%% -*- coding:utf-8 -*-
\documentclass[ number=1
                ,series=tbls,
	        %,blackandwhite
	        ,smallfont
	        %,draftmode  
		  ]{../langscibook}                          


\usepackage{makros.2e}

\usepackage{pstricks,pst-node}

%\nodemargin5pt%\treelinewidth2pt\arrowwidth6pt\arrowlength10pt
\psset{nodesep=5pt} %,linewidth=0.8pt,arrowscale=2}
\psset{linewidth=0.5pt}

% compile with texlive 2013

\begin{document}
\thispagestyle{empty}

%\oneline{%
\begin{pspicture}(0,0)(9.4,4.2)
%\psgrid
\rput[Bl](0,0){%
\begin{tabular}[b]{@{}ccc@{}}
Phänomen && Modell\\
\rnode{phen}{\fbox{\begin{tabular}{c}
linguistische\\
Objekte\\
\end{tabular}}}&&\rnode{modell}{\fbox{\begin{tabular}{c}
Merkmal-\\
strukturen\\
\end{tabular}}}\\[10ex]
&\rnode{theorie}{\fbox{\begin{tabular}{c}
Merkmal-\\
beschreibungen\\
\end{tabular}}}\\
&formale Theorie\\
\end{tabular}}
%\anodeconnect[l]{modell}[r]{phen}%
\ncline{->}{modell}{phen}\nbput{modelliert}
%\ncdiag[angleA=180,angleB=45]{->}{modell}{theorie}
\psline{<-}(5.4,1.6)(6,2.4)
\rput[Bl](6,2){wird von Theorie lizenziert} % früher Erfüllung, jetzt FR Änderung
\psline{->}(5,1.7)(5.6,2.4)
\rput[Bl](3.4,2.0){legt fest}
%\nccurve{<-}{modell}{theorie}\nbput{legt fest}
\ncline{->}{theorie}{phen}\naput{sagt vorher}%
%\aanodeconnect[b]{modell}[tr]{theorie}%
%\anodeconnect[tl]{theorie}[b]{phen}%
\end{pspicture}


\end{document}
