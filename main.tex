%% -*- coding:utf-8 -*-
\documentclass[output=book
%                ,series=tbls,
%		,modfonts
%		,nonflat
%		,multiauthors
%	        ,collection
%	        ,collectionchapter
%	        ,collectiontoclongg
 	        ,biblatex  
                ,booklanguage=german
%                ,babelshorthands
%                ,showindex
                ,newtxmath
%                ,colorlinks, citecolor=brown % for drafts
%                ,draftmode
% 	        ,coverus
                ,uniformtopskip
                ,tblseight % to get the fonts for the Chinese title of GT
		  ]{langscibook} 


\usepackage{ifthen}
\provideboolean{draft}
\setboolean{draft}{false}

\newtoggle{draft}\togglefalse{draft}
%\newtoggle{draft}\toggletrue{draft}
%\newtoggle{finished}\togglefalse{finished}



% load memoize before german.sty

%\usepackage{nomemoize} 
\usepackage{memoize} % The figures have been created with an earlier version of texlive. Recreating
                     % them changes the layout of the pages, so we have to rely on them. Unless ... 16.03.2023
%\memoizeset{readonly}
% see below for further settings

%% -*- coding:utf-8 -*-
\author{Stefan Müller}
\title{Head-Driven Phrase Structure Grammar}
\subtitle{Eine Einführung\newlineCover 
\textit{\LARGE {\spaceskip=3.5pt Vierte überarbeitete Auf\/lage}}}
\renewcommand{\lsImpressumCitationText}{%
Stefan Müller. \lsYear. \textit{Head-Driven Phrase Structure Grammar: Eine Einführung. Vierte
  überarbeitete Auf\kern0pt lage}. (Textbooks in Language Sciences~). Berlin: Language Science Press.
}

\typesetter{Stefan Müller}


\BackTitle{Head-Driven Phrase Structure Grammar}
\BackBody{
Das Buch ist eine abgeschlossene Einführung in das Framework der Head-Driven Phrase Structure Grammar. In einem einführenden Kapitel wird der Übergang von einfachen Phrasenstrukturgrammatiken zu den komplexeren Repräsentationen mit Merkmalbeschreibungen motiviert. Das zweite Kapitel führt allgemein verständlich in den Formalismus der Merkmalstrukturen ein. In den verbleibenden Kapiteln werden phänomenbezogen verschiedene Grammatikbereiche diskutiert, wobei die Analysen jeweils für das Deutsche ausgearbeitet werden. Der Schwerpunkt liegt hierbei auf syntaktischen Phänomenen (Konstituentenstruktur, Konstituentenstellung, Kasus, Passiv, Kongruenz, Verbalkomplex und Partikelverben), die Morphologie (Flexion und Derivation) wird aber ebenfalls behandelt, und es wird gezeigt, wie die Bedeutung morphologischer und syntaktischer Konstruktionen kompositional bestimmt werden kann.

In den Analyseabschnitten wurde bewußt auf die Diskussion von Alternativen verzichtet. Den
Alternativen ist jeweils ein eigener Abschnitt gewidmet, der dem fortgeschrittenen Leser den
Vergleich mit anderen HPSG-Ansätzen aber auch mit Ansätzen aus der Konstruktionsgrammatik (CxG), der
Lexikalisch-Funktionalen-Grammatik (LFG) und mit Varianten der Government \& Binding-Theorie (GB)
ermöglichen soll. Bei der Diskussion der Alternativen spielen sowohl formale Eigenschaften der
jeweiligen Theorien als auch empirische Befunde aus dem Deutschen und anderen Sprachen eine Rolle. 

Das Buch mit der Virtuellen Maschine für Grammix richtet sich gleichermaßen an Lehrende und Lernende
der Germanistik, der allgemeinen Sprachwissenschaft und der Computerlinguistik. 

\bigskip
%~
%\smallskip
\vfill


\medskip

\noindent
„Der Inhalt des Buches ist komplex und behandelt eine nicht einfache Thematik. Für Studierende
sprachwissenschaftlicher Fächer ist dieses Lehrbuch aber durchaus geeignet und auch in seinem Aufbau
didaktisch wohlüberlegt.“ Jens Fleischhauer. 2010. \emph{Roterdorn}. 

\medskip

\noindent
„Einerseits ist in der vorliegenden Einführung ein sehr gut gegliederter didaktischer Aufbau
hervorzuheben, andererseits kamen viele neue Beispielanalysen komplexer syntaktischer
Strukturen der deutschen Sprache dazu, was für Neulinge auf dem Gebiet einen echten Gewinn für die
Verarbeitung dieser absolut nicht einfachen Materie bringt und auch für Insider neue Perspektiven
der Erweiterung dieses faszinierenden Ansatzes bietet. [\ldots] Hervorzuheben ist, dass Müller es
versteht, sehr klar und kleinschrittig viele weitere wichtige syntaktische Phänomene des Deutschen
nachvollziehbar einzuführen.“ \href{https://doi.org/10.1515/infodaf-2009-2-361}{Markus J. Weininger. 2009. \emph{Informationen Deutsch als Fremdsprache}}. 


\medskip

\noindent
%„HPSG – \rot{H}ier \rot{p}räsentiert \rot{s}ich \rot{G}rammatik in Theorie \& Anwendung von kompetentester Hand. Nicht nur für Syntaxinteressierte ein „Must-Have“ und noch dazu ein „Easiest-to-Have“, da Open Access dank Language Science Press!“ Hubert Haider, 2024.

\noindent
„HPSG – \emph{Hier präsentiert sich Grammatik} in Theorie \& Anwendung von kompetentester
Hand. Nicht nur für Syntaxinteressierte ein „Must-Have“ und noch dazu ein „Easiest-to-Have“, da Open
Access dank Language Science Press!“ Hubert Haider. 2024.

\medskip

\noindent
„Ich freue mich sehr, dass Stefan Müllers exzellente Einführung  in die HPSG nun in einer
überarbeiteten Auflage bei Language Science Press als Open-Access-Publikation erscheint, womit
Studierende und Forscher/innen eine einfache Möglichkeit erhalten, sich in dieses wichtige
Grammatikmodell detailliert einzuarbeiten und über den neuesten Stand seiner Entwicklung zu
informieren.“ Joachim Jacobs, 2024.

}



\dedication{Ich widme dieses Buch Brigitte Narr als Dank für ihren unermüdlichen Einsatz für die
  Sprachwissenschaft.\\Sie gehört definitiv zu den Guten im Verlagswesen.}

% otherwise the distance between lines is too big. St. Mü. 14.02.2025
\renewcommand{\lsDedicationFont}{\fontsize{15pt}{8mm}\selectfont}

%\renewcommand{\lsISBNdigital}{978-3-96110-408-6}
%\renewcommand{\lsISBNhardcover}{978-3-98554-066-2} Lehrbuch -> no hardcover
%\renewcommand{\lsISBNsoftcover}{978-3-98554-066-2}
%\renewcommand{\lsBookDOI}{10.5281/zenodo.7733033}
\renewcommand{\lsSeries}{tbls} % use lowercase acronym, e.g. sidl, eotms, tgdi
%\renewcommand{\lsSeriesNumber}{12} %will be assigned when the book enters the proofreading stage
%% \renewcommand{\lsURL}{http://langsci-press.org/catalog/book/25} % contact the coordinator for the right number
%\renewcommand{\lsID}{353}

% \proofreader{%
% Amir Ghorbanpour,
% Wilson Lui,
% Lachlan Mackenzie,
% Rebecca Madlener,
% Katja Politt,
% Janina Rado,
% Brett Reynolds,
% Lea Schäfer,
% Annika Schiefner,
% Troy E. Spier,
% George Walkden,
% Jeroen van de Weijer}
%% -*- coding:utf-8 -*-

\usepackage{csquotes}

% \up for relative clause chapter


% http://tex.stackexchange.com/questions/38607/no-room-for-a-new-dimen
%\usepackage{etex}\reserveinserts{28}


% for subnodes in trees
\usepackage{tcolorbox}
\tcbuselibrary{skins}
\newtcbox{\mybox}[1][]{empty,shrink tight,nobeforeafter,on line,before upper=\vphantom{gM},remember as=#1,top=2pt,bottom=2pt}



\newcommand{\page}{}


% http://tex.stackexchange.com/questions/229500/tikzmark-and-xelatex
% temporary fix, remove later
%\newcount\pdftexversion \pdftexversion140 \def\pgfsysdriver{pgfsys-dvipdfm.def} \usepackage{tikz} \usetikzlibrary{tikzmark}



% \justify to switch of \raggedright in translations
%\usepackage{ragged2e}



%\usepackage{metalogo} % xelatex

\usepackage{multicol}

\usepackage{bookmark}

%\usepackage{my-ccg-ohne-colortbl}




% This has side effects on my-ccg commands do no know why
%\usepackage{./langsci/langsci-optional}

\usepackage{todonotes}

% used to be in this package
\providecommand{\citegen}{}
\renewcommand{\citegen}[2][]{\citeauthor{#2}'s (\citeyear*[#1]{#2})}
\providecommand{\lsptoprule}{}
\renewcommand{\lsptoprule}{\midrule\toprule}
\providecommand{\lspbottomrule}{}
\renewcommand{\lspbottomrule}{\bottomrule\midrule}
\providecommand{\largerpage}{}
\renewcommand{\largerpage}[1][1]{\enlargethispage{#1\baselineskip}}


\usepackage{./styles/oneline}

%\let\oneline\onelinetextwidthhack


\usepackage{./styles/biblatex-series-number-checks}

% not needed for 2024. It does not exist.
%\usepackage{langsci-basic}
%\usepackage{langsci-optional}
%\usepackage{langsci-lgr}
%\newcommand{\M}{\textsc{m}{}\xspace}        % use at own risk



\usepackage{langsci-branding}
%\usepackage[danger]{langsci-lgr}
\usepackage{langsci-lgr}

\usepackage{graphicx}

\usepackage{soul}

%\usepackage{./styles/mycommands}% \spacebr


\usepackage{langsci-gb4e}
\usepackage{styles/jambox}

% fixes problem with to much vertical space between \zl and \eal due to the \nopagebreak
% command.
\makeatletter
\def\@exe[#1]{\ifnum \@xnumdepth >0%
                 \if@xrec\@exrecwarn\fi%
                 \if@noftnote\@exrecwarn\fi%
                 \@xnumdepth0\@listdepth0\@xrectrue%
                 \save@counters%
              \fi%
                 \advance\@xnumdepth \@ne \@@xsi%
                 \if@noftnote%
                        \begin{list}{(\thexnumi)}%
                        {\usecounter{xnumi}\@subex{#1}{\@gblabelsep}{0em}%
                        \setcounter{xnumi}{\value{equation}}}
% this is commented out here since it causes additional space between \zl and \eal 06.06.2020
%                        \nopagebreak}%
                 \else%
                        \begin{list}{(\roman{xnumi})}%
                        {\usecounter{xnumi}\@subex{(iiv)}{\@gblabelsep}{\footexindent}%
                        \setcounter{xnumi}{\value{fnx}}}%
                 \fi}
\makeatother


\makeatletter
\def\eas{\ifnum\@xnumdepth=0\begin{exe}[(34)]\else\begin{xlist}[iv.]\fi\ex\begin{tabular}[t]{@{}p{.99\linewidth}@{}}}
\makeatother

\settowidth\jamwidth{(German)}

%\usepackage{subfig}

%\renewcommand{\xbar}{X̅\xspace}


%\usepackage[external,linguistics]{styles/forest/forest}


% for reasons I do not understand this cannot be moved further down and 
% the loading of forest further down cannot be removed. St. Mü. 26.01.2017
% It breaks the dependency grammar trees in forest.
\usepackage{langsci-forest-setup}

% for Germanic history tree
\useforestlibrary{edges} 

%\usepackage{memoize} 
%\usepackage{nomemoize} % use this if memoize caues chaos.
%\memoizeset{
%  memo filename prefix={germanic.memo.dir/},
%}
% uncomment this if your figures change frequently and you do not want memoize to externalize them.
%\memoizeset{readonly}




% has to be loaded after forest-setup because of incompatibilities with the dg-style.
%\usepackage{german}\selectlanguage{ngerman}
%\usepackage[ngerman]{babel}

%\selectlanguage{ngerman}

\usepackage{./styles/merkmalstruktur,./styles/abbrev,./styles/makros.2020,./styles/my-xspace,./styles/article-ex,./styles/additional-langsci-index-shortcuts,
./styles/de-date,./styles/my-theorems,styles/mycommands}

\usepackage{langsci-avm}

\let\vref\ref
\newcommand\figuresref[2]{%
Figure~\ref{#1} and Figure~\ref{#2}%
}

\if0

% loaded in macros.2e \usepackage[english]{varioref}
% do not stop and warn! This will be tested in the final version
%\usepackage[english]{varioref}
%\vrefwarning

\newcommand\thefiguresref[2]{%
 \vrefpagenum\firstnum{#1}%
 \vrefpagenum\secondnum{#2}%
\ifthenelse{\equal\firstnum\secondnum}%
{the Figures~\ref{#1} and~\ref{#2}\vpageref[]{#1}}%
{the Figures~\ref{#1} and~\ref{#2}}}

\newcommand\figuresref[2]{%
 \vrefpagenum\firstnum{#1}%
 \vrefpagenum\secondnum{#2}%
\ifthenelse{\equal\firstnum\secondnum}%
{\iflanguage{german}{%
die Abbildungen~\ref{#1} und~\ref{#2} \vpageref[]{#1}%
}% end German
{Figures~\ref{#1} and~\ref{#2} \vpageref[]{#1}}}%
{\iflanguage{german}{%
Abbildung~\ref{#1}\vpageref[]{#1} und Abbildung~\ref{#2} \vpageref{#2}%
}% end German
{%
Figure~\ref{#1}\vpageref[]{#1} and Figure~\ref{#2} \vpageref{#2}}}%
}


\newcommand\pagerefonlyifdifferent[2]{%
\vrefpagenum\firstnum{#1}%
\vrefpagenum\secondnum{#2}%
\ifthenelse{\equal\firstnum\secondnum}%
{}
{\vpageref{#2}}}



\newcommand\figuretwoonsamepagesref[2]{%
 \vrefpagenum\firstnum{#1}%
 \vrefpagenum\secondnum{#2}%
\ifthenelse{\equal\firstnum\secondnum}%
%{ on the same page as Figure~\ref{#1}}%
{}%
{ on page~\vpageref{#2}}%
}
\renewcommand{\reftextcurrent}{}

\newcommand\refORregion[2]{%
 \vrefpagenum\firstnum{#1}%
 \vrefpagenum\secondnum{#2}%
\ifthenelse{\equal\firstnum\secondnum}%
{\pageref{#1}}%
{\pageref{#1}--\pageref{#2}}%
}

%\let\reftextfaceafter\reftextafter
%\let\reftextfacebefore\reftextbefore

\addto\extrasenglish{%
     \renewcommand\reftextfaceafter {\reftextafter}%
     \renewcommand\reftextfacebefore {\reftextbefore}%
}

\fi

% draw a grid for getting the coordinates
\usepackage{./styles/tikz-grid}

% Adapted from https://tex.stackexchange.com/questions/255234/how-does-one-pick-control-points-to-control-b%C3%A9zier-curves-in-tikz
% \DrawControl{(12,4)}{1}\DrawControl{(-4,4)}{2};  
\newcommand\DrawControl[2]{
  \node[circle,fill=red,inner sep=2pt,label={above:$#1$},label={[black]below:{\footnotesize#2}}] at #1 {};
}

% for offsets in trees
%\newlength{\offset}
%\newlength{\offsetup}

\ifxetex
\usepackage{./styles/de-hyp-utf8}
\else
\usepackage{./styles/de-hyp}
\fi

\usepackage{appendix}



% http://tex.stackexchange.com/questions/3223/subscripts-for-primed-variables
%
% to get 
% {}[ af   [~]\sub{V} ]\sub{V$'$}
%
% typeset properly. Thanks, Sebastian.
%
\usepackage{subdepth}


%\usepackage{caption}



\usepackage{langsci-tbls}
% do not need identation for enumerate since we are in a box anyway.
\usepackage{enumitem}



% for abbreviations 02.05.2020
\usepackage{tabularx}


% remove when finished
\usepackage{proofread}


%% -*- coding:utf-8 -*-


\hypersetup{bookmarksopenlevel=0}

\setcounter{secnumdepth}{4}

\let\citew\citet

\let\citews\textcites

\newcommand{\page}{}


\newcommand{\danish}{\jambox{(\ili{Danish})}}
\newcommand{\dutch}{\jambox{(\ili{Dutch})}}
\newcommand{\english}{\jambox{(\ili{English})}}
\newcommand{\german}{\jambox{(\ili{German})}}
\newcommand{\yiddish}{\jambox{(\ili{Yiddish})}}
\newcommand{\icelandic}{\jambox{(\ili{Icelandic})}}




%\newcommand{\rot}[1]{{\color{red}#1}}
% like Lab Phon
%\newcommand{\rot}[1]{{\color{lsLightWine}#1}}
% like Niger Congo
\newcommand{\rot}[1]{{\color{lsRed}#1}}
\newcommand{\rotbf}[1]{\rot{#1}}
\newcommand{\rotit}[1]{\rot{#1}}
%\newcommand{\gruen}[1]{{\color{green}#1}}
% like OGS
\newcommand{\gruen}[1]{{\color{lsDarkGreenOne}#1}}
\newcommand{\gruenbf}[1]{\gruen{#1}}
\newcommand{\gruensc}[1]{\gruen{#1}}
\newcommand{\gruenit}[1]{\gruen{#1}}

%\newcommand{\blau}[1]{{\color{blue}#1}}

% Topics at the grammar icours interface
% lsMidDarkBlue

% EOTMS
% lsMidBlue
\newcommand{\blau}[1]{{\color{lsMidBlue}#1}}

\newcommand{\blaubf}[1]{\blau{#1}}
\newcommand{\blausc}[1]{\blau{#1}}
\newcommand{\blauit}[1]{\blau{#1}}



\let\mc=\multicolumn


\newcommand{\sigle}[1]{\tiny{#1}}


%% now loaded by the langsci class
% \iftoggle{draft}{
% \usepackage{todonotes}
% }{
% \usepackage[disable]{todonotes}
% }

\iftoggle{draft}{}{
\presetkeys{todonotes}{disable}{}
}

\newcommand{\todostefan}[1]{\todo[color=orange!80]{\footnotesize #1}\xspace}
\newcommand{\todosatz}[1]{\todo[color=red!40]{\footnotesize #1}\xspace}

\newcommand{\inlinetodostefan}[1]{\todo[color=green!40,inline]{\footnotesize #1}\xspace}

\newcommand{\inlinetodoopt}[1]{\todo[color=green!40,inline]{\footnotesize #1}\xspace}
\newcommand{\inlinetodoobl}[1]{\todo[color=red!40,inline]{\footnotesize #1}\xspace}

\newcommand{\itd}[1]{\iftoggle{draft}{\inlinetodoobl{#1}}{}}
\newcommand{\itdobl}[1]{\iftoggle{draft}{\inlinetodoobl{#1}}{}}
\newcommand{\itdopt}[1]{\iftoggle{draft}{\inlinetodoopt{#1}}{}}

% for editing, remove later
%\usepackage{xcolor}
\newcommand{\iaddpages}{\iftoggle{draft}{\yel[add pages]{pages}\xspace}}

\newcommand{\addpages}{\iftoggle{draft}{\todostefan{add pages}}\xspace}
\newcommand{\addsource}{\iftoggle{draft}{\todostefan{add source}}\xspace}
\newcommand{\addglosses}{\iftoggle{draft}{\todostefan{add glosses}}\xspace}



%% % This sets the default for the positioning
%% % will be in the main class
%% \makeatletter
%% \renewcommand{\fps@figure}{htbp}
%% \renewcommand{\fps@table}{htbp}
%% \makeatother
%% \renewcommand{\floatpagefraction}{0.7}	% require fuller float pages
%% 	% N.B.: floatpagefraction MUST be less than topfraction !!




\robustify\textsc
\robustify\textit


% https://tex.stackexchange.com/questions/95014/aligning-overline-to-italics-font/95079#95079
\newbox\usefulbox

\makeatletter
    \def\getslant #1{\strip@pt\fontdimen1 #1}

    \def\skoverline #1{\mathchoice
     {{\setbox\usefulbox=\hbox{$\m@th\displaystyle #1$}%
        \dimen@ \getslant\the\textfont\symletters \ht\usefulbox
        \divide\dimen@ \tw@ 
        \kern\dimen@ 
        \overline{\kern-\dimen@ \box\usefulbox\kern\dimen@ }\kern-\dimen@ }}
     {{\setbox\usefulbox=\hbox{$\m@th\textstyle #1$}%
        \dimen@ \getslant\the\textfont\symletters \ht\usefulbox
        \divide\dimen@ \tw@ 
        \kern\dimen@ 
        \overline{\kern-\dimen@ \box\usefulbox\kern\dimen@ }\kern-\dimen@ }}
     {{\setbox\usefulbox=\hbox{$\m@th\scriptstyle #1$}%
        \dimen@ \getslant\the\scriptfont\symletters \ht\usefulbox
        \divide\dimen@ \tw@ 
        \kern\dimen@ 
        \overline{\kern-\dimen@ \box\usefulbox\kern\dimen@ }\kern-\dimen@ }}
     {{\setbox\usefulbox=\hbox{$\m@th\scriptscriptstyle #1$}%
        \dimen@ \getslant\the\scriptscriptfont\symletters \ht\usefulbox
        \divide\dimen@ \tw@ 
        \kern\dimen@ 
        \overline{\kern-\dimen@ \box\usefulbox\kern\dimen@ }\kern-\dimen@ }}%
     {}}
    \makeatother



% \newcommand{\questions}[1]{~\newline\vspace*{-10mm}
% {\mmzset{disable}%
% \tblssy{people}{Kontrollfragen}{\setlist{leftmargin=*}#1}}}
% %\tblssy{people}{Comprehension questions}{#1}}

% \newcommand{\exercises}[1]{{\mmzset{disable}%
% \tblssy{pencil}{Übungsaufgaben}{\setlist{leftmargin=*}#1}}}
% %\tblssy{pencil}{Exercises}{#1}}

% \newcommand{\furtherreading}[1]{%~\newline\vspace*{-10mm}
% {\mmzset{disable}%
% \tblssy{book}{Literaturhinweise}{#1}}}

% \newcommand{\greyboxrest}[1]{
% {\mmzset{disable}%
% \begin{mdframed}[style=greyexercise]
% #1
% \end{mdframed}}}


% \newcommand{\questions}[1]{~\newline\vspace*{-10mm}
% {\mmzset{disable}%
% \tblssy{people}{Kontrollfragen}{\setlist{leftmargin=*}#1}}}
% %\tblssy{people}{Comprehension questions}{#1}}

% \newcommand{\exercises}[1]{{\mmzset{disable}%
% \tblssy{pencil}{Übungsaufgaben}{\setlist{leftmargin=*}#1}}}
% %\tblssy{pencil}{Exercises}{#1}}

% \newcommand{\furtherreading}[1]{%~\newline\vspace*{-10mm}
% {\mmzset{disable}%
% \tblssy{book}{Literaturhinweise}{#1}}}

% \newcommand{\greyboxrest}[1]{
% {\mmzset{disable}%
% \begin{mdframed}[style=greyexercise]
% #1
% \end{mdframed}}}


% \mdfdefinestyle{greyexercisenologo}{%
% 	everyline=true,ignorelastdescenders=true,
% 	linewidth=0pt,backgroundcolor=\tblsboxcolor,
% 	innerleftmargin=5mm, innerrightmargin=5mm, innerbottommargin=5mm, innertopmargin=5mm,
% 	frametitleaboveskip=15mm, frametitlebelowskip=5mm,frametitlerule=false, repeatframetitle=false
% }



% Is mdframed used as backend?
\notbool{langsci@tbls@tcolorbox}{%
\newcommand{\questions}[1]{~\newline\vspace*{-10mm}
{\mmzset{disable}%
\tblssy{people}{Comprehension questions}{\setlist{leftmargin=*}#1}}}
%\tblssy{people}{Comprehension questions}{#1}}

\newcommand{\exercises}[1]{{\mmzset{disable}%
\tblssy{pencil}{Exercises}{\setlist{leftmargin=*}#1}}}
%\tblssy{pencil}{Exercises}{#1}}

\newcommand{\furtherreading}[1]{%~\newline\vspace*{-10mm}
{\mmzset{disable}%
\tblssy{book}{Further reading}{#1}}}

\newcommand{\greyboxrest}[1]{
{\mmzset{disable}%
\begin{mdframed}[style=greyexercisenologo]
#1
\end{mdframed}
}}

\mdfdefinestyle{greyexercisenologo}{%
	everyline=true,ignorelastdescenders=true,
	linewidth=0pt,backgroundcolor=\tblsboxcolor,
	innerleftmargin=5mm, innerrightmargin=5mm, innerbottommargin=5mm, innertopmargin=5mm,
	frametitleaboveskip=15mm, frametitlebelowskip=5mm,frametitlerule=false, repeatframetitle=false
}}{}

% Is tcolorbox used as backend?
\ifbool{langsci@tbls@tcolorbox}{%
\newcommand{\questions}[1]{{\mmzset{disable}%
\begin{tblsfilledsymbol}{Comprehension questions}{people}\setlist{leftmargin=*}#1\end{tblsfilledsymbol}}}
\newcommand{\exercises}[1]{{\mmzset{disable}%
\begin{tblsfilledsymbol}{Exercises}{pencil}\setlist{leftmargin=*}#1\end{tblsfilledsymbol}}}
\newcommand{\furtherreading}[1]{{\mmzset{disable}%
\begin{tblsfilledsymbol}{Further reading}{book}#1\end{tblsfilledsymbol}}}}{}




% get rid of these morewrite messages:
% https://tex.stackexchange.com/questions/419489/suppressing-messages-to-standard-output-from-package-morewrites/419494#419494
\ExplSyntaxOn
\cs_set_protected:Npn \__morewrites_shipout_ii:
  {
    \__morewrites_before_shipout:
    \__morewrites_tex_shipout:w \tex_box:D \g__morewrites_shipout_box
    \edef\tmp{\interactionmode\the\interactionmode\space}\batchmode\__morewrites_after_shipout:\tmp
  }
\ExplSyntaxOff

\newcommand{\term}[1]{\emph{\isi{#1}}}



% Felix 09.06.2020: copy code from the third line into localcommands.tex: https://github.com/langsci/langscibook#defined-environments-commands-etc
\patchcmd{\mkbibindexname}{\ifdefvoid{#3}{}{\MakeCapital{#3} }}{\ifdefvoid{#3}{}{#3 }}{}{\AtEndDocument{\typeout{mkbibindexname could not be patched.}}}



% This does a linebreak for \gll for long sentences leaving space for the language at the right
% margin.
% St.Mü. 17.06.2021
% \newcommand{\longexampleandlanguage}[2]{
% \begin{tabularx}{.99\linewidth}[t]{@{}X@{}p{\widthof{(#2)}}@{}}
% \begin{minipage}[t]{\linewidth-1em}
% #1
% \end{minipage} & (\ili{#2})
% \end{tabularx}}

% From stackexchange:
% https://tex.stackexchange.com/questions/600695/limiting-space-used-by-gb4e-and-adding-language-information-to-the-right/649414#649414
\newcommand{\longexampleandlanguage}[2]{
%\begin{tabularx}{\linewidth}[t]{@{}X@{}p{\widthof{(#2)}}@{}}
\begin{minipage}[t]{\linewidth-1em-\widthof{(#2)}}
#1
\end{minipage} 
\hfill
\begin{minipage}[t]{\widthof{(#2)}}
 (\ili{#2})
\end{minipage}
%\end{tabularx}
}


% This is needed for the index since the ! will be interpreted as subitem.
\newcommand{\dslmath}{$/\!/$}



% http://tex.stackexchange.com/questions/230300/doing-something-like-psframebox-in-tikz#230306
\tikzset{
frbox/.style={
  rounded corners,
  draw,
  thick,
  inner sep=5pt
  }
}
\newcommand\TZbox[1]{\tikz{\node[frbox,baseline] {#1};}}

\newgray{hell}{.85}
%\newcommand{\highlight}[1]{\psframebox[linecolor=hell,fillcolor=hell,fillstyle=solid]{#1}}
\newcommand{\highlight}[1]{\colorbox{hell}{#1}}

\newcommand{\nom}{{\normalfont\textit{nom}}}
\newcommand{\gen}{{\normalfont\textit{gen}}}
\newcommand{\dat}{{\normalfont\textit{dat}}}
\newcommand{\acc}{{\normalfont\textit{acc}}}

%\newcommand{\mod}{\textsc{mod}\xspace}  % wegen beamer.cls nicht in abbrev.sty
\newcommand{\rel}{\textsc{rel}\xspace}  % wegen avm.sty nicht in abbrev.sty

% for offsets in trees
\newlength{\offset}
\newlength{\offsetup}
%% -*- coding:utf-8 -*-
%% hyphenation points for line breaks
%% Normally, automatic hyphenation in LaTeX is very good
%% If a word is mis-hyphenated, add it to this file
%%
%% add information to TeX file before \begin{document} with:
%% %% -*- coding:utf-8 -*-
%% hyphenation points for line breaks
%% Normally, automatic hyphenation in LaTeX is very good
%% If a word is mis-hyphenated, add it to this file
%%
%% add information to TeX file before \begin{document} with:
%% %% -*- coding:utf-8 -*-
%% hyphenation points for line breaks
%% Normally, automatic hyphenation in LaTeX is very good
%% If a word is mis-hyphenated, add it to this file
%%
%% add information to TeX file before \begin{document} with:
%% \include{localhyphenation}
\hyphenation{
Adjektiv-stamm-lexikon
Ar-gu-ment-rol-le
Ar-gu-ment-rol-len
Ar-gu-ment-rollen-rea-li-sie-rung-en
Aus-schluss-klau-sel
As-pekt
Bay-er
             be-schrei-ben
Bei-spiel-ab-lei-tung
Be-schrän-kun-gen
             Bilder-nomina
             Bin-dungs-theorie
Cog-ni-tive
Con-fer-ence
con-straint
con-straints
CO-LING
             Do-mi-nanz-sche-ma
             Do-mi-nanz-sche-ma-ta
De-mons-tra-tiv-pro-no-men
Dem-ske
Dis-kus-sion
Druck-erzeug-nis
Druck-erzeug-nis-sen
dunk-les
ent-wi-ckeln
	     Er-wei-te-rung
             %Flexions-merkmale
Er-werbs-pha-se
Er-werbs-pro-blem
Er-werbs-pro-zess-es
Fi-nit-heit
Funk-tions-verb-ge-fü-ges
Gar-aus
ge-typ-ter
             Gram-ma-tik
Gram-ma-tik-re-gel
Gram-ma-tik-re-geln
             Grund-for-ma-lis-mus
Grund-ope-ra-ti-on-en
Hilfs-verb-in-ver-sion
             Im-ple-men-tation
In-kon-sis-tenz-en
In-ter-na-tio-nal
             In-ter-net-an-bie-ter
	     in-tran-si-tiv
	     in-tran-si-tiv-es
             In-fi-ni-tiv-verb-phra-se
Jo-shi
             Kasus-an-for-de-rung
	     Ka-te-go-rien
	     Ka-te-go-rien-in-ven-tar
             Kon-so-nan-ten-um-wand-lung
             Kon-struk-tio-nen
Kop-fes
Kopf-merk-mals-prin-zip
             Kopf-toch-ter
             Kor-pus-an-fra-gen
             Kor-pus-an-fra-ge
             le-xi-ka-lisch
lin-gu-is-tisch-er
Lin-gu-is-tik
Lis-ten-en-des
	     Lo-kal-an-ga-ben
             mathe-ma-tische
	     Mehr-deu-tig-keiten
	     Mehr-deu-tig-keit
             Merk-mal-struk-tur
miss-trau-en
Mit-tel-feld-ele-men-te
Mor-phol-o-gy
New-meyer
    nicht-erga-tives
             nicht-erga-tiven
nicht-idio-ma-tisch
nicht-idio-ma-tische
nicht-idio-ma-tischer
nicht-idio-ma-tischen
             %Nominal-phrase
             %Nominal-umgebungen
             NON-LOC
             Ober-feld-um-stel-lung
             Ob-jekt-an-he-bung
Ob-jekt-an-he-bungs-verb
Ob-jekt-an-he-bungs-ver-ben
             Ob-li-que-ness
             Para-phra-sen
             Parse-stra-te-gien
Par-ti-kel-verb-re-gel
Par-ti-zip-ein-trag
Pas-siv-ana-ly-se
Pas-siv-ana-ly-sen
             Pos-ses-siv-pro-no-men
             Pos-ses-siv-pro-no-mi-na
Pro-blem
Pro-jekt-er-geb-nis-se
Quar-tals-er-fol-ge
Rat-ten-fä-nger
Re-dun-danz
Re-dun-danz-re-gel
Re-dun-danz-re-geln
             re-fe-renz-iden-tisch
Re-fe-renz-ob-jekt
             Re-la-tiv-satz-ele-ment
	     Rest-feld-ele-men-te
Satz-en-de
Satz-mus-tern
             sub-sum-iert
Schwär-me-rei
se-man-tics
Se-man-tik-spe-zi-fi-ka-tion
	     Si-tu-ations-se-man-tik
Sprach-er-werb
Sprach-er-werbs
Sprach-er-werbs-phä-no-me-ne
Sprach-er-werbs-pro-blem
Sprach-er-werbs-theo-rien
sprach-spe-zi-fisch
sprach-spe-zi-fische
sprach-spe-zi-fisch-es
             Stan-dard-ab-fol-ge
Stan-ford
Sta-tis-tik-kom-po-nen-ten
sta-tis-tische
Struk-tur-tei-lung
Sub-jekt-ex-trak-tion
             syn-tak-tische
Ta-ges-leis-tung
teil-idio-ma-tisch
teil-idio-ma-ti-schen
             Teil-spezi-fi-ka-tio-nen
             Teil-spezi-fi-ka-tion
The-o-ret-i-cal
Tür-ki-schen
             Typ-beschrei-bung
             Typ-beschrei-bung-en
             Typ-hier-ar-chie
             Typ-konzept
             Typ-sys-tem
	     unter-schied-licher
Ver-ar-bei-tungs-sys-tems
             %Ver-ben
             %Verbal-phrasen-teilen
             %Verbal-pro-jek-tion
             %Verbal-pro-jek-tion-en
             Ver-bal-kom-plex-unter-bre-chung
Verb-be-we-gung
Verb-be-we-gungs-spur
Verb-ebe-ne
             Verb-end-stel-lung
             Verb-en-dung
             Verb-en-dung-en
Verb-erst-le-xi-kon-re-gel
	     Verb-erst-stel-lung
Verb-erst-sät-ze
Verb-erst-sät-zen
Verb-letzt-sät-ze
Verb-letzt-sät-zen
Verb-letzt-stel-lung
Verb-letzt-po-si-tion
	     Verb-zweit-stel-lung
             Verb-form
             Verb-kom-ple-men-ten
             Verb-kom-ple-men-te
Verb-mo-bil
             Verb-phra-se
             Verb-phra-sen
Verb-pro-jek-tion
Verb-pro-jek-tion-en
             Vor-le-sung
             Vor-feld-el-lip-se
             Vor-feld-el-lip-sen
	     ver-wen-det-en
             war-um
             War-um
weg-ge-las-sen
             Wur-zel-kno-ten
             Zusammen-setzen
             Zu-stands-pas-siv
             Zu-stands-pas-sivs
Wort-ein-heit
Wun-der-lich
Fink-bei-ner
Zeit-schrift-en-ar-ti-kel
}
\hyphenation{
Adjektiv-stamm-lexikon
Ar-gu-ment-rol-le
Ar-gu-ment-rol-len
Ar-gu-ment-rollen-rea-li-sie-rung-en
Aus-schluss-klau-sel
As-pekt
Bay-er
             be-schrei-ben
Bei-spiel-ab-lei-tung
Be-schrän-kun-gen
             Bilder-nomina
             Bin-dungs-theorie
Cog-ni-tive
Con-fer-ence
con-straint
con-straints
CO-LING
             Do-mi-nanz-sche-ma
             Do-mi-nanz-sche-ma-ta
De-mons-tra-tiv-pro-no-men
Dem-ske
Dis-kus-sion
Druck-erzeug-nis
Druck-erzeug-nis-sen
dunk-les
ent-wi-ckeln
	     Er-wei-te-rung
             %Flexions-merkmale
Er-werbs-pha-se
Er-werbs-pro-blem
Er-werbs-pro-zess-es
Fi-nit-heit
Funk-tions-verb-ge-fü-ges
Gar-aus
ge-typ-ter
             Gram-ma-tik
Gram-ma-tik-re-gel
Gram-ma-tik-re-geln
             Grund-for-ma-lis-mus
Grund-ope-ra-ti-on-en
Hilfs-verb-in-ver-sion
             Im-ple-men-tation
In-kon-sis-tenz-en
In-ter-na-tio-nal
             In-ter-net-an-bie-ter
	     in-tran-si-tiv
	     in-tran-si-tiv-es
             In-fi-ni-tiv-verb-phra-se
Jo-shi
             Kasus-an-for-de-rung
	     Ka-te-go-rien
	     Ka-te-go-rien-in-ven-tar
             Kon-so-nan-ten-um-wand-lung
             Kon-struk-tio-nen
Kop-fes
Kopf-merk-mals-prin-zip
             Kopf-toch-ter
             Kor-pus-an-fra-gen
             Kor-pus-an-fra-ge
             le-xi-ka-lisch
lin-gu-is-tisch-er
Lin-gu-is-tik
Lis-ten-en-des
	     Lo-kal-an-ga-ben
             mathe-ma-tische
	     Mehr-deu-tig-keiten
	     Mehr-deu-tig-keit
             Merk-mal-struk-tur
miss-trau-en
Mit-tel-feld-ele-men-te
Mor-phol-o-gy
New-meyer
    nicht-erga-tives
             nicht-erga-tiven
nicht-idio-ma-tisch
nicht-idio-ma-tische
nicht-idio-ma-tischer
nicht-idio-ma-tischen
             %Nominal-phrase
             %Nominal-umgebungen
             NON-LOC
             Ober-feld-um-stel-lung
             Ob-jekt-an-he-bung
Ob-jekt-an-he-bungs-verb
Ob-jekt-an-he-bungs-ver-ben
             Ob-li-que-ness
             Para-phra-sen
             Parse-stra-te-gien
Par-ti-kel-verb-re-gel
Par-ti-zip-ein-trag
Pas-siv-ana-ly-se
Pas-siv-ana-ly-sen
             Pos-ses-siv-pro-no-men
             Pos-ses-siv-pro-no-mi-na
Pro-blem
Pro-jekt-er-geb-nis-se
Quar-tals-er-fol-ge
Rat-ten-fä-nger
Re-dun-danz
Re-dun-danz-re-gel
Re-dun-danz-re-geln
             re-fe-renz-iden-tisch
Re-fe-renz-ob-jekt
             Re-la-tiv-satz-ele-ment
	     Rest-feld-ele-men-te
Satz-en-de
Satz-mus-tern
             sub-sum-iert
Schwär-me-rei
se-man-tics
Se-man-tik-spe-zi-fi-ka-tion
	     Si-tu-ations-se-man-tik
Sprach-er-werb
Sprach-er-werbs
Sprach-er-werbs-phä-no-me-ne
Sprach-er-werbs-pro-blem
Sprach-er-werbs-theo-rien
sprach-spe-zi-fisch
sprach-spe-zi-fische
sprach-spe-zi-fisch-es
             Stan-dard-ab-fol-ge
Stan-ford
Sta-tis-tik-kom-po-nen-ten
sta-tis-tische
Struk-tur-tei-lung
Sub-jekt-ex-trak-tion
             syn-tak-tische
Ta-ges-leis-tung
teil-idio-ma-tisch
teil-idio-ma-ti-schen
             Teil-spezi-fi-ka-tio-nen
             Teil-spezi-fi-ka-tion
The-o-ret-i-cal
Tür-ki-schen
             Typ-beschrei-bung
             Typ-beschrei-bung-en
             Typ-hier-ar-chie
             Typ-konzept
             Typ-sys-tem
	     unter-schied-licher
Ver-ar-bei-tungs-sys-tems
             %Ver-ben
             %Verbal-phrasen-teilen
             %Verbal-pro-jek-tion
             %Verbal-pro-jek-tion-en
             Ver-bal-kom-plex-unter-bre-chung
Verb-be-we-gung
Verb-be-we-gungs-spur
Verb-ebe-ne
             Verb-end-stel-lung
             Verb-en-dung
             Verb-en-dung-en
Verb-erst-le-xi-kon-re-gel
	     Verb-erst-stel-lung
Verb-erst-sät-ze
Verb-erst-sät-zen
Verb-letzt-sät-ze
Verb-letzt-sät-zen
Verb-letzt-stel-lung
Verb-letzt-po-si-tion
	     Verb-zweit-stel-lung
             Verb-form
             Verb-kom-ple-men-ten
             Verb-kom-ple-men-te
Verb-mo-bil
             Verb-phra-se
             Verb-phra-sen
Verb-pro-jek-tion
Verb-pro-jek-tion-en
             Vor-le-sung
             Vor-feld-el-lip-se
             Vor-feld-el-lip-sen
	     ver-wen-det-en
             war-um
             War-um
weg-ge-las-sen
             Wur-zel-kno-ten
             Zusammen-setzen
             Zu-stands-pas-siv
             Zu-stands-pas-sivs
Wort-ein-heit
Wun-der-lich
Fink-bei-ner
Zeit-schrift-en-ar-ti-kel
}
\hyphenation{
Adjektiv-stamm-lexikon
Ar-gu-ment-rol-le
Ar-gu-ment-rol-len
Ar-gu-ment-rollen-rea-li-sie-rung-en
Aus-schluss-klau-sel
As-pekt
Bay-er
             be-schrei-ben
Bei-spiel-ab-lei-tung
Be-schrän-kun-gen
             Bilder-nomina
             Bin-dungs-theorie
Cog-ni-tive
Con-fer-ence
con-straint
con-straints
CO-LING
             Do-mi-nanz-sche-ma
             Do-mi-nanz-sche-ma-ta
De-mons-tra-tiv-pro-no-men
Dem-ske
Dis-kus-sion
Druck-erzeug-nis
Druck-erzeug-nis-sen
dunk-les
ent-wi-ckeln
	     Er-wei-te-rung
             %Flexions-merkmale
Er-werbs-pha-se
Er-werbs-pro-blem
Er-werbs-pro-zess-es
Fi-nit-heit
Funk-tions-verb-ge-fü-ges
Gar-aus
ge-typ-ter
             Gram-ma-tik
Gram-ma-tik-re-gel
Gram-ma-tik-re-geln
             Grund-for-ma-lis-mus
Grund-ope-ra-ti-on-en
Hilfs-verb-in-ver-sion
             Im-ple-men-tation
In-kon-sis-tenz-en
In-ter-na-tio-nal
             In-ter-net-an-bie-ter
	     in-tran-si-tiv
	     in-tran-si-tiv-es
             In-fi-ni-tiv-verb-phra-se
Jo-shi
             Kasus-an-for-de-rung
	     Ka-te-go-rien
	     Ka-te-go-rien-in-ven-tar
             Kon-so-nan-ten-um-wand-lung
             Kon-struk-tio-nen
Kop-fes
Kopf-merk-mals-prin-zip
             Kopf-toch-ter
             Kor-pus-an-fra-gen
             Kor-pus-an-fra-ge
             le-xi-ka-lisch
lin-gu-is-tisch-er
Lin-gu-is-tik
Lis-ten-en-des
	     Lo-kal-an-ga-ben
             mathe-ma-tische
	     Mehr-deu-tig-keiten
	     Mehr-deu-tig-keit
             Merk-mal-struk-tur
miss-trau-en
Mit-tel-feld-ele-men-te
Mor-phol-o-gy
New-meyer
    nicht-erga-tives
             nicht-erga-tiven
nicht-idio-ma-tisch
nicht-idio-ma-tische
nicht-idio-ma-tischer
nicht-idio-ma-tischen
             %Nominal-phrase
             %Nominal-umgebungen
             NON-LOC
             Ober-feld-um-stel-lung
             Ob-jekt-an-he-bung
Ob-jekt-an-he-bungs-verb
Ob-jekt-an-he-bungs-ver-ben
             Ob-li-que-ness
             Para-phra-sen
             Parse-stra-te-gien
Par-ti-kel-verb-re-gel
Par-ti-zip-ein-trag
Pas-siv-ana-ly-se
Pas-siv-ana-ly-sen
             Pos-ses-siv-pro-no-men
             Pos-ses-siv-pro-no-mi-na
Pro-blem
Pro-jekt-er-geb-nis-se
Quar-tals-er-fol-ge
Rat-ten-fä-nger
Re-dun-danz
Re-dun-danz-re-gel
Re-dun-danz-re-geln
             re-fe-renz-iden-tisch
Re-fe-renz-ob-jekt
             Re-la-tiv-satz-ele-ment
	     Rest-feld-ele-men-te
Satz-en-de
Satz-mus-tern
             sub-sum-iert
Schwär-me-rei
se-man-tics
Se-man-tik-spe-zi-fi-ka-tion
	     Si-tu-ations-se-man-tik
Sprach-er-werb
Sprach-er-werbs
Sprach-er-werbs-phä-no-me-ne
Sprach-er-werbs-pro-blem
Sprach-er-werbs-theo-rien
sprach-spe-zi-fisch
sprach-spe-zi-fische
sprach-spe-zi-fisch-es
             Stan-dard-ab-fol-ge
Stan-ford
Sta-tis-tik-kom-po-nen-ten
sta-tis-tische
Struk-tur-tei-lung
Sub-jekt-ex-trak-tion
             syn-tak-tische
Ta-ges-leis-tung
teil-idio-ma-tisch
teil-idio-ma-ti-schen
             Teil-spezi-fi-ka-tio-nen
             Teil-spezi-fi-ka-tion
The-o-ret-i-cal
Tür-ki-schen
             Typ-beschrei-bung
             Typ-beschrei-bung-en
             Typ-hier-ar-chie
             Typ-konzept
             Typ-sys-tem
	     unter-schied-licher
Ver-ar-bei-tungs-sys-tems
             %Ver-ben
             %Verbal-phrasen-teilen
             %Verbal-pro-jek-tion
             %Verbal-pro-jek-tion-en
             Ver-bal-kom-plex-unter-bre-chung
Verb-be-we-gung
Verb-be-we-gungs-spur
Verb-ebe-ne
             Verb-end-stel-lung
             Verb-en-dung
             Verb-en-dung-en
Verb-erst-le-xi-kon-re-gel
	     Verb-erst-stel-lung
Verb-erst-sät-ze
Verb-erst-sät-zen
Verb-letzt-sät-ze
Verb-letzt-sät-zen
Verb-letzt-stel-lung
Verb-letzt-po-si-tion
	     Verb-zweit-stel-lung
             Verb-form
             Verb-kom-ple-men-ten
             Verb-kom-ple-men-te
Verb-mo-bil
             Verb-phra-se
             Verb-phra-sen
Verb-pro-jek-tion
Verb-pro-jek-tion-en
             Vor-le-sung
             Vor-feld-el-lip-se
             Vor-feld-el-lip-sen
	     ver-wen-det-en
             war-um
             War-um
weg-ge-las-sen
             Wur-zel-kno-ten
             Zusammen-setzen
             Zu-stands-pas-siv
             Zu-stands-pas-sivs
Wort-ein-heit
Wun-der-lich
Fink-bei-ner
Zeit-schrift-en-ar-ti-kel
}


%% -*- coding:utf-8 -*-
%\languageshorthands{ngerman}\useshorthands*{"}

%%%%%%%%%%%%%%%%%%%%%%%%%%%%%%%%%%%%%%%%%%%%%%%%%%%%%%%%%%%%
%
% gb4e

% fixes problem with to much vertical space between \zl and \eal due to the \nopagebreak
% command.
\makeatletter
\def\@exe[#1]{\ifnum \@xnumdepth >0%
                 \if@xrec\@exrecwarn\fi%
                 \if@noftnote\@exrecwarn\fi%
                 \@xnumdepth0\@listdepth0\@xrectrue%
                 \save@counters%
              \fi%
                 \advance\@xnumdepth \@ne \@@xsi%
                 \if@noftnote%
                        \begin{list}{(\thexnumi)}%
                        {\usecounter{xnumi}\@subex{#1}{\@gblabelsep}{0em}%
                        \setcounter{xnumi}{\value{equation}}}
% this is commented out here since it causes additional space between \zl and \eal 06.06.2020
%                        \nopagebreak}%
                 \else%
                        \begin{list}{(\roman{xnumi})}%
                        {\usecounter{xnumi}\@subex{(iiv)}{\@gblabelsep}{\footexindent}%
                        \setcounter{xnumi}{\value{fnx}}}%
                 \fi}
\makeatother

\makeatletter
\def\eas{\ifnum\@xnumdepth=0\begin{exe}[(34)]\else\begin{xlist}[iv.]\fi\ex\begin{tabular}[t]{@{}p{0.999\linewidth}@{}}\leavevmode}
\makeatother



%%%%%%%%%%%%%%%%%%%%%%%%%%%%%%%%%%%%%%%%%%%%%%%%%%%%%%%%%%
%
% biblatex

% biblatex sets the option autolang=hyphens
%
% This disables language shorthands. To avoid this, the hyphens code can be redefined
%
% https://tex.stackexchange.com/a/548047/18561

\makeatletter
\def\hyphenrules#1{%
  \edef\bbl@tempf{#1}%
  \bbl@fixname\bbl@tempf
  \bbl@iflanguage\bbl@tempf{%
    \expandafter\bbl@patterns\expandafter{\bbl@tempf}%
    \expandafter\ifx\csname\bbl@tempf hyphenmins\endcsname\relax
      \set@hyphenmins\tw@\thr@@\relax
    \else
      \expandafter\expandafter\expandafter\set@hyphenmins
      \csname\bbl@tempf hyphenmins\endcsname\relax
    \fi}}
\makeatother


% the package defined \attop in a way that produced a box that has textwidth
%
\def\attop#1{\leavevmode\begin{minipage}[t]{.995\linewidth}\strut\vskip-\baselineskip\begin{minipage}[t]{.995\linewidth}#1\end{minipage}\end{minipage}}


%%%%%%%%%%%%%%%%%%%%%%%%%%%%%%%%%%%%%%%%%%%%%%%%%%%%%%%%%%%%%%%%%%%%


% Don't do this at home. I do not like the smaller font for captions.
% This does not work. Throw out package caption in langscibook
% \captionsetup{%
% font={%
% stretch=1%.8%
% ,normalsize%,small%
% },%
% width=\textwidth%.8\textwidth
% }
% \setcaphanging


% this was loaded when newtxmath was used as an option

    %% Unfortunately, this is NOT extensively tested!
%            \usepackage{xpatch}
%            \xpretocmd{\textsuperscript}
%            {{\sbox0{$\textstyle x$}}}


\usepackage[figuresright]{rotating}

\usepackage[german]{varioref}
% do not stop and warn! This will be tested in the final version
\vrefwarning

\usepackage{pstricks,pst-node}

\usepackage{tikz-qtree}
% has strange side effects
%\tikzset{every tree node/.style={align=left, anchor=north}}
\tikzset{every roof node/.append style={inner sep=0.1pt,text height=2ex,text depth=0.3ex}}


\newcommand{\NOTE}[1]{}
%\newcommand{\NOTE}[1]{\marginpar{#1}}

\memoizeset{
  memo filename prefix={hpsg.memo.dir/},
  register=\todo{O{}+m},
  prevent=\todo,
}


% \renewcommand{\itd}[1]{}
% \renewcommand{\itdobl}[1]{}
% \renewcommand{\itdopt}[1]{}

% \renewcommand{\addpages}{}
% \renewcommand{\addsource}{}
% \renewcommand{\addglosses}{}


%\bibliography{germanic}
\bibliography{bib-abbr,biblio-xdata-en,biblio}


\input hpsg-lehrbuch-include

\end{document}

%%% Local Variables:
%%% TeX-command-extra-options: "-shell-escape"
%%% End:
