%% -*- coding:utf-8 -*-
%%%%%%%%%%%%%%%%%%%%%%%%%%%%%%%%%%%%%%%%%%%%%%%%%%%%%%%%%
%%   $RCSfile: 4-hpsg-komplementation.tex,v $
%%  $Revision: 1.20 $
%%      $Date: 2007/02/13 11:00:11 $
%%     Author: Stefan Mueller (CL Uni-Bremen)
%%    Purpose: 
%%   Language: LaTeX
%%%%%%%%%%%%%%%%%%%%%%%%%%%%%%%%%%%%%%%%%%%%%%%%%%%%%%%%%


\chapter{Dominanzstrukturen und Prinzipien}
\label{chap-komplementation}

\begin{figure}
\centerline{%
\begin{forest}
%sm edges
[NP
  [Det [dem]]
  [N
    [Eichhörnchen]]]
\end{forest}}
\caption{\label{fig-dem-Eichhörnchen}\emph{dem Eichhörnchen}}
\end{figure}

\begin{figure}
\centering
\centerline{%
\begin{forest}
%sm edges
[V
  [NP
    [Aicke]]
  [V
    [NP
      [Det [dem]]
      [N [Eichhörnchen]]]
    [V
      [NP
        [Det [die]]
        [N [Nuss]]]
      [V
        [gibt]]]]]
\end{forest}}
\caption{\label{fig-bin-Aicke-dem-Eichhörnchen-die-Nuss-gibt}Binär verzweigende Kopf-Komplement-Strukturen}
\end{figure}


% \begin{figure}
% \centering
% \centerline{%
% \begin{forest}
% sm edges
% [{V[comps \sliste{} ]}
%   [{\ibox{1} NP[type{nom}]}
%     [Aicke]]
%   [{V[comps \sliste{ \ibox{1} }]}
%     [{\ibox{2} NP[type{dat}]} 
%       [dem Eichhörnchen, roof]]
%     [{V[comps \sliste{ \ibox{1}, \ibox{2} }]}
%       [{\ibox{3} NP[type{acc}]}
%         [die Nuss,roof]]
%       [{V[comps \sliste{ \ibox{1}, \ibox{2}, \ibox{3} }]}
%         [gibt]]]]]
% \end{forest}}
% \caption{\label{fig-saettigung-valenz}Abarbeitung der Valenzliste des Verbs}
% \end{figure}


% \begin{figure}
% \settowidth{\offset}{V[type{fi}}
% \settowidth{\offsetup}{V[type{fin}}
% \centerline{
% \begin{forest}
% sm edges, for tree={l+=\baselineskip}
% [V{[type{fin}, comps \eliste]}, name=fin1
% 	[\ibox{1} NP{[type{nom}]}
% 		[Aicke]]
% 	[V{[type{fin}, comps \sliste{ \ibox{1} }]}, name=fin2
% 		[\ibox{2} NP{[\textit{dat}]}
% 			[dem Eichhörnchen,roof]]
% 		[V{[type{fin}, comps \sliste{ \ibox{1}, \ibox{2} }]}, name=fin3
% 			[\ibox{3} NP{[\textit{acc}]}
% 				[die Nuss,roof]]
% 			[V{[type{fin}, comps \sliste{ \ibox{1}, \ibox{2}, \ibox{3} }]}, name=fin4
% 				[gibt]]]]]	
% tikz={\draw[<->] ($(fin1.south west)+(\offsetup,0)$) to ($(fin2.north west)+(\offset,0)$);
%       \draw[<->] ($(fin2.south west)+(\offsetup,0)$) to ($(fin3.north west)+(\offset,0)$);
%       \draw[<->] ($(fin3.south west)+(\offsetup,0)$) to ($(fin4.north west)+(\offset,0)$);}
% \end{forest}
% }
% \caption{\label{fig-projektion-head-feat}Projektion der Kopfmerkmale des Verbs}
% \end{figure}
% \begin{figure}
% \centerline{%
% \begin{forest}
% type hierarchy
% [sign,
%    calign=fixed angles,
%    calign angle=60
%   [word]
%   [phrase
%     [non-headed-phrase]
%     [headed-phrase
%       [head-specifier-phrase]
%       [head-complement-phrase]
%       [head-adjunct-phrase]]]]
% \end{forest}}
% \caption{\label{fig-type-sign}Typhierarchie für type{sign}: alle Untertypen von type{headed"=phrase} erben Beschränkungen.}
% \end{figure}




