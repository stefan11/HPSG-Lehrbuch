%% -*- coding:utf-8 -*-

\usepackage{csquotes}

% \up for relative clause chapter


% http://tex.stackexchange.com/questions/38607/no-room-for-a-new-dimen
%\usepackage{etex}\reserveinserts{28}


% for subnodes in trees
%\usepackage{tcolorbox}
%\tcbuselibrary{skins}
%\newtcbox{\mybox}[1][]{empty,shrink tight,nobeforeafter,on line,before upper=\vphantom{gM},remember as=#1,top=2pt,bottom=2pt}



% http://tex.stackexchange.com/questions/229500/tikzmark-and-xelatex
% temporary fix, remove later
%\newcount\pdftexversion \pdftexversion140 \def\pgfsysdriver{pgfsys-dvipdfm.def} \usepackage{tikz} \usetikzlibrary{tikzmark}

% for MRS
\usetikzlibrary{tikzmark}

% for bezier curves (bounding box) for scoped MRSes
\usetikzlibrary{bbox}

% This is a style, that causes \draw to always start at the south of a node.
% It is used in MRSes. Handles are defined with \subnode (\subnode{h3}{h3} and instead of drawing
% from h3.south we can then draw from h3.
% invoced by:
% \begin{tikzpicture}[remember picture,->,dashed,from south line to]
% 
\tikzset{from south line to/.style={to path=(\tikztostart.south)--(\tikztotarget)\tikztonodes}}

\newcounter{mrstikz}
\AddToHook{env/tikzpicture/before}{\stepcounter{mrstikz}\tikzset{tikzmark prefix=mrs-tikz\themrstikz}}

\tikzset{mrs/.style={remember picture,from south line to,->,dashed}}
\tikzset{scoped mrs/.style={remember picture,from south line to,bezier bounding box}}


% \justify to switch of \raggedright in translations
%\usepackage{ragged2e}



%\usepackage{metalogo} % xelatex

\usepackage{multicol}

\usepackage{bookmark}

%\usepackage{my-ccg-ohne-colortbl}




% This has side effects on my-ccg commands do no know why
%\usepackage{./langsci/langsci-optional}

\usepackage{todonotes}

% used to be in this package
\providecommand{\citegen}{}
\renewcommand{\citegen}[2][]{\citeauthor{#2}'s (\citeyear*[#1]{#2})}
\providecommand{\lsptoprule}{}
\renewcommand{\lsptoprule}{\midrule\toprule}
\providecommand{\lspbottomrule}{}
\renewcommand{\lspbottomrule}{\bottomrule\midrule}
\providecommand{\largerpage}{}
\renewcommand{\largerpage}[1][1]{\enlargethispage{#1\baselineskip}}


\usepackage{./styles/oneline}

%\let\oneline\onelinetextwidthhack


\usepackage{./styles/biblatex-series-number-checks}

% not needed for 2024. It does not exist.
%\usepackage{langsci-basic}
%\usepackage{langsci-optional}
%\usepackage{langsci-lgr}
%\newcommand{\M}{\textsc{m}{}\xspace}        % use at own risk



\usepackage{langsci-branding}
%\usepackage[danger]{langsci-lgr}
\usepackage{langsci-lgr}

\usepackage{graphicx}

\usepackage{soul}

%\usepackage{./styles/mycommands}% \spacebr


\usepackage{langsci-gb4e}

\let\eal\ealnoraggedright

\usepackage{styles/jambox}

% fixes problem with to much vertical space between \zl and \eal due to the \nopagebreak
% command.
\makeatletter
\def\@exe[#1]{\ifnum \@xnumdepth >0%
                 \if@xrec\@exrecwarn\fi%
                 \if@noftnote\@exrecwarn\fi%
                 \@xnumdepth0\@listdepth0\@xrectrue%
                 \save@counters%
              \fi%
                 \advance\@xnumdepth \@ne \@@xsi%
                 \if@noftnote%
                        \begin{list}{(\thexnumi)}%
                        {\usecounter{xnumi}\@subex{#1}{\@gblabelsep}{0em}%
                        \setcounter{xnumi}{\value{equation}}}
% this is commented out here since it causes additional space between \zl and \eal 06.06.2020
%                        \nopagebreak}%
                 \else%
                        \begin{list}{(\roman{xnumi})}%
                        {\usecounter{xnumi}\@subex{(iiv)}{\@gblabelsep}{\footexindent}%
                        \setcounter{xnumi}{\value{fnx}}}%
                 \fi}
\makeatother


\makeatletter
\def\eas{\ifnum\@xnumdepth=0\begin{exe}[(34)]\else\begin{xlist}[iv.]\fi\ex\begin{tabular}[t]{@{}p{.99\linewidth}@{}}}
\makeatother

\settowidth\jamwidth{(German)}

%\usepackage{subfig}

%\renewcommand{\xbar}{X̅\xspace}


%\usepackage[external,linguistics]{styles/forest/forest}


% for reasons I do not understand this cannot be moved further down and 
% the loading of forest further down cannot be removed. St. Mü. 26.01.2017
% It breaks the dependency grammar trees in forest.
\usepackage{langsci-forest-setup}

% for Germanic history tree
\useforestlibrary{edges} 

%\usepackage{memoize} 
%\usepackage{nomemoize} % use this if memoize caues chaos.
%\memoizeset{
%  memo filename prefix={germanic.memo.dir/},
%}
% uncomment this if your figures change frequently and you do not want memoize to externalize them.
%\memoizeset{readonly}




% has to be loaded after forest-setup because of incompatibilities with the dg-style.
%\usepackage{german}\selectlanguage{ngerman}
%\usepackage[ngerman]{babel}

%\selectlanguage{ngerman}


\usepackage{./styles/merkmalstruktur,./styles/makros.2020,./styles/my-xspace,./styles/abbrev,./styles/article-ex,./styles/additional-langsci-index-shortcuts,
./styles/de-date,./styles/my-theorems,styles/mycommands}

\usepackage{langsci-avm}

\let\vref\ref
\newcommand\figuresref[2]{%
Figure~\ref{#1} and Figure~\ref{#2}%
}

\if0

% loaded in macros.2e \usepackage[english]{varioref}
% do not stop and warn! This will be tested in the final version
%\usepackage[english]{varioref}
%\vrefwarning

\newcommand\thefiguresref[2]{%
 \vrefpagenum\firstnum{#1}%
 \vrefpagenum\secondnum{#2}%
\ifthenelse{\equal\firstnum\secondnum}%
{the Figures~\ref{#1} and~\ref{#2}\vpageref[]{#1}}%
{the Figures~\ref{#1} and~\ref{#2}}}

\newcommand\figuresref[2]{%
 \vrefpagenum\firstnum{#1}%
 \vrefpagenum\secondnum{#2}%
\ifthenelse{\equal\firstnum\secondnum}%
{\iflanguage{german}{%
die Abbildungen~\ref{#1} und~\ref{#2} \vpageref[]{#1}%
}% end German
{Figures~\ref{#1} and~\ref{#2} \vpageref[]{#1}}}%
{\iflanguage{german}{%
Abbildung~\ref{#1}\vpageref[]{#1} und Abbildung~\ref{#2} \vpageref{#2}%
}% end German
{%
Figure~\ref{#1}\vpageref[]{#1} and Figure~\ref{#2} \vpageref{#2}}}%
}


\newcommand\pagerefonlyifdifferent[2]{%
\vrefpagenum\firstnum{#1}%
\vrefpagenum\secondnum{#2}%
\ifthenelse{\equal\firstnum\secondnum}%
{}
{\vpageref{#2}}}



\newcommand\figuretwoonsamepagesref[2]{%
 \vrefpagenum\firstnum{#1}%
 \vrefpagenum\secondnum{#2}%
\ifthenelse{\equal\firstnum\secondnum}%
%{ on the same page as Figure~\ref{#1}}%
{}%
{ on page~\vpageref{#2}}%
}
\renewcommand{\reftextcurrent}{}

\newcommand\refORregion[2]{%
 \vrefpagenum\firstnum{#1}%
 \vrefpagenum\secondnum{#2}%
\ifthenelse{\equal\firstnum\secondnum}%
{\pageref{#1}}%
{\pageref{#1}--\pageref{#2}}%
}

%\let\reftextfaceafter\reftextafter
%\let\reftextfacebefore\reftextbefore

\addto\extrasenglish{%
     \renewcommand\reftextfaceafter {\reftextafter}%
     \renewcommand\reftextfacebefore {\reftextbefore}%
}

\fi

% draw a grid for getting the coordinates
\usepackage{./styles/tikz-grid}

% Adapted from https://tex.stackexchange.com/questions/255234/how-does-one-pick-control-points-to-control-b%C3%A9zier-curves-in-tikz
% \DrawControl{(12,4)}{1}\DrawControl{(-4,4)}{2};  
\newcommand\DrawControl[2]{
  \node[circle,fill=red,inner sep=2pt,label={above:$#1$},label={[black]below:{\footnotesize#2}}] at #1 {};
}

% for offsets in trees
%\newlength{\offset}
%\newlength{\offsetup}

\usepackage{./styles/de-hyp-utf8}

\usepackage{appendix}



% http://tex.stackexchange.com/questions/3223/subscripts-for-primed-variables
%
% to get 
% {}[ af   [~]\sub{V} ]\sub{V$'$}
%
% typeset properly. Thanks, Sebastian.
%
\usepackage{subdepth}


%\usepackage{caption}


% alt
%\usepackage[mdframed]{langsci-tbls}
\usepackage{langsci-tbls}
% do not need identation for enumerate since we are in a box anyway.
\usepackage{enumitem}



% for abbreviations 02.05.2020
\usepackage{tabularx}


% remove when finished
\usepackage{proofread}


% without loading german.sty abbreviations with "= following them do not work. babel does not
% recognize the " escape provided in my-xspace. St. Mü. 06.03.2025
%\usepackage{ngerman}


% \usepackage{pst-node}
% \psset{nodesep=4pt}

\definecolor{fugreen}{rgb}{0.6,0.8,0}


%\usepackage[libertine]{newtxmath}


\usepackage[figuresright]{rotating}

\usepackage[german]{varioref}
% do not stop and warn! This will be tested in the final version
\vrefwarning
