%% -*- coding:utf-8 -*-
\author{Stefan Müller}
\title{Head-Driven Phrase Structure Grammar}
\subtitle{Eine Einführung\newlineCover 
\textit{\LARGE {\spaceskip=3.5pt Vierte überarbeitete Auf\/lage}}}
\renewcommand{\lsImpressumCitationText}{%
Stefan Müller. \lsYear. \textit{Head-Driven Phrase Structure Grammar: Eine Einführung. Vierte
  überarbeitete Auf\kern0pt lage}. (Textbooks in Language Sciences~). Berlin: Language Science Press.
}

\typesetter{Stefan Müller}


\BackTitle{Head-Driven Phrase Structure Grammar}
\BackBody{
Das Buch ist eine abgeschlossene Einführung in das Framework der Head-Driven Phrase Structure Grammar. In einem einführenden Kapitel wird der Übergang von einfachen Phrasenstrukturgrammatiken zu den komplexeren Repräsentationen mit Merkmalbeschreibungen motiviert. Das zweite Kapitel führt allgemein verständlich in den Formalismus der Merkmalstrukturen ein. In den verbleibenden Kapiteln werden phänomenbezogen verschiedene Grammatikbereiche diskutiert, wobei die Analysen jeweils für das Deutsche ausgearbeitet werden. Der Schwerpunkt liegt hierbei auf syntaktischen Phänomenen (Konstituentenstruktur, Konstituentenstellung, Kasus, Passiv, Kongruenz, Verbalkomplex und Partikelverben), die Morphologie (Flexion und Derivation) wird aber ebenfalls behandelt, und es wird gezeigt, wie die Bedeutung morphologischer und syntaktischer Konstruktionen kompositional bestimmt werden kann.

In den Analyseabschnitten wurde bewußt auf die Diskussion von Alternativen verzichtet. Den
Alternativen ist jeweils ein eigener Abschnitt gewidmet, der dem fortgeschrittenen Leser den
Vergleich mit anderen HPSG-Ansätzen aber auch mit Ansätzen aus der Konstruktionsgrammatik (CxG), der
Lexikalisch-Funktionalen-Grammatik (LFG) und mit Varianten der Government \& Binding-Theorie (GB)
ermöglichen soll. Bei der Diskussion der Alternativen spielen sowohl formale Eigenschaften der
jeweiligen Theorien als auch empirische Befunde aus dem Deutschen und anderen Sprachen eine Rolle. 

Das Buch mit der Virtuellen Maschine für Grammix richtet sich gleichermaßen an Lehrende und Lernende
der Germanistik, der allgemeinen Sprachwissenschaft und der Computerlinguistik. 

\bigskip
%~
%\smallskip
\vfill


\medskip

\noindent
„Der Inhalt des Buches ist komplex und behandelt eine nicht einfache Thematik. Für Studierende
sprachwissenschaftlicher Fächer ist dieses Lehrbuch aber durchaus geeignet und auch in seinem Aufbau
didaktisch wohlüberlegt.“ Jens Fleischhauer. 2010. \emph{Roterdorn}. 

\medskip

\noindent
„Einerseits ist in der vorliegenden Einführung ein sehr gut gegliederter didaktischer Aufbau
hervorzuheben, andererseits kamen viele neue Beispielanalysen komplexer syntaktischer
Strukturen der deutschen Sprache dazu, was für Neulinge auf dem Gebiet einen echten Gewinn für die
Verarbeitung dieser absolut nicht einfachen Materie bringt und auch für Insider neue Perspektiven
der Erweiterung dieses faszinierenden Ansatzes bietet. [\ldots] Hervorzuheben ist, dass Müller es
versteht, sehr klar und kleinschrittig viele weitere wichtige syntaktische Phänomene des Deutschen
nachvollziehbar einzuführen.“ \href{https://doi.org/10.1515/infodaf-2009-2-361}{Markus J. Weininger. 2009. \emph{Informationen Deutsch als Fremdsprache}}. 


\medskip

\noindent
%„HPSG – \rot{H}ier \rot{p}räsentiert \rot{s}ich \rot{G}rammatik in Theorie \& Anwendung von kompetentester Hand. Nicht nur für Syntaxinteressierte ein „Must-Have“ und noch dazu ein „Easiest-to-Have“, da Open Access dank Language Science Press!“ Hubert Haider, 2024.

\noindent
„HPSG – \emph{Hier präsentiert sich Grammatik} in Theorie \& Anwendung von kompetentester
Hand. Nicht nur für Syntaxinteressierte ein „Must-Have“ und noch dazu ein „Easiest-to-Have“, da Open
Access dank Language Science Press!“ Hubert Haider. 2024.

\medskip

\noindent
„Ich freue mich sehr, dass Stefan Müllers exzellente Einführung  in die HPSG nun in einer
überarbeiteten Auflage bei Language Science Press als Open-Access-Publikation erscheint, womit
Studierende und Forscher/innen eine einfache Möglichkeit erhalten, sich in dieses wichtige
Grammatikmodell detailliert einzuarbeiten und über den neuesten Stand seiner Entwicklung zu
informieren.“ Joachim Jacobs, 2024.

}



\dedication{Ich widme dieses Buch Brigitte Narr als Dank für ihren unermüdlichen Einsatz für die
  Sprachwissenschaft.\\Sie gehört definitiv zu den Guten im Verlagswesen.}

% otherwise the distance between lines is too big. St. Mü. 14.02.2025
\renewcommand{\lsDedicationFont}{\fontsize{15pt}{8mm}\selectfont}

%\renewcommand{\lsISBNdigital}{978-3-96110-408-6}
%\renewcommand{\lsISBNhardcover}{978-3-98554-066-2} Lehrbuch -> no hardcover
%\renewcommand{\lsISBNsoftcover}{978-3-98554-066-2}
%\renewcommand{\lsBookDOI}{10.5281/zenodo.7733033}
\renewcommand{\lsSeries}{tbls} % use lowercase acronym, e.g. sidl, eotms, tgdi
%\renewcommand{\lsSeriesNumber}{12} %will be assigned when the book enters the proofreading stage
%% \renewcommand{\lsURL}{http://langsci-press.org/catalog/book/25} % contact the coordinator for the right number
%\renewcommand{\lsID}{353}

% \proofreader{%
% Amir Ghorbanpour,
% Wilson Lui,
% Lachlan Mackenzie,
% Rebecca Madlener,
% Katja Politt,
% Janina Rado,
% Brett Reynolds,
% Lea Schäfer,
% Annika Schiefner,
% Troy E. Spier,
% George Walkden,
% Jeroen van de Weijer}