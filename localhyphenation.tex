%% -*- coding:utf-8 -*-
%% hyphenation points for line breaks
%% Normally, automatic hyphenation in LaTeX is very good
%% If a word is mis-hyphenated, add it to this file
%%
%% add information to TeX file before \begin{document} with:
%% %% -*- coding:utf-8 -*-
%% hyphenation points for line breaks
%% Normally, automatic hyphenation in LaTeX is very good
%% If a word is mis-hyphenated, add it to this file
%%
%% add information to TeX file before \begin{document} with:
%% %% -*- coding:utf-8 -*-
%% hyphenation points for line breaks
%% Normally, automatic hyphenation in LaTeX is very good
%% If a word is mis-hyphenated, add it to this file
%%
%% add information to TeX file before \begin{document} with:
%% %% -*- coding:utf-8 -*-
%% hyphenation points for line breaks
%% Normally, automatic hyphenation in LaTeX is very good
%% If a word is mis-hyphenated, add it to this file
%%
%% add information to TeX file before \begin{document} with:
%% \include{localhyphenation}
\hyphenation{
Adjektiv-stamm-lexikon
Ar-gu-ment-rol-le
Ar-gu-ment-rol-len
Ar-gu-ment-rollen-rea-li-sie-rung-en
Aus-schluss-klau-sel
As-pekt
Bay-er
             be-schrei-ben
Bei-spiel-ab-lei-tung
Be-schrän-kun-gen
             Bilder-nomina
             Bin-dungs-theorie
Cog-ni-tive
Con-fer-ence
con-straint
con-straints
CO-LING
             Do-mi-nanz-sche-ma
             Do-mi-nanz-sche-ma-ta
De-mons-tra-tiv-pro-no-men
Dem-ske
Dis-kus-sion
Druck-erzeug-nis
Druck-erzeug-nis-sen
dunk-les
ent-wi-ckeln
	     Er-wei-te-rung
             %Flexions-merkmale
Er-werbs-pha-se
Er-werbs-pro-blem
Er-werbs-pro-zess-es
Fi-nit-heit
Funk-tions-verb-ge-fü-ges
Gar-aus
ge-typ-ter
             Gram-ma-tik
Gram-ma-tik-re-gel
Gram-ma-tik-re-geln
             Grund-for-ma-lis-mus
Grund-ope-ra-ti-on-en
Hilfs-verb-in-ver-sion
             Im-ple-men-tation
In-kon-sis-tenz-en
In-ter-na-tio-nal
             In-ter-net-an-bie-ter
	     in-tran-si-tiv
	     in-tran-si-tiv-es
             In-fi-ni-tiv-verb-phra-se
Jo-shi
             Kasus-an-for-de-rung
	     Ka-te-go-rien
	     Ka-te-go-rien-in-ven-tar
             Kon-so-nan-ten-um-wand-lung
             Kon-struk-tio-nen
Kop-fes
Kopf-merk-mals-prin-zip
             Kopf-toch-ter
             Kor-pus-an-fra-gen
             Kor-pus-an-fra-ge
             le-xi-ka-lisch
lin-gu-is-tisch-er
Lin-gu-is-tik
Lis-ten-en-des
	     Lo-kal-an-ga-ben
             mathe-ma-tische
	     Mehr-deu-tig-keiten
	     Mehr-deu-tig-keit
             Merk-mal-struk-tur
miss-trau-en
Mit-tel-feld-ele-men-te
Mor-phol-o-gy
New-meyer
    nicht-erga-tives
             nicht-erga-tiven
nicht-idio-ma-tisch
nicht-idio-ma-tische
nicht-idio-ma-tischer
nicht-idio-ma-tischen
             %Nominal-phrase
             %Nominal-umgebungen
             NON-LOC
             Ober-feld-um-stel-lung
             Ob-jekt-an-he-bung
Ob-jekt-an-he-bungs-verb
Ob-jekt-an-he-bungs-ver-ben
             Ob-li-que-ness
             Para-phra-sen
             Parse-stra-te-gien
Par-ti-kel-verb-re-gel
Par-ti-zip-ein-trag
Pas-siv-ana-ly-se
Pas-siv-ana-ly-sen
             Pos-ses-siv-pro-no-men
             Pos-ses-siv-pro-no-mi-na
Pro-blem
Pro-jekt-er-geb-nis-se
Quar-tals-er-fol-ge
Rat-ten-fä-nger
Re-dun-danz
Re-dun-danz-re-gel
Re-dun-danz-re-geln
             re-fe-renz-iden-tisch
Re-fe-renz-ob-jekt
             Re-la-tiv-satz-ele-ment
	     Rest-feld-ele-men-te
Satz-en-de
Satz-mus-tern
             sub-sum-iert
Schwär-me-rei
se-man-tics
Se-man-tik-spe-zi-fi-ka-tion
	     Si-tu-ations-se-man-tik
Sprach-er-werb
Sprach-er-werbs
Sprach-er-werbs-phä-no-me-ne
Sprach-er-werbs-pro-blem
Sprach-er-werbs-theo-rien
sprach-spe-zi-fisch
sprach-spe-zi-fische
sprach-spe-zi-fisch-es
             Stan-dard-ab-fol-ge
Stan-ford
Sta-tis-tik-kom-po-nen-ten
sta-tis-tische
Struk-tur-tei-lung
Sub-jekt-ex-trak-tion
             syn-tak-tische
Ta-ges-leis-tung
teil-idio-ma-tisch
teil-idio-ma-ti-schen
             Teil-spezi-fi-ka-tio-nen
             Teil-spezi-fi-ka-tion
The-o-ret-i-cal
Tür-ki-schen
             Typ-beschrei-bung
             Typ-beschrei-bung-en
             Typ-hier-ar-chie
             Typ-konzept
             Typ-sys-tem
	     unter-schied-licher
Ver-ar-bei-tungs-sys-tems
             %Ver-ben
             %Verbal-phrasen-teilen
             %Verbal-pro-jek-tion
             %Verbal-pro-jek-tion-en
             Ver-bal-kom-plex-unter-bre-chung
Verb-be-we-gung
Verb-be-we-gungs-spur
Verb-ebe-ne
             Verb-end-stel-lung
             Verb-en-dung
             Verb-en-dung-en
Verb-erst-le-xi-kon-re-gel
	     Verb-erst-stel-lung
Verb-erst-sät-ze
Verb-erst-sät-zen
Verb-letzt-sät-ze
Verb-letzt-sät-zen
Verb-letzt-stel-lung
Verb-letzt-po-si-tion
	     Verb-zweit-stel-lung
             Verb-form
             Verb-kom-ple-men-ten
             Verb-kom-ple-men-te
Verb-mo-bil
             Verb-phra-se
             Verb-phra-sen
Verb-pro-jek-tion
Verb-pro-jek-tion-en
             Vor-le-sung
             Vor-feld-el-lip-se
             Vor-feld-el-lip-sen
	     ver-wen-det-en
             war-um
             War-um
weg-ge-las-sen
             Wur-zel-kno-ten
             Zusammen-setzen
             Zu-stands-pas-siv
             Zu-stands-pas-sivs
Wort-ein-heit
Wun-der-lich
Fink-bei-ner
Zeit-schrift-en-ar-ti-kel
}
\hyphenation{
Adjektiv-stamm-lexikon
Ar-gu-ment-rol-le
Ar-gu-ment-rol-len
Ar-gu-ment-rollen-rea-li-sie-rung-en
Aus-schluss-klau-sel
As-pekt
Bay-er
             be-schrei-ben
Bei-spiel-ab-lei-tung
Be-schrän-kun-gen
             Bilder-nomina
             Bin-dungs-theorie
Cog-ni-tive
Con-fer-ence
con-straint
con-straints
CO-LING
             Do-mi-nanz-sche-ma
             Do-mi-nanz-sche-ma-ta
De-mons-tra-tiv-pro-no-men
Dem-ske
Dis-kus-sion
Druck-erzeug-nis
Druck-erzeug-nis-sen
dunk-les
ent-wi-ckeln
	     Er-wei-te-rung
             %Flexions-merkmale
Er-werbs-pha-se
Er-werbs-pro-blem
Er-werbs-pro-zess-es
Fi-nit-heit
Funk-tions-verb-ge-fü-ges
Gar-aus
ge-typ-ter
             Gram-ma-tik
Gram-ma-tik-re-gel
Gram-ma-tik-re-geln
             Grund-for-ma-lis-mus
Grund-ope-ra-ti-on-en
Hilfs-verb-in-ver-sion
             Im-ple-men-tation
In-kon-sis-tenz-en
In-ter-na-tio-nal
             In-ter-net-an-bie-ter
	     in-tran-si-tiv
	     in-tran-si-tiv-es
             In-fi-ni-tiv-verb-phra-se
Jo-shi
             Kasus-an-for-de-rung
	     Ka-te-go-rien
	     Ka-te-go-rien-in-ven-tar
             Kon-so-nan-ten-um-wand-lung
             Kon-struk-tio-nen
Kop-fes
Kopf-merk-mals-prin-zip
             Kopf-toch-ter
             Kor-pus-an-fra-gen
             Kor-pus-an-fra-ge
             le-xi-ka-lisch
lin-gu-is-tisch-er
Lin-gu-is-tik
Lis-ten-en-des
	     Lo-kal-an-ga-ben
             mathe-ma-tische
	     Mehr-deu-tig-keiten
	     Mehr-deu-tig-keit
             Merk-mal-struk-tur
miss-trau-en
Mit-tel-feld-ele-men-te
Mor-phol-o-gy
New-meyer
    nicht-erga-tives
             nicht-erga-tiven
nicht-idio-ma-tisch
nicht-idio-ma-tische
nicht-idio-ma-tischer
nicht-idio-ma-tischen
             %Nominal-phrase
             %Nominal-umgebungen
             NON-LOC
             Ober-feld-um-stel-lung
             Ob-jekt-an-he-bung
Ob-jekt-an-he-bungs-verb
Ob-jekt-an-he-bungs-ver-ben
             Ob-li-que-ness
             Para-phra-sen
             Parse-stra-te-gien
Par-ti-kel-verb-re-gel
Par-ti-zip-ein-trag
Pas-siv-ana-ly-se
Pas-siv-ana-ly-sen
             Pos-ses-siv-pro-no-men
             Pos-ses-siv-pro-no-mi-na
Pro-blem
Pro-jekt-er-geb-nis-se
Quar-tals-er-fol-ge
Rat-ten-fä-nger
Re-dun-danz
Re-dun-danz-re-gel
Re-dun-danz-re-geln
             re-fe-renz-iden-tisch
Re-fe-renz-ob-jekt
             Re-la-tiv-satz-ele-ment
	     Rest-feld-ele-men-te
Satz-en-de
Satz-mus-tern
             sub-sum-iert
Schwär-me-rei
se-man-tics
Se-man-tik-spe-zi-fi-ka-tion
	     Si-tu-ations-se-man-tik
Sprach-er-werb
Sprach-er-werbs
Sprach-er-werbs-phä-no-me-ne
Sprach-er-werbs-pro-blem
Sprach-er-werbs-theo-rien
sprach-spe-zi-fisch
sprach-spe-zi-fische
sprach-spe-zi-fisch-es
             Stan-dard-ab-fol-ge
Stan-ford
Sta-tis-tik-kom-po-nen-ten
sta-tis-tische
Struk-tur-tei-lung
Sub-jekt-ex-trak-tion
             syn-tak-tische
Ta-ges-leis-tung
teil-idio-ma-tisch
teil-idio-ma-ti-schen
             Teil-spezi-fi-ka-tio-nen
             Teil-spezi-fi-ka-tion
The-o-ret-i-cal
Tür-ki-schen
             Typ-beschrei-bung
             Typ-beschrei-bung-en
             Typ-hier-ar-chie
             Typ-konzept
             Typ-sys-tem
	     unter-schied-licher
Ver-ar-bei-tungs-sys-tems
             %Ver-ben
             %Verbal-phrasen-teilen
             %Verbal-pro-jek-tion
             %Verbal-pro-jek-tion-en
             Ver-bal-kom-plex-unter-bre-chung
Verb-be-we-gung
Verb-be-we-gungs-spur
Verb-ebe-ne
             Verb-end-stel-lung
             Verb-en-dung
             Verb-en-dung-en
Verb-erst-le-xi-kon-re-gel
	     Verb-erst-stel-lung
Verb-erst-sät-ze
Verb-erst-sät-zen
Verb-letzt-sät-ze
Verb-letzt-sät-zen
Verb-letzt-stel-lung
Verb-letzt-po-si-tion
	     Verb-zweit-stel-lung
             Verb-form
             Verb-kom-ple-men-ten
             Verb-kom-ple-men-te
Verb-mo-bil
             Verb-phra-se
             Verb-phra-sen
Verb-pro-jek-tion
Verb-pro-jek-tion-en
             Vor-le-sung
             Vor-feld-el-lip-se
             Vor-feld-el-lip-sen
	     ver-wen-det-en
             war-um
             War-um
weg-ge-las-sen
             Wur-zel-kno-ten
             Zusammen-setzen
             Zu-stands-pas-siv
             Zu-stands-pas-sivs
Wort-ein-heit
Wun-der-lich
Fink-bei-ner
Zeit-schrift-en-ar-ti-kel
}
\hyphenation{
Adjektiv-stamm-lexikon
Ar-gu-ment-rol-le
Ar-gu-ment-rol-len
Ar-gu-ment-rollen-rea-li-sie-rung-en
Aus-schluss-klau-sel
As-pekt
Bay-er
             be-schrei-ben
Bei-spiel-ab-lei-tung
Be-schrän-kun-gen
             Bilder-nomina
             Bin-dungs-theorie
Cog-ni-tive
Con-fer-ence
con-straint
con-straints
CO-LING
             Do-mi-nanz-sche-ma
             Do-mi-nanz-sche-ma-ta
De-mons-tra-tiv-pro-no-men
Dem-ske
Dis-kus-sion
Druck-erzeug-nis
Druck-erzeug-nis-sen
dunk-les
ent-wi-ckeln
	     Er-wei-te-rung
             %Flexions-merkmale
Er-werbs-pha-se
Er-werbs-pro-blem
Er-werbs-pro-zess-es
Fi-nit-heit
Funk-tions-verb-ge-fü-ges
Gar-aus
ge-typ-ter
             Gram-ma-tik
Gram-ma-tik-re-gel
Gram-ma-tik-re-geln
             Grund-for-ma-lis-mus
Grund-ope-ra-ti-on-en
Hilfs-verb-in-ver-sion
             Im-ple-men-tation
In-kon-sis-tenz-en
In-ter-na-tio-nal
             In-ter-net-an-bie-ter
	     in-tran-si-tiv
	     in-tran-si-tiv-es
             In-fi-ni-tiv-verb-phra-se
Jo-shi
             Kasus-an-for-de-rung
	     Ka-te-go-rien
	     Ka-te-go-rien-in-ven-tar
             Kon-so-nan-ten-um-wand-lung
             Kon-struk-tio-nen
Kop-fes
Kopf-merk-mals-prin-zip
             Kopf-toch-ter
             Kor-pus-an-fra-gen
             Kor-pus-an-fra-ge
             le-xi-ka-lisch
lin-gu-is-tisch-er
Lin-gu-is-tik
Lis-ten-en-des
	     Lo-kal-an-ga-ben
             mathe-ma-tische
	     Mehr-deu-tig-keiten
	     Mehr-deu-tig-keit
             Merk-mal-struk-tur
miss-trau-en
Mit-tel-feld-ele-men-te
Mor-phol-o-gy
New-meyer
    nicht-erga-tives
             nicht-erga-tiven
nicht-idio-ma-tisch
nicht-idio-ma-tische
nicht-idio-ma-tischer
nicht-idio-ma-tischen
             %Nominal-phrase
             %Nominal-umgebungen
             NON-LOC
             Ober-feld-um-stel-lung
             Ob-jekt-an-he-bung
Ob-jekt-an-he-bungs-verb
Ob-jekt-an-he-bungs-ver-ben
             Ob-li-que-ness
             Para-phra-sen
             Parse-stra-te-gien
Par-ti-kel-verb-re-gel
Par-ti-zip-ein-trag
Pas-siv-ana-ly-se
Pas-siv-ana-ly-sen
             Pos-ses-siv-pro-no-men
             Pos-ses-siv-pro-no-mi-na
Pro-blem
Pro-jekt-er-geb-nis-se
Quar-tals-er-fol-ge
Rat-ten-fä-nger
Re-dun-danz
Re-dun-danz-re-gel
Re-dun-danz-re-geln
             re-fe-renz-iden-tisch
Re-fe-renz-ob-jekt
             Re-la-tiv-satz-ele-ment
	     Rest-feld-ele-men-te
Satz-en-de
Satz-mus-tern
             sub-sum-iert
Schwär-me-rei
se-man-tics
Se-man-tik-spe-zi-fi-ka-tion
	     Si-tu-ations-se-man-tik
Sprach-er-werb
Sprach-er-werbs
Sprach-er-werbs-phä-no-me-ne
Sprach-er-werbs-pro-blem
Sprach-er-werbs-theo-rien
sprach-spe-zi-fisch
sprach-spe-zi-fische
sprach-spe-zi-fisch-es
             Stan-dard-ab-fol-ge
Stan-ford
Sta-tis-tik-kom-po-nen-ten
sta-tis-tische
Struk-tur-tei-lung
Sub-jekt-ex-trak-tion
             syn-tak-tische
Ta-ges-leis-tung
teil-idio-ma-tisch
teil-idio-ma-ti-schen
             Teil-spezi-fi-ka-tio-nen
             Teil-spezi-fi-ka-tion
The-o-ret-i-cal
Tür-ki-schen
             Typ-beschrei-bung
             Typ-beschrei-bung-en
             Typ-hier-ar-chie
             Typ-konzept
             Typ-sys-tem
	     unter-schied-licher
Ver-ar-bei-tungs-sys-tems
             %Ver-ben
             %Verbal-phrasen-teilen
             %Verbal-pro-jek-tion
             %Verbal-pro-jek-tion-en
             Ver-bal-kom-plex-unter-bre-chung
Verb-be-we-gung
Verb-be-we-gungs-spur
Verb-ebe-ne
             Verb-end-stel-lung
             Verb-en-dung
             Verb-en-dung-en
Verb-erst-le-xi-kon-re-gel
	     Verb-erst-stel-lung
Verb-erst-sät-ze
Verb-erst-sät-zen
Verb-letzt-sät-ze
Verb-letzt-sät-zen
Verb-letzt-stel-lung
Verb-letzt-po-si-tion
	     Verb-zweit-stel-lung
             Verb-form
             Verb-kom-ple-men-ten
             Verb-kom-ple-men-te
Verb-mo-bil
             Verb-phra-se
             Verb-phra-sen
Verb-pro-jek-tion
Verb-pro-jek-tion-en
             Vor-le-sung
             Vor-feld-el-lip-se
             Vor-feld-el-lip-sen
	     ver-wen-det-en
             war-um
             War-um
weg-ge-las-sen
             Wur-zel-kno-ten
             Zusammen-setzen
             Zu-stands-pas-siv
             Zu-stands-pas-sivs
Wort-ein-heit
Wun-der-lich
Fink-bei-ner
Zeit-schrift-en-ar-ti-kel
}
\hyphenation{
Adjektiv-stamm-lexikon
Ar-gu-ment-rol-le
Ar-gu-ment-rol-len
Ar-gu-ment-rollen-rea-li-sie-rung-en
Aus-schluss-klau-sel
As-pekt
Bay-er
             be-schrei-ben
Bei-spiel-ab-lei-tung
Be-schrän-kun-gen
             Bilder-nomina
             Bin-dungs-theorie
Cog-ni-tive
Con-fer-ence
con-straint
con-straints
CO-LING
             Do-mi-nanz-sche-ma
             Do-mi-nanz-sche-ma-ta
De-mons-tra-tiv-pro-no-men
Dem-ske
Dis-kus-sion
Druck-erzeug-nis
Druck-erzeug-nis-sen
dunk-les
ent-wi-ckeln
	     Er-wei-te-rung
             %Flexions-merkmale
Er-werbs-pha-se
Er-werbs-pro-blem
Er-werbs-pro-zess-es
Fi-nit-heit
Funk-tions-verb-ge-fü-ges
Gar-aus
ge-typ-ter
             Gram-ma-tik
Gram-ma-tik-re-gel
Gram-ma-tik-re-geln
             Grund-for-ma-lis-mus
Grund-ope-ra-ti-on-en
Hilfs-verb-in-ver-sion
             Im-ple-men-tation
In-kon-sis-tenz-en
In-ter-na-tio-nal
             In-ter-net-an-bie-ter
	     in-tran-si-tiv
	     in-tran-si-tiv-es
             In-fi-ni-tiv-verb-phra-se
Ja-cken-doff
Jo-shi
             Kasus-an-for-de-rung
	     Ka-te-go-rien
	     Ka-te-go-rien-in-ven-tar
             Kon-so-nan-ten-um-wand-lung
             Kon-struk-tio-nen
Kop-fes
Kopf-merk-mals-prin-zip
             Kopf-toch-ter
             Kor-pus-an-fra-gen
             Kor-pus-an-fra-ge
             le-xi-ka-lisch
lin-gu-is-tisch-er
Lin-gu-is-tik
Lis-ten-en-des
	     Lo-kal-an-ga-ben
             mathe-ma-tische
	     Mehr-deu-tig-keiten
	     Mehr-deu-tig-keit
             Merk-mal-struk-tur
miss-trau-en
Mit-tel-feld-ele-men-te
Mor-phol-o-gy
New-meyer
    nicht-erga-tives
             nicht-erga-tiven
nicht-idio-ma-tisch
nicht-idio-ma-tische
nicht-idio-ma-tischer
nicht-idio-ma-tischen
             %Nominal-phrase
             %Nominal-umgebungen
             NON-LOC
             Ober-feld-um-stel-lung
             Ob-jekt-an-he-bung
Ob-jekt-an-he-bungs-verb
Ob-jekt-an-he-bungs-ver-ben
             Ob-li-que-ness
             Para-phra-sen
             Parse-stra-te-gien
Par-ti-kel-verb-re-gel
Par-ti-zip-ein-trag
Pas-siv-ana-ly-se
Pas-siv-ana-ly-sen
             Pos-ses-siv-pro-no-men
             Pos-ses-siv-pro-no-mi-na
Pro-blem
Pro-jekt-er-geb-nis-se
Quar-tals-er-fol-ge
Rat-ten-fä-nger
Re-dun-danz
Re-dun-danz-re-gel
Re-dun-danz-re-geln
             re-fe-renz-iden-tisch
Re-fe-renz-ob-jekt
             Re-la-tiv-satz-ele-ment
	     Rest-feld-ele-men-te
Satz-en-de
Satz-mus-tern
             sub-sum-iert
Schwär-me-rei
se-man-tics
Se-man-tik-spe-zi-fi-ka-tion
	     Si-tu-ations-se-man-tik
Sprach-er-werb
Sprach-er-werbs
Sprach-er-werbs-phä-no-me-ne
Sprach-er-werbs-pro-blem
Sprach-er-werbs-theo-rien
sprach-spe-zi-fisch
sprach-spe-zi-fische
sprach-spe-zi-fisch-es
             Stan-dard-ab-fol-ge
Stan-ford
Sta-tis-tik-kom-po-nen-ten
sta-tis-tische
Struk-tur-tei-lung
Sub-jekt-ex-trak-tion
             syn-tak-tische
Ta-ges-leis-tung
teil-idio-ma-tisch
teil-idio-ma-ti-schen
             Teil-spezi-fi-ka-tio-nen
             Teil-spezi-fi-ka-tion
The-o-ret-i-cal
Tür-ki-schen
             Typ-beschrei-bung
             Typ-beschrei-bung-en
             Typ-hier-ar-chie
             Typ-konzept
             Typ-sys-tem
	     unter-schied-licher
Ver-ar-bei-tungs-sys-tems
             %Ver-ben
             %Verbal-phrasen-teilen
             %Verbal-pro-jek-tion
             %Verbal-pro-jek-tion-en
             Ver-bal-kom-plex-unter-bre-chung
Verb-be-we-gung
Verb-be-we-gungs-spur
Verb-ebe-ne
             Verb-end-stel-lung
             Verb-en-dung
             Verb-en-dung-en
Verb-erst-le-xi-kon-re-gel
	     Verb-erst-stel-lung
Verb-erst-sät-ze
Verb-erst-sät-zen
Verb-letzt-sät-ze
Verb-letzt-sät-zen
Verb-letzt-stel-lung
Verb-letzt-po-si-tion
	     Verb-zweit-stel-lung
             Verb-form
             Verb-kom-ple-men-ten
             Verb-kom-ple-men-te
Verb-mo-bil
             Verb-phra-se
             Verb-phra-sen
Verb-pro-jek-tion
Verb-pro-jek-tion-en
             Vor-le-sung
             Vor-feld-el-lip-se
             Vor-feld-el-lip-sen
	     ver-wen-det-en
             war-um
             War-um
weg-ge-las-sen
             Wur-zel-kno-ten
             Zusammen-setzen
             Zu-stands-pas-siv
             Zu-stands-pas-sivs
Wort-ein-heit
Wun-der-lich
Fink-bei-ner
Zeit-schrift-en-ar-ti-kel
}