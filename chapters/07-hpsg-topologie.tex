%% -*- coding:utf-8 -*-
%%%%%%%%%%%%%%%%%%%%%%%%%%%%%%%%%%%%%%%%%%%%%%%%%%%%%%%%%
%%   $RCSfile: 8-hpsg-topologie.tex,v $
%%  $Revision: 1.13 $
%%      $Date: 2007/02/13 11:00:11 $
%%     Author: Stefan Mueller (CL Uni-Bremen)
%%    Purpose: 
%%   Language: LaTeX
%%%%%%%%%%%%%%%%%%%%%%%%%%%%%%%%%%%%%%%%%%%%%%%%%%%%%%%%%


\chapter{Ein topologisches Modell des deutschen Satzes}
\label{topo}

\is{Topologie|(}%
In diesem Kapitel werden einige Grundbegriffe eingeführt, die ich in den folgenden
Kapiteln benutze. 
Andere, ausführlichere Einführungen in die Topologie deutscher Sätze
findet man in \citew{Reis80a}, \citew{Hoehle86} und \citew{Askedal86}.

\section{Verbstellungstypen}

Man teilt deutsche Sätze in Abhängigkeit von der Stellung des finiten Verbs 
in drei verschiedene Klassen ein:
\is{Verbstellung!-erst-}
\is{Verbstellung!-zweit-}
\is{Verbstellung!-letzt-}
\begin{itemize}
\item Sätze mit Verbendstellung
\item Sätze mit Verberststellung
\item Sätze mit Verbzweitstellung
\end{itemize}
Beispiele dafür sind folgende Sätze:
\eal
\ex (Peter hat erzählt,) dass er das Eis gegessen \emph{hat}.
\ex \emph{Hat} Peter das Eis gegessen?
\ex Peter \emph{hat} das Eis gegessen.
\zl

% Hoehle83:Kapitel 4.1 gibt Beispiele für Imperativsätze mit V2
% 
% denBesten83a:53
% Niederländisch: Hans mag dich nicht?
%
% S. 62 verb initial declaratives 



\section{Satzklammer, Vorfeld, Mittelfeld und Nachfeld}

Man kann feststellen, dass das finite Verb mit seinen verbalen 
Komplementen nur in (\mex{0}a) eine Einheit bildet.
In (\mex{0}b) und (\mex{0}c) hängen Verb und verbale Komplemente 
nicht zusammen, sind diskontinuierlich.
\is{Konstituente!diskontinuierliche}
Man teilt den deutschen Satz auf Grundlage dieser Verteilungen 
in mehrere Bereiche ein. In (\mex{0}b) und (\mex{0}c) rahmen 
die Verbteile den Satz ein. Man spricht deshalb von der Satzklammer\is{Satzklammer}.
Sätze mit Verbendstellung werden meistens durch unterordnende Konjunktionen
wie \emph{weil}, \emph{dass} oder dergleichen
eingeleitet. Diese Konjunktionen stehen an der Stelle, an der
in Verberst- bzw.\ Verbzweitsätzen das finite Verb steht. Man rechnet sie
deshalb mit zur linken Satzklammer.
Mit Hilfe dieses Begriffs von der Satzklammer kann man den deutschen Satz 
in Vorfeld, Mittelfeld und Nachfeld einteilen: Das Vorfeld ist der Bereich
vor der linken Satzklammer. Das Mittelfeld befindet sich zwischen linker und rechter
Satzklammer, und das Nachfeld ist der Bereich rechts der rechten Satzklammer.
Beispiele zeigt die Übersicht in Tabelle~\vref{bsp-topo}.
%\newpage
\is{Vorfeld|see{Feld}}
\is{Feld!Vor-}
\is{Mittelfeld|see{Feld}}
\is{Feld!Mittel-}
\is{Nachfeld|see{Feld}}
\is{Feld!Nach-}
\begin{table}[htbp]
\begin{sideways}
%{\tiny
\begin{tabular}{lllll}
Vorfeld & linke Kl & Mittelfeld                             & rechte Klammer & Nachfeld                   \\ \\
Aicke    & schläft.                                                                                            \\
Aicke    & hat           &                                        & geschlafen.                                 \\
Aicke    & erkennt       & Conny.                                                                               \\
Aicke    & färbt         & den Mantel                             & um             & den Conny kennt.           \\
Aicke    & hat           & Conny                                  & erkannt.                                    \\
Aicke    & hat           & Conny als sie aus dem Zug stieg sofort & erkannt.                                    \\
Aicke    & hat           & Conny sofort                           & erkannt        & als sie aus dem Zug stieg. \\
Aicke    & hat           & Conny zu erkennen                      & behauptet.                                  \\
Aicke    & hat           &                                        & behauptet      & Conny zu erkennen.         \\ \\
        & Schläft       & Aicke?                                                                                \\
        & Schlaf!                                                                                              \\
        & Iss            & jetzt dein Eis                         & auf!                                        \\
        & Hat           & er doch das ganze Eis alleine          & gegessen.                                   \\  \\
        & weil          & er das ganze Eis alleine               & gegessen hat   & ohne mit der Wimper zu zucken.    \\
        & weil          & er das ganze Eis alleine               & essen können will   & ohne gestört zu werden.    \\
%        & wer           & das ganze Eis alleine                  & gegessen hat.                               \\
\end{tabular}
\end{sideways}
\caption{\label{bsp-topo}Beispiele für die Besetzung topologischer Felder}
\end{table}

Die rechte Satzklammer kann mehrere Verben enthalten und wird auch 
Verbalkomplex\is{Verbalkomplex} (\emph{verbal complex} oder \emph{verb cluster}) genannt.

\section{Zuordnung zu den Feldern}

Prädikative Adjektive verhalten sich in Bezug auf ihre Stellungsmöglichkeiten in vielerlei Hinsicht
wie Verben (vergleiche Kapitel~\ref{sec-pvp}), weshalb ich das Adjektiv in (\mex{1}) ebenfalls der
rechten Satzklammer zuordne.
\ea
Aicke ist seiner Frau treu.
\z

\noindent
Wie die Beispiele in Tabelle~\ref{bsp-topo} zeigen, müssen nicht immer
alle Felder besetzt sein, selbst die linke Satzklammer kann leer bleiben,
wenn man die Kopula\is{Kopula} \emph{sein} wie in den folgenden Beispielen weglässt:
\eal
\ex
%Doch 
{}[\ldots]
egal,      was  noch  passiert, der Norddeutsche Rundfunk             steht  schon   jetzt als Gewinner fest.\footnote{
        Spiegel, 12/1999, S.\,258.
}
\ex Interessant, zu erwähnen, daß ihre Seele völlig    in Ordnung war.\footnote{
        Michail Bulgakow, \emph{Der Meister und Margarita}. München: Deutscher Taschenbuch Verlag. 1997, S.\,422.
      }
\ex
Ein Treppenwitz der    Musikgeschichte, daß die Kollegen   von Rammstein vor    fünf Jahren noch im      Vorprogramm   von Sandow spielten.\footnote{
         Flüstern \& Schweigen, taz, 12.07.1999, S.\,14. %war das englisch? 07.12.1999, p.\,14
}
\zl
Die Sätze in (\mex{0}) entsprechen denen in (\mex{1}):
\eal
\ex 
Was noch passiert, ist egal, \ldots
\ex
Interessant ist zu erwähnen, dass ihre Seele völlig in Ordnung war.
\ex 
Ein Treppenwitz der Musikgeschichte ist, dass die Kollegen von Rammstein vor fünf Jahren noch im Vorprogramm von Sandow spielten.
\zl
Wenn Felder leer bleiben, ist es mitunter nicht ohne weiteres
ersichtlich, welchen Feldern Konstituenten zuzuordnen sind.
Für die Beispiele in (\mex{-1}) muss man die Kopula an der entsprechenden Stelle einsetzen und kann
dann feststellen, dass sich jeweils eine Konstituente dem Vorfeld zuordnen lässt und welche Felder
den restlichen Konstituenten zugeordnet werden müssen.

Im folgenden Beispiel von Hermann \citet[\page13]{Paul1919a} liefert die Einsetzung der Kopula
ein anderes Ergebnis (\mex{1}b):
\eal
\ex Niemand da?
\ex Ist niemand da?
\zl
Es handelt sich um einen Fragesatz. \emph{niemand} ist also in (\mex{0}a)
nicht dem Vorfeld, sondern dem Mittelfeld zuzuordnen.

In (\mex{1}) gibt es Elemente im \vf, in der linken Satzklammer
und im Mittelfeld. Die rechte Satzklammer ist leer.
\ea
Er gibt der Frau das Buch, die er kennt.
\z
Wie ist nun der Relativsatz \emph{die er kennt} einzuordnen? Gehört
er zum Mittelfeld oder zum Nachfeld?
Dies lässt sich mit der sogenannten Rangprobe\is{Rangprobe} \citep[\page72]{Bech55a} herausfinden:
Der Satz in (\mex{0}) wird ins Perfekt gesetzt. Da infinite Verben
die rechte Satzklammer besetzen, kann man mit dieser Umformung das Mittelfeld
klar vom Nachfeld abgrenzen. Die Beispiele in (\mex{1}) zeigen,
dass der Relativsatz nicht im \mf stehen kann, es sei denn als Teil
einer komplexen Konstituente mit dem Kopfnomen \emph{Frau}.
\eal
\ex[]{
Er hat der Frau das Buch gegeben, die er kennt.
}
\ex[*]{
Er hat der Frau das Buch, die er kennt, gegeben.
}
\ex[]{
Er hat der Frau, die er kennt, das Buch gegeben.
}
\zl

\noindent
Diese Rangprobe hilft nicht, wenn der Relativsatz wie in (\mex{1}) am Ende des Satzes neben seinem Bezugsnomen steht:
\ea
Er gibt das Buch der Frau, die er kennt.
\z
Setzt man (\mex{0}) ins Perfekt, so kann das Hauptverb vor oder nach dem Relativsatz stehen:
\eal
\ex Er hat das Buch der Frau gegeben, die er kennt.
\ex Er hat das Buch der Frau, die er kennt, gegeben.
\zl
In (\mex{0}a) ist der Relativsatz extraponiert, in (\mex{0}b) ist er Bestandteil der Nominalphrase
\emph{der Frau, die er kennt} und steht als solcher innerhalb der NP im Mittelfeld. Für (\mex{-1}) kann man sich also
nicht auf den Test verlassen. Man geht davon aus, dass in (\mex{-1}) der Relativsatz zur NP gehört,
da das die einfachere Struktur ist, denn wenn der Relativsatz im Nachfeld steht, muss es sich um eine
Verschiebung des Relativsatzes aus der NP heraus handeln, \dash, die NP"=Struktur muss ohnehin
angenommen werden und die Verschiebung kommt noch dazu.

%% Die Einordnung von Interrogativphrasen und Relativphrasen wird in der theoretischen Literatur
%% verschieden gehandhabt. Theoretisch gibt es drei Möglichkeiten für die Zuordnung von \emph{wer}
%% in (\mex{1}) zu einem Stellungsfeld: \emph{wer} könnte im Vorfeld, in der linken Satzklammer oder
%% im Mittelfeld stehen.
%% \ea
%% Ich möchte wissen, wer das ganze Eis alleine gegessen hat.
%% \z
%% Die letzte Möglichkeit ist die unplausibelste, da Interrogativ- und Relativphrasen aus anderen
%% Teilsätzen vorangestellt worden sein können, was keine Eigenschaft von Mittelfeldelementen ist.
%% Mittelfeldelemente gehören (von einigen wenigen Ausnahmen abgesehen, die nur unter eingeschränkten
%% Bedingungen möglich sind) immer zu den Verben in den Satzklammern. Wie die Beispiele in (\mex{1})
%% zeigen, kann die Phrase, die das Relativpronomen enthält, durchaus zu einem Verb gehören,
%% dessen Projektion sich im Nachfeld befindet. Die Zugehörigkeit der Relativphrase ist durch
%% ein \_$_i$ gekennzeichnet.
%% \eal
%% \label{bsp-richter-top}
%% \ex eine Tat, [\sub{VP} die begangen zu haben]$_i$ Hans sich weigert [\sub{VP} dem Richter \_$_i$ zu gestehen]\footnote{
%%         \citew[\page48a]{Haider85c}.
%% }\label{bsp-richter}
%% \ex ein Buch, [\sub{VP} das zu lesen]$_i$ der Professor glaubt [\sub{VP} den Studenten \_$_i$ empfehlen zu müssen]\footnote{
%%         \citew[Kapitel~7.3.2]{Grewendorf88a}.
%% }
%% \zl


\section{Rekursion}
\label{sec-topo-rekursion}

Wie\is{Rekursion|(} schon \citet[\page82]{Reis80a} festgestellt hat, kann das Vorfeld,
wenn es eine komplexe Konstituente enthält, selbst wieder
in Felder unterteilt sein und \zb ein Nachfeld enthalten.
In (\mex{1}b) befindet sich \emph{für lange lange Zeit} und in
(\mex{1}d) \emph{dass du kommst} innerhalb des Vorfelds rechts der
rechten Satzklammer \emph{verschüttet} bzw.\ \emph{gewusst}, \dash
innerhalb des Vorfelds im Nachfeld.
\eal
\ex Die Möglichkeit, etwas zu verändern, ist damit verschüttet
      für lange lange Zeit.
\ex {}[Verschüttet für lange lange Zeit] ist damit die Möglichkeit, 
      etwas zu ver"-ändern.
\ex Wir haben schon seit langem gewusst, dass du kommst.
\ex {}[Gewusst, dass du kommst,] haben wir schon seit langem.
\zl

\noindent
Elemente im Mittelfeld und Nachfeld können natürlich ebenso wie
Vorfeldkonstituenten eine interne Struktur haben und somit wieder
in Felder unterteilt werden. Zum Beispiel ist \emph{dass} die linke Satzklammer des Teilsatzes
\emph{dass du kommst} in (\mex{0}c), \emph{du} steht im Mittelfeld und \emph{kommst} bildet die
rechte Satzklammer des Teilsatzes.%
\is{Topologie|)}\is{Rekursion|)}


%\section*{Kontrollfragen}

\questions{
\begin{enumerate}
\item Wie sind die Begriffe Vorfeld, Mittelfeld, Nachfeld und linke bzw.\ rechte
      Satzklammer definiert?
\end{enumerate}
}

%\section*{Übungsaufgaben}

\exercises{
\begin{enumerate}
\item Bestimmen Sie die Satzklammern und das Vorfeld, Mittelfeld und Nachfeld in den folgenden Sätzen! Gehen Sie auch auf die
  Felderstruktur eingebetteter Sätze ein!
\eal
\ex Aicke isst.
\ex Der Delfin hilft dem Kind, den jeder kennt.
\ex Der Delfin dem Kind, das jeder kennt.
%\ex Die Studenten behaupten, nur wegen der Hitze einzuschlafen.
\ex Die Student*innen haben behauptet, nur wegen der Hitze einzuschlafen.
\ex Dass Aicke nicht kommt, ärgert Conny.
\ex Einen Roman lesen, der sie langweilt, würde sie nie.
\zl
\end{enumerate}
}

%\section*{Literaturhinweise}

\furtherreading{
\citet{Reis80a} begründet, 
warum die Feldertheorie für die Beschreibung
der Konstituentenstellungsdaten im Deutschen sinnvoll und notwendig ist.

\citet{Hoehle86} erwähnt noch ein weiteres Feld links des Vorfeldes,
das man für Linksherausstellungen wie die in (\mex{1}) braucht.
\ea
Der Mittwoch, der passt mir gut.
\z
\itdopt{hier geht was mit den Abständen durcheinander.}
Höhle geht auch auf die historische Entwicklung der Feldertheorie ein.

%Die Begriffe Vorfeld, Mittelfeld und Nachfeld sind ausführlicher in 
%\citew[Kapitel~4]{Grewendorf88a} erklärt.

\citet{Woellstein2010a-u} gibt in einem kleinen Lehrbuch einen Überblick zu verschiedenen Auffassungen zu topologischen
Feldern.

Die wohl umfangreichste Arbeit zu topologischen Feldern hat Tilman Höhle 1983 fast
fertiggestellt. Sie wurde jahrzehntelang kopiert und in interessierten Kreisen weitergegeben. 2018
wurden die \emph{Topologischen Felder} in einem Sammelband mit allen Arbeiten von Tilman Höhle zum
ersten Mal einer breiten Öffentlichkeit zugänglich gemacht \citep{HoehleTopo}. 
}