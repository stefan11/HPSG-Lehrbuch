%% -*- coding:utf-8 -*-
%%%%%%%%%%%%%%%%%%%%%%%%%%%%%%%%%%%%%%%%%%%%%%%%%%%%%%%%%
%%   $RCSfile: hpsg-morphologie.tex,v $
%%  $Revision: 1.5 $
%%      $Date: 2006/10/11 13:45:38 $
%%     Author: Stefan Mueller (CL Uni-Bremen)
%%    Purpose: 
%%   Language: LaTeX
%%%%%%%%%%%%%%%%%%%%%%%%%%%%%%%%%%%%%%%%%%%%%%%%%%%%%%%%%

\chapter{Argumentation für eine Theorie bzw.\ gegen andere Theorien}
\label{chap-argumentation}


In diesem Kapitel soll erörtert werden, was eine gute Theorie ausmacht
und wie man für oder gegen Theorien argumentieren kann. Der Leser mag
es merkwürdig finden, dass diese Diskussion zum Schluß des Buches stattfindet.
Der Grund hierfür ist, dass erst zu diesem Zeitpunkt das nötige Wissen
über Phänomene und Beschreibungsmittel vorhanden ist, um über diese Dinge reden
zu können.


\section{Kriterien für eine gute Theorie}
\label{sec-kriterien}

Für eine bestimmte Sprache oder einen Sprachausschnitt lassen sich unendlich viele
Grammatiken angeben. Das kann man sich an der Grammatik in (\mex{2}) klar machen,
die den folgenden Satz beschreibt:
\ea
Kim sleeps.
\z
\ea\label{gram-kim-sleeps-v}
\begin{tabular}[t]{@{}l@{ }l}       
{S}  & {$\to$ NP, V}
\end{tabular}\hspace{2cm}%
\begin{tabular}[t]{@{}l@{ }l}
{NP} & {$\to$ kim}\\
{V} & {$\to$ sleeps}\\
\end{tabular}
\z
Die Grammatik in (\mex{1}) beschreibt genau dieselben Sätze wie die in (\mex{0}),
enthält aber eine zusätzliche Regel:
\ea
\begin{tabular}[t]{@{}l@{ }l}       
{S}  & {$\to$ NP, V}\\
PP   & {$\to$ P, Det}\\
\end{tabular}\hspace{2cm}%
\begin{tabular}[t]{@{}l@{ }l}
{NP} & {$\to$ kim}\\
{V} & {$\to$ sleeps}\\
\end{tabular}
\z
Die zweite Grammatikregel ist völlig unsinnig, schadet aber im konkreten Fall nicht,
da die Grammatik weder Regeln für P noch für Det enthält. Auf diese Weise können wir aus
der Grammatik in (\mex{-1}) unendlich viele andere Grammatiken durch Hinzunahme unsinniger
Regeln erzeugen. Vergleicht man die Grammatiken in (\mex{-1}) und (\mex{0}) miteinander,
ist klar, dass die Grammatik in (\mex{-1}) die am besten für ihren Zweck, nämlich die Beschreibung
von (\mex{-2}), geeignete ist. Die Anzahl der verwendeten Regeln ist also ein Kriterium für die
Güte von Grammatiken. Wie sieht es nun mit der folgenden Grammatik aus?

\ea
\begin{tabular}[t]{@{}l@{ }l}       
{S}  & {$\to$ NP, VP}\\
\end{tabular}\hspace{2cm}%
\begin{tabular}[t]{@{}l@{ }l}
{NP} & {$\to$ kim}\\
{VP} & {$\to$ sleeps}\\
\end{tabular}
\z
Die beiden Grammatiken in (\ref{gram-kim-sleeps-v}) und (\mex{0}) sind -- betrachtet man die Sprache, die sie beschreiben -- identisch,
als Linguisten wissen wir aber, was passiert, wenn man die Grammatiken erweitern will: Zur Grammatik
in (\mex{0}) können wir einfach eine Regel für die Ableitung von VPen mit transitiven Verben und
einen Lexikoneintrag für ein transitives Verb hinzufügen und können dann sofort einen entsprechenden
Satz analysieren:
\ea
\begin{tabular}[t]{@{}l@{ }l}       
{S}  & {$\to$ NP, VP}\\
VP   & $\to$ V, NP\\
\end{tabular}\hspace{2cm}%
\begin{tabular}[t]{@{}l@{ }l}
{NP} & {$\to$ kim}\\
{VP} & {$\to$ sleeps}\\
V    & $\to$ likes\\
\end{tabular}
\z
Da wir die Kategorie VP haben, können wir die Grammatik auch so erweitern, dass Adjunkte
mit der VP kombiniert werden können. Die Grammatik in (\ref{gram-kim-sleeps-v}) können wir 
nicht so einfach erweitern. Wir müßten die Kategorien umbenennen oder komplizierte Regeln
schreiben. Man kann also auch von zwei Grammatiken, die eigentlich denselben Sprachausschnitt
beschreiben, sagen, welche die bessere ist, wenn man sich überlegt, wie die betreffenden
Grammatiken erweitert werden können.

Hier ist noch eine einschränkende Anmerkung nötig: Nicht immer ist die Theorie, die mit weniger
Einheiten eine bestimmte Sprache beschreibt, die bessere, denn man muss auch berücksichtigen,
was modelliert werden soll. Zum Beispiel konnte experimentell nachgewiesen werden, dass
Menschen bei der Verarbeitung von komplexen Wörtern oder Phrasen mitunter direkt auf in ihrem
Kopf abgespeicherte Muster zurückgreifen können, obwohl die komplexen Wörter bzw.\ Phrasen
regulär nach den Gesetzmäßigkeiten der jeweiligen Sprache gebildet sind.\NOTE{Quellen? Lüdeling?} Ob solche Einheiten
als Ganzes gespeichert werden oder nicht, hängt mit der Häufigkeit der Verwendung dieser
Einheiten zusammen. Will man also direkt modellieren, welche Elemente im mentalen Lexikon eines
Menschen enthalten sind, so muss man für bestimmte Wörter neben der allgemeinen Wortbildungsregel
auch noch die Ergebnisse der Regelanwendung auf"|führen. Nach den oben beschriebenen Kriterien
wäre eine solche Grammatik schlechter als eine, die nur die Wortbildungsregeln auf"|führt
und auf die Listung der Ergebnisse verzichtet. Berücksichtigt man aber die experimentell gewonnenen
Einsichten, so ist eine Grammatik, die hochfrequente Wörter, deren Existenz im mentalen Lexikon
nachgewiesen werden kann, redundant repräsentiert, bei entsprechendem Modellierungsziel besser geeignet.

Genauso wie man zu jeder Phrasenstrukturgrammatik unendlich viele Grammatiken angeben kann,
die dieselbe Sprache beschreiben,
kann man zu jeder Repräsentation in Merkmalbeschreibungen noch weitere Merkmale hinzufügen. Zu der
in diesem Buch entwickelten Theorie könnte man für jedes Wort noch ein binäres
Merkmal \textsc{ist-länger-als-siebzehn} annehmen, das den Wert + hat, wenn das Wort mehr als 17 Buchstaben
hat, und $-$ sonst. \emph{Straßenbahnhaltestelle} hätte dann den Wert + und \emph{ja} den Wert $-$.
Die so erweiterte Grammatik würde aber dann nichts anderes tun als die in den vorigen Kapiteln
beschriebene. Es ist klar, dass wir ein solches Merkmal nicht brauchen und dass die Grammatik
ohne dieses Merkmal besser ist.

Wir haben also bisher Sparsamkeit und Erweiterbarkeit als Kriterien für gute Theorien
kennengelernt. Ein weiteres Kriterium ist die Stärke der Vorhersagen, die eine Grammatik macht,
und die Konsequenzen, die daraus zu ziehen sind, wenn die Vorhersagen falsch sind.
In einer Theorie, die ganz allgemeine Regel"=Schemata und Prinzipien verwendet, werden im allgemeinen
viel stärkere Vorhersagen gemacht als in Theorien, die einfach vorhandene Daten klassifizieren.
Stellt sich eine Vorhersage der Theorie als falsch heraus, so müssen die allgemeinen Prinzipien
geändert werden, was mitunter einen erheblichen Eingriff in das System bedeutet,
bei klassifizierenden Theorien muss man einfach bisher unbekannte Daten neu klassifizieren.

Das folgende Beispiel aus der Grammatikentwicklung soll das zuletzt Gesagte etwas verdeutlichen:
Ich habe Analysen für die sogenannte \emph{dritte Konstruktion}\is{dritte Konstruktion} und für die Voranstellung von Teilphrasen
(siehe Kapitel~\ref{sec-pvp}) implementiert und 
danach in einem Aufsatz von Klaus Netter \citeyearpar{Netter91} die folgenden Sätze gefunden:
\eal
\ex {}[Versucht, zu lesen], hat er das Buch nicht.
\ex {}[Versucht, einen Freund vorzustellen], hat er ihr noch nie.
\zl
Mir war völlig schleierhaft, wie diese Sätze zu analysieren sind. Ich habe den Computer eingeschaltet,
die Sätze eingegeben, mir die Analyse angesehen und verstanden, wie die bereits einzeln beschriebenen
Phänomene interagieren. Das heißt, die implementierte Grammatik machte Vorhersagen in Bezug auf
Sätze, über die ich noch nie nachgedacht hatte.

Wir können also folgendes festhalten:
Muss man zwischen zwei Analysen/""Theorien entscheiden, so sollte man die nehmen, die sich durch folgendes
auszeichnet:
\begin{itemize}
\item ein möglichst kleines Inventar an Beschreibungsmitteln (Merkmale, Typen, Regeln, 
      automatische Typ- oder Regelberechnung, relationale Beschränkungen),
\item eine möglichst geringe Anzahl/geringe Mächtigkeit von verwendeten Elementen 
      aus dem Inventar der Beschreibungsmittel,
\item Erweiterbarkeit und
\item möglichst starke Vorhersagen.
\end{itemize}


\section{Argumentation gegen Theorien unter Bezugnahme auf Daten}

Leider ist es so, dass in vielen Bereichen der Linguistik sehr schlampig mit Daten
umgegangen wird. Dies ist wahrscheinlich das Resultat einer Gegenbewegung
von Wissenschaftlern, die sich von allein an Korpusbelegen orientierter Forschung
absetzen wollten. Diskutierte Daten sind oft nur durch Introspektion der jeweiligen
Autoren abgesichert. In den Anfangszeiten bestimmter Richtungen
in der theoretischen Linguistik im vergangenen Jahrhundert war ein solches Vorgehen 
auch unproblematisch, denn die theoretischen Modelle behandelten unkontroverse Fälle.
Mit zunehmender Komplexität der Theorien wurden auch komplexere Daten diskutiert 
und mitunter auch Daten zur Motivation von Theorien verwendet, deren
Akzeptabilitätsstatus unklar ist \citep{Fanselow2004b}.
Man findet dann Aussagen wie \emph{in meinem Dialekt
ist folgender Satz grammatisch} oder \emph{Es gibt Sprecher, die den Satz akzeptieren,
aber man kann keine regionale Verteilung oder sonstige Abhängigkeit vom Register, sozialem
Status oder ähnlichem feststellen}. Insbesondere Aussagen der letzten Art sind mit Vorsicht
zu genießen, da sie schlecht verifizierbar sind. Nicht verifizierbare Aussagen haben
in wissenschaftlichen Arbeiten aber nichts zu suchen:\footnote{
  Siehe hierzu auch die DFG"=Richtlinien zur Sicherung guter wissenschaftlicher Praxis (Kommission, 1998)\nocite{DFG98a}.%
}
Im besten Fall verschwenden sie
nur Zeit und Ressourcen derjenigen, die die Publikationen lesen, im schlechtesten Fall werden die
Daten von anderen Wissenschaftlern aufgegriffen und zur Motivation von Analysen in
anderen Sprachen verwendet. Es entsteht somit ein riesiges Kartenhaus, dessen
Grundlage wacklige Datenbeurteilungen sind. Nach einiger Zeit ist nicht mehr nachzuvollziehen,
welche Theorieteile wirklich motiviert sind.

Es ist einfach, auf der Grundlage von Daten gegen Analysen zu argumentieren, wenn Autoren
explizit auf Konsequenzen ihrer Analyse hinweisen. Wenn Autoren \zb schreiben, dass ihre
Theorie die Unmöglichkeit der NP"=Extraposition im Deutschen dadurch erklärt, dass Kasus vom Verb
nur nach links zugewiesen werden kann, dann reicht es zu zeigen, dass NP"=Extraposition im Deutschen
möglich ist, um die Theorie zu widerlegen.\footnote{
  Siehe \citew[Kapitel~13.1]{Mueller99a} und \citew[ix--xii]{Mueller2002b} zur Extraposition von Nominalphrasen.%
}
Die Widerlegung von aussagekräftigen Theorien ist ein fruchtbarer Prozeß: Die Theorie war gut,
da sie bestimmte Vorhersagen gemacht hat. Die Vorhersagen wurden durch die Daten nicht bestätigt,
weshalb die Theorie revidiert werden muss und man hinterher eine bessere Theorie bekommt.

Wird nicht explizit auf bestimmte Daten eingegangen, dann ist es nicht ohne weiteres möglich,
eine Theorie zu widerlegen. Man kann \zb nicht sagen, dass eine bestimmte Theorie %der \emph{be}"=Derivation\is{Derivation}
falsch ist, wenn sie nicht alle im Korpus belegten Muster %von \emph{be}"=Verben 
beschreibt. Die Theorie ist dann nicht falsch, sondern unvollständig. Man kann dann argumentieren,
dass der Aufwand, den man für die Erklärung der restlichen Fälle treiben müßte, sehr groß (oder größer als bei einer
anderen diskutierten Analyse) wäre oder dass bestimmte bereits getroffene Entscheidungen eine Analyse der verbleibenden Fälle ausschließt,
eine solche Argumentation ist aber oft schwierig. 
%% Im konkreten Fall ging es um eine morphembasierte
%% Analyse der \emph{be}"=Derivation, die einer konstruktionsbasierten\is{Konstruktionsgrammatik (CxG)}
%% Analyse unterlegen sein sollte \citep{MR2001a}. Es wurden verschiedene Konstruktionen 
%% für die verschiedenen belegten Muster vorgeschlagen. Natürlich
%% kann man ganz parallel verschiedene \emph{be}"=Präfixe\is{Pr"afix} vorschlagen, die Kosten sind dieselben.

Jetzt wird auch klar, warum Theorien, die Daten nur mittels Typhierarchien klassifizieren,
wenig interessant sind: Wirft man Vertretern solcher Ansätze vor, dass ein bestimmtes Phänomen
nicht erfaßt wird, so ist die Antwort: Doch, dafür gibt es einen bestimmten Typ, der von allgemeineren
Typen erbt. Wie aus der Diskussion im Abschnitt~\ref{sec-kriterien} aber klar geworden sein sollte,
hat die Stipulation eines Typs seinen Preis. Eine Theorie, die mit weniger Typen und mit wenigen
konstruktionsspezifischen Typen und ohne automatische Berechnung von zusätzlichen Typen
auskommt, ist die bessere Theorie und macht stärkere Vorhersagen.




%\section{Argumentation für Theorien unter Bezugnahme auf Daten}


\section{Argumentation gegen Theorien wegen Übergenerierung}

Im Abschnitt~\ref{sec-kriterien} haben wir gesagt, dass eine Theorie möglichst starke Vorhersagen
machen sollte. Das schließt ein, dass die Theorie nur grammatischen Sätzen eine Struktur zuweisen
darf. Man sagt, dass eine Theorie, die auch ungrammatische Sätze zuläßt, übergeneriert.\is{"Ubergenerierung} Im folgenden
will ich kurz zeigen, dass es auch legitim sein kann, übergenerierende Theorien zu entwickeln bzw.\
dass mitunter Argumentation in Bezug auf Übergenerierung nicht gerechtfertigt ist.

In der Einleitung wurden im Rahmen der Diskussion der Konstituententests die Beispiele
in (\ref{bsp-mehr-vf}) -- hier als (\mex{1}) wiederholt -- angeführt:
\eal
\ex {}[Trocken] [durch die Stadt] kommt man am Wochenende auch mit der BVG.\footnote{
        taz berlin, 10.07.1998, S.\,22.
      }
\ex {}[Wenig] [mit Sprachgeschichte] hat der dritte Beitrag in dieser Rubrik zu tun, [\ldots]\footnote{
  ZS für Dialektologie und Linguistik, LXIX, 3/2002, S.\,339.
}
\zl
In \citew{Mueller2005d}\is{Vorfeldbesetzung} habe ich eine Analyse für solche Sätze vorgestellt, die -- so wie sie jetzt ist --
übergeneriert. Sie läßt Sätze mit beliebig vielen von einem leeren verbalen Kopf abhängenden
Konstituenten im Vorfeld zu. Man kann mir nun vorwerfen, eine Analyse entwickelt zu haben,
die nicht restriktiv genug ist, die Analyse ist jedoch wegen der mangelnden Restriktivität 
nicht falsch: Die Beispiele zeigen, dass vor dem finiten Verb mehrere Konstituenten stehen können,
die in keiner direkten Beziehung zueinander stehen. Will man solche Beispiele nicht völlig aus grammatischen
Beschreibungen ausklammern (und das sollte man nicht, denn es gibt massenhaft Belege für dieses Muster,
siehe \citew{Mueller2003b}), dann muss es für diese Beispiele syntaktische Strukturen geben. Einen
Vorschlag dafür, wie diese aussehen können, habe ich in dem zitierten Aufsatz gemacht. Die pragmatischen
Beschränkungen, die für diese Muster gelten, sind noch nicht bekannt, was darauf zurückzuführen ist, dass
diese Art Daten bisher in der Literatur bis auf wenige Ausnahmen nicht diskutiert wurden. Man muss
also entsprechende Teiltheorien für Prosodie\is{Prosodie} und Informationsstruktur\is{Informationsstruktur} und die Interaktion mit
der Vorfeldbesetzung entwickeln, um dann Aussagen darüber treffen zu können, warum eben doch nicht
beliebige Kombinationen von Konstituenten im Vorfeld stehen können.
Es ist also legitim, Teilaspekte von Phänomenen zu analysieren, wenn es möglich ist, Übergenerierung
durch die Formalisierung weiterer Beschränkungen aus anderen linguistischen Teilbereichen einzuschränken bzw.\
ganz zu eliminieren.

Ich möchte noch ein weiteres Beispiel diskutieren, bei dem Übergenerierung unterstellt wurde, aber
in Wirklichkeit einfach nur keine Aussage über einen bestimmten Phänomenbereich gemacht wurde:
\citet[\page73]{Gunkel2003b} schreibt folgendes:
\begin{quote}
Dass nicht alle nichtergativen Verben passivierbar sind wird in Teilen der HPSG-Literatur immer
noch ignoriert. So schlägt Pollard (1994: 282 et passim) eine Analyse des \emph{werden}"=Passivs vor, derzufolge
alle nichtergativen Verben mit referentiellem Subjekt passivierbar sind. In analoger Weise -- wenn auch in technisch
anderer Durchführung -- ist die Passivanalyse von Heinz/Matiasek (1994: 216, 224f.) konzipiert, die wesentlich auf
Haider (1986a) aufbaut und nach der alle Verben mit designiertem Argument passivierbar sind. Angesichts von nichtpassivierbaren
Verben wie in (2-26) aufgeführt, sind solche Vorschläge hinfällig.
\exewidth{(2-26)}
\begin{exe}
\exi{(2-26)}\begin{xlist}[iv.]
\ex blühen, brennen, funktionieren, glühen, ruhen
\ex entsprechen, fehlen, gefallen, gehören
\ex ärgern, faszinieren, freuen, schmerzen, stören, wundern
\ex haben, bekommen, besitzen, erhalten, kriegen
\ex beinhalten, enthalten, fassen
\zl
\citep[\page73]{Gunkel2003b}
\end{quote}

\noindent
Gunkel merkt in einer Fußnote an: \emph{Müller folgt Pollards Ansatz, obwohl er an anderer Stelle selbst bemerkt, dass
Nichtergativität keineswegs Passivfähigkeit impliziert.}

Gunkels Argumentation ist nicht korrekt. Man kann eine Analyse nur dann verwerfen, wenn man zeigen kann,
dass es nicht möglich ist, weitere Beschränkungen zu formulieren, die bestimmte Strukturen/""Sätze ausschließen.
Auf dieselbe Weise, wie er gegen die Pollardsche/""Haidersche Passivanalyse argumentiert, könnte Gunkel gegen
jede Syntaxtheorie argumentieren, die Strukturen für (\mex{1}) vorschlägt:
\eal
\ex[]{
Karl wäscht den Schlafanzug.
}
\ex[*]{
Der Schlafanzug wäscht Karl.
}
\zl
Der Unterschied in (\mex{0}) kann über Selektionsrestriktionen\is{Selektionsrestriktion} bzw.\ unter
Bezug auf Weltwissen\is{Weltwissen} erklärt werden, das muss aber nicht unbedingt Gegenstand
syntaktischer Abhandlungen sein. 

Genauso ist es nicht unmöglich, Beschränkungen für Thema"=Verben zu finden, die entsprechende Passivierungen
von Thema"=Verben ausschließen, aber mit der restlichen Analyse kompatibel sind. Gunkel kann also Autoren nur
vorwerfen, dass sie die Thema"=Verben nicht behandelt haben, aber er kann nicht behaupten, dass die Theorien
allgemein unbrauchbar seien.

Gunkel fährt fort: \emph{Einer der Gründe für die stillschweigende Gleichsetzung von Ergativität und
  Nichtpassivierbarkeit} [\ldots].
Aus dieser Passage wird klar, dass niemand explizit die von Gunkel unterstellte Gleichsetzung gemacht hat.
Die Analysen von Haider, Heinz und Matiasek, Kathol, Pollard und von mir haben erklärt, warum
unakkusative Verben normalerweise nicht 
%bzw.\ nur unter bestimmten pragmatischen Bedingungen (\citealp[\page350]{Ruzicka89}; 
%\citealp[\page205]{Wunderlich85}; \citealp[\page290]{Mueller99a}) 
passivierbar sind, sie haben auch erklärt, 
was beim Passiv passiert, wenn es möglich ist, sie haben aber nicht gesagt,
dass das Passiv für alle unergativischen Verben möglich ist. Es ist notwendigerweise so, dass Theorien, die nur den
syntaktischen Aspekt eines Phänomens betrachten, semantische Aspekte nicht berücksichtigen. Sie sind
deshalb nicht falsch, sondern höchstens unvollständig. 
%% Im Übrigen wäre es ein Leichtes gewesen, so wie
%% Gunkel es auf Seite~100 vorschlägt, dem Lexikoneintrag für das Passivhilfsverb \emph{werden} eine Restriktion
%% hinzuzufügen, die besagt, dass das Partizip eine Agens"=Rolle haben muss (Das wurde übrigens auch von
%% \citet[\page17]{Haider86} explizit vorgeschlagen.). Man braucht dann allerdings einen sehr weiten Agens"=Begriff.
%% Normalerweise wird \zb angenommen, dass der Sehende in (\mex{1}) ein Experiencer ist:
%% \eal
%% \ex Er sah den Einbrecher.
%% \ex Der Einbrecher wurde gesehen.
%% \zl


%% \section*{Kontrollfragen}


%% \begin{enumerate}
%% \item
%% \end{enumerate}

%% \section*{Übungsaufgaben}
%% \begin{enumerate}
%% \item Suchen Sie in der Fachliteratur ein Beispiel für fehlerhafte Argumentation.
%% \end{enumerate}


\label{lastpage}
