%% -*- coding:utf-8 -*-
%%%%%%%%%%%%%%%%%%%%%%%%%%%%%%%%%%%%%%%%%%%%%%%%%%%%%%%%%
%%   $RCSfile: 4-hpsg-komplementation.tex,v $
%%  $Revision: 1.20 $
%%      $Date: 2007/02/13 11:00:11 $
%%     Author: Stefan Mueller (CL Uni-Bremen)
%%    Purpose: 
%%   Language: LaTeX
%%%%%%%%%%%%%%%%%%%%%%%%%%%%%%%%%%%%%%%%%%%%%%%%%%%%%%%%%


\chapter{Kopf-Argument-Strukturen}
\label{chap-komplementation}

In diesem Kapitel wird genauer auf die Repräsentation von Grammatikregeln
im Formalismus der HPSG eingegangen. Es wird gezeigt, wie Bäume mittels Merkmalbeschreibungen
repräsentiert werden können. Typhierarchien werden zur Klassifikation verschiedener
Grammatikregeln benutzt, und es wird gezeigt, wie Generalisierungen in bezug
auf Grammatikregeln erfaßt werden können. Die Lexikoneinträge, die bisher verwendet wurden,
werden stärker strukturiert, so dass es möglich wird, die Projektion
%% \NOTE{Sollte das noch irgendwo rein?
%%   Unter Projektion\is{Projektion} versteht man in der Linguistik das Übertragen von Merkmalen von einer Kategorie
%%   auf eine diese unmittelbar dominierende Kategorie. Bei einer Übertragung von Information in die
%%   andere Richtung spricht man von Perkolation\is{Perkolation}. Wie sich zeigen wird, ist es in der
%%   HPSG nicht sinnvoll, die Richtungen zu unterscheiden, da Projektion/""Perkolation durch
%%   Strukturteilung modelliert wird. Die Information ist einfach identisch, und es ist nicht sinnvoll,
%%   über eine Richtung der Informationsweitergabe zu sprechen.%
%% } 
von Merkmalen, die für Phrasen wichtig sind, elegant zu beschreiben. 

\section{Die Modellierung von Konstituentenstruktur mit Hilfe von Merkmalstrukturen}

Wie bereits angedeutet, dienen Merkmalbeschreibungen in der HPSG
als einheitliches Beschreibungsinventar für morphologische Regeln, Lexikoneinträge
und syntaktische Regeln. Die Bäume, die bisher zu sehen waren, sind nur Visualisierungen
der Zusammenhänge, sie haben keinen theoretischen Status. Genauso gibt es
in der HPSG keine Ersetzungsregeln.\is{Phrasenstrukturgrammatik}\footnote{
  In vielen Computerimplementationen zur Verarbeitung von HPSG"=Grammatiken werden
  zur Verbesserung der Verarbeitungsgeschwindigkeit allerdings Phrasenstrukturregeln verwendet.}
Die Funktion der Phrasenstrukturregeln wird von Merkmalbeschreibungen übernommen.
Man trennt zwischen unmittelbarer Dominanz\is{Dominanz!unmittelbare} (ID) und linearer Präzedenz\is{Präzedenz} (LP).
Die Dominanzinformation wird mittels \textsc{dtr}"=Merkmalen (Kopf"|tochter und Nicht-Kopf"|töchter)
repräsentiert, Information über Präzedenz ist implizit in \phon enthalten.
In diesem Kapitel beschäftigen wir uns nur mit der Dominanz. Zu Abfolgebeschränkungen
wird im Kapitel~\ref{chap-Konstituentenreihenfolge} mehr gesagt.
(\mex{1}) zeigt die Repräsentation der \phonwe in einer Merkmalbeschreibung, die dem Baum in
Abbildung~\vref{fig-dem-mann-fs} entspricht.
\begin{figure}
\centerline{%
\begin{forest}
sm edges
[NP
  [Det,baseline
    [dem]]
  [N
    [Eichhörnchen]]]
\end{forest}}
\caption{\label{fig-dem-mann-fs}\emph{dem Eichhörnchen}}
\end{figure}
\ea
\ms{ 
  phon     & \phonliste{ dem Eichhörnchen }\\
  head-dtr & \onems{ phon \phonliste{ Eichhörnchen }  \\
                 }\\
  non-head-dtrs & \liste{ \onems{ phon \phonliste{ dem } \\
                            }}\\
}
\z
In (\mex{0}) gibt es genau eine Kopf"|tochter (\textsc{head-dtr}\isfeat{head-dtr}).
Die Kopf"|tochter ist immer die Tochter, die den Kopf enthält. In einer Struktur mit den Töchtern 
{\em das\/} und {\em Bild von Maria\/} wäre {\em Bild von Maria\/} die Kopf"|tochter,
da {\em Bild\/} der Kopf ist. Es kann im Prinzip mehrere Nicht"=Kopf"|töchter geben. 
Würde man wie in Abbildung~\vref{fig-Aicke-dem-Affen-den-Stock-gibt-flat} flache Strukturen
für einen Satz mit einem ditransitiven finiten Verb annehmen, hätte man \zb drei Nicht"=Kopf"|töchter.\isfeat{non-head-dtrs}
Auch binär verzweigende Strukturen ohne Kopf sind sinnvoll (siehe Kapitel~\ref{chap-rs} zur Analyse
von Relativsätzen).



%% \begin{tabular}[t]{@{}c@{\hspace{10mm}}c@{}}%llll}
%% \multicolumn{2}{c}{\rnode{1}{NP[\begin{tabular}[t]{@{}l}
%%                                           \textsc{head-dtr} \ibox{2}, \\
%%                                           \textsc{non-head-dtrs} \liste{\ibox{1}}]\\
%%                                           \end{tabular}}
%% }\\
%% \\[2ex]
%% \rnode{2}{\ibox{1} Det} & \rnode{3}{\ibox{2} N}\\
%% \\[2ex]
%% \rnode{4}{dem}                     & \rnode{5}{Eichhörnchen}\\
%% \end{tabular}
%% \ncline{1}{2} \ncline{1}{3}
%% \ncline{2}{4} \ncline{3}{5}


Das Regelschema (\ref{regelschema-psg-comps}) aus dem vorigen Kapitel -- hier der Übersichtlichkeit
halber als (\mex{1}) wiederholt -- kann man analog in Merkmalbeschreibungen ausdrücken.
\ea
\label{regelschema-psg-comps-zwei}
\begin{tabular}[t]{@{}lll}
H[SUBCAT \ibox{1}] & $\to$ & H[SUBCAT \ibox{1} $\oplus$ \sliste{ \ibox{2} } ] \ibox{2}\\
\end{tabular}
\z
Das zeigt das folgende Schema\is{Schema!Kopf"=Argument"=}:
\begin{schema}[Kopf-Argument-Schema (binär verzweigend, vorläufige Version)]
\label{schema-bin-prel}
\type{head-argument-phrase}\istype{head"=argument"=phrase} \impl\\
\onems{
      comps \ibox{1} \\
      head-dtr$|$comps \ibox{1} $\oplus$ \sliste{ \ibox{2} } \\
      non-head-dtrs \sliste{ \ibox{2} }\\
      }
\end{schema}
Schema~\ref{schema-bin-prel} gibt an,
welche Eigenschaften ein linguistisches Objekt vom Typ \type{head"=argument"=phrase} haben muss.
Der Pfeil\is{\impl} steht in Schema~\ref{schema-bin-prel} für eine Implikation\is{Implikation},
nicht für den Ersetzungspfeil, wie er in Phrasenstrukturregeln vorkommt. Man kann das Schema also
wie folgt lesen: Wenn ein Objekt vom Typ \type{head"=argument"=phrase} ist, muss es die Eigenschaften
haben, die auf der rechten Seite der Implikation angegeben sind. Konkret heißt das,
dass solche Objekte immer eine Valenzliste besitzen, die \iboxt{1} entspricht, dass sie eine Kopf"|tochter haben,
die eine Valenzliste hat, die sich in die zwei Teillisten \ibox{1} und \sliste{ \ibox{2} }
unterteilen läßt, und dass sie eine Nicht"=Kopf"|tochter haben, die mit dem letzten Element
in der \compsl der Kopf"|tochter \iboxb{2} kompatibel ist. 

Wie ebenfalls im vorigen Kapitel angemerkt, sind Artikel"=Nomen"=Kombinationen
und Verb"=NP"=Kombinationen zwei mögliche Instantiierungen für (\mex{0}):
\ea
\begin{tabular}[t]{@{}lll}
N[SUBCAT \ibox{1}] & $\to$ & Det N[SUBCAT \ibox{1} $\oplus$ \sliste{ \textrm{Det} } ]\\
V[SUBCAT \ibox{1}] & $\to$ & V[SUBCAT \ibox{1} $\oplus$ \sliste{ NP } ]~~ NP\\
\end{tabular}
\z
Entsprechend kann man das Schema~\ref{schema-bin-prel} instantiieren und bereits
mit diesem einen Dominanzschema komplexe Strukturen analysieren, die dem Baum in
Abbildung~\vref{fig-bin-Aicke-die-Nuss-dem-Eichhörnchen-gibt} entsprechen.
\begin{figure}[htp]
\centering
\centerline{%
\begin{forest}
sm edges
[{V[\comps \sliste{} ]}
  [{\ibox{1} NP[\type{nom}]}
    [Aicke]]
  [{V[\comps \sliste{ \ibox{1} }]}
    [{\ibox{2} NP[\type{acc}]} 
      [die Nuss, roof]]
    [{V[\comps \sliste{ \ibox{1}, \ibox{2} }]}
      [{\ibox{3} NP[\type{dat}]}
        [dem Eichhörnchen,roof]]
      [{V[\comps \sliste{ \ibox{1}, \ibox{2}, \ibox{3} }]}
        [gibt]]]]]
\end{forest}}
\caption{\label{fig-bin-Aicke-die-Nuss-dem-Eichhörnchen-gibt}Binär verzweigende Kopf-Argument-Strukturen}
\end{figure}
Man sieht recht deutlich, dass die Struktur in (\mex{1}) eigentlich dem auf die Seite gelegten
Baum entspricht.\footnote{
  \textsc{non-head-dtrs} habe ich mit \textsc{nh-dtrs} abgekürzt.%
}

\ea
%\resizebox{\linewidth}{!}{
\onems[head-argument-phrase]{
phon      \phonliste{ Aicke die Nuss dem Eichhörnchen gibt }\\
head-dtr \onems{
phon     \phonliste{ die Nuss dem Eichhörnchen gibt }\\
head-dtr \onems{
      phon     \phonliste{ dem Eichhörnchen gibt }\\
      head-dtr \onems{ phon \phonliste{ gibt }\\
                       } \\
      nh-dtrs \liste{ \onems{ 
                                                  phon     \phonliste{ dem Eichhörnchen }\\
                                                  head-dtr \onems{ phon \phonliste{ Eichhörnchen }  \\
                                                                 }\\
                                                  nh-dtrs \liste{ \onems{ phon \phonliste{ dem } \\
                                                                              }}\\
                                                  }
                       }\\
}\\
nh-dtrs \liste{ \onems{ 
                                                  phon     \phonliste{ die Nuss }\\
                                                  head-dtr \onems{ phon \phonliste{ Nuss }  \\
                                                                 }\\
                                                  nh-dtrs \liste{ \onems{ phon \phonliste{ die } \\
                                                                              }}\\
                                                  }
                       }\\
} \\
nh-dtrs \liste{ \onems{phon \phonliste{ Aicke }\\
                       } \\
}\\
}
%}
\z
Die Struktur in (\mex{0}) enthält nur den Ausschnitt aus der Beschreibung des linguistischen
Zeichens, der etwas über \phonwe und den Status der Töchter (Kopf"|tochter vs.\ Nicht"=Kopf"|tochter)
aussagt. (\mex{1}) zeigt andere Details der Struktur wie die Information über Wortart
und Valenz:
\ea
\label{ms-dem-mann-gibt}
\onems[head-argument-phrase~]{
      phon    \phonliste{ dem Eichhörnchen gibt }\\
      comps \ibox{1} \liste{ NP[\nom], NP[\acc]} \\
      head-dtr \onems{ phon \phonliste{ gibt }\\
                                     comps \ibox{1}  $\oplus$ \liste{ \ibox{2} } \\
                       } \\
      non-head-dtrs \liste{ \ibox{2} \onems[head-argument-phrase~]{ 
                                                  phon  \phonliste{ dem Eichhörnchen }\\
                                                  p-o-s \type{noun}\\
                                                  comps \liste{} \\
                                                  head-dtr \ldots \\
                                                  non-head-dtrs \ldots \\
                                                  }
                       }\\
}
\z
Natürlich hat die Repräsentation für \emph{dem Eichhörnchen} selbst wieder eine interne Struktur,
die aber aus Gründen der Übersichtlichkeit weggelassen wurde.
Abbildung~\vref{fig-saettigung-valenz} zeigt die Valenzinformation in Baumdarstellung.
\begin{figure}
\centerline{%
\begin{forest}
sm edges
[{V[\comps \sliste{} ]}
  [{\ibox{1} NP[\type{nom}]}
    [Aicke]]
  [{V[\comps \sliste{ \ibox{1} }]}
    [{\ibox{2} NP[\type{acc}]} 
      [die Nuss, roof]]
    [{V[\comps \sliste{ \ibox{1}, \ibox{2} }]}
      [{\ibox{3} NP[\type{dat}]}
        [dem Eichhörnchen,roof]]
      [{V[\comps \sliste{ \ibox{1}, \ibox{2}, \ibox{3} }]}
        [gibt]]]]]
\end{forest}}
\caption{\label{fig-saettigung-valenz}Abarbeitung der Valenzliste des Verbs}
\end{figure}

%% \ea
%% \onems[head-argument-phrase~]{
%%       phon    \phonliste{ er die Nuss dem Eichhörnchen gibt }\\
%%       comps \ibox{1} \liste{ } \\
%%       head-dtr \onems[head-argument-phrase]{ phon \phonliste{ die Nuss dem Eichhörnchen gibt }\\
%%                                        comps \ibox{1}  $\oplus$ \liste{ \ibox{2} } \\
%%                        } \\
%%       non-head-dtrs \liste{ \ibox{2} \onems[word]{ 
%%                                        phon \phonliste{ er }\\
%%                                        p-o-s \textit{noun}\\
%%                                        comps  \liste{} \\
%%                                       }
%%                        }\\
%% }
%% \z





\section{Projektion von Kopfeigenschaften}
\label{sec-kopfeigenschaften}

Auf Seite~\pageref{bsp-projektion-v-merkmale} wurde gezeigt, dass
die Verbform zu den Merkmalen zählt, die für die Distribution von Verbalprojektionen
wichtig ist. Die Wortart und Information über Finitheit muss also an der Maximalprojektion
von Verben wie \emph{gibt} repräsentiert sein. Das verdeutlicht Abbildung~\vref{fig-projektion-head-feat}.
\begin{figure}
\settowidth{\offset}{V[\type{fi}}
\settowidth{\offsetup}{V[\type{fin}}
\centerline{
\begin{forest}
sm edges, for tree={l+=\baselineskip}
[V{[\type{fin}, \comps \eliste]}, name=fin1
	[\ibox{1} NP{[\type{nom}]}
		[Aicke]]
	[V{[\type{fin}, \comps \sliste{ \ibox{1} }]}, name=fin2
		[\ibox{2} NP{[\textit{dat}]}
			[dem Eichhörnchen,roof]]
		[V{[\type{fin}, \comps \sliste{ \ibox{1}, \ibox{2} }]}, name=fin3
			[\ibox{3} NP{[\textit{acc}]}
				[die Nuss,roof]]
			[V{[\type{fin}, \comps \sliste{ \ibox{1}, \ibox{2}, \ibox{3} }]}, name=fin4
				[gibt]]]]]	
tikz={\draw[<->] ($(fin1.south west)+(\offsetup,0)$) to ($(fin2.north west)+(\offset,0)$);
      \draw[<->] ($(fin2.south west)+(\offsetup,0)$) to ($(fin3.north west)+(\offset,0)$);
      \draw[<->] ($(fin3.south west)+(\offsetup,0)$) to ($(fin4.north west)+(\offset,0)$);}
\end{forest}
}
\caption{\label{fig-projektion-head-feat}Projektion der Kopfmerkmale des Verbs}
\end{figure}
Die Verbform\isfeat{vform} ist bisher noch gar nicht in Merkmalbeschreibungen enthalten gewesen.
Eine mögliche Merkmalsgeometrie zeigt (\mex{1}).
\ea
      \ms{ phon   & list~of~phoneme strings\\
           p-o-s  & p-o-s \\
           vform  & vform \\ 
           comps & list~of~signs\\
         }\\
\z
Diese Struktur ist für das, was erreicht werden soll, aber wenig geeignet.
Wir wollen immer sowohl die Information über die Wortart und die Verbform gemeinsam
projizieren. Man könnte das zwar erreichen, indem man jeweils einzelne Strukturteilungen
zwischen den \textsc{p-o-s}- und \textsc{vform}"=Werten einer verbalen Kopf"|tochter und
ihrer Mutter vornimmt, eine Bündelung der Information, die projiziert werden soll,
ist jedoch adäquater: Man führt ein neues Merkmal \textsc{head}\isfeat{head} ein, dessen Wert
eine komplexe Struktur ist, die alle Merkmale enthält, deren Werte für die
Erklärung der Distributionseigenschaften\label{page-kopf-merkmal} der Gesamtprojektion relevant sind.
\ea
      \ms{ phon & list~of~phoneme strings\\
           head & \ms{
                  p-o-s & p-o-s\\
                  vform & vform\\
                  } \\
           comps & list~of~signs\\
         }
\z
Nun ist es aber so, dass Köpfe je nach ihrer Wortart ganz verschiedene Eigenschaften
besitzen und somit auch unterschiedliche Merkmale projiziert werden müssen.
Das Merkmal \textsc{vform} ist nur für Verben sinnvoll. Pränominale Adjektive und 
Nomina projizieren dagegen Kasus. Verben selegieren zwar Argumente, die einen
bestimmten Kasus haben müssen, sie haben aber selbst keine Kasusmarkierung.
Man könnte alle Merkmale, die projiziert werden können, in einer Struktur wie
(\mex{1}) zusammenfassen und dann sagen, dass \textsc{case} bei Verben und \vform
bei Nomina keine Werte (bzw.\ einen Wert wie \emph{none}) haben.
\ea
      \ms{
                  p-o-s & p-o-s\\
                  vform & vform\\
                  case  & case\\
         }
\z
Besser ist es jedoch, davon auszugehen, dass die Merkmalstrukturen für
Verben und Nomina Objekte verschiedenen Typs sind: Merkmalstrukturen,
die Verben modellieren, sind vom Typ \type{verb}\istype{verb} und haben deshalb
ein Merkmal \textsc{vform}. Merkmalstrukturen, die Nomina modellieren,
sind vom Typ \type{noun}\istype{noun} und haben ein
Kasusmerkmal\isfeat{case}\label{page-ref-case-feat}. (\mex{1}) und (\mex{2}) zeigen entsprechende
Merkmalbeschreibungen.
\ea
\ms[verb]{
vform & vform\\
}
\z
\ea
        \ms[noun]{
        case & case\\
        }
\z

\noindent
Der Lexikoneintrag für \emph{gibt}, der auf Seite~\pageref{le-gibt-1}
angegeben wurde, kann jetzt wie folgt präzisiert werden:
\ea
\textit{gibt\/}:\\
\ms[word]{ 
     phon   & \phonliste{ gibt }\\
     head   & \ms[verb]{ vform & fin \\} \\
     comps & \liste{ NP[\textit{nom\/}], NP[\textit{acc}], NP[\textit{dat}] } \\
}
\z
Ein Lexikoneintrag enthält also phonologische Information, Informationen über die Kopfeigenschaften (\emph{part of speech}, Verbform, \ldots)
und Valenzinformation (eine Liste von Merkmalbeschreibungen).

Bis jetzt gibt es in der hier dargestellten Theorie noch nichts, was sicherstellen
würde, dass die Information, die in Lexikoneinträgen unter \textsc{head} repräsentiert
ist, auch zum Mutterknoten kommt. Dies wird durch das Kopfmerkmalsprinzip (\emph{Head Feature Principle})
sichergestellt.
\begin{prinzip-break}[Kopf"|merkmalprinzip~(\textit{Head~Feature~Principle})]\is{Prinzip!Kopfmerkmals-}
\label{prinzip-hfp}
In einer Struktur mit Kopf sind die Kopfmerkmale der Mutter identisch (teilen die Struktur)
mit den Kopfmerkmalen der Kopf"|tochter.
\end{prinzip-break}
Bisher wurde erst ein Dominanzschema behandelt, aber in den kommenden Kapiteln werden
noch weitere Schemata dazukommen, \zb Schemata für Kopf"=Adjunkt"=Strukturen und
für die Abbindung von Fernabhängigkeiten. Das Kopfmerkmalsprinzip ist ein allgemeines Prinzip,
das für alle durch diese Schemata lizenzierten Strukturen erfüllt sein muss. Es muss --
wie oben verbal ausgedrückt -- für alle Strukturen mit Kopf gelten. Formal kann man
das erfassen, indem man Strukturen einteilt in solche mit Kopf und solche ohne
und denen, die einen Kopf haben, den Typ \type{headed"=phrase} zuweist.
Der Typ \type{head"=argument"=phrase} -- das ist der Typ, den die Beschreibung in
Schema~\ref{schema-bin-prel} auf Seite~\pageref{schema-bin-prel} hat -- ist ein Untertyp
von \type{headed"=phrase}. Objekte eines bestimmten Typs haben immer 
alle Eigenschaften, die Objekte entsprechender Obertypen haben. Es sei an das Beispiel
aus Kapitel~\ref{sec-formalismus-typen} erinnert: Ein Objekt des Typs \textit{frau}
hat alle Eigenschaften des Typs \textit{person}. Darüber hinaus haben diese Objekte
noch weitere, spezifischere Eigenschaften, die andere Untertypen von \type{person}
nicht teilen.

Man kann also formal eine Beschränkung für einen Obertyp festlegen und erreicht damit
automatisch, dass alle Untertypen genauso beschränkt sind. Das Kopfmerkmalsprinzip
läßt sich wie folgt formalisieren:
\ea
\type{headed"=phrase}\istype{headed"=phrase} \impl
\ms{ 
head \ibox{1}\\
head-dtr$|$head \ibox{1}\\
} 
\z
Der Pfeil\is{\impl} entspricht der Implikation\is{Implikation} aus der Logik.
(\mex{0}) kann man wie folgt lesen:
Wenn eine Struktur vom Typ \type{headed"=phrase} ist, dann muss gelten, dass
der Wert von \textsc{head} mit dem Wert von \textsc{head-dtr$|$head} identisch ist.
Einen Ausschnitt aus der Typhierarchie unter \type{sign} zeigt Abbildung~\vref{fig-type-sign}.
\begin{figure}
\centerline{%
\begin{forest}
type hierarchy
[sign,
 for tree={
   calign=fixed angles,
   calign angle=60
 } 
  [word]
  [phrase
    [non-headed-phrase]
    [headed-phrase
      [\ldots]
      [head-argument-phrase]
      [\ldots]]]]
\end{forest}}
\caption{\label{fig-type-sign}Typhierarchie für \type{sign}: alle Untertypen von \type{headed"=phrase} erben Beschränkungen.}
\end{figure}
\type{word} und \type{phrase} sind Unterklassen sprachlicher Zeichen.
Phrasen kann man unterteilen in Phrasen ohne Kopf (\type{non"=headed"=phrase})
und Phrasen mit Kopf (\type{headed"=phrase}). Auch für Phrasen vom Typ
\type{non"=headed"=phrase} bzw.\ \type{headed"=phrase} gibt es Untertypen.
Den Typ \type{head"=argument"=phrase} haben wir bereits besprochen. Auf andere
Typen werden wir in den kommenden Kapiteln eingehen.

Die Beschreibung in (\mex{1}) zeigt das Kopf"=Argument"=Schema 
von Seite~\pageref{schema-bin-prel} zusammen mit den
Beschränkungen, die der Typ \type{head"=argument"=phrase} von \type{headed"=phrase} erbt.
\eas
Kopf-Argument-Schema + Kopfmerkmalsprinzip:\\
\ms[head-argument-phrase~]{
head   & \ibox{1} \\
comps & \ibox{2} \\[2mm]
head-dtr  & \ms{ head   & \ibox{1} \\
                 comps & \ibox{2} $\oplus$ \sliste{ \ibox{3} } \\
               } \\
non-head-dtrs  & \sliste{ \ibox{3} } \\
}
\zs
(\mex{1}) zeigt eine Beschreibung der Struktur, die durch das Schema~\ref{schema-bin-prel} lizenziert wird.
Zusätzlich zur Valenzinformation, die schon in (\ref{ms-dem-mann-gibt}) enthalten war,
ist in (\mex{1}) noch die Kopf"|information spezifiziert, und es ist zu sehen, wie
das Kopfmerkmalsprinzip die Projektion der Merkmale erzwingt: Der Kopfwert der gesamten Struktur
\iboxb{1} entspricht dem Kopfwert des Verbs \emph{gibt}.
\ea
\onems[head-argument-phrase~]{
      phon  \phonliste{ dem Eichhörnchen gibt }\\
      head  \ibox{1}\\
      comps \ibox{2} \sliste{ NP[\nom], NP[\acc]} \\
      head-dtr \onems[word]{ phon \phonliste{ gibt }\\
                                     head \ibox{1} \ms[verb]{vform & fin \\
                                                            }\\
                                     comps \ibox{2}  $\oplus$ \sliste{ \ibox{3} } \\
                       } \\
      non-head-dtrs \liste{ \ibox{3} \onems[head-argument-phrase~]{ 
                                        phon \phonliste{ dem Eichhörnchen }\\
                                        head \ms[noun]{ cas & dat\\
                                                      } \\
                                        comps   \eliste \\
                                        head-dtr \ldots\\
                                        non-head-dtrs \ldots\\
                                     }
                       }\\
}
\z

\noindent
Für den gesamten Satz \emph{er die Nuss dem Eichhörnchen gibt} bekommt man -- wie schon in
Abbildung~\ref{fig-projektion-head-feat} dargestellt -- eine Struktur, die durch (\mex{1})
beschrieben wird:
\ea
\ms{
head \ms[verb]{vform & fin \\
              }\\
comps \eliste\\
}
\z
Diese Beschreibung entspricht dem Satzsymbol S in der Grammatik auf
Seite~\pageref{bsp-grammatik-psg}, wobei (\mex{0}) noch Information über die Verbform enthält.


Damit sind bereits einige der wesentlichen Konzepte der HPSG"=Theorie eingeführt. Im folgenden
Kapitel wird gezeigt, wie man den semantischen Beitrag von Wörtern repräsentieren und den
semantischen Beitrag von Phrasen kompositionell bestimmen kann.






\questions{

\begin{enumerate}
\item Wie kann man Konstituentenstrukturen mit Hilfe von Merkmalstrukturen modellieren?
\item Wodurch zeichnen sich Köpfe gegenüber Nichtköpfen aus?
\item Wie wird erreicht, dass in bezug auf den Kopf relevante Information auch auf phrasalem
      Niveau verfügbar ist?
\end{enumerate}

}

\exercises{

\begin{enumerate}
\item Zeichnen Sie einen Syntaxbaum für (\mex{1}):
\ea
dass der Mann das spannende Buch liest
\z
Markieren Sie die Kanten im Baum mit Ad für Adjunkt, Ar für Argument und
H für Kopf.
\item Geben Sie die vollständige Merkmalbeschreibung für die Wortfolge in (\mex{1}), wie sie in
  \emph{dass das Kind schläft} vorkommt,  an:
      \ea
      das Kind schläft
      \z
\item\label{uebung-grammix-kapitel4}
      Dem Buch liegt eine CD mit Implementationen von Grammatiken bei, die
      den jeweiligen Kapiteln entsprechen. Die CD enthält ein eigenes Betriebssystem,
      so dass Sie unabhängig vom auf dem Computer installierten Betriebssystem
      mit der CD arbeiten können. Voraussetzung ist lediglich ein Intel"=kompatibler Prozessor.

      Konfigurieren Sie Ihren Computer so, dass er von einer eingelegten CD bootet und starten
      Sie den Computer von der Grammix"=CD. Es sollte ein Startbildschirm im Corporate
      Design der Universität Bremen erscheinen. Wenn der Bootvorgang abgeschlossen ist,
      sehen Sie einen Bildschirm mit einem Begrüßungsfenster in einem Web"=Browser und
      mit verschiedenen Icons, die den einzelnen Grammatiken
      entsprechen. Klicken Sie auf das Icon, unter dem \emph{Grammatik 4} steht. Es werden sich
      zwei Fenster öffnen, und die Grammatik~4 wird im größeren der beiden Fenster geladen.
      Nach Beendigung des Ladevorgangs erscheint ein Prompt (\texttt{>>>}). An dieser Stelle
      können Sie Sätze eingeben, die dann vom System mittels der jeweils geladenen Grammatik
      analysiert werden. Die Lexika, die zu den Grammatiken gehören, sind sehr klein,
      aber der Leser kann sie bei Bedarf selbst erweitern.

      Weitere Details zum Arbeiten mit der Grammix"=CD finden Sie im Begrüßungsfenster,
      das beim Starten der CD geöffnet wird, oder auf der zur CD gehörenden Web"=Seite \url{https://hpsg.hu-berlin.de/Software/Grammix/}.

      Analysieren Sie die folgenden Sätze:
      \eal
      \ex Der Mann schläft.
      \ex der Mann die Frau kennt
      \zl
      Nach der Eingabe der Sätze öffnet sich ein weiteres Fenster, das Chart"=Display.
      Das Chart"=Display ist eine Visualisierung der Prozesse, die bei der automatischen
      Analyse von Sätzen ablaufen. Die einzelnen Teilphrasen, die von der
      Grammatik lizenziert werden, sind durch Kanten im Chart"=Display dargestellt. Die Kanten
      kann man anzeigen lassen, indem man sie anklickt.

      Nach einer erfolgreichen Analyse einer Eingabe öffnet sich ein weiteres Fenster mit einem
      Syntaxbaum. Der Syntaxbaum enthält die Phonologie"=Werte der im Baum enthaltenen
      linguistischen Objekte und Information über den Typ dieser Objekte.
      Klickt man die Knoten im Baum an, wird die zugehörige Merkmalbeschreibung
      angezeigt. Ein weiterer Klick bringt sie wieder zum Verschwinden. Mit der rechten Maustaste
      kann man Teile von Beschreibungen ausblenden. Durch Klicken auf die Boxen für die Strukturteilung
      kann man die Werte, die zu den entsprechenden Boxen gehören, ein- und ausblenden.

      Lassen Sie sich die Merkmalbeschreibungen für die Beispiele in (\mex{0}) anzeigen!
      Überlegen Sie, wieso die Merkmale die Werte haben, die sie haben!
      
      Geben Sie die ungrammatischen Wortfolgen in (\mex{1}) (ohne den Stern) ein:
      \eal
      \ex[*]{
      Mann schläft.
      }
      \ex[*]{
      Der Mann kennt.
      }
      \zl
      Inspizieren Sie mit Hilfe des Chart"=Displays die Parse"=Chart! Welche Wörter
      werden zu Wortgruppen zusammengebaut, welche nicht? Warum ist das so?
\end{enumerate}
}
