%% -*- coding:utf-8 -*-
%%%%%%%%%%%%%%%%%%%%%%%%%%%%%%%%%%%%%%%%%%%%%%%%%%%%%%%%%
%%   $RCSfile: 5-hpsg-semantik.tex,v $
%%  $Revision: 1.22 $
%%      $Date: 2008/09/30 09:14:41 $
%%     Author: Stefan Mueller (CL Uni-Bremen)
%%    Purpose: 
%%   Language: LaTeX
%%%%%%%%%%%%%%%%%%%%%%%%%%%%%%%%%%%%%%%%%%%%%%%%%%%%%%%%%

\chapter{Semantik}
\label{chap-sem}


Innerhalb\is{Semantik|(} der HPSG"=Forschung gibt es zwei verschiedene Ansätze
zur Beschreibung der Bedeutung sprachlicher Ausdrücke. In den ersten
HPSG"=Arbeiten \citep{ps,ps2,Mueller99a} und auch in einigen Umsetzungen 
innerhalb der Computerlinguistik wurde die Situationssemantik
\citep*{BP83a,CMP90,Devlin92}\nocite{BP87a} verwendet.
Eine Publikation neueren Datums, die ebenfalls Situationssemantik verwendet,
ist \citew{GSag2000a-u}.
Einige aktuellere theoretische Arbeiten (\zb \citew{Kiss2001a}) benutzen
\emph{Minimal Recursion Semantics}\is{Minimal Recursion Semantics@\emph{Minimal Recursion Semantics} (MRS)}
(MRS) \citep*{CFPS2005a}. MRS ist für computerlinguistische Systeme gut geeignet, da
Skopusbeziehungen\is{Skopus} unterspezifiziert dargestellt werden können, was eine effizientere
Verarbeitung der Grammatiken durch Computer ermöglicht.

Im folgenden werde ich Situationssemantik nutzen, da die Theorie und ihre
Umsetzung in Merkmalstrukturen leichter zu verstehen ist.

\section{Die Situationssemantik}

In der Situationssemantik\is{Situationssemantik} werden Situationen\is{Situation} beschrieben.
Situationen sind durch Sachverhalte\is{Sachverhalt}, die in ihnen wahr sind, charakterisiert.
Dinge von einer gewissen zeitlichen Dauer, die zur kausalen Ordnung der
Welt gehören, die man wahrnehmen kann, auf die man reagieren kann,
werden als \emph{Individuen}\is{Individuum} bezeichnet (\emph{Karl}, \emph{die Frau}).
Dazu gehören auch Konzepte (\emph{die Angst}, \emph{das Versprechen}).
Sachverhalte sind Relationen zwischen solchen Individuen.
Relationen können verschiedene Stelligkeiten haben:
\begin{itemize}
\item nullstellig: \relation{regnen} (\emph{Es regnet.}) 
\item einstellig: \relation{ sterben} (\emph{Er stirbt.})
\item zweistellig: \relation{lieben}  (\emph{Er liebt sie.})
\item dreistellig: \relation{geben}   (\emph{Er gibt ihm den Aufsatz.})
\item vierstellig: \relation{kaufen}  (\emph{Er kauft den Mantel vom Händler für fünf Euro.})
\end{itemize}
Die Bezeichnungen der genannten Relationen entsprechen den Infinitivformen
der Verben. Für die Repräsentation der Tatsache, dass jemand jemanden liebt,
ist Flexionsinformation nicht relevant, weshalb man sich für die Bedeutungsrepräsentation
auf eine beliebige Form des Verbs geeinigt hat. Außerdem wird für die Bedeutungsrepräsentation
von der konkreten Realisierung im Satz abstrahiert. In der Prädikatenlogik\is{Prädikatenlogik} wird
\zb dem Satz (\mex{1}a) die Repräsentation (\mex{1}b) zugeordnet.
\eal
\ex Peter liebt Maria.
\ex \relation{lieben}(Peter', Maria')
\zl
Dieselbe Bedeutung kann aber auch durch den Passivsatz in (\mex{1}) ausgedrückt werden.
\ea
Maria wird von Peter geliebt.
\z

\noindent
Um Generalisierungen in bezug auf die Verbindung zwischen Syntax und Semantik (das sogenannte
\emph{Linking}\is{Linking}) vornehmen zu können, verwendet man zur Kennzeichnung der Argumente\is{Argument}
in Relationen semantische Rollen\is{semantische Rolle} \citep{Fillmore68,Fillmore77,Kunze91}. Beispiele für
solche Rollen sind: 
\textsc{agens}\is{semantische Rolle!Agens}, 
\textsc{patiens}\is{semantische Rolle!Patiens}, 
\textsc{experiencer}\is{semantische Rolle!Experiencer}, 
\textsc{source}\is{semantische Rolle!Source}, 
\textsc{goal}\is{semantische Rolle!Goal}, 
\textsc{thema}\is{semantische Rolle!Thema}, 
\textsc{location}\is{semantische Rolle!Location}, 
\textsc{trans-obj}\is{semantische Rolle!Trans-Obj}, 
\textsc{instrument}\is{semantische Rolle!Instrument}, 
\textsc{means}\is{semantische Rolle!Means} und 
\textsc{proposition}\is{semantische Rolle!Proposition}.
Die Zuordnung von Argumenten zu einzelnen Rollen kann von Autor zu Autor stark variieren.
\citet{Dowty91a} formuliert Kriterien dafür, welche Rolle man für welche Art von Argument
verwenden sollte. Er schlägt Bezeichnungen wie Proto"=Agens und Proto"=Patiens vor.
Ich werde im folgenden die etablierten Rollenbezeichnungen verwenden.

%\subsection{Parametrisierte Sachverhalte}

In der Situationssemantik werden Relationen immer in doppelten spitzen Klammern
geschrieben. Innerhalb der Klammern steht an erster Stelle der Name der Relation, gefolgt
von Paaren aus Rollenbezeichnungen und Variablen. (\mex{1}) zeigt Beispiele für Relationen, die einem Verb, einem Adjektiv
und einem Nomen entsprechen:\footnote{
  In der Situationssemantik gehört noch ein Polaritätsbit zu den semantischen Repräsentationen, das
  anzeigt, ob ein gewisser Sachverhalt in einer Situation gilt oder nicht.
  Da das hier aber nicht benötigt wird, lasse ich es im folgenden weg.%
}
\ea\is{Verb}\is{Nomen}\is{Adjektiv}
\begin{tabular}[t]{@{}ll}
Verb:     & $\ll schlagen, agens:X, patiens:Y\gg$\\
Adjektiv: & $\ll interessant, thema: X\gg$\\
Nomen:    & $\ll mann, instance: X\gg$\\
\end{tabular}
\z
Die Variablen in (\mex{0}) werden auch \emph{Parameter}\is{Parameter} genannt. Sachverhalte,
die solche Variablen enthalten, bezeichnet man als 
\emph{parametrisierte Sachverhalte}\is{Sachverhalt!parametrisierter ~}
(\emph{parametrized state of affairs} (\emph{psoa}))\label{page-psoa}\is{psoa}\is{SOA}. Parameter
können durch Sachverhalte restringiert werden. (\mex{2}) zeigt eine Repräsentation
für (\mex{1}):
\ea
\label{bsp-mann-schlaegt-hund}
Der Mann schlägt den Hund.
\z
\ea
\label{ex-sachverhalt}
$\ll schlagen, agens:X, patiens:Y\gg$\\
$X|\ll mann, instance:X\gg,$\\
$Y|\ll hund, instance:Y\gg$
\z
Der Sachverhalt \emph{schlagen} hat die beiden Parameter X und Y. X
bezieht sich auf etwas, das eine Instanz von \emph{mann} ist, und Y bezieht
sich auf etwas, das eine Instanz von \emph{hund} ist.

Die Bedeutungsrepräsentation für den Satz (\mex{1}) ist ganz analog:
\ea
Das Buch ist interessant.
\z
\ea
$\ll interessant, thema:X\gg$\\
$X|\ll buch, instance:X\gg$
\z
X ist das Thema der \relation{interessant}"=Relation,
und X bezieht sich auf ein Buch.

\section{Die Repräsentation parametrisierter Sachverhalte mit Hilfe von Merkmalstrukturen}
\label{sec-rep-semantik-merkmalstruktur}

Die situationssemantischen Ausdrücke lassen sich sehr leicht durch Merkmalstrukturen
modellieren: Die Relationsbezeichnung wird zu einem Typ, und alle Rolle"=Variable"=Paare
werden innerhalb der Merkmalstruktur als Merkmal"=Wert"=Paare aufgeführt. Wir erhalten
zum Beispiel die beiden folgenden Strukturen:
\ea
\ms[schlagen]
{ 
 agens & X \\
 patiens & Y\\
}
\z

\ea
\ms[mann]
{ 
 inst & X \\
}
\z


\section{Repräsentation des \textsc{content}-Wertes}


Die semantische Information muss in Lexikoneinträge integriert
werden. Das geschieht als Wert des Merkmals \textsc{content} (Abkürzung \textsc{cont}\isfeat{cont}).
Erweitert man die bisherige Beschreibung einfach um dieses Merkmal, so erhält man (\mex{1}):
\ea
      \ms{ phon   & list~of~phoneme strings\\
           head   & head \\
           comps & list~of~signs\\
           cont   & cont\\
         }\\
\z
Es ist aber sinnvoll, noch eine stärkere Gliederung vorzunehmen und zwischen syntaktischer und semantischer
Information zu trennen. Dazu führt man das Merkmal \textsc{category}\isfeat{cat} (Abkürzung \textsc{cat}) ein. Der Wert
von \textsc{cat} ist eine Merkmalstruktur mit den Merkmalen \textsc{head} und \textsc{comps}. (\mex{1})
zeigt eine entsprechende Beschreibung:
\ea
\label{geom-cat-cont}
      \ms{ phon & list~of~phoneme strings\\
           cat  & \ms{ head   & head\\
                            comps & list~of~signs\\
                          } \\
           cont & cont\\
         }
\z
Mit einer solchen Aufteilung der Information wird es möglich, durch eine einfache
Strukturteilung nur syntaktische Information zu teilen. Zum Beispiel möchte
man Fälle\is{Koordination|(} symmetrischer\label{page-symmetrische-koordination} Koordination wie in (\mex{1}) so behandeln, dass
man einfach die \textsc{cat}"=Werte der Konjunkte identifiziert:
\eal
\ex {}[der Mann] und [die Frau]
\ex Er [kennt] und [liebt] diese Schallplatte.
\ex Er ist [dumm] und [arrogant].
\zl
In den jeweiligen Beispielen in (\mex{0}) sind Konjunkte verschiedener Kategorien (N, V, A) und 
verschiedener Sättigungsgrade (gesättigt,
zwei offene Argumente, ein offenes Argument) miteinander
koordiniert worden. Allen Koordinationen ist jedoch gemein, dass die Konjunkte gleiche syntaktische
Eigenschaften haben. Dies wird durch Identifikation der \textsc{cat}"=Werte der Konjunkte erreicht.
Der \textsc{cat}"=Wert der gesamten Koordination entspricht den \textsc{cat}"=Werten der Konjunkte:
Die Verbindung aus \emph{kennt} und \emph{liebt} verhält sich so, wie sich \emph{kennt} und \emph{liebt}
einzeln verhalten würden.\is{Koordination|)}


\section{Nominale Objekte}
\label{sec-index}

Nominale Objekte führen einen Index (\textsc{ind}) ein. Diesen Index kann man
sich wie eine Variable vorstellen. Zum semantischen Beitrag nominaler Objekte
können auch Restriktionen (\textsc{restrictions}) zum eingeführten Index\isfeat{ind}\isfeat{restr} zählen.
Die Restriktionen schränken die Menge der Objekte ein, auf die man sich mit einem referentiellen
Index beziehen kann. Mit \relation{mann}(x) bezieht man sich auf ein \emph{x} mit der
Eigenschaft, Mann zu sein. Mit den beiden Restriktionen \relation{mann}(x) und \relation{klug}(x)
bezieht man sich auf eine Teilmenge der Männer, nämlich genau auf die, die klug sind.
Der Wert von \textsc{restrictions} ist eine Liste der jeweiligen Restriktionen.\footnote{
  Bei \citet[\page26]{ps2} ist der Wert von \textsc{restrictions} eine Menge\is{Menge}.
  Mengen unterscheiden sich von Listen dadurch, dass nichts über die Reihenfolge 
  der Elemente ausgesagt wird. Wie in Kapitel~\ref{kap-merkmalstrukturen} bereits angemerkt, ist die
  Formalisierung von Mengen recht kompliziert, weshalb ich auf die Einführung von Mengen verzichtet habe.%
}
(\mex{1}) zeigt den Lexikoneintrag für \emph{Buch}:
\ea
\label{le-buch}
Lexikoneintrag für \textit{Buch\/}:\\
\ms{ 
  cat & \ms{ head & noun\\
             comps & \sliste{ Det } \\
           } \\
  cont &  \ms
           { ind & \ibox{1} \ms{ per & 3 \\
                                 num & sg \\
                                 gen & neu \\
                               } \\
             restr & \liste{ \ms[buch]{ inst & \ibox{1} \\ }} \\
           } \\
}
\z
Zu einem Index gehören auch Merkmale wie Person\isfeat{per},\is{Person} Numerus\isfeat{num} und Genus\isfeat{gen}.\is{Kasus}\is{Genus}\is{Numerus}
Diese sind für die Bestimmung der Referenz/""Koreferenz wichtig. So kann
sich \emph{sie} in (\mex{1}) nur auf \emph{Frau}, nicht aber auf \emph{Buch}
beziehen. \emph{es} dagegen kann nicht auf \emph{Frau} referieren.
\ea
Die Frau$_i$ kauft ein Buch$_j$. Sie$_i$ liest es$_j$.
\z
Pronomina müssen im allgemeinen mit dem Element, auf das sie sich beziehen, in Person,
Numerus und Genus übereinstimmen. Die Indizes werden dann entsprechend identifiziert.
In der HPSG wird das mittels Strukturteilung gemacht. Man spricht auch von Koindizierung\is{Koindizierung}.
(\mex{1}) zeigt Beispiele für die Koindizierung von Reflexivpronomina\is{Pronomen!Reflexiv-}:
\eal
\ex Ich$_i$ sehe mich$_i$.
\ex Du$_i$ siehst dich$_i$.
\ex Er$_i$ sieht sich$_i$.
\ex Wir$_i$ sehen uns$_i$.
\ex Ihr$_i$ seht euch$_i$.
\ex Sie$_i$ sehen sich$_i$.
\zl
Welche Koindizierungen möglich und welche zwingend sind, wird durch die Bindungstheorie\is{Bindungstheorie}
geregelt. \citet{PS92a,ps2} haben eine Bindungstheorie im Rahmen der HPSG vorgestellt, die viele Probleme
nicht hat, die man bekommt, wenn man Bindung in bezug auf Verhältnisse in Bäumen erklärt. Allerdings bleiben
einige Fragen offen, die soweit ich weiß auch in anderen Theorien noch nicht zufriedenstellend beantwortet
wurden. Einige der Probleme wurden in \citew[Kapitel~20]{Mueller99a} diskutiert. In diesem Buch werde
ich auf die Bindungstheorie nicht weiter eingehen. Da es keine zufriedenstellenden Bindungstheorien gibt,
werde ich Bindungsdaten auch nicht für die Argumentation für bestimmte syntaktische Strukturen verwenden.

Die Struktur in (\mex{1}) zeigt einen Lexikoneintrag für das Pronomen \emph{er}. Das Pronomen hat eine
leere Valenzliste, da es keine weiteren Konstituenten benötigt, um als vollständige Nominalphrase in Sätzen
verwendet zu werden.
\ea
Lexikoneintrag für \emph{er}:\\
\ms{ 
  cat & \ms{ head   & noun\\
             comps & \sliste{ } \\
           } \\
  cont &  \ms
           { ind & \ms{ per & 3   \\
                        num & sg  \\
                        gen & mas \\
                      } \\
             restr & \liste{ } \\
           } \\
}
\z
Die Liste der Restriktionen ist die leere Liste, da wir, wenn wir einen Satz wie (\mex{1}) äußern, nur
wissen, dass wir von einem Individuum sprechen, auf das wir uns entweder deiktisch beziehen oder das
im bisherigen Diskurs mit einem Nomen in der dritten Person Singular maskulin erwähnt wurde.
\ea
Er schläft.
\z
Wir wissen außerdem noch, dass dieses Individuum schläft, aber diese Information kommt vom Verb, sie
gehört nicht zu den Relationen, die das Subjekt \emph{er} zur Gesamtbedeutung des Satzes beisteuert.

Im folgenden werden die Abkürzungen in (\mex{1}) verwendet:
\ea
\label{abkuerzungen-sem}
\begin{tabular}[t]{@{}lp{4.8cm}@{\hspace{0mm}}lp{4.2cm}@{}}
NP$_{[3,sg,fem]}$     & \onems{ cat \ms{ head & \type{noun} \\
                                         comps & \liste{} \\
                                       } \\
                                cont$|$ind \ms{ per & 3 \\
                                                num & sg \\
                                                gen & fem \\
                                              } \\
                              }  &
NP$_{\ibox{1}}$ & \onems{ cat  \ms{ head & \type{noun} \\
		                  comps & \liste{} \\
                                } \\
                       cont$|$ind \ibox{1}\\
                     }\\\\
\is{:}%
\baro{N}:\label{ex-abkuerzung-nbar} \ibox{1} & \ms{ cat & \ms{ head   & noun \\
                                     comps & \sliste{ Det } \\
                                   } \\
                          cont & \ibox{1} \\
                        } \\
\end{tabular}
\z
Tiefer gestellte Werte beziehen sich auf Index"=Merkmale, der Doppelpunkt
auf den Wert von \cont und \baro{N} entspricht einer Nominalprojektion
mit dem Bar"=Level~1 aus der \xbart. In der HPSG handelt es sich bei \baro{N}
um ein nominales Objekt, das in seiner Valenzliste noch einen Determinierer enthält.





\section{Repräsentation parametrisierter Sachverhalte mit Merkmalstrukturen}

Den parametrisierten Sachverhalt aus (\ref{ex-sachverhalt}), hier der Übersichtlichkeit
halber als (\mex{1}) wiederholt, kann man wie in (\mex{2}) mittels Merkmalbeschreibungen beschreiben.
\ea
\label{ex-sachverhalt-zwei}
$\ll schlagen, agens:X, patiens:Y\gg$\\
$X|\ll mann, instance:X\gg,$\\
$Y|\ll hund, instance:Y\gg$
\z
\ea
\ms[schlagen]
{ 
 agens & \ibox{1} \\
 patiens & \ibox{2} \\
}
\bigskip

\ms{
ind & \ibox{1} \ms{ per & 3 \\
                    num & sg \\
                    gen & mas \\
                  } \\
restr & \liste{ \ms[mann]{ 
                inst & \ibox{1} \\
                }}\\
}
\hspace{1.5cm}
\ms{
ind & \ibox{2} \ms{ per & 3 \\
                    num & sg \\
                    gen & mas \\
                  }\\
restr & \liste{ \ms[hund]{
                inst & \ibox{2} \\
                }} \\
}
\z
Dabei steht die \ibox{1} in der oberen Beschreibung und in der linken unteren Beschreibung
für den gleichen Wert. Entsprechendes gilt für die \ibox{2}. Strukturteilung über mehrere Merkmalbeschreibungen
hinweg wie in (\mex{0}) ist eigentlich nicht sinnvoll, aber die Teilstrukturen in (\mex{0}) sind 
nur Ausschnitte aus einer größeren Beschreibung, und innerhalb von größeren Beschreibungen kann man natürlich solche Strukturteilungen
vornehmen. Welche Gesamtstrukturen lizenziert werden, wird in Kürze gezeigt werden.

\section{Linking}
\label{sec-linking}
\is{Linking|(}

Bisher gibt es noch keine Zuordnung der Elemente in der Valenzliste\is{Valenz} zu
Argumentrollen\is{semantische Rolle} im semantischem Beitrag.
Eine solche Verbindung wird auch \emph{Linking} genannt.
(\mex{1}) zeigt, wie einfach das Linking in der HPSG funktioniert.
Die referentiellen Indizes der Nominalphrasen sind mit den semantischen Rollen identifiziert.\is{Argument}
\ea
\label{le-geben}
\textit{gibt\/} (finite Form):\\
\ms
{ cat & \ms{ head & \ms[verb]
                     { vform & fin \\} \\
             comps & \liste{ NP[\textit{nom\/}]\ind{1}, NP[\textit{acc}]\ind{2}, NP[\textit{dat}]\ind{3}   } \\
           } \\
  cont &  \ms[geben]{ agens & \ibox{1} \\
                      thema & \ibox{2} \\
                      goal  & \ibox{3} \\
           } \\
}
\z
Dadurch dass für die Argumentrollen allgemeine Bezeichnungen wie \textsc{agens} und \textsc{patiens}
verwendet werden, kann man Generalisierungen über Valenzklassen und
Argumentrollenrealisierungen aufstellen. Man kann \zb Verben in Verben mit Agens, Verben mit Agens
und Thema, Verben mit Agens und Patiens usw.\ unterteilen. Verschiedene Valenz/""Linking"=Muster kann man
in Typhierarchien spezifizieren und die spezifischen Lexikoneinträge diesen Mustern zuordnen,
\dash von den entsprechenden Typen Beschränkungen erben lassen. Eine Typbeschränkung für
Verben mit Agens, Thema und Goal hat die Form in (\mex{1}):
\ea
\onems
{ cat$|$comps \liste{ []\ind{1}, []\ind{2}, []\ind{3}  } \\[1mm]
  cont  \ms[agens-thema-goal-rel~]{ agens & \ibox{1} \\
                      thema & \ibox{2} \\
                      goal  & \ibox{3} \\
           } \\
}
\z
[]\ind{1} steht dabei für ein Objekt mit beliebiger syntaktischer Kategorie und dem Index
\iboxt{1}. 
%% \etag steht für einen unspezifizierten Wert. In (\mex{0}) kann \etag die leere Liste
%% sein oder aber auch eine Liste mit einem oder mehreren Elementen. Wichtig ist nur, dass
%% die \compsl mit drei Elementen beginnt, deren Index mit Werten von Argumentrollen innerhalb von \cont
%% identisch ist.
Der Typ für die Relation \textit{geben\/} ist Untertyp von \textit{agens-thema-goal-rel\/}.
Der Lexikoneintrag für das Wort \emph{geben} bzw.\ den Stamm \stem{geb} hat das Linking"=Muster in (\mex{0}).

Zu Linking"=Theorien im Rahmen der HPSG siehe \citew{Davis96a-u} und \citew{Wechsler91a-u}.
\is{Linking|)}

\section{Der semantische Beitrag von Phrasen}
\label{sec-semp-i}

Bisher wurde nur etwas darüber ausgesagt, wie der Bedeutungsbeitrag von Lexikoneinträgen repräsentiert
werden kann. Wie sich der semantische Beitrag von Phrasen ergibt, ist noch nicht geklärt.
Es ist klar, dass der semantische Beitrag einer Phrase wesentlich von deren Kopf bestimmt
wird. In Kopf"=Argument"=Strukturen ist der semantische Hauptbeitrag der Phrase mit dem
Beitrag des Kopfes identisch. Das verdeutlicht Abbildung~\vref{abb-sem-proj}.
\begin{figure}
\centering%
\oneline{%
\begin{forest}
sm edges, for tree={l sep+=\baselineskip}
[{V[\type{fin}, \comps \sliste{ } ]}, name=1
  [ {\ibox{1} NP[\type{nom}]}
    [Aicke]]
  [{V[\type{fin}, \comps \sliste{ \ibox{1} } ]}, name=2
    [ {\ibox{2} NP[\type{dat}]}
      [dem Eichhörnchen, roof]]
    [{V[\type{fin}, \comps \sliste{ \ibox{1}, \ibox{2} } ]}, name=3
      [ {\ibox{3} NP[\type{acc}]}
        [die Nuss, roof]]
      [{V[\type{fin}, \comps \sliste{ \ibox{1}, \ibox{2}, \ibox{3} } ]}, name=4
        [gibt]]]]]
\draw[<->] (4) to[out=east,in=east] (3);
\draw[<->] (3) to[out=east,in=east] (2);
\draw[<->] (2) to[out=east,in=east] (1);
\node()[above right of=4,xshift=6em,yshift=\baselineskip]{$geben(a,e,n)$};
\end{forest}}
\caption{\label{abb-sem-proj}Projektion des semantischen Beitrags des Kopfes}
\end{figure}
$a$, $e$ und $n$ stehen dabei verkürzend für \emph{Aicke}, \emph{dem Eichhörnchen} und \emph{die Nuss}.

Diese Identifikation des semantischen Beitrags von Kopf"|tochter und Mutterknoten
stellt das Semantikprinzip sicher:
% In Strukturen mit Kopf, die keine Kopf-Adjunkt-Strukturen sind, ist der
% semantische Beitrag der Mutter identisch mit dem der Kopf"|tochter.
%
\begin{prinzip-break}[Semantikprinzip (vorläufige Version)]
\label{semp-i}
In Strukturen, in denen es eine Kopf"|tochter gibt, ist der
semantische Beitrag der Mutter identisch mit dem der Kopf"|tochter.
\end{prinzip-break}
Diese Beschränkung gilt nicht für Kopf"=Adjunkt"=Strukturen, 
die im Kapitel~\ref{chap-adjunkte} behandelt werden.
Führt man einen Typ \textit{head-non-adjunct-phrase} ein, von dem alle Typen
erben, die keine Adjunktstrukturen sind, kann man das Semantikprinzip wie folgt formalisieren:
\ea
\textit{head-non-adjunct-phrase}\istype{head"=non"=adjunct"=phrase} \impl
\ms{
cont & \ibox{1} \\
head-dtr$|$cont & \ibox{1} \\
}
\z

\noindent
Abbildung~\vref{abb-sign-non-ha-struc} zeigt, wie der Typ \type{head"=non"=adjunct"=phrase}
in die in Abbildung~\ref{fig-type-sign} entwickelte Hierarchie unter \textit{sign} integriert werden muss.
\begin{figure}
\centerline{
\begin{forest}
type hierarchy
[sign,
 for tree={
   calign=fixed angles,
   calign angle=60
 } 
  [word]
  [phrase
    [non-headed-phrase]
    [headed-phrase, l sep+=\baselineskip
      [\ldots]
      [head-non-adjunct-phrase, calign with current
        [head-argument-phrase]]
      [\ldots]]]]
\end{forest}}
\caption{\label{abb-sign-non-ha-struc}Typhierarchie für \textit{sign}}
\end{figure}

Die Merkmalbeschreibung in (\mex{1})\vpageref{ha-struc-mit-constraints} zeigt alle Beschränkungen,
die für Strukturen vom Typ \textit{head-argument-phrase\/} gelten.
Das sind die Beschränkungen des Schemas~\ref{schema-bin-prel} und die von \textit{headed-phrase\/} und 
\textit{head-non-adjunct-phrase\/} ererbte Information:
\ea
Head-Argument-Schema + Kopfmerkmalsprinzip + Semantikprinzip:\\
\onems[head-argument-phrase~]{
\label{ha-struc-mit-constraints}
cat  \ms{ head   & \ibox{1} \\
           comps & \ibox{2} \\ 
         }\\
cont  \ibox{3}\\[2mm]
head-dtr   \ms{ cat & \ms{ head   & \ibox{1} \\
                            comps & \ibox{2} $\oplus$ \sliste{ \ibox{4} } \\
                          } \\
                 cont & \ibox{3} \\
               }\\
non-head-dtrs  \sliste{ \ibox{4} } \\
}
\z

\noindent
Die Verbindung zwischen semantischem Beitrag des Verbs und seinen Argumenten wird im Lexikoneintrag
des Verbs hergestellt. Somit ist dafür gesorgt, dass \emph{e}, \emph{b} und \emph{m} in 
Abbildung~\ref{abb-sem-proj} den Indizes der jeweiligen NPen entsprechen. Zusätzlich zur Hauptbedeutung
\relation{geben}($e,b,m$) müssen die zu den Indizes gehörenden Restriktionen in der Repräsentation der Bedeutung
des gesamten Ausdrucks enthalten sein. Das erreicht man, indem man die semantische Information aus den
NPen ebenfalls im Baum nach oben gibt. Dazu kann man \zb eine Liste verwenden, in die beim Zusammenbau
komplexer Einheiten der Bedeutungsbeitrag von NPen aufgenommen wird. Leider ist die Sache nicht ganz so einfach,
da es in Sätzen mit Quantoren\is{Quantor} wie \emph{alle}, \emph{jeder}, \emph{ein} usw.\ komplexe Interaktionen
zwischen mehreren Quantoren geben kann, die zu verschiedenen Lesarten führen \citep{Frey93a,Kiss2001a}.
Auf diesen Phänomenbereich kann hier nicht weiter eingegangen werden.
Einige Anmerkungen finden sich noch im Kapitel~\ref{sec-spec-kopf}.%
\is{Semantik|)}


\begin{comment}
Wir hatten für den Satz (\ref{bsp-mann-schlaegt-hund}) die semantische
Repräsentation in (\ref{ex-sachverhalt}) die hier als (\ref{ex-sachverhalt-drei})
wiederholt ist:
\ea
\label{ex-sachverhalt-drei}
$\ll schlagen, agens:X, patiens:Y\gg$\\
$X|\ll mann, instance:X\gg,$\\
$Y|\ll hund, instance:Y\gg$
\z
Bisher wurde nur gezeigt, wie der semantische Hauptbeitrag des Verbs zum 
Beitrag der gesamten Struktur wird, aber auch die Information über die beiteiligten
Nominalphrasen muss am Mutterknoten repräsentiert sein. Dazu wird das Merkmal \textsc{indices}\isfeat{indices}
verwendet \citep[Kapitel~4.4]{ps}. Es steht innerhalb einer Merkmalstruktur, 
die Information über den Äußerungskontext enthält \citep[\page22]{ps2}.
Das Merkmal, das die Information über den Äußerungskontext enthält, heißt \textsc{context}\isfeat{context}.
%Beim Zusammenbau der 
\ea
\onems
{   cont \ibox{1} \ms{ ind & \ibox{2} \ms{  per & 3 \\
                                            num & sg \\
                                            gen & mas \\
                                         } \\
                       restr & \liste{\ms[mann]{
                                      instance & \ibox{2} \\
                                      }} \\
                     } \\
   context$|$inds   \textrm{\sliste{ \ibox{1} }} \\
}
\z

\noindent
Der \textsc{indices}"=Wert eines phrasalen Zeichens ergibt sich aus der Verknüpfung
der \textsc{indices}"=Werte der Töchter. (\mex{1}) zeigt die Auswirkungen einer
entsprechenden Beschränkung für binär verzweigende Strukturen, die eine Kopf"|tochter haben:
\ea
\onems{
context$|$inds  \ibox{1} $\oplus$ \ibox{2} \\
head-dtr$|$context$|$inds \ibox{1} \\
non-head-dtrs \liste{ \onems{
                       context$|$inds \ibox{2} \\
                      }}\\
}
\z

Zum Abschluß dieses Kapitels soll eine Repräsentation für den Satz in (\mex{1})
angegeben werden, die Information über den semantischen Beitrag des Verbs und die Indices
enthält.
\ea
weil er den Hund schlägt.
\z
(\mex{1})\vpageref{rep-er-den-hund-schlaegt} zeigt die Struktur des Satzes, wobei
die interne Struktur der Phrase \emph{der Hund} aus Platzgründen und aus Gründen der Übersichtlichkeit
weggelassen wurde.
\begin{figure}[htbp]
\ea
\label{rep-er-den-hund-schlaegt}
Repräsentation für \emph{er den Hund schlägt}:\\
\onems{
phon \phonliste{ er den Hund schlägt }\\
cat  \ms{ head  & \ibox{1} \ms[verb]{ vform & fin\\
                           }\\
          comps & \sliste{ }\\
        }\\
cont \ibox{2} \ms[schlagen]{
       agens & \ibox{3}\\
       patiens & \ibox{4}\\
       }\\
context$|$inds \liste{ \ibox{5} \ms{
                       ind & \ibox{3} \ms{ per & 3\\
                                           num & sg\\
                                           gen & mas\\
                                         }\\
                       restr & \sliste{ }\\
                       }, \ibox{6} \ms{
                       ind & \ibox{4} \ms{ per & 3\\
                                           num & sg\\
                                           gen & mas\\
                                         }\\
                       restr & \liste{ \ms[hund]{
                                       inst & \ibox{4}\\
                                       }}\\
                       }}\\
head-dtr \onems{ phon \phonliste{ den Hund schlägt }\\
                 cat  \ms{ head & \ibox{1} \\
                           comps & \sliste{ \ibox{7} }\\
                         }\\
                 cont \ibox{2}\\
                 context$|$inds \liste{ \ibox{6} }\\
                 head-dtr \ms{ phon \phonliste{ schlägt }\\
                               cat  \ms{ head  & \ibox{1} \\
                                         comps & \sliste{ \ibox{7}, \ibox{8} }\\
                                       }\\
                               cont \ibox{2} \\
                               context$|$inds \liste{}\\
                             }\\
                 non-head-dtrs \liste{ \ibox{8} \onems{
                                        phon \phonliste{ den Hund }\\
                                        cat  \ms{ head  & \ms[noun]{ case & acc\\
                                                                   }\\
                                                  comps & \sliste{ }\\
                                                }\\
                                        cont \ibox{6} \\
                                        context$|$inds \liste{ \ibox{6} }\\
                                     }\\ }\\
               }\\
non-head-dtrs \liste{ \ibox{7} \onems{
                phon \phonliste{ er }\\
                cat  \ms{ head  & \ms[noun]{ case & nom\\
                                           }\\
                          comps & \sliste{ }\\
                        }\\
                cont \ibox{5} \\
                context$|$inds \liste{ \ibox{5} }\\
                }\\ }\\
}
\z
\end{figure}
\end{comment}

%\section*{Kontrollfragen}

\questions{
\begin{enumerate}
\item Was versteht man unter Linking?
\item Wie wird in der HPSG eine Verbindung zwischen Form und Bedeutung eines (komplexen)
      sprachlichen Zeichens hergestellt?
\end{enumerate}
}

%\section*{Übungsaufgaben}

\exercises{
\begin{enumerate}
\item Wie kann man den semantischen Beitrag von \emph{lacht} repräsentieren? Geben Sie den
  Lexikoneintrag für \emph{lacht} an und berücksichtigen Sie das Linking.
\item Geben Sie die vollständige Merkmalbeschreibung für die Wortfolge in (\mex{1}), wie sie in
  \emph{dass er lacht} vorkommt,  an: 
      \ea
      er lacht
      \z
      Dabei soll sowohl die syntaktische Struktur als auch der Bedeutungsbeitrag berücksichtigt werden.

\item Laden Sie die zu diesem Kapitel gehörende Grammatik von der Grammix"=CD
(siehe Übung~\ref{uebung-grammix-kapitel4} auf Seite~\pageref{uebung-grammix-kapitel4}).
Im Fenster, in dem die Grammatik geladen wird, erscheint zum Schluß eine Liste von Beispielen.
Geben Sie diese Beispiele nach dem Prompt ein und wiederholen Sie die in diesem Kapitel besprochenen
Aspekte.
\end{enumerate}
}
