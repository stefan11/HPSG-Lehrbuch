%% -*- coding:utf-8 -*-
%%%%%%%%%%%%%%%%%%%%%%%%%%%%%%%%%%%%%%%%%%%%%%%%%%%%%%%%%
%%   $RCSfile: 91-hpsg-udc.tex,v $
%%  $Revision: 1.19 $
%%      $Date: 2008/09/30 09:14:41 $
%%     Author: Stefan Mueller (CL Uni-Bremen)
%%    Purpose: 
%%   Language: LaTeX
%%%%%%%%%%%%%%%%%%%%%%%%%%%%%%%%%%%%%%%%%%%%%%%%%%%%%%%%%

\chapter{Nichtlokale Abhängigkeiten}
\label{chap-nla}

Bisher können wir Verbletzt und Verberstsätze analysieren. In diesem Kapitel werden wir
die Mechanismen kennenlernen, die zu einer adäquaten Behandlung der Vorfeld- und
der Nachfeldbesetzung benötigt werden. Damit werden wir dann in der Lage sein, Verbzweitsätze
zu analysieren.

\section{Verschiedene Arten von Fernabhängigkeiten}

Das\is{Extraktion|(}\is{nichtlokale Abhängigkeit|(} Vorfeld kann mit einer Konstituente (Adjunkt, Subjekt oder Komplement) besetzt sein (\citealp[Kapitel~2.4]{Erdmann1886a};
\citealp[\page69, S.\,77]{Paul1919a}), weshalb das Deutsche zu den Verbzweitsprachen\is{Verbzweitsprache} gezählt wird.\footnote{%
  Die Kategorisierung des Deutschen als Verbzweitsprache scheint im Widerspruch zur Aussage zu stehen, 
  dass das Deutsche eine Verbletztsprache\is{Verbletztsprache} (siehe Seite~\pageref{page-verbletzt}) ist. Dem ist
  jedoch nicht so, da bei der Klassifizierung von Sprachen nach der Zugehörigkeit zu den Verbzweitsprachen
  andere Eigenschaften eine Rolle spielen als bei der Klassifikation nach der Zugehörigkeit zu einem
  der Grundmuster SVO, SOV, VSO, VOS, OSV bzw.\ OVS. Das Deutsche wird zu den SOV"=Sprachen gezählt,
  weil davon ausgegangen wird, dass diese Abfolge die zugrundeliegende Abfolge ist. Das Verb kann jedoch auch nach 
  vorn gestellt werden. Im Gegensatz zum Englischen kann im Deutschen nicht nur
  das Subjekt vor dem finiten Verb stehen, sondern auch Adjunkte und Komplemente, was eine Eigenschaft
  von V2"=Sprachen ist. Das Englische ist eine SVO"=Sprache aber keine V2"=Sprache. Das Dänische ist
  eine SVO"=Sprache aber auch gleichzeitig eine V2"=Sprache \citep{Vikner95a-u,MOeDanish}.%
}
(\mex{1}) zeigt einige Beispiele für mögliche Vorfeldbesetzungen:
\ea
\label{fragen}
\begin{tabular}[t]{@{}l@{~}ll}
a. & Schläft Karl?                                        & Karl schläft.\\
b. & Kauft Karl diese Jacke?                              & Karl kauft diese Jacke.\\
   &                                                      & Diese Jacke kauft Karl.\\
c. & Kauft Karl morgen diese Jacke?                       & Morgen kauft Karl diese Jacke.\\
d. & Wird die Jacke von Karl gekauft?                     & Von Karl wird die Jacke gekauft.\\
e. & Ist Maria schön?                                     & Schön ist Maria.\\
f. & Muss man sich kämmen?                                 & Man muss sich kämmen.\\
   &                                                      & Sich kämmen muss man.\\
g. & Glaubt Karl, dass Maria ihn liebt?                    & Dass Maria ihn liebt, glaubt Karl.\\
h. & Lacht Karl, weil er den Trick kennt?   & Weil er den Trick kennt, lacht Karl.\\
i. & Schlaf jetzt endlich!                                & Jetzt schlaf endlich!\\
\end{tabular}
\z


\noindent
Man könnte versucht sein, die Sätze mit Vorfeldbesetzung als einfache Anordnungsvarianten
der Verberstsätze aufzufassen: Das Verb würde dann mit dem Kopf"=Argument"=Schema mit seinen
Argumenten kombiniert, wobei diese links oder rechts des Verbs angeordnet werden
dürfen. Abbildung~\vref{abb-kathol-flach-vf} zeigt, wie eine solche Analyse für (\mex{1}) bei der
Annahme flacher Strukturen aussehen würde.
\ea
Dem Affen gibt Aicke den Stock.
\z
\begin{figure}
\centerline{%
\begin{forest}
sm edges
[{V[\type{fin}, \comps \eliste]},l sep+=.5\baselineskip
  [\ibox{2} {NP[\type{dat}]} [dem Affen, roof]]
  [{V[\type{fin}, \comps \sliste{ \ibox{1}, \ibox{2}, \ibox{3} } ]} [gibt]]
  [\ibox{1} {NP[\type{nom}]} [Aicke]]
  [\ibox{3} {NP[\type{acc}]} [den Stock, roof]]]
\end{forest}}
\caption{\label{abb-kathol-flach-vf}Analyse von \emph{Dem Affen gibt Aicke den Stock.} mit
  einer flachen Struktur, die das Vorfeld enthält}
\end{figure}
Bei der Annahme binär verzweigender Strukturen bekäme man die Analyse in
Abbildung~\ref{abb-kathol-binaer-vf}.
\begin{figure}
\centerline{%
\begin{forest}
sm edges
[{V[\type{fin}, \comps \eliste]}, s sep+=1.5em 
  [\ibox{2} {NP[\type{dat}]} [dem Affen, roof]]
  [{V[\type{fin}, \comps \sliste{ \ibox{2} } ]}, s sep+=1em 
    [{V[\type{fin}, \comps \sliste{ \ibox{2}, \ibox{3} } ]} 
      [{V[\type{fin}, \comps \sliste{ \ibox{1}, \ibox{2}, \ibox{3} } ]} [gibt]]
      [\ibox{1} {NP[\type{nom}]} [Aicke]]]
    [\ibox{3} {NP[\type{acc}]} [den Stock, roof]]]]
\end{forest}}%
\caption{\label{abb-kathol-binaer-vf}Analyse von \emph{Dem Affen gibt Aicke den Stock.} mit
  binär verzweigenden Strukturen}
\end{figure}

%\noindent
Solche Analysen würden funktionieren, wenn die Argumente nur in unmittelbarer Umgebung des Verbs angeordnet
werden könnten, \dash, wenn es sich bei den Vorfeldbesetzungen um lokale Umstellungen handelte.
%% JB: Das sind die Alternativenkapitel, die man nicht unbedingt lesen muss!
%% In einer Analyse mit flachen Strukturen (Kapitel~\ref{sec-flache-strukturen}), 
%% mit variabler Verzweigungsrichtung oder in einer Linearisierungsgrammatik, wie sie in Kapitel~\ref{sec-Linearisierung} besprochen wurde,
%% wäre ein solches Vorgehen technisch möglich. Bei variabler Verzweigungsrichtung würden einige Argumente des Verbs 
%% rechts vom Verb realisiert und ein Argument links des Verbs.
%% \citet*{NSW94a} haben für Voranstellung von Idiomteilen einen Linearisierungsansatz vorgeschlagen
%% und auch \citet[Kapitel~6.3]{Kathol95a} entwickelt eine Linearisierungsanalyse
%% für einfache Voranstellungen wie die in (\mex{0}). 
Allerdings gibt es Daten, die zeigen, dass es sich bei der Vorfeldbesetzung nicht um eine lokale Umstellung handeln kann.
So gehören in (\mex{1}) die Konstituenten im \vf zu einem tief eingebetteten Kopf:
\eal
\ex\label{bsp-um-zwei-millionen}
{}[Um zwei Millionen Mark]$_i$ soll er versucht haben, [eine Versicherung \_$_i$ zu betrügen].\footnote{
         taz, 04.05.2001, S.\,20.
}
\ex
"`Wer$_i$, glaubt\iw{glauben} er, daß er \_$_i$ ist?"' erregte sich ein Politiker vom Nil.\footnote{
        Spiegel, 8/1999, S.\,18.
}
\ex Wen$_i$ glaubst du, daß ich \_$_i$ gesehen habe.\footnote{
    \citew[\page84]{Scherpenisse86a}.
    }
\ex {}[Gegen ihn]$_i$ falle es den Republikanern hingegen schwerer, [ [ Angriffe \_$_i$] zu lancieren].\footnote{
  taz, 08.02.2008, S.\,9
}
\zl
\emph{um zwei Millionen Mark} gehört zu \emph{betrügen} und nicht zu einem der Verben \emph{soll}, \emph{versucht}
oder \emph{haben}. Da sich die Verbalphrase \emph{eine Versicherung zu betrügen} im Nachfeld befindet, kann
also \emph{um zwei Millionen Mark} nicht aufgrund einer Umstellung vorn realisiert worden sein.

Deshalb werden Verbzweitsätze in den meisten Grammatiken für das Deutsche zu Verberstsätzen in Beziehung gesetzt.
Die Stelle, an der die Vorfeldkonstituente stehen würde, wird meist durch einen `\_' gekennzeichnet. Man nennt
diesen Platzhalter auch Spur\is{Spur}, Lücke oder \emph{Trace} bzw.\ \emph{Gap}.
Vergleiche auch Kapitel~\ref{sec-v1}.
Das zur Lücke gehörige Element im Vorfeld wird auch Füller\is{F"uller} genannt.

Abhängigkeiten, die über Phrasengrenzen hinweggehen, nennt man\NOTE{JB: schwer verständlich, Def-Abgrenzung}
\emph{Fernabhängigkeiten}\is{Fernabhängigkeit} oder \emph{nichtlokale Abhängigkeiten}. Die Anzahl der Phrasengrenzen, die bei der Voranstellung überschritten werden können, ist im Prinzip
unbegrenzt. Auch Satzgrenzen können überschritten werden. Bei zu hoher Komplexität ergeben sich
dann allerdings Verarbeitungsprobleme für den Menschen. 
Die englische Bezeichnung für Fernabhängigkeiten, die mehrere Satzgrenzen überschreiten können,
ist \emph{unbounded dependencies}. Im Unterschied dazu gibt es Fernabhängigkeiten, die
begrenzt sind, aber dennoch nicht lokal. Diese werden auch \emph{long distance dependencies}
genannt. Ein Beispiel für solch eine Fernabhängigkeit ist die Extraposition\is{Extraposition|(}.
In (\mex{1}a) ist ein Relativsatz extraponiert worden, der \emph{Frau} im \mf modifiziert.
In (\mex{1}b) ist ein Infinitivkomplement ins Nachfeld gestellt worden. 
\eal
\label{bsp-extrap}
\ex Der Mann hat [der Frau \_$_i$] den Apfel gegeben, [die er am schönsten fand]$_i$.
\ex Der Mann hat \_$_i$ behauptet, [einer Frau den Apfel gegeben zu haben]$_i$.
\zl
Wie bei den Sätzen
in (\ref{fragen}) könnte man für (\mex{0}b) eine lokale Umstellung annehmen, doch ist eine
Theorie, die mit nur einem Mechanismus zur Erklärung bestimmter Abfolgen auskommt, einer
Theorie, die zwei verschiedene Analysen verwendet, vorzuziehen.

Dass es sich bei der Extraposition wirklich um eine nichtlokale Abhängigkeit handelt,
zeigen die Beispiele in (\mex{1}):
\eal
\ex Karl hat mir [von [der Kopie [einer Fälschung [eines Bildes [einer Frau \_$_i$]]]]] erzählt,
    [die schon lange tot ist]$_i$.
\ex Ich habe [von [dem Versuch [eines Beweises [der Vermutung \_$_i$]]]] gehört,\\
       {}[dass es Zahlen gibt, die die folgenden Bedingungen erfüllen]$_i$.
\zl
In (\mex{0}a) gehört der Relativsatz zu einem Nomen, das in fünf maximale Phrasen eingebettet
ist. Zu solchen und anderen Extrapositionsdaten siehe auch \citew[Kapitel~13.1]{Mueller99a}
und \citew{Mueller2004d,Mueller2007c}. Im Gegensatz zur Vorfeldbesetzung ist die Extraposition jedoch keine
\emph{unbounded dependency}, denn die Verschiebung von Material nach rechts ist durch die Satzgrenze
beschränkt:
% Ist das die richtige Referenz?? \citep[\page 166]{Ross67}:
\eal
\ex[]{
Karl hat, dass [\sub{S} John \_$_i$ erzählt hat, [dass Maria schläft,]$_i$ ] nicht wirklich behauptet.
}
\ex[*]{
Karl hat, dass [\sub{S} John \_$_i$ erzählt hat] nicht wirklich behauptet, [dass Maria schläft]$_i$.
}
\ex[]{
Der Mann, [\sub{S} der \_$_i$ behauptet hat, [dass Maria nicht kommt]$_i$ ], steht da drüben.
}
\ex[*]{
Der Mann, [\sub{S} der \_$_i$ behauptet hat], steht da drüben, [dass Maria nicht kommt]$_i$.
}
\zl
Extrapositionsanalysen finden sich in \citealp{Keller95b}, \citealp{Bouma96} und
\citealp[Kapitel~13.2]{Mueller99a}. Spezielle Analysen für die Relativsatzextraposition findet man in
\citew{Kiss2005a} und \citew{Crysmann2004a}. In diesem Buch werde ich nicht weiter auf Extraposition
eingehen.\is{Extraposition|)}

Nichtlokale\is{Relativsatz|(} Abhängigkeiten sind auch innerhalb von Relativsätzen zu beobachten: Die Phrase, die
das Relativpronomen enthält, kann zu tiefer eingebetteten Konstituenten gehören, wie die
Beispiele in (\mex{1}) und (\mex{2}) zeigen:
\eal
\label{bsp-nla-rs}
\ex\iw{bitten}
das Thema, [über das]$_i$ er Peter gebeten hat, [\sub{VP} [einen Vortrag\iw{Vortrag} \_$_i$] zu halten],
\ex\iw{versuchen} 
Das Tor, [von dem]$_i$ Stein versuchte, [das Schild mit der Aufschrift \frqq Zu verkaufen\flqq{} \_$_i$ zu entfernen], sank mit einem
Klagelaut um.\footnote{
  Judith Hermann, \emph{Sommerhaus, später}, Frankfurt: S.\,Fischer Verlags GmbH, 3.\, Auf"|lage, 2001, S.\,148.%
}
\zl
\ea
Wollen wir mal da hingehen, wo$_i$\iw{wo!Relativpronomen} Jochen gesagt hat, [dass es \_$_i$ so gut schmeckt]?
\z\is{Relativsatz|)}
Relativsätze werden im Kapitel~\ref{chap-rs} behandelt. Interrogativnebensätze\is{Interrogativsatz} sind syntaktisch ähnlich aufgebaut.
Sie werden in diesem Buch nicht behandelt. Im folgenden wenden wir uns der Vorfeldbesetzung zu.


\section{Vorfeldbesetzung}

Im letzten Kapitel haben wir bereits Spuren kennengelernt. Die Abhängigkeit zwischen vorangestelltem
Verb und Verbspur war aber lokaler Natur, was dadurch erfaßt wurde, dass die entsprechende Information
als Teil der Kopf"|information nach oben gereicht wurde. Wie wir aber im vorigen Abschnitt gesehen haben,
können Fernabhängigkeiten mehrere Maximalprojektionen überschreiten. Eine Projektion von Merkmalen
bis zur phrasalen Ebene ist also nicht ausreichend. Deshalb wird die Datenstruktur noch einmal geändert,
und es wird zusätzlich zu der unter \local repräsentierten lokal relevanten Information noch ein
Merkmal für Information über Fernabhängigkeiten eingeführt. Dieses Merkmal heißt \nonloc.
Die Datenstruktur in (\ref{geom-loc}) auf Seite~\pageref{geom-loc} wird also zu (\mex{1}) erweitert:
\ea
\label{geom-nonloc}
\ms{
phon & list~of~phoneme strings\\
loc  & \ms{ cat  & \ms{ head   & head \\
                        subcat & list of signs\\
                      } \\
            cont & cont\\
          }\\
nonloc & nonloc\\
}
\z
Der \nonlocw ist selbst noch weiter strukturiert:
\ea
\ms[nonloc]{
 que & \type{list~of~npros} \\
 rel & \type{list~of~indices} \\
 slash & \type{list~of~local~structures} \\ %\\
 %extra & \ms[list~of~local~structures]{} \\
}
\z
Dabei ist \textsc{que} für die Analyse von Interrogativsätzen wichtig. Da diese in diesem Buch nicht behandelt
werden, wird es im folgenden weggelassen. \textsc{rel} ist eine Liste referentieller Indizes von Relativpronomina. Auf
\textsc{rel} wird im Kapitel~\ref{chap-rs} über Relativsätze noch genauer eingegangen.
\textsc{slash} ist eine Liste von \type{local}"=Objekten. Diese Liste wird zur Analyse
der Vorfeldbesetzung benötigt. Für die Analyse von Relativsätzen und Interrogativnebensätzen
ist sie ebenfalls von Bedeutung.


Im vorigen Kapitel haben wir die Verberststellung für Sätze wie (\mex{1}a) analysiert, nun wenden wir uns
dem Beispiel (\mex{1}b) zu:
\eal
\ex Kennt$_i$ das Buch jeder \_$_i$?
\ex {}[Das Buch]$_j$ kennt$_i$ \_$_j$ jeder \_$_i$?
\zl
Die Analyse der Vorfeldbesetzung kann nur unter Einbeziehung der Analyse der Verberststellung erklärt
werden. Entsprechende Darstellungen würden aber sehr komplex werden. Ich erkläre deshalb
die Mechanismen im folgenden an einer Struktur für die Verberststellung, die man in einem
Ansatz %mit variabler Verzweigungsrichtung %bzw.\ mit Linearisierungsdomänen 
bekommen würde, bei dem das Verb in Erststellung wie auch in Abbildung~\ref{abb-kathol-binaer-vf}
direkt mit seinen Argumenten kombiniert wird. \Dh statt der korrekten Analyse in (\mex{0}b) erkläre
ich zuerst die Analyse in (\mex{1}):
\ea
{}[Das Buch]$_j$ [[kennt \_$_j$] jeder].\label{ex-das-buch-kennt}
\z 
Die komplette Analyse mit korrekter Modellierung der Verberststellung zeigt
die Abbildung~\vref{abb-das-buch-kennt-jeder}.\NOTE{FB: Lieber Englisch oder Kapitel 9/10 in anderer Reihenfolge.}


Wie\is{Spur!Extraktionsspur|(} bei der Analyse der Verbbewegung geht man davon aus, dass es an der Stelle, an der
das Objekt normalerweise stehen würde, eine Spur gibt, die die Eigenschaften des Objekts
hat. Das Verb kann also seine Valenzanforderungen lokal befriedigen. Die Information darüber,
dass eine Kombination mit einer Spur und nicht mit einem echten Argument erfolgt ist,
wird innerhalb des entstehenden komplexen Zeichens repräsentiert und im Baum nach oben gereicht.
Die Fernabhängigkeit kann dann weiter oben im Baum durch ein Element im Vorfeld abgebunden werden.

Eingeführt wird die Fernabhängigkeit durch die Spur, die eine
Merkmalstruktur, die dem \localw des geforderten Arguments entspricht, in ihrer
\slashl hat. (\mex{1}) zeigt eine Beschreibung der Spur, wie sie für die Analyse von (\mex{0}) benötigt wird:

\eas
\label{le-spur-acc-o-kennen}
Spur für das Akkusativkomplement von \emph{kennen} (vorläufig):\\
\ms[word]{
 phon & \phonliste{} \\
loc   & \ibox{1} \ms{ cat \ms{ head & \ms[noun]{
                                                     cas & acc\\
                                                     } \\
                                              subcat & \liste{}\\
                                            } \\
                                   }\\
nonloc & \ms{ slash & \sliste{ \ibox{1} } \\
                                                %extra & \liste{} \\
            } \\
}
\zs

\noindent
Da Spuren keine interne Struktur, \dash keine Töchter haben, sind sie vom Typ \type{word}.
Die Spur hat die Eigenschaften eines Akkusativobjekts. Dass das Akkusativobjekt
an der Stelle der Spur fehlt, ist durch den entsprechenden Wert in \slasch ausgedrückt.\is{Spur!Extraktionsspur|)} 
%
Abbildung~\vref{abb-das-buch-kennt-er-variable-verzweigung} zeigt das Weiterreichen
der Information.
\begin{figure}
\centerline{%
\begin{forest}
sm edges
[V\feattab{\subcat \eliste,\\
           \slasch \eliste}, s sep+=3ex 
  [{NP\ibox{1}[\type{acc}]} [das Buch,roof]]
  [V\feattab{
      \subcat \sliste{ },\\
      \slasch \sliste{ \ibox{1} }}, s sep+=1ex
    [V\feattab{
      \subcat \sliste{ \ibox{2} },\\
      \slasch \sliste{ \ibox{1} }}
      [V\feattab{
        \subcat \sliste{ \ibox{2}, \ibox{3} },\\
        \slasch \sliste{ }} [kennt]]
      [\ibox{3} NP\feattab{\textsc{loc} \ibox{1} \type{acc},\\
                           \slasch \sliste{ \ibox{1} }}
        [\trace]]]
    [{\ibox{2} NP[\type{nom}]}
      [jeder]]]]
\end{forest}}
\caption{\label{abb-das-buch-kennt-er-variable-verzweigung}Perkolation von nichtlokaler Information in einer Struktur mit variabler Verzweigung}
\end{figure}
Für das Weiterleiten der \textsc{nonloc}"=Information ist das folgende Prinzip der nichtlokalen Merkmale verantwortlich.

\begin{samepage}
\is{Prinzip!nonloc-@\textsc{nonloc}-}
\begin{prinzip-break}[Prinzip der nichtlokalen Merkmale]
Der Wert des \textsc{nonloc}"=Merkmals eines phrasalen Zeichens ist die Vereinigung der
\textsc{non\-loc}"=Werte der Töchter des Zeichens abzüglich der abgebundenen Elemente.
\end{prinzip-break}
\end{samepage}

\noindent
Wie auch bei den in Kapitel~\ref{sec-prinzipien} besprochenen Prinzipien gibt es eine
Formalisierung dieses Prinzips, der wir uns dann nach der Diskussion des Kopf"=Füller"=Schemas
zuwenden.

Das Kopf"=Füller"=Schema lizenziert den obersten Knoten in Abbildung~\ref{abb-das-buch-kennt-er-variable-verzweigung}:
\begin{schema}[Kopf"=Füller"=Schema]
\label{hf-schemaa}\is{Schema!Kopf"=Füller"=}
~\\\samepage
\type{head-filler-phrase}\istype{head"=filler"=phrase} \impl\\
\ms{ 
nonloc$|$slash &   \eliste\\
head-dtr      & \onems{ loc$|$cat \onems{ head \ms[verb]{vform & fin\\
                                                                 initial & \textrm{+}\\
                                                                }\\
                                                  subcat \liste{}\\
                                               }\\
                             nonloc$|$slash   \sliste{ \ibox{1} }\\
                        }\\
non-head-dtrs & \liste{ \onems{ loc \ibox{1}\\
                                           nonloc$|$slash \liste{} \\
                                 }}\\
   }
\end{schema}
Dieses Schema kombiniert einen finiten Satz mit Verb in Verberststellung (\textsc{initial}+) und einem Element in \textsc{slash}
mit einer Nicht-Kopf"|tochter, deren \textsc{local}"=Wert identisch mit diesem \textsc{slash}"=Element ist.
In der Struktur werden keine Argumente gesättigt. \type{head-filler-phrase} ist Untertyp von \type{head-non-argument-phrase}
(siehe Abbildung~\vref{abb-sign-hf-struc}).
\begin{figure}
\oneline{
% Das tut so, als wäre head-argument-phrase eine Tochter von head-non-argument-phrase, zeichnet aber
% keine Kante. Die Kanten werden dann alle mit draw per Hand gezeichnet.
\begin{forest}
type hierarchy,
 % for tree={
 %   calign=fixed angles,
 %   calign angle=50
 % } 
[phrase
    [non-headed-phrase]
    [headed-phrase, l sep*=2, for children={l sep*=4}
      [head-non-adjunct-phrase, name=non-adjunct, 
        [head-adjunct-phrase,no edge, name=adjunct]]
      [head-non-argument-phrase, name=non-arg
        [head-argument-phrase,no edge, name=argument]]
      [head-non-filler-phrase, name= non-filler
        [head-filler-phrase, no edge, name=filler]]
      [head-non-specifier-phrase, name=non-specifier
        [head-specifier-phrase, no edge, name=specifier]]]]
\draw (non-adjunct.south) to (argument.north);
\draw (non-adjunct.south) to (filler.north);
\draw (non-adjunct.south) to (specifier.north);
\draw (non-arg.south) to (adjunct.north);
\draw (non-arg.south) to (filler.north);
\draw (non-arg.south) to (specifier.north);
\draw (non-filler.south) to (adjunct.north);
\draw (non-filler.south) to (argument.north);
\draw (non-filler.south) to (specifier.north);
\draw (non-specifier.south) to (adjunct.north);
\draw (non-specifier.south) to (argument.north);
\draw (non-specifier.south) to (filler.north);
\end{forest}}
\caption{\label{abb-sign-hf-struc}Typhierarchie für \type{phrase}}
\end{figure}
Dadurch wird der \subcatw der Kopf"|tochter automatisch mit dem der Mutter identifiziert (siehe Abschnitt~\ref{sec-valp} zum Valenzprinzip\is{Prinzip!Valenz-}).
Der semantische Beitrag kommt vom Verb (der Kopf"|tochter). Das folgt aus dem Semantikprinzip, denn der
Typ \type{head-filler-phrase} ist ein Untertyp von \type{head-non-adjunct-phrase}.

Die Abbildung~\ref{abb-sign-hf-struc} enthält auch einen Typ \type{head"=non"=filler"=phrase}. Die Beschränkung
für diesen Typ sieht wie folgt aus:
\eas
\type{head"=non"=filler"=phrase}\istype{head"=non"=filler"=phrase} \impl\\
\onems{ 
nonloc$|$slash \ibox{1} $\oplus$ \ibox{2}\\
head"=dtr$|$nonloc$|$slash \ibox{1} \\
non-head"=dtrs  \sliste{ [ nonloc$|$slash \ibox{2} ] }\\
}
\zs
Diese Implikation sorgt dafür, dass in allen bisher eingeführten Strukturen außer der eben besprochenen
die \slashwe der Töchter zum \slashw der Mutter verknüpft werden.
Zusammen mit Teilen der Beschränkungen für den Typ \type{head-filler-phrase}, die in (\mex{1})
wiederholt sind, stellt (\mex{0}) die Formalisierung des Prinzips der nichtlokalen Merkmale dar.
\eas
\type{head-filler-phrase}\istype{head"=filler"=phrase} \impl\\
\onems{ 
nonloc$|$slash            \eliste\\
head-dtr$|$nonloc$|$slash \sliste{ \ibox{1} }\\
non-head-dtrs             \sliste{ [ loc \ibox{1} ] }\\
   }
\zs

\noindent
In\is{Spur!Extraktionsspur|(} (\ref{le-spur-acc-o-kennen}) wurde eine Spur für das Akkusativobjekt von \emph{kennen} angegeben.
Genau wie bei der Analyse der Verbbewegung ist es jedoch nicht nötig, verschiedene Spuren mit unterschiedlichen
Eigenschaften im Lexikon zu haben. Ein allgemeiner Eintrag wie der in (\mex{1}) ist ausreichend:

\eas
\label{le-extraktionsspur}
Extraktionsspur: \\
\onems[word]{
 phon   \phonliste{} \\
 loc    \ibox{1} \\
 nonloc$|$slash  \sliste{ \ibox{1} } \\ 
%\ms{ %que   & \liste{} \\
                 %                               rel & \liste{} \\
%                                                slash & \sliste{ \ibox{1} } \\ 
                                                %extra & \liste{} \\
%                                              } \\ 
}
\zs
Das liegt daran, dass der Kopf die \textsc{local}"=Eigenschaften seiner Argumente und damit auch die
\textsc{local}"=Eigenschaften von Spuren, die mit ihm kombiniert werden, ausreichend bestimmt. Durch
die Identifikation des Objekts in der \subcatl mit der Spur und durch die Identifikation der
Information in \textsc{slash} mit der Information über das Element im Vorfeld wird sichergestellt,
dass nur Elemente im Vorfeld realisiert werden, die zu den Anforderungen in der \subcatl
passen. Genauso funktioniert die Analyse von vorangestellten Adjunkten: Dadurch dass der
\localw der Konstituente im Vorfeld über die Vermittlung durch \textsc{slash} mit dem \localw der Spur
identifiziert wird, ist ausreichend Information über die Art der Spur vorhanden.\is{Spur!Extraktionsspur|)}


Im folgenden zeige ich, wie die Extraktionsanalyse
mit der Verberstanalyse aus Kapitel~\ref{sec-v1} kombiniert werden kann.
Abbildung~\vref{abb-das-buch-kennt-jeder} zeigt die Analyse für (\ref{ex-das-buch-kennt}),
hier als (\mex{1}) mit Markierungen für die Verbspur wiederholt:
\ea
{}[Das Buch]$_i$ kennt$_j$  \_$_i$ jeder \_$_j$.
\z
\begin{figure}
% %\centerline{\mbox{\includegraphics[width=0.9\textwidth]{netter-verb-movement}}}
% %\centerline{%
% \resizebox{\textwidth}{!}{%
% %\nodemargin5pt
% \begin{tabular}[t]{@{}c@{\hspace{4mm}}c@{\hspace{4mm}}c@{\hspace{4mm}}c@{\hspace{4mm}}c@{}}%llll}
% \multicolumn{2}{@{}c}{\rnode{3all}{V[\begin{tabular}[t]{@{}l}
%                                  \textsc{subcat} \rnode{3}{\eliste},\\
%                                  \textsc{slash} \eliste]\\
%                                  \end{tabular}}
%                                           }\\
% \\*[4ex]
% \rnode{npbuch}{NP[\textsc{loc} \ibox{1} \type{acc}]} & \multicolumn{2}{c}{\rnode{1}{V[\begin{tabular}[t]{@{}l}
%                                           \textsc{subcat} \sliste{},\\
%                                           \textsc{slash} \sliste{ \ibox{1} }]\\
%                                           \end{tabular}}
%                                           }\\
% \\*[4ex]
% &\rnode{2}{V[\textsc{subcat} \sliste{ \ibox{2} }]} & \multicolumn{3}{l}{\rnode{3}{\ibox{2} V[\begin{tabular}[t]{@{}l}
%                                                                     \textsc{subcat} \sliste{ },
%                                            \textsc{slash} \sliste{ \ibox{1} }]
%                                                                     \end{tabular}}
%                                                        }\\
% \\
% & \hspace{8ex}V1-LR\\*[3ex]
% & \rnode{kennt-vl}{V[\textsc{subcat} \sliste{ \ibox{3}, \ibox{4} }]}  & \rnode{4}{\ibox{4} [\begin{tabular}[t]{@{}l@{}}
%                                                                         \textsc{loc} \ibox{1},\\
%                                                                         \textsc{slash} \sliste{ \ibox{1} }] \\
%                                                                         \end{tabular}}& \multicolumn{2}{l}{\rnode{5}{V[\begin{tabular}[t]{@{}l}
%                                                                                                        \textsc{subcat} \sliste{ \ibox{4} }]\\
%                                                                                                        \end{tabular}}
%                                                                                            }\\
% \\*[4ex]
% &                                   &                                   & \rnode{6}{\ibox{3} NP[\type{nom}]} & \rnode{7}{V[\begin{tabular}[t]{@{}l@{}}
%                                                                                                  \textsc{subcat} \sliste{ \ibox{3}, \ibox{4} }] \\
%                                                                                                  \end{tabular}}\\
% \\*[3ex]
% d\rnode{buch}{as Buc}h & \rnode{8}{kennt}                     & \rnode{9}{$-$}                  & \rnode{10}{jeder} & \rnode{11}{$-$}\\
% \end{tabular}
% \ncline{3all}{npbuch}\ncline{3all}{1}%
% \ncline{1}{2}\ncline{1}{3}%
% \ncline{3}{4}\ncline{3}{5}%
% \ncline{5}{6}\ncline{5}{7}%
% \ncline{2}{kennt-vl}\ncline{kennt-vl}{8}%
% \ncline{4}{9}%
% \ncline{6}{10}%
% \ncline{7}{11}%
% \ncline{npbuch}{buch}%
% }
\oneline{%
\begin{forest}
sm edges,
for level={3}{l+=\baselineskip}
[V\feattab{\subcat \eliste,\\
           \slasch \eliste}, s sep+=1em
  [{NP\ibox{1}[\type{acc}]}
     [das Buch, roof]]
  [V\feattab{\subcat \eliste,\\
             \slasch  \sliste{ \ibox{1} }}
     [{V[\subcat \sliste{ \ibox{2} }]}
       [{V[\subcat \sliste{ \ibox{3}, \ibox{4} }]},edge label={node[midway,right]{V1-LR}}
         [kennt]]]
     [\ibox{2} V\feattab{\subcat \eliste, \slasch \sliste{ \ibox{1} }}
         [\ibox{4} \feattab{\textsc{loc} \ibox{1},\\
                            \slasch \sliste{ \ibox{1} } }
         [\trace]]
       [V\feattab{\subcat \sliste{ \ibox{4} }}
         [{\ibox{3} NP[\type{nom}]}
           [jeder]]
         [V\feattab{\subcat \sliste{ \ibox{3}, \ibox{4} }}
           [\trace]]]]]]
\end{forest}}
\caption{\label{abb-das-buch-kennt-jeder}Analyse für:\ \emph{Das Buch kennt jeder.} kombiniert mit der Verbbewegungsanalyse für die Verberststellung}
\end{figure}
Die Verbbewegungsspur für \emph{kennt} wird mit einer Nominativ"=NP und einer Extraktionsspur kombiniert. 
Die Extraktionsspur steht im Beispiel für das Akkusativobjekt. Das Akkusativobjekt ist 
in der \subcatl des Verbs beschrieben \iboxb{4}. Über den Verbbewegungsmechanismus
gelangt die Valenzinformation, die im ursprünglichen Eintrag für \emph{kennt} enthalten ist
(\sliste{ \ibox{3}, \ibox{4} }), zur Verbspur. Die Kombination der Projektion der Verbspur mit der
Extraktionsspur verläuft genau so, wie wir es bisher gesehen haben. Der \slashw der Extraktionsspur
wird im Baum nach oben gereicht und letztendlich durch das Kopf"=Füller"=Schema abgebunden.


Die wesentlichen Punkte der Analyse kann man wie folgt zusammenfassen:
\begin{itemize}
\item Nichtlokale Information wird nach oben weitergereicht.
\item Das erfolgt über Strukturteilung.
\item Die Information über nichtlokale Abhängigkeiten ist gleichzeitig an den entsprechenden Knoten präsent.
\end{itemize}
Diese Analyse kann Sprachen erklären, in denen bestimmte Elemente in Abhängigkeit davon flektieren,
ob sie Bestandteil einer Konstituente sind, durch die eine Fernabhängigkeit hindurchgeht. \citet*{BMS2001a} nennen 
Irisch\il{Irisch}, Chamorro\il{Chamorro}, Palauan\il{Palauan}, Isländisch\il{Isländisch},
Kikuyu\il{Kikuyu}, Ewe\il{Ewe}, Thompson Salish\il{Thompson Salish}, Moore\il{Moore}, 
Französisch\il{Französisch}, Spanisch\il{Spanisch} und Jiddisch\il{Jiddisch} als Beispiele für
solche Sprachen und geben entsprechende Literaturverweise. Da in HPSG"=Analysen die Information Schritt für Schritt
weitergegeben wird, können alle Knoten in der Mitte einer Fernabhängigkeit auf die Elemente, die in
Fernabhängigkeiten eine Rolle spielen, zugreifen.



\section{Anhang 1: Der semantische Beitrag von V1- bzw. V2-Sätzen}
\label{sec-semantik-v1-v2}

%%%  24.08.2011 Was ist mit positionalem `es'?
%%%  Das muesste dann ja auch durch SLASH gehen.
%
% Satztyp --> Satzart

Neben\is{Satztyp|(} der Verbsemantik spielt die Verbstellung eine entscheidende Rolle bei der
Bestimmung der Bedeutung eines Satzes. So bedeuten die Sätze in (\mex{1}) verschiedenes:
\eal
\ex Kommt Peter heute? 
\ex Peter kommt heute.
\zl
Der erste Satz ist eine Frage, mit der der Wahrheitsgehalt der Aussage \emph{Peter kommt heute}
erfragt wird. Der zweite Satz sagt aus, dass es eine Tatsache ist, dass Peter heute kommt. Die beiden
Sätze unterscheiden sich lediglich durch ihre Verbstellung voneinander. Der Satz mit Verberststellung
ist ein Entscheidungsfragesatz\is{Fragesatz} und der mit Verbzweitstellung ist ein
Aussagesatz\is{Aussagesatz}. Man muss jedoch vorsichtig sein, denn der Satztyp läßt sich
nicht ohne weiteres an der Position des Verbs festmachen. So gibt es zum Beispiel auch Fragen wie
(\mex{1}), in denen ein Fragepronomen vor dem Finitum steht.
\ea
Wer kommt heute?
\z
Und es gibt elliptische Äußerungen, in denen keine Konstituente vor dem finiten Verb zu sehen ist:
\ea
A: Und was ist mit Peter?\\
B: Kommt heute.
\z
Im Antwortsatz in (\mex{0}) wurde das Subjekt weggelassen. Es würde normalerweise vor dem Finitum im Vorfeld
stehen, weshalb man auch von Vorfeldellipse\is{Vorfeldellipse} spricht.
Bei Verben mit optionalen Argumenten kann man mitunter sogar den Aussagesatz mit Vorfeldellipse
nicht von einem Fragesatz unterscheiden, wenn man einfach nur die Abfolge der Wörter betrachtet:
\eal
\label{bsp-hat-er-gegessen}
\ex Hat er schon gegessen?
\ex Und was ist mit dem Kuchen?\\
    Hat er schon gegessen.
\zl

\noindent
Außerdem ist zu berücksichtigen, dass es auch Imperativsätze mit den beiden Verbstellungen gibt:
\eal
\ex Gib du mir jetzt das Buch!
\ex Jetzt gib du mir das Buch!
\zl

\noindent
Die Verberststellung kann auch in Konditionalsätzen vorkommen:
\eal
\ex Gibst du mir das Buch, helfe ich dir.
\ex Hätte er Karl das Buch rechtzeitig gegeben, hätte Karl ihm helfen können.
\zl
Das sind längst nicht alle Verwendungsweisen für Sätze mit Verberst"= bzw.\ Verbzweitstellung,
für einen detaillierteren Überblick siehe \citew[Kapitel~4]{Zifonun97a}.

Die Bedeutungskomponente, die durch die Verbstellung in den Satz kommt, kann man zu einem
großen Teil bereits im Lexikon bestimmen, und zwar in der Lexikonregel, die für die Lizenzierung
des Verbs in Erststellung verantwortlich ist \citep[\page 205]{Kiss95b}. Das mag sich merkwürdig anhören, denn es scheint
so zu sein, dass das Verb nicht wissen kann, ob es in einem Verberst"= oder Verbzweitsatz verwendet
wird. Dem ist aber nicht so, denn das Verb in Erststellung verlangt eine Projektion der Verbspur
als Argument, und diese Projektion hat entweder ein Element in \slasch oder nicht. Wenn ein Element
in \slasch enthalten ist, kann die Kombination aus Verb und Verbspurprojektion nur zu einem Verbzweitsatz
führen, da Elemente in \slasch abgebunden werden müssen. Ist nichts in \slasch enthalten, kann auch nichts
vor das Finitum gestellt werden, und das Verb kann also nur in einem Verberstsatz verwendet werden.

Die Verberstlexikonregel von Seite~\pageref{lr-verb-movement2} muss entsprechend modifiziert werden.
Die neue Version in (\mex{1}) bettet den semantischen Beitrag der Projektion der Verbspur \iboxb{2}
unter eine zusätzliche Relation ein.
\eas
\label{lr-verb-movement3}
Lexikonregel\is{Lexikonregel!Verberststellung} für Verb in Erststellung:\\
\begin{tabular}[t]{@{}l@{}}
\ms{
loc & \ibox{1} \ms{ cat$|$head & \ms[verb]{ vform & fin\\
                                             initial & $-$\\
                                          }\\
                  }\\
} $\mapsto$\\*
\onems{
loc \onems{ cat  \ms{ head & \ms[verb]{ vform & fin\\
                                        initial & $+$\\
                                        dsl     & none\\
                                      }\\
                           subcat & \sliste{ \onems{ loc \onems{ cat \onems{ head \ms[verb]{
                                                                             dsl & \ibox{1}\\
                                                                             }\\
                                                                      subcat \eliste\\
                                                                    }\\
                                                           cont \ibox{2}\\
                                                         }\\
                                              }}\\
                         }\\
                   cont \ms{
                        soa-rel & \ibox{2}\\
                        }\\
             }\\
}
\end{tabular}
\zs
Der genaue Typ der Relation ist jedoch unterspezifiziert und wird in Abhängigkeit vom
\slashw der Projektion der Verbspur festgelegt. Das wird durch die folgenden Implikationen geregelt:

\eas
Verberstsätze:\istype{verb-initial-lr}\\
\onems[verb-initial-lr]{
loc$|$cat$|$subcat \sliste{ [ nonloc$|$slash \eliste ] }\\
} \impl\\
\ms{
loc$|$cont \type{conditional\_or\_imperative\_or\_interrogative}\\
}
\zs

\eas
\label{impl-v2}%
Verbzweitsätze:\istype{verb-initial-lr}\\
\onems[verb-initial-lr]{
loc$|$cat$|$subcat \sliste{ [ nonloc$|$slash \sliste{ [ ] } ] }\\
} \impl\\
\ms{
loc$|$cont \type{assertion\_or\_imperative\_or\_interrogative}\\
}
\zs
(\mex{-1}) und (\mex{0}) beziehen sich auf den Typ der Lexikonregel (\type{verb-initial-lr}).
Dieser ist in (\ref{lr-verb-movement3}) nicht sichtbar, aber in Kapitel~\ref{sec-lr}
auf Seite~\pageref{pageref-lr-mit-dtr} wurde erklärt, dass die Notation in (\ref{lr-verb-movement3})
als Variante einer Lexikonregel gesehen werden kann, die durch eine typisierte Merkmalstruktur
modelliert wird. \type{verb-initial-lr} ist dann der Typ der entsprechenden Struktur.
Die Implikationen in (\mex{-1}) und (\mex{0}) unterscheiden sich im Hinblick auf den \slashw
des Elements in der \subcatl. Ist dieser die leere Liste, so kann das Verb nur in der ersten
Position realisiert werden.\footnote{
  Das positionale \emph{es}\is{positionales \emph{es}} in Sätzen wie (i.a) kann entweder als Funktor analysiert werden, der einen
  Fragesatz als Argument nimmt, die Fragesemantik unterdrückt und eine entsprechende
  Aussagesatzsemantik beisteuert (Tibor Kiss\aimention{Tibor Kiss}, p.\,M.\ 2006), oder aber als
  extrahiertes Adjunkt, das selbst nichts zur Semantik des von ihm modifizierten Kopfes
  beiträgt. Bei der zweiten Analyse muss man sicherstellen, dass dieses Adjunkt wirklich extrahiert
  ist, \dash, dass das \emph{es} im Mittelfeld ausgeschlossen ist.
\eal
\ex[]{
Es kamen drei Männer herein.
}
\ex[*]{
dass es drei Männer hereinkamen
}
\zllast
}
Der semantische Beitrag muss also vom Typ \type{conditional}, \type{imperative}
oder \type{interrogative} sein. Der Typ \type{conditional\_or\_imperative\_or\_interrogative} ist
ein Obertyp dieser drei Typen, und man muss weiter Information wie zum Beispiel Information über
die Flexion\is{Flexion} des Verbs (Imperativ oder nicht, Verknüpfung mit anderen Sätzen, Intonation\is{Intonation}) heranziehen,
um den genauen Typ der Relation zu bestimmen.

Enthält die \slashl ein Element, so muss das Verb an zweiter Stelle realisiert werden.
Es kann sich dann nur um eine Aussage, einen Imperativ oder eine Frage handeln. Auch in diesem
Fall braucht man für die Bestimmung der genauen Relation weitere Information. Flexionsinformation
und Information über die Realisierung von Argumenten kann Evidenz für das Vorliegen eines
Imperativs sein. Ist das vorangestellte Element ein Fragepronomen, so liegt eine Frage vor,
ansonsten kann es sich um eine Frage oder einen Aussagesatz handeln. Der Bedeutungsbeitrag in (\mex{1})
läßt sich nur unter Bezug auf die Intonation bestimmen:
\eal
\ex Peter kommt morgen?
\ex Peter kommt morgen.
\zl

\noindent
Der aufmerksame Leser wird sich fragen, wie man den korrekten Bedeutungsbeitrag für die Sätze in
(\ref{bsp-hat-er-gegessen}) bekommt. Die Analyse des Fragesatzes ist einfach: Das optionale Argument
von \emph{essen} wird nicht realisiert. In (\ref{bsp-hat-er-gegessen}a) liegt ein ganz normaler Verberstsatz
vor. Da es sich nicht um einen Konditionalsatz handelt, da keine Imperativmorphologie vorliegt
und da der Satz mit Fragesatzintonation gesprochen wird, handelt es sich um einen Fragesatz,
\dash, nur der Typ \type{interrogative} kommt als semantischer Beitrag in Frage. In der Analyse von
(\ref{bsp-hat-er-gegessen}b) wird der Lexikoneintrag mit dem Akkusativobjekt verwendet. Das Akkusativobjekt
wird durch eine Spur abgebunden und befindet sich dann in der \slashl des Arguments des Verbs
in Erststellung. Deshalb sind die Bedingungen der Beschränkung in \pref{impl-v2} erfüllt, und
der semantische Beitrag muss demzufolge \type{assertion\_or\_imperative\_or\_in\-ter\-rog\-a\-tive} sein.
Da weder Imperativmorphologie noch Frageintonation vorliegt, ist nur die Relation \type{assertion}
angebracht. Das Element in \slasch wird dann entweder durch ein spezielles Dominanzschema für
die Vorfeldellipse abgebunden \citep{Mueller2004e} oder ein leeres Element im Vorfeld fungiert als Füller in einer 
Füller"=Kopf"=Struktur.
\is{Satztyp|)} 



\section{Anhang 2: Interaktion mit der Informationsstruktur}
\label{sec-udc-is}

Eine\is{Informationsstruktur|(} Eigenschaft der Analyse wurde bisher noch nicht erklärt: Bei der Voranstellung nimmt
die Extraktionsspur immer die höchste Position im Mittelfeld ein. Das heißt bei der Analyse
von (\mex{1}a) geht man von der Struktur in (\mex{1}b) und nicht von der Struktur in (\mex{1}c) aus:
\eal
\ex Das Buch kennt jeder.
\ex {}[Das Buch]$_i$ kennt$_j$ \_$_i$ jeder \_$_j$.
\ex {}[Das Buch]$_i$ kennt$_j$ jeder \_$_i$ \_$_j$.
\zl
Das mag überraschen, denn die Normalstellung\is{Normalabfolge} der Argumente von \emph{kennen} ist ja Nominativ vor Akkusativ.
Prinzipiell läßt die Analyse bisher beide Abfolgen zu: Da wir sowohl (\mex{1}a) als auch (\mex{1}b)
analysieren können (siehe Kapitel~\ref{sec-mf}), wären ohne eine weitere Einschränkung
auch beide Strukturen in (\mex{0}) möglich.%% \footnote{
%%   \citet{Nerbonne94a} benutzt die zwei Analysemöglichkeiten in 
\eal
\ex Kennt jeder das Buch?
\ex Kennt das Buch jeder?
\zl
\citet{Fanselow2003d} und \citet{Frey2004a} argumentieren jedoch dafür, dass der Satz in (\mex{-1}a) nach dem
Muster von (\mex{-1}b) analysiert werden sollte. Frey hat festgestellt, dass die pragmatischen\is{Pragmatik}
Bedingungen für die Voranstellung von Elementen aus einem einfachen Satz ins Vorfeld den Bedingungen der Voranstellung
im Mittelfeld entsprechen. Insbesondere kann man feststellen, dass die Plazierung von Subjekten (\mex{1}a), 
bestimmten Objekten (\mex{1}b--c) und bestimmten Adverbialen (\mex{1}d--e) im Vorfeld pragmatisch
unmarkiert 
% Seitenzahlen. Bei GMueller2004 steht das verteilt.
ist (\citealp{Lenerz77}, \citealp[\page 73--74]{Haider84c}, \citealp{Fanselow2003d};
G.\,\citealp[\page 189]{GMueller2004a}, \citealp{Frey2004a}),\NOTE{Koster zitieren: \citealp{Koster78a-u}} 
\dash, es muss sich beim vorangestellten Element nicht zwangsläufig 
um einen Topik\is{Topik} oder um einen Fokus\is{Fokus} handeln.\footnote{
  Die Beispiele in (\mex{1}) und (\mex{2}) sind von \citet{Frey2004a}. 
%%
%% Das stimmt aber nicht, denn Berman nimmt ja für lokale Umstellung keine Extraktion an, deshalb
%% darf dann auch irgendwas, was zur lokalen f-Struktur beiträgt in SpecCP stehen.
%%
%% Frey merkt auch an, dass die
%%   Beispiele mit pragmatisch unmarkierten Beispielen \lfg"=Ansätze wiederlegen, die davon ausgehen,
%%   dass Elemente im Vorfeld die gramatikalisierten Diskursfunktionen Subjekt (\subj), Topik (\topic)
%%   oder Fokus (\focus) haben müssen (\citew{Berman97a},\cite[\page 29--30]{Berman2003a}, \citew[\page 181]Bresnan2001a}).%
}

\eal
\label{bsp-voranstellung-pragmatik}
\ex Karl hat das Paket weggebracht.
\ex Dem Karl hat das Spiel gut gefallen.
\ex Einem Mitbewohner wurde die Geldbörse entwendet.
\ex Leider hat keiner dem alten Mann geholfen.
\ex In Europa spielen Jungen gerne Fußball.
\zl
Dass das Vorfeld nicht mit einer pragmatischen Funktion wie Topik oder Fokus verbunden ist,
wird noch klarer, wenn man sich die Wetterverben\is{Verb!Wetter-} ansieht:
\ea
Es wird bald regnen.
\z
Das Expletivum in (\mex{0}) ist nicht referentiell, kann also keinen Anschluß an bereits Gesagtes bilden.

Der Zusammenhang zwischen der pragmatischen Funktion, die ein Element in der ersten Position des Mittelfelds
haben würde, und der Vorfeldpositionierung wird erfaßt, wenn man annimmt, 
dass die Vorfeldbesetzung bei einfachen Sätzen durch das Element erfolgt,
das in einem Verbletztsatz am weitesten links im Mittelfeld stehen würde.

\citet{Frey2004a} weist darauf hin, dass Sätze wie (\mex{1}) gesondert behandelt werden müssen:
\eal
\ex[]{
weil es den Otto friert
}
\ex[]{\label{bsp-den-otto-friert-es}
Den Otto friert es.
}
\ex[*]{
weil den Otto es friert
}
\zl
Obwohl \emph{den Otto} nicht vor dem Pronomen\is{Pronomen!Linearisierung} stehen kann, ist die Stellung in (\mex{0}b) unmarkiert.
Frey behandelt das \emph{es} deshalb als Klitikon\is{Klitisierung}, das für die Besetzung des Vorfeldes keine Rolle
spielt. Alternativ kann man -- wie das in der HPSG üblich ist -- annehmen, dass die Abfolge der Elemente
im Mittelfeld über Linearisierungsbeschränkungen\is{Linearisierung!-sregel} geregelt wird, und diese würden Pronomina im Mittelfeld
vor der Spur anordnen, so dass man auch die Unmarkiertheit von (\mex{0}b) ableiten kann.%
\is{Informationsstruktur|)}%
\is{Extraktion|)}
%
%% Da das in unserer Analyse genau das Element ist, das zuletzt mit dem Verb kombiniert wird, erzwingt
%% die folgende Beschränkung, das nur dieses Element extrahiert werden kann.
%% \ea
%% \label{implikation-extraktion-hoechstes-element}
%% \ms[head-argument-phrase]{
%% non-head-dtrs & \liste{ \ms{ phon & \eliste\\ } }\\
%% } \impl \ms{ synsem$|$loc$|$cat$|$subcat & \eliste \\ }
%% \z
%% Die Beschränkung drückt Folgendes aus: Wenn die Nicht"=Kopf"|tochter in einer Kopf"=Argument"=Struktur
%% eine Spur ist, dann muss die gesamte Struktur vollständig gesättigt sein, \dash, die Nicht"=Kopf"|tochter
%% ist das höchste Argument.


%% Die Implikation wird ja nicht benutzt, da das mit den Pronomina nicht hinhaut.
%% \is{Verbalkomplex|(}%
%% Die folgenden Daten scheinen den Ansatz zu widerlegen, denn obwohl eine Voranstellung von \emph{in der Kneipe gelesen}
%% möglich ist, ist die Voranstellung im Mittelfeld schlecht.
%% \eal
%% \ex[]{
%% In der Kneipe gelesen hat er das Buch.
%% }
%% \ex[*]{
%% weil in der Kneipe gelesen er das Buch hat
%% }
%% \zl
%% Im Vorgriff auf das Kapitel~\ref{chap-verbalkomplex} kann ich hier sagen, dass die Kombination
%% von \emph{gelesen} bzw.\ \emph{in der Kneipe gelesen} mit \emph{hat} nicht über das Kopf"=Argument"=Schema,
%% sondern über ein spezielles Schema für Prädikatskomplexe läuft. Das Prädikatskomplexschema erlaubt keine
%% lokale Umstellung von Argumenten, weshalb Abfolgen wie die in (\mex{0}b) durch das Prädikatskomplexschema
%% nicht lizenziert sind. Allerdings läßt das Prädikatskomplexschema die Voranstellung von Elementen
%% ins Vorfeld zu (siehe Kapitel~\ref{sec-pvp}). Die Implikation in (\ref{implikation-extraktion-hoechstes-element})
%% sagt nur etwas über Kopf"=Argument"=Strukturen aus, betrifft also das Prädikatskomplexschema nicht.
%% \is{Verbalkomplex|)}%


\section{Alternativen}


\NOTE{Kathol95a vielleicht noch diskutieren}

In diesem Abschnitt sollen zwei Alternativen zu Teilbereichen der hier vorgestellten
Analyse untersucht werden: Die Bestimmung der Satztypen aufgrund von Abfolgemustern und die
lexikalische Einführung von Fernabhängigkeiten.

\subsection{Bestimmung des Satztyps in Abhängigkeit von der Reihenfolge sichtbarer Elemente}
\label{sec-satztypen-kathol}

\mbox{}\citet{Kathol95a,Kathol97a,Kathol2000a,Kathol2001a} hat eine Theorie des deutschen\is{Linearisierung!-sdom"ane|(}
Satzes entwickelt, die die von Mike Reape in die HPSG eingeführten Linearisierungsdomänen
benutzt (Siehe \citew{Reape90a,Reape92a,Reape94a} und Kapitel~\ref{sec-Reape-Linearisierung}).
Die Linearisierungsdomänen verbaler Köpfe werden nach den im Kapitel~\ref{topo} eingeführten topologischen Feldern
eingeteilt. Die Felder werden durchnumeriert: 1 = Vorfeld, 2 = linke Satzklammer, 3 = Mittelfeld
und 4 = rechte Satzklammer. Linearisierungsbeschränkungen sorgen dann dafür, dass Elemente,
die den entsprechenden Feldern zugewiesen wurden, auch in der bekannten Reihenfolge stehen,
\dash 1"=Elemente vor 2"=Elementen usw.

Kathol schlägt vor, den Satztyp in Abhängigkeit von sichtbarem Material\is{leere Kategorie|(} zu bestimmen.
\citet{Kathol97a} weist die klassische CP/IP"=Analyse der deutschen Satzstruktur aufgrund von
Lernbarkeitserwägungen zurück und argumentiert für eine nicht"=abstrakte Syntax, \dash
eine Syntax, in der nur die Reihenfolge sichtbarer Elemente eine Rolle spielt und in der
abstrakte syntaktische Objekte wie \zb leere funktionale Köpfe keine Rolle spielen (S.\,89).


\citet{Kathol2001a} definiert die Satztypen mit Bezug auf die Domänenelemente.
Er nimmt an, dass alle Satztypen Untertypen der folgenden drei Typen sind:
\eal
\label{clause-types}
\ex\label{v1-clause-type} \type{V1-clause} \impl \onems{
                                                  S[\type{fin}]\\
                                                  dom  \liste{ \onems[2]{
                                                               V[\type{fin}]\\
                                                               }, \ldots }
                                                  }

\ex\label{v2-clause-type} \type{V2-clause} \impl \onems{
                                                  S[\type{fin}]\\
                                                  dom \liste{ [ \type{1} ], \onems[2]{
                                                                              V[\type{fin}]\\
                                                                              }, \ldots }
                                                  }
\ex\label{subord-clause-types} \type{subord-clause} \impl \onems{
                                                           S[\type{fin}]\\
                                                           dom  \liste{ \ldots, \onems[2]{
                                                                                \textsc{head} $\neg$ V[\type{fin}]\\
                                                                                }, \ldots } 
                                                           }
\zl
Die Beschränkung für den ersten Typ besagt, dass das finite Verb an erster Stelle stehen muss,
die für den zweiten Typ sagt, dass es ein Element im Feld \type{1} geben muss und dass das
finite Verb unmittelbar anschließend im Feld \type{2} steht. Für Sätze vom Typ
\type{subord-clause} muss gelten, dass im Feld \type{2} (der linken Satzklammer) kein finites
Verb steht. Das finite Verb muss in solchen Sätzen also in einem anderen Feld stehen. Da das
finite Verb aber nur in einer der beiden Satzklammern stehen kann, bleibt nur die rechte Satzklammer
-- also das Feld~\type{4} -- übrig.

Kathol kreuzklassifiziert die Typen in (\mex{0}) mit den Typen \type{declarative}, \type{wh-interrogative} und
\type{polar}. Die entsprechende Typhierarchie zeigt Abbildung~\vref{fig-clausal-types}.
% For this paper it is sufficient to know that \type{v2-clause} and \type{declarative} have a common
% subtype and that \type{v1-clause} and \type{polar} have a common subtype. \type{v2-clause} and \type{polar}
% and \type{v1-clause} and \type{declarative} are incompatible, respectively.
%
\begin{figure}
\begin{forest}
type hierarchy, auto name,
before typesetting nodes={
where={> O_={tier}{max}}{l*=1.5}{},
%for tree={alias/.option=content},
},
[finite-clause
  [internal-syntax, partition
    [root
      [v2 
        [r-wh-int, tier=max]
        [r-decl, tier=max]]
      [v1 [r-pol-int, tier=max]]]
    [subord,]]
  [clausality, partition
    [inter
      [wh, tier=wh-polar-decl
        [s-wh-int, tier=max]]
      [polar, tier=wh-polar-decl
        [s-pol-int, tier=max]]]
    [decl, tier=wh-polar-decl
        [s-decl, tier=max]]]]
\draw (wh.south) -- (r-wh-int.north);
\draw (subord.south) -- (s-wh-int.north);
\draw (subord.south) -- (s-pol-int.north);
\draw (subord.south) -- (s-decl.north);
\draw (polar.south) -- (r-pol-int.north);
\draw (decl.south) -- (r-decl.north);
\end{forest}
\caption{\label{fig-clausal-types}Satztypen nach \citew{Kathol2001a}}
\end{figure}
\itdopt{Überschneidungen vermeiden.}
Die Boxen stehen hier für sogenannte Partitionen\is{Partition}, \dash, linguistische Objekte müssen
jeweils einen der Typen haben, die direkt von einer Box dominiert werden.

Ein solcher Ansatz zur Satztypbestimmung wäre sehr attraktiv, wenn es eine eineindeutige Beziehung
zwischen der Abfolge sichtbarer Elemente und dem Satztyp gäbe. Dem ist jedoch nicht so: In elliptischen
Äußerungen kann man alle Elemente, die in Kathols Satztypbestimmung eine Rolle spielen, weglassen.
Wie bereits in den Kapiteln~\ref{sec-konst-test-probleme-voranstellung}
und~\ref{sec-flache-strukturen} erwähnt wurde, scheint bei oberflächlicher Betrachtung 
das Vorfeld auch mehrfach besetzbar zu sein. Die relevanten Beispiele werden im folgenden diskutiert.

Kathol definiert Verberst- und Verbzweitsätze als Sätze, in deren linker Satzklammer
ein finites Verb steht. Es gibt im Deutschen aber Äußerungen wie die in (\mex{1}),
in denen es (im Hauptsatz) kein Finitum gibt (siehe auch \citew[\page 41]{Paul1919a} für weitere
Beispiele).
\eal
\ex Doch egal,      was  noch  passiert, der Norddeutsche Rundfunk             steht  schon   jetzt als Gewinner fest.\footnote{
        Spiegel, 12/1999, S.\,258.
}
\ex Interessant, zu erwähnen, daß ihre Seele völlig    in Ordnung war.\footnote{
        Michail Bulgakow, \emph{Der Meister und Margarita}. München: Deutscher Taschenbuch Verlag. 1997, S.\,422.
      }
\ex Ein Treppenwitz der    Musikgeschichte, daß die Kollegen   von Rammstein vor    fünf Jahren noch im      Vorprogramm   von Sandow spielten.\footnote{
        Flüstern \& Schweigen, taz, 12.07.1999, S.\,14. %war das englisch? 07.12.1999, p.\,14
}
\zl
In den Äußerungen in (\mex{0}) wurde die Kopula\is{Kopula|(} \emph{sein} weggelassen.
Diese Äußerungen entsprechen den Sätzen in (\mex{1}):
\eal
\ex Doch was noch passiert, ist egal, \ldots
\ex Zu erwähnen, dass ihre Seele völlig in Ordnung war, ist interessant.
\ex Dass die Kollegen von Rammstein vor fünf Jahren noch im Vorprogramm von Sandow spielten ist ein Treppenwitz der Musikgeschichte.
\zl
Die Kopula, die mit Adjektiven benutzt wird, leistet keinen eigenen semantischen Beitrag,
sie stellt lediglich die Kongruenzinformation und die verbalen Merkmale zur Verfügung, die von anderen Prädikaten
gebraucht werden, die das Adjektiv + Kopula einbetten \citep[\page 41]{Paul1919a}.
Wie die Beispiele in (\mex{-1}) zeigen, kann die Kopula weggelassen werden. Das Ergebnis sind dann
Sätze ohne finites Verb.

Die Beispiele in (\mex{-1}) sind Deklarativsätze, \dash, sie sollten dem Muster in
(\ref{v2-clause-type}) entsprechen. (\mex{1}) ist ein Beispiel für eine verblose Frage.
Dieser Satz entspricht einem Verberstsatz mit der Kopula an der ersten Position,
\dash, er sollte dem Muster in (\ref{v1-clause-type}) entsprechen.
\ea
Niemand da?\footnote{
        \citew[\page 13]{Paul1919a}.
}
\z

\noindent
Man kann die Satztypbestimmung retten, indem man ein phonologisch leeres Verb annimmt.\footnote{
  Siehe auch \citew{Bender2000a} und \citew*[\page 464]{SWB2003a} für Analysen
  mit einer phonologisch leeren Kopula für \emph{African American Vernacular English}\il{African American Vernacular English}.%
}\is{Kopula|)}
\citew[Kapitel~5.4.1]{Kathol95a} schließt jedoch phonologisch leere Domänenelemente explizit aus.
%I gives a lexical entry for a trace as used in nonlocal dependency constructions.
Eine weitere Argumentation gegen leere Elemente findet sich in \citew[\page 38]{Kathol2001a}.
In Kathols nicht"=abstrakter Syntax haben leere Elemente keinen Platz.

Ein anderes Phänomen, das ebenfalls problematisch für den Linearisierungsansatz ist,
ist die sogenannte Vorfeldellipse, die auch \emph{Topic Drop} genannt
wird. \citet{Huang84},\NOTE{FB: Scheint ein Widerspruch zu früherer Anmerkung zu sein, dass das
  Vorfeld nichts mit pragmatischen Funktionen zu tun hat.}
\citet{Fries88b} und \citet{Hoffmann97a} diskutieren diese Konstruktion im Detail.
Wie bereits im Abschnitt~\ref{sec-semantik-v1-v2} festgestellt wurde, läßt sich in bestimmten
Fällen die Vorfeldellipse nicht von Fragen unterscheiden, wenn man nur die lineare Abfolge
von realisierten Elementen betrachtet. Das entsprechende Beispiel sei hier wiederholt:
\eal
\label{bsp-hat-er-gegessen-zwei}
\ex Hat er schon gegessen?
\ex Und was ist mit dem Kuchen?\\
    Hat er schon gegessen.
\zl
Die Sätze unterscheiden sich nur in bezug auf ihre Intonation.

Die Bestimmung des Satztyps würde funktionieren, wenn man annehmen würde, dass das Vorfeld
durch ein leeres Element besetzt sein kann. Dann wäre (\mex{0}b) als Verbzweitsatz analysierbar.

Will man keine leeren Elemente verwenden, bleibt nur, weitere Muster aufzuschreiben. Für die Vorfeldellipse
müßte man dann \zb folgende Beschränkung formulieren:
\ea
\type{V1-topic-drop-clause} \impl \onems{
                                                  S[\type{fin}]\\
                                                  dom  \liste{ \onems[2]{
                                                               V[\type{fin}]\\
                                                               }, \ldots }
                                                  }
\z
Der Typ \type{V1-topic-drop-clause} wäre dann ein Untertyp von \type{root}, und man müßte
noch einen gemeinsamen Untertyp von \type{V1-topic-drop-clause} und \type{decl} definieren.

Genauso könnte man Typen für die Fälle mit fehlender Kopula\is{Kopula} definieren. Auf diese Weise würde
man jedoch nur alle Abfolgemuster aufzählen, ohne die Zusammenhänge zwischen diesen Mustern
zu erfassen. Für die bisher diskutierten Fälle ist es zwar unschön, aber doch immerhin
möglich, sie in die Typhierarchie zu integrieren. Betrachtet man jedoch die Beispiele mit scheinbar
mehrfacher Vorfeldbesetzung, wie sie schon im Kapitel~\ref{sec-konst-test-probleme-voranstellung}
diskutiert wurden, sieht man, dass ein rein linearisierungsbasierter Ansatz ohne leere Elemente
erhebliche Probleme bekommt.
\eal
\ex {}[Nichts] [mit  derartigen     Entstehungstheorien] hat es natürlich zu tun, wenn \ldots\footnote{
        K. Fleischmann, \emph{Verbstellung und Relieftheorie}, München, 1973, S.\,72.
        zitiert nach \citew[\page 135]{vdVelde78a}.}
\label{bsp-nichts-mit-derartigen-udc}
\ex {}[Trocken] [durch   die Stadt] kommt man am     Wochenende auch mit  der BVG.\footnote{
        taz berlin, 10.07.1998, S.\,22.
      }\label{bsp-trocken-durch-die-stadt-udc}
\ex\label{bsp-alle-traeume-udc}
    {}[Alle Träume] [gleichzeitig]  lassen sich nur  selten verwirklichen.\footnote{
        Broschüre der Berliner Sparkasse, 1/1999.
        }
\zl
In diesen Beispielen steht das finite Verb nicht an zweiter Stelle, obwohl es sich um Deklarativsätze
handelt. Eine Möglichkeit, das zu erfassen, wäre, die Typbeschränkung in (\ref{v2-clause-type}) so aufzuweichen,
dass beliebig viele Elemente vor dem Verb in der linken Satzklammer stehen können:
\ea
 \type{Vn-clause} \impl \onems{
                                                  S[\type{fin}]\\
                                                  dom \liste{ [ \type{1} ], \ldots, \onems[2]{
                                                                              V[\type{fin}]\\
                                                                              }, \ldots }
                                                  }
\z
Das ist aber problematisch, da Kathol Linearisierungsbeschränkungen annimmt, die dafür sorgen,
dass alle Domänenelemente entsprechend ihrer Feldnummer angeordnet sind. Daraus folgt, dass
vor \type{2} nur \type{1}"=Elemente stehen können. Für die Sätze in (\mex{1}) ergibt sich
deshalb, dass \emph{los} und \emph{damit} und \emph{den Atem} und \emph{an} bzw.\ \emph{gut}
und \emph{an} im Vorfeld stehen müssen.
\eal
\label{bsp-particle-vf-udc}
\ex\label{bsp-los-damit-udc} 
\emph{Los} damit \emph{geht} es schon am 15.\ April.\footnote{
        taz, 01.03.2002, S.\,8.%
    }
\ex\label{bsp-den-atem-an-haelt-udc}
Den Atem an hielt die ganze Judenheit des römischen Reichs und weit hinaus über die Grenzen.\footnote{
        Lion Feuchtwanger, \emph{Jud Süß}, S.\,276, zitiert nach \citew[\page56]{Grubacic65a}.
}
\ex Sein Vortrag wirkte [\ldots] %(trotz Verweis auf seine Familie, natürlich) 
ein wenig arrogant, nicht zuletzt wegen seiner Anmerkung,
neulich habe er bei der Premiere des neuen "`Luther"'"=Films in München neben
Sir Peter Ustinov und Uwe Ochsenknecht gesessen. %Ansatz
Gut \emph{an kommt}\iw{ankommen} dagegen die Rede des Jokers im Kandidatenspiel: des Thüringer Landesbischofs Christoph Kähler (59).\footnote{
        taz, 04.11.2003, S.\,3, siehe auch \citew[\page313]{Mueller2005d}.%
}
\zl
In einem solchen Setting kann man nicht erklären, wieso die Sätze in (\mex{1}) schlecht sind:
\eal
\ex[*]{
An den Atem hielt die ganze Judenheit.
}
\ex[*]{
An gut kommt dagegen die Rede des Jokers im Kandidatenspiel.
}
\zl
In der Analyse der Sätze in (\mex{-1}), die ich in \citew{Mueller2005d} vertrete, bildet
die Verbpartikel die rechte Satzklammer in einem komplexen Vorfeld
und \emph{den Atem} bzw.\ \emph{gut} stehen davor. Es ist jedoch nicht so, dass die Verbpartikel
und das jeweils andere Element als eigenständige Konstituenten im Vorfeld stehen. Würde man eine
eigenständige Voranstellung der Elemente annehmen, so ließe sich nicht erklären, wieso die Sätze
in (\mex{0}) ungrammatisch sind, denn prinzipiell kann eine Partikel auch an erster Stelle stehen,
wie (\ref{bsp-los-damit-udc}) zeigt. Mit einem intern strukturierten komplexen Vorfeld sind die Verhältnisse
in (\mex{-1}) und (\mex{0}) dagegen erklärbar, denn dann ist die Abfolge in (\ref{bsp-los-damit-udc}) eine
Extraposition\is{Extraposition} von \emph{damit}, wie sie auch sonst möglich ist (\mex{1}a), und die
Beispiele in (\mex{0}) sind schlecht, weil \emph{den Atem} und \emph{gut} auch ansonsten nicht ins
Nachfeld gestellt werden können, wie (\mex{1}b,c) zeigen:
\eal
\ex[]{
Es geht schon am 15.\ April los damit.
}
\ex[*]{
Deshalb hielt die ganze Judenheit an den Atem.
}
\ex[*]{
Die Rede kommt an gut.
}
\zl

\noindent
Eine Alternative zur Aufweichung der Beschränkung, dass nur eine Konstituente vor dem finiten
Verb stehen darf, besteht darin anzunehmen, dass die Elemente vor dem Finitum eine Konstituente
bilden. Ich habe vorgeschlagen, die Elemente mit dem leeren verbalen Kopf zu kombinieren, der
auch im Kapitel~\ref{sec-v1} für die Analyse der Verbstellung verwendet wurde. Wenn man keine
leeren Köpfe annehmen will, bleibt nur, spezielle Grammatikregeln anzunehmen, die die Elemente
im Vorfeld zu größeren Konstituenten kombinieren. Die Daten, die ich in \citew{Mueller2003b}
diskutiert habe, zeigen jedoch, dass sehr viele verschiedene Arten von Kombinationen
im Vorfeld möglich sind. So können zwei Argumente (\ref{bsp-nichts-mit-derartigen-udc}) oder wie in
(\ref{bsp-trocken-durch-die-stadt-udc}) und in (\ref{bsp-alle-traeume-udc}) Adjunkte und Argumente
gemeinsam im Vorfeld auf"|treten. Die Beispiele in (\ref{bsp-particle-vf-udc}) zeigen, dass eine
Verbpartikel zusammen mit einem anderen Element im Vorfeld stehen kann. Somit braucht man mindestens
drei zusätzliche Grammatikregeln, die Argumente, Adjunkte und Prädikatskomplexbestandteile zu Verben projizieren und auf diese
Weise erreichen, was ein einziger leerer Kopf leisten würde. In Analysen der Verbstellung,
die einen leeren Kopf für das Verb in Endstellung annehmen, ist der entsprechende leere Kopf
ohnehin vorhanden, muss also nicht stipuliert werden. In einer Linearisierungsanalyse gibt es
den leeren Kopf dagegen nicht, weshalb die Stipulation eines leeren Kopfes oder der drei Grammatikregeln
zur Vermeidung des leeren Kopfes nur für die Analyse der scheinbar mehrfachen Vorfeldbesetzung
in die Grammatik aufgenommen werden müssen. Aus diesem Grund ist der in Kapitel~\ref{sec-v1}
und im Abschnitt~\ref{sec-semantik-v1-v2} entwickelten Analyse der Satztypen im Deutschen
gegenüber der linearisierungsbasierten Analyse der Vorzug zu geben.%
\is{leere Kategorie|)}\is{Linearisierung!-sdom"ane|)}


\subsection{Lexikalische Einführung von Fernabhängigkeiten}
\label{sec-lex-intro-udc}

\mbox{}\citet*{BMS2001a}\is{Extraktion|(} entwickeln eine Analyse der Fernabhängigkeiten, bei der
die Information über die Fernabhängigkeit im Lexikon eingeführt wird. Es gibt
eine Liste, die Information über die Valenz eines Kopfes enthält (\argst)\isfeat{arg-st}.
Von dieser Liste gibt es eine Abbildung auf eine weitere Liste (\textsc{deps})\isfeat{deps},
die sowohl Argumente als auch Adjunkte enthält. 
\ea
Argument Structure Extension \citep*[\page 12]{BMS2001a}\\
\type{verb} \impl \ms{
arg-st & \ibox{1}\\
deps   & \ibox{1} $\oplus$ list\textrm{(}`adverbial'\textrm{)}\\
}
\z
Von dieser Liste gibt es schließlich eine Abbildung auf die \textsc{subj}\isfeat{subj}- und \textsc{comps}\isfeat{comps}"=Liste.
\textsc{subj} und \textsc{comps} regeln die Kombinatorik in der Syntax, auf diese Merkmale
wird in den Dominanzschemata Bezug genommen. (\subj und \comps entsprechen der hier verwendeten
\subcatl, wobei Subjekte getrennt von Komplementen repräsentiert werden.)

\eas
Argument Realization \citep*[\page 12]{BMS2001a}\\
\type{word} \impl \ms{
subj  & \ibox{1} \\
comps & \ibox{2} $\ominus$ list\textrm{(}`gap-ss'\textrm{)}\\
deps  & \ibox{1} $\oplus$ \ibox{2}\\
} 
\zs
Dabei ist \type{gap-ss} ein Untertyp von \type{synsem}, für den gilt, dass der \localw mit einem
Element in \slasch identisch ist. Das Subtraktionszeichen $\ominus$\is{$\ominus$}\isrel{$\ominus$} wird dazu benutzt, alle Objekte
vom Typ \type{gap-ss} von der Liste \iboxt{2} abzuziehen. Das heißt, die Liste der abhängigen
Elemente ist \textsc{deps}. Davon werden die Elemente abgezogen, die Gaps sind, und das Ergebnis der Subtraktion wird mit der
\textsc{comps}"=Liste identifiziert.

% Nö, denn SUBJ könnte ja auch leer sein.
%% Eine solche Abbildung sagt voraus, dass Subjekte nicht extrahiert werden können,
%% was für das Englische richtig ist, für das Deutsche aber nicht. 
Paßt man das
Argumentrealisierungsprinzip an die in diesem Buch vertretene Merkmalsgeometrie an,
bekommt man für finite Verben folgendes (siehe Kapitel~\ref{chap-passiv}):
\ea
Argument Realization\\
\type{word} \impl \ms{
subcat & \ibox{1} $\ominus$ list\textrm{(}`gap-ss'\textrm{)}\\
deps   & \ibox{1}\\
} 
\z


\noindent
Für ein ditransitives Verb wie \emph{geben} bekommt man in einem solchen Ansatz
unter anderem die drei \subcatlen in (\mex{1}b--d):
\eal
\ex \liste{ NP[\type{nom}], NP[\type{acc}], NP[\type{dat}] }
\ex \liste{ \hphantom{NP[\type{nom}],~}NP[\type{acc}], NP[\type{dat}] }
\ex \liste{ NP[\type{nom}],\hphantom{~NP[\type{acc}],~}NP[\type{dat}] }
\ex \liste{ NP[\type{nom}], NP[\type{acc}]\hphantom{,~NP[\type{dat}]} }
\zl
Das Subjekt, das Akkusativ"=Objekt bzw.\ das Dativ"=Objekt ist in (\mex{0}b--d) extrahiert, wird
also nicht von \textsc{deps} auf die \subcatl übernommen.

\citet[\page147--148]{Nerbonne94a} argumentiert gegen die Verwendung von Spuren\is{leere Kategorie} und für
die lexikalische Einführung von Fernabhängigkeiten, da man bei der Verwendung von Spuren
das Problem der Serialisierung\is{Serialisierung} der Spuren hat. Wie sich in den letzten Jahren gezeigt hat,
ist das aber kein Problem, sondern im Gegenteil erwünscht.
Die Diskussion in Abschnitt~\ref{sec-udc-is} hat gezeigt, dass das Element im Vorfeld
bei Voranstellung aus dem lokalen Satz dieselbe pragmatische Funktion hat, die es
haben würde, wenn es ganz links im Mittelfeld stehen würde. In einem Ansatz mit Spuren kann
man die entsprechenden Beschränkungen formulieren, \dash, man kann erzwingen, dass die
Spur am weitesten links im Mittelfeld steht, und durch die Interaktion mit Linearisierungsregeln\is{Linearisierung!-sregel}
läßt sich die informationsstrukturelle Funktion der extrahierten Konstituente bestimmen.
Im lexikonbasierten Ansatz läßt sich hingegen nicht feststellen, ob die Argumente
des Verbs definit oder indefinit sind, ob es unbetonbare Pronomina sind oder nicht usw.
Diese Information ist in den Beschreibungen NP[\type{nom}] nicht enthalten, da man nicht im
Lexikon festlegen muss, dass \emph{geben} mit einer definiten Nominalphrase vorkommt. Die Beschreibung
NP[\type{nom}] sagt nichts über die Definitheit/""Indefinitheit des Subjekts von \emph{geben} aus
und läßt damit sowohl Definitheit als auch Indefinitheit zu.
Da die extrahierten Elemente in bezug auf ihre informationsstrukturellen Eigenschaften nicht
bestimmt sind, sind also auch die informationsstrukturellen Eigenschaften, die der zum extrahierten
Element gehörende Füller haben muss, nicht beschränkt.
%%
%% \eal
%% \ex weil dem Mann eine Frau ein Buch gibt
%% \ex Dem Mann gibt eine Frau ein Buch.
%% \zl
Der einzige Ausweg scheint hier zu sein, dass man alle Information, die im Zusammenhang
mit der Linearisierung relevant ist, im Lexikon spezifiziert. Man würde dann für jede
Abfolge einen eigenen Lexikoneintrag mit entsprechend geordneter Valenzliste annehmen
und verlangen, dass die Elemente der Valenzliste von rechts nach links abgebunden werden
(also mit \emph{append} statt \emph{delete}). Für das Verb \emph{geben}, wie es in (\mex{1}a)
benutzt wird, müßte man also eine \subcatl wie in (\mex{1}b) annehmen:
\eal
\ex weil dem Mann eine Frau ein Buch gibt
\ex \liste{ NP[\type{dat}, \textsc{def}+], NP[\type{nom}, \textsc{def}$-$], NP[\type{acc}, \textsc{def}$-$] }
\zl
\citet{Uszkoreit86b} hat eine solche Analyse vorgeschlagen.
Da \citet*{BMS2001a} Adjunkte genau wie Argumente in den \textsc{deps}"=Listen verwalten, würde die Liste
für (\mex{1}a) auch ein Adjunkt enthalten:
\eal
\ex weil dem Mann hoffentlich eine Frau ein Buch gibt
\ex \liste{ NP[\type{dat}, \textsc{def}+], Adjunct, NP[\type{nom}, \textsc{def}$-$], NP[\type{acc}, \textsc{def}$-$] }
\zl
Die beiden folgenden Beispiele von \citet{Frey2004a}\NOTE{Seitenzahl} zeigen, dass
Frame"=Adverbiale (\mex{1}a) im Gegensatz zu Instrumental"=Präpositionalphrasen (\mex{1}b) ins
Vorfeld gestellt werden können, ohne dass die Sätze pragmatisch markiert wären.
\eal
\ex In Europa spielen Jungen gerne Fußball. 
\ex Mit dem Hammer hat Otto das Fenster eingeschlagen.
\zl
Das entspricht der Markiertheit/""Unmarkiertheit der entsprechenden Sätze in (\mex{1}):
\eal
\ex dass in Europa Jungen gerne Fußball spielen
\ex dass Otto mit dem Hammer das Fenster eingeschlagen hat
\ex dass mit dem Hammer Otto das Fenster eingeschlagen hat
\zl
Um diesen Unterschied im Lexikon erfassen zu können, muss man bereits bei der lexikalischen
Einführung der Adjunkte und Fernabhängigkeiten sagen, welcher Art die Adverbiale im Vorfeld
sein werden, \dash, die Tatsache, dass eine Instrumental-PP bzw.\ ein Frame"=Adverbial oder Satzadverb
extrahiert wurde, muss bereits im Lexikoneintrag eines Verbs enthalten sein.

Überlegt man sich, was das genau bedeutet, so sieht man, dass die gesamte Sprache einfach im Lexikon
aufgeschrieben wird. Einzig die genaue phonologische Realisierung und die semantische Relation
werden ausgelassen. 
%% Der Ansatz ist also genauso unbefriedigend wie der oberflächenorientierte Ansatz
%% von Kathol, der einfach auf"|tretende Muster klassifiziert.

Darüber hinaus gibt es das Problem, dass man normalerweise davon ausgeht, dass Elemente in Valenzlisten
Objekte vom Typ \type{synsem} sind (siehe Kapitel~\ref{chap-lokalitaet} zur Lokalität der Selektion)
und dass deshalb im Lexikon phonologische Information abhängiger Elemente nicht zur Verfügung steht,
da \phon nicht Teil der in \type{synsem}"=Objekten enthaltenen Information ist. Man kann also nicht
erkennen, ob ein unbetonbares Pronomen\is{Pronomen!Linearisierung} vor dem zu extrahierenden Element steht oder nicht, was für
die Analyse von Sätzen wie (\ref{bsp-den-otto-friert-es}) -- hier als (\mex{1}) wiederholt -- ein
Problem darstellt. 
\ea
\label{bsp-den-otto-friert-es-zwei}
Den Otto friert es.
\z
Vielleicht könnte man dieses Problem lösen, indem man ein syntaktisches Merkmal \textsc{betonbar}
innerhalb von \synsem stipuliert, aber \citet[\page 178]{Abraham95a-u} formuliert für die Abfolge
von Pronomina die Regel Dentale\is{Dental} vor Labialen\is{Labial}. Information über die Aussprache von syntaktischem
Material gehört aber eindeutig zu \phon und nicht zu \synsem.
%
%
%Dental [n] und der Labial [m] soll das ihn < ihm bedeuten? Vielleicht gibt es da ja Unterschiede
% zu anderen Pronomina und diese gewichtete Regel spielt wirklich eine Rolle.
%
%



\begin{comment}
\subsection{Mittelfeldabfolge und Vorfeldbesetzung in LFG}


LFG\is{Lexical Functional Grammar@\emph{Lexical Functional Grammar} (LFG)|(}
ist im Gegensatz zur HPSG keine monostratale Theorie. Es gibt eine Beschreibung für die
Konstituentenstruktur (c"=Struktur\is{c"=Struktur}) und eine für die funktionale Struktur (f"=Struktur\is{f"=Struktur}). Die
f"=Struktur enthält Informationen über grammatische Funktionen, die wie \subj, \obj
(Akkusativobjekt), \objz (Dativobjekt) und \comp von einem Verb regiert sein müssen oder wie \topic,
\focus und \adj (Adjunkt) unregiert sind. In einer 
\fstr gibt es ein \predm, das eine Repräsentation des Prädikats in einem gewissen lokalen Kontext und
die vom Prädikat regierten grammatischen Funktionen als Wert hat. Von der
c"=Struktur gibt es eine Abbildung auf die f"=Struktur. Die Diskussion der Analyse des Satzes
(\mex{1}) in Abbildung~\vref{abb-vl-lfg} soll das verdeutlichen.
\ea
{}dass der Mann dem Jungen hilft
\z
\begin{figure}
\centerline{%
\begin{tabular}{@{}ccc}
\multicolumn{3}{c}{\rnode{vp}{VP}}\\[2ex]
\rnode{npnom_oben}{(\up \textsc{subj}) = \down} & \rnode{npacc_oben}{(\up \textsc{obj}) = \down} & \rnode{v_oben}{\up = \down}\\
\rnode{npnom}{NP[nom]}  & \rnode{npdat}{NP[dat]} & \rnode{v}{V}\\[3ex]
d\rnode{David}{er Man}n & d\rnode{s}{em Junge}n  & \rnode{hilft}{hilft}\\
\end{tabular}%
\ncline{vp}{npnom_oben}\ncline{vp}{npacc_oben}\ncline{vp}{v_oben}%
\ncline{npnom}{David}\ncline{npdat}{s}\ncline{v}{hilft}%
\hspace{4em}%
\rnode{alles}{\lfgms{ pred & `helfen\sliste{\subj,\objz}'\\
         subj  & \rnode{fmann}{\lfgms{ pred & `Mann' \\
                                     }}\\
         tense & PRESENT\\
         obj2  & \rnode{fjunge}{\lfgms{ pred & `Junge'\\
                                    }}\\
       }}
{%\nodemargin1pt
\ncline{<->}[r]{vp}[l][340]{alles}{6em}%
\ncline{<->}[r]{v}[l]{alles}{3em}%
\ncline{<->}[r][60]{npnom}[l][330]{fmann}{6em}%
\ncline{<->}[r][340]{npdat}[l][40]{fjunge}{6em}%
}}
\caption{\label{abb-vl-lfg}Beispiel"=Analyse für \emph{der Mann dem Jungen hilft}}
\end{figure}
Die Annotation (\up \textsc{subj}) = \down{} besagt, dass der Subjektwert der \fstr, die zum Mutterknoten
(der VP) gehört, identisch ist mit der \fstr des Knotens unterhalb der Annotation (NP[nom]). Durch
entsprechende \cstr"=Annotationen kann man erreichen, dass bestimmte Phrasenstrukturpositionen auf
grammatische Funktionen in der \fstr des zugehörigen Kopfes abgebildet werden. Bei Köpfen wird die
\fstr einfach an den Mutterknoten weitergegeben (\up = \down). In Abbildung~\ref{abb-vl-lfg} kommt
der Hauptanteil der \fstr vom Verb \emph{hilft}. Das Verb steuert den \pred"= und den \textsc{tense}"=Wert bei. Die beiden NPen
liefern die Werte für \subj und \obj in dieser \fstr.

Nun stellt sich die Frage, wie die zugehörige \cstr aussieht. Eine Möglichkeit ist eine
Phrasenstrukturregel wie (\mex{1}):\footnote{
  Siehe auch \citew[\page 110]{Bresnan2001a-u} und \citew[Abschnitt~2.2]{Dalrymple2006a} zu einer entsprechenden Regel mit
  optionalen Konstituenten auf der rechten Regelseite.
}

\ea
\phraserule{VP}{
\rulenode{(NP[nom])\\*(\up\ \subj) = \down}
\rulenode{(NP[dat])\\*(\up\ \objz) = \down}
\rulenode{(NP[acc])\\*(\up\ \obj) = \down}
\rulenode{(V)\\* \up = \down}}
\z
In dieser Regel sind alle Bestandteile der rechten Regelseite als optional gekennzeichnet. Dass
obligatorische Argumente nicht weggelassen werden, wird durch Wohlgeformtheitsbedingungen für
\fstren sichergestellt, die besagen, dass alle im \predw vorkommenden grammatischen Funktionen auch
in einer \fstr realisiert sein müssen (\emph{Completeness}\is{Vollständigkeit}\is{Completeness@\emph{Completeness}|see{Vollst"andigkeit}}). Dafür dass \emph{helfen} nicht mit einem
Akkusativobjekt realisiert wird, sorgt eine Wohlgeformtheitsbedingung, die verlangt, dass in einer
\fstr keine Argumentfunktionen auf"|tauchen dürfen, die nicht in \pred vorkommen 
(\emph{Coherence}\is{Coherence@\emph{Coherence}|see{Koh"arenz}}\is{Koh"arenz!LFG}).

Das Problem mit (\mex{0}) ist nun, dass man mit (\mex{0}) genau eine Anordnung der Argumente im
Mittelfeld ableiten kann. Das liegt daran, dass LFG keine Trennung zwischen Dominanz\is{Dominanz} und Präzedenz\is{Pr"azedenz}
annimmt. Durch die Spezifikation der Kasuswerte in (\mex{0}) ist also die Reihenfolge der Argumente
genau festgelegt.

Man könnte nun einen Ansatz mit "`Bewegung"' und Spuren annehmen, wie das oft auch in der \gbt
gemacht wird. Diese Analysen wurden bereits im Kapitel~\ref{sec-Scrambling-Skopus} kritisiert. Will
man keinen bewegungsbasierten Ansatz, so bleiben zwei Möglichkeiten: explizite Aufzählung der
möglichen Anordnungsmuster und andere Zuordnung der grammatischen Funktionen. Beide sind nicht
unproblematisch, wie ich im folgenden darlegen werde: Bei der ersten Variante, der expliziten
Aufzählung der Abfolgemuster ergibt sich ein Problem mit den als optional markierten Konstituenten.
Für die Analyse des Satzes in (\mex{1}) würde man wohl die Stellungsvariante (\mex{2}) der Regel in (\mex{0}) verwenden.
\ea
dass ihm keiner hilft
\z
\ea
\phraserule{VP}{
\rulenode{(NP[dat])\\*(\up\ \objz) = \down}
\rulenode{(NP[nom])\\*(\up\ \subj) = \down}
\rulenode{(NP[acc])\\*(\up\ \obj) = \down}
\rulenode{(V)\\* \up = \down}}
\z
Das Problem ist jedoch, dass es außerdem die Regeln in (\mex{1}) geben muss, damit die entsprechenden
Anordnungen bei Sätzen mit ditransitiven Verben analysierbar sind:
\eal
\ex
\phraserule{VP}{
\rulenode{(NP[dat])\\*(\up\ \objz) = \down}
\rulenode{(NP[acc])\\*(\up\ \obj) = \down}
\rulenode{(NP[nom])\\*(\up\ \subj) = \down}
\rulenode{(V)\\* \up = \down}}
\ex
\phraserule{VP}{
\rulenode{(NP[acc])\\*(\up\ \obj) = \down}
\rulenode{(NP[dat])\\*(\up\ \objz) = \down}
\rulenode{(NP[nom])\\*(\up\ \subj) = \down}
\rulenode{(V)\\* \up = \down}}
\zl
Die Regeln in (\mex{0}) unterscheiden sich von der in (\mex{-1}) nur durch die Position der
Akkusativ"=NP. Diese ist jedoch optional, wird in der \fstr von \emph{hilft} nicht verlangt und wird
deshalb bei einer Analyse von (\mex{-2}) auch nicht realisiert. In der Analyse von (\mex{-2}) können
somit alle drei Regeln angewendet werden, weshalb man für Sätze mit zweistelligen Verben immer drei
Analysen bekommt.\is{Mehrdeutigkeit!unechte} Dies läßt sich nur beheben, indem man die umstellbaren Argumente obligatorisch
macht und für alle Anordnungsmuster eine eigene Regel annimmt. Damit ist man aber wieder bei einer
Grammatik angelangt, die von Chomsky und den Vertretern der GPSG als nicht adäquat eingestuft wurde.

Ein alternativer Ansatz wurde von Berman (\citeyear[Abschnitt~2.1.3]{Berman96a-u}; \citeyear[\page 37]{Berman2003a}) 
vorgeschlagen. Sie geht von binär verzweigenden Strukturen aus. Die VP besteht bei ihr nur aus dem
Verbalkomplex und einem V0. V0 kann bei \citet[\page 34]{Berman96a-u} ein overtes Verb oder eine
Verbspur\is{Spur!Verb-} sein, \citet[\page 41]{Berman2003a} geht dagegen von einer Analyse mit optionalem finiten
Verb in der VP aus. Bei Verberstsätzen steht das finite Verb in C und steuert von dort seine
\fstr"=Information bei. Die Argumente stehen bei Berman nicht innerhalb der VP, sondern außerhalb
und werden über die folgende Regel mit der VP verbunden:
\ea
\phraserule{VP}{
\rulenode{ZP\\*\up = \down}
\rulenode{VP\\*\up = \down}}
\z

\noindent
ZP steht dabei für alle Phrasen, die sowohl im Vorfeld als auch im Mittelfeld auf"|treten können,
\dash für Subjekte, Objekte verschiedenster Art und Adjunkte (S.\,30). (\mex{1}) zeigt einen vereinfachten Ausschnitt
der ZP"=Regel für DPen:
\ea
\label{LFG-ZP-Regel}
\phraserule{ZP}{
\rulenode{\begin{tabular}[t]{@{}l@{~}l@{}}
          DP\\
          \{  & (\up\ \subj) = \down\\
          $|$ & (\up\ \vcomp{}* \{\obj $|$ \objz $|$ \ldots\}) = \down \}\\
          \ldots\\
          \end{tabular} } }
\z
Der senkrechte Strich steht in (\mex{0}) für \emph{Disjunktion}\is{Disjunktion}. Die Regel besagt, dass die
abgeleitete DP das \subj, \obj bzw.\ \objz in der \fstr der Mutter 
beisteuert. Da die \fstr der ZP in der Regel (\mex{-1}) mit der \fstr der VP identifiziert wird
(\up = \down), trägt die DP also zur \fstr des Verbs in der VP bei. Der Zusatz \vcomp{}* im zweiten
Disjunkt in (\mex{0}) wird in Bermans Analyse benötigt, da das Hauptverb in Sätzen mit Hilfsverben
unter die Hilfsverben eingebettet sein kann. Die \fstr des Hauptverbs ist dann in die \fstr des
Hilfsverb unter dem Merkmal \vcomp eingebettet. \vcomp{}* steht für einen beliebig langen Pfad aus
\vcomp"=Merkmalen (\textsc{vcomp$|$vcomp$|$vcomp \ldots}). Insbesondere ist auch ein Pfad der länge
Null möglich. In diesem Fall steuert die DP ihre \fstr direkt als \obj"= oder \objz"=Wert zur
übergeordneten \fstr bei. Ausdrücke wie \vcomp{}* werden auch zur Modellierung von
Fernabhängigkeiten verwendet \citep{KZ89a}. Da es damit eine gewisse Variabilität in der Abbildung von der \cstr
zur \fstr gibt, spricht man auch von \emph{funktionaler Ungewissheit} (\emph{functional
uncertainty}). \citet[\page 31--33]{Berman97a} gibt weitere Ersetzungsregeln für ZP an, die aus ZP
Präpositionalphrasen, Adverbphrasen und Adjektivphrasen ableiten.
% 
% Abschnitt 2.3.3 eigene syntaktische Kategorie für VPen im Vorfeld

\citet{Berman2003a} gibt keine Phrasenstrukturregeln an. Sie formuliert aber
die beiden folgenden Beschränkungen, die DPen in Abhängigkeit von ihrem Kasus eine grammatische
Funktion zuordnen (S.\,37):
\eal
\ex (\down \textsc{case}) = NOM $\Rightarrow$ (\up \subj) = \down{}
\ex (\down \textsc{case}) = ACC $\Rightarrow$ (\up \obj) = \down{}
\zl
Solche Bedingungen sind aber problematisch, denn nicht alle Nominative sind Subjekte und nicht alle
Akkusative sind Objekte. In (\mex{1}a, b) gibt es neben dem Subjekt bzw.\ Objekt Prädikate im
Nominativ bzw.\ im Akkusativ. In (\mex{1}c) liegt ein Akkusativ"=Adjunkt vor. Siehe
Kapitel~\ref{sec-sem-kasus} zu anderen Beispielen für Adjunkt"=Nominalphrasen.
\eal
\ex Er ist ein Lügner.
\ex Er nannte ihn einen Lügner.
\ex Er arbeitete den ganzen Tag.
\zl

\noindent
Man muss also eine sehr viel genauere Fallunterscheidung vornehmen. Das geht wohl am einfachsten in
einer disjunktiven Regel wie der in (\ref{LFG-ZP-Regel}).

Für die Vorfeldbesetzung nimmt \citet[\page]{Berman2003a} Strukturen wie die in
Abbildung~\vref{Abb-LFG-Vorfeld} an.
\begin{figure}

\centerline{%
\begin{tabular}{@{}ccc@{}}
\multicolumn{2}{c}{\rnode{cp}{CP}}\\[4ex]
\rnode{npnom_oben}{(\up \textsc{subj}) = \down} & \rnode{c1_oben}{\up = \down}\\
\rnode{npnom}{NP[nom]}  & \rnode{c1}{C$'$}\\[4ex]
                       & \rnode{c_oben}{\up = \down} & \rnode{vp_oben}{\up = \down}\\
                       & \rnode{c}{C}              & \rnode{vp}{VP}\\[4ex]
                       &                          & \rnode{npacc_oben}{(\up \textsc{obj}) = \down}\\
                       &                          & \rnode{npacc}{NP[dat]}\\[5ex]
D\rnode{David}{er Man}n &\rnode{ver}{hilft}   & de\rnode{s}{m Jung}en\\
\end{tabular}%
\ncline{cp}{c1_oben}\ncline{cp}{npnom_oben}%
\ncline{c1}{c_oben}\ncline{c1}{vp_oben}%
\ncline{vp}{npacc_oben}%
\ncline{c}{ver}\ncline{npnom}{David}\ncline{npacc}{s}%
\hspace{4em}
\raisebox{3em}{\rnode{alles}{\lfgms{ pred & `helfen\sliste{\subj,\objz}'\\
         subj & \rnode{fmann}{\lfgms{ pred &  `Mann' \\
                   }}\\
         tense & PRESENT\\
         obj2  & \rnode{fjunge}{\lfgms{ pred & `Junge'\\
                   }}\\
       }}}
\ncline{<->}[r]{vp}[l]{alles}{3em}%
\ncline{<->}[r]{c1}[l]{alles}{3em}%
\ncline{<->}[r][30]{c}[l]{alles}{6em}%
\ncline{<->}[r][25]{npnom}[l][330]{fmann}{3em}%
\ncline{<->}[r]{npacc}[l][40]{fjunge}{3em}%
}

\caption{\label{Abb-LFG-Vorfeld}Analyse für \emph{Der Mann hilft dem Jungen.}}
\end{figure}
In \citew[\page 20]{Berman96a-u} gibt sie eine Regel für die CP, die der folgenden entspricht:
\ea
\phraserule{CP}{
\rulenode{ZP\\*\up = \down}
\rulenode{C\\*\up = \down}}
\z
Diese Regel besagt, dass im Vorfeld dieselben Phrasen stehen können, die auch im Mittelfeld
vorkommen. Das ist nicht ganz richtig, da zum Beispiel die Reflexivpronomina, die zu inheränt
reflexiven Verben gehören, nicht vorfeldfähig sind:
\eal
\ex[]{
weil er sich wirklich gut erholt hat
}
\ex[*]{
Sich hat er wirklich gut erholt.
}
\zl
Sätze wie (\mex{0}b) lassen sich leicht über zusätzliche Beschränkungen für die Vorfeldposition
ausschließen. Problematisch ist jedoch die \fstr"=Annotation in der ZP"=Regel in
(\ref{LFG-ZP-Regel}), denn die besagt, dass eine ZP Information zu einem \subj, \obj bzw.\ \objz in der übergeordneten
\fstr oder zu einem über \vcomp"=Pfade zugänglichen \obj bzw.\ \objz in der übergeordneten \fstr beiträgt.

Damit ist es aber nicht möglich nicht"=lokale Fälle von Vorfeldbesetzung wie
(\ref{bsp-um-zwei-millionen}) -- hier als (\mex{1}) wiederholt -- richtig zu erfassen:
\ea
\label{bsp-um-zwei-Millionen-zwei}
{}[Um zwei Millionen Mark]$_i$ soll er versucht haben, [eine Versicherung \_$_i$ zu betrügen].\footnote{
         taz, 04.05.2001, S.\,20.
}
\z
Das Präpositionalobjekt in (\mex{0}) hängt weder von \emph{soll}, noch von \emph{haben}, noch von
\emph{versucht} ab. Das sind die drei Verben, deren \fstren man über \vcomp{}* erreichen
könnte. \zuie wie \emph{eine Versicherung zu betrügen} werden in der LFG normalerweise als \xcomp
analysiert \citep{Dalrymple2006a}. \citet[\page 17]{Berman96a-u} argumentiert jedoch dafür, ihnen auch
die grammatische Funkton \obj zuzuordnen. Damit man (\mex{0}) analysieren kann, müßte man
(\ref{LFG-ZP-Regel}) wie folgt abändern:
\ea
\phraserule{ZP}{
\rulenode{\begin{tabular}[t]{@{}l@{~}l@{}}
          DP\\
          \{  & (\up\ \subj) = \down\\
          $|$ & (\up\ \{\vcomp $|$ \obj{}\}* \{\obj $|$ \objz $|$ \ldots\}) = \down \}\\
          \ldots\\
          \end{tabular} } }
\z
In (\mex{0}) steht nun statt \vcomp{}* der Ausdruck \{\vcomp $|$ \obj{}\}*. Dieser erlaubt unter
anderem eine beliebig lange Folge von \obj"=Merkmalen. Die Vorfeldkonstituente in
(\ref{bsp-um-zwei-Millionen-zwei}) kann also einen \fstr"=Beitrag zur \fstr unter
\textsc{vcomp$|$""vcomp$|$""obj} leisten. Damit ergibt sich aber ein neues
Problem: Wenn eine DP im Akkusativ das Objekt zu einer \fstr oder aber auch ein beliebig tief in die
\fstr eingebettetes Objekt beitragen kann, dann gibt es nichts, was eine (\mex{1}c) entsprechende
Lesart von (\mex{1}a) ausschließt.
\eal
\ex Diesen Hund hat er Peter gezwungen, sofort einzuschläfern.
\ex Er hat Peter gezwungen, diesen Hund sofort einzuschläfern.
\ex Er hat diesen Hund gezwungen, Peter sofort einzuschläfern.
\zl
\emph{diesen Hund} befindet sich in (\mex{0}a) im Vorfeld und kann somit den Wert eines beliebigen \obj in der
Gesamtstruktur füllen, \ua auch den \objw von \emph{gezwungen}. \emph{Peter} steht im Mittelfeld und
kann zur \fstr von \emph{hat gezwungen} beitragen. Da \emph{einzuschläfern} eine grammatische
Funktion in der \fstr von \emph{hat gezwungen} hat und da Fernabhängigkeiten -- wie (\mex{1}a) zeigt
-- in die VP hineinreichen, kann \emph{Peter} also auch den \objw von \emph{einzuschläfern} beitragen. Alle
Strukturen sind kohärent und vollständig, es gibt also nichts, was die unerwünschte Analyse in
(\mex{0}c) ausschließt. Die Konsequenz ist, dass man für die Konstituenten im Mittelfeld eine andere
\fstr"=Annotation braucht als für die im Vorfeld: Objekte im Vorfeld können zum obersten Verb(alkomplex) oder
aber zu tiefer eingebetteten Verben gehören, Objekte im Mittelfeld dagegen nur zu den Verben
im entsprechenden Kohärenzfeld (siehe Kapitel~\ref{chap-anhebung}). Das heißt aber, dass man an zwei verschiedenen
Stellen in der Grammatik sagen muss, dass (bestimmte) Nominative \subj sind, dass (bestimmte)
Akkusative \obj sind usw. 
%
%Alternativ zur Verwendung einer Annotation der Vorfeldkonstituente, die
%mit funktionaler Ungewißheit eine zur Vorfeldkonstituente passende f"=Struktur sucht, kann man auch
%eine Spur annehmen und die Spur entsprechend annotieren. Die Spur würde sich dann eine Konstituente
%im Vorfeld suchen
%
Im Gegensatz zu der hier beschriebenen Theorie im Rahmen der LFG gibt es
in der HPSG keine grammatischen Funktionen. Eine Nominalphrase kann als Argument, als Adjunkt oder
prädikativ auf"|treten. Welche Funktion vorliegt, wird durch den Lexikoneintrag des Nomens
bestimmt. Fernabhängigkeiten interagieren mit der Funktion nicht, da die Vorfeldkonstituente mit einer
Spur identifiziert wird und somit alle relevante Information in beiden Teilstrukturen präsent ist:
Befindet sich im Vorfeld eine Akkusativ"=NP die Verbprojektionen modifiziert, so ist diese
Information auch bei der Spur präsent und wird im Mittelfeld in einer entsprechenden
Kopf"=Adjunkt"=Struktur mit dem Verb verknüpft. Parallel funktioniert das für Argumente bzw.\
prädikative Nominalphrasen.
\is{Lexical Functional Grammar@\emph{Lexical Functional Grammar} (LFG)|)}%
\end{comment}
\is{Extraktion|)}\is{nichtlokale Abhängigkeit|)}


%\section*{Kontrollfragen}

\questions{
\begin{enumerate}
\item Nennen Sie Beispiele für Fernabhängigkeiten.
\item Warum kann man Fernabhängigkeiten nicht wie Umstellungen im Mittelfeld behandeln?
\item Was unterscheidet die HPSG"=Analysen von transformationsgrammatischen Analysen?
\end{enumerate}
}

%\section*{Übungsaufgaben}

\exercises{
\begin{enumerate}
%% \item Suchen Sie einen Beleg für eine idiomatische Redewendung, bei der sich ein Idiombestandteil im \vf
%%       befindet.
\item Skizzieren Sie einen Baum für den folgenden Satz:
\ea
Diese Wohnung mietet er.
\z
Geben Sie die \slashwe und die \dslwe an.

\item Laden Sie die zu diesem Kapitel gehörende Grammatik von der Grammix"=CD
(siehe Übung~\ref{uebung-grammix-kapitel4} auf Seite~\pageref{uebung-grammix-kapitel4}).
Im Fenster, in dem die Grammatik geladen wird, erscheint zum Schluß eine Liste von Beispielen.
Geben Sie diese Beispiele nach dem Prompt ein und wiederholen Sie die in diesem Kapitel besprochenen
Aspekte.
\end{enumerate}
}


%\section*{Literaturhinweise}

\furtherreading{
Die Analyse der Fernabhängigkeiten in der HPSG geht auf Arbeiten von \citet{Gazdar81} im Rahmen der
\gpsg zurück. Einen guten Überblick über Extraktionsphänomene im Englischen, die Geschichte der
Extraktionsanalysen und einen Vergleich mit transformationellen Analysen geben
\citet{LH2006a}. \citet{Thiersch78a} hat als erster im Rahmen der \gbt eine Analyse der Verbstellung
im Deutschen vorgelegt, die der hier vorgestellten entspricht. \citet{Uszkoreit87a} hat eine Analyse
der Vorfeldbesetzung im Rahmen der GPSG entwickelt, die ebenfalls die von Gazdar entwickelten
Mechanismen zur Analyse von Fernabhängigkeiten verwendet.

Die Analyse der Fernabhängigkeiten in der HPSG wird von \citet{BC2021a} besprochen. Ihr Artikel
diskutiert verschiedene Arten der Einführung von Fernabhängigkeiten, Resumptivpronomina, Paradoxa,
bei denen die Füller nicht zur Lücke passen, und auch Voranstellung vs. Extraposition.
}


% Jong Bok: Which man and which woman did [[Tom hug t] and [Mary kiss t]] respectively?


