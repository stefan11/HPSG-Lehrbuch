%% -*- coding:utf-8 -*-
%%%%%%%%%%%%%%%%%%%%%%%%%%%%%%%%%%%%%%%%%%%%%%%%%%%%%%%%%
%%   $RCSfile: hpsg-kongruenz.tex,v $
%%  $Revision: 1.13 $
%%      $Date: 2007/02/13 11:00:12 $
%%     Author: Stefan Mueller (CL Uni-Bremen)
%%    Purpose: 
%%   Language: LaTeX
%%%%%%%%%%%%%%%%%%%%%%%%%%%%%%%%%%%%%%%%%%%%%%%%%%%%%%%%%


\chapter{Kongruenz}
\label{Kapitel-kongruenz}
\is{Kongruenz|(}

In diesem Kapitel werden Kongruenzphänomene behandelt. Man sagt, dass zwei linguistische
Objekte kongruieren, wenn sie in bestimmten morpho"=syntaktischen Merkmalen (\zb Kasus,
Person, Numerus, Genus) immer übereinstimmen.
So gibt es im Deutschen zum Beispiel Subjekt"=Verb"=Kongruenz: 
Verb und Subjekt müssen in Person und Numerus übereinstimmen.
Wie wir gleich sehen werden, sind Kongruenzphänomene in der HPSG sehr einfach
zu beschreiben.

\section{Subjekt-Verb-Kongruenz}
\is{Kongruenz!Subjekt"=Verb"=|(}

Im Deutschen gibt es Subjekt"=Verb"=Kongruenz. Person und Numerus des Subjekts
müssen zur Form des Verbs passen:
\eal
\ex Ich lache.
\ex Du lachst.
\ex Er/sie/es lacht.
\ex Wir lachen.
\ex Ihr lacht.
\ex Sie lachen.
\zl
Die Analyse der Subjekt"=Verb"=Kongruenz liegt auf der Hand: Das Verb selegiert
das Subjekt und kann also auch Anforderungen an Person- und Numerus"=Werte stellen \citep[\page 82]{ps2}.
Den \localw des Lexikoneintrags für das finite Verb \emph{lachst} zeigt (\mex{1}):
\ea
\localw für \emph{lachst}:\\
\ms{ cat & \ms{ head & \ms[verb]{ vform & fin\\
                                }\\
                spr & \eliste\\
                comps & \liste{ NP[\nom]\ind{1} }\\
               }\\
     cont & \ms[lachen]{
            agens & \ibox{1} \ms{
                             per & 2\\
                             num & sg\\
                             }\\
            }\\
}
\z
In diesem Lexikoneintrag sind die Person- und Numerus"=Werte des Subjektindexes \iboxb{1}
spezifiziert. Wird \emph{lachst} mit einer Subjektsnominalphrase kombiniert, muss diese zweite Person Singular
sein.

Interessant sind Fälle, in denen Formen zusammenfallen. So kann zum Beispiel das Wort
\emph{gab} sowohl mit einem Subjekt in der ersten Person als auch mit einem in der dritten Person vorkommen:
\eal
\ex Ich gab ihm das Buch.
\ex Er  gab ihm das Buch.
\zl
Das kann man modellieren, indem man zwei Lexikoneinträge für \emph{gab} annimmt. Bei einem solchen Vorgehen
wird allerdings nicht erfasst, dass diese beiden Einträge bis auf den Wert des \textsc{person}"=Merkmals absolut
identisch sind. Statt auf der Ebene der Lexikoneinträge eine Disjunktion zu verwenden, verwendet man
die Disjunktion im Wert des \textsc{person}"=Merkmals.\footnote{
  Die beiden Ansätze sind logisch äquivalent. Das kann man sich verdeutlichen, wenn man die
  beiden Repräsentationen in (i) vergleicht:
  \eal
  \ex {} [\textsc{per} 1] $\vee$ [\textsc{per} 3]
  \ex {} [\textsc{per} 1 $\vee$ 3]
  \zl
  Beide beschreiben dieselbe Menge von Objekten. Dennoch wird die Repräsentation in (ii.b) der
  in (ii.a) vorgezogen, da der Numerus"=Wert nicht wie in (ii.a) mehrfach aufgeführt werden
  muss und es somit von vornherein ausgeschlossen ist, dass die Numerus"=Werte in den Disjunkten
  verschieden sein können.
  \eal
  \ex {} \ms{ per & 1 \\
              num & sg \\ } $\vee$ \ms{ per & 3 \\
                                        num & sg \\ }
  \ex \ms{ per & 1 $\vee$\! 3\\
           num & sg \\ }
  \zl
Die Disjunktion lässt sich komplett vermeiden, wenn man entsprechende Typen annimmt. Der \perw von
\emph{gab} wäre dann \type{1\_or\_3} mit den Untertypen \type{1} und \type{3}.
}
Für \emph{gab} hat man also:
\eas
\label{le-gab}
\localw für \emph{gab}:\\
\ms
{ cat & \ms{ head & \ms[verb]
                     { vform & fin \\
                     } \\
             spr    & \eliste\\
             arg-st & \liste{ NP[\textit{nom\/}]\ind{1}, NP[\textit{acc}]\ind{2}, NP[\textit{dat}]\ind{3}   } \\
           } \\
  cont &  \ms{ ltop & \ibox{4}\\
               ind  & \ibox{5}\\
               rels & \liste{ \ms[geben]{ lbl  & \ibox{4}\\ 
                                          arg0 & \ibox{5}\\
                                          arg1 & \ibox{1} \ms{
                                                   per & 1 $\vee$\! 3\\
                                                   num & sg\\
                                                 }\\
                                          arg2 & \ibox{2} \\
                                          arg3 & \ibox{3} \\
           }}}
}
\zs
Verwendet man diesen Lexikoneintrag für die Analyse von (\mex{-1}a), so wird die Agens"=Rolle
von \emph{geben} mit dem Index des Pronomens \emph{ich} identifiziert. Der \textsc{person}"=Wert von
\emph{ich} ist \type{1}, weshalb die Kombination der Beschränkung \type{1} $\vee$\! \type{3} mit dem \textsc{person}"=Wert 
des Pronomens \type{1} ergibt. Bei der Analyse von (\mex{-1}b) ist das Ergebnis der Kombination
der \textsc{person}"=Werte \type{3}.

Verlangt ein Verb keine Nominalphrase im Nominativ, muss es in der dritten Person Singular stehen.
Beispiele zeigt (\mex{1}):
\eal
\ex Ihm graut vor der Prüfung.
\ex\label{sp-dass-er-kommt-gefaellt-mir}
Dass Aicke kommt, gefällt mir.
\ex Den Kindern wird geholfen.
\ex Heute wird gefeiert!
\ex Hier scheint gefeiert zu werden.\label{bsp-scheint-gefeiert-zu-werden}
\zl
Formen wie \noword{graust} oder \word{gefällst} mit Satzargument nach dem Muster von
(\ref{sp-dass-er-kommt-gefaellt-mir}) gibt es im Lexikon nicht. 
Die Interaktion von Kongruenzbedingungen mit der Flexionsmorphologie\is{Flexion} wird 
im Kapitel~\ref{sec-morph-flex-anal} genauer besprochen. Hier soll lediglich festgehalten werden,
dass \zb die Endung für die zweite Person Singular nur mit einem Verb verknüpft
werden kann, das ein entsprechend kongruierendes Subjekt hat. Da das bei \emph{grauen} nicht der Fall
ist, können die Formen der zweiten Person Singular nicht gebildet werden.
Bei (\mex{0}c,e) ist die Sache etwas schwieriger, da \emph{wird} und \emph{scheint} Verben sind,
die mit vielen verschiedenen Verben vorkommen können. Wie das Subjekt aussieht bzw.\ ob es überhaupt
eins gibt, wird von den jeweils anderen Verben bestimmt.\is{Verb!Anhebungs-}
Siehe hierzu auch Kapitel~\ref{sec-subjektanhebung-anal} und Kapitel~\ref{sec-passive-anal}.
Die Flexionsregeln erzwingen
das Vorhandensein eines Subjekts, wenn die Endung von der dritten Person Singular verschieden
ist. Der Unterschied in (\mex{1}) ist also durch das Zusammenspiel mehrerer Beschränkungen
erklärt: Ob \emph{scheinst} ein Subjekt hat oder nicht, hängt vom eingebetteten Verb ab,
ob der Satz grammatisch ist oder nicht, hängt vom Flexionsmorphem ab, das im Fall
von \emph{scheinst} ein Subjekt verlangt und im Fall von \emph{scheint} sowohl mit
subjektlosen Verben als auch mit Verben, die ein Subjekt haben, kompatibel ist.
\eal
\ex[*]{
Dir scheinst zu grauen.
}
\ex[]{
Du scheinst zu schlafen.
}
\ex[]{
Ihm scheint zu grauen.
}
\ex[]{
Er scheint zu schlafen.
}
\zl
Das gilt genauso für das Passivhilfsverb\is{Passiv} \emph{werden}:
\eal\label{bsp-kongruenz-passiv}
\ex[]{
Die Männer werden geschlagen.
}
\ex[*]{
Die Männer wird geschlagen.
}
\ex[*]{
Den Männern werden geholfen.
}
\ex[]{
Den Männern wird geholfen.
}
\zl
Das Akkusativobjekt von \emph{schlagen} wird bei Passivierung zum
Subjekt und muss deshalb mit dem finiten Verb kongruieren. Das Dativobjekt
von \emph{helfen} wird nicht zum Subjekt, weshalb der gesamte Satz
dann kein Subjekt hat und das finite Verb in der dritten Person Singular
stehen muss.

Die Subjekt"=Verb"=Kongruenz\label{page-kongruenz-scheinen} soll im Folgenden noch präzisiert werden.
Dazu müssen wir jedoch auf die Kapitel~\ref{Kapitel-Verbalkomplex} und~\ref{Kapitel-anhebung}
zum Verbalkomplex und das Kapitel~\ref{Kapitel-passiv} zum Passiv vorgreifen:
Ich gehe davon aus, dass Verbalkomplexe im Deutschen so analysiert werden,
dass die Argumente des eingebetteten Verbs zu Argumenten\is{Argument!-anziehung|(} des gesamten Komplexes
werden. In den obigen Beispielen bilden dann \emph{zu grauen} und \emph{scheinen}
bzw.\ \emph{zu schlafen} und \emph{scheinst} einen Komplex. Selbiges gilt auch für
das Passivhilfsverb und das eingebettete Partizip.
Für die Beispiele in (\mex{-1}) und (\mex{0}) ergeben sich die folgenden \compslen:
\ea
\begin{tabular}[t]{@{}l@{ }l@{\hspace{5ex}}l@{}}
  &                          &      \textsc{comps}\\[2mm]
a.&zu schlafen scheinst:     & \nliste{NP[\nom]}\\[2mm]
b.&zu grauen scheint:        & \nliste{NP[\dat]}\\[2mm]
c.&geschlagen werden:        & \nliste{NP[\nom]}\\[2mm]
d.&geholfen wird:            & \nliste{NP[\dat]}\\
\end{tabular}
\z
Diese Valenzlisten enthalten jeweils nur ein Argument, und bei Nominativ-NPen muss es
Kongruenz geben, bei Nicht"=Nominativ"=NPen muss das Verb in der dritten Person stehen.

Die Verhältnisse sind bei Verben mit mehr Argumenten genauso. (\mex{2}) zeigt
die \compslen für die Verbalkomplexe in (\mex{1}):
\eal
\ex weil den Studenten vor der Prüfung zu grauen scheint
\ex weil die Frauen das Spiel zu gewinnen scheinen
\ex weil die Bücher dem Kind überreicht werden
\zl
 \ea
\begin{tabular}[t]{@{}l@{ }l@{\hspace{5ex}}l@{}}
  &                          &      \comps\\[2mm]
a.&zu grauen scheint:        & \nliste{NP[\dat], PP[\type{vor}]}\\[2mm]
b.&zu gewinnen scheinen:     & \nliste{NP[\nom], NP[\acc]}\\[2mm]
c.&überreicht werden:        & \nliste{NP[\nom], NP[\dat]}\\[2mm]
\end{tabular}
\z
Gibt es ein Nominativelement, so muss das Verb mit ihm übereinstimmen,
ist das wie in (\mex{-1}a) nicht der Fall, so muss das Verb in der dritten Person Singular stehen.

Bisher stand das Nominativ"=Element immer an der ersten Stelle in der \compsl. Das ist aber nicht
immer so. Im Kapitel~\ref{sec-remote-passive-phen} wird das sogenannte Fernpassiv\is{Passiv!Fern-}
diskutiert, das sich dadurch auszeichnet, dass Objekte des eingebetteten Verbs zum Subjekt werden
können. Die hier interessierenden Beispiele sind die mit dem Objektkontrollverb \emph{erlauben}
(siehe auch Beispiel (\ref{bsp-auskosten-fernpassiv}) auf S.\,\pageref{bsp-auskosten-fernpassiv}).
In einem Satz wie (\mex{1}) wird das Objekt von \emph{lesen} zum Subjekt des gesamten Satzes.
\ea
Die Aufsätze wurden ihm zu lesen erlaubt.\footnote{%
Siehe S.\,\pageref{bsp-Zeitung-zu-lesen-erlaubt} für die Diskussion eines ähnlichen Beispiels von
\citew[\page309]{Bech55a}.
}
\z
Die Valenzlisten für den Aktivsatz in (\mex{1}) und den Passivsatz in (\mex{0}) zeigt (\mex{2}):
\ea
Ich habe ihm die Aufsätze zu lesen erlaubt.
\z
 \ea
\begin{tabular}[t]{@{}l@{ }l@{\hspace{5ex}}l@{}}
  &                          & \textsc{comps}\\[2mm]
a.&zu lesen erlaubt habe:    & \sliste{NP[\nom], NP[\dat], NP[\acc]}\\[2mm]
b.&zu lesen erlaubt wurde:   & \sliste{NP[\dat], NP[\nom]}\\
\end{tabular}
\z
Bei der Passivierung wird das Subjekt von \emph{erlauben} unterdrückt, so dass
das Dativobjekt an erster Stelle steht. Das Objekt von \emph{lesen} wird im
Nominativ realisiert und muss auch mit dem Verb kongruieren.\is{Argument!-anziehung|)}

Man kann die Generalisierung in Bezug auf die obigen Daten wie folgt zusammenfassen:
Das Verb muss mit der NP im Nominativ\is{Kasus!Nominativ} kongruieren, wenn es keine
gibt, muss das Verb in der dritten Person Singular stehen.\footnote{
  Normalerweise gibt es nur eine NP im Nominativ in einer Valenzliste. In Kopula"=Konstruktionen\is{Kopula} wie
  (i) kommen zwei Nominative vor:
  \ea
  Er ist ein Lügner.
  \z
  Hier kann man davon ausgehen, dass das Verb mit der am wenigsten obliquen NP im Nominativ (dem
  Subjekt) kongruiert. Die Beschreibung der Kongruenzphänomene innerhalb von Kopulakonstruktionen ist
  jedoch nicht trivial \parencites[\page 197--198]{Reis82}[\page 274]{Mueller99a}.%
}
Es ist nun interessant zu sehen, dass dieses Kongruenzprinzip\is{Prinzip!Kongruenz-} auch für andere Sprachen
wie \zb Spanisch\il{Spanisch}\footnote{
Zu den spanischen Daten siehe \citet{VV2000a-u}. Die Analyse, die \citeauthor{VV2000a-u} vorschlagen,
ist im Gegensatz zu der hier vorgeschlagenen keine prinzipielle, sondern nur eine einzelsprachliche
Analyse.
}, Hindi\il{Hindi} und \ili{Isländisch} korrekte Vorhersagen macht \parencites{MSHindi}[\page 229]{MuellerGermanic}.
Hindi hat wesentlich komplexere Kongruenzphänomene \citep{Arsenault2002a}, und man kann auch Subjekt nicht ohne weiteres
mit Nominativ gleichsetzen. Durch ein Kongruenzprinzip, das sicherstellt, dass das Verb
mit dem am wenigsten obliquen Nominativ"=Element kongruiert, werden die Hindi"=Daten und auch die
isländischen Daten korrekt erfasst.
\is{Kongruenz!Subjekt"=Verb"=|)}

\section{NP-interne Kongruenz}
\label{sec-np-kongruenz}

In\is{Kongruenz!NP"=interne|(}\is{Adjektiv|(}\is{Determinator|(}\is{Nomen|(}
der Nominalphrase müssen Determinator, Adjektiv und Nomen in Kasus\is{Kasus}, Numerus\is{Numerus}, Genus\is{Genus} 
und Flexionsklasse\is{Flexion!-sklasse} übereinstimmen. (\mex{1}) zeigt einige Beispiele:
\eal
\ex der große  Baum
\ex des großen Baumes
\ex ein großer Baum
\zl
In (\mex{0}a,b) sieht man, dass Artikel, Adjektiv und Nomen je nach Kasus variieren. (\mex{0}c)
zeigt, dass es im Vergleich zu (\mex{0}a) Unterschiede in der Adjektivform gibt, wenn
der indefinite Artikel benutzt wird. Um den Unterschied zwischen (\mex{0}a) und (\mex{0}c)
erfassen zu können, teilt man Determinatoren und Adjektive in Flexionsklassen ein und
verlangt, dass die jeweils in NPen realisierten Elemente bezüglich ihrer Flexionsklassen
zueinander passen.

Wie bei \emph{gab} fallen in der Flexion im Nominalbereich viele Formen zusammen.
So ist zum Beispiel der Kasuswert von \emph{Baum} die Disjunktion
$nom \vee dat \vee acc$.
Im Singular ist nur der Genitiv von \emph{Baum} morphologisch gekennzeichnet.

Betrachtet man die Beispiele in (\mex{0}), könnte man annehmen, dass die Kongruenz
in der Flexionsklasse nur den Determinator und das Adjektiv betrifft. Dem ist aber nicht so,
denn es gibt eine (kleine) Klasse von Nomen, die wie Adjektive mit dem Determinator kongruieren.
Zu diesen Nomina gehören \emph{Gesandter}, \emph{Verwandter}, \emph{Beamter}.
\eal
\ex ein (frustrierter) Beamter
\ex der (frustrierte)  Beamte
\zl
Diese Nomina werden wie Adjektive flektiert, Genus und Flexionsklasse sind morphologisch markiert.
%ps2:372


Traditionell werden die Adjektivformen in drei Flexionsklassen\is{Flexion!-sklasse|(} eingeteilt: schwach, gemischt
und stark. Dieter Wunderlich\aimention{Dieter Wunderlich} hat 1988 in einer unveröffentlichten Arbeit, die von \citet[\page66]{ps2}
zitiert wird, vorgeschlagen, nur noch zwei Klassen zu verwenden: stark und schwach. Mit dieser
Einteilung ergibt sich folgende Tabelle, wobei \type{strong}\istype{strong} heißt, dass der Determinator
stark sein muss und \type{weak}\istype{weak}, dass der Determinator schwach sein muss.
\ea
\begin{tabular}[t]{lllll}
Form   & \textsc{dtype}  & \textsc{num} & \textsc{gen}  & \textsc{case}\\
kluge  & \type{dtype}        & \type{sg}   & \type{fem}     & $nom \vee acc$\\
       & \type{strong}       & \type{sg}   & \type{mas}     & $nom$\\
       & \type{strong}       & \type{sg}   & \type{neu}     & $nom \vee acc$\\
       & \type{weak}         & \type{pl}   & \type{genus}   & $nom \vee acc$\\
klugen & \type{strong}       & \type{sg}   & \type{genus}   & $gen \vee dat$\\
       & \type{strong}       & \type{pl}   & \type{genus}   & $case$\\
       & \type{dtype}        & \type{sg}   & \type{mas}     & $acc$\\
       & \type{weak}         & \type{sg}   & $mas \vee neu$ & $gen$\\
       & \type{weak}         & \type{pl}   & \type{genus}   & $dat$\\
klugem & \type{weak}         & \type{sg}   & $mas \vee neu$ & $dat$\\
kluger & \type{weak}         & \type{sg}   & \type{fem}     & $gen \vee dat$\\
       & \type{weak}         & \type{pl}   & \type{genus}   & $gen$\\
       & \type{weak}         & \type{sg}   & \type{mas}     & $nom$\\
kluges & \type{weak}         & \type{sg}   & \type{neu}     & $nom \vee acc$\\
\end{tabular}
\z
Die Typen \type{dtype}\istype{dtype}, \type{case}\istype{case} und \type{genus}\istype{genus} stehen dabei für den jeweils
allgemeinsten Typ, \dash, es gibt keine Einschränkung in Bezug auf den Wert von \textsc{dtype}\isfeat{dtype}, 
\textsc{case}\isfeat{case} bzw.\ \textsc{genus}\isfeat{genus} für die entsprechenden Adjektivformen.

Die definiten Artikel haben die Flexionsklasse \type{strong}. Alle flektierten Formen
des Determinators \emph{ein} werden ebenfalls zu den starken Determinatoren gezählt,
wohingegen die unflektierte Form als schwacher Determinator gilt.
\eal
\ex ein kluger Mann   (schwacher Determinator)
\ex einem klugen Mann (starker Determinator)
\zl
In Nominalphrasen ohne Determinator müssen die Adjektive den \textsc{dtype}"=Wert \type{weak} haben.\is{Flexion!-sklasse|)}
\eal
\ex mit frischer Milch
\ex mit gutem Wein
\ex mit trockenem Getreide
\zl
Die Kongruenz zwischen Determinator und Nomen wird genauso sichergestellt wie die zwischen Subjekt und Verb:
Das Nomen selegiert einen Determinator, der in den entsprechenden morpho"=syntaktischen Merkmalen
mit ihm übereinstimmt:
\eas
\label{le-mann}%
\localw für \emph{Mann}:\\
\ms
{ cat & \ms{ head & \ms[noun]
                     { case & \ibox{1} $nom \vee dat \vee acc$\\
                     } \\
             spr    & \nliste{ [] }\\
             arg-st & \nliste{ Det[\textsc{case} \ibox{1}, \textsc{num} \ibox{2}, \textsc{gen} \ibox{3}] }\\
           } \\
  cont &  \ms{
          ltop & \ibox{4}\\
          ind &  \ibox{5} \ms{
                 per & 3\\
                 num & \ibox{2} sg\\
                 gen & \ibox{3} mas\\
                 }\\
          rels & \liste{ \ms[mann]{ 
                          lbl  & \ibox{4}\\
                          arg0 & \ibox{5}\\
                          } } \\
          }\\
}
\zs
Hierbei ist es wichtig, dass die Kasuswerte des Determinators mit denen des Nomens
identifiziert sind. Der Wert ist identisch, denn wäre er nur gleich, würde die Grammatik
ungrammatische Sätze zulassen. Mit einem Lexikoneintrag, der den \catw in (\mex{1}) hat,
könnte man den Satz (\mex{2}) analysieren.
\ea
\catw für \emph{Mann} (falsch!):\\
\ms{ head & \ms[noun]
                     { case & $nom \vee dat \vee acc$\\
                     } \\
     spr    & \nliste{ [] }\\
     arg-st & \nliste{ Det[\textsc{case} $nom \vee dat \vee acc$, \textsc{num} \type{sg}, \textsc{gen} \type{mas}] }\\
}
\z
\ea[*]{
Dem Mann schläft.
}
\z
Das liegt daran, dass \emph{dem} den Kasuswert \type{dat} hat. Die Kombination der Beschränkung
$nom \vee dat \vee acc$ mit \type{dat} ist \type{dat}, \dash, man kann \emph{dem}
mit \emph{Mann} kombinieren. Durch die Identifikation der Beschreibung des Artikels
in der \textsc{spr}"=Liste mit dem \textsc{synsem}"=Wert des Determinators wird der Kasuswert
in der \textsc{spr}"=Liste in (\mex{-1}) zu \type{dat}, der Kasuswert des Nomens bleibt
davon aber unberührt, ist also weiterhin $nom \vee dat \vee acc$. Somit wäre die Nominalphrase
\emph{dem Mann} mit den Anforderungen des Verbs \emph{schläft} kompatibel, was der
Datenlage nicht entspricht.
Bei (\ref{le-mann}) gibt es dagegen kein Problem: Wenn der Determinator mit dem Nomen
kombiniert wird, wird der Kasuswert in der \textsc{spr}"=Liste spezifischer und da dieser
Wert mit dem Kasuswert des Nomens identisch ist, hat die gesamte NP den Kasuswert, der
sich aus der Kombination der Kasuswerte des Nomens und des Determinators ergibt.


Für Nomina wie \emph{Beamter} wird zusätzlich noch die Flexionsklasse\is{Flexion!-sklasse} des Determinators
spezifiziert. (\mex{1}) zeigt ein Beispiel:
\ea
\label{le-beamter}%
\localw für \emph{Beamter}:\\
\ms
{ cat & \ms{ head & \ms[noun]
                     { case & \ibox{1} $nom \vee dat \vee acc$\\
                     } \\
             spr    & \nliste{ Det[\textsc{case} \ibox{1}, \textsc{num} \ibox{2}, \textsc{gen} \ibox{3}, \textsc{dtype} \type{weak}] }\\
             comps & \eliste\\
           } \\
  cont &  \ms{
          ltop & \ibox{4}\\
          ind &  \ibox{5} \ms{
                 per & 3\\
                 num & \ibox{2} sg\\
                 gen & \ibox{3} mas\\
                 }\\
          rels & \liste{ \ms[beamter]{ 
                          ltop & \ibox{4}\\
                          arg0 & \ibox{5}\\
                          }
                        }\\
         }\\
}
\z

\noindent
Die Kongruenz der Adjektive ist nicht ganz so einfach zu erfassen, aber auch nicht wirklich schwierig:
Ein Adjektiv modifiziert eine \nbar"=Projektion, \dash eine Nominalprojektion, die noch mit einem Determinator
kombiniert werden muss. Das Adjektiv kann also Beschränkungen über die Eigenschaften des Determinators
formulieren. Insbesondere kann es Kasus, Genus, Numerus und Flexionsklasse des Determinators bestimmen.
Da das Nomen selbst mit dem Determinator in Kasus, Numerus, Genus und Flexionsklasse\is{Flexion!-sklasse} kongruiert, ist somit
auch die Kongruenz des Adjektivs mit dem Nomen gewährleistet. (\mex{1}) zeigt das am Beispiel
des Adjektivs \emph{interessantes}, das bereits im Kapitel über Modifikation auf Seite~\pageref{le-kleinen-sem}
besprochen wurde:

\eas
\label{le-interessantes-agr}%
Adjektiveintrag mit Kongruenzinformation (\localw für \emph{interessantes}):\\
\ms
 { cat & \ms{ head & \onems[adj]
                      { %prd & $-$ \\
                        mod$|$loc \onems{ cat \onems{spr \liste{~ \textrm{Det[\begin{tabular}[t]{@{}l@{}}
                                                          \textsc{case} $nom \vee acc$,\\
                                                          \textsc{num} \type{sg},
                                                          \textsc{gen} \type{neu},\\
                                                          \textsc{dtype} \type{weak}]\\
                                                          \end{tabular}}~~~}\\
                                                    comps \eliste}\\
                                       cont \ms{ ind   & \ibox{1} } \\
                                     }\\
                      } \\
               spr    & \eliste\\
               comps & \eliste \\
             } \\
   cont & \ms{ ltop  & \ibox{2}\\ 
               ind   & \ibox{3} \\
               restr & \liste{ \ms[interessant]{
                                ltop & \ibox{2}\\ 
                                arg0 & \ibox{3}\\
                                arg1 & \ibox{1}\\ 
                               } } } \\
}
\zs

\noindent
In (\mex{0}) wird anders als im auf Seite~\pageref{le-kleinen-sem}
angegebenen Eintrag nur auf die morpho"=syntaktischen Eigenschaften des Determinators Bezug genommen.
Über den referentiellen Index des Adjektivs wird nichts gesagt, außer dass er mit dem des Nomens
identisch sein muss. Behandelt man die morpho"=syntaktischen Merkmale des Determinators als
Kopfmerkmale, ist sichergestellt, dass die morpho"=syntaktischen Eigenschaften
des Determinators und des Adjektivs von den semantischen Eigenschaften des Nomens unabhängig sind.
Das ist wichtig für Nomina wie \emph{Mädchen} und \emph{Weib}, die innerhalb der NP das Genus Neutrum
haben, auf die man sich aber sowohl mit \emph{es} (neutrum) als auch mit \emph{sie} (femininum) beziehen kann:
\eal
\ex "`Farbe bringt die meiste Knete!"' verriet ein 14jähriges türkisches {\em Mädchen\/}, {\em die\/} die Mauerstückchen am
      Nachmittag am Checkpoint Charlie an Japaner und US-Bürger verkauft.\footnote{
        taz, 14.06.90, S.\,6.
      }
\ex Es ist ein junges {\em Mädchen\/}, {\em die\/} auf der Suche nach CDs bei Bolzes reinschaut.\footnote{
        taz, 13.03.96, S.\,11.
      }
\zl
%Solche Sätze sind jedoch selten und kommen im heutigen Deutsch praktisch nicht vor.
\eal
\label{bsp-neu-neu-goethe}
\ex Das {\em Mädchen\/}, {\em das\/} Rosen und andere Blumen herumtrug, bot ihm \emph{ihren} Korb
dar, \ldots\footnote{
        Goethe, {\em Wilhelm Meisters Lehrjahre\/}, Hamburger Ausgabe, Band 7, S.\,90.
      }
\ex Nun sitz ich hier, wie ein altes {\em Weib\/}, {\em das\/} ihr Holz von Zäunen stoppelt
      und \emph{ihr}\iw{ihr!Possessivum} Brot an den Türen, um ihr hinsterbendes, freudloses Dasein noch einen
      Augenblick zu verlängern und zu erleichtern.\footnote{
        Goethe, {\em Die Leiden des jungen Werther\/}, Hamburger Ausgabe, Band 6, S.\,99.
      }\label{bsp-goethe-altes-weib}
\zl
Der Genus"=Wert im semantischen Index von \emph{Mädchen} und \emph{Weib} muss also \type{fem} $\vee$ \type{neu} sein, Artikel
und Adjektiv dürfen aber nie \type{fem} sein.\footnote{
  Zu Problemen, die Nomina wie \emph{Mädchen} für die Bindungstheorie aufwerfen,
  siehe \citew[\page417--418]{Mueller99a}.%
\itdopt{check Wechsler \& Zlatic.}
}
Das wird korrekt erfasst, wenn \emph{Mädchen} von seinem Determinator verlangt,
dass dieser das Genus Neutrum hat. Adjektive beziehen sich immer auf die Merkmale des Determinators (in der \textsc{spr}"=Liste
des Nomens).

Behandelt man Possessivpronomina\is{Pronomen!Possessiv-|(} als Determinatoren, folgt zwangsläufig, dass die Kongruenzmerkmale
der Determinatoren keine semantischen Merkmale sein können, wie das \zb von \citet[\page 88]{ps2}
im Zusammenhang mit deutschen Kongruenzphänomenen vorgeschlagen wurde, 
denn das Pronomen \emph{seine} in (\mex{1}) referiert auf ein Maskulinum (\emph{Junge}) oder Neutrum
(\emph{Mädchen}, \emph{Kind}), kongruiert aber mit einem Femininum:
\ea
seine Freundin
\z
Den Eintrag für das Possessivum\is{Pronomen!Possessiv-} zeigt (\mex{1}):
\eas
\label{le-seine-agr}%
Lexikoneintrag für \emph{seine}:\\
\ms{ 
  cat & \ms{ head  & \ms[det]{
                     case & nom $\vee$ acc\\
                     gen & fem \\
                     dtype & strong \\
                     spec & \ms{ cont & \ms{ ltop & \ibox{1}\\
                                             ind  & \ibox{2}\\
                                           }}\\
                     }\\
             arg-st & \eliste\\
           } \\
  cont &  \ms{
%       ltop & \ibox{3}\\
       ind  & \ibox{3} \ms{
                       per & 3\\
                       num & sg\\
                       gen & mas $\vee$ neu\\
                      }\\ 
       rels & \liste{ \ms[def\_q]{
                      arg0 & \ibox{2}\\ 
                      rstr & \ibox{4}\\ 
%                      body &\\
                      }, \ms[besitzen\_rel]{
                            lbl  & \ibox{1}\\     % pronoun_q(x,h, h) h1:pronoun_rel(x),
                                % h1:besitzen(x,y) ^ h1:buch(y)
%                           arg0 & event\\
                            arg1 & \ibox{3}\\
                            arg2 & \ibox{2}
                      } }\\
       hcons & \liste{ \ms[qeq]{
                       harg & \ibox{4}\\
                       larg & \ibox{1}\\
                       }
               }\\
           } \\
}
\zs
\is{Pronomen!Possessiv-|)}\is{Kongruenz!NP"=interne|)}%
\is{Adjektiv|)}\is{Determinator|)}\is{Nomen|)}
%% \section{Unterspezifikation und Typen}

%% Die Lexikoneintrag in (\ref{le-interessantes-agr}) und (\ref{le-seine-agr}) sind in Bezug auf ihren
%% Kasus unterspezifiziert. Sie können sowohl in Nominativnominalphrasen als auch in Akkusativnominalphrasen
%% vorkommen.

\begin{comment}
\section{Alternativen}

Analysen für Kongruenzphänomene müssen erfassen, dass Kongruenzbeziehungen zwischen
Elementen bestehen können, die sich nicht im selben lokalen Baum befinden.
So kongruieren in Abbildung~\vref{fig-Du-Maria-erwartest} jeweils Determinator und Nomen
bzw.\ Subjekt und Verb, obwohl sie nicht Töchter eines gemeinsamen Mutterknotens sind.
\begin{figure}[htbp]
%% \centerline{\begin{pspicture}(0.6,0)(8,6.2)
%% %\psgrid
%% \rput[B](1,0){\rnode{Peter}{Du}}
%% \rput[B](3,0){\rnode{Maria}{Maria}}
%% \rput[B](6,0){\rnode{erwartet}{erwartest}}
%% %
%% \rput[B](6,2){\rnode{v}{V}}
%% \rput[B](3,2){\rnode{np1}{NP}}
%% %
%% \rput[B](4.5,4){\rnode{vs}{V}}
%% \rput[B](1,4){\rnode{np2}{NP}}
%% %
%% \rput[B](3,6){\rnode{vp}{V}}
%% %
%% \psset{angleA=-90,angleB=90,arm=0pt}
%% %
%% \ncdiag{v}{erwartet}
%% \ncdiag{np2}{Peter}\ncdiag{np1}{Maria}
%% \ncdiag{vs}{np1}\ncdiag{vs}{v}
%% \ncdiag{vp}{np2}\ncdiag{vp}{vs}
%% %
%% \end{pspicture}}
\hfill \begin{tabular}[b]{@{}cccc@{}}
\multicolumn{2}{c}{\rnode{np}{NP}}\\[4ex]
\rnode{det}{Det} & \multicolumn{2}{c}{\rnode{n1}{N}}\\[4ex]
                & \rnode{a1}{A}            & \multicolumn{2}{c}{\rnode{n2}{N}}\\[4ex]
                &                         & \rnode{a2}{A}           & \rnode{n3}{N}\\[4ex]
\rnode{dem}{dem} & \rnode{kleinen}{kleinen} & \rnode{gruenen}{grünen} & \rnode{maennchen}{Männchen}\\
\end{tabular}
\ncline{np}{det}\ncline{np}{n1}%
\ncline{n1}{a1}\ncline{n1}{n2}%
\ncline{n2}{a2}\ncline{n2}{n3}%
\ncline{det}{dem}%
\ncline{a1}{kleinen}%
\ncline{a2}{gruenen}%
\ncline{n3}{maennchen}%
\hfill \begin{tabular}[b]{@{}cccc@{}}
\multicolumn{2}{c}{\rnode{v1}{V}}\\[4ex]
\rnode{np1}{NP} & \multicolumn{2}{c}{\rnode{v2}{V}}\\[4ex]
               & \rnode{np2}{NP}  & \rnode{v3}{V}\\[4ex]
\rnode{du}{du}  & \rnode{ihn}{ihn} & \rnode{erw}{erwartest}\\
\end{tabular}
\ncline{v1}{np1}\ncline{v1}{v2}%
\ncline{v2}{np2}\ncline{v2}{v3}%
\ncline{np1}{du}%
\ncline{np2}{ihn}%
\ncline{v3}{erw}%
\hfill\mbox{}
\caption{\label{fig-Du-Maria-erwartest}Kongruenz in nichtlokalen Bäumen}
\end{figure}
%
\citet{Kathol99b} entwickelt eine Theorie, in der Kongruenzmerkmale an allen kongruierenden
Elementen repräsentiert sind. Die Kongruenzmerkmale sind bei ihm Kopfmerkmale. Er vergleicht
seine Theorie mit der von \citet{ps2}, die auch der Analyse in diesem Kapitel zugrunde liegt.
In der Analyse der Subjekt"=Verb"=Kongruenz von Pollard und Sag sind Eigenschaften wie
Person und Numerus nur in der Beschreibung des Subjekts enthalten, nicht jedoch in der Beschreibung
des Verbs, wenn man von der Valenzinformation absieht.
Pollard \& Sags Analyse skizziert Kathol wie in (\mex{1}a) und seine eigene wie in (\mex{1}b):
\eal
\ex\label{ps-agr} \begin{tabular}[t]{@{}ll@{}}
\multicolumn{2}{c}{\rnode{x}{x}}\\[3ex]
\rnode{functor}{x / y[F:\iboxt{1}]} & \rnode{argument}{y[F:\iboxt{1}]}\\[2ex]
\end{tabular}
\ncline{x}{functor}
\ncline{x}{argument}

\ex \begin{tabular}[t]{@{}ll@{}}
\multicolumn{2}{c}{\rnode{x2}{x[F:\iboxt{1}]}}\\[3ex]
\rnode{functor2}{x[F:\iboxt{1}] / y[F:\iboxt{1'}]} & \rnode{argument2}{y[F:\iboxt{1'}]}\\
\end{tabular}
\ncline{x2}{functor2}
\ncline{x2}{argument2}
\zl
Dabei steht \emph{x} für den Kopf und \emph{y} für das selegierte Element. \emph{F} ist ein
Kongruenzmerkmal. In Kathols Ansatz wird \ibox{1} mit \ibox{1'} identifiziert, so dass
die Kongruenzinformation an Kopf und selegiertem Element und auch an der gesamten Projektion
vorhanden ist.

%% Kathol nennt verschiedene Argumente gegen Theorien wie (\ref{ps-agr}):
%% \begin{enumerate}
%% \item Im Gegensatz zur Rektion, in der ein Element die Eigenschaften eines anderen bestimmt, gibt
%% es bei der Kongruenz Markierungen an beiden der beteiligten Elemente. Kathol verweist auf 
%% \citew{Zwicky86} für Evidenz dafür, wegen dieses Unterschieds auch andere theoretische Mittel
%% für die Beschreibung der Phänomene zu benutzen.
%% \item Latein
%% \item Subjektlose Verben im Deutschen.

%% \end{enumerate}

%% Arnold Zwickys Artikel beschäftigt sich mit einer Beschreibung der Adjektivkongruenz im Rahmen
%% der \gpsg.

Kathol führt Daten aus dem Latein an, die dafür sprechen, Kongruenzinformation einheitlich
an allen beteiligten Elementen zu repräsentieren.
\ea
\gll illarum             duarum            bonarum            feminarum\\
     dieser$_{gen}$.\fem.\pl{} zwei$_{gen}$.\fem.\pl{} guten$_{gen}$.\fem.\pl{} Frauen$_{gen}$.\fem.\pl\\
\z
Wenn die entsprechende Kongruenzinformation für die beteiligten Nomina und Adjektive an derselben
Stelle in den Beschreibungen der entsprechenden Wörter repräsentiert ist, kann man für
die entsprechenden morphologischen Prozesse generalisierte Regeln verwenden. Wird die zu den
Endungen gehörende Information unterschiedlich repräsentiert, müssen gänzlich verschiedene
morphologische Regeln angenommen werden.

Was die einheitliche Behandlung der Morphologie angeht, ist Kathols Ansatz vorzuziehen. Es gibt jedoch
auch Argumente gegen diesen Ansatz, die im Folgenden präsentiert werden sollen:
In Kathols Ansatz haben also alle Knoten in der Abbildung~\ref{fig-Du-Maria-erwartest} links die
Kongruenzmerkmale Maskulin, Singular, Dativ, strong. Insbesondere haben auch die gesamte NP und
alle Teilprojektionen diese Eigenschaften. Genauso haben alle Verb"=Knoten in Abbildung~\ref{fig-Du-Maria-erwartest} rechts
die Eigenschaft zweite Person Singular zu sein.

Wir haben im Kapitel~\ref{sec-kopfeigenschaften} auf Seite~\pageref{page-kopf-merkmal}
die Kopfeigenschaften als die für die Distribution einer Phrase relevanten Eigenschaften 
definiert. Für Nominalphrasen ist \zb ihr Kasus relevant, da dieser ausschlaggebend dafür ist,
mit welchen Verben eine Nominalphrase vorkommen kann:
\eal
\ex[]{
Der Mann schläft.
}
\ex[*]{
Dem Mann schläft.
}
\ex[]{
Ein Mann schläft.
}
\zl
Die Flexionsklasse des Determinators ist jedoch für die Distribution der Nominalphrase unerheblich:
An Stellen, an denen \emph{der Mann} stehen kann, kann auch \emph{ein Mann} stehen.\footnote{
  Diese Aussage ist zu relativieren, denn in Abhängigkeit von der Informationsstruktur\is{Informationsstruktur} einer
  Äußerung kann eine indefinite Nominalphrase an Stellen schlecht sein, an denen eine definite
  erlaubt ist und umgekehrt. Dies liegt dann aber an den semantischen Eigenschaften der Determinatoren
  und nicht an der für NP"=interne Kongruenz wichtigen Flexionsklasse.%
}
\end{comment}
\is{Kongruenz|)}


%\section*{Kontrollfragen}

\questions{
\begin{enumerate}
\item Welche Kongruenzphänomene kennen Sie?
\end{enumerate}
}

%\section*{Übungsaufgaben}

\exercises{
\begin{enumerate}
\item Überlegen Sie, wie die Kongruenzverhältnisse in den folgenden beiden Sätzen erfasst werden:
\ea
Das kluge Mädchen konnte über diese Dummheit nur lächeln. Sie wusste längst, dass das Gegenteil
richtig war.
\z

\item Laden Sie die zu diesem Kapitel gehörende Grammatik von der Grammix"=CD
(siehe Übung~\ref{uebung-grammix-kapitel4} auf Seite~\pageref{uebung-grammix-kapitel4}).
Im Fenster, in dem die Grammatik geladen wird, erscheint zum Schluss eine Liste von Beispielen.
Geben Sie diese Beispiele nach dem Prompt ein und wiederholen Sie die in diesem Kapitel besprochenen
Aspekte.

\end{enumerate}
}

%\section*{Literaturhinweise}

\furtherreading{
\citet[Kapitel~2]{ps2} setzen sich ausführlich mit Kongruenzphänomenen auseinander und diskutieren auch alternative
Ansätze. In diesem Kapitel wurde semantische Kongruenz (mit Bezug auf Index"=Merkmale) und
syntaktische Kongruenz (mit Bezug auf Kopfmerkmale) vorgestellt. Eine Arbeit, die sich detailliert
mit den verschiedenen Kongruenzformen beschäftigt, ist \citew{Kathol99b}. 
Ein gutes Buch zur Analyse von Kongruenz im Rahmen der HPSG ist
\citew{WZ2003a}. \citet{Wechsler2024a} fasst den Forschungsstand zusammen.
}