%% -*- coding:utf-8 -*-
%%%%%%%%%%%%%%%%%%%%%%%%%%%%%%%%%%%%%%%%%%%%%%%%%%%%%%%%%
%%   $RCSfile: 1-hpsg-einleitung.tex,v $
%%  $Revision: 1.19 $
%%      $Date: 2008/09/30 09:14:41 $
%%     Author: Stefan Mueller (CL Uni-Bremen)
%%    Purpose: 
%%   Language: LaTeX
%%%%%%%%%%%%%%%%%%%%%%%%%%%%%%%%%%%%%%%%%%%%%%%%%%%%%%%%%


\chapter{Einleitung}

In diesem Kapitel soll erklärt werden, warum man sich überhaupt mit Syntax
beschäftigt (Abschnitt~\ref{sec-wozu-syntax}) und 
warum es sinnvoll ist, die Erkenntnisse zu formalisieren (Abschnitt~\ref{sec-formal}).
Einige Grundbegriffe werden in den Abschnitten~\ref{konstituententests}--\ref{sec-intro-arg-adj} eingeführt. 
Die Abschnitte~\ref{sec-grammatikmodelle}--\ref{sec-xbar}
beschäftigen sich mit Vorgängertheorien, die die HPSG beeinflusst haben.
Im Abschnitt~\ref{sec-grundlegendes} werden einige Eigenschaften der HPSG
vorgestellt, insbesondere die Modellierung sprachlicher Äußerungen mittels
Merkmalstrukturen. Dieser Abschnitt dient zur Überleitung ins nächste Kapitel,
in dem dann der Formalismus der Merkmalstrukturen vorgestellt wird.

\section{Wozu Syntax?}
\label{sec-wozu-syntax}

Die sprachlichen Ausdrücke, die wir verwenden, haben eine Bedeutung. Es handelt
sich um sogenannte Form"=Bedeutungspaare \citep{Saussure16a-de}. Dem Wort \emph{Baum}
mit seiner bestimmten orthographischen Form oder einer entsprechenden Aussprache
wird die Bedeutung \emph{baum}$'$ zugeordnet. Aus kleineren sprachlichen Einheiten
können größere gebildet werden: Wörter können zu Wortgruppen verbunden werden
und diese zu Sätzen.

Die Frage, die sich nun stellt, ist folgende: Braucht man ein formales System, das
diesen Sätzen eine Struktur zuordnet? Würde es nicht ausreichen, so wie wir
für \emph{Baum} ein Form"=Bedeutungspaar haben, entsprechende Form"=Bedeutungspaare
für Sätze aufzuschreiben? Das wäre im Prinzip möglich, wenn eine Sprache eine
endliche Aufzählung von Wortfolgen wäre. Nimmt man an,
dass es eine maximale Satzlänge und eine maximale Wortlänge und somit eine endliche
Anzahl von Wörtern gibt, so ist die Anzahl der bildbaren Sätze endlich.
Allerdings ist die Zahl der bildbaren Sätze selbst bei Begrenzung der Satzlänge riesig.
Und die Frage, die man dann beantworten muss, ist: Was ist die Maximallänge für
Sätze? Zum Beispiel kann man die Sätze in (\mex{1}) verlängern:
\eal
\ex Dieser Satz geht weiter und weiter und weiter und weiter \ldots
\ex {}[Ein Satz ist ein Satz] ist ein Satz.
\zl
In (\mex{0}b) wird etwas über die Wortgruppe \emph{ein Satz ist ein Satz} ausgesagt, nämlich dass sie
ein Satz ist. Genau dasselbe kann man natürlich auch vom gesamten Satz (\mex{0}b) behaupten und den Satz
entsprechend um \emph{ist ein Satz} erweitern.

Durch Erweiterungen wie die in (\mex{0}) können wir enorm lange und komplexe Sätze bilden.\footnote{
Manchmal wird behauptet, dass wir in der Lage wären, unendlich lange Sätze zu bilden (\citealp*[\page
 117]{NKN2001a}; \citealp[\page 3]{KS2008a-u}; Dan Everett in \citew{OW2012a} bei 25:19;
 \citealp*[\page 67]{Chesi2015a-u}; \citealp[\page 5]{Lin2017a}; \citealp[\page 2]{Martorell2018a};
 \href{https://en.wikipedia.org/wiki/Biolinguistics\#Minimalist_Program}{Wikipedia-Eintrag für
   Biolinguistics/Minimalism}, 2019-10-17)
% was there till 2019-11-03 fixed on 2019-11-09 
% https://en.wikipedia.org/w/index.php?title=Biolinguistics&oldid=924370751
% "This mechanism allows people to combine words into infinite strings."
 oder dass Chomsky so etwas behauptet hätte \citep[\page 341]{Leiss2003a}.

Das ist nicht richtig, da jeder Satz irgendwann einmal enden
 muss. Auch in der Theorie der formalen Sprachen in der Chomskyschen Tradition gibt es keinen
 unendlich langen Satz, vielmehr wird von bestimmten formalen Grammatiken eine Menge mit unendlich
 vielen endlichen Sätzen beschrieben (\citealp[\page 13]{Chomsky57a}). Zu Rekursion\is{Rekursion} in
 der Grammatik und Behauptungen zur Unendlichkeit unserer Sprache siehe auch \citew{PS2010a} und
 \citew[Abschnitt~11.1.1.8]{MuellerGTBuch1}.
} Die Festsetzung einer Grenze, bis zu der solche Kombinationen zu unserer Sprache gehören, wäre
willkürlich (\citealp[\page 208]{Harris57a}; \citealp[\page 23]{Chomsky57a}). Auch ist die Annahme, dass
solche komplexen Sätze als Gesamtheit in unserem Gehirn gespeichert sind, unplausibel. Man kann für
hochfrequente Muster bzw.\ idiomatische Kombinationen mit psycholinguistischen Experimenten zeigen,
dass sie als ganze Einheit gespeichert sind, das ist für Sätze wie die in (\mex{0}) jedoch nicht der
Fall. Es muss also eine Strukturierung der Äußerungen, es muss bestimmte wiederkehrende Muster
geben. Solche Muster aufzudecken, zu beschreiben und zu erklären, ist die Aufgabe der Syntax.

Wenn man nicht annehmen will, dass Sprache nur eine Liste von Form"=Bedeutungspaaren
ist, dann muss es ein Verfahren geben, die Bedeutung komplexer Äußerungen aus
den Bedeutungen der Bestandteile der Äußerungen zu ermitteln.
Die Syntax sagt etwas über die Art und Weise der Kombination der beteiligten
Wörter aus, etwas über die Struktur einer Äußerung.
So hilft uns zum Beispiel das Wissen über Subjekt"=Verb"=Kongruenz bei der Interpretation
der Sätze in (\mex{1}c,d):
\eal
\ex Die Frau schläft.
\ex Die Mädchen schlafen.
\ex Die Frau kennt  die Mädchen.
\ex Die Frau kennen die Mädchen.
\zl
Die Sätze in (\mex{0}a,b) zeigen, dass ein Subjekt im Singular bzw.\ Plural
ein entsprechend flektiertes Verb braucht. In (\mex{0}a,b) verlangt das Verb nur ein
Argument, so dass die Funktion von \emph{die Frau} bzw.\ \emph{die Mädchen} klar ist.
In (\mex{0}c,d) verlangt das Verb zwei Argumente, und \emph{die Frau} und \emph{die Mädchen}
könnten an beiden Argumentstellen auf"|treten. Die Sätze könnten also bedeuten, dass
die Frau jemanden kennt oder dass jemand die Frau kennt. Durch die Flexion des Verbs und
Kenntnis der syntaktischen Gesetzmäßigkeiten des Deutschen weiß der Hörer
aber, dass es für (\mex{0}c,d) jeweils nur eine Lesart gibt.



\section{Warum formal?}
\label{sec-formal}

Die folgenden beiden Zitate geben eine Begründung für die Notwendigkeit
formaler Beschreibung von Sprache:  
\begin{quote}
Precisely constructed models for linguistic structure can play an
important role, both negative and positive, in the process of discovery 
itself. By pushing a precise but inadequate formulation to
an unacceptable conclusion, we can often expose the exact source
of this inadequacy and, consequently, gain a deeper understanding
of the linguistic data. More positively, a formalized theory may 
automatically provide solutions for many problems other than those
for which it was explicitly designed. Obscure and intuition-bound
notions can neither lead to absurd conclusions nor provide new and
correct ones, and hence they fail to be useful in two important respects. 
I think that some of those linguists who have questioned
the value of precise and technical development of linguistic theory
have failed to recognize the productive potential in the method
of rigorously stating a proposed theory and applying it strictly to
linguistic material with no attempt to avoid unacceptable conclusions 
by ad hoc adjustments or loose formulation.
\citep[\page5]{Chomsky57a}
\end{quote}

\begin{quote}
As is frequently pointed out but cannot be overemphasized, an important goal
of formalization in linguistics is to enable subsequent researchers to see the defects
of an analysis as clearly as its merits; only then can progress be made efficiently.
\citep[\page322]{Dowty79a}
\end{quote}
%
Wenn wir linguistische Beschreibungen formalisieren, können wir leichter
erkennen, was genau eine Analyse bedeutet. Wir können feststellen, welche
Vorhersagen sie macht, und wir können alternative Analysen ausschließen.




\section{Konstituenten}
\label{konstituententests}

Betrachtet man den Satz in (\mex{1}), so hat man das Gefühl, dass bestimmte Wörter zu einer
Einheit gehören.
\ea
Alle Studenten lesen während der vorlesungsfreien Zeit Bücher.
\z
In diesem Abschnitt sollen Tests vorgestellt werden, die Indizien für eine engere
Zusammengehörigkeit von Wörtern darstellen. Wenn von einer \emph{Wortfolge}\is{Wortfolge}
die Rede ist, ist eine beliebige linear zusammenhängende Folge von Wörtern gemeint, 
die nicht unbedingt syntaktisch oder semantisch zusammengehörig sein müssen, \zb
\emph{Studenten lesen während} in (\mex{0}). Mehrere Wörter, die eine strukturelle Einheit bilden,
werden dagegen als \emph{Wortgruppe}\is{Wortgruppe}, \emph{Konstituente}\is{Konstituente}
oder \emph{Phrase}\is{Phrase} bezeichnet. Den Trivialfall stellen immer einzelne Wörter dar, die
natürlich immer eine strukturelle Einheit aus einem einzelnen Element bilden.

\subsection{Konstituententests}

Für den Konstituentenstatus gibt es Tests, die in den folgenden Abschnitten besprochen werden.
Wie im Abschnitt~\ref{sec-status-der-ktests} gezeigt werden wird, gibt es Fälle, bei denen die
blinde Anwendung der Tests zu unerwünschten Resultaten führt.

\subsubsection{Substituierbarkeit}

Kann man eine Wortfolge %einer bestimmten Kategorie 
in einem Satz gegen eine andere Wortfolge so austauschen,\is{Austauschbarkeit}\is{Substituierbarkeit} dass
wieder ein akzeptabler Satz entsteht, so ist das ein Indiz dafür, dass 
die beiden Wortfolgen Konstituenten bilden.

In (\mex{1}) kann man \emph{den Mann} durch \emph{eine Frau} ersetzen, was ein Indiz dafür
ist, dass beide Wortfolgen Konstituenten sind.
\eal
\ex Aicke kennt [den Mann].
\ex Aicke kennt [eine Frau].
\zl

\noindent
Genauso kann man in (\mex{1}a) die Wortfolge \emph{das Buch zu lesen} durch
\emph{dem Affen den Stock zu geben} ersetzen.
\eal
\ex Aicke versucht, das Buch zu lesen.\label{ex-das-buch-zu-lesen}
\ex Aicke versucht, dem Affen den Stock zu geben.
\zl


\subsubsection{Pronominalisierungstest}

Alles,\is{Pronominalisierungstest}
worauf man sich mit einem Pronomen beziehen kann, ist eine Konstituente. 
In (\mex{1}) kann man sich \zb mit \emph{es} auf die Wortfolge \emph{das Kind} beziehen:
\eal
\ex {}[Das Kind] schläft.
\ex Es schläft.
\zl

\noindent
Auch auf Konstituenten wie \emph{das Buch zu lesen} in \pref{ex-das-buch-zu-lesen}
kann man sich mit Pronomina beziehen, wie (\mex{1}) zeigt:
\eal
\ex Aicke versucht, das Buch zu lesen.
\ex Kirby versucht das auch.
\zl


\subsubsection{Fragetest}

Was\is{Fragetest} sich erfragen lässt, ist eine Konstituente.
        \eal
        \ex {}[Das Kind] schläft.
        \ex Wer schläft?
        \zl
        Der Fragetest ist ein Spezialfall des Pronominalisierungstests: Man bezieht sich mit
        einem Fragepronomen auf eine Wortfolge.

Die Konstituenten wie \emph{das Buch zu lesen} in \pref{ex-das-buch-zu-lesen} kann man erfragen,
wie (\mex{1}) zeigt:
\ea
Was versucht Aicke?
\z

\subsubsection{Verschiebetest}

Wenn Wortfolgen\is{Permutationstest|(}\is{Verschiebetest|(}\is{Umstelltest|(} ohne Beeinträchtigung der Akzeptabilität des Satzes verschoben
bzw.\ umgestellt werden können, ist das ein Indiz dafür, dass sie eine Konstituente bilden.

        In (\mex{1}) sind \emph{keiner} und \emph{dieses Buch} auf verschiedene Weisen angeordnet,
        was dafür spricht, \emph{dieses} und \emph{Buch} als zusammengehörig zu betrachten.
        \eal
        \ex[]{
          weil keiner [dieses Buch] kennt
          }
        \ex[]{
          weil dieses Buch keiner kennt
          }
        \zl
Es ist jedoch nicht sinnvoll, \emph{keiner dieses} als Konstituente zu analysieren,
da die Sätze in (\mex{1}) und auch andere vorstellbare Abfolgen, die durch
Umstellung von \emph{keiner dieses} gebildet werden können, unakzeptabel sind:\footnote{
  Ich verwende folgende Markierungen für Sätze: `*'\is{*} wenn ein Satz ungrammatisch ist.
  `\#'\is{\#} wenn der Satz eine Lesart hat, die nicht der relevanten Lesart entspricht.
  `\S'\is{\S} wenn der Satz aus semantischen oder informationsstrukturellen Gründen abweichend ist,
  \zb weil das Subjekt belebt sein müsste, aber im Satz unbelebt ist, oder weil es einen
  Konflikt gibt zwischen der Anordnung der Wörter im Satz und der Markierung bekannter
  Information durch die Verwendung von Pronomina.%
}
\eal
\ex[*]{
weil Buch keiner dieses kennt
}
\ex[*]{
weil Buch kennt keiner dieses
}
\zl

\noindent
Auch Konstituenten wie \emph{das Buch zu lesen} in \pref{ex-das-buch-zu-lesen}
sind umstellbar:
\eal
\ex Aicke hat noch nicht das Buch zu lesen versucht.
\ex Aicke hat das Buch zu lesen noch nicht versucht.
\ex Aicke hat noch nicht versucht, das Buch zu lesen.
\zl
\is{Permutationstest|)}\is{Verschiebetest|)}\is{Umstelltest|)}

\subsubsection{Voranstellungstest}

Eine besondere Form der Umstellung bildet die Voranstellung. Normalerweise steht
in Aussagesätzen genau eine Konstituente vor dem finiten Verb:
\eal
\label{bsp-v2}
\ex[]{
[Alle Kinder] lesen während dieser Zeit Bücher.
}
\ex[]{
[Bücher] lesen alle Kinder während dieser Zeit.
}
\ex[*]{
[Alle Kinder] [Bücher] lesen während dieser Zeit.
}
\ex[*]{
[Bücher] [alle Kinder] lesen während dieser Zeit.
}
\zl
Die Voranstellbarkeit einer Wortfolge ist als starkes Indiz für deren Konstituentenstatus
zu werten.

Diesen Test findet man auch als Satzgliedtest in der traditionellen Grammatik. Der
Begriff der Konstituente ist jedoch allgemeiner, da er auch Fälle wie (\mex{1}) erfasst:
\eal
\ex Märchen erzählen sollte man niemandem.
\ex dass man niemandem Märchen erzählen sollte
\zl
\emph{Märchen erzählen} ist kein Satzglied, aber dennoch bilden \emph{Märchen} und
\emph{erzählen} in (\mex{0}a) eine strukturelle Einheit, also eine Konstituente.
  

\subsubsection{Koordinationstest}

Lassen\is{Koordination!-stest} sich Wortfolgen koordinieren, so ist das ein Indiz dafür, dass die koordinierten Wortfolgen
jeweils Konstituenten sind.

In (\mex{1}) werden \emph{die Kinder} und \emph{die Eltern} koordinativ verknüpft.
Die gesamte Koordination ist dann das Subjekt von \emph{lachen}.
Das ist ein Indiz dafür, dass \emph{die Kinder}
und \emph{die Eltern} Konstituenten bilden.
\ea
{}[Die Kinder] und [die Eltern] lachen.
\z
Das Beispiel in (\mex{1}) zeigt, dass sich auch Wortgruppen mit \zui koordinieren lassen:
\ea
Er hat versucht, [das Buch zu lesen] und [es dann unauffällig verschwinden zu lassen].
\z

\subsection{Bemerkungen zum Status der Tests}
\label{sec-status-der-ktests}

Es wäre schön, wenn die vorgestellten Tests immer eindeutige Ergebnisse liefern würden,
weil dadurch die empirischen Grundlagen, auf denen Theorien aufgebaut werden, klarer
wären. Leider ist dem aber nicht so. Vielmehr gibt es bei jedem der Tests Probleme,
auf die ich im Folgenden eingehen will.

\subsubsection{Expletiva}
\is{Pronomen!Expletiv-|(}

Es gibt eine besondere Klasse von Pronomina, die sogenannten Expletiva, die sich nicht
auf Dinge oder Personen beziehen, also nicht referieren. Ein Beispiel ist das \emph{es} in
(\mex{1}).
\eal
\ex[]{
Es regnet.
}
\ex[]{
Regnet es?
}
\ex[]{\label{bsp-weil-es-jetzt-regnet}
weil es jetzt regnet
}
\zl
Wie die Beispiele in (\mex{0}) zeigen, kann das \emph{es} am Satzanfang
oder nach dem Verb stehen. Es kann auch durch ein Adverb vom Verb getrennt sein.
Dies spricht dafür, \emph{es} als eigenständige Einheit zu betrachten.

Allerdings gibt es Probleme mit den Tests: Zum einen ist \emph{es} nicht
uneingeschränkt umstellbar, wie (\mex{1}a) und (\mex{2}b) zeigen.
\eal
\ex[*]{\label{bsp-weil-jetzt-es-regnet}
weil jetzt es regnet
}
\ex[]{
weil jetzt keiner wegläuft
}
\zl
\eal
\ex[]{\label{bsp-Conny-sah-es-regnen}
Conny sah es regnen.
}
\ex[*]{\label{bsp-es-sah-Conny-regnen}
Es sah Conny regnen.
}
\ex[]{
Conny sah den Hasen weglaufen.
}
\ex[]{
Den Hasen sah Conny weglaufen.
}
\zl
Im Gegensatz zum Akkusativ \emph{den Hasen} in (\mex{0}c,d) kann das Expletivum in (\mex{0}b) nicht
vorangestellt werden.

Zum anderen schlagen auch Substitutions- und Fragetest fehl:
\eal
\ex[*]{
Eine Person/sie regnet.
}
\ex[*]{
Wer/was regnet?
}
\zl
Genauso schlägt der Koordinationstest fehl:
\ea[*]{
Es und eine Person regnet/regnen.
}
\z
Dieses Fehlschlagen der Tests lässt sich leicht erklären: Schwach betonte
Pronomina wie \emph{es} stehen bevorzugt vor anderen Argumenten, direkt nach
der Konjunktion (\emph{weil} in (\ref{bsp-weil-es-jetzt-regnet})) bzw.\
direkt nach dem finiten Verb (\ref{bsp-Conny-sah-es-regnen}) (siehe \citealt[\page 570]{Abraham95a-u}). Wird, wie
in (\ref{bsp-weil-jetzt-es-regnet}), ein Element vor das Expletivum gestellt,
wird der Satz ungrammatisch. Der Grund für die Ungrammatikalität von
(\ref{bsp-es-sah-Conny-regnen}) liegt in einer generellen Abneigung des 
Akkusativ"=\emph{es} dagegen, die erste Stelle im Satz einzunehmen. Es gibt zwar Belege für
solche Muster, aber in diesen ist das \emph{es} immer referentiell \parencites[\page162]{Lenerz94a}[\page4]{GS97a}.

%        \citet[\page162]{Lenerz94a} schreibt (\ref{bsp-geld-weg}) Peter Gallmann\aimention{Peter Gallmann}
%        zu. (\ref{bsp-experten}) stammt von Santorini und ist nach \citew[\page4]{GS97a} zitiert.%
%
Dass auch der Substitutionstest und der Fragetest fehlschlagen, ist ebenfalls
nicht weiter verwunderlich, denn das \emph{es} ist nicht referentiell.
Man kann es höchstens durch ein anderes Expletivum wie \emph{das} ersetzen.
Wenn wir das Expletivum durch etwas Referentielles ersetzen, bekommen wir semantische Abweichungen.
Natürlich ist es auch nicht sinnvoll, nach etwas semantisch Leerem zu fragen oder
sich darauf mit einem Pronomen zu beziehen.
\is{Pronomen!Expletiv-|)}


Daraus folgt: Nicht alle Tests müssen positiv ausfallen, damit eine Wortfolge als Konstituente gelten kann,
\dash, die Tests stellen keine notwendige Bedingung dar.

%% \subsubsection{Pronominalisierung}

%% Der Pronominalisierungstest sagt, dass alles, worauf man sich mit einem Pronomen beziehen kann,
%% eine Konstituente ist. Betrachtet man den folgenden kleinen Text, so sieht man, dass
%% der Pronominalisierungstest nicht unproblematisch ist.
%% \ea
%% Über Mozart hat er schon viele Bücher gelesen. Sie waren alle interessant.
%% \z
%% Das Personalpronomen \emph{sie} bezieht sich auf \emph{viele Bücher über Mozart},
%% in (\mex{0}) stehen die Bestandteile von \emph{viele Bücher über Mozart} aber nicht zusammen.
%% Man kann über (\mex{1}) sagen, dass \emph{viele Bücher über Mozart} eine Konstituente
%% bilden, aber in (\mex{0}) ist ein Bestandteil dieser Wortgruppe vorangestellt worden.
%% \ea
%% Hat er schon viele Bücher über Mozart gelesen?
%% \z
%%
%% diskontinuierliche Konstituente


\subsubsection{Der Verschiebetest}

Der Verschiebetest\is{Verschiebetest} ist in Sprachen mit relativ freier Konstituentenstellung problematisch, da sich
nicht immer ohne weiteres sagen lässt, was verschoben wurde. Zum Beispiel stehen die Wörter
\emph{gestern dem Kind} in (\mex{1}) an jeweils unterschiedlichen Positionen:
\eal
\ex weil alle gestern dem Kind geholfen haben
\ex weil gestern dem Kind alle geholfen haben
\zl
Man könnte also annehmen, dass \emph{gestern} gemeinsam mit \emph{dem Kind} umgestellt wurde. Eine
alternative Erklärung für die Abfolgevarianten in (\mex{1}) liegt aber darin anzunehmen, dass
Adverbien an beliebiger Stelle im Satz stehen können und dass in (\mex{0}b) nur \emph{dem Kind} vor
\emph{alle} gestellt wurde. Man sieht auf jeden Fall, dass \emph{gestern} und \emph{dem Kind} nicht
in einer semantischen Beziehung stehen und dass man sich auch nicht mit einem Pronomen auf die
gesamte Wortfolge beziehen kann. Obwohl es so aussieht, als sei das Material zusammen umgestellt
worden, ist es also nicht sinnvoll anzunehmen, dass es sich bei \emph{gestern dem Kind} um eine
Konstituente handelt.

% Engel94a:148  Ich mag es. -> keine Umstellung möglich

% zu kompliziert
%% \eal
%% \ex Der Hund bellt oft.
%% \ex (dass) oft der Hund bellt
%% \zl



\subsubsection{Voranstellung}
\is{Vorfeldbesetzung|(}
\label{sec-konst-test-probleme-voranstellung} 

Wie bei der Diskussion von (\ref{bsp-v2}) erwähnt, steht im Deutschen normalerweise
eine Konstituente vor dem Finitum. Voranstellbarkeit vor das finite Verb wird mitunter
sogar als ausschlaggebendes Kriterium für Satzglied- bzw.\ Konstituentenstatus genannt \citep[\page
783]{Duden2005}. Als Beispiel sei hier die Definition aus \citew{Bussmann83a} aufgeführt, die in
\citew{Bussmann90a} nicht mehr enthalten ist:
\begin{quote}
\textbf{Satzgliedtest}\is{Satzglied} [Auch: Konstituententest]. Auf der $\to$ Topikalisierung
beruhendes Verfahren zur Analyse komplexer Konstituenten. Da bei Topikalisierung
jeweils nur eine Konstituente bzw.\ ein $\to$ Satzglied an den Anfang gerückt werden kann,
lassen sich komplexe Abfolgen von Konstituenten (\zb Adverbialphrasen) als
ein oder mehrere Satzglieder ausweisen; in \textit{Ein Taxi quält sich im Schrittempo
durch den Verkehr} sind \textit{im Schrittempo} und \textit{durch den Verkehr}
zwei Satzglieder, da sie beide unabhängig voneinander in Anfangsposition gerückt werden
können. \citep[\page446]{Bussmann83a}
\end{quote}

Aus dem Zitat ergeben sich die beiden folgenden Implikationen:
\begin{itemize}
\item Teile des Materials können einzeln vorangestellt werden. $\to$\\
      Das Material bildet keine Konstituente.
\item Material kann zusammen vorangestellt werden. $\to$\\
      Das Material bildet eine Konstituente.
\end{itemize}
Wie ich zeigen werde, sind beide problematisch.

Die erste ist wegen Beispielen wie (\mex{1}) problematisch:
\eal
\ex Keine Einigung erreichten Schröder und Chirac über den Abbau der Agrarsubventionen.\footnote{tagesschau, 15.10.2002, 20:00.}
\ex Über den Abbau der Agrarsubventionen erreichten Schröder und Chirac keine Einigung.
\zl
Obwohl Teile der Nominalphrase \emph{keine Einigung über den Abbau der Agrarsubventionen}
einzeln vorangestellt werden können, wollen wir die Wortfolge als eine Nominalphrase (NP) analysieren,
wenn sie wie in (\mex{1}) nicht vorangestellt ist.
\ea
Schröder und Chirac erreichten keine Einigung über den Abbau der Agrarsubventionen.
\z
Die Präpositionalphrase \emph{über den Abbau der Agrarsubventionen} hängt semantisch von \emph{Einigung}
ab (\emph{Sie einigen sich über die Agrarsubventionen.}).

Diese Wortgruppe kann auch gemeinsam vorangestellt werden:
\ea
Keine Einigung über den Abbau der Agrarsubventionen erreichten Schröder und Chirac.
\z
In theoretischen Erklärungsversuchen geht man davon aus, dass \emph{keine Einigung über den Abbau der Agrarsubventionen}
eine Konstituente ist, die unter gewissen Umständen aufgespalten\is{NP"=Aufspaltung} werden kann. In solchen Fällen können die einzelnen
Teilkonstituenten wie in (\mex{-2}) unabhängig voneinander umgestellt werden.



Die zweite Implikation ist ebenfalls problematisch, da es Sätze
wie die in (\mex{1}) gibt:
\eal
\label{bsp-mehr-vf}
\ex\label{bsp-trocken-durch-die-stadt}
{}[Trocken] [durch die Stadt] kommt man am Wochenende auch mit der BVG.\footnote{
        taz berlin, 10.07.1998, S.\,22.
      }
\ex {}[Wenig] [mit Sprachgeschichte] hat der dritte Beitrag in dieser Rubrik zu tun, [\ldots]\footnote{
  Zeitschrift für Dialektologie und Linguistik, LXIX, 3/2002, S.\,339.
}
\zl
In (\mex{0}) befinden sich mehrere Konstituenten vor dem finiten Verb, die nicht in einer syntaktischen
oder semantischen
%\NOTE{JB: intuitiv doch} 
Beziehung zueinander stehen. Was es genau heißt,
in einer syntaktischen bzw.\ semantischen Beziehung zueinander zu stehen, wird in den
Kapiteln~\ref{chap-valenz} und~\ref{chap-sem} noch genauer erklärt. Beispielhaft sei hier nur
für (\mex{0}a) gesagt, dass \emph{trocken} ein Adjektiv ist, das in (\mex{0}a) \emph{man}
als Subjekt hat und außerdem etwas über den Vorgang des Durch"=die"=Stadt"=Kommens aussagt,
sich also auf das Verb bezieht. Wie (\mex{1}b) zeigt, kann \emph{durch die Stadt} nicht
mit dem Adjektiv \emph{trocken} kombiniert werden:
\eal
\ex[]{
Man ist/bleibt trocken.
}
\ex[*]{
Man ist/bleibt trocken durch die Stadt.
}
\zl
Genauso ist \emph{durch die Stadt} eine Richtungsangabe, die syntaktisch vollständig ist
und nicht mit einem Adjektiv kombiniert werden kann:
\eal
\ex[]{
der Weg durch die Stadt
}
\ex[*]{
der Weg trocken durch die Stadt
}
\zl
Das Adjektiv \emph{trocken} hat also weder syntaktisch noch semantisch etwas mit 
der Präpositionalphrase \emph{durch die Stadt} zu tun. Beide Phrasen haben jedoch gemeinsam,
dass sie sich auf das Verb beziehen bzw.\ von diesem abhängen.

Man mag dazu neigen, die Beispiele in (\ref{bsp-mehr-vf}) als Ausnahmen
abzutun. Das ist jedoch nicht gerechtfertigt, wie wir in breit angelegten empirischen
Studien gezeigt haben \parencites{Mueller2003b}[Abschnitt~3.1]{MuellerGS}{Bildhauer2011a}.

Würde man \emph{trocken durch die Stadt} aufgrund des Testergebnisses als Konstituente bezeichnen
und annehmen, dass \emph{trocken durch die Stadt} wegen der Existenz von (\ref{bsp-trocken-durch-die-stadt})
auch in (\mex{1}) als Konstituente zu behandeln ist, 
wäre der Konstituentenbegriff entwertet, da man mit den Konstituententests ja gerade semantisch
bzw.\ syntaktisch zusammengehörige Wortfolgen ermitteln will.%\NOTE{Johannes Bubenzer: Unverständlich! Wieso?}
\footnote{
  Die Daten kann man mit einem leeren verbalen Kopf\is{Spur!Verb-}\is{leere Kategorie} vor dem finiten Verb analysieren,
  so dass letztendlich wieder genau eine Konstituente vor dem Finitum steht \citep{Mueller2005d}.
  Trotzdem sind die Daten für Konstituententests problematisch, da die Konstituententests
  ja entwickelt wurden, um zu bestimmen, ob \zb \emph{trocken} und \emph{durch die Stadt}
  bzw.\ \emph{wenig} und \emph{mit Sprachgeschichte} in (\mex{1}) Konstituenten bilden. Ein
  Satzglied ist \emph{trocken durch die Stadt} auf keinen Fall.%
}

\eal
\ex Man kommt am Wochenende auch mit der BVG trocken durch die Stadt.
\ex Der dritte Beitrag in dieser Rubrik hat wenig mit Sprachgeschichte zu tun.
\zl
Voranstellbarkeit ist also nicht hinreichend für den Konstituentenstatus der Wortfolge an anderen
Stellen im Satz.

Wir haben auch gesehen, dass es sinnvoll ist, Expletiva als Konstituenten zu behandeln,
obwohl diese im Akkusativ nicht voranstellbar sind (siehe auch (\ref{bsp-Conny-sah-es-regnen})):
\eal
\ex[]{
Er bringt es bis zum Professor.
}
\ex[\#]{
Es bringt er zum Professor.
} 
\zl
Es gibt weitere Elemente, die ebenfalls nicht vorangestellt werden können. Als Beispiel
seien noch die mit inhärent reflexiven Verben\is{Verb!inhärent reflexives} verwendeten Reflexivpronomina\is{Pronomen!Reflexiv-} genannt:
\eal
\ex[]{
Aicke hat sich nicht erholt.
}
\ex[*]{
Sich hat Aicke nicht erholt.
}
\zl
Daraus folgt, dass Voranstellbarkeit kein notwendiges Kriterium für den Konstituentenstatus
ist. Somit ist Voranstellbarkeit weder hinreichend noch notwendig.
\is{Vorfeldbesetzung|)}


\subsubsection{Koordination}

Koordinationsstrukturen\is{Koordination|(}\is{Koordination!-stest} wie die in (\mex{1}) sind ebenfalls problematisch:
\ea
\label{ex-gapping}
%Peter gab ihm einen Apfel und ihr eine Tomate.
Deshalb kaufte Aicke einen Esel und Conny ein Pferd.
\z
Bilden \emph{Aicke einen Esel} und \emph{Conny ein Pferd} jeweils Konstituenten?

Diese Wörter kann man nicht gemeinsam umstellen:\footnote{
  Der Bereich vor dem finiten Verb wird auch \emph{Vorfeld}\is{Feld!Vor-} genannt (siehe Kapitel~\ref{topo}).
  Scheinbar mehrfache Vorfeldbesetzung ist im Deutschen unter bestimmten
  Bedingungen möglich. Siehe dazu auch den vorigen Abschnitt, insbesondere
  die Diskussion der Beispiele in (\ref{bsp-mehr-vf}) auf Seite~\pageref{bsp-mehr-vf}.
  Das Beispiel in (\mex{1}) ist jedoch bewusst so konstruiert worden, 
  dass sich ein Subjekt mit im Vorfeld befindet, was aus Gründen, die 
  mit den informationsstrukturellen Eigenschaften solcher Vorfeldbesetzungen
  zusammenhängen, nur sehr eingeschränkt möglich ist \citep[\page 118]{MBC2012a}. Siehe auch \citew{dKM2003a} zu Subjekten
  in vorangestellten Verbalphrasen.%
}
\ea[*]{
Aicke einen Esel kaufte deshalb.
}
\z

Eine Ersetzung durch Pronomina ist nicht ohne Ellipse möglich:
\eal
\ex[\#]{
Deshalb kaufte sie.
}
\ex[*]{
Deshalb kaufte ihn.
}
\zl
Die Pronomina stehen nicht für die zwei logischen Argumente von \emph{kaufen}, die
in (\ref{ex-gapping})  \zb durch \emph{der Mann} und \emph{einen Esel} realisiert sind,
sondern nur für jeweils eins.
\is{Koordination|)}

Daraus folgt: Auch wenn einige Tests erfüllt sind, muss es noch lange nicht sinnvoll sein,
eine Wortfolge als Konstituente einzustufen, \dash, die Tests stellen keine hinreichende
Bedingung dar. Auch hier gilt wie bei der scheinbar mehrfachen Vorfeldbesetzung im vorigen Abschnitt: Es kann durchaus
sinnvoll sein, für \emph{Aicke einen Esel} in (\ref{ex-gapping}) Konstituentenstatus
anzunehmen. Daraus kann man aber für den Konstituentenstatus in (\mex{1}) keine Schlüsse ziehen:
\ea
Deshalb kaufte Aicke einen Esel.
\z


Zusammenfassend kann man sagen, dass die Konstituententests, wenn man sie ohne Wenn und Aber
anwendet, nur Indizien liefern. Ist man sich der erwähnten problematischen Fälle bewusst,
kann man mit den Tests aber doch einigermaßen klare Vorstellungen davon bekommen, welche
Wörter als Einheit analysiert werden sollten.

\section{Köpfe}
\is{Kopf|(}

Der Kopf"=Begriff spielt in der Kopfgesteuerten Phrasenstrukturgrammatik
eine wichtige Rolle, wie man unschwer am Namen der Theorie erkennen kann.

Der Kopf einer Wortgruppe/""Konstituente/""Phrase/""Projektion ist dasjenige Element,
das die wichtigsten Eigenschaften der Wortgruppe/""Konstituente/""Phrase/""Projektion bestimmt.
Gleichzeitig steuert der Kopf den Aufbau der Phrase, \dash, der Kopf verlangt
die Anwesenheit bestimmter anderer Elemente in seiner Phrase. Die Köpfe sind
in den folgenden Beispielen kursiv gesetzt:
\eal
\ex \emph{Schläft} der Affe?
\ex \emph{Erwartet} der Delphin den Fisch?
\ex \emph{Hilft} der Delphin diesem Kind?
\ex \emph{in} diesem Haus
\ex ein \emph{Delphin}
\zl
Die Verben bestimmen den Kasus ihrer jeweiligen Argumente (der Subjekte und Objekte).
In (\mex{0}d) bestimmt die Präposition den Kasus der Nominalphrase \emph{diesem Haus} und
leistet auch den semantischen Hauptbeitrag: Ein Ort wird beschrieben. (\mex{0}e)
ist umstritten: Es gibt sowohl Wissenschaftler, die annehmen, dass der Determinator
der Kopf ist\NOTE{\cite{Brame81a-u}} 
\parencites[\page 6]{Ajdukiewicz35a-u}{VH77a-u,Brame82a}[\page 90--92]{Hudson84a-u}{Hellan86a,Abney87a,Netter94,Netter98a}, als auch solche, die annehmen,
dass das Nomen der Kopf ist \parencites{vanLangendonck94a}[\page 49]{ps2}{Demske2001a}{Hudson2004a}{Bruening2009a}{MyPM2021a}{MuellerHeadless}.
% Dalrymple textbook
Ich schließe mich für den Rest des Buches den letzteren an. Argumente für den Kopfstatus des Nomens
finden sich in Kapitel~\ref{sec-dp-analyse}, eine länger Diskussion findet sich in \citew{MyPM2021a}.

Die Kombination eines Kopfes mit einer anderen Konstituente wird \emph{Projektion
des Kopfes}\is{Projektion} genannt. Eine Projektion, die alle notwendigen Bestandteile zur Bildung
einer vollständigen Phrase enthält, wird \emph{Maximalprojektion}\is{Projektion!Maximal-}
genannt. Ein \emph{Satz}\is{Satz} ist die Maximalprojektion eines finiten Verbs.

%% Die Hauptkategorien und die Merkmale, die auf der phrasalen Ebene relevant sind, sind
%% in der Tabelle~\vref{tab-merkmale} dargestellt.
Beispiele für Kategorien und die Merkmale, die auf der phrasalen Ebene relevant sind, sind
in der Tabelle~\vref{tab-merkmale} dargestellt.
\begin{table}
\oneline{\begin{tabular}{ll}\lsptoprule
Kategorie    & projizierte Merkmale\\\midrule
Verb         & Wortart, Verbform\is{Verbform} (\textit{fin}, \textit{bse}, \textit{inf}, \textit{ppp})\\
Nomen        & Wortart, Kasus\is{Kasus} (\textit{nom}, \textit{gen}, \textit{dat}, \textit{acc})\\
             & Person\is{Person} (\emph{1}, \emph{2}, \emph{3}), Numerus (\emph{sg}, \emph{pl})\\
%Präposition  & Kategorie, (Form der Präposition (\textit{an}, \textit{auf}, \ldots), Kasus der NP)\\
Adjektiv     & Wortart, bei flektierten Formen Kasus\is{Kasus|(}, Genus\is{Genus|(} (\emph{mas}, \emph{fem}, \emph{neu}), Numerus\is{Numerus|(}\\
             & und Flexionsklasse\is{Flexion!-sklasse|(} (\emph{strong}, \emph{weak})\\\lspbottomrule
\end{tabular}}
\caption{\label{tab-merkmale}Beispiele für Hauptkategorien und projizierte Merkmale}
\end{table}
Dabei steht \type{fin}\istype{fin} für \emph{finit}, \type{bse}\istype{bse} für Infinitiv ohne \emph{zu}
und \textit{inf}\istype{inf} für Infinitiv mit \emph{zu}. \textit{ppp}\istype{ppp} steht für Verben
im Partizip II. Die Information über die Verbform ist \zb wichtig,
weil bestimmte Verben Projektionen mit einer bestimmten Verbform verlangen:
\eal
\label{bsp-projektion-v-merkmale}
\ex[]{
Dem Kind helfen will der Delphin nicht.
}
\ex[]{
Dem Kind geholfen hat der Delphin nicht.
}
\ex[*]{
Dem Kind geholfen will der Delphin nicht.
}
\ex[*]{
Dem Kind helfen hat der Delphin nicht.
}
\zl
\emph{wollen} verlangt immer ein Verb in der \emph{bse}"=Form bzw.\ eine Verbphrase, die ein Verb in
der \textit{bse}"=Form enthält. \emph{haben} verlangt dagegen ein Verb bzw.\ eine Verbphrase in der \textit{ppp}"=Form.
%% Bei Präpositionen muss man zwischen solchen unterscheiden, die nur zur rein formalen
%% Kennzeichnung von Argumenten dienen (\mex{1}a) und solchen, die einen eigenen
%% semantischen Beitrag leisten (\mex{1}b):
%% \eal
%% \ex Er wartet auf den Installateur.
%% \ex Er wartet auf dem Installateur.
%% \zl 
%% Für die Beschreibung der Argumentpräpositionalphrasen ist es wichtig zu wissen,
%% welche Präposition in der PP vorkommt, denn das Verb lässt nur bestimmte Präpositionalobjekte
%% zu:
%% \ea[*]{
%% Er wartet an den Installateur.
%% }
%% \z
%% Der Kasus der NP, die innerhalb

Genauso ist bei Nomina der Kasus des Nomens bzw.\ der mit dem Nomen im Kasus übereinstimmenden
Elemente wichtig: \emph{den Kindern} ist eine Dativ"=NP, die mit Verben wie \emph{helfen}, aber
nicht mit \emph{kennen} kombiniert werden kann:
\eal
\ex[]{
Wir helfen den Kindern.
}
\ex[*]{
Wir kennen den Kindern.
}
\zl
Die Eigenschaft des Nomens \emph{Kindern}, im Dativ zu stehen, ist für die ganze NP relevant.

Als drittes Beispiel sollen noch die Adjektive diskutiert werden: Adjektive bzw.\ Adjektivphrasen
müssen in Kasus, Genus, Numerus und Flexionsklasse zum Artikel und dem Nomen in der Nominalphrase
passen:
\eal
\ex[]{
ein kluger Beamter
}
\ex[]{
eine kluge Beamte
}
\ex[]{
eines klugen Beamten
}
\ex[*]{
eine kluger Beamte
}
\zl
Das Adjektiv ist morphologisch (in (\mex{0}) durch die Endungen \suffix{e}, \suffix{en} und
\suffix{er}) entsprechend markiert. Die Beispiele in (\mex{1}) zeigen, dass Adjektive im Deutschen
auch mit weiterem Material kombiniert werden, \dash komplexere Phrasen bilden können:
\eal
\ex[]{
ein dem König treuer Beamter
}
\ex[]{
eine dem König treue Beamte
}
\ex[]{
eines dem König treuen Beamten
}
\ex[*]{
eine dem König treuer Beamte
}
\zl
Damit man nun Fälle wie (\mex{0}d) ausschließen kann, muss sichergestellt sein, dass die Information
über Kasus, Genus, Numerus und Flexionsklasse genau wie bei \emph{kluger} auch bei \emph{dem König
treuer} als Eigenschaft der gesamten Phrase vermerkt ist. Die Flexion des Adjektivs bestimmt also, in welchen Kontexten
die gesamte Adjektivphrase vorkommen kann.
\is{Kopf|)}\is{Kasus|)}\is{Genus|)}\is{Numerus|)}\is{Flexion!-sklasse|)}



\section{Argumente und Adjunkte}
\label{sec-intro-arg-adj}

\is{Argument|(}\is{Adjunkt|(}
Konstituenten im Satz stehen in verschiedenartigen Beziehungen zu ihrem Kopf.
Man unterscheidet zwischen Argumenten und Adjunkten. Die syntaktischen Argumente
eines Kopfes entsprechen meistens dessen logischen Argumenten. So kann man
die Bedeutung von (\mex{1}a) in der Prädikatenlogik\is{Prädikatenlogik} als (\mex{1}b) darstellen:
\eal
\ex Aicke liebt Conny.
\ex $lieben'(Aicke', Conny')$
\zl
Die logische Repräsentation in (\mex{0}b) ähnelt der Äußerung in (\mex{0}a),
abstrahiert aber von der Stellung der Wörter und deren Flexion.
\emph{Aicke} und \emph{Conny} sind syntaktische Argumente des Verbs \emph{liebt},
und die entsprechenden Bedeutungsrepräsentationen sind Argumente der Relation $lieben'$
im logischen Sinn. 
Konstituenten, die nicht zur Kernbedeutung ihres Kopfes beitragen,
sondern darüber hinausgehende Information beisteuern, werden \emph{Adjunkte}
genannt. Ein Beispiel ist das Adverb \emph{sehr} in (\mex{1}):
\ea
Aicke liebt Conny sehr.
\z
Es sagt etwas über die Intensität der durch das Verb beschriebenen Relation aus.
Weitere Beispiele für Adjunkte sind Adjektive (\mex{1}a) und Relativsätze (\mex{1}b):
\eal
\ex\label{bsp-ein-graues-Eichhörnchen}
ein \emph{graues} Eichhörnchen
\ex das Kind, \emph{das Conny kennt}
\zl
Adjunkte haben folgende syntaktische bzw.\ semantische Eigenschaften:
\eal
\label{adj-kriterien}
\ex Adjunkte füllen keine semantische Rolle.
\ex Adjunkte sind optional.\is{Optionalität}
\ex Adjunkte sind iterierbar.\is{Iterierbarkeit}
\zl
Die Phrase in (\ref{bsp-ein-graues-Eichhörnchen}) kann man durch ein weiteres Adjunkt erweitern:
\ea
ein großes graues Eichhörnchen
\z
Sieht man von Verarbeitungsproblemen ab, die sich aufgrund zu hoher Komplexität
für menschliche Hörer/""Sprecher ergeben würden, so kann eine solche Erweiterung
mit Adjektiven beliebig oft erfolgen. Argumente können dagegen nicht mehrfach 
in einer Phrase realisiert werden:
\ea[*]{
Die Kinder die Eltern schlafen.
}
\z
Wenn der Schlafende benannt worden ist, kann man keine weitere Nominalgruppe
im Satz unterbringen, die sich auf andere Schlafende bezieht. Will man ausdrücken,
dass mehrere Personen schlafen, muss das wie in (\mex{1}) mit Hilfe von
Koordination geschehen:
\ea
Die Kinder und die Eltern schlafen.
\z
Man beachte, dass die Kriterien in (\ref{adj-kriterien}) zur Bestimmung von Adjunkten
nicht hinreichend sind, da es auch syntaktische
Argumente gibt, die keine semantische Rolle füllen (das \emph{es} in (\mex{1}a)) bzw.\ optional
sind (\emph{Pizza} in (\mex{1}b)).
\eal
\ex Es regnet.
\ex Wir essen (Pizza).
\zl

\noindent
Normalerweise legen Köpfe die syntaktischen Eigenschaften ihrer Argumente
ziemlich genau fest. So schreibt ein Verb vor, welche Kasus seine nominalen
Argumente haben können bzw.\ müssen. Zum Beispiel verlangt \emph{bezichtigen} einen Genitiv, der
Dativ ist unmöglich.
\eal
\ex[]{
Aicke bezichtigt Conny des Mordes.
}
\ex[*]{
Aicke bezichtigt Conny dem Mord.
}
\ex[]{
Aicke hilft dem Opfer.
}
\ex[*]{
Aicke hilft des Opfers.
}
\zl
Auch die Präposition von Präpositionalobjekten und der Kasus der Nominalphrase
im Präpositionalobjekt wird vorgeschrieben (siehe \citealt[\page 78]{Eisenberg94a}):
\eal
\ex[]{
Aicke denkt an die Eisenbahn.
}
\ex[\#]{
Aicke denkt an der Eisenbahn.
}
\ex[]{
Aicke hängt an der Eisenbahn.
}
\ex[*]{
Aicke hängt an die Eisenbahn.
}
\zl
Der Kasus von Nominalphrasen in modifizierenden Präpositionalphrasen hängt dagegen
von ihrer Bedeutung ab. Direktionale (also eine Richtung angebende) Präpositionalphrasen 
verlangen normalerweise eine Nominalphrase im Akkusativ (\mex{1}a), lokale (also einen Ort spezifizierende)
PPen nehmen einen Dativ zu sich (\mex{1}b):
\eal
\ex Aicke geht in die Schule / auf den Weihnachtsmarkt / unter die Brücke.
\ex Aicke schläft in der Schule / auf dem Weihnachtsmarkt / unter der Brücke.
\zl

\noindent
Einen interessanten Fall stellen Verben wie \emph{sich befinden} dar. Sie können
nicht ohne eine Ortsangabe stehen:
\ea[*]{
Wir befinden uns.
}
\z
Wie die Ortsangabe realisiert werden kann, ist aber sehr frei, weder die syntaktische
Kategorie noch die Art der Präposition %oder der Kasus (der ist aber immer Dativ ...)
in der Präpositionalphrase wird vorgeschrieben:
\ea
Wir befinden uns hier / unter der Brücke / neben dem Eingang / im Bett.
\z
Lokalangaben wie \emph{hier} oder \emph{unter der Brücke} werden im Kontext anderer Verben (\zb
\emph{schlafen}) als Adjunkte eingestuft, für Verben wie \emph{sich befinden} muss man aber wohl
annehmen, dass diesen Ortsangaben der Status eines obligatorischen syntaktischen Arguments zukommt.\footnote{
  In ähnlichem Zusammenhang wird meist das Verb \emph{wohnen} diskutiert und
  die Präpositionalphrase in (i.b) wird als valenznotwendig\is{Valenz} eingestuft 
  (Siehe \zb \citew[Kapitel~2]{Steinitz69a}, \citew[\page127]{HS73a}, \citew[\page 54]{Bierwisch88a-u-kopiert}, \citew[\page99]{Engel94}, 
\citew*[\page119]{Kaufmann95a}, \citew[\page 21]{Abraham2005a}).
%Fanselow2003b:21
Einfache Sätze mit \emph{wohnen} ohne Angabe des Ortes oder der Umstände sind meist
        schlecht. 
\eal
\ex[?]{
Aicke wohnt.
}
\ex[]{
Aicke wohnt in Bremen.
}
\ex[]{
Aicke wohnt allein.
}
\zl
Wie (ii) zeigt, ist es jedoch nicht gerechtfertigt, sie generell auszuschließen:
        \eal
        \ex Das Landgericht Bad Kreuznach wies die Vermieterklage als unbegründet zurück,
            die Mieterfamilie kann wohnen bleiben. (Mieterzeitung 6/2001, S.\,14)
        \ex Die Bevölkerungszahl explodiert. Damit immer mehr Menschen wohnen können, wächst Hongkong, die Stadt, 
            und nimmt sich ihr Terrain ohne zu fragen.  (taz, 31.07.2002, S.\,25)
        \ex Wohnst Du noch, oder lebst Du schon? (strassen|feger, Obdachlosenzeitung Berlin, 01/2008,
        S.\,3)%(IKEA-Werbung, Anfang 2003)
\ex Wer wohnt, verbraucht Energie -- zumindest normalerweise. (taz, berlin, 15.12.2009, S.\,23)
\ex Ich will doch nur wohnen! (ZDF, 15.08.2024, 22:35)
        \zl
Wenn man Sätze ohne Modifikator aber nicht ausschließen will, dann wäre
die Präpositionalphrase in (i.b) ein optionaler Modifikator, der trotzdem
zu den Argumenten des Verbs gezählt wird. Das scheint wenig sinnvoll. \emph{wohnen}
sollte also einfach als intransitives Verb behandelt werden.

(i.a) dürfte deswegen schlecht sein, weil der Nutzen einer solchen Äußerung gering ist,
denn normalerweise wohnt jeder Mensch irgendwo (manche leider unter der Brücke).
In (ii.a) ist klar, dass die Familie in der gemieteten Wohnung lebt. Der Ort des Wohnens muss nicht
genannt werden, da klar ist, dass es sich um eine Mietwohnung handelt. Wichtig ist für (ii.a) nur, 
ob die Familie die Wohnung weiter benutzen kann oder nicht.
Genauso ist in (ii.b) die Tatsache des Wohnens und nicht der Ort wichtig.

Siehe auch \citew{GA2001a} zu Adjunkten, die in bestimmten Kontexten aus pragmatischen\is{Pragmatik} 
Gründen obligatorisch sind.%
}
Das Verb selegiert eine Ortsangabe, stellt aber keine Anforderungen an die syntaktische
Kategorie. Die Ortsangaben verhalten sich semantisch wie die anderen Adjunkte, die wir bereits
kennengelernt haben. Wenn ich nur die semantischen Aspekte einer Kombination von Adjunkt und Kopf
betrachte, nenne ich das Adjunkt auch \emph{Modifikator}.\is{Modifikator} Unter den Begriff Modifikator lassen sich
auch die Ortsangaben bei \emph{befinden} fassen.
Modifikatoren sind normalerweise Adjunkte, \dash optional\is{Optionalität}, in Fällen wie denen mit \emph{befinden} aber auch
(obligatorische) Argumente.

Argumente werden in der HPSG in Subjekte\is{Subjekt} und Komplemente\is{Komplement}
unterteilt. Nicht alle Köpfe müssen ein Subjekt haben (siehe Kapitel~\ref{sec-subjekt-valenz}),
so dass die Menge der Argumente eines Kopfes durchaus auch der Menge der Komplemente eines Kopfes entsprechen
kann. Die Begriffe werden aber nicht synonym verwendet, wie das in anderen Schulen
der Linguistik der Fall ist.
% \item Genitive?
% \eal
% \ex der Mantel des Versicherungsvertreters
% \ex der Mantel des Vaters des Versicherungsvertreters
% \zl
% \ea
% Ereignisse dieser Art der letzten Jahre \citep[\page257]{Heringer73a}
% \z
\is{Argument|)}\is{Adjunkt|)}



\section{Verschiedene Grammatikmodelle}
\label{sec-grammatikmodelle}

Im Abschnitt~\ref{konstituententests} wurden Konstituententests vorgestellt. Mit Hilfe dieser Tests kann
man Wortfolgen in Teile zerlegen. Linguistische Theorien weisen den Teilen eine bestimmte
Struktur zu. Welche Aspekte dabei im Mittelpunkt stehen und welche Strukturen letztendlich
angenommen werden, unterscheidet sich dabei unter Umständen sehr stark von Theorie zu Theorie.

Die folgende Liste ist eine Aufzählung einiger Theorien oder Frameworks mit dazugehörigen
Publikationen. Wenn es größere Arbeiten zum Deutschen/""auf Deutsch gibt, sind diese ebenfalls
aufgeführt.
\begin{itemize}
\item Dependenzgrammatik (DG)\is{Dependenzgrammatik (DG)}\\\citep{Kern1884a-u,Tesniere59a-u,Tesniere2015a-u,HB69a-u,HB98a,Kunze75a-u,Hudson90a-u,Weber92a,Heringer96a-u}\nocite{Tesniere80a-u}
\item Kategorialgrammatik (CG)\is{Kategorialgrammatik (CG)}\\
\citep*{Ajdukiewicz35a-u,Montague74b-u,Dowty79a,BFZ97a,Steedman2000a-u}
\item Phrasenstrukturgrammatik\is{Phrasenstrukturgrammatik} (PSG)\nocite{Bloomfield33a-u}\\
      \citep*{BHPS61a} % Bar-Hillel hat auch Bußmann zietiert
\item Transformationsgrammatik\is{Transformation} und deren Nachfolger
      \begin{itemize}
      \item Transformationsgrammatik\\\citep{Chomsky57a,Bierwisch63a}
      \item Government \& Binding\is{Government and Binding (GB)@\textit{Government and Binding} (GB)}\\\citep{Chomsky81a,SS88a,Grewendorf88a}
      \item Minimalismus\is{Minimalistisches Programm}\\\citep{Chomsky95a-u,Grewendorf2002a}
      \end{itemize}
\item Relational Grammar (RG)\\
      \citep{Perlmutter83a-ed}
\item Tree Adjoining Grammar\is{Tree Adjoining Grammar@\emph{Tree Adjoining Grammar} (TAG)}\\
      \citep*{JLT75a-u,Joshi87a-u,JS97a}      
\item Generalisierte Phrasenstrukturgrammatik (\gpsg)\\
      \citep*{GKPS85a}
\item Lexikalisch Funktionale Grammatik (LFG)\is{Lexical Functional Grammar@\emph{Lexical Functional Grammar} (LFG)}\\
      \citep{Bresnan82a-ed,Bresnan2001a,BF96a-ed,Berman2003a}
\item Head-Driven Phrase Structure Grammar (HPSG)\\\citep*{ps,ps2,Kiss95b,Sag97a,Mueller99a,Mueller2002b,HPSGHandbook}
\item Konstruktionsgrammatik (CxG)\is{Konstruktionsgrammatik (CxG)}\\\citep*{FKoC88a,KF99a,Goldberg95a,Goldberg2006a,FS2006a-ed}
\end{itemize}

\noindent
Das HPSG-Handbuch \citep{HPSGHandbook} enthält auch Kapitel, die HPSG mit vielen der hier genannten
Theorien vergleichen. Eine ausführliche Einführung in verschiedene Theorien gibt
\citet{MuellerGT-Eng}. Von diesem Buch gibt es eine frühere deutsche Version, in der aber die Kapitel
über Minimalismus und Dependenzgrammatik noch nicht enthalten sind \citep{MuellerGTBuch}.

Sieht man vom Minimalismus ab, so finden sich Einsichten aus all diesen Grammatiktheorien
in den Analysen, die ich im Folgenden vorstellen werde.\footnote{
  Zu einer kurzen Beschreibung der einzelnen Frameworks und einer historischen Einordnung
  siehe auch \citew*[Anhang B]{SWB2003a}.%
}
Den Ausgangspunkt für die Motivation
der komplexen Strukturen, die in der HPSG angenommen werden, bilden Phrasenstrukturen, wie sie
in der Phrasenstrukturgrammatik verwendet werden. Phrasenstrukturgrammatiken sind Gegenstand des
nächsten Abschnitts.


\section{Phrasenstrukturgrammatiken}
\label{sec-psg}
\is{Phrasenstrukturgrammatik|(}

Wörter können anhand ihrer Flexionseigenschaften und ihrer Distribution einer Wortart zugeordnet werden.
So ist \emph{weil} in (\mex{1}) eine Konjunktion, \emph{dem} und \emph {den}
sind Artikel und werden zu den Determinierern gezählt. \emph{Affen} und \emph{Stock} sind Nomina und \emph{gibt}
ist ein Verb.
\ea
\label{bsp-weil-Aicke-dem-Affen-den-Stock-gibt}
weil Aicke dem Affen den Stock gibt
\z
Mit den in Abschnitt~\ref{konstituententests} eingeführten Tests kann man nun feststellen, dass
die einzelnen Wörter sowie die Wortgruppen \emph{dem Affen} und \emph{den Stock} Konstituenten bilden.
Diesen sollen Symbole zugewiesen werden. Da das Nomen ein wesentlicher Bestandteil der Wortgruppen
\emph{dem Affen} und \emph{den Stock} ist, nennt man diese Wortgruppen Nominalphrasen, was mit NP abgekürzt wird. Das Pronomen \emph{er}
kann an denselben Stellen stehen wie volle Nominalphrasen, weshalb man das Pronomen auch der
Kategorie NP zuordnen kann.

Phrasenstrukturgrammatiken geben Regeln vor, die etwas darüber aussagen, welche Symbole Wörtern
zugeordnet werden und wie sich komplexere Einheiten zusammensetzen. Eine einfache Phrasenstrukturgrammatik,
mit der man (\mex{0}) analysieren kann, ist in (\mex{1}) zu sehen:\footnote{
  Die Konjunktion \emph{weil} wird vorerst ignoriert. Da die Behandlung von Sätzen mit Verberst- oder
  Verbzweitstellung weitere Überlegungen voraussetzt, werden hier nur Verbletztsätze besprochen.
  Zu den anderen Verbstellungen siehe Kapitel~\ref{sec-v1} und~\ref{chap-nla}.%
}
\ea
\label{bsp-grammatik-psg}
\begin{tabular}[t]{@{}l@{ }l}
{NP} & {$\to$ D, N}\\          
{S}  & {$\to$ NP, NP, NP, V}
\end{tabular}\hspace{2cm}%
\begin{tabular}[t]{@{}l@{ }l}
{NP} & {$\to$ Aicke}\\
{D}  & {$\to$ den}\\
{D}  & {$\to$ dem}\\
\end{tabular}\hspace{8mm}
\begin{tabular}[t]{@{}l@{ }l}
{N} & {$\to$ Affen}\\
{N} & {$\to$ Stock}\\
{V} & {$\to$ gibt}\\
\end{tabular}
\z
Dabei kann man eine Regel wie NP $\to$\is{$\to$} D, N so verstehen, dass eine Nominalphrase -- also etwas,
dem das Symbol NP zugeordnet wird -- aus einem Determinator (D) und einem Nomen (N) bestehen kann.

Man kann den Satz in (\mex{-1}) mit der Grammatik in (\mex{0}) zum Beispiel auf die folgende Art und Weise analysieren:
Man nimmt das erste Wort im Satz und überprüft, ob es eine Regel gibt, auf deren rechter Regelseite
das Wort vorkommt. Wenn dem so ist, ersetzt man das Wort durch das Symbol in der linken Regelseite.
Das geschieht in den Zeilen 2--4, 6--7 und 9 der Ableitung in (\mex{1}).
Wenn es zwei oder mehrere Symbole gibt, die in einer rechten Regelseite gemeinsam vorkommen, dann
kann man diese durch das jeweilige Symbol in der linken Seite der Regel ersetzen. Das passiert in
den Zeilen 5, 8 und 10. 
%\begin{figure}
\ea
\label{bsp-anwendung-grammatik}
\begin{tabular}[t]{@{}r|l@{~~~}l@{~~~}l@{~~~}l@{~~~}l@{~~~}l@{\hspace{1cm}}l}
 & \multicolumn{6}{l}{Wörter und Symbole} & angewendete Regeln\\\hline
 1 & Aicke         & dem          & Affen          & den          & Stock & gibt                \\
 2 & {NP}          & dem          & Affen          & den          & Stock & gibt & {NP $\to$ er}  \\
 3 & NP            & {D}          & Affen          & den          & Stock & gibt & {D $\to$ dem}  \\
 4 & NP            & D            & {N}           & den          & Stock & gibt & {N $\to$ Affen} \\
 5 & NP            &              & {NP}          & den          & Stock & gibt & {NP $\to$ D, N}\\
 6 & NP            &              & NP            & {D}          & Stock & gibt & {D $\to$ den}  \\
 7 & NP            &              & NP            & D            & {N}  & gibt & {N $\to$ Stock} \\
 8 & NP            &              & NP            &              & {NP} & gibt & {NP $\to$ D, N}\\
 9 & NP            &              & NP            &              & NP   & {V} & {V $\to$ gibt}  \\
10 &               &              &               &              &      & {S} & {S $\to$ NP, NP, NP, V}\\
\end{tabular}
\z
%\vspace{-\baselineskip}\end{figure}
In (\mex{0}) bin ich von einer Folge von Wörtern ausgegangen und haben gezeigt, dass man die Regeln
der Phrasenstrukturgrammatik so anwenden kann, dass ein Satzsymbol abgeleitet werden kann. Genauso gut
hätte man die Schritte in umgekehrter Reihenfolge durchführen können, dann hätte man aus dem
Satzsymbol mit den Schritten~9--1 eine Wortfolge abgeleitet. Durch andere Wahl von Ersetzungsregeln
kann man von S ausgehend \ua auch die Wortfolge \emph{Aicke den Stock dem Affen gibt} ableiten. Man sagt,
die Grammatik erzeugt, lizenziert bzw.\ generiert eine Menge von Sätzen.

Die Ableitung in (\mex{0}) kann auch als Baum dargestellt werden. Das zeigt Abbildung~\vref{fig-Aicke-dem-Affen-den-Stock-gibt-flat}.
\begin{figure}[htbp]
\centerline{
\begin{forest}
sm edges
[S
  [NP [Aicke] ]
  [NP
    [Det [dem] ]
    [N [Affen] ] 
  ]
  [NP
    [Det [den] ]
    [N [Stock] ] 
  ]
  [V [gibt] ]
]
\end{forest}
}
\caption{\label{fig-Aicke-dem-Affen-den-Stock-gibt-flat}Analyse von \emph{Aicke dem Affen den Stock gibt}}
\end{figure}
Die Symbole im Baum werden \emph{Knoten}\is{Knoten} genannt. Man sagt, dass S die NP"=Knoten
und den V"=Knoten \emph{unmittelbar dominiert}\is{Dominanz}. Die anderen Knoten im Baum werden von S ebenfalls dominiert,
aber nicht unmittelbar dominiert\is{Dominanz!unmittelbare}. Will man über Beziehungen von Knoten zueinander reden,
verwendet man Verwandschaftsbezeichnungen. So ist in Abbildung~\ref{fig-Aicke-dem-Affen-den-Stock-gibt-flat}
S der \emph{Mutterknoten}\is{Knoten!Mutter-} der drei NP"=Knoten und des V"=Knotens. 
Die NP"=Knoten und V sind \emph{Schwestern}\is{Knoten!Schwester-}, da sie denselben Mutterknoten haben.
Hat ein Knoten zwei Töchter\is{Knoten!Tochter-} liegt eine \emph{binär\is{binär} verzweigende Struktur}\is{Verzweigung}\is{Verzweigung!binäre} vor.
% stimmt nicht, das ist relativ
%hat er mehr als zwei Töchter, spricht man auch von \emph{flacher Verzweigung}. 
Gibt es genau eine Tochter, spricht man von einer \emph{unären\is{unär}\is{Verzweigung!unäre} Verzweigung}. Zwei Konstituenten sind 
\emph{adjazent}\is{Adjazenz}, wenn sie direkt nebeneinander stehen.

In vielen linguistischen Publikationen werden Phrasenstrukturregeln nicht mehr angegeben. Es werden
nur Baumrepräsentationen oder äquivalente, kompakter darstellbare Klammerausdrücke wie \zb (\mex{1}) verwendet.
\ea
{}[\sub{S} [\sub{NP} Aicke] [\sub{NP} [\sub{D} dem] [\sub{N} Affen]]  [\sub{NP} [\sub{D} den] [\sub{N} Stock]] [\sub{V} gibt]]
\z

\noindent
Hier muss darauf hingewiesen werden, dass die Grammatik in (\ref{bsp-grammatik-psg}) nicht die einzig mögliche Grammatik
für den Beispielsatz in (\ref{bsp-weil-Aicke-dem-Affen-den-Stock-gibt}) ist. 
Es gibt unendlich\label{page-unendlich-viele-grammatiken} viele Grammatiken, 
die zur Analyse solcher Sätze verwendet werden können \citep[Übung~1]{MuellerGT-Eng5}.
Eine ist zum Beispiel die in (\mex{1}):
\ea\label{psg-binaer}
\begin{tabular}[t]{@{}l@{ }l@{}}
NP & $\to$ D, N  \\
V  & $\to$ NP, V\\
\end{tabular}\hspace{2cm}%
\begin{tabular}[t]{@{}l@{ }l}
{NP} & {$\to$ er}\\
{D}  & {$\to$ den}\\
{D}  & {$\to$ dem}\\
\end{tabular}\hspace{8mm}
\begin{tabular}[t]{@{}l@{ }l}
{N} & {$\to$ Affen}\\
{N} & {$\to$ Stock}\\
{V} & {$\to$ gibt}\\
\end{tabular}
\z
Diese Grammatik lizenziert binär verzweigende Strukturen, wie die in Abbildung~\ref{fig-Aicke-dem-Affen-den-Stock-gibt-bin}.


\begin{figure}
\centerline{
\begin{forest}
sm edges
[V
  [NP [Aicke] ]
  [V
    [NP
      [Det [dem] ]
      [N [Affen] ] ]
    [V
      [NP
        [Det [den] ]
        [N [Stock] ] ]
      [V [gibt] ] ] ] ]
\end{forest}
}
\caption{\label{fig-Aicke-dem-Affen-den-Stock-gibt-bin}Analyse von \emph{Aicke dem Affen den Stock gibt} mit binärer Verzweigung}
\end{figure}

%\noindent
Sowohl die Grammatik in (\mex{0}) als auch die in (\mex{-3}) ist zu ungenau.\footnote{
  Bei der Grammatik in (\ref{psg-binaer}) kommt noch dazu, dass man nicht sagen kann,
  wann eine Äußerung vollständig ist, da für alle Kombinationen von V und NP das Symbol
  V verwendet wird. Man braucht irgendwo in der Grammatik eine Repräsentation der Stelligkeit
  des Verbs. Siehe hierzu Kapitel~\ref{chap-valenz}. Die Strukturen, die wir in den folgenden
  Kapiteln motivieren werden, ähneln eher denen in Abbildung~\ref{fig-Aicke-dem-Affen-den-Stock-gibt-bin} als denen in Abbildung~\ref{fig-Aicke-dem-Affen-den-Stock-gibt-flat}.%
}
Nimmt man noch Lexikoneinträge für \emph{ich} und \emph{das} in die Grammatik auf, so werden
fälschlicherweise die Sätze in (\mex{1}b--d) lizenziert:
\eal
\ex[]{
Aicke dem Affen den Stock gibt
}
\ex[*]{
ich dem Affen den Stock gibt
}
\ex[*]{
Aicke dem Affen dem Stock gibt
}
\ex[*]{
Aicke dem Affen das Stock gibt
}
\zl
In (\mex{0}b) ist die Subjekt"=Verb"=Kongruenz verletzt, \emph{ich} und \emph{gibt} passen
nicht zueinander. In (\mex{0}c) sind die Kasusanforderungen des Verbs nicht erfüllt.
\emph{gibt} verlangt ein Akkusativ"=Objekt. In (\mex{0}d) ist die Determinator"=Nomen"=Kongruenz
verletzt. \emph{das} und \emph{Stock} passen nicht zueinander, da die Genus"=Eigenschaften der
beiden Wörter verschieden sind.

Im Folgenden soll überlegt werden, wie man die Grammatik verändern muss, damit die Sätze in (\mex{0}b--d)
nicht mehr erzeugt werden. Wenn wir Subjekt"=Verb"=Kongruenz erfassen wollen, müssen
wir die folgenden sechs Fälle abdecken, da im Deutschen das Verb mit dem Subjekt in Person (1, 2, 3)
und Numerus (sg, pl) übereinstimmen\is{Kongruenz} muss:
\eal\jamwidth=8cm\relax
\ex Ich schlafe.  \jam(1, sg)
\ex Du schläfst.  \jam(2, sg)
\ex Er/sie/es schläft.   \jam(3, sg)
\ex Wir schlafen. \jam(1, pl)
\ex Ihr schlaft.  \jam(2, pl)
\ex Sie schlafen. \jam(3, pl)
\zl
Die Verhältnisse können wir mit Grammatikregeln erfassen, indem wir die Menge der verwendeten
Symbole vergrößern. Statt der Regel S $\to$ NP, NP, NP, V verwenden wir dann:
\ea
\begin{tabular}[t]{@{}l@{ }l}
S  & $\to$ NP\_1\_sg, NP, NP, V\_1\_sg\\
S  & $\to$ NP\_2\_sg, NP, NP, V\_2\_sg\\
S  & $\to$ NP\_3\_sg, NP, NP, V\_3\_sg\\
S  & $\to$ NP\_1\_pl, NP, NP, V\_1\_pl\\
S  & $\to$ NP\_2\_pl, NP, NP, V\_2\_pl\\
S  & $\to$ NP\_3\_pl, NP, NP, V\_3\_pl\\
\end{tabular}
\z
Das heißt, man benötigt sechs Symbole für Nominalphrasen, sechs Symbole für Verben und 
sechs Regeln statt einer.

Um die Kasuszuweisung durch das Verb zu erfassen, kann man analog Kasusinformation in die
Symbole aufnehmen. Man erhält dann Regeln wie die folgenden:
\ea
\label{ditrans-ps-regeln}
\begin{tabular}[t]{@{}l@{ }l}
S  & $\to$ NP\_1\_sg\_nom, NP\_dat, NP\_acc, V\_1\_sg\_ditransitiv\\
S  & $\to$ NP\_2\_sg\_nom, NP\_dat, NP\_acc, V\_2\_sg\_ditransitiv\\
S  & $\to$ NP\_3\_sg\_nom, NP\_dat, NP\_acc, V\_3\_sg\_ditransitiv\\
S  & $\to$ NP\_1\_pl\_nom, NP\_dat, NP\_acc, V\_1\_pl\_ditransitiv\\
S  & $\to$ NP\_2\_pl\_nom, NP\_dat, NP\_acc, V\_2\_pl\_ditransitiv\\
S  & $\to$ NP\_3\_pl\_nom, NP\_dat, NP\_acc, V\_3\_pl\_ditransitiv\\
\end{tabular}
\z
Da wir Nominalphrasen in vier Kasus unterscheiden müssen, haben wir dann
insgesamt (3 * 2) + 3 = 9 Symbole für verschiedene NPen. Da die Verben zu den Nominalphrasen
in der Regel passen müssen, weil man \zb Verben, die drei Argumente nehmen, von solchen,
die nur zwei bzw.\ eins nehmen, unterscheiden können muss (\mex{1}), muss man entsprechend auch
die Menge der Symbole für Verben aufblähen.\is{Valenz}
\eal
\ex[]{
Aicke schläft.
}
\ex[*]{
Aicke schläft das Buch.
}
\ex[]{
Aicke kennt das Buch.
}
\ex[*]{
Aicke kennt.
}
\zl
In den obigen Regeln ist diese Information über nötige Argumente in Form der
Markierung `ditransitiv' enthalten.

Um die Determinator"=Nomen"=Kongruenz wie in (\mex{1}) zu erfassen, müssen wir Information 
über Genus (fem, mas, neu), Numerus (sg, pl) und Kasus (nom, gen, dat, akk)
in die Regeln für Nominalphrasen integrieren.
\eal
\ex der Roman, die Geschichte, das Buch (Genus)
\ex das Buch, die Bücher (Numerus)
\ex des Buches, dem Buch (Kasus)
\zl
Aus der Regel NP $\to$ D, N wird (\mex{1}):
\ea
%\resizebox{\linewidth}{!}{
\begin{tabular}[t]{@{}l@{ }l@{\hspace{4mm}}l@{ }l}
NP\_3\_sg\_nom  & $\to$ D\_fem\_sg\_nom, N\_fem\_sg\_nom \\
NP\_3\_sg\_nom  & $\to$ D\_mas\_sg\_nom, N\_mas\_sg\_nom \\
NP\_3\_sg\_nom  & $\to$ D\_neu\_sg\_nom, N\_neu\_sg\_nom \\
NP\_3\_pl\_nom  & $\to$ D\_fem\_pl\_nom, N\_fem\_pl\_nom \\
NP\_3\_pl\_nom  & $\to$ D\_mas\_pl\_nom, N\_mas\_pl\_nom \\
NP\_3\_pl\_nom  & $\to$ D\_neu\_pl\_nom, N\_neu\_pl\_nom \\[2mm]
\end{tabular}

\begin{tabular}[t]{@{}l@{ }l@{\hspace{4mm}}l@{ }l}
NP\_gen  & $\to$ D\_fem\_sg\_gen, N\_fem\_sg\_gen \\     
NP\_gen  & $\to$ D\_mas\_sg\_gen, N\_mas\_sg\_gen \\     
NP\_gen  & $\to$ D\_neu\_sg\_gen, N\_neu\_sg\_gen \\     
NP\_gen  & $\to$ D\_fem\_pl\_gen, N\_fem\_pl\_gen \\     
NP\_gen  & $\to$ D\_mas\_pl\_gen, N\_mas\_pl\_gen \\     
NP\_gen  & $\to$ D\_neu\_pl\_gen, N\_neu\_pl\_gen \\[2mm]
%\ldots & Dativ                                              \\[2mm]
%\ldots & Akkusativ\\[2mm]
\end{tabular}

\begin{tabular}[t]{@{}l@{ }l@{\hspace{4mm}}l@{ }l}
NP\_dat  & $\to$ D\_fem\_sg\_dat, N\_fem\_sg\_dat \\     
NP\_dat  & $\to$ D\_mas\_sg\_dat, N\_mas\_sg\_dat \\     
NP\_dat  & $\to$ D\_neu\_sg\_dat, N\_neu\_sg\_dat \\     
NP\_dat  & $\to$ D\_fem\_pl\_dat, N\_fem\_pl\_dat \\     
NP\_dat  & $\to$ D\_mas\_pl\_dat, N\_mas\_pl\_dat \\     
NP\_dat  & $\to$ D\_neu\_pl\_dat, N\_neu\_pl\_dat \\[2mm]
\end{tabular}

\begin{tabular}[t]{@{}l@{ }l@{\hspace{4mm}}l@{ }l}
NP\_akk  & $\to$ D\_fem\_sg\_akk, N\_fem\_sg\_akk \\     
NP\_akk  & $\to$ D\_mas\_sg\_akk, N\_mas\_sg\_akk \\     
NP\_akk  & $\to$ D\_neu\_sg\_akk, N\_neu\_sg\_akk \\     
NP\_akk  & $\to$ D\_fem\_pl\_akk, N\_fem\_pl\_akk \\     
NP\_akk  & $\to$ D\_mas\_pl\_akk, N\_mas\_pl\_akk \\     
NP\_akk  & $\to$ D\_neu\_pl\_akk, N\_neu\_pl\_akk \\[2mm]
\end{tabular}
%}
\z
Wir brauchen also 24 Symbole für Determinatoren, 24 Symbole für Nomen und 24 Regeln statt einer.


\section{Erweiterung der PSG durch Merkmale}


Phrasenstrukturgrammatiken, die nur atomare Symbole verwenden, sind problematisch,
da Generalisierungen nicht erfasst werden. Wir können zwar als Menschen erkennen, dass
das Symbol NP\_3\_sg\_nom für eine Nominalphrase steht, weil es die Buchstabenfolge NP
enthält, aber formal ist das Symbol ein Symbol wie jedes andere in der Grammatik. Die Grammatik
würde genauso funktionieren, wenn man statt NP\_akk das Symbol Z80 verwenden würde. Somit kann die
Gemeinsamkeit aller Symbole, die für NPen verwendet werden, nicht erfasst werden. 
Genauso können wir nicht erfassen, dass die Regeln in (\mex{0}) etwas gemeinsam haben.
Formal ist ihnen nur gemeinsam, dass sie ein Symbol auf der linken Regelseite und 
zwei Symbole auf der rechten Regelseite haben.

Die Lösung besteht in der Einführung von Merkmalen, die den Kategoriesymbolen
zugeordnet werden, und der Möglichkeit, die Werte solcher Merkmale in Regeln zu identifizieren.
Für das Kategoriesymbol NP kann man \zb die Merkmale für Person, Numerus und Kasus einführen:
\ea
\begin{tabular}[t]{@{}l@{ }l}
NP(3,sg,nom)  & $\to$ D(fem,sg,nom), N(fem,sg,nom)\\
NP(3,sg,nom)  & $\to$ D(mas,sg,nom), N(mas,sg,nom)\\
\end{tabular}
\z
Für Determinatoren und Nomina gibt es ein Genusmerkmal. Verwendet man statt der
Werte in (\mex{0}) Variablen, erhält man Regelschemata wie das in (\mex{1}):
\ea
\begin{tabular}[t]{@{}l@{ }l@{ }l}
NP({3},{Num},{Kas}) & $\to$ & D(Gen,{Num},{Kas}), N(Gen,{Num},{Kas})\\
\end{tabular}
\z
Die Werte der Variablen sind hierbei egal. Wichtig ist nur, dass sie übereinstimmen.
Der Wert des Personenmerkmals (erste Stelle in NP(3,Num,Kas)) ist durch die Regel auf `3' festgelegt.

Die Regeln in (\ref{ditrans-ps-regeln}) können wie in (\mex{1}) zusammengefasst werden:
\ea
\begin{tabular}[t]{@{}l@{ }l@{ }l}
S  & $\to$ & NP({Per1},{Num1},{nom}), \\
   &       & NP(Per2,Num2,{dat}),\\
   &       & NP(Per3,Num3,{akk}),\\
   &       & V({Per1},{Num1},ditransitiv)\\
\end{tabular}
\z
Durch die Identifikation von Per1 und Num1 beim Verb und beim Subjekt wird Subjekt"=Verb"=Kongruenz erzwungen.
Bei den anderen NPen sind die Werte dieser Merkmale egal. Die Kasus der NPen sind explizit festgelegt.
\is{Phrasenstrukturgrammatik|)}
%% Eine Regel wie (\mex{0}) wirft natürlich die Frage auf, ob es Regeln geben kann,
%% in denen nur der Per"=Wert oder nur der Num"=Wert identisch sein muss? Die Tatsache,
%% dass immer beide Werte gemeinsam übereinstimmen mü

%% \item Gruppierung von Information $\to$ stärkere Generalisierung, stärkere Aussage\\[2ex]

%% \begin{tabular}{@{}l@{ }l@{ }l}
%% S  & $\to$ & NP(Agr1,nom),\\
%%    &       & NP(Agr2,dat),\\
%%    &       & NP(Agr3,akk),\\
%%    &       & V(Agr1)\\\\
%% \end{tabular}

%% wobei Agr ein Merkmal mit komplexen Wert ist: \zb agr(1,sg)
%% \end{itemize}


\section{Die \xbar-Theorie}
\label{sec-xbar}
\is{X-Theorie@\xbar-Theorie|(}

Im vorigen Abschnitt wurde gezeigt, wie man sehr spezifische Phrasenstrukturregeln
zu allgemeineren Regelschemata generalisieren kann. Diese Abstraktion lässt sich
noch weitertreiben: Statt explizite Kategoriesymbole wie V, N oder P zu verwenden, kann man auch
an deren Stelle Variablen setzen. Eine solche Form der Abstraktion findet man in
der \xbar-Theorie (sprich X-bar"=Theorie) \citep{Chomsky70a,Jackendoff77}. Diese Form abstrakter Regeln
spielt in den verschiedensten Theorien eine Rolle. Beispielhaft seien
erwähnt: Government \& Binding \indexgb \citep{Chomsky81a,SS88a,Grewendorf88a},
Generalized Phrase Structure Grammar\indexgpsg \citep*{GKPS85a}
und Lexical Functional Grammar\indexlfg \citep{Bresnan82a-ed,Bresnan2001a,BF96a-ed}.

(\mex{1}) zeigt die \xbar-Schemata und eine mögliche Instantiierung, in der
für X die Kategorie N eingesetzt wurde, und Beispielwortgruppen, die mit den Regeln
abgeleitet werden können:
\ea
\(
\begin{array}[t]{@{}l@{\hspace{7mm}}l@{\hspace{7mm}}l@{}}
\xbar\mbox{-Regel} & \mbox{mit Kategorien} & \mbox{Beispiel}\\[2mm]
\overline{\overline{\mbox{X}}} \rightarrow \overline{\overline{\mbox{Spezifikator}}}~~\xbar & \overline{\overline{\mbox{N}}} \rightarrow \overline{\overline{\mbox{DET}}}~~\overline{\mbox{N}} & \mbox{das [Bild von Conny]} \\
\xbar \rightarrow \xbar~~\overline{\overline{\mbox{Adjunkt}}}             & \overline{\mbox{N}} \rightarrow \overline{\mbox{N}}~~\overline{\overline{\mbox{REL\_SATZ}}} & \mbox{[Bild von Conny] [das alle} \\
                            &                                             & \mbox{kennen]}\\
\xbar \rightarrow \overline{\overline{\mbox{Adjunkt}}}~~\xbar             & \overline{\mbox{N}} \rightarrow \overline{\overline{\mbox{ADJ}}}~~\overline{\mbox{N}} & \mbox{schöne [Bild von Conny]}\\
\xbar \rightarrow \mbox{X}~~\overline{\overline{\mbox{Komplement}}}*               & \overline{\mbox{N}} \rightarrow \mbox{N}~~\overline{\overline{\mbox{P}}} & \mbox{Bild [von Conny]}\\
\end{array}
\)
\z
Für X kann aber auch jede andere Wortart eingesetzt werden (\zb V, A oder P).
Das englische \emph{bar} bedeutet Balken. Das X ohne Balken steht in den obigen
Regeln für ein lexikalisches Element. Ein lexikalisches Element kann mit
all seinen Komplementen kombiniert werden. Der `*' in der letzten Regel
steht für beliebig viele Wiederholungen des Symbols, hinter dem er steht.
Das Ergebnis der Kombination eines lexikalischen Elements mit seinen Komplementen\is{Komplement}
ist eine neue Projektionsstufe von X: die Projektionsstufe eins, markiert durch einen
Balken. \xbar kann dann mit Adjunkten kombiniert werden. Diese können links oder
rechts von \xbar stehen. Das Ergebnis der Kombination ist wieder ein \xbar, \dash
die Projektionsstufe wird durch die Kombination mit Adjunkten\is{Adjunkt} nicht verändert.
Vollständige Projektionen, Maximalprojektionen also, werden durch zwei Balken
markiert. Für ein X mit zwei Balken schreibt man auch XP\is{XP}.
Eine XP besteht aus einem Spezifikator\is{Spezifikator} und einem \xbar. Je nach theoretischer
Ausprägung zählen Subjekte von Sätzen und Determinatoren in Nominalphrasen zu den Spezifikatoren.
In Nicht"=Kopf"=Positionen dürfen nur Maximalprojektionen vorkommen,
deshalb befinden sich über Komplement, Adjunkt und Spezifikator jeweils zwei Balken.

In der Theorie, die nun im Folgenden entwickelt wird, werden nicht alle Annahmen
der \xbart geteilt. Insbesondere die letzte Annahme wird verworfen. Dass die
hier entwickelte Theorie deshalb nicht weniger restriktiv ist als Theorien,
die sich streng an das \xbar-Schema halten, haben \citet{Pullum85a} und \citet{KP90a}
gezeigt.

Bevor wir aber zu Grammatikregeln kommen, muss ein gewisser formaler Apparat eingeführt
werden, der zur exakten Beschreibung der Grammatikregeln benötigt wird. Dies wird
im Kapitel~\ref{kap-merkmalstrukturen} geschehen. An dieser Stelle möchte
ich noch einen kurzen Überblick über HPSG geben, der auf die folgenden Kapitel einstimmen soll.
\is{X-Theorie@\xbar-Theorie|)}





\section{Einordnung der HPSG}
\label{sec-grundlegendes}

HPSG hat folgende Eigenschaften: Es ist eine lexikonbasierte Theorie,
\dash, der wesentliche Bestandteil der linguistischen Zusammenhänge
befindet sich in den Beschreibungen von Wörtern. HPSG ist zeichenbasiert
im Sinne Saussures \citeyearpar{Saussure16a-de}, \dash, Form und Bedeutung
sprachlicher Zeichen sind stets gemeinsam repräsentiert. Getypte
Merkmalstrukturen werden zur Modellierung
aller relevanten Information benutzt. Die Strukturen kann man mit Merkmalbeschreibungen wie der in
(\mex{1}) beschreiben. Lexikoneinträge, Phrasen
und Prinzipien werden mit denselben formalen Mitteln modelliert und beschrieben.
Generalisierungen über Wortklassen oder Regelschemata werden
mittels Vererbungshierarchien erfasst.
HPSG ist eine monostratale Theorie, \dash, Phonologie, Syntax
und Semantik werden in einer Struktur repräsentiert. Es gibt keine
getrennten Beschreibungsebenen wie zum Beispiel in der Government \& Binding"=Theorie.
(\mex{1}) zeigt Auszüge aus einer Repräsentation des Wortes \emph{Grammatik}:
\ea
\ms[word]{
\rnode{4}{phonology}   & \phonliste{ Grammatik } \\[1mm]
syntax-semantics \ldots & \ms[local]{ \rnode{5}{category}  & \ms[category]{ head & \ms[noun]{ case & \ibox{1}\\
                                                       }\\[6mm]
                                       comps & \liste{ Det[\textsc{case}~\ibox{1}] } \\
                                     } \\[6mm]
              \rnode{6}{content} & \ldots \ms[grammatik]{ inst & X \\
                                   }\\
            }\\
}
\z
An diesem Beispiel kann man schon erkennen, dass diese Merkmalbeschreibung
Information über die Phonologie, die syntaktische Kategorie und den Bedeutungsbeitrag
des Wortes \emph{Grammatik} enthält. Was die kursiven Wörter und die Zahlen in Boxen
bedeuten, warum die Information so wie in (\mex{0}) strukturiert ist und wie solche Lexikoneinträge
mit den Regelschemata zusammenwirken, wird in den folgenden Kapiteln gezeigt.

\section{Grundlegendes zu den Daten}
\label{sec-grundlegendes-zu-daten}

Bevor wir uns im folgenden Kapitel den formalen Details und im Anschluss daran den einzelnen
grammatischen Problemen zuwenden, soll zum Abschluss dieses einleitenden Kapitels noch einiges zum
Umgang mit Daten und zum Grammatikschreiben allgemein gesagt werden.

Es gibt verschiedene Arten, Linguistik zu betreiben, und verschiedene Ansichten darüber,
was Grammatiken leisten sollen. Die Grammatik, die in diesem Buch entwickelt wird,
modelliert einen Ausschnitt der deutschen Sprache. Dass die vorgestellte Grammatik nicht die gesamte
Sprache abdecken kann, wird man verstehen, wenn man die ersten Kapitel gelesen hat:
Analysen müssen sorgfältig motiviert werden, und es gibt komplexe Interaktionen
zwischen verschiedenen Phänomenen. Wie dieses Buch zeigt, braucht man bereits \pageref{last-page-hpsg-teil} Seiten,
um die wichtigsten Phänomene abzuhandeln. Das bedeutet aber nicht, dass der Versuch, eine explizite
Grammatik für eine Sprache zu entwickeln, ein sinnloses Unterfangen ist, wie \zb 
\citet[Kapitel~10]{Sampson2001a-u} behauptet. Sampson untersucht ein annotiertes
Korpus (eine Menge von Beispielsätzen, denen eine Struktur zugeordnet wurde)
auf Vorkommen von Nominalphrasenmustern und spekuliert, dass es bei Vergrößerung des Korpus
immer wieder Muster geben wird, die bisher nicht aufgetreten sind. Somit -- so seine Argumentation --
ist der Versuch, eine Grammatik zu schreiben, die wenige Regeln enthält und grammatische
von ungrammatischen Sätzen unterscheidet, zum Scheitern verurteilt.
Sampson geht sogar so weit zu behaupten, dass man nicht sinnvoll zwischen grammatischen und
ungrammatischen Sätzen unterscheiden kann. Sampson diskutiert den Einwand, dass die Konstruktion
[\sub{NP} Nomen Artikel] nie für eine Grammatik des Englischen angenommen würde und dass deshalb
Grammatiken des Englischen korrekt vorhersagen, dass \emph{bread the} keine englische Nominalphrase ist.
Er schreibt dazu folgendes:
\begin{quote}
But to suggest that the construction is not just very unusual but actually impossible in English
is merely a challenge to think of a plausible context for it, and in this case (as usual in such cases)
it is not at all hard to meet the challenge. There would be nothing even slightly strange, in a discussion
of foreign languages, in saying \emph{Norwegians put the article after the noun, in their language
they say things like bread the is on table the} -- an utterance which contains two examples
of Culy's `impossible construction'. Talking about foreign languages is one valid use of the English
language, among countless others. \citep[\page177]{Sampson2001a-u}
\end{quote}
Denkt man ein bisschen über diese Sichtweise nach, sieht man, wie absurd dieses Argument ist. Denn
man könnte auf diese Weise alle Sprachen der Welt zu Bestandteilen der Englischen Sprache machen.
Je nachdem, wieweit Sampson mit seiner Ansicht gehen will, wären dann \zb ganze deutsche Sätze wie in (\mex{1}a)
oder deutsche Sätze mit Wort"=für"=Wort"=Übersetzung wie in (\mex{1}b) Bestandteil des Englischen.
\eal
\ex I think that Germans say "`weil ich diesen Ansatz komisch finde"'.
\ex I think that Germans say \emph{because I this proposal funny consider}.
\zl
Genauso würden alle früheren Sprachstufen des Deutschen, alle Sprachentwicklungsdaten,
alle Sprachfehler im Deutschen und natürlich auch in anderen Sprachen zu Bestandteilen des
Englischen, denn über diese Themen wird auf Englisch publiziert.

Jemand, der Grammatiken entwickelt, müsste dann eine deutsche Grammatik mit englischen Wörtern
als Teilgrammatik des Englischen annehmen. Es ist klar, dass man, wenn man erfassen will, welche
Regeln für die Bildung von Wortgruppen in einer Sprache gelten, nicht die Regeln einer anderen
Sprache formulieren will. Dass es nicht sinnvoll ist, Metabetrachtungen zu Sprache innerhalb
einer Grammatik zu modellieren, zeigen auch die folgenden Beispiele:
\eal
\ex I believe that the sequence \emph{the the the the word leave know the that him strange the} is word salad.
\ex In German, you can not use the sequence corresponding to \emph{this because funny I consider proposal}.
\zl
Die kursiv geschriebene Folge in (\mex{0}a) ist arbiträr, und man könnte sie durch die Regeln in (\mex{1}) erzeugen:
\ea
wörter $\to$ wort wort\\
wörter $\to$ wörter wort
\z
Diese Regeln erzeugen beliebige Wortfolgen. Man kann mit der ersten Regel Folgen der Länge zwei
ableiten. Wenn man die zweite und die erste Regel gemeinsam anwendet, kann man Folgen der Länge drei
anwenden. Bei Mehrfachverwendung der zweiten Regel kann man beliebig lange Folgen erzeugen.
Die Regeln sind jedoch absolut uninteressant und 
haben nichts mit den direkt verwendeten sprachlichen Fähigkeiten des Menschen zu tun.

Sampsons Position ist eine extreme\is{Extremismus} Position, und wie so oft sind extreme Positionen als Gegenstück
zu anderen nicht minder extremen Positionen entstanden. In den Anfangstagen der generativen Grammatik ging man davon aus,
dass eine Sprache dadurch gekennzeichnet ist, dass sie durch eine endliche Menge von Ersetzungs- und
Transformationsregeln erzeugt werden kann. Ein Satz, der nicht durch die Regelmenge lizenziert ist,
gilt als nicht zu der betreffenden Sprache gehörend. Die Grammatiker, die
seit den fünfziger Jahren in den von Chomsky dominierten Richtungen gearbeitet haben, verließen
sich für die Beurteilung von Sätzen meist auf ihre Intuition. Mittels Introspektion teilten sie
Äußerungen in wohlgeformte und nicht wohlgeformte ein und versuchten dann für die wohlgeformten
Äußerungen Regeln zu formulieren. Ein solches Vorgehen war in den ersten Jahren der Theorieentwicklung
durchaus gerechtfertigt, denn bei einfachen Sätzen wie \zb Sätzen mit einem intransitiven Verb
und einem Subjekt kann man durchaus problemlos die Wohlgeformtheit des Satzes beurteilen.
Obwohl man auch für Sätze wie \emph{Er schläft.} oder \emph{Max rülpst.} leicht Korpusbelege\is{Korpus}
wie die in (\mex{1}) oder (\mex{2}) finden kann, wäre es Zeitverschwendung, danach zu suchen.
\eal
\ex Er schläft.\footnote{
  taz, 30.04.2004, S.\,3 und taz, 03.05.1997, S.\,18.
}
\ex Wenn man ihn dann beispielsweise in eine Lage bringt, wo er isoliert ist, oder man verhindert, 
    daß er schläft, dann kommt er in eine Situation, wo er nur noch hinaus will.\footnote{
  taz, 17.04.1999, S.\,3.
}
\ex Andrea Rupprecht arbeitet meist am PC, wenn er schläft oder wenn der Vater sich um ihn kümmert.\footnote{
  taz bremen,  08.03.1999, S.\,26.
}
\zl
\ea
Max rülpst.\footnote{
  Axel Hacke. \emph{Der kleine Erziehungsberater}. München: Verlag Antje Kunstmann. 1992, S.\,20--22.%
}
\z
Leider sind jedoch nicht alle Bereiche unseres Sprachvermögens der Introspektion\is{Introspektion} zugänglich, und
spätestens zu dem Zeitpunkt, zu dem die generative Grammatik den Bereich sehr simpler Sätze verlassen
hatte und mit komplexen Beispielsätzen für Strukturen zu argumentieren begann, ergaben sich Probleme.
Die negative Beurteilung von Daten wird benutzt, um bestimmte Strukturen auszuschließen. Oft kann man
jedoch zeigen, dass es durchaus Belege für die angeblich unmöglichen Strukturen gibt (siehe auch \citew{Mueller2007c,MM2009a}).
Dieser Punkt wird auch in dem obigen Zitat von Sampson angesprochen: Mitunter gibt es Äußerungskontexte, in denen
Äußerungen völlig normal erscheinen, und die jeweiligen Autoren sind nur nicht auf einen solchen gekommen.
Somit wird ein Theoriegebäude entwickelt, das leicht durch Daten zum Einsturz gebracht werden kann.
Da der Anspruch ist, eine Theorie zu entwickeln, die für alle Sprachen gleichermaßen funktioniert
und möglichst dieselben Strukturen und Prinzipien für alle Sprachen oder bestimmte Sprachgruppen
annimmt, werden Ergebnisse bei der Erforschung von Einzelsprachen auch für die Motivation
bestimmter Strukturen in anderen Sprachen verwendet. Das heißt, dass ein einzelner einflussreicher
Artikel, der auf nicht abgesicherten Behauptungen beruht, dazu führen kann, dass Arbeiten
in einem theoretischen Rahmen über Jahre oder Jahrzehnte entweder nicht ausreichend begründet 
oder sogar wirklich falsch sind.

Behauptungen in Bezug auf die Nicht"=Existenz von Strukturen lassen sich mit Korpusdaten gut widerlegen,
weshalb ich in diesem Buch an Stellen, die kontrovers diskutiert werden (oder kontrovers diskutiert
werden könnten) auf Korpusbelege zurückgreife. In Korpora findet man jedoch auch Beispiele wie die in (\mex{1}):
\ea
Studenten stürmen mit Flugblättern und Megafon die Mensa und rufen alle auf zur Vollversammlung
in der Glashalle \emph{zum kommen}. \emph{Vielen} bleibt das Essen im Mund stecken und \emph{kommen sofort mit}.\footnote{
  Streikzeitung der Universität Bremen, 04.12.2003, S.\,2. Die Markierung im Text ist von mir.%
}
\z
Solche Daten zieht man nicht zur Motivation von Theorien heran, da sie nicht wohlgeformt sind.
Sicher ist es interessant zu untersuchen, was beim Verfassen dieser Wortfolgen schief gegangen ist, aber
sie sollten nicht direkt von einer Grammatik beschrieben werden. Man muss also Korpusbelege zusätzlich
noch mit der Methode der Introspektion absichern. Misstraut man der eigenen Urteilsfähigkeit, kann
man Sprecher der jeweiligen Sprache befragen oder andere psycholinguistische Experimente
durchführen.

\begin{sloppypar}
Ein anderer Kritikpunkt Sampsons ist die Ja/""Nein"=Unterscheidung, die generative Grammatiken in Bezug
auf die Grammatikalität von Äußerungen treffen. So wird der erste Satz in (\mex{1}) als grammatisch eingestuft,
die folgenden Sätze werden dagegen gleichermaßen als ungrammatisch zurückgewiesen.
\end{sloppypar}

\eal
\ex[]{
Der Delphin hilft diesem Kind.
}
\ex[*]{
Der Delphin helfen diesem Kind.
}
\ex[*]{
Delphin der helfen diesem Kind.
}
\ex[*]{
Delphin der helfen Kind diesem.
}
\zl
Kritiker wenden hier zu Recht ein, dass man bei (\mex{0}b--d) Abstufungen in der Akzeptabilität
feststellen kann: In (\mex{0}b) gibt es keine Kongruenz zwischen Subjekt und Verb, in (\mex{0}c)
stehen zusätzlich \emph{Delphin} und \emph{der} in der falschen Reihenfolge, und in (\mex{0}d) ist \emph{Kind}
und \emph{diesem} vertauscht. Genauso verletzen die Sätze in (\mex{-1}) zwar grammatische
Regeln des Deutschen, sind aber noch interpretierbar.
Sampsons Kritik wird jedoch sofort hinfällig, wenn man Grammatik als ein System
von Wohlgeformtheitsbeschränkungen versteht. Eine Äußerung ist um so schlechter,
je mehr Wohlgeformtheitsbedingungen sie verletzt \citep[\page 26--27]{PS2001a}. Die
Wohlgeformtheitsbedingungen kann man auch wichten, 
so dass man erklären kann, warum bestimmte Verletzungen zu stärkeren Abweichungen führen als andere.
\itdopt{Quelle}
Für die Gewichtung von Beschränkungen kann man auch Korpusdaten und entsprechende Verfahren
heranziehen, so dass man von den im Bereich der Korpuslinguistik entwickelten Techniken profitiert.
In diesem Buch wird auf Gewichte nicht eingegangen, es interessiert nur, welche Strukturen prinzipiell wohlgeformt sind.






\questions{
\begin{enumerate}
\item Wodurch unterscheidet sich ein Kopf in einer Wortgruppe von Nicht"=Köpfen?
\item Gibt es in (\mex{1}) einen Kopf?
      \eal
      \ex es
      \ex Schlaf!
      \ex schnell
      \zl
\item Wodurch unterscheiden sich Argumente von Adjunkten?
\item Bestimmen Sie die Köpfe, die Adjunkte und die Argumente im Satz (\mex{1}) und in den Bestandteilen des Satzes:
  \ea
  Aicke hilft den kleinen Kindern in der Schule.
  \z
\end{enumerate}
}


\exercises{
\begin{enumerate}
\item\label{ua-psg-eins}
      Auf Seite~\pageref{page-unendlich-viele-grammatiken} habe ich behauptet, dass
      es unendlich viele Grammatiken gibt, die (\ref{bsp-weil-Aicke-dem-Affen-den-Stock-gibt})
      analysieren können.
      Überlegen Sie sich, wieso diese Behauptung richtig ist.
%%       Beweisen Sie, dass das richtig ist.
%%
%%       Hilfestellung: Eine Möglichkeit, eine Behauptung zu beweisen, besteht darin, einen indirekten
%%       Beweis zu führen. Beim indirekten Beweis geht man davon aus, dass das, was man beweisen will,
%%       falsch ist, und zeigt, dass diese Annahme zu einem Widerspruch führt.
\item Stellen Sie Überlegungen dazu an, wie man ermitteln kann, welche der unendlich vielen Grammatiken
      die beste ist bzw.\ welche der Grammatiken die besten sind.
\item Schreiben Sie eine Phrasenstrukturgrammatik, mit der man u.\,a.\ die Sätze in (\mex{1})
      analysieren kann, die die Wortfolgen in (\mex{2}) aber nicht zulässt.
      \eal
      \ex[]{
      Der Delphin hilft dem Kind.
      }
      \ex[]{
      Conny gibt ihr das Buch.
      }
      \ex[]{
      Conny wartet auf ein Wunder.
      }
%       \ex[]{
%       Er wartet neben dem Bushäuschen auf ein Wunder.
%       }
      \zl
      \eal
      \ex[*]{
        Der Delphin hilft er.
      }
      \ex[*]{
        Conny gibt ihr den Buch.
      }
      \zl 

\end{enumerate}
}

%% \section*{Literaturhinweise}


%% Die Frage, was in einer Struktur als Kopf zu gelten hat, wird innerhalb
%% verschiedener theoretischer Ansätze mitunter verschieden beantwortet.
%% Zu einem Vergleich und einer Diskussion siehe \citew{Zwicky85a}.


%      <!-- Local IspellDict: de -->

