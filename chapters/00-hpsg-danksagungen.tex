%% -*- coding:utf-8 -*-

\chapter*{Danksagungen}

Ich danke allen Studierenden der Universitäten Bremen und Potsdam, 
an denen ich das Buch ausprobieren konnte.
Felix Bildhauer,
Johannes Bubenzer, % Uni Potsdam, Mann mit Bart aus Hamburg, Script
Daniel Clerc, % Kommentare zu Übungsaufgaben
Anna Iwanow, % Uni Potsdam, Script
Katarina Klein, 
Till Kolter, 
Oleg Lichtenwald, 
Haitao Liu,
Frank Richter,
Wolfgang Seeker,
Wilko Steffens,
Ralf Vogel
und
Arne Zeschel
danke ich für Kommentare zu früheren Versionen dieses Buches.
%Felix Bildhauer, 
Olivier Bonami, % Situationssemantik
Gosse Bouma, 
Ann Copestake, 
Gisbert Fanselow, 
Kerstin Fischer, 
Dan Flickinger, 
Martin Forst,
Frederik Fouvry,
Tibor Kiss, 
Valia Kordoni, 
Detmar Meurers, 
Ivan Sag, 
Manfred Sailer,       % Idioms
Jan-Philipp Soehn,    % Idioms
Anatol Stefanowitsch,
Gertjan van Noord
und
Shravan Vasishth % Performance/Competence
%und Arne Zeschel 
danke ich für Diskussionen.

Beim Tutorial \emph{Grammar Implementation}, das im Rahmen des von der DFG geförderten Netzwerks
Cogeti (Constraintbasierte Grammatik: Empirie, Theorie und Implementierung) am Seminar für
Computerlinguistik der Universität Heidelberg durchgeführt wurde,
konnte ich noch einige kleinere Fehler in der Grammix"=CD finden. Ich danke allen Teilnehmern des Tutoriums.

Bei Felix Bildhauer und Renate Schmidt möchte ich mich für das Korrekturlesen des fast fertigen
Manuskripts bedanken. Die Fehler, die jetzt noch im Buch enthalten sind, habe ich gestern bei
letzten Änderungen reingemacht.

~\medskip

\noindent
Bremen, 12.\ Februar, 2007\hfill Stefan Müller


Ich bedanke mich bei Jürgen Bohnemeyer\aimention{J{\"u}rgen Bohnemeyer} und bei Jacob Maché\aimention{Jacob Mach{\'e}} für Diskussion bzw.\ für Kommentare zur
ersten Auf"|lage, bei Antonio Machicao y Priemer\aimention{Antonio {Machicao y Priemer}} für
Kommentare zur zweiten Auf"|lage und bei Tibor Kiss\aimention{Tibor Kiss}, Timm Lichte\aimention{Timm Lichte} und Adam
Przepiórkowski\aimention{Adam Przepiórkowski} für Kommentare zur dritten Auf"|lage. Steve Wechsler
danke ich für Diskussion zur Kongruenz.


% Timm Lichte: Falsches Verb in Erklärungen für Passivbeispiel.
% Tibor Kiss 6.5.2014: Merkmalstruktur statt beschreibung, Symbole in PSGen.
% Pied Piping gegen PSGen? To Do.
% Zehui Guo zu Typo bei Umstellung dat < acc Order. 04.05.2025
% Iman Mohsen selber Typo bei  Umstellung dat < acc Order. 13.05.2025
% Davide Baleani Typo 19.05.2025
Dankbar bin ich auch Nele Arnold, Davide Baleani, Karla Jerabeck, Zehui Guo, Anja Herrmann, Iman Mohsen and Sophie Reule für Kommentare, Hinweise auf Typos und Fragen.

Jakob Maché danke ich für Kommentare zum Semantik-Kapitel.
Antonio Machicao y Priemer und Giuseppe Varaschin danke ich für die Diskussion von semantischen Fragen.

Sašo Živanović möchte ich allerherzlichst für die Arbeit danken, die in \texttt{forest} geflossen
ist. Mit diesem Paket wurden alle Bäume und alle Typhierarchien in diesem Buch gesetzt. In einem
früheren Leben habe ich alle Bäume mit Tabellen erzeugt, weil es damals noch keine wirklich guten
\LaTeX"=Pakete zum Setzen von Bäumen gab. Sašo hat mir gezeigt, wie ich bestimmte Probleme sehr
elegant lösen kann. Er hat dann auch das \texttt{memoize}"=Paket entwickelt, mit dem man \texttt{tikz}"=Grafiken
externalisieren kann, so dass sich die Compile"=Zeit verringert. Beide Pakete sind mit dem
HPSG"=Handbuch und diesem Buch getestet und haben sich bewährt. Vielen, vielen Dank für all die Zeit,
die in diese Projekte geflossen ist.

Ebenfalls danken möchte ich allen \LaTeX"=Expert*innen, die \href{https://tex.stackexchange.com/users/18561/stefan-m%c3%bcller?tab=questions}{meine Fragen auf Stackexchange}
beantwortet haben. Oft sehr, sehr schnell, so dass ich mitunter schon in wenigen Minuten eine
Antwort hatte. Ihr seid alle großartig!

Es ist schön, dass der \href{https://github.com/stefan11/HPSG-Lehrbuch}{Code dieses Buches} offen ist, so dass auch andere von Euch lernen können.
\todostefan{URL auf github aktualisieren}


