%% -*- coding:utf-8 -*-
%%%%%%%%%%%%%%%%%%%%%%%%%%%%%%%%%%%%%%%%%%%%%%%%%%%%%%%%%
%%   $RCSfile: 3-hpsg-valenz-psg.tex,v $
%%  $Revision: 1.17 $
%%      $Date: 2008/09/30 09:14:41 $
%%     Author: Stefan Mueller (CL Uni-Bremen)
%%    Purpose: 
%%   Language: LaTeX
%%%%%%%%%%%%%%%%%%%%%%%%%%%%%%%%%%%%%%%%%%%%%%%%%%%%%%%%%

\chapter{Valenz und Grammatikregeln}
\label{chap-valenz}

%% \NOTE{
%% Haitao Liu:
%% A small note: the title of the third chapter is "`Valenz unu and
%% Grammatikregeln"'. I can understand there the term "`valence"' should lie
%% within the framework of HPSG, but it seems to me if you can mention 2 or
%% 3 important documents of valency theory outside HPSG, it is helpful to
%% the readers. Why I hope to see that, because there are many documents in
%% German about valency theory, it is good if we could link the term to
%% traditional use of the term. I have noted that in the reference already
%% is the name of Helbig's valency dictionary. 
%% }
In diesem Kapitel geht es um die Interaktion zwischen Grammatikregeln und Valenzinformation.
Der Begriff der Valenz stammt aus der Chemie. Atome können sich mit anderen Atomen
zu mehr oder weniger stabilen Molekülen verbinden. Wichtig für die Stabilität ist, wie
Elektronenschalen besetzt sind. Eine Verbindung mit anderen Atomen kann dazu führen,
dass eine Elektronenschale voll besetzt ist, was dann zu einer stabilen Verbindung führt.
%\footnote{
%\url{http://www.quantenwelt.de/atomphysik/modelle/bindungen/valenz.html}.
%\url{http://www.iap.uni-bonn.de/P2K/periodic_table/valences.html}. \urlchecked{09}{11}{2006}.%
%}
Die Valenz\is{Valenz|(} sagt etwas über die Anzahl der Wasserstoffatome aus,
die mit einem Atom eines Elements verbunden werden können. In der Verbindung H$_2$O
hat Sauerstoff die Valenz 2. Man kann nun die Elemente in Valenzklassen einteilen.
Elemente mit einer bestimmten Valenz werden im Periodensystem von Mendeleev
in einer Spalte repräsentiert.

Dieses Konzept wurde von \citet{Tesniere59a-u}\nocite{Tesniere80a-u}
auf die Linguistik übertragen: Ein Kopf braucht bestimmte Argumente,
um eine stabile Verbindung einzugehen. Wörter mit der gleichen Valenz -- also mit der gleichen
Anzahl und Art von Argumenten\is{Argument} -- werden in Valenzklassen eingeordnet, da sie sich in Bezug
auf die Verbindungen, die sie eingehen, gleich verhalten. Abbildung~\vref{abb-chemie-valenz}
zeigt Beispiele aus der Chemie und der Linguistik.
\begin{figure}
\centerline{
\begin{forest}
[O
  [H] 
  [H] ]
\end{forest}
\hspace{5em}
\begin{forest}
[helfen
 [Aicke]
 [Conny] ]
\end{forest}
}
\caption{\label{abb-chemie-valenz}Verbindung von Sauerstoff mit Wasserstoff und Verbindung
eines Verbs mit seinen Argumenten}
\end{figure}

Weitere Information zum Valenzbegriff in der Linguistik findet man in
\citew{AEEHHL2003a-ed,AEEHHL2006a-ed}. Das Buch von Tesnière wurde 1980 in Teilen ins Deutsch
übersetzt \citep{Tesniere80a-u}. Seit 2015 ist eine vollständige Übersetzung ins Englische verfügbar \citep{Tesniere2015a-u}.

Im folgenden wird gezeigt, wie Valenzinformation so repräsentiert werden kann,
dass man statt vieler spezifischer Phrasenstrukturregeln ganz allgemeine Schemata 
zur Lizenzierung syntaktischer Strukturen verwenden kann.

\section{Repräsentation von Valenzinformation}

Die in Kapitel~\ref{sec-psg} diskutierten Phrasenstrukturgrammatiken\is{Phrasenstrukturgrammatik}
haben den Nachteil, dass man sehr viele Regeln für die verschiedenen Valenzmuster
braucht. (\mex{1}) zeigt beispielhaft einige solche Regeln und die
dazugehörigen Verben.
\ea
\label{psg-valenz}
\begin{tabular}[t]{l@{~$\to$~}l@{\hspace{4em}}l}
      S & NP, V                             & \emph{X schläft}\\
      S & NP, NP, V                         & \emph{X Y erwartet}\\
      S & NP, PP[\textit{über\/}], V           & \emph{X über Y spricht}\\
      S & NP, NP, NP, V                     & \emph{X Y Z gibt}\\
      S & NP, NP, PP[\textit{mit\/}], V        & \emph{X Y mit Z dient}\\
      \end{tabular}
\z
Damit die Grammatik keine falschen Sätze erzeugt, muss man dafür sorgen, 
dass Verben nur mit passenden Regeln verwendet werden können.
\eal
\ex[*]{
dass Kirby das Buch schläft
}
\ex[*]{
dass Kirby erwartet
}
\ex[*]{
dass Kirby über den Mann erwartet 
}
\zl
Man muss also Verben (allgemein Köpfe) in Valenzklassen einordnen. Diese
Valenzklassen müssen den Grammatikregeln zugeordnet sein. Damit
ist die Valenz doppelt kodiert: Zum einen sagt man in den Regeln etwas
darüber aus, welche Elemente zusammen vorkommen müssen/""können,
und zum anderen ist Valenzinformation im Lexikon enthalten.

Um solcherart redundante Repräsentation zu vermeiden, nimmt
man in der HPSG -- wie in der Kategorialgrammatik\is{Kategorialgrammatik (CG)} -- Beschreibungen der Argumente eines
Kopfes in die lexikalische Repräsentation des Kopfes auf. 

Ein weiteres Argument dafür, dass Valenzinformation im Lexikon verfügbar sein muss, ist, dass
bestimmte morphologische Prozesse Bezug auf Valenzinformation nehmen. So ist zum Beispiel die \bard
nur für Verben produktiv verwendbar, die ein Akkusativobjekt verlangen: \emph{unterstützbar} ist
bildbar, \noword{helfbar} dagegen nicht.  

In der Beschreibung von Köpfen gibt es verschiedene Valenzmerkmale (\textsc{spr} für Spezifikatoren und \textsc{comps}
für Komplemente). Diese enthalten Beschreibungen
der Objekte, die mit einem Kopf kombiniert werden müssen, damit eine
vollständige Phrase vorliegt. Vorerst ist nur der \compsw wichtig. (\mex{1}) zeigt Beispiele für die Verben aus (\mex{-1}):
\ea
\begin{tabular}[t]{@{}lll}
      Verb             & \comps\\
      \emph{schlafen} & \sliste{ NP }\\
      \emph{erwarten} & \sliste{ NP, NP }\\
      \emph{sprechen} & \sliste{ NP, PP[\type{über}] }\\
      \emph{geben}    & \sliste{ NP, NP, NP }\\
      \emph{dienen}   & \sliste{ NP, NP, PP[\type{mit}] }\\  
      \end{tabular}
\z
\comps ist die Abkürzung für \textsc{subcategorization}. Man spricht auch davon,
dass ein bestimmter Kopf für bestimmte Argumente subkategorisiert\is{Subkategorisierung} ist. Diese Redeweise kommt
wohl daher, dass die Köpfe bezüglich ihrer Wortart bereits kategorisiert sind
und dann durch die Valenzinformation weitere Unterklassen (wie \zb intransitives oder transitives Verb)
%\glossary{name={intransitives Verb},description={Intransitive Verben können im Gegensatz zu transitiven Verben kein Akkusativobjekt binden bzw. benötigen zusätzlich eine Präposition.}}
gebildet werden. Im Englischen gibt es auch die Verwendung \emph{X subcategorizes for Y}, was soviel wie
\emph{X verlangt Y} oder \emph{X selegiert Y}\is{Selektion} heißt. Man sagt auch, dass X Y \emph{regiert}.\is{Rektion}
% Rektion schließt in der Dependenzgrammatik auch Adjunkte ein. Das ist also vielleicht etwas
% irreführend, aber irgendwo muss Rektion noch erwähnt werden.

Statt spezifische Regeln wie die in (\ref{psg-valenz}) zu verwenden, kann man Regelschemata wie die in (\mex{1})
für Kopf"=Argument"=Kombination benutzen:
\ea
\label{psg-regeln-append}
\begin{tabular}[t]{@{}l@{~}l@{~$\to$~}l}
a.& V[\comps \ibox{A}] &  \ibox{B}, V[\comps \ibox{A} $\oplus$ \sliste{ \ibox{B} } ]\\
b.& A[\comps \ibox{A}] &  \ibox{B}, A[\comps \ibox{A} $\oplus$ \sliste{ \ibox{B} } ]\\
c.& N[\comps \ibox{A}] &  \ibox{B}, N[\comps \ibox{A} $\oplus$ \sliste{ \ibox{B} } ]\\
d.& P[\comps \ibox{A}] &  P[\comps \ibox{A} $\oplus$ \sliste{ \ibox{B} } ], \ibox{B}\\
\end{tabular}
\z
In den Boxen verwende ich \emph{A} und \emph{B} statt 1 oder 2, damit man nicht mit den
Strukturteilungen in den folgenden Abbildungen durcheinanderkommt.

Die Schemata lizenzieren die Verbindung von V, A, N bzw.\ P mit einem Element aus der
\compsl.
Dabei ist `$\oplus$'\is{$\oplus$} (\emph{append}\is{Relation!\emph{append}}) eine Relation zur Verknüpfung zweier Listen:\\
\ea
\begin{tabular}[t]{@{}l@{~}l@{}}
\phonliste{ x, y } = & \phonliste{ x } $\oplus$ \phonliste{ y } oder\\
                     & \phonliste{} $\oplus$ \phonliste{ x, y } oder\\
                     & \phonliste{ x, y } $\oplus$ \phonliste{}\\
\end{tabular}
\z
Die Anwendung der Regel (\mex{-1}a) soll anhand der Beispiele in (\mex{1}) erklärt werden:
\eal
\ex dass Aicke schläft
\ex\label{bsp-weil-peter-maria-erwartet}
dass Aicke Conny erwartet
\zl
Die \compsl im Lexikoneintrag von \emph{schläft} enthält genau eine NP. Setzt man
den Lexikoneintrag für \emph{schläft} in die Regel in (\ref{psg-regeln-append}a) ein,
ergibt sich (\mex{1}):
\ea
\label{ex-Regel-mit-einer-NP-in-Valenz}
V[\comps \ibox{A}] $\to$ \ibox{B} NP, V[\comps \ibox{A} \sliste{} $\oplus$ \sliste{ \ibox{B} NP } ]
\z
\emph{append} teilt die \compsl von \emph{schläft} in zwei Teile. 
Ein Teil ist eine einelementige Liste, die die NP enthält, der zweite Teil ist die leere Liste. 
Die leere Liste entspricht der Liste der noch zu sättigenden Argumente.
Abbildung~\vref{fig-Aicke-schlaeft} zeigt den durch das Regelschema lizenzierten Baum.
\begin{figure}
\centerline{
\begin{forest}
sm edges
[{V[\comps \sliste{}]}
  [{\ibox{1} NP[\type{nom}]}
    [Aicke]]
  [{V[\comps \sliste{ \ibox{1} }]}
    [schläft]]]
\end{forest}}
%
\caption{\label{fig-Aicke-schlaeft}Analyse für \emph{Aicke schläft}}
\end{figure}

Lässt man die Ausdrücke in eckigen Klammern in (\mex{0}) weg, bleibt nur (\mex{1}) übrig.
\ea
V $\to$ NP, V
\z
Obwohl die Regel in (\mex{0}) genauso aussieht wie die Phrasenstrukturregeln, die
wir im Kapitel~\ref{sec-psg} behandelt haben, ist sie anderer Natur:
Die Information darüber, dass ein Verb mit einer NP kombiniert wird, wird nicht in
den Regeln spezifiziert, sondern wird über die \comps{}"=Information durch das Verb
beigesteuert.

Die \compsl im Lexikoneintrag von \emph{erwartet} enthält genau zwei NPen. Setzt man
den Lexikoneintrag für \emph{erwartet} in die erste Regel in (\ref{psg-regeln-append}) ein,
ergibt sich (\mex{1}):
\ea
V[\comps \ibox{A}] $\to$ \ibox{B} NP, V[\comps \ibox{A} \sliste{ NP } $\oplus$ \sliste{ \ibox{B} NP } ]
\z
Das Ergebnis der Regelanwendung ist V[\comps \sliste{ NP }]. Das entspricht dem
Lexikoneintrag von \emph{schläft}. Die Regel kann also wie in (\ref{ex-Regel-mit-einer-NP-in-Valenz}) noch einmal
angewendet werden, so dass man dann eine vollständig gesättigte Phrase bekommt. Das zeigt
Abbildung~\vref{fig-Aicke-Conny-erwartet}.

%\psset{xunit=1cm,yunit=5.4mm}
\begin{figure}
\centerline{%
\begin{forest}
sm edges
[{V[\comps \sliste{}]}
  [{\ibox{1} NP[\type{nom}]}
    [Aicke]]
  [{V[\comps \sliste{ \ibox{1} }] }
    [{\ibox{2} NP[\type{acc}]} 
      [Conny]]
    [{V[\comps \sliste{ \ibox{1}, \ibox{2} }] }
      [erwartet]]]]
\end{forest}}
\caption{\label{fig-Aicke-Conny-erwartet}Analyse für \emph{Aicke Conny erwartet}}
\end{figure}

Die Regeln in (\ref{psg-regeln-append}) kann man nun auf zweierlei Weise verallgemeinern:
Zuerst kann man von der Abfolge der Konstituenten auf der rechten Regelseite abstrahieren
(dazu später noch mehr). Man erhält dann (\mex{1}):
\ea
\label{abstraktion-linearisierung}
      \begin{tabular}[t]{@{}lll}
      V[\comps \ibox{A}] & $\to$ & V[\comps \ibox{A} $\oplus$ \sliste{ \ibox{B} } ] \ibox{B}\\
      N[\comps \ibox{A}] & $\to$ & N[\comps \ibox{A} $\oplus$ \sliste{ \ibox{B} } ] \ibox{B}\\
      A[\comps \ibox{A}] & $\to$ & A[\comps \ibox{A} $\oplus$ \sliste{ \ibox{B} } ] \ibox{B}\\
      P[\comps \ibox{A}] & $\to$ & P[\comps \ibox{A} $\oplus$ \sliste{ \ibox{B} } ] \ibox{B}\\
      \end{tabular}
\z
In einem zweiten Schritt kann man, wie das in den \xbar-Schemata gemacht wurde, über
die Kategorie des Kopfes abstrahieren. Man erhält dann ein abstraktes Schema, das
die Regeln in (\ref{psg-regeln-append}) bzw.\ in (\mex{0}) zusammenfasst:
\ea
\label{regelschema-psg-comps}
\begin{tabular}[t]{@{}lll}
H[\comps \ibox{A}] & $\to$ & H[\comps \ibox{A} $\oplus$ \sliste{ \ibox{B} } ] \ibox{B}\\
\end{tabular}
\z
`H' steht hierbei für irgendeinen Kopf (also \zb A, N, P oder V).
%% Zwei mögliche Instantiierungen dieses Schemas zeigt (\mex{1}):
%% \ea
%% \begin{tabular}[t]{@{}lll}
%% N[\comps \ibox{1}] & $\to$ & Det N[\comps \ibox{1} $\oplus$ \sliste{ Det } ]\\
%% V[\comps \ibox{1}] & $\to$ & V[\comps \ibox{1} $\oplus$ \sliste{ NP } ]~~ NP\
%% \end{tabular}
%% \z

Nach diesen Vorbemerkungen zur Repräsentation von Valenzinformation können wir uns nun
dem Aufbau von Lexikoneinträgen zuwenden: Lexikoneinträge werden immer durch Merkmalstrukturen modelliert. Information über die
Aussprache (\textsc{phonology}\isfeat{phon}), die Wortart (\textsc{part-of-speech}, \textsc{p-o-s}) und die Valenz
(hier vorerst nur \comps)\isfeat{comps} kann wie folgt dargestellt werden:
\ea
\label{le-gibt-1}
\textit{gibt\/} (finite Form):\\
\ms[word]{ 
     phon   & \phonliste{ gibt } \\
     p-o-s  & verb\\
     comps & \liste{ NP[\type{nom}], NP[\type{dat}], NP[\type{acc}]   } \\
}
\z
Als Wert von \phon müssten phonologische Umschriften der Wörter angegeben werden,
was oft der Einfachheit halber nicht gemacht wird.
Statt dessen verwendet man die orthographische Form. Zur Behandlung der Phonologie in HPSG siehe
\citew{BK94b} und \citew{Hoehle99a-u}.

NP[\type{nom}], NP[\type{dat}] und NP[\type{acc}] stehen für komplexe Merkmalstrukturen, die intern
genauso aufgebaut sind wie der Eintrag für \emph{gibt} in (\mex{0}). Wie die Kasusinformation
repräsentiert wird, wird auf Seite~\pageref{page-ref-case-feat} erklärt. Die Reihenfolge der
Elemente entspricht der unmarkierten Reihenfolge, \dash der Reihenfolge, die man in den meisten
Äußerungskontexten verwenden kann und die intonatorisch unmarkiert ist \citep{Hoehle82a}. Bei dreistelligen Verben ist
das meistens die Reihenfolge \type{nom}, \type{dat}, \type{acc}, aber bei Verben wie
\emph{aussetzen} ist die Reihenfolge \type{acc} vor \type{dat} die unmarkierte \parencites[\page 249]{Wegener85b}[S.\,60]{Hoberg81a}. 
%
% Die Reihenfolge der Elemente in der \compsl entspricht der Obliqueness"=Hierarchie\is{Obliqueness}
% von \citet{KC77a} und \citet{Pullum77a}:
% \begin{table}[H]
% \resizebox{\linewidth}{!}{%
% \begin{tabular}{@{}l@{\hspace{1ex}}l@{\hspace{1ex}}l@{\hspace{1ex}}l@{\hspace{1ex}}l@{\hspace{1ex}}l@{}}
% SUBJECT $=>$ & DIRECT $=>$ & INDIRECT $=>$ & OBLIQUES $=>$ & GENITIVES $=>$  & OBJECTS OF \\
%              & OBJECT      & OBJECT        &               &                 & COMPARISON 
% \end{tabular}%
% }\label{page-obliquen-h}
% \end{table}
% \noindent
% Diese Hierarchie gibt die unterschiedliche syntaktische Aktivität der grammatischen Funktion wieder.
% Elemente, die weiter links stehen, kommen eher in bestimmten syntaktischen Konstruktionen vor. Beispiele
% für syntaktische Konstruktionen, in denen Obliqueness eine Rolle spielt, sind:
% \begin{itemize}
% \item Ellipse\is{Ellipse} \citep{Klein85}
% \item Vorfeldellipse\is{Vorfeldellipse} \citep{Fries88b}
% \item freie Relativsätze\is{Relativsatz!freier}
%       \citep{Bausewein90,Pittner95b,Mueller99b}
% \item Passiv\is{Passiv} \citep{KC77a}
% \item Zustandsprädikate\is{Zustandsprädikat} \citep{Mueller2001c,Mueller2002b,Mueller2008a}
% \item Bindungstheorie\is{Bindungstheorie} (\citealp{Grewendorf85a}; \citealp{PS92a,ps2})
% \end{itemize}

% \noindent
% Eine Alternative für die Anordnung der Elemente in der \compsl wäre die Abfolge 
% NP[\textit{nom\/}], NP[\textit{dat}], NP[\textit{acc}], die ebenfalls bei vielen Phänomenen eine Rolle spielt. 
% Ein Beispiel für ein solches Phänomen ist die Konstituentenstellung. Diese ist im Deutschen relativ frei. Statt (\mex{1}a) kann
% man in bestimmten Kontexten auch (\mex{1}b) äußern:
% \eal
% \ex weil der Mann der Frau das Buch gibt
% \ex weil der Mann das Buch der Frau gibt
% \zl
% Die Anzahl der Kontexte, in denen (\mex{0}b) geäußert werden kann, ist jedoch kleiner
% als die, in denen (\mex{0}a) geäußert werden kann. Man bezeichnet die Abfolge in (\mex{0}a)
% deshalb auch als präferierte Abfolge oder als Normalabfolge\is{Normalabfolge} \citep{Hoehle82a}.

% In der Literatur wird mitunter von einer größeren Verbnähe des Akkusativs\is{Kasus!Akkusativ} gesprochen,
% und es wird behauptet, dass eine Voranstellung des Verbs mit einem Dativ\is{Kasus!Dativ} nicht möglich
% ist (\citealt{Haftka81a}; \citealt{Haider82}; \citealt{Wegener90}; \citealt{Zifonun92a}).
% Zu dieser Behauptung gibt es Gegenbeispiele bei \citet{Uszkoreit87a}, \citet{SS88a}, %zitieren Thiersch82a-u
% \citet{Oppenrieder91a}, \citet{Grewendorf93} und G.\ \citet[\page5]{GMueller98a}.
% (\mex{1}) zeigt Korpusbelege, in denen ein Dativobjekt, das normalerweise vor dem Akkusativ
% angeordnet werden würde, zusammen mit einem infiniten Verb vorangestellt wurde:

% \eal
% \label{bsp-syntax-pvp-besonders}
% \ex Besonders Einsteigern empfehlen möchte ich Quarterdeck Mosaic, dessen gelungene grafische 
%     Oberfläche und Benutzerführung auf angenehme Weise über die ersten Hürden 
%     hinweghilft, obwohl sich die Funktionalität auch nicht zu verstecken braucht.\footnote{
%       c't, 9/95, S.\,156.}
% \ex Der Nachwelt hinterlassen hat sie eine aufgeschlagene \textit{Hör zu\/} und einen kurzen
%     Abschiedsbrief: \ldots\footnote{
%       taz, 18.11.98, S.\,20.}
% \zl
% Man könnte annehmen, dass der Dativ nur bei bestimmten Verben zusammen mit dem Verb vorangestellt
% werden kann und dass bei solchen Verben dafür der Akkusativ nicht mit dem Verb gemeinsam
% voranstellbar ist. Man würde also für einige Verben die Hierarchie SU $>$ DO $>$ IO und für andere
% SU $>$ IO $>$ DO annehmen. Die Beispiele in (\mex{1}) zeigen jedoch, dass mit demselben Verb Dativ +
% V und Akkusativ + V vorangestellt werden können.
% \eal
% \label{bsp-acc-dat-pvp}
% \ex Den Wählern erzählen sollte man diese Geschichte nicht.
% \ex Märchen erzählen sollte man den Wählern nicht.
% \zl

\section{Spezifikatoren}

Wir haben bisher nur das Valenzmerkmal \comps benutzt. Es gibt 
noch das Merkmal \spr, das für die Repräsentation von Spezifikatoren verwendet wird. Für das Deutsche spielt \spr in Nominalstrukturen eine
Rolle. Ein Nomen wie \emph{Schwester} selegiert seinen Determinator über das \sprm. Komplemente, wie
sie bei relationalen Nomina wie \emph{Schwester}, \emph{Bild} oder Nominalisierungen wie \emph{Vorlesen} möglich sind, werden über \comps
selegiert. Abbildung~\ref{fig-die-Schwester-von-Aicke} zeigt die Analyse von (\mex{1}a):
\eal
\ex die Schwester von Aicke
\ex das Bild vom Gleimtunnel
\ex das Vorlesen von Büchern
\zl 
\begin{figure}
\centerfit{%
\begin{forest}
sm edges
[{NP[\spr \eliste, \comps \eliste]}
   [Det [die] ]
   [N\feattab{
      \spr \nliste{ Det }, \comps \sliste{}}
     [N\feattab{
         \spr \nliste{ Det },\\
         \comps \nliste{ PP[\type{von}] }} [Schwester] ]
        [{PP[\type{von}]} [von Aicke,roof] ] ] ]
\end{forest}}
\caption{\label{fig-die-Schwester-von-Aicke}Analyse des Beispiels \emph{die Schwester von Aicke}}
\end{figure}

Für die Analyse von Spezifikator-Kopf-Strukturen braucht man ein Schema, das analog zu dem in
(\ref{regelschema-psg-comps}) ist. Eine vorläufige Version ist in (\mex{1}) zu sehen:
\ea
\label{regelschema-psg-spr}
\begin{tabular}[t]{@{}lll}
H[\spr \ibox{A}] & $\to$ & \ibox{B} H[\spr \ibox{A}  $\oplus$ \sliste{ \ibox{B} }  ]\\
\end{tabular}
\z
Wir werden darauf im nächsten Kapitel zurückkommen.

Das Deutsche zählt typologisch zu den Subjekt-Objekt-Verb-Sprachen (SOV), so wie \zb auch Afrikaans,
Niederländisch, Japanisch, Koreanisch und Persisch. Das Englische, die skandinavischen Sprachen und die
romanischen Sprachen werden zu den Subjekt"=Verb"=Objekt"=Sprachen (SVO) gezählt. Ich habe in diesem Kapitel
alle Argumente von finiten Verben gleich behandelt. Sie sind alle in der \compsl repräsentiert. Das
ist berechtigt, weil Subjekte im Deutschen keinen besonderen Status haben, ja, es muss nicht mal
ein Subjekt im Satz geben. Das wird im kommenden Abschnitt genauer besprochen. Bei den SVO-Sprachen
ist das jedoch anders: Schon durch die Stellung hat das Subjekt einen anderen Status. Subjekte
stehen links vom Verb, andere Argumente rechts. Das wird in HPSG-Grammatiken dadurch erfasst, dass
für Subjekte ein separates Valenzmerkmal verwendet wird. In manchen Grammatiken ist das das \subjm,
in anderen das \sprm. Da ich das \subjm anders verwende, benutze ich wie \citet*[Chapter~4.3]{SWB2003a} das \sprm
für die Repräsentation des Subjekts in SVO-Sprachen \citep[Kapitel~4.3]{MuellerGermanic}.

Die Struktur in Abbildung~\ref{fig-Aicke-expects-Conny} ist das Gegenstück zu
Abbildung~\ref{fig-Aicke-Conny-erwartet}. Die Komplemente stehen im Englischen rechts des Verbs, das
Subjekt steht links. Für das Subjekt gibt es eine besondere Position. Im Deutschen ist das nicht der
Fall. Da kann das Subjekt irgendwo stehen, Hauptsache, es steht links des Verbs. Gemeinsam mit den
anderen Argumenten.
\begin{figure}
\centerfit{%
\begin{forest}
sm edges
[{V[\spr \eliste, \comps \eliste]}
   [{NP[\type{nom}]} [Aicke] ]
   [V\feattab{
      \spr \sliste{ NP[\type{nom}] }, \comps \sliste{}}
     [V\feattab{
         \spr \sliste{ NP[\type{nom}] },\\
         \comps \sliste{ NP[\type{acc}] }} [expects] ]
        [{NP[\type{acc}]} [Conny] ] ] ]
\end{forest}}
\caption{\label{fig-Aicke-expects-Conny}Analyse des Beispiels \emph{Aicke expects Conny} in der
  SVO-Sprache Englisch}
\end{figure}



\section{Die Argumentstruktur}


Will man erfassen, was alle Sprachen (oder zumindest viele Sprachen) gemeinsam haben, ist es
sinnvoll, eine zugrundeliegende Struktur anzunehmen: die Argumentstruktur (\argst). Die Argumentstruktur ist
eine Liste, in der alle Argumente eines Kopfes enthalten sind. Diese Argumente werden dann auf die
Valenzmerkmale des jeweiligen Kopfes verteilt. Für das Englische kommt das Subjekt in die \sprl und
die Komplemente auf \comps. Für das Deutsche bleibt die \sprl finiter Verben leer und alle Argumente
kommen auf \comps. (\mex{1}) und (\mex{2}) zeigen prototypische Verben im Englischen und im Deutschen:

\ea
\label{ex-spr-comps-arg-st}
\begin{tabular}[t]{@{}ll@{~~}l@{~~}l@{}}
Verb          & \spr                      & \comps                                     & \argst\\
\emph{sleep}  & \sliste{ NP[\type{nom}] } & \sliste{}                                  & \sliste{ NP[\type{nom}] }\\
\emph{expect} & \sliste{ NP[\type{nom}] } & \sliste{ NP[\type{acc}] }                  & \sliste{ NP[\type{nom}], NP[\type{acc}] }\\
\emph{speak}  & \sliste{ NP[\type{nom}] } & \sliste{ PP[\type{about}] }                & \sliste{ NP[\type{nom}], PP[\type{about}] }\\
\emph{give}   & \sliste{ NP[\type{nom}] } & \sliste{ NP[\type{acc}], NP[\type{acc}] }  & \sliste{ NP[\type{nom}], NP[\type{acc}], NP[\type{acc}] }\\
\emph{serve}  & \sliste{ NP[\type{nom}] } & \sliste{ NP[\type{acc}], PP[\type{with}] } & \sliste{ NP[\type{nom}], NP[\type{acc}], PP[\type{with}] }\\  
\end{tabular}
\z

\ea
\label{ex-spr-comps-arg-st}
\begin{tabular}[t]{@{}ll@{~~}l@{~~}l@{}}
Verb            & \spr                      & \comps                                      & \argst\\
\emph{sclafen}  & \sliste{ } & \sliste{ NP[\type{nom}] }                                  & \sliste{ NP[\type{nom}] }\\
\emph{erwarten} & \sliste{ } & \sliste{ NP[\type{nom}], NP[\type{acc}] }                  & \sliste{ NP[\type{nom}], NP[\type{acc}] }\\
\emph{sprechen} & \sliste{ } & \sliste{ NP[\type{nom}], PP[\type{über}] }                 & \sliste{ NP[\type{nom}], PP[\type{about}] }\\
\emph{geben}    & \sliste{ } & \sliste{ NP[\type{nom}], NP[\type{dat}], NP[\type{acc}] }  & \sliste{ NP[\type{nom}], NP[\type{dat}], NP[\type{acc}] }\\
\emph{dienen}   & \sliste{ } & \sliste{ NP[\type{nom}], NP[\type{dat}], PP[\type{with}] } & \sliste{ NP[\type{nom}], NP[\type{dat}], PP[\type{with}] }\\  
\end{tabular}
\z

In der Literatur findet man das Argumentrealisierungsprinzip, das für die Verteilung von Argumenten
auf Valenzlisten sorgt.
\ea
Argumentrealisierungsprinzip\is{Prinzip!Argumentrealisierungs-} (vereinfachte Version):\\
\type{word} \impl
\avm{
[spr    & \1\\
 comps  & \2\\
 arg-st & \1 \+ \2]
}
%\itdopt{This allows non-cannonicals in \ibox{1}. Not sure this is what is intended.} 
\z
Die Formalisierung dieses Prinzips scheint einfach zu sein: Die \argstl wird in zwei Teile geteilt
und der erste Teil bildet die \sprl und der zweite die \compsl. Für das Deutsche kann man dann
für Verben einfach sagen, dass die \sprl die leere Liste ist. Es gibt aber – je nach Analyse – noch andere Phänomene,
die mit der Verteilung der Argumente auf Valenzlisten interagieren. \citet[\page 171]{GSag2000a-u} analysieren
die Voranstellung von Konstituenten (siehe Kapitel~\ref{chap-nla}) so, dass die vorangestellten Konstituenten nicht mehr in den
Valenzmerkmalen auftauchen. Das muss dann im Argumentrealisierungsprinzip berücksichtigt
werden. Auch muss irgendwo geregelt werden, wie viele Elemente überhaupt in der \sprl vorkommen
können. Ohne eine solche Beschränkung könnten im Prinzip alle Elemente aus der \argstl in der \sprl
stehen und keins in der \compsl. Das alles soll uns hier nicht kümmern: Für deutsche Verben
ist die \sprl leer und alle Argumente finiter Verben werden unter \comps aufgeführt.

Die einheitliche \argstl hat den Vorteil, dass bestimmte Eigenschaften von Sprachen sprachübergreifend
gleich behandelt werden können. So kann zum Beispiel die Beziehung zwischen Syntax und Semantik (das
so genannte Linking\is{Linking}) auf \argst gemacht werden \citep*{DKW2024a}. Wir werden uns das im Kapitel~\ref{Kapitel-Semantik}
über Semantik genauer ansehen. Die Argumentstrukturliste ist für die Kasusvergabe
und für die Bindungstheorie relevant.
\is{Valenz|)}
  

\section{Alternativen}

In diesem Abschnitt werden alternative Vorschläge aus anderen Grammatikmodellen diskutiert.
Die Abschnitte, in denen solche Diskussionen stattfinden, sind anders geschrieben als das restliche
Lehrbuch. Sie enthalten ausführliche Literatur- und Datendiskussionen, wie das für wissenschaftliche
Texte nötig ist. Außerdem gibt es mitunter Querverweise auf spätere Kapitel. Dem Neueinsteiger
wird geraten, diese Abschnitte zu überspringen und sie erst beim zweiten Lesen des Buches zu berücksichtigen.

Im vorangegangenen Abschnitt wurde das Subjekt (finiter Verben) genauso wie andere Argumente des Verbs
in der Valenzliste repräsentiert und auch in syntaktischen Strukturen wie die anderen
Argumente behandelt. Das wird in anderen Grammatiktheorien anders gesehen.

Für Sätze wie (\ref{bsp-weil-peter-maria-erwartet}) wird in der \gbt oft die Struktur
in Abbildung~\vref{fig-Aicke-Conny-erwartet-ip} angenommen (siehe \zb \citealt[\page 49]{Grewendorf88a}).
%
\begin{figure}
\centerline{%
\begin{forest}
sm edges
[IP
  [NP [Aicke, roof]]
  [\hphantom{$'$}I$'$ 
    [VP
      [\hphantom{$'$}V$'$
        [NP [Conny,roof]]
        [\hphantom{$^0$}\vnull [erwartet]]]]
    [\hphantom{$^0$}\inull [\trace]]]]
\end{forest}}
\caption{\label{fig-Aicke-Conny-erwartet-ip}Analyse von \emph{Aicke Conny erwartet} mit IP/VP"=Unterscheidung}
\end{figure}
Die Kategorie I ist eine funktionale Kategorie\is{Kategorie!funktionale!I}, die in Grammatiken des Englischen durch
das Stellungsverhalten von Hilfsverben motiviert werden kann. Im Deutschen verhalten sich
Hilfsverben in Bezug auf ihre Stellung aber wie Vollverben \citep[\page 86]{Ross69a-u}, so dass eine Zuordnung von
Voll- und Hilfsverben zu verschiedenen Kategorien nicht gerechtfertigt wäre.
In Analysen, die dem in Abbildung~\ref{fig-Aicke-Conny-erwartet-ip}
skizzierten Ansatz folgen, geht man davon aus, dass das Subjekt\is{Subjekt} einen Sonderstatus hat und
nicht innerhalb der Projektion des Verbs (der VP) steht. Das Subjekt wird deshalb auch
\hypertarget{externesArgument}{\emph{externes Argument}}\is{Argument!externes} genannt. Es lassen sich durchaus
Subjekt"=Objekt"=Asymmetrien feststellen, ob das allerdings eine strukturell andere
Behandlung von Subjekten rechtfertigt, ist eine kontrovers diskutierte Frage 
(vergleiche \zb \citealt*{Haider82,Grewendorf83a,Kratzer84a,Webelhuth85a,%
Sternefeld85b,%
Scherpenisse86a,%S. 31, Kapitel~4
Fanselow87a,Grewendorf88a,Duerscheid89a,Webelhuth90,Oppenrieder91a,Wilder91a,Haider93a,Grewendorf93,%
Frey93a,%S.30
%Duerscheid89a:60
Lenerz94a,%
Meinunger2000a%S. 30
}). 
In Abbildung~\vref{fig-Aicke-Conny-erwartet} werden alle Komplemente des Verbs auf die gleiche Weise gesättigt.
Die maximale Projektion ist eine Projektion des Verbs.
Es gibt in dieser Analyse für Sätze mit finiten Verben
also keine besonders lizenzierte Zwischenstufe VP wie \zb in HPSG"=Grammatiken des Englischen 
(vergleiche \citealt*[Kapitel~6.1]{ps}, 
\citealt*[Kapitel~1.4]{ps2}, \citealt[\page 34]{GSag2000a-u}).\footnote{
	Hierbei ist unwesentlich, ob man flache oder binär verzweigende Strukturen
	annimmt (siehe Kapitel~\ref{sec-konstituentenreihenfolge-alternativen}). Wesentlich ist, ob das Subjekt durch
	ein eigenes Schema gesättigt wird.
}
% 
Sowohl Subjekt als auch Objekte des finiten Verbs sind Elemente der \compsl und werden in Strukturen
gleichen Typs mit dem Verb verbunden.
Das heißt, es gibt keine strukturelle Unterscheidung zwischen Subjekt und 
Objekten. In der GB"=Literatur geht man meistens
davon aus, dass Subjekte nicht subkategorisiert sind (siehe \zb \citew[\page26--28]{Chomsky93a}, \citew[\page33]{Hoekstra87a}).
Die Anwesenheit eines Subjekts wird durch das Erweiterte Projektionsprinzip\is{Prinzip!Erweitertes Projektionsprinzip} (\emph{Extended Projection
Principle} = EPP) erzwungen:
% Carnie2002a:175
\ea
Erweitertes Projektionsprinzip (EPP):
\begin{itemize}
\item Lexikalische Information muss syntaktisch realisiert werden.
\item Jeder Satz enthält ein Subjekt. 
\end{itemize}
\z

\noindent
\citet[Abschnitt~2]{Bresnan82c} hat gezeigt, dass die Behauptungen,
die eine solche Sonderbehandlung des Subjekts rechtfertigen sollen, empirisch falsch sind.

% Bresnan82c:349f zu Subjekt, Selektionsrestriktion und Subjektidiomen

\subsection{Subjekte als Bestandteil von Idiomen}

% Chomsky2007a:23 Why do idioms typically exclude EA?

\is{Idiom|(}%
Eine dieser Behauptungen ist, dass es keine Idiome gibt, die ein Subjekt haben, das Bestandteil des
Phraseologismus ist, wobei ein Objekt frei belegbar ist (\citealp[\page50--51]{Marantz81a-u}).
\citet[\page349--350]{Bresnan82c} diskutiert folgende Beispiele, die zeigen, dass es durchaus
Idiome gibt, deren Subjekt fester Bestandteil des Idioms ist, wobei Nicht"=Subjekt"=Teile
frei belegt werden können:
\eal
\ex\label{cat-tounge}
The cat's got x's tongue.
\glt `X kann nicht sprechen.'
\ex\label{what-is-eating-x}
What's eating x?
\glt `Warum ist X so gereizt?'
%% \ex X's goose is cooked.
%% \glt `X hat Probleme und es gibt keinen Ausweg.'
\zl
\citet[\page29]{Marantz84a} merkt an, dass Beispiele wie (\ref{cat-tounge}) für die Diskussion seiner
Behauptung irrelevant sind, da das freie Element nicht das Objekt ist. Zu (\ref{what-is-eating-x}) schreibt
Marantz:
\begin{quote}
From the point of view of the present theory, it is important that this apparent subject idiom
has no S-internal syntax, for it is precisely S-internal syntax that is at issue. \emph{What's eating NP?}
is not a combination of subject and verb, forming a predicate on the object, but rather a combination of \emph{wh}-question
syntax, progressive aspect, plus subject and verb---that is, a complete sentence frame---with an open slot
for an argument. \citep[\page27]{Marantz84a}
\end{quote}
Wenn man diesem Ansatz folgt, kann man nicht erfassen, wieso (\ref{what-is-eating-x}) anderen englischen
Sätzen ähnelt und wieso (\ref{what-is-eating-x}) intern den normalen syntaktischen Gesetzmäßigkeiten folgend strukturiert ist.
Aus diesem Grund wird idiomatischen Phrasen in HPSG"=Ansätzen sehr wohl eine interne Struktur zugesprochen 
(siehe \citew{KE94a,Sailer2000a,SS2003a,Soehn2006a} zu Idiomanalysen im Rahmen der HPSG). 

Andere Beispiele, die Bresnan angeführt hat, erklärt Marantz mit der Annahme, dass es sich bei
den Subjekten nicht eigentlich um Subjekte handelt, sondern um zugrundeliegende Objekte. Die
Verben in den entsprechenden Wendungen werden in die Klasse der unakkusativischen Verben eingeteilt
(Zur Unterscheidung zwischen unakkusativischen und unergativischen Verben siehe Kapitel~\ref{sec-unakkusativitaet}).

Die Behauptung, dass Subjekte nie Bestandteile von idiomatischen Wendungen sind, die ein frei belegbares
Argument enthalten, ist weit verbreitet (siehe \zb \citew[\page92]{denDikken95a},
\citew[\page112--116]{Kratzer96a}) und wird auch explizit für das Deutsche formuliert, \zb bei \citet[\page50]{Grewendorf2002a}.
Dass die Behauptung für das Deutsche nicht richtig ist -- und also auch nicht für alle Sprachen richtig sein kann -- ,
zeigen Beispiele wie die in (\mex{1}) und (\mex{2}), die zum Teil schon von \citet[\page178]{Reis82} diskutiert
wurden:\footnote{
(\ref{bsp-sticht-der-hafer}), (\ref{bsp-reitet-der-teufel}), (\ref{bsp-felle-davonschwimmen}),
(\ref{bsp-reisst-der-geduldfaden}) und (\ref{bsp-platzt-der-kragen})
 sind von \citet[\page178]{Reis82}. Siehe auch \citet[\page 173]{Haider93a} für Beispiele,
die denen in (\mex{1}) ähneln.%
}
\eal
\ex\label{bsp-sticht-der-hafer}
weil ihn der Hafer sticht          % Reis
\ex\label{bsp-reitet-der-teufel}
weil ihn der Teufel reitet         % Reis, Haider93a:173
\ex\label{bsp-felle-davonschwimmen}
weil ihm alle Felle davonschwimmen % Reis
\ex weil ihn der Schlag trifft     % Haider93a:173
\ex\label{ex-ihn-die-wut-packt}
weil ihn die Wut packt         % Haider SynCom
\ex weil ihm der Kopf raucht       % Keil97:171
\ex\label{kein-hahn-danach-kräht}
weil kein Hahn danach kräht    % Burger:25          ist von Burger, aber in anderem Kontext zitiert
\ex 
weil nach ihm kein Hahn kräht                  % Haider93a:173
\ex 
weil ihn der Esel im Galopp verloren hat   % Haider93a:173
\ex\label{spatzen-gepfiffen}
weil die Spatzen das vom Dach gepfiffen haben  % Haider93a:173
\zl
\eal
\ex\label{bsp-reisst-der-geduldfaden}
weil ihm der Geduldsfaden reißt
\ex\label{bsp-platzt-der-kragen}
weil ihm der Kragen geplatzt ist
\ex weil ihm ein Stein vom Herzen fällt
\ex weil bei ihm Hopfen und Malz verloren ist% Keil97:171
%
%\ex Ihm läuft es kalt den Rücken hinunter.    = ein Schauer
\ex weil ihm der Atem stockt
\zl
\citet[\page208--209]{Marantz97a} formuliert eine Variante seiner Behauptung, die besagt, dass ein Agens\is{semantische Rolle!Agens} 
nicht Bestandteil eines Idioms sein
darf. Wie die Beispiele in (\ref{bsp-reitet-der-teufel}) und (\ref{kein-hahn-danach-kräht})
zeigen, ist das ebenfalls falsch.

\is{Verb!unakkusativisches|(}
\citet[\page89]{Scherpenisse86a} behauptet in einer Arbeit zum Deutschen wie auch \citet{Marantz84a}
in Bezug auf englische Beispiele,
dass es sich bei Beispielen mit Subjekten als Idiombestandteil um sogenannte unakkusativische Verben
handelt, das heißt, dass die Subjekte keine wirklichen Subjekte,
sondern zugrundeliegende Objekte sind. 
\is{Passiv|(}%
Wäre diese Annahme richtig,
dürften keine Passivvarianten von Idiomen mit festem Subjekt existieren,
da bei der Passivierung das Subjekt unterdrückt wird und unakkusativische
Verben eben kein zugrundeliegendes Subjekt haben (siehe Kapitel~\ref{chap-passiv}).
Betrachtet man die oben aufgeführten Beispiele, so stellt man fest, dass die Verben in
(\mex{-1}) in der wörtlichen Lesart durchaus passivierbar sind.
Dass manche Passivierungen der Idiombeispiele schlecht sind, ist
von anderen Idiomen bekannt. In vielen der Beispiele ist der Nominativ
unbelebt, was Einfluss auf die Passivierbarkeit haben dürfte.\footnote{
        Siehe jedoch \citew[Kapitel~3.1.2]{Mueller2002b} zu Passiven mit
        unbelebtem Subjekt.%
}
Beispiele für Passivierungen auch mit der idiomatischen Variante zeigen (\mex{1}) %\fromto{\mex{1}}{\mex{2}}
und (\ref{wut-gepackt}a, b):

\eal
\ex Was schon von den Dächern gepfiffen wurde, jetzt ist es amtlich: In \emph{Britische Besatzer unterstützten protestantische Mörder} berichtet der Spiegel über die Vorlage eines offiziellen Untersuchungsberichtes über die Zusammenarbeit der britischen Besatzer mit der Ulster Defence Association (UDA), einer der größten paramilitärischen Gruppen in Nordirland.\footnote{
  \url{http://witch.muensterland.org/2003/04/18.html}. 05.01.2007.
}
\ex Was vor wenigen Wochen nur vereinzelt von den Dächern gepfiffen wurde, nahm in der letzten Woche konkrete Formen an: US-Milliardär Haim Saban, 58 Jahre, hat den Hamburger Verleger Heinz Bauer beim Bieten um die Konkursmasse des Kirch-Konzerns besiegt und gehört damit nun zu den wichtigen Figuren auf dem deutschen Medienmarkt.\footnote{
  \url{http://heim.at/paysuspends/kulturmedien/saban.htm}.
}
\ex In einer Pressemitteilung bestätigte die WWF am 23.03.2001 offiziell die Nachricht, die von allen Spatzen längst von den Dächern gepfiffen wurde.\footnote{
  \url{http://people.freenet.de/wwf-hp/WWFschlucktWCW.htm}. 05.01.2007.
}
\zl

%% %\emph{Der Esel hat X im Galopp verloren.} kann ebenfalls passiviert werden, wie die Beispiele in (\mex{1}) zeigen:
%% \eal
%% %\ex Wer weiß, ob dieses armselig dummerhaftige Individuum einfach vom Esel im Galopp verloren wurde oder ob sich da nicht 'n Dackel mit 'ner Lokomotive gemischt hat und dabei so'n Schwachköppchen herausgemendelt wurde... ?!\footnote{nur noch im Cache
%% %
%% \ex  ob das kleine Menschenkind vom Storch gebracht oder vom Esel im Galopp verloren oder als Geschenk des Himmels von einem Engel geleitet wird\footnote{
%%   http://www.offenes-forum-glaube.de/Longard/EngelW00.html. 11.12.2004
%% }
%% \ex Januar 1982 wurde ich, laut Aussage meiner Eltern, vom Esel im Galopp verloren.\footnote{
%%   http://www.twenchatter.com/Profile/schnee.HTM. 11.12.2004
%% }
%% \zl
\noindent
Beispiel (\mex{1}) zeigt ein Zustandspassiv von (\ref{bsp-sticht-der-hafer}), und die Beispiele in
(\mex{2}) sind Passive von (\ref{ex-ihn-die-wut-packt}), wobei (\ref{bsp-von-wut-gepackt}) ein
pränominales Partizip mit passivischer Argumentstruktur enthält: 
\ea
Vom Hafer gestochen\footnote{
  taz, 06.04.2000, S.\,20.
}
\z
\eal
\label{wut-gepackt}
\ex Und doch kann sich kaum eine Frau davon freisprechen, dass sie von Empörung oder gar heiliger Wut gepackt wird, sobald sie eines Mannes auf ihrem Territorium ansichtig wird.\footnote{
  taz bremen, 29.11.1997, S.\,26.
}
\ex "`Ich war so narrisch, weil ich im ersten Lauf so einen Scheiß gefahren bin"', 
    sagte nach dem Rennen Michaela Gerg, die nach einem verkorksten ersten Lauf von Wut gepackt wurde, 
    im zweiten Durchgang Bestzeit fuhr und noch auf den siebten Rang kam.\footnote{
      taz, 29.01.1990, S.\,13.
    }
\ex\label{bsp-von-wut-gepackt}
Mit 2:6 gab sie den ersten Satz übernervös ab, drehte daraufhin von Wut gepackt teuflisch auf und holte sich den zweiten Satz unter Zuhilfenahme göttlicher Passierschläge und perfekter Lobs ebenfalls mit 6:2.\footnote{
  taz, 22.10.1991, S.\,13.
}
\zl

\noindent
In (\mex{1}) und (\mex{2}) liegen weitere Passivierungen bzw.\ pränominale Partizipien mit passivischer
Argumentstruktur vor.
\eal
\label{vom-teufel-geritten}
\ex 
"`Iordannis wird vom Teufel geritten"', witzelte Vangelis.\footnote{
Jentzsch, Kerstin, Seit die Götter ratlos sind, München: Heyne 1999 [1994] S.\,227.
}
\ex Wird Gerda vom Teufel geritten?\footnote{
Dietlof Reiche, Unterwegs mit Gerda, in: Die Zeit 17.09.1998, S.\,71.
}
\ex
Er wurde verhext, oder sie wurde vom Teufel geritten.\footnote{
Schwanitz, Dietrich, Männer, Frankfurt a.M.: Eichborn 2001, S.\,241.
}
\zl
\eal
\ex viele vom Schlag getroffene und Lahme aber wurden geheilt.\footnote{
        http://members.tirol.com/vineyard.grk/predigt.html. 22.09.2003.
}
\ex Jeffrey, der Sohn des vom Schlag Getroffenen, besucht seinen Vater im Krankenhaus.\footnote{
         Frankfurter Allgemeine Zeitung; 13.11.1986. \url{http://www.davidlynch.de/bluefaz86.html}.\newline 22.09.2003.%
}
\zl
Das zeigt, dass die Idiome \emph{Die Spatzen pfeifen X von den Dächern}, \emph{X packt die Wut},
\emph{der Teufel reitet X} und \emph{X trifft der Schlag} syntaktisch aktiv sind. Eine Grammatiktheorie
sollte erfassen, dass die Aktivstrukturen der Idiome genauso zu den Passivstrukturen in Beziehung stehen,
wie das bei nichtidiomatischen Wendungen der Fall ist.
%dass also bei diesen Idiomen keine Inkorporierung der Idiombestandteile
%in das Verb vorliegen kann.
%% Hier meint er wohl im Idiom schon selbst.
%% Bsp: Das ist an den Haaren herbeigezogen.
%%
%% Die Passivierungen in (\ref{vom-teufel-geritten}) zeigen außerdem, dass Marantz' Behauptung \citeyearpar[\page208]{Marantz97a},
%% dass Vorgangspassive nicht mit Idiomen verträglich sind, sondern nur Zustandspassive, falsch ist.%

\citet[\page435]{Sternefeld85a} merkt an, dass die Nominative in den von Marga Reis
diskutierten Idiomen sich wie Objekte verhalten, da sie adjazent
zum Verb auf"|treten und somit dieselben Stellungseigenschaften wie Objekte haben.
Sternefeld nimmt eine Inkorporationsanalyse\is{Inkorporation} für solche Idiome an,
\dash, die Nominativphrase bildet eine feste Einheit mit dem Verb.
Die Beispiele (\ref{kein-hahn-danach-kräht}) und (\ref{spatzen-gepfiffen}) zeigen, dass die Idiombestandteile nicht unbedingt
adjazent zum Verb sein müssen. Außerdem ist das Idiom \emph{Die Spatzen pfeifen X von den Dächern.} passivierbar, was zeigt, dass
es nicht legitim wäre, es als unveränderbare Einheit einfach im Lexikon aufzuführen.
\emph{die Spatzen} ist ein syntaktisch normales Subjekt, das sich bei Umformungen
wie Passivierung auch ganz normal verhält.

Zusammenfassend kann man also sagen:
\begin{itemize}
\item Es gibt Idiome, deren Subjekt Idiombestandteil ist und die gleichzeitig ein frei belegbares Objekt haben.
\item Eine Teilklasse dieser Idiome enthält Verben, die nicht unakkusativisch sind.
\item Es gibt Idiome mit festem Subjekt, die passiviert werden können bzw.\ passivische adjektivische Partizipien bilden.
\item Es gibt Idiome mit festem Subjekt, deren Bestandteile (inklusive Subjekt) umstellbar sind.
\end{itemize}
Damit ist Marantz' Behauptung für eine Sprache widerlegt und kann somit auch nicht für alle Sprachen gültig sein.
\is{Passiv|)}
\is{Idiom|)}\is{Verb!unakkusativisches|)}

\subsection{Expletivpronomina in Objektposition}
\label{sec-expletivum-in-obj-position}

\is{Pronomen!Expletiv-|(}%
Zum anderen wird behauptet, dass Expletivpronomina nicht in Objektsposition auf"|treten.
%\citep[S.\,175]{Carnie2002a}.\footnote{
%  Carnie schreibt `usually'
Doch auch das
ist falsch, wie die Beispiele in (\mex{1}) zeigen.\footnote{
  Die Beispiele (\mex{1}b--e) sind aus \citew[\page 107]{Tarvainen2000a}.
}
\eal
\ex Er hat es weit gebracht.\footnote{
        \citew[\page172]{Lenerz81a}.
}
\ex Ich habe es heute eilig.
\ex Sie hat es ihm angetan.
\ex Er hat es auf sie abgesehen.
\ex Ich meine es gut mit dir.
\zl
% Wir ließen es dabei bewenden [\ldots]\footnote{
%        Murakami Haruki, \emph{Hard-boiled Wonderland und das Ende der Welt}, suhrkamp taschenbuch, 3197, 2000,
%        Übersetzung Annelie Ortmanns und Jürgen Stalph, S.\,73
%}

%% (\mex{0}) widerlegt übrigens auch Hoekstras Behauptung \citeyearpar[\page35]{Hoekstra87a},
%% dass die Anzahl der Argumente nicht größer als die Anzahl der durch ein Prädikat 
%% zugewiesenen semantischen Rollen sein kann. 
\noindent
Man könnte die Behauptung retten, indem man annimmt, dass \emph{es} eine
Quasi"=Theta"=Rolle\is{semantische Rolle!Quasi"=Theta"=Rolle} bekommt, wie das mitunter für
Wetterverben\is{Verb!Wetter-} wie die in (\mex{1}) angenommen wird
  \citep[\page324--327]{Chomsky93a}\nocite{Chomsky81a}.\footnote{ 
  \citet{Berman99a} zitiert ein Manuskript von Christian Fortmann\aimention{Christian Fortmann} 
  mit den Beispielen in (\mex{1}).%
}
\eal
\ex Gestern hat es geblitzt, ohne zu donnern.
\ex Gestern hat es geregnet, anstatt zu schneien.
\zl
Die Behauptung, dass \emph{es} eine semantische Rolle zugewiesen bekommt, wird dadurch
motiviert, dass man sagt, dass sich das \emph{es} in (\mex{0}) genauso verhält wie eine
normale Nominalphrase in bestimmten Infinitivkonstruktionen.
Man sagt, dass in (\mex{1}) das Subjekt von \emph{fährt} mit dem Subjekt von \emph{anzustrengen}
referenzidentisch ist, sich also auf dasselbe Individuum bezieht (in (\mex{1}) auf \emph{Aicke}).
\ea
Aicke fährt schnell, ohne sich anzustrengen.
\z
Man kann aber Fälle wie (\mex{-1}) auch erfassen, ohne annehmen
zu müssen, dass \emph{es} eine semantische Rolle bekommt: Man muss für (\mex{-1}) und (\mex{0})
lediglich verlangen, dass der semantische Index\is{Index} der beiden
Subjekte identisch\is{Koindizierung} ist. In (\mex{-1}) handelt es sich um einen
nicht"=referentiellen Index, in (\mex{0}) dagegen um einen referentiellen 
(zu semantischen Indices siehe Kapitel~\ref{sec-index}).

Ich nehme im Folgenden an, dass sowohl expletive Objekte als auch expletive Subjekte
vom entsprechenden Kopf syntaktisch, nicht aber semantisch selegiert werden.
Durch die Einführung von Quasi"=Theta"=Rollen würde der Begriff der semantischen Rolle 
syntaktisiert\label{page-syntaktisierung} und somit entwertet.
\is{Pronomen!Expletiv-|)}
% Das ist das Theta-Kriterium (Fanselow91a:46)

\subsection{Subjektlose Prädikate}
\label{sec-subjekt-valenz}
% Berman99: Safir85, Cardinaletti90, Grewendorf89, Vikner95 -> expletives pro

% Hoekstra87a:72
Die Behauptung, dass alle Sätze ein Subjekt haben müssen, die Bestandteil des
\emph{Extended Projection Principle}\is{Extended Projection Principle@\emph{Extended Projection Principle}} ist,
ist für das Deutsche ebenfalls problematisch,
wie die Beispiele in (\mex{1}) zeigen:
\eal
\iw{grauen}
\ex Den Studenten graut vor der Prüfung.
\ex Heute wird getanzt.\is{Passiv!unpersönliches}
\ex Mir ist schlecht. 
\zl
Wenn man Subjekte mit in der Valenzinformation von Köpfen repräsentiert, dann
ist der Unterschied zwischen Prädikaten, die ein Subjekt selegieren, und subjektlosen
Prädikaten leicht zu erfassen: Bei subjektlosen Prädikaten wie \emph{grauen}
gibt es in der Valenzinformation einfach kein Subjekt.
Das \emph{Extended Projection Principle} wurde benutzt, um das Auf"|treten von Expletivpronomina
(nichtreferierenden \emph{it} bzw.\ \emph{es}) in Subjektsposition vorherzusagen.
Man könnte für Fälle wie (\mex{0}) behaupten, dass es ein Subjekt gibt, dass
dieses jedoch abstrakt ist und nicht phonetisch realisiert wird\is{leere Kategorie}. Folgt
man einer solchen Erklärung, bleibt aber offen, warum Subjekte, die für
die Bedeutung von Ausdrücken irrelevant sind, mal phonetisch leer realisiert werden
können (im Fall von \emph{grauen}) und mal -- wie bei \emph{regnen} --
als \emph{es} realisiert werden müssen \citep[\page80]{Fanselow91a}. 
% Grewendorf89 Kapitel 3,90b expletives pro bei Passiv
Zu subjektlosen Verben und dem
\emph{Extended Projection Principle} siehe auch \citew[Kapitel~6.2.1]{Haider93a}, 
\citews{Berman99a}[Kapitel~4]{Berman2003a} und \citew[\page 176]{Fanselow2000b}.
Zu leeren Subjekten und Kontrolle\is{Kontrolle} siehe \citew[\page912]{Nerbonne86b}.

%% \subsection{Prädikate mit fakultativ expletivem Subjekt}
%%
%% Hubert Haider (p.\,M., 2005) hat mich noch auf die folgenden Daten hingewiesen,
%% die schon Hermann \citet[\S 24]{Paul1919a} diskutiert:\footnote{
%%   Siehe auch \citew[\page156]{SG82a}, wo das Beispiel (i) diskutiert wird.
%% \ea
%% Oft treibt es ihn ins Gasthaus.
%% \zlast
%% }
%% \eal
%% \ex Mich ergreift es.
%% \ex Mich hält es nicht länger.
%% \ex Mich schüttelt/juckt/reizt/treibt/lockt/reißt/zieht/drängt es. 
%% \zl
%% Diese Verben können sowohl mit einem persönlichen als auch mit einem expletiven
%% Subjekt benutzt werden. Würde man diese Eigenschaft nicht als lexikalische Eigenschaft
%% von Verben erfassen, so würde man vorhersagen, dass man alle Verben mit einem expletiven
%% Subjekt verwenden kann. So müsste nach Haider (\mex{1}) auch bedeuten können,
%% dass sich gestern ein Anrufensereignis mit mir als Angerufenem ereignete.
%% \ea
%% Gestern hat es mich angerufen.
%% \z
%% Das ist jedoch nicht der Fall. Es handelt sich also um Eigenheiten der Verben
%% in (\mex{-1}) und dies muss sich in der Argumentstruktur dieser Verben widerspiegeln.
%% Das Subjekt muss demzufolge Bestandteil der Argumentstruktur von Verben sein.


%% nee is ja Quatsch, denn dann müsste es ja in SLASH landen.
%%
%% \citet{Aranovich2003a} macht einen interessanten Vorschlag zur Repräsentation
%% von leeren Subjekten: Er nimmt ein leeres Subjekt in der Argumentstruktur (\argst)
%% an, das dann nicht in die Valenzlisten aufgenommen wird.
%%
%% Bei finiten Verben wird das Subjekt gemeinsam mit den Objekten realisiert, bei
%% infinite Verbphrasen wie \zb \emph{Conny zu lieben} in (\mex{1})
%% taucht allerdings kein Subjekt an der Oberfläche auf.
%% \ea
%% Karl behauptet, Conny zu lieben.
%% \z
%% Bei infiniten Verbphrasen gibt es also strukturelle Unterschiede zwischen Subjekt und Objekten:
%% Subjekte werden nicht realisiert, andere Argumente müssen dagegen (wenn sie nicht optional sind)
%% realisiert werden.
%% % Kratzer84: finite IP mit Subjekt und infinite VP
%% Genaueres dazu findet man in den Kapiteln~\ref{passiv} und~\ref{kontrolle}.
%% Ein weiterer Vorteil der separaten Auf"|listung des Subjekts infiniter
%% Verben liegt darin, dass sowohl Verbphrasen wie die oben auf"|geführte als
%% auch finite Sätze Maximalprojektionen sind und somit ihr ähnliches
%% syntaktisches Verhalten erklärt werden kann.
%%
%% \citet*{Borsley87} plädierte schon 1987 dafür, 
%% das Subjekt nicht in die Sub\-cat"=Liste auf"|zunehmen.
%% \citet*{Kiss92} dagegen schlug vor, nur das Subjekt 
%% infiniter Verben in einer Liste als Kopf"|merkmal zu repräsentieren.
%% Nur in finiten Sätzen wird das Subjekt gesättigt. Deshalb ist
%% das Subjekt auch nur bei finiten Verbformen ein Element der Sub\-cat"=Liste. 
%% Bei nichtfiniten Verbformen dagegen ist das Subjekt nicht Element der Sub\-cat"=Liste. 
%% Statt dessen gibt es ein Kopf"|merkmal \textsc{subj}, das als Wert eine maximal 
%% einelementige Liste hat. Diese Liste enthält das Subjekt bei Verben,
%% die ein Subjekt haben, und ist leer bei Verben, die kein Subjekt haben 
%% (\zb \emph{grauen} in {\em Mir graut.\/}).\is{Verb!subjektloses}
%% % Hoekstra87a: Perkolation der Subjekt-Theta-Rolle = SUBJ-Merkmal



\questions{
\begin{enumerate}
\item Wofür steht `$\oplus$'?
\item Was ist die Obliqueness"=Hierarchie?
\end{enumerate}}

\exercises{
\begin{enumerate}
\item Geben Sie die Valenzlisten für folgende Wörter an:
      \eal
      \ex er
      \ex ihre (in \emph{ihre Ankündigung})
      \ex schnarcht
      \ex denkt
      \ex graut
      \zl
\end{enumerate}
}



%      <!-- Local IspellDict: de -->

