%% -*- coding:utf-8 -*-
%%%%%%%%%%%%%%%%%%%%%%%%%%%%%%%%%%%%%%%%%%%%%%%%%%%%%%%%%
%%   $RCSfile: 9-hpsg-konstituentenreihenfolge.tex,v $
%%  $Revision: 1.17 $
%%      $Date: 2008/09/30 09:14:41 $
%%     Author: Stefan Mueller (CL Uni-Bremen)
%%    Purpose: 
%%   Language: LaTeX
%%%%%%%%%%%%%%%%%%%%%%%%%%%%%%%%%%%%%%%%%%%%%%%%%%%%%%%%%

\chapter{Konstituentenreihenfolge}
\label{Kapitel-Konstituentenreihenfolge}

Das Deutsche ist eine Sprache mit relativ freier Konstituentenstellung.
Außerdem wird das Deutsche typologisch zu den Verbletztsprachen (SOV) gezählt.
In deklarativen Hauptsätzen und in Fragesätzen steht das Verb jedoch an zweiter
bzw.\ an erster Stelle. In den folgenden Abschnitten soll erklärt werden, wie man
die relativ freie Konstituentenstellung behandeln kann, und in Abschnitt~\ref{sec-v1}
wird gezeigt, wie die Verberststellung zur Verbletztstellung in Beziehung gesetzt werden
kann.

\section{Anordnung von Konstituenten im Mittelfeld}
\label{sec-mf}

Das Deutsche ist eine Sprache mit relativ freier Konstituentenstellung.
Im \mf können Argumente in nahezu beliebiger Abfolge angeordnet werden.
So kann man statt der Abfolge in (\mex{1}a) auch die Abfolgen in (\mex{1}b--f)
verwenden:
\eal
\label{bsp-perm-mf}
\ex
\label{ex-weil Aicke dem Affen den Stock gibt}
weil Aicke dem Affen den Stock gibt
\ex weil Aicke den Stock dem Affen gibt
\ex weil den Stock Aicke dem Affen gibt
\ex weil den Stock dem Affen Aicke gibt
\ex weil dem Affen Aicke den Stock gibt
\ex weil dem Affen den Stock Aicke gibt
\zl
Bei den Abfolgen in (\mex{0}b--f) muss man die Konstituenten anders betonen
und die Menge der Kontexte, in denen der Satz mit der jeweiligen Abfolge
geäußert werden kann, ist gegenüber der Menge der Kontexte, in denen (\mex{0}a)
geäußert werden kann, eingeschränkt \citep{Hoehle82a}. Man nennt die
Abfolge in (\mex{0}a) deshalb auch die \wis{Normalabfolge} bzw.\ die unmarkierte
Abfolge.

Außer den Argumenten können sich noch Adjunkte im \mf befinden. Diese können
an beliebigen Positionen zwischen den Argumenten stehen. Für (\mex{0}b) ergeben
sich \zb folgende Möglichkeiten:
\eal
\label{bsp-adj-mf}
\ex weil morgen Aicke dem Affen den Stock gibt
\ex weil Aicke morgen dem Affen den Stock gibt
\ex weil Aicke dem Affen morgen den Stock gibt
\ex weil Aicke dem Affen den Stock morgen gibt
\zl
Skopustragende Adjunkte unterscheiden sich aber von Argumenten dadurch, dass man mehrere
Adjunkte im Mittelfeld nicht einfach umordnen kann, ohne die Bedeutung des Satzes zu ändern.
Die beiden Sätze in (\mex{1}) bedeuten verschiedene Sachen:\footnote{
  Im Gegensatz dazu ändert sich bei sogenannten intersektiven Adjunkten die Hauptbedeutung nicht:
\eal
\ex weil er morgen in der Schule singt
\ex weil er in der Schule morgen singt
\zl
Die beiden Sätze in (i) unterscheiden sich hinsichtlich der Kontexte, in denen sie geäußert
werden können, aber in beiden Sätzen wird etwas darüber ausgesagt, was jemand morgen tut und wo
das entsprechende Ereignis stattfindet.%
}
\eal\is{Adverb!Skopus}
\label{bsp-Skopus-absichtlich-nicht}
\ex
\label{bsp-absichtlich-nicht}
weil er absichtlich nicht lacht
\ex\label{bsp-nicht-absichtlich}
weil er nicht absichtlich lacht
\zl

\noindent
In der HPSG"=Theorie wurde eine große Anzahl von alternativen Vorschlägen zur Erklärung der
Daten in \fromto{\mex{-2}}{\mex{0}} diskutiert. Bei der Behandlung der Mittelfeldabfolgen
spielt immer auch die Behandlung der Verbstellung eine Rolle. Diese wird jedoch erst im
folgenden Abschnitt behandelt. Wichtig für die Auswahl des richtigen Ansatzes sind
bestimmte Arten von Vorfeldbesetzung. Vorfeldbesetzung ist Gegenstand des Kapitels~\ref{Kapitel-nla},
so dass es erst im Anschluss an das Kapitel~\ref{Kapitel-nla} möglich ist, alternative
Analysen für \fromto{\mex{-2}}{\mex{0}} zu besprechen.

Bisher wurden binär verzweigende Strukturen angenommen (Schema~\ref{schema-bin-prel} auf
Seite~\pageref{schema-bin-prel} und Schema~\ref{ha-schema-prel} auf Seite~\pageref{ha-schema-prel}).
Mit den entsprechenden Schemata können wir (\ref{ex-weil Aicke dem Affen den Stock gibt}), (\ref{bsp-adj-mf}) und auch (\ref{bsp-Skopus-absichtlich-nicht})
ohne Probleme analysieren: Bei der Analyse von (\ref{ex-weil Aicke dem Affen den Stock gibt}) wird das Verb mit dem jeweils
letzten Element der \compsl kombiniert. Bei der Analyse der Sätze in (\mex{-1})
wird eine bestimmte Verbalprojektion mit einem Adjunkt kombiniert, und die verbleibenden
Argumente werden danach gesättigt. Die unterschiedliche Bedeutung der Sätze in (\ref{bsp-Skopus-absichtlich-nicht})
ergibt sich daraus, dass in (\ref{bsp-absichtlich-nicht}) \emph{absichtlich} mit der Projektion \emph{nicht lacht} verbunden
wird, wohingegen in (\ref{bsp-nicht-absichtlich}) \emph{nicht} mit \emph{absichtlich lacht} kombiniert wird.

Es bleibt zu klären, wodurch die Abfolgen in (\mex{-2}b--f) lizenziert sind. Diese
Sätze lassen sich nach einer leichten Änderung des Kopf"=Komplement"=Schemas ebenfalls
analysieren: Das Schema muss so abgeändert werden, dass es die Sättigung von Komplementen
in beliebiger Reihenfolge erlaubt. Das wird dadurch erreicht, dass die \compsl nicht
in einen Listenanfang und einen einelementigen Rest sondern in drei Teile geteilt wird: Der erste
Teil ist eine Liste beliebiger Länge (\,\ibox{1} im Schema~\ref{schema-Kopf-Komplementschema-prel2}), der zweite eine einelementige Liste (\sliste{
  \ibox{2} }) und der dritte wieder eine
Liste beliebiger Länge \iboxb{3}.\footnote{
  \citet{Gunji86a} hat in seiner Analyse des Japanischen\il{Japanisch} die Verwendung eines
  mengenwertigen Valenzmerkmals vorgeschlagen, wodurch ebenfalls eine variable Sättigungsreihenfolge
  entsteht. \citet{HN89a}, \citet{Pollard90a} und \citet*{EEU92a} nehmen für das Deutsche
  ein mengenwertiges Valenzmerkmal an. Die Verwendung eines listenwertigen Merkmals hat den
  Vorteil, dass die Elemente geordnet sind und man somit Beschränkungen, die sich auf
  Hierarchien beziehen, formulieren kann. Siehe \citew{MuellerBinding} zur Bindungstheorie und
  Kapitel~\ref{Kapitel-Kasus} zur Kasusvergabe. 
%Seite~\pageref{page-obliquen-h} zur Obliqueness"=Hierarchie. % Nee, steht da nicht drin. \citep[\page ]{Mueller2004b}.
% wohl wegen ARG-ST

  \citet[\page185]{FR92} und \citet[\page218--223]{Kiss95b} nehmen ebenfalls eine Valenzliste und ein Valenzprinzip an, das es erlaubt,
  einen Kopf mit einem beliebigen Element aus der Valenzliste zu kombinieren.%
}
Das Kopf"=Komplement"=Schema hat also folgende Form:
\begin{samepage}
\begin{schema}[Kopf-Komplement-Schema (binär verzweigend, vorläufige Version)]
\label{schema-Kopf-Komplementschema-prel2}\is{Schema!Kopf"=Komplement"=}
\type{head"=complement"=phrase}\istype{head"=complement"=phrase} \impl\\
\onems{
      cat$|$comps \ibox{1} $\oplus$ \ibox{3}\\
      head-dtr$|$cat$|$comps \ibox{1} $\oplus$ \sliste{ \ibox{2} } $\oplus$ \ibox{3} \\
      non-head-dtrs \sliste{ \ibox{2} }\\
}
\end{schema}
\end{samepage}
Mit diesem Schema kann man alle Abfolgen in (\ref{bsp-perm-mf}) analysieren. Das soll
anhand der einfacheren Sätze in (\mex{1}) verdeutlicht werden.
%\enlargethispage{\baselineskip} Dann kommt es weiter hinten zum Umbruch in (8).
\eal
\ex weil niemand den Roman kennt
\ex weil den Roman niemand kennt
\zl
Die Analyse von (\mex{0}a) zeigt Abbildung~\vref{abb-niemand-den-Roman-kennt}
und die von (\mex{0}b) 
Abbildung~\ref{abb-den-Roman-niemand-kennt}.%
\begin{figure}
\begin{forest}
sm edges
[{V[\comps \eliste]}
  [{\ibox{1} NP[\textit{nom}]}
    [niemand]]
  [{V[\comps \nliste{ \ibox{1} }]}
    [{\ibox{2} NP[\type{acc}]}
       [den Roman,roof]]
    [{V[\comps \nliste{ \ibox{1}, \ibox{2} } ]}
      [kennt]]]]
\end{forest}
\caption{\label{abb-niemand-den-Roman-kennt}Analyse für \emph{niemand den Roman kennt}}
\end{figure}%
\begin{figure}
\begin{forest}
sm edges
[{V[\comps \eliste]}
  [{\ibox{2} NP[\type{acc}]}
     [den Roman,roof]]
  [{V[\comps \nliste{ \ibox{2} }]}
    [{\ibox{1} NP[\textit{nom}]}
      [niemand]]
    [{V[\comps \nliste{ \ibox{1}, \ibox{2} } ]}
      [kennt]]]]
\end{forest}
\caption{\label{abb-den-Roman-niemand-kennt}Analyse für \emph{den Roman niemand kennt}}
\end{figure}
Die beiden Analysen unterscheiden sich einzig und allein darin, welches Element der \compsl
zuerst abgebunden wird: In Abbildung~\ref{abb-niemand-den-Roman-kennt} wird zuerst das Akkusativobjekt
mit dem Verb kombiniert und in Abbildung~\ref{abb-den-Roman-niemand-kennt} zuerst das Subjekt. In der
Analyse in Abbildung~\ref{abb-niemand-den-Roman-kennt} ist bei der Kombination von \emph{den Roman} und
\emph{kennt} die \ibox{3} aus Schema~\ref{schema-Kopf-Komplementschema-prel2} die leere Liste und \ibox{1} die Liste
\sliste{ NP[\type{nom}] }. Die Verkettung von \ibox{1} und \ibox{3} ist demzufolge ebenfalls
\sliste{ NP[\type{nom}] }, weshalb der \compsw der Verknüpfung aus \emph{den Roman} und \emph{kennt}
die Liste \sliste{ NP[\type{nom}] } ist (\sliste{ \ibox{1} } in
Abbildung~\ref{abb-niemand-den-Roman-kennt}).  Das Nominativargument wird im zweiten Schritt gesättigt:
Bei der Verbindung von \emph{niemand} mit \emph{den Roman kennt} sind sowohl \ibox{1} als auch \ibox{3}
aus Schema~\ref{schema-Kopf-Komplementschema-prel2} die leere Liste. Das Ergebnis der Kombination ist eine vollständig
gesättigte Phrase. In der Analyse in Abbildung~\ref{abb-den-Roman-niemand-kennt} ist die \ibox{3}
die Liste \sliste{ NP[\type{acc}] } und \ibox{1} die leere Liste. Die Verkettung von \ibox{1} und
\ibox{3} ist \sliste{ NP[\type{acc}] }. Die Kombination von \emph{niemand kennt} mit NP[\type{acc}]
erfolgt im zweiten Schritt.

Sprachen mit fester
Konstituentenstellung\is{Konstituentenstellung!feste}\is{Konstituentenstellung!freie} (wie das
Englische\il{Englisch}) unterscheiden sich von Sprachen wie dem Deutschen dadurch, dass sie die
Argumente von einer Seite beginnend abbinden (im Englischen\il{Englisch} wird das Subjekt in einer separaten
Valenzliste repräsentiert, so dass das Subjekt links des Verbs und alle anderen Argumente rechts des
Verbs realisiert werden), wohingegen Sprachen mit freierer Konstituentenstellung die Argumente
in beliebiger Reihenfolge abbinden können. In Sprachen mit fester Konstituentenstellung ist also
\ibox{1} bzw.\ \ibox{3} die leere Liste. Da die Strukturen des Deutschen nicht hinsichtlich \ibox{1}
bzw.\ \ibox{3} beschränkt sind, \dash, \ibox{1} und \ibox{3} können die leere Liste sein oder auch
Elemente enthalten, ist die Intuition erfasst, dass es in Sprachen mit freier Konstituentenstellung
weniger Beschränkungen gibt, als in Sprachen mit fester Konstituentenstellung. Man vergleiche das
mit der Analyse, die \citet{Kayne94a-u} im Minimalistischen\is{Minimalistisches Programm} Rahmen ausgearbeitet hat. Kayne geht
davon aus, dass alle Sprachen aus der [Spezifikator [Kopf Komplement]]"=Grundstellung abgeleitet sind
(\citet{Laenzlinger2004a} hat eine entsprechende Analyse für das Deutsche als
SVO"=Sprache vorgestellt). In solchen Analysen sind Sprachen wie
Englisch der einfache Fall und Sprachen mit freier Stellung erfordern einen erheblichen
theoretischen Aufwand, wohingegen man in der hier vorgestellten Analyse stärkere theoretische
Beschränkungen formulieren muss, wenn die Sprache stärkeren Restriktionen in Bezug auf die
Umstellbarkeit von Konstituenten unterliegt. Auch die Komplexität der lizenzierten Strukturen
unterscheidet sich in der HPSG nicht wesentlich von Sprache zu Sprache. Die Sprachen unterscheiden
sich lediglich in der Art der Verzweigung.\footnote{ 
Das schließt natürlich nicht aus, dass die
  jeweiligen Strukturen von Menschen entsprechend der Art der Verzweigung verschieden gut zu
  verarbeiten sind. Siehe \citew{Gibson98a,Hawkins99a} und \citet[Abschnitt~11.3]{MuellerGTBuch1}.
}$^,$\footnote{%
\citet[\page 18]{Haider97c} hat darauf hingewiesen, dass sich VX"=Sprachen von XV"=Sprachen bei
Analysen wie der hier vorgestellten in der Art der Verzweigung unterscheiden und dass das bei bestimmten
theoretischen Annahmen Auswirkungen auf die Bindungstheorie\is{Bindungstheorie} hat, \dash auf die
Beschränkungen bzgl.\ möglicher Bezüge referentieller Ausdrücke. Für HPSG"=Analysen
ist die Verzweigungsrichtung jedoch irrelevant, denn die Bindungsprinzipien sind mit Hilfe von
o"=Kommando\is{o"=Kommando} definiert \citep[Kapitel~6]{ps2} und o"=Kommando nimmt Bezug auf die
Obliqueness"=Hierarchie\is{Obliqueness}, \dash auf die Reihenfolge der Elemente in der \argstl oder
einer speziellen Liste für Bindungsverhältnisse und nicht auf die Richtung, in der diese Elemente
mit ihrem Kopf kombiniert werden. Zu einer Übersicht über die verschiedenen Ansätze bzgl.\ Bindung
in der HPSG siehe \citew{MuellerBinding}. 
}


\section{Linearisierungsregeln}

Wie\is{Linearisierung!-sregel|(} auf der Seite~\pageref{abstraktion-linearisierung} bei der Diskussion der Regelschemata in
(\ref{abstraktion-linearisierung}) angemerkt, sind die Dominanzschemata in der HPSG abstrakte
Repräsentationen, die nur etwas über die Bestandteile einer Phrase (unmittelbare Dominanz)
aussagen, jedoch keine Beschränkungen über die Abfolge von Töchtern (lineare Präzedenz\is{Präzedenz})
enthalten. Eine solche Trennung zwischen unmittelbarer Dominanz\is{Dominanz} (\emph{immediate dominance})
und linearer Abfolge (\emph{linear precedence}) gab es schon in der \gpsg
\citep*{GKPS85a}. Grammatiken, die eine solche Trennung zwischen \emph{immediate dominance} und
\emph{linear precedence} vornehmen, nennt man \emph{ID/LP"=Grammatiken}. Regeln, die die lineare Abfolge
beschränken, werden \emph{LP"=Regeln}\is{LP"=Regel} genannt.

Zur Motivation dieser Trennung erinnern wir uns an die einfache Phrasenstrukturgrammatik,
die im Abschnitt~\ref{sec-psg} auf Seite~\pageref{ditrans-ps-regeln} diskutiert wurde. Will man
mit solchen Phrasenstrukturregeln ausdrücken, dass alle Abfolgen in (\ref{bsp-perm-mf}) möglich sind,
dann braucht man für ditransitive Verben die folgenden sechs Phrasenstrukturregeln:
\ea
\begin{tabular}[t]{@{}l@{ }l@{ }l@{ }l@{ }l}
S  & $\to$ NP[nom],& NP[dat], & NP[acc], & V\\
S  & $\to$ NP[nom],& NP[acc], & NP[dat], & V\\
S  & $\to$ NP[acc],& NP[nom], & NP[dat], & V\\
S  & $\to$ NP[acc],& NP[dat], & NP[nom], & V\\
S  & $\to$ NP[dat],& NP[nom], & NP[acc], & V\\
S  & $\to$ NP[dat],& NP[acc], & NP[nom], & V\\
\end{tabular}
\z
Die Gemeinsamkeit, die diese sechs Regeln haben, wird in (\mex{0}) nicht erfasst. Abstrahiert man
dagegen von der linearen Abfolge und lässt Regeln nur etwas über Dominanz aussagen, so erhält man
statt der Regeln in (\mex{0}) nur die eine Regel in (\mex{1}):
 \ea
S $\to$ NP[nom] NP[dat] NP[acc] V
\z
(\mex{0}) besagt, dass der S"=Knoten drei Nominalphrasen und ein Verb in beliebiger Reihenfolge
dominiert. Das ist natürlich zu wenig restriktiv, da so auch zugelassen wird, dass das Verb an beliebiger
Stelle zwischen den Nominalphrasen steht. Entsprechend formulierte Linearisierungsbeschränkungen
schließen dann Abfolgen wie NP[nom] NP[acc] V NP[dat] aus.

\subsection{Töchter und phonologischer Beitrag}

Zu Beschränkungen in Bezug auf die Abfolge wurde in der in den vorigen Kapiteln entwickelten
HPSG"=Grammatik noch nichts gesagt. Betrachtet man das
Schema~\ref{schema-Kopf-Komplementschema-prel2} auf S.\,\pageref{schema-Kopf-Komplementschema-prel2}, so sieht man,
dass die Kopf"|tochter und die Nicht"=Kopf"|töchter im Schema übereinander und nicht in irgendeiner
Reihenfolge nebeneinander stehen. Durch das Schema wird also keine Reihenfolge festgelegt. Bei binär
verzweigenden Strukturen mit Kopf gibt es somit zwei Möglichkeiten für die Anordnung der
Konstituenten in der nach Reihenfolge geordneten \dtrsl:
\begin{itemize}
\item \begin{tabular}[t]{@{}ll@{}}
Kopf kommt zuerst:                                   & Beispiel:\\
%
\ms{
head-dtr & \ibox{1}\\
non-head-dtrs & \nliste{ \ibox{2} } \\
dtrs & \nliste{ \ibox{1}, \ibox{2} }
}&%
\onems{
head-dtr \ibox{1}\\
non-head-dtrs \nliste{ \ibox{2} } \\
dtrs \nliste{ \ibox{1} [\phon \phonliste{ schläft }], \ibox{2} [\phon \phonliste{ Aicke }]}
}\\
\end{tabular}
\item \begin{tabular}[t]{@{}ll@{}}
Kopf kommt zum Schluss:                            & Beispiel:\\
%
\ms{
head-dtr & \ibox{1}\\
non-head-dtrs & \nliste{ \ibox{2} } \\
dtrs & \nliste{ \ibox{2}, \ibox{1} }
}&%
\onems{
head-dtr \ibox{1}\\
non-head-dtrs \nliste{ \ibox{2}  } \\
dtrs \nliste{ \ibox{2} [\phon \phonliste{ Aicke }], \ibox{1} [\phon \phonliste{ schläft }] }
}\\
\end{tabular}
\end{itemize}
Da die Elemente in der \dtrsl entsprechend ihrer Reihenfolge in der jeweiligen Phrase angeordnet sind, kann man
den phonlogischen Beitrag der Phrase einfach bestimmen: Er ergibt sich aus der Verkettung der
jeweiligen phonologischen Beiträge in der \dtrsl:
\ea
\avm{ 
[phon & \1 \+ \2\\
 dtrs & < [phon \1], [phon \2] > ]
}
\z
Oder allgemeiner:
\ea
\type{phrase} \impl\\
\avm{
[ phon & \texttt{collect-phon}(\ibox{1})\\
  dtrs & \ibox{1}]
}
\z
\texttt{collect-phon} geht dabei durch die Liste der Töchter und sammelt analog zu
\texttt{collet-rels} auf S.\,\pageref{constraint-collect-rels} die \phonwe auf.
%
% \begin{itemize}
% \item \begin{tabular}[t]{@{}ll@{}}
% der Kopf kommt zuerst:                                   & Beispiel:\\
% %
% \ms{
% phon & \ibox{1} $\oplus$ \ibox{2}\\
% head-dtr & \onems{ phon \ibox{1}}\\
% non-head-dtrs & \liste{ \onems{ phon \ibox{2} }} \\
% }&%
% \ms{
% phon & \phonliste{ schläft, Karl}\\
% head-dtr & \onems{ phon \phonliste{ schläft } }\\
% non-head-dtrs & \phonliste{ \onems{ phon \phonliste{ Karl } }} \\
% }\\
% \end{tabular}
% \item \begin{tabular}[t]{@{}ll@{}}
% der Kopf kommt zum Schluss:                            & Beispiel:\\
% %
% \ms{
% phon & \ibox{2} $\oplus$ \ibox{1}\\
% head-dtr & \onems{ phon \ibox{1}}\\
% non-head-dtrs & \liste{ \onems{ phon \ibox{2} }} \\
% }&%
% \ms{
% phon & \phonliste{Karl, schläft }\\
% head-dtr & \onems{ phon \phonliste{ schläft } }\\
% non-head-dtrs & \liste{ \onems{ phon \phonliste{ Karl } }} \\
% }\\
% \end{tabular}
% \end{itemize}

\subsection{Linearisierungsregeln für Komplemente}

Mit den bisher formulierten Beschränkungen kann man also auch Phrasen wie die in (\mex{1}) und (\mex{2}) ableiten:
\eal
\ex[*]{
{}[[den Schrank] in]
}
\ex[*]{
{}dass [er [es [gibt ihm]]]
}
\zl
In (\mex{0}a) steht die Präposition \emph{in} rechts der Nominalphrase, obwohl sie eigentlich links von
ihr stehen müsste, und in (\mex{0}b) handelt es sich um einen Verbletztsatz, in dem aber \emph{gibt} und \emph{ihm}
vertauscht sind, \dash, eigentlich müsste das Verb an letzter Stelle stehen. 

Linearisierungsregeln sagen etwas über die Reihenfolge von zwei beschriebenen Objekten aus.
Es gibt verschiedene Arten von Linearisierungsregeln: Manche nehmen nur Bezug auf Merkmale der jeweiligen
Objekte, andere beziehen sich auf die syntaktische Funktion (Kopf\is{Kopf}, Argument\is{Argument}, Adjunkt\is{Adjunkt}, \ldots) und
wieder andere mischen beides.
Wenn man für alle Köpfe ein Merkmal \textsc{initial}\isfeat{initial} annimmt, dann kann man Köpfe,
die ihren Argumenten vorangehen, den \initialw `+' geben und Köpfen, die ihren Argumenten folgen,
den Wert `--'. Die Linearisierungsregeln in (\mex{1}) sorgen dann dafür, dass ungrammatische
Abfolgen wie die in (\mex{0}a,b) ausgeschlossen sind.
\eal
\ex\label{lp-ini-Komplement} Head[\initial+] $<$ Complement
\ex Complement $<$ Head[\initial --]
\zl
Präpositionen haben den \initialw `+', und Regel (\mex{0}a) erzwingt somit die Anordnung des präpositionalen
Kopfes vor dem nominalen Argument. Verben in Letztstellung haben den \initialw `--'. (\mex{0}b) schließt
also die Abfolge in (\mex{-1}b) aus.

Die Konsequenz der Linearisierungsregel in (\ref{lp-ini-Komplement}) ist, dass jetzt nur noch die beiden 
Kopf"=Komplement"=Strukturen durch die Grammatik lizenziert werden, die durch (\mex{1}a) und
(\mex{1}b) beschrieben werden:\footnote{
  Sätzen wie in (i) werden im Kapitel~\ref{Kapitel-nla} besprochen.
  \ea
  Das Buch kennt jeder.
  \zlast
}
\eal
\ex \onems{
cat$|$comps \ibox{1} $\oplus$ \ibox{2}\\
head-dtr \ibox{3} \ms{ cat  & \ms{ head$|$initial $+$\\
                                   comps \ibox{1} $\oplus$ \sliste{ \ibox{4} } $\oplus$ \ibox{2} \\
                     }\\
            }\\
non-head-dtrs \sliste{ \ibox{4} }\\
dtrs \sliste{ \ibox{3}, \ibox{4} }
}
\ex \onems{
cat$|$comps \ibox{1} $\oplus$ \ibox{2}\\
head-dtr \ibox{3} \ms{ cat  & \ms{ head$|$initial $-$\\
                                   comps \ibox{1} $\oplus$ \sliste{ \ibox{4} } $\oplus$ \ibox{2} \\
                     }\\
            }\\
non-head-dtrs \sliste{ \ibox{4} }\\
dtrs \sliste{ \ibox{4}, \ibox{3} }
}
\zl
Diese beiden Beschreibungen entsprechen dem Schema~\ref{schema-Kopf-Komplementschema-prel2},
enthalten aber zusätzlich Information über den \initialw der
Kopf"|tochter und die entsprechende Verknüpfung der \phonwe
der Töchter: In (\mex{0}a) stehen alle Elemente aus der \phonl der Kopf"|tochter
links der Elemente aus der \phonl der Nichtkopftochter. In (\mex{0}b)
ist es andersherum.


\subsection{Linearisierungsregeln für Adjunkte}

Genauso müssen die Abfolgen in (\mex{1}) ausgeschlossen werden:
\eal
\label{bsp-Linearisierung-Adjunkt}
\ex[*]{
dass [er [es [ihm [gibt nicht]]]]
}
\ex[*]{
{}[der [Mann kluge]]
}
\ex[*]{
{}[das [[am Wald] Haus]]
}
\zl
Die beiden Regeln in (\mex{1}) sorgen dafür, dass Prämodifikatoren immer vor den Köpfen, die sie modifizieren,
angeordnet werden und dass Postmodifikatoren immer nach den Köpfen angeordnet werden.\isfeat{pre"=modifier}
\eal
\label{lp-head-argument-initial}
\ex Adjunct[\textsc{pre-modifier} +] $<$ Head
\ex Head $<$ Adjunct[\textsc{pre-modifier} --]
\zl
Die erste Linearisierungsregel sorgt dafür, dass die Phrasen in (\ref{bsp-Linearisierung-Adjunkt}a,b) ausgeschlossen sind, und die zweite
schließt (\ref{bsp-Linearisierung-Adjunkt}c) aus. (\mex{1}) zeigt die entsprechenden Anordnungen der
Elemente in der \dtrsl, die möglich sind:
\eal
\ex \onems{
head-dtr \ibox{3} \\
non-head-dtrs \sliste{ \ibox{4} \ms{ cat|head \ms{ pre-modifier $+$\\
                                                   mod \ibox{3} }\\ 
                     } }\\
dtrs \sliste{ \ibox{4}, \ibox{3} }
}
\ex \onems{
head-dtr \ibox{3} \\
non-head-dtrs \sliste{ \ibox{4} \ms{ cat|head \ms{ pre-modifier $-$\\
                                                   mod \ibox{3} }\\ 
                     } }\\
dtrs \sliste{ \ibox{3}, \ibox{4} }
}
\zl


Bevor wir uns den Spezifikator"=Kopf"=Strukturen zuwenden, möchte ich noch kurz
auf die Stellung von Adjunkten in verbalen Projektionen eingehen. Sie können überall zwischen den
Komplementen des Verbs stehen, wie die Beispiele in (\ref{bsp-adj-mf}) zeigen. Die Sätze kann man analysieren,
indem man annimmt, dass ein Adjunkt sich prinzipiell mit einer beliebigen Verbprojektion
verbinden kann. In (\mex{1}) sind entsprechende Beispielstrukturen zu sehen:
\eal
\label{bsp-adj-mf-zwei}
\ex weil morgen [Aicke dem Affen den Stock gibt]
\ex weil Aicke morgen [dem Affen den Stock gibt]
\ex weil Aicke dem Affen morgen [den Stock gibt]
\ex weil Aicke dem Affen den Stock morgen gibt
\zl
In (\mex{0}a) wurde \emph{gibt} mit all seinen Argumenten kombiniert und erst danach
mit dem Adverb\is{Adverb}. In (\mex{0}b) wurde \emph{gibt} mit den Objekten kombiniert. Nach der
Modifikation der Phrase \emph{dem Affen den Stock gibt} durch \emph{morgen} wird dann
in einem weiteren Schritt das Subjekt gesättigt. (\mex{0}c) und (\mex{0}d) zeigen Beispiele mit kleineren
Projektionen von \emph{gibt}. Genauso lassen sich natürlich die Sätze in (\mex{1})
analysieren:
\eal
\ex weil Aicke den Stock morgen [dem Affen gibt]
\ex weil Aicke den Stock dem Affen morgen gibt
\zl
Diese Sätze sind Varianten von (\mex{-1}c) und (\mex{-1}d), die sich von den ersten Sätzen
nur durch die Anordnung der Objekte unterscheiden. Da das Kopf"=Komplement"=Schema die Kombination
eines Kopfes mit seinen Komplementen in beliebiger Reihenfolge gestattet, kann auch wie in (\mex{0}a)
das Dativobjekt mit dem Verb kombiniert werden, bevor das Adverb die entsprechende Wortgruppe
modifiziert.%
\is{Linearisierung!-sregel|)}

%% Wie \citet[\page25--26]{HR98a}\NOTE{Seitenzahl prüfen} und \citet[\page214--215]{Fanselow2003b}
%% gehe ich also davon aus, dass Adjunkte nicht wie Argumente variabel angeordnet werden, sondern
%% dass Adjunkte im Mittelfeld mit beliebig komplexen Projektionen kombiniert werden können, woraus
%% sich dann der Eindruck der Stellungsfreiheit ergibt.

\section{Linearisierungsregeln für Spezifikatoren}
\label{sec-spr}

Im vorigen Abschnitt wurden Kopf"=Komplement"=Strukturen mit dem Kopf in Erst-
bzw.\ Letztstellung diskutiert. Betrachtet man die Nominalphrase in (\mex{1}),
sieht man, dass der Determinator\is{Determinator!Stellung|(} links des Nomens steht und alle anderen abhängigen
Elemente rechts.
\ea
\label{ex-die-Zerstörung-der-Stadt-dds}
die Zerstörung der Stadt durch die Soldaten
\z
In früheren Varianten der HPSG wurden alle Argumente eines Kopfes in einer Liste, der \subcatl,
repräsentiert \parencites[Kapitel~1--8]{ps2}[Abschnitt~2.2]{Mueller99a}.
Man kann diese Fälle nicht ohne Weiteres mit dem Kopf"=Komplement"=Schema behandeln, da die Linearisierungsregel
(\ref{lp-head-argument-initial}) je nach \textsc{initial}"=Wert des Nomens die Stellungen in (\mex{1})
erzwingen würde:
\eal
\ex[*]{
Zerstörung die der Stadt durch die Soldaten
}
\ex[*]{
die der Stadt durch die Soldaten Zerstörung
}
\zl
Die Beispiele (\mex{-1}) und (\mex{0}b) zeigen, dass die Argumente, die semantisch von \emph{Zerstörung} abhängen, rechts des
Nomens stehen müssen, was für den \textsc{initial}"=Wert `+' für Nomina spricht. \citet[\page
164--165]{Mueller99a} verwendet deshalb spezifische Regeln, die besagen, dass Determinatoren vor Nomina
stehen müssen und Nicht-Determinatoren Nomina folgen müssen.

Würde man annehmen, dass der Determinator\is{Determinator!-phrase} der Kopf ist, bekäme man Strukturen,
in denen die Argumente immer nur auf einer Seite des Kopfes stehen. (\mex{1}) zeigt die
Klammerstruktur für die DP"=Analyse von (\ref{ex-die-Zerstörung-der-Stadt-dds}):
\ea
{}[\sub{DP} [\sub{Det} die] [\sub{NP} [\sub{N} Zerstörung] [\sub{DP} der Stadt] [\sub{PP} durch die Soldaten]]]
\z
\itdopt{check DP und Possesiva}
%Im Kapitel~\ref{sec-dp-analyse}
\citet{MuellerHeadless} und \citet{MyPM2021a} argumentieren jedoch gegen DP-Analysen, weshalb ich
bei der in der HPSG auch ansonsten üblichen NP"=Analyse geblieben bin (siehe \citealt{VanEyndeNominalStructures}
für einen Überblick).
Auch ist es nicht so, dass diese Art Abfolge etwas völlig Ungewöhnliches ist.
In Subjekt"=Verb"=Objekt"=Sprachen wie dem Englischen steht das Subjekt immer links des Verbs, aber alle
anderen Argumente stehen rechts. Diesen Unterschieden in der Verzweigungsrichtung wird
durch eine eigene Grammatikregel Rechnung getragen \citep[\page 38, \page 362]{ps2}.
Ich nehme also -- wie schon in Kapitel~\ref{Abschnitt-SPR-Merkmal} dargelegt -- an, dass es für Determinatoren (allgemeiner Spezifikatoren) ein eigenes
Valenzmerkmal (\textsc{spr}\isfeat{spr}) gibt. Die entsprechende Analyse von
\pref{ex-die-Zerstörung-der-Stadt-dds} zeigt Abbildung~\vref{Abbildung-die-Zerstorung}.
\begin{figure}
\begin{forest}
sm edges
[N\feattab{\spr \eliste,\\
           \comps \eliste}
  [\ibox{1} Det [die]]
  [N\feattab{\spr \sliste{ \ibox{1} },\\
           \comps \eliste}
    [N\feattab{\spr \sliste{ \ibox{1} },\\
           \comps \sliste{ \ibox{2} }}
      [N\feattab{\spr \sliste{ \ibox{1} },\\
           \comps \sliste{ \ibox{2}, \ibox{3} }}
         [Zerstörung]]
      [\ibox{3} {NP[\type{gen}]}
        [der Stadt,roof]]]
    [\ibox{2} {PP[\type{durch}]}
      [durch die Soldaten,roof]]]]
\end{forest}
\caption{NP-Analyse mit Valenzmerkmal \spr}\label{Abbildung-die-Zerstorung} 
\end{figure}
Die Kombination einer \nbar mit dem Determinator wird dann durch das folgende Schema lizenziert, das analog zum Kopf"=Argument"=Schema
auf Seite~\pageref{schema-Kopf-Komplementschema-prel2} ist.
\begin{schema}[Spezifikator-Kopf-Schema]\is{Schema!Spezifikator"=Kopf"=}
\label{schema-spr-h}
\type{head-specifier"=phrase}\istype{head"=specifier"=phrase} \impl\\
\onems{
      cat$|$spr \ibox{1} \\
      head-dtr$|$cat  \ms{ spr    & \ibox{1} $\oplus$ \sliste{ \ibox{2} } \\
                           comps & \eliste \\
                         }\\
      non-head-dtrs \sliste{ \ibox{2} }\\
      }
\end{schema}

\noindent
Für das Nomen \emph{Zerstörung} wird der folgende \catw angenommen:
\ea
\ms{ head & \ms[noun]{ initial & $+$\\
                     }\\
     spr & \nliste{ Det }\\
           comps & \sliste{ NP[\gen], PP[\type{durch}] }\\
         }
\z
Das Kopf"=Komplement"=Schema lizenziert die Phrasen  \emph{Zerstörung der Stadt}
und  \emph{Zerstörung der Stadt durch die Soldaten}. Die Phrase  \emph{Zerstörung der Stadt durch
  die Soldaten} hat eine leere \compsl und einen Determinator in \textsc{spr}. Damit kann sie als
Kopf"|tochter in das Schema~\ref{schema-spr-h} eingesetzt werden und dann mit einem Determinator zur
Nominalphrasen in (\ref{ex-die-Zerstörung-der-Stadt-dds}) kombiniert werden. Zur Analyse von
Nominalstrukturen siehe auch Kapitel~\ref{Abschnitt-Valenzprinzip}.

Die Anordnung der Elemente in der NP wird dann durch die Linearisierungsregel in (\ref{lp-ini-Komplement})
zusammen mit der LP"=Regel in (\mex{1}) erzwungen, die besagt, dass Spezifikatoren vor ihren
jeweiligen Köpfen stehen:
\ea
\label{lp-Specifier-Head}
Specifier $<$ Head
\z
Damit steht die Nicht"=Kopftochter immer vor der Kopftochter:
\ea
\onems{
cat$|$spr \ibox{1}\\
head-dtr \ibox{2} \ms{ cat|spr \ibox{1} $\oplus$ \sliste{ \ibox{3} } \\
                     }\\
non-head-dtrs \sliste{ \ibox{3} }\\
dtrs \sliste{ \ibox{3}, \ibox{2} }
}
\z

Die Regel in (\ref{lp-ini-Komplement}) sorgt dafür, dass \emph{Zerstörung} vor allen Elementen steht, die in seiner
\compsl enthalten sind, \dash vor \emph{der Stadt} und \emph{durch die Soldaten}, und die Regel in
(\ref{lp-Specifier-Head}) sorgt dafür, dass der Determinator \emph{die} vor \emph{Zerstörung der Stadt durch die
Soldaten} angeordnet wird.\is{Determinator!Stellung|)}

Im folgenden Abschnitt wird die Analyse der Verbstellung erklärt. Das ist wohl der komplexeste Teil
des Buches. Ich würde gern, wie im Abschnitt~\ref{Abschnitt-Skopus} über Unterspezifikation von Qunatoren"=Skopus, die
Leser*innen davon abzuhalten versuchen, den folgenden Abschnitt zu lesen, aber leider ist die
Analyse der Verbstellung so zentral, dass jeder und jede diesen Abschnitt lesen muss.

\section{Verberststellung}
\label{sec-v1}\label{Abschnitt-V1}

Das Deutsche zählt typologisch zu den SOV"=Sprachen\is{SOV}\is{Verbletztsprache|(}. Innerhalb der
Transformationsgrammatik\is{Government and Binding (GB)@\textit{Government and Binding\/} (GB)} 
wird deshalb angenommen, dass die Stellung Subjekt Objekt Verb die zugrundeliegende Stellung ist \parencites{Bach62a}[\page34]{Bierwisch63a}{Reis74a}[Kapitel~1]{Thiersch78a}.\footnote{%
  Bierwisch schreibt die Annahme einer zugrundeliegenden Verbletztstellung Fourquet
  zu \citep{Fourquet57a}. Eine Übersetzung des von Bierwisch zitierten
  französischen Manuskripts kann man in \citew[\page117--135]{Fourquet70a}
  finden.%
}
Sätze wie (\mex{1}b, c), bei denen das finite Verb an der ersten oder zweiten Stelle steht,
gelten als aus Verbletztsätzen durch Umstellung des finiten Verbs abgeleitet.
\eal
\ex dass Aicke dem Affen gestern den Stock gegeben hat
\ex Hat Aicke dem Affen gestern den Stock gegeben?
\ex Aicke hat dem Affen gestern den Stock gegeben.
\zl
Dabei wird folgendes Bild verwendet: Das finite Verb kämpft mit der Konjunktion um die linke Satzklammer:
Wenn die Konjunktion in der linken Satzklammer steht, muss das finite Verb in die rechte Satzklammer.
Ansonsten steht das Finitum in der linken Satzklammer.\footnote{
  Dieses Bild trifft nicht für alle germanischen Sprachen zu. Zum Beispiel im Jiddischen werden
  Komplementierer mit Verbzweitsätzen kombiniert \citep[\page 196]{Diesing2004a}.
} Man geht davon aus, dass die subordinierende Konjunktion
der Kopf ist. Diese wird auch Komplementierer\is{Komplementierer} genannt.

Ähnliche Ansätze gibt es auch in der kategorialgrammatischen GPSG"=Variante von \citet[\page
110]{Jacobs86a}, innerhalb der Lexical Functional Grammar \citep{Berman96a-u} und innerhalb der HPSG
\label{Seite-V1-via-Lexikonregel}\parencites{KW91a}{Oliva92b}{Netter92}{Frank94}{Kiss95a}{Feldhaus97}{Meurers2000b}{Mueller2005c,MuellerGS}.\addpages

Die Annahme der Verbletztstellung\label{page-verbletzt} als Grundstellung wird durch folgende Beobachtungen
motiviert:\footnote{
  Zu den Punkten 1 und 2 siehe auch \citew[\page34--36]{Bierwisch63a}. Zum Punkt
  3 siehe \citew[Abschnitt~2.3]{Netter92}.%
}
\begin{enumerate}
\item Sogenannte Verbzusätze oder Verbpartikel bilden mit dem Verb eine enge Einheit.
\eal
\ex weil er morgen anfängt
\ex Er fängt morgen an.
\zl
Diese Einheit ist nur in der Verbletztstellung zu sehen, was dafür spricht, diese
Stellung als Grundstellung anzusehen. 

Verben, die aus einem Nomen durch \wis{Rückbildung} entstanden sind, können oft nicht
in ihre Bestandteile geteilt werden, und Verbzweitsätze sind dadurch ausgeschlossen:
\eal
\ex[]{
weil sie das Stück heute uraufführen
}
\ex[*]{
Sie uraufführen heute das Stück.
}
\ex[*]{
Sie führen heute das Stück urauf.
}
\zl
\itdopt{vorausdrucken, verdreifachen}
Es gibt also nur die Stellung, von der man auch annimmt, dass sie die Grundstellung ist.
\item Verben in infiniten Nebensätzen und in durch eine Konjunktion eingeleiteten
finiten Nebensätzen stehen immer am Ende (von Ausklammerungen ins Nachfeld abgesehen):
\eal
\ex Der Clown versucht, Kurt-Martin die Ware zu geben.
\ex dass der Clown Kurt-Martin die Ware gibt
\zl
\item\is{Skopus|(} Im Abschnitt~\ref{sec-mf} wurden die Beispiele (\ref{bsp-absichtlich-nicht}) diskutiert,
und es wurde darauf hingewiesen, dass die Skopusbeziehung der Adverbien
von ihrer Reihenfolge abhängt. Das links stehende Adverb hat Skopus über das folgende Adverb
und das Verb in Letztstellung.\footnote{%
An dieser Stelle darf nicht verschwiegen werden,
  dass es Ausnahmen von der Regel zu geben scheint, dass weiter links stehende Modifikatoren
  Skopus über Modifikatoren rechts von ihnen haben. \citet*[\page47]{Kasper94a}
  diskutiert die Beispiele in (i), die auf  \citet*[\page137]{BV72} zurückgehen.
\eal
\label{bsp-peter-liest-gut-wegen}
\ex Peter liest gut wegen der Nachhilfestunden.
\ex Peter liest wegen der Nachhilfestunden gut.
\zl
% Kiss95b:212
  Wie \citet[Abschnitt~6]{Koster75a} und \citet*[\page67]{Reis80a} gezeigt haben, sind diese Daten jedoch nicht
  zwingend, weil in den Beispielen die rechte Satzklammer nicht besetzt ist, und es sich
  somit nicht zwangsläufig um eine normale Umstellung im Mittelfeld handeln muss, sondern um 
  eine Extraposition\is{Extraposition} handeln kann. Wie Koster und Reis festgestellt haben,
  sind die von Kasper diskutierten Beispiele sogar ungrammatisch,
  wenn man die rechte Satzklammer ohne Ausklammerung der Kausalbestimmung besetzt:
\eal
\ex[*]{
Hans hat gut wegen der Nachhilfestunden gelesen.
}
\ex[]{
Hans hat gut gelesen wegen der Nachhilfestunden.
}
\zl
  Das folgende Beispiel von \citet[\page 383]{Crysmann2004a}
  zeigt allerdings, dass auch bei besetzter Satzklammer eine Anordnung der Adjunkte
  möglich ist, in der ein weiter rechts stehendes Adjunkt Skopus über ein links stehendes hat:
\ea
Da muss es schon erhebliche Probleme mit der Ausrüstung gegeben haben, da wegen schlechten
  Wetters ein Reinhold Messmer niemals aufgäbe.
%\ex Stefan  ist wohl deshalb krank geworden, weil er äußerst hart wegen der Konferenz in Bremen gearbeitet hat.
\z
Das ändert jedoch nichts an der Tatsache, dass die entsprechenden
Sätze in (\ref{bsp-absichtlich-nicht-anal}) und (\ref{bsp-absichtlich-nicht-anal-v1}) unabhängig von der Verbstellung
dieselbe Bedeutung haben. Die allgemeine Bedeutungskomposition muss man jedoch eventuell
auf die von Crysmann beschriebene Weise behandeln.%
\itdopt{check wie Berthold das gemacht hat}
}
Das wurde dann mittels folgender Struktur erklärt:
\eal
\label{bsp-absichtlich-nicht-anal}
\ex weil er [absichtlich [nicht lacht]]
\ex weil er [nicht [absichtlich lacht]]
\zl
Nun kann man feststellen, dass sich bei Verberststellung die Skopusverhältnisse nicht ändern.
Nimmt man an, dass Sätze mit Verberststellung eine Struktur haben, die der in (\mex{0})
ähnelt, dann ist diese Tatsache automatisch erklärt. Eine solche Parallelität kann man
erreichen, indem man ein leeres Element annimmt, das den Platz des Verbs in (\mex{0}) füllt
und bis auf die Tatsache, dass es phonologisch leer ist, identisch
mit dem normalen Verb ist, \dash, es hat dieselbe Valenz und leistet auch denselben semantischen
Beitrag. Dieses leere Element (auch \emph{Spur}\is{Spur!Verb-|(} oder \emph{Lücke}
oder \emph{Trace} bzw.\ \emph{Gap} genannt)
ist in (\mex{1}) als \_$_i$ gekennzeichnet. 
Dass das Verb \emph{lacht} zu dieser Spur gehört, wird durch den gemeinsamen
Index gekennzeichnet.
\eal
\label{bsp-absichtlich-nicht-anal-v1}
\ex Lacht$_i$ er [absichtlich [nicht \_$_i$]]?
\ex Lacht$_i$ er [nicht [absichtlich \_$_i$]]?
\zl\is{Skopus|)}
%\item Verum-Fokus
\nocite{Hoehle88a,Hoehle97a}
\end{enumerate}\is{Verbletztsprache|)}
Zur Verdeutlichung soll die Analyse des Satzes in (\ref{bsp-kennt-jemand-den-Roman}) erklärt werden:
\eal
\ex dass jemand den Roman kennt\label{bsp-dass-jemand-den-Roman-kennt}
\ex Kennt$_i$ jemand den Roman \_$_i$?\label{bsp-kennt-jemand-den-Roman}
\zl
Für die Spur in (\mex{0}b) könnte man den folgenden Lexikoneintrag annehmen:\todostefan{Zehui:
  Welchen Typ hat die Verbspur?}
\eas
Verbspur für \emph{kennt}:\\
\ms{
phon & \phonliste{}\\
cat  & \ms{ head & \ms[verb]{ vform & fin\\
                            }\\
            comps & \sliste{ \npnom\ind{1}, \npacc\ind{2} }\\
          }\\
cont & \ms{
       ind  & \ibox{3}\\
       ltop & \ibox{4}\\
       rels & \liste{ \ms[kennen]{
                       lbl & \ibox{4}\\
                       arg0 & \ibox{3}\\
                       arg1 & \ibox{1}\\
                       arg2 & \ibox{2}\\
                     } } }\\
}
\zs
Dieser Lexikoneintrag unterscheidet sich vom normalen Verb \emph{kennt} nur in seinem \phonw.

Die syntaktischen Aspekte der Analyse von (\ref{bsp-kennt-jemand-den-Roman}) sind in Abbildung~\vref{verb-movement-syn-simple}
dargestellt. 
\begin{figure}
\begin{forest}
sm edges
[{V[\comps \eliste]}
  [V [kennt]]
  [{V[\comps \eliste]}
    [{\ibox{1} NP[\type{nom}]}
      [jemand]]
    [{V[\comps \sliste{ \ibox{1} } ]}
      [{\ibox{2} NP[\type{acc}]}
        [den Roman, roof]]
      [{V[\comps \sliste{ \ibox{1}, \ibox{2} } ]}
        [\trace]]]]]
\end{forest}
\caption{\label{verb-movement-syn-simple}Analyse von \emph{Kennt jemand den Roman?}}
\end{figure}
Die Kombination der Spur mit \emph{den Roman} und \emph{jemand} folgt den Regeln und Prinzipien, die
wir bisher kennengelernt haben. Es stellt sich aber sofort die Frage, wodurch das Verb \emph{kennt}
in Abbildung~\ref{verb-movement-syn-simple} lizenziert wird und welchen Status es hat.

Will man erfassen, dass sich das finite Verb in Initialstellung wie ein Komplementierer verhält
\citep{HoehleKomplementierer}, so liegt es nahe, \emph{kennt} in
Abbildung~\ref{verb-movement-syn-simple} Kopfstatus zuzuschreiben und \emph{kennt} eine gesättigte
Verbalprojektion mit Verbletztstellung selegieren zu lassen. Finite Verben in Initialstellung
unterscheiden sich dann von Komplementierern dadurch, dass sie eine Projektion einer Verbspur
verlangen, wohingegen Komplementierer Projektionen von overten Verben verlangen:
\eal
\ex dass [jemand den Roman kennt]
\ex Kennt [jemand den Roman \_ ]
\zl
Nun ist es aber normalerweise nicht so, dass \emph{kennen} einen vollständigen Satz selegiert und
sonst nichts weiter, wie es zur Analyse des Verberstsatzes mit \emph{kennt} als Kopf notwendig
wäre. Auch muss sichergestellt werden, dass die Verbalprojektion, mit der \emph{kennt} kombiniert
wird, genau die zu \emph{kennt} gehörige Verbspur enthält. Könnte sie nämlich die Verbspur
enthalten, die zu \emph{gibt} gehört, so würde man Sätze wie (\mex{1}b) analysieren können: 
\eal
\ex[]{ Gibt [Aicke dem Affen den Stock \_$_{gibt}$].  } 
\ex[*]{
\label{bsp-kennt-gibt}
Kennt [Aicke dem Affen den Stock \_$_{gibt}$].
}
\zl
In den obigen Erläuterungen wurde die Zusammengehörigkeit von vorangestelltem Verb und Verbspur
durch eine Koindizierung ausgedrückt. In HPSG wird Identität von Information immer durch Strukturteilung
hergestellt. Das Verb in Initialstellung muss also fordern, dass die Spur genau die Eigenschaften des
Verbs hat, die das Verb hätte, wenn es sich in Letztstellung befände. Die Information, die geteilt
werden muss, ist also sämtliche syntaktische und semantische Information, \dash alle bisher eingeführten
Merkmale bis auf das \phonm. Um die benötigte Strukturteilung vornehmen zu können, strukturieren
wir unsere Merkmalstrukturen stärker: Syntaktische und semantische Information wird unter dem
Pfad \textsc{local}\isfeat{loc} gebündelt. Die Datenstruktur in (\ref{geom-cat-cont}) auf Seite~\pageref{geom-cat-cont}
wird also zu (\mex{1}) erweitert:
\ea
\label{geom-loc}
\ms{ phon & list~of~phoneme strings\\
     loc & \ms{
           cat  & \ms{ head   & head\\
                       spr & list~of~signs\\
                       comps & list~of~signs\\
                     } \\
           cont & cont\\
         }\\
   }
\z

\noindent
Die Verbspur für \emph{kennt} hat dann die folgende Form:

\eas
\label{le-verbspur-kennt-ohne-dsl}%
Verbspur für \emph{kennt}:\\\samepage
\ms{
phon & \phonliste{}\\
loc  & \ms{ cat  & \ms{ head & \ms[verb]{ vform & fin\\
                                          ini   & $-$\\
                                        }\\
                        comps & \nliste{ \npnom\ind{1}, \npacc\ind{2} }\\
                      }\\
            cont & \ms{
       ltop & \ibox{3}\\
       ind  & \ibox{4}\\
       rels & \liste{ \ms[kennen]{
                       lbl & \ibox{3}\\
                       arg0 & \ibox{4}\\
                       arg1 & \ibox{1}\\
                       arg2 & \ibox{2}\\
                     } } }
          }\\
}
\zs

\noindent
Unter \textsc{local} steht nun alle Information, die in lokalen Kontexten eine Rolle spielt.
Diese Information wird zwischen Spur und Verb geteilt. Bisher ist es jedoch noch nicht möglich,
eine entsprechende Strukturteilung vorzunehmen, denn das Verb \emph{kennt} kann ja nur
die Eigenschaften der Projektion der Spur (dem gesamten Satz mit leerem Verb) selegieren und die \compsl der selegierten
Projektion ist die leere Liste, was uns wieder zu dem Problem mit (\ref{bsp-kennt-gibt}) bringt. Man
muss also sicherstellen, dass die gesamte Information über die Verbspur am obersten Knoten
ihrer Projektion verfügbar ist. Das kann man erreichen, indem man ein entsprechendes Kopfmerkmal
einführt, dessen Wert eine \local"=Objekt ist, das alle relevanten Eigenschaften (Valenz und
Semantik) mit der Spur teilt. Dieses Merkmal wird \textsc{dsl} genannt. Es steht für \emph{double
  slash} und wurde so genannt, weil es eine ähnliche Funktion wie das \slashm hat, mit dem wir uns
im nächsten Kapitel beschäftigen werden.\footnote{% 
  Das Merkmal \dsl wurde von \citet*{Jacobson87} zur Beschreibung von Kopfbewegung 
  für englische\il{Englisch} invertierte Strukturen eingeführt. \citet{Borsley89}
  hat diese Idee aufgegriffen, innerhalb der HPSG"=Theorie umgesetzt und 
  auch gezeigt, wie Kopfbewegung im CP/IP"=System der \gbt mittels \textsc{dsl} modelliert werden
  kann.
  Die Einführung des \textsc{dsl}"=Merkmals zur Beschreibung von Bewegungsprozessen
  in die Theorie der HPSG ist dadurch motiviert, dass es sich bei solcherart Bewegung
  im Gegensatz zu Fernabhängigkeiten, wie sie im Kapitel~\ref{Kapitel-nla}
  besprochen werden, um eine lokale Bewegung handelt.

  Der Vorschlag, Information über die Verbspur als Teil der Kopf"|information nach oben zu reichen,
  stammt von \citet[\page 187]{Oliva92b}.%
}
(\mex{1}) zeigt den angepassten Eintrag für die Verbspur:

\eas
\label{le-verbspur-kennt}%
Verbspur für \emph{kennt} mit \dslm für die Projektion von Valenz und semantischen Eigenschaften:\\
\samepage
\ms{
phon & \phonliste{}\\
loc  & \ms{ cat  & \ms{ head & \onems[verb]{ vform \type{fin}\\
                                                      ini   $-$\\
                                                      dsl   \ms{ cat & \ms{ spr   & \ibox{1}\\ 
                                                                            comps & \ibox{2}\\
                                                                         }\\
                                                                 cont & \ibox{3}}}\\
                                 spr   & \ibox{1}\\
                                 comps & \ibox{2} \nliste{ \npnom\ind{4}, \npacc\ind{5} }\\
                      }\\
            cont & \ibox{3} \ms{
              ltop & \ibox{6}\\
              ind  & \ibox{7}\\
              rels & \liste{ \ms[kennen]{
                             lbl & \ibox{6}\\
                             arg0 & \ibox{7}\\
                             arg1 & \ibox{4}\\
                             arg2 & \ibox{5}\\
                           } } } }\\
}
\zs
Durch die Teilung der Werte innerhalb von \dslw mit denen der Verbspur in (\mex{0}) ist die Information
über syntaktische und semantische Information der Verbspur auch an ihrer Maximalprojektion
verfügbar, und das Verb in Erststellung kann sicherstellen, dass die Projektion der Spur zu ihm passt.
\is{Spur!Verb-|)}

Der spezielle Lexikoneintrag für die Verberststellung wird durch die folgende
Regel\is{Lexikonregel!Verberststellung|(} lizenziert:\footnote{
Im Kapitel~\ref{Abschnitt-Lexikonregeln} wurde erklärt, dass man Lexikonregeln auch so formalisieren
kann, dass das Eingabeelement eine Tochter ist. In früheren Publikationen habe ich -- vielen
vorausgehenden Veröffentlichungen folgend (siehe S.\,\pageref{Seite-V1-via-Lexikonregel}) -- die
Verberstregel immer als Lexikonregel erklärt und in einer Fußnote darauf hingewiesen, dass es sich
eigentlich um eine unäre Regel handelt. Es ist nötig, eine syntaktische Regel zu verwenden, da als
Tochter das Ergebnis einer Koordination vorkommen kann:
\ea
Kirby kennt und liebt diese Band.
\z
Für die Analyse von (i) muss die Verberstregel auf \emph{kennt und liebt} angewendet
werden. Lexikonregeln können immer nur auf Lexikonelemente (Stämme oder Wörter) angewendet werden,
\emph{kennt und liebt} ist aber eine Wortgruppe. In diesem Buch werde ich also die Verberstregel
gleich als unäre Regel bzw.\ unäres Schema erklären, damit der Text auch synchron mit der
Computerimplementation ist. 
}
\eas
\label{lr-verb-movement}%
Regel für Verb in Erststellung (vorläufige Version):\\
\type{verb-initial-rule} \impl\\
\ms{
loc$|$cat & \ms{ head & \ms[verb]{vform & fin\\
                                          initial & $+$\\
                                          dsl     & none\\
                                 }\\
                 comps & \sliste{ \onems{ loc$|$cat \onems{ head|dsl \ibox{1}\\
                                                            comps \eliste }}}\\
                         }\\
dtrs & \liste{ \ms{
loc & \ibox{1} \ms{ cat$|$head & \ms[verb]{ vform & fin\\
                                                     initial & $-$\\
                                             }\\
                  }\\
} }
}
\zs

\noindent
Das durch diese Regel lizenzierte Verb selegiert eine maximale Projektion der Verbspur,
die dieselben lokalen Eigenschaften wie das Eingabeverb hat. Das wird durch die
Koindizierung des \localw{}es des Eingabeverbs und des \dslw{}es der selegierten
Verbalprojektion erreicht. Eingabe für die Regel können nur finite Verben in Endstellung
(\textsc{initial}$-$) sein. Die Ausgabe ist ein Verb in Erststellung (\textsc{initial}+). In
früheren Auf"|lagen dieses Buches habe ich in der Valenzliste des Ausgabeverbs noch angegeben, dass das
Komplement eine Verbalprojektion sein muss (\head \type{verb}). Das ist aber nicht nötig, denn der
\dslw enthält ja die Information darüber, dass die Spur den \localw \ibox{1} (finites Verb in Letztstellung) haben
muss. Daraus ergibt sich für die Spur, dass sie ein Verb sein muss, da ja in der Spur der \dslw und der
\localw identifiziert werden, und somit kommen als Komplemente des Verberstverbs nur
Verbalprojektionen in Frage. 
%Linearisierungsregeln nehmen auf das \textsc{initial}"=Merkmal Bezug und sorgen für
%die korrekte Anordnung der Köpfe in lokalen Bäumen.
%
Die Analyse von (\ref{bsp-kennt-jemand-den-Roman}) zeigt Abbildung~\vref{verb-movement-syn}.
\begin{figure}
\begin{forest}
sm edges
[S
  [{V[\comps \sliste{ \ibox{1} }]} 
    [{V[\comps \ibox{2} ]},tier=np,edge label={node[midway,right]{V1-R}} 
       [kennt$_j$] ] ]
    [{\ibox{1} V\feattab{\textsc{dsl|comps} \ibox{2},\\
                 \comps \eliste}}
         [{\ibox{3} NP[\type{nom}]},tier=np [jemand] ]
         [{V\feattab{\textsc{dsl|comps} \ibox{2},\\
                 \comps \sliste{ \ibox{3} }}}
           [{\ibox{4} NP[\type{acc}]} [den Roman, roof] ]
           [{V\feattab{\textsc{dsl|comps} \ibox{2},\\
                 \comps \ibox{2} \sliste{ \ibox{3}, \ibox{4} }}} [\_$_j$] ] ] ] ] ]
\end{forest}
\caption{\label{verb-movement-syn}Veranschaulichung der Analyse von \emph{Kennt jemand den Roman?}}
\end{figure}
V1-R steht für die Verberst"=Regel.


Die Regel in (\mex{0}) lizenziert ein Verb, das einen Satz (\iboxt{1} in
Abbildung~\ref{verb-movement-syn}) selegiert. Der \dslw dieses Satzes entspricht dem \locw der in
die Regel eingesetzten Tochter. Teil des \dslwes ist auch die Valenzinformation, die in
Abbildung~\ref{verb-movement-syn} dargestellt ist \iboxb{2}. Da \dsl ein Kopfmerkmal ist, ist der
\dslw der VP mit dem der Verbspur identisch, und da die Information über Valenz und Semantik im
\locw der Verbspur mit der im \dslw identifiziert ist, ist auch die \comps"=Information des Verbs
\emph{kennen} in der Spur verfügbar. Die Kombination der Spur mit ihren Argumenten erfolgt genau so,
wie das bei normalen Verben der Fall wäre.


In der Regel in (\ref{lr-verb-movement}) wurde noch nichts zur Semantik gesagt. Man geht im allgemeinen davon aus, dass
die Verbspur in Verberstsätzen auch semantisch für das Verb in Erststellung steht und
dass Verberstsätze entsprechend ihren Verbletztgegenstücken interpretiert werden. Dies
wird dadurch modelliert, dass man den semantischen Beitrag parallel zur Valenzinformation
durch den Baum fädelt. 
Bei der Kombination mit Argumenten sorgt das Semantikprinzip dafür, 
dass die Information über den \textsc{index}"=Wert (die Ereignisvariable) und der \ltopw entlang der Kopfprojektion im Baum nach
oben gereicht wird. Im letzten Projektionsschritt in Abbildung~\ref{verb-movement-syn}
ist das Verb in Erststellung der Kopf und deshalb wird auch der semantische Beitrag
dieses Verbs projiziert. In der Regel (\mex{1}) für das Verb in Erststellung wird
der semantische Beitrag der Projektion der Spur in Endstellung (\textsc{ind} \ibox{2}, \ltop \ibox{3}) mit
den entsprechenden Werten des Verbs in Erststellung identifiziert:

\eas
\label{lr-verb-movement2}%
Regel für Verb in Erststellung (mit semantischem Beitrag, vorläufig):\\
\type{verb-initial-rule} \impl\\
\ms{
loc & \ms{ cat & \ms{ head & \ms[verb]{ vform & fin\\
                                      initial & $+$\\
                                      dsl     & none }\\
                    comps & \sliste{ \onems{ loc \ms{ cat & \onems{ head|dsl \ibox{1}\\
                                                                    comps \eliste }\\
                                                      cont & \ms{ ind & \ibox{2}\\
                                                                  ltop & \ibox{3}} }\\
                                              }}\\
                  }\\
            cont & \ms{ ind & \ibox{2}\\
                        ltop & \ibox{3}\\
                        rels & \eliste\\
                        hcons & \eliste} }\\
dtrs & \liste{ \ms{
               loc & \ibox{1} \ms{ cat$|$head & \ms[verb]{ vform & fin\\
                                                           initial & $-$\\
                                                         }\\
                                 }}}
}
\zs
% \medskip
% bug medskip is necessary when \flushright is used

\noindent
Bei der Kombination des Verbs in Erststellung mit der Projektion der Verbspur wird der \ltopw und
der Index von der Verbspurprojektion übernommen und wegen des Semantikprinzips dann weiter nach oben
gereicht und so zum Beitrag der gesamten Konstruktion. Abbildung~\vref{verb-movement-sem} zeigt die
semantischen Aspekte der Verbbewegungsanalyse mit der Spur in (\ref{le-verbspur-kennt}) und der
Verberstregel in (\ref{lr-verb-movement2}).
\begin{figure}
\oneline{%
\begin{tikzpicture}[remember picture]
\begin{forestintikzpicture}
sm edges, for tree={l sep+=\baselineskip}
[V\ms{cont \ms{ltop & \subnode[inner sep=0pt]{ltop5}{\ibox{1}} h1\\
                   rels & \nliste{ \ldots, h1:k(j, r) }}}, baseline
            [V\ms{cont  & \ms{ltop & \subnode[inner sep=0pt]{ltop4}{\ibox{1}} h1\\
                              rels & \eliste }\\
                  comps & \sliste{ \ibox{2} dsl|cont \subnode[inner sep=0pt]{dsl0}{\ibox{3}} }}
                    [V\ms{cont \subnode[inner sep=0pt]{c1}{\ibox{3}} \ms{ltop & h1\\
                                                          rels & \nliste{ h1:k(j, r) } }}, tier=np,edge label={node[midway,right]{V1-R}}
                            [kennt$_j$]]]
            [\ibox{2} V\ms{dsl|cont \subnode[inner sep=0pt]{dsl1}{\ibox{3}},\\
                  cont \ms{ltop & \subnode[inner sep=0pt]{ltop3}{\ibox{1}} h1,\\
                           rels & \nliste{ \ldots, h1:k(j, r) }}}
                    [NP{[\textit{nom}]}, tier=np
                            [jemand]]
                    [V\ms{dsl|cont \subnode[inner sep=0pt]{dsl2}{\ibox{3}},\\
                          cont \ms{ltop & \subnode[inner sep=0pt]{ltop2}{\ibox{1}} h1\\
                                   rels & \nliste{ \ldots, h1:k(j, r) }}}
                            [NP{[\textit{acc}]}
                                    [den Roman, roof]]
                            [V\ms{dsl|cont \subnode[inner sep=0pt]{dsl3}{\ibox{3}},\\
                                  cont \subnode[inner sep=0pt]{cont-trace}{\ibox{3}}\ms{ltop & \subnode[inner sep=0pt]{ltop1}{\ibox{1}} h1\\
                                                                                        rels & \nliste{ h1:k(j, r) } }}
                                    [\trace$_j$]]]]]
%
% use mark=* after smooth to see the dots that are used, mark=none disables the drawing
%
% The following does not draw the first and the last mark
% \draw[blue] plot [smooth,mark=*,mark indices={2,3}] coordinates {(ltop4.80) (0,-1.4) (2.4,-1.6) (ltop3.north)};
%
% alternative:
%\filldraw[red] (0,-1.4) circle [radius=2pt] (2.4,-1.6) circle [radius=2pt];
%
\begin{scope}[overlay,<->,shorten >=2pt, shorten <=2pt]
%\draw[gray] (-6,-10) to[grid with coordinates] (9,1);
% Lexical rule
\draw[brown] plot [smooth,mark=none,mark indices={2,3,4}] coordinates {(c1.north) (-3.3,-4) (-2.8,-3.8) (-1.6, -3.8) (dsl0.south west)};
% [3] links zu [3] rechts
\draw[blue]  plot [smooth] coordinates {(dsl0.north) (-0.8,-2) (1,-1.9) (dsl1.north west)};
% DSL percolates down from clause to trace.
\draw[red]   plot [smooth] coordinates {(dsl1.east) (5.2,-1.0) (6.4,-2.6) (dsl2.east)};
\draw[red]   plot [smooth] coordinates {(dsl2.east) (7.3,-3) (8,-4.5) (dsl3.east)};
% LTOP is projected upwards from trace to complete verbal projection
\draw[green] plot [smooth] coordinates {(ltop2.45) (7,-3.5) (7.5,-5) (ltop1.45)};
\draw[green] plot [smooth] coordinates {(ltop3.60) (5,-1.5) (6,-3) (ltop2.55)};
% LTOP von V1 übernommen aus COMPS
% [1] links zu [1] rechts 
\draw[blue] plot [smooth,mark=none,mark indices={2,3}] coordinates {(ltop4.80) (0,-1.2) (2.4,-1.4) (ltop3.north)};
% LTOP is projected from V1 head to complete V1 clause
\draw[green] plot [smooth,mark=none,mark indices={2}] coordinates {(ltop4.135) (-2.6,.2) (ltop5.135)};
\end{scope}
\end{forestintikzpicture}
\end{tikzpicture}
}
\caption{\label{verb-movement-sem}Veranschaulichung der Analyse von \emph{Kennt jemand den Roman?}:
  Braun = Verberstregel, Blau = Selektion, Grün = Semantikprinzip, Rot = Kopfmerkmalsprinzip}
\end{figure}
Rein technisch müsste das \textit{h1} des Eingabeverbs für die V1"=Regel in Abbildung~\ref{verb-movement-sem} auch \iboxt{1} sein, denn
der \contw des Eingabeverbs ist ja identisch mit dem der Verbspur. Ich habe das aus
Darstellungsgründen jedoch weggelassen, damit man besser sehen kann, welchen Weg die
\ltop"=Information nimmt. Durch die Verberstregel wird aus dem Verb, wie es in Letztstellung
vorkommen würde, ein V1"=Verb. Dieses selegiert eine Verbspur und in \dsl ist der \contw des
ursprünglichen Verbs \iboxb{3} enthalten. Das Komplement des Verbs in Initialstellung \iboxb{2} wird
mit der Verbalprojektion identifiziert, wodurch der \dslw der Verbalprojektion dann den \contw des
Verbletztverbs \iboxb{3} enthält. Da \dsl ein Kopfmerkmal ist, wird der \dslw mit anderen
Kopfmerkmalen entlang des Kopfpfades weitergegeben, so dass er auch an der Verbspur vorhanden
ist. Da Valenz und semantische Information im \dslw mit den entsprechenden Werten der Spur (\ref{le-verbspur-kennt}) identifiziert wird, ist \ibox{3} auch identisch mit dem
\contw der Verbspur. Das heißt, dass die Relation kennen(jemand, roman) auch Element der
\relsl der Verbspur ist. Das Handel dieser Relation ist -- wie beim ursprünglichen Verb auch --
h1. Nach Semantikprinzip wird der \ltopw den Kopfpfad entlang nach oben gegeben und das V1"=Verb
übernimmt den \ltopw von seinem Komplement. Da das V1"=Verb der Kopf des gesamten Satze ist, wird
der \ltopw des V1"=Verbs dann zum \ltopw des gesamten Satzes. Man beachte, dass das Verb in
Erststellung keine Relationen über \rels beisteuert. Das ist wichtig, denn die Relationen kommen vom
Satzende. Würde auch das Verb in Erststellung Relationen beisteuern, wären diese doppelt in der
\relsl: einmal vom Verb in Erststellung und einmal von der Verbspur in Letztstellung.\footnote{
  Das wird im Kapitel~\ref{Abschnitt MRS und Fernabhängigkeiten} revidiert. Für die Analyse von
  Fernabhängigkeiten ist es nicht sinnvoll, das Merkmal \rels innerhalb von \cont zu haben, wenn man
  Spuren verwendet, weil die Relationen dann sowohl im Vorfeld als auch an der Stelle der Spur in
  die \relsl aufgenommen werden. In der revidierten Version der Verbbewegungsregel kommen die
  Relationen vom Verb in Erststellung und die Verbspur und die Extraktionsspur führen selbst keine
  Relationen ein.
}

Die Analyse in Abbildung~\ref{verb-movement-sem} mag etwas kompliziert erscheinen, da die semantische
Information einerseits über \dsl vom Verb in Verberststellung zur Verbspur und andererseits wieder von
der Verbspur zum Verb in Erststellung weitergereicht wird. Anhand eines Beispiels mit einem Adjunkt
wird aber klar, dass diese kompliziert erscheinende Behandlung gerechtfertigt ist. Die Analyse des
Satzes (\mex{1}) zeigt Abbildung~\vref{verb-movement-adjunkt-sem}.
\ea
Kennt jemand den Roman nicht?
\z
\begin{figure}
\oneline{%
\begin{forest}
sm edges, for tree={l sep+=2\baselineskip}
[V\ms{cont \ms{ltop  & \subnode{ltop5}{\ibox{1}} h2\\
               rels  & \nliste{ \ldots, h2:n(h3), h1:k(j, r) }\\
               hcons & \nliste{ \ldots, h3 \qeq h1 }}}, baseline
	[V\ms{cont  & \ms{ltop & \subnode{ltop4}{\ibox{1}} h2\\
                          rels & \eliste\\
                          hcons & \eliste }\\
              comps & \sliste{ \ibox{2} dsl|cont \subnode{dsl0}{\ibox{3}} }}
		[V\ms{cont \subnode{c1}{\ibox{3}} \onems{ltop \type{h1}\\
                                                         rels  \nliste{ h1:k(j, r) }\\[1mm]
                                                         hcons  \eliste}}, tier=np,edge label={node[midway,right]{V1-R}}
			[kennt$_j$]]]
	[\ibox{2} V\ms{dsl|cont \subnode{dsl1}{\ibox{3}}\\
              cont \ms{ltop & \subnode{ltop3}{\ibox{1}} h2\\
                       rels & \nliste{ \ldots, h2:n(h3), h1:k(j, r) }\\[1mm]
                       hcons & \nliste{ \ldots, h3 \qeq h1 }}}
		[NP{[\textit{nom}]}, tier=np
			[jemand]]
		[V\ms{dsl|cont \subnode{dsl2}{\ibox{3}},\\
                      cont \ms{ltop & \subnode{ltop2}{\ibox{1}} h2\\
                               rels & \nliste{ \ldots, h2:n(h3), h1:k(j, r) }\\[1mm]
                               hcons & \nliste{ \ldots, h3 \qeq h1 }}}, s sep+=2em
			[NP{[\textit{acc}]}%,fit=band
				[den Roman, roof]]
                        [V\ms{dsl|cont \subnode{dsl3}{\ibox{3}},\\
                              cont \subnode{cont-trace}{\ibox{3}}\ms{ltop & \subnode{ltop1}{\ibox{1}} h2\\
                                                                     rels & \nliste{ h2:n(h3), h1:k(j, r) }\\[1mm]
                                                                     hcons & \nliste{ h3 \qeq h1 }
                                                                   }}
                          [Adv\ms{cont \ms{ltop & \subnode{ltop1}{\ibox{1}} h2\\
                                           rels & \nliste{ h2:n(h3) }\\[1mm]
                                           hcons & \nliste{ h3 \qeq h1 } } }
                             [nicht]]
			  [V\ms{dsl|cont \subnode{dsl3}{\ibox{3}},\\
                                cont \subnode{cont-trace}{\ibox{3}}\onems{ltop \type{h1}\\
                                                                          rels \nliste{ h1:k(j, r) }\\[1mm]
                                                                          hcons \eliste }}
				[\trace$_j$]]]]]]
\end{forest}}
\caption{\label{verb-movement-adjunkt-sem}Veranschaulichung semantischer Aspekte der Analyse von \emph{Kennt jemand den Roman nicht?}}
\end{figure}
Anders als bei der Analyse des unnegierten Satzes ist nicht h1 das \ltop, sondern das Handle
der Negation h2. Dieses Handle wird nach oben gegeben. Der \contw des Verbs, wie es in Endstellung
vorkommen würde, wird via \dsl mit dem \contw der Spur geteilt. Der \hconsw für die
Skopusbeschränkungen von \emph{kennt} und somit auch der Verbspur ist die leere Liste. Die Negation führt die Relation \relation{nicht} mit dem
Handle h2 ein, \relation{nicht} hat das Argument h3 und h3 ist \qeq mit h1, dem Handle der Relation
\relation{kennen}. Die \relation{nicht}"=Relation und die \relation{kennen} werden in \rels nach
oben gegeben, die Skopusbeschränkung von \emph{nicht} in \hcons. Das Verb in Initialstellung
übernimmt von seinem Komplement wieder \ltop und \textsc{index} und von dort wird h2 zum \ltop des
gesamten Satzes. Man sieht also, dass man nicht einfach die Semantik vom Verb in Erststellung nach
oben geben darf, sondern den Umweg über die Spur nehmen muss, denn durch Adjunkte im Mittelfeld,
kann der Beitrag des Verbs in Initialstellung noch in sein Gegenteil verkehrt werden.
\is{Lexikonregel!Verberststellung|)}


Es wäre unbefriedigend, wenn man für jedes Verb eine spezielle Spur haben müsste. Das ist aber 
zum Glück nicht nötig, da eine allgemeine Spur der Form in (\mex{1}) für die Analyse
von Sätzen mit Verbbewegung ausreicht.\footnote{%
  In früheren Arbeiten \parencites{Mueller2005c}[408]{MuellerOrder2024} habe ich eine zyklische Spur
  verwendet, in der der \localw mit dem \dslw
\itdopt{Seite für lB}
  identifiziert wurde. Das bedeutet so viel wie: Ich fehle mir selbst. Diese Analyse wurde von
  \citet[207]{Meurers2000b} vorgeschlagen. Sie ist sehr elegant, aber leider nicht mit der Analyse
  der Koordination verträglich, die ich hier verwende. Sie Abschnitt~\ref{sec-Verbbewegung mit zyklischen Strukturen} für eine Diskussion.
}
\eas
Verbspur:\\
\ms{
phon & \eliste\\
loc & \ms{ cat & \ms{ head & \onems[verb]{
                             vform \type{fin}\\
                             ini   $-$\\
                             dsl   \ms{ cat & \ms{ spr   & \ibox{1}\\ 
                                                     comps & \ibox{2}\\
                                                   }\\
                                          cont & \ibox{3}\\
                                        }}\\
                      spr & \ibox{1}\\
                      comps & \ibox{2}\\
                    }\\
          cont & \ibox{3}\\
        }\\
}
\zs
(\mex{1}) ist ein Lexikoneintrag für ein Element, das nicht ausgesprochen wird. Es hat die Wortart
Verb, ist finit und muss in Letztstellung stehen (\textsc{ini}$-$). \sprw, \compsw und \contw wird mit dem
Wert in \dsl geteilt.

Diese Analyse der Verbstellung ist die komplexeste Analyse im vorliegenden Buch. Wenn man sie verstanden
hat, braucht man nichts mehr zu fürchten. Die wichtigsten Punkte seien hier noch einmal zusammengefasst:
\begin{itemize}
\item Eine Verberstregel lizenziert für jedes finite Verb eine besondere Form des Verbs.
\item Dieser Lexikoneintrag steht in Initialstellung und verlangt als Argument eine vollständige Projektion
      einer Verbspur.
\item Die Projektion der Verbspur muss einen \dslw haben, der dem \localw der Tochter in der Verberstregel
      entspricht.
\item Da \dsl ein Kopfmerkmal ist, ist der selegierte \dslw auch an der Spur präsent.
\item Da der \dslw der Spur mit deren \localw identisch ist, ist der \localw der Spur also
      auch mit dem \localw der Tochter in der Verberstregel identisch.
\end{itemize}

\noindent
Nach der Diskussion der Analyse von Verberstsätzen in diesem Kapitel wird im folgenden Kapitel
die Analyse von Verbzweitsätzen erklärt. Vorher sollen aber noch alternative HPSG"=Ansätze zur Konstituentenstellung
und die Behandlung der Konstituentenstellung in anderen Theorien diskutiert werden.

\section{Alternativen}

Abschnitt~\ref{sec-konstituentenreihenfolge-alternativen} diskutiert einige Aspekte bisheriger
HPSG"=Ansätze. Abschnitt~\ref{sec-alternativen-konstituentenreihenfolge-gb} setzt sich mit der
Behandlung der Konstituentenstellung innerhalb der GB"=Theorie bzw.\ innerhalb des Minimalistischen
Programms auseinander.

\subsection{Alternative HPSG-Ansätze}
\label{sec-konstituentenreihenfolge-alternativen}

Alternative HPSG"=Ansätze werden ausführlich in \citew{Mueller2004b} und in
\citew{Mueller2005c,Mueller2005d,MuellerGS} diskutiert. Hier seien nur einige Punkte kurz erwähnt.

\subsubsection{Verbbewegung mit zyklischen Strukturen}
\label{sec-Verbbewegung mit zyklischen Strukturen}

\citet[207]{Meurers2000b} hat eine zyklische Verbspur bzw.\ sogar noch viel allgemeiner eine Spur
für Kopfbewegung vorgeschlagen:
\eas
\label{le-verbspur}%
Allgemeine Kopfspur:\is{Spur!Verb-}\\
\ms{
phon & \phonliste{}\\
loc  & \ibox{1} \ms{ cat$|$head$|$dsl   & \ibox{1}\\
                   }\\
}
\zs
Mit dieser zyklischen Struktur muss man nicht einmal die Wortart und sonstigen Eigenschaften der
Verbspur wie \initialw und Finitheit angeben. Dies mag auf den ersten Blick verwundern, doch wenn
man sich das Zusammenspiel der Verberstregel (\ref{lr-verb-movement2}) und die Perkolation des
\textsc{dsl}"=Merkmals im Baum genau ansieht, wird man erkennen, dass der \dslw der Verbprojektion
und somit der \localw der Verbspur durch den \localw des Eingabeverbs bestimmt ist. In
Abbildung~\ref{verb-movement-syn} und~\ref{verb-movement-sem} ist \emph{kennt} die Eingabe zur
Verbbewegungsregel. Der \localw der Verbspur entspricht also dem \localw von \emph{kennt}. Durch die
entsprechenden Strukturteilungen wird sichergestellt, dass bei einer Analyse des Satzes
(\ref{bsp-kennt-jemand-den-Roman}) der \localw der Verbspur genau dem entspricht, was in
(\ref{le-verbspur-kennt}) angegeben ist. Wie oben schon erwähnt, enthält die geteilte Information
auch die Wortart der Spur, so dass die Wortart nicht in der Beschreibung der vom Verb in
Initialstellung selegierten Projektion angegeben werden muss.

Diese Verbspur ist wunderschön und viel besser als die, die ich im vorigen Abschnitt vorgestellt
habe, es gibt nur ein Problem, auf das auch \citet[207]{Meurers2000b} bereits hingewiesen hat. Mit
der zyklischen Verbspur kann man nicht in Lexikoneinträgen von Verbletztverben festlegen, dass deren
\dslw \type{none} ist, denn die Verbletztverben sind Töchter der Verberstregel und ihr \localw ist
identisch mit dem \dslw der selegierten Verbprojektion. Wäre der \dslw eines Verbs in Letztstellung
\type{none}, dann würden Verben in Erststellung in ihrer \compsl ein Element mit dem Pfad
\textsc{loc|cat|head|dsl|cat|head|dsl} und dem Wert \type{none} haben. Der \headw der selegierten
Projektion ist mit dem \headw der Verbspur identisch. Das heißt, der \headw der Spur enthält
\textsc{dsl|cat|head|dsl} \type{none}. Da es aber in der Verbspur einen Zyklus gibt,
der \dsl mit \local gleichsetzt, muss es nach \dsl immer weiter gehen. \type{none} kommt als Wert
nicht in Frage. Das bedeutet, dass der \dslw von Verben in Letztstellung nicht \type{none} sein darf,
wenn man so eine schöne Spur wie in (\ref{le-verbspur}) verwenden will. Dann ergibt sich aber das
Problem, dass außer der erwünschten Verbspur auch Verbletztverben in Sätzen mit Verbinitialstellung
auftreten können:
\ea[*]{
Kennt jemand den Roman kennt.
}
\z
(\mex{0}) könnte analysiert werden, wenn das erste Vorkommen von \emph{kennt} als Ausgabe der Verbbewegungsregel
analysiert wird und wenn der \textsc{dsl}"=Wert des zweiten \emph{kennt} nicht restringiert ist, so dass das
zweite \emph{kennt} dieselbe Rolle wie die Verbspur in der Analyse übernehmen könnte. Ich habe
deshalb lange Zeit in Anlehnung an einen Vorschlag von \citet[207]{Meurers2000b} die folgende
Beschränkung vorgeschlagen. Sie besagt, dass ein Verb, wenn es overt realisiert
wird und in syntaktische Strukturen eintritt, den \textsc{dsl}"=Wert \emph{none} haben muss. 
\ea
\label{constraint-DSL}
\ms{ head-dtr & \ms[word]{ phon & non-empty-list \\
                         }
} \impl
\onems{ loc$|$cat$|$head$|$dsl \type{none}\\
            }
\z
Abbildung~\ref{abb-jemand-den-Roman-kennt-dsl} verdeutlicht die Konfiguration:
\begin{figure}
\begin{forest}
sm edges
[{V[\head \ibox{1} ]}
  [{NP[\textit{nom}]}
    [niemand]]
  [{V[\head \ibox{1} [\dsl \type{none}]]}
    [{NP[\type{acc}]}
       [den Roman,roof]]
    [{V[\head \ibox{1}]}
      [kennt]]]]
\end{forest}
\caption{\label{abb-jemand-den-Roman-kennt-dsl}\dslwe für \emph{jemand den Roman kennt}}
\end{figure}
Die Beschränkung wird angewendet, wenn \emph{kennt} als Kopftochter mit dem Akkusativobjekt
kombiniert wird. Da die Kopfmerkmale entlang des Kopfpfades geteilt werden, hat auch \emph{kennt}
den \dslw \type{none}. Im Lexikon kann es aber einen unspezifizierten \dslw haben.

Die Beschränkung in (\ref{constraint-DSL}) unterscheidet sich von der von
\citet[\page207]{Meurers2000b} angegebenen \ua dadurch, dass die Kopftochter im Antezedens vom Typ
\emph{word} sein muss. Ohne diese Einschränkung auf \type{word} würde das Constraint auch auf
Projektionen der Verbspur anwendbar sein und somit wohlgeformte Sätze ausschließen.

Diese Analyse funktioniert und ist Bestandteil der Grammatiken im CoreGram-Projekt
\citep{MuellerCoreGram}, es gibt aber ein Problem, das ich übersehen habe: Die Beschränkung in
(\ref{constraint-DSL}) erfasst alle Vorkommen von Verben in Kopfpositionen, in
Koordinationsstrukturen sind die Verben bzw.\ ihre Projektionen aber nicht in Kopfposition. Das
bedeutet, dass trotz der Implikation in (\ref{constraint-DSL}) ungrammatische Sätze zugelassen
werden. (\mex{1}) zeigt einige Beispiele:
\eal
\label{ex-Probleme bei zyklischen Verbspuren mit Koordination}
\ex[*]{
Der Affe kennt$_j$ [[lacht] und [den Stock \trace$_j$]].
}
\ex[*]{
Der Affe gibt$_j$ [[lacht] und [dem Mann den Stock \trace$_j$]].
}
\zl
Die Struktur in (\mex{0}b) entsteht dadurch, dass eine Verbspur für \emph{gibt} mit zwei Argumenten kombiniert
wird, so dass nur noch ein Subjekt gebraucht wird. Dann wird die Projektion \emph{dem Mann den Stock
  \trace$_j$} mit dem intransitiven Verb \emph{lacht} koordiniert. Dabei wird der \dslw des
Verbletztverbs mit dem der Verbspur identifiziert. Es entsteht also ein \emph{lacht} mit einem
\dslw, der später mit dem \localw von \emph{gibt} identifiziert wird. Unsinnige und unerwünschte
Strukturen. Man kann nun alles Mögliche versuchen, um die Strukturen mit weiteren Implikationen
auszuschließen, aber das ist nicht so einfach, denn man möchte Verben mit gefüllten \dslwen
koordinieren können, denn das war ja der Ausgangspunkt: Sätze wie (\mex{1}) sollten analysiert werden.
\ea
Schläft und schnarcht Aicke?
\z
Man könnte die Koordination von Konjunkten mit gefülltem \dslw generell ausschließen. Auf den ersten
Blick scheint das eine schlechte Idee zu sein, weil es so scheint, als könnten koordinierte
Verbalprojektionen für die Analyse von Sätzen wie (\mex{1}) hilfreich sein:
\ea
Kennt$_j$ [Aicke den Roman \trace$_j$] und [Conny den Film \trace$_j$]?
\z
Das ist aber eine Täuschung. Da die Verbspur in beiden Konjunkten dieselbe ist (Sie ist Teil von
\head und muss deshalb in beiden Konjunkten gleich sein, da die \catwe identifiziert werden.) Ist
auch die Valenzliste des Verbs in beiden Konjunkten identisch. Das bedeutet, dass (\mex{0})
überhaupt nicht analysiert werden kann, denn \emph{Aicke} und \emph{Conny} und \emph{den Roman} und
\emph{den Film} sind nicht identifizierbar. Selbst wenn die Konjunkte aus phonologisch und
semantisch kompatiblem Material bestünden, wäre die Analyse letztendlich falsch, da die Variablen
der NPen identifiziert würden und somit die sich ergebende MRS nicht wohlgeformt wäre. Sätze wie
(\mex{0}) muss man also anders analysieren. Siehe \citews{Crysmann2003c}{BS2004a} zu Vorschlägen für
ähnliche Strukturen im Englischen und \citew{AC2024a} zu einem allgemeinen Überblick über
Koordination in HPSG.

Wenn man also Koordinationen mit Projektionen von Verbspuren grundsätzlich ausschließt, werden die
problematischen Abfolgen in (\ref{ex-Probleme bei zyklischen Verbspuren mit Koordination}) nicht
mehr lizenziert. Allerdings verbleiben noch Folgen wie (\mex{1}):
\ea[*]{
\label{ex-Schläft und schnarcht Aicke schläft und schnarcht}
Schläft und schnarcht Aicke schläft und schnarcht?
}
\z
Diese kann man nicht ausschließen, denn genau diese Koordination in der rechten Satzklammer braucht
man ja als Eingabe der Verberstregel.

Ein weiteres Problem ergibt sich daraus, dass bei Koordinationen die \catwe der Konjunkte
identifiziert werden \citep[\page 202]{ps2}. Abbildung~\ref{Abbildung Kennt und
  liebt Aicke diese Band} zeigt, gelangt auf diese Weise der \localw von \emph{kennt und liebt}
\iboxb{2} innerhalb der \catwe der beiden Konjunkte zu den Verben \emph{kennt} und \emph{liebt}. 
%\mmznext{recompile} % for one compilation
\begin{figure}
\begin{forest}
sm edges
[S
  [{V[\comps \sliste{ \ibox{1} }]} 
    [{V[\textsc{loc} \ibox{2} [\textsc{cat} \ibox{3}]]$_j$},tier=np,edge label={node[midway,right]{V1-R}} 
       [{V[\textsc{cat} \ibox{3} [\textsc{h|dsl} \ibox{2}]]} [kennt] ]
        [Coord
          [Coord [und]]
          [{V[\textsc{cat} \ibox{3} [\textsc{h|dsl} \ibox{2}]]} [liebt]]]]]
    [{\ibox{1} V\feattab{\textsc{dsl} \ibox{2},\\
                 \comps \eliste}}
         [{NP[\type{nom}]},tier=np [Aicke] ]
         [V\feattab{\textsc{dsl} \ibox{2}}
           [{NP[\type{acc}]} [diese Band, roof] ]
           [{V[\textsc{l} \ibox{2} [\textsc{cat|h|dsl} \ibox{2}]]} [\_$_j$] ] ] ] ]
\end{forest}
\caption{Verteilung der \catwe und somit auch \dslwe in Koordinationen bei Erststellung}\label{Abbildung Kennt und
  liebt Aicke diese Band}
\end{figure}
Damit landet auch der \contw der Koordination der beiden Verben im \dslw der einzelnen Verben.
Rein technisch, von der Menge der lizenzierten Strukturen, ist das kein Problem, aber es ist dennoch
unerwartet und vor allem sorgt es dafür, dass man nicht verlangen kann, dass der \dslw eines Verbs
immer identisch mit dem \localw des Verbs ist, wenn das Verb einen von \type{none} verschiedenen
\dslw hat. Mit solch einer Beschränkung hätte man die falsche Koordination der in der rechten
Satzklammer in (\ref{ex-Schläft und schnarcht Aicke schläft und schnarcht}) ausschließen können,
aber man braucht genau das in der linken Satzklammer.

Die Schlussfolgerung ist, dass man, wenn man die Analyse der koordinierten Verben in Initialstellung
beibehalten will, auf die zyklische Verbspur verzichten muss. Da man ohne zyklische Strukturen auch
nicht die unplausiblen Verteilungen von \dslwen aus Abbildung~\ref{Abbildung Kennt und liebt Aicke
  diese Band} bekommt, bevorzuge ich eine nicht-zyklische Analyse.


\subsubsection{Flache Strukturen mit freier Linearisierung des Verbs}
\label{sec-flache-strukturen}

\citet{Uszkoreit87a}\is{Verzweigung!flache|(} hat eine GPSG"=Grammatik für das Deutsche entworfen, die davon ausgeht,
dass das Verb mit seinen Argumenten in einem lokalen Baum realisiert wird. Da das Verb
und dessen Argumente vom selben Knoten dominiert werden, können sie nach GPSG"=Annahmen beliebig
angeordnet werden, solange bestimmte innerhalb der Theorie spezifizierte Linearisierungsbeschränkungen\is{Linearisierung!-sregel}
nicht verletzt werden. \citet{Pollard90a} hat Uszkoreits Ansatz für eine Beschreibung der Satzstruktur
des Deutschen in seine HPSG"=Analyse übernommen. Die Analyse des Satzes (\ref{bsp-perm-mf}) zeigt
Abbildung~\ref{abb-uszkoreit-pollard}.
\begin{figure}
\begin{forest}
sm edges, for tree={l sep=+2\baselineskip}
[{V[\type{fin}, \comps \eliste{}]}
  [{\ibox{1} NP[\type{nom}]} 
    [Aicke,roof]]
  [{\ibox{3} NP[\type{acc}]} 
    [den Stock, roof]]
  [{\ibox{2} NP[\type{dat}]}
    [dem Affen, roof]]
  [{V[\type{fin}, \comps \nliste{ \ibox{1}, \ibox{2}, \ibox{3} }] }
    [gibt]]]
\end{forest}
\caption{\label{abb-uszkoreit-pollard}Analyse von \emph{Aicke den Stock dem Affen gibt} mit flachen Strukturen}
\end{figure}

%% Mitunter wird mit Lernbarkeitsargumenten\is{Lernbarkeit} gegen flache Strukturen argumentiert \citep[Kapitel~2.5]{Haegeman94a-u}.
%% Die Behauptung ist, dass man bei flach verzweigenden Strukturen mehr Regeln benötigt und dass demzufolge
%% Grammatiken mit flach verzweigenden Strukturen schwerer zu erlernen wären. Dieses Argument kann jedoch
%% nur verwendet werden, wenn die flach verzweigenden Regeln wirklich einzeln als verschiedene Phrasenstrukturregeln
%% formuliert werden (wie \zb in der GPSG von Uszkoreit). Verwendet man abstrakte Schemata wie die
%% im Kapitel~\ref{sec-xbar} angegebenen \xbar-Regeln bzw.\ die HPSG"=Schemata ist die Anzahl der benötigten
%% Regeln gleich, denn die Regeln enthalten keine Information darüber, wieviele Argumente ein bestimmter
%% Kopf hat, diese Information ist im Kopf selbst enthalten. Wenn ein Kopf zwei Argumente verlangt (\emph{erwarten})
%% dann gibt es in einer flach verzweigenden Struktur, die diesen Kopf als Kopf"=Tochter hat, drei Töchter:
%% die Kopf"=Tochter und zwei Nicht"=Kopf"=Töchter. Bei einem dreistelligen Verb wie \emph{geben} gäbe
%% es entsprechend vier Töchter. Das Schema, das diese Strukturen lizenziert, ist jeweils dasselbe. Zu erlernen
%% wäre lediglich, dass es in einer Sprache bzw.\ in Sprachen die Möglichkeit gibt, einen Kopf
%% mit seinen Argumenten zu kombinieren.
%%
%% Lernbarkeit ist also keine Eigenschaft, die helfen würde, zwischen den beiden Alternativen zu unterscheiden.
%% Es gibt jedoch einige relevante Unterschiede, die im folgenden diskutiert werden sollen:
Wie \citet[\page220]{Netter92} anmerkt sind Adjunkte in solch flache Strukturen nicht ohne weiteres zu
integrieren, da die Bestimmung des semantischen Beitrags der Gesamtphrase sich auf den Beitrag
der einzelnen Adjunkte beziehen muss. \citet{Kasper94a} hat gezeigt, wie Adjunkte in so eine flache
Analyse integriert werden können. Er verwendet komplexe relationale Beschränkungen, die alle
Adjunkttöchter nacheinander in die Berechnung der Gesamtbedeutung einbeziehen. Da relationale
Beschränkungen ein sehr mächtiges Beschreibungsmittel sind, sind Ansätze, die sie vermeiden bzw.\
sich auf einfache Beschränkungen wie \emph{append} beschränken, vorzuziehen.

Ansätze mit flachen Strukturen haben den Vorteil, dass sie ohne leere Köpfe\is{leere Kategorie} für die Beschreibung der
Verbstellung auskommen. Bei solchen Analysen scheint es jedoch nicht möglich zu sein, die scheinbar
mehrfache Vorfeldbesetzung\is{Vorfeldbesetzung} \citep{Mueller2005d,MuellerGS} auf adäquate Weise zu analysieren. Beispiele für
solche problematischen Voranstellungen sind in (\mex{1}) aufgeführt (siehe auch Seite~\pageref{bsp-mehr-vf}):
\eal
\label{bsp-smvfb}
%% \ex Fachleute erkennen Salzschäden an Bäumen relativ gut. [\ldots] Es kann also einige
%%     Winterperioden dauern, bis man den Effekt beobachten kann. [\ldots] [Exakt] [auf das Salz]
%%     kann man es tatsächlich erst zurückführen, wenn es für den Baum zu spät ist.\footnote{
%%         taz berlin, 22.12.2003, S.\,22
%%     }
\ex {}[Trocken] [durch die Stadt] kommt man am Wochenende auch mit der BVG.\footnote{
        taz berlin, 10.07.1998, S.\,22.
      }
% [unverholen] oder [unverholen verärgert]
\ex {}Unverhohlen verärgert auf Kronewetters Vorwurf reagierte Silke Fischer.\footnote{
    taz berlin, 23.04.2004, S.\,21.
}
\ex {}[Hart] [ins Gericht] ging Klug mit dem Studienkontenmodell der Landesregierung.\footnote{
  taz nord, 19.02.2004, S.\,24.
  }
\zl
In diesen Beispielen befinden sich zwei Konstituenten vor dem finiten Verb, obwohl das Deutsche
zu den Verbzweitsprachen gerechnet wird, weil in Aussagesätzen
normalerweise eben nur genau eine Konstituente vor dem Finitum steht (zu den Details siehe Kapitel~\ref{Kapitel-nla}).
Weitere Beispiele und eine ausführliche Diskussion findet man in \citew{Mueller2003b} und \citew{Bildhauer2011a}
bzw.\ in der \href{http://hpsg.hu-berlin.de/~stefan/Pub/mehr-vf-ds.html}{im WWW zugänglichen Datensammlung}\footnote{
  \url{http://hpsg.hu-berlin.de/~stefan/Pub/mehr-vf-ds.html}. \mytoday%
}.

Bei einem Ansatz, der Verbspuren annimmt, kann man davon ausgehen, dass sich in (\mex{0}) eine Projektion
einer solchen Spur im Vorfeld befindet, \dash, dass das gesamte Material vor dem Finitum eine komplexe Konstituente
bildet. Bei Linearisierungsansätzen gibt es dagegen einfach keine Möglichkeit, 
die Konstituenten im Vorfeld zu einer Konstituente zusammenzufassen. Man
könnte natürlich -- wie in \citew{Mueller2002f,Mueller2002c,Mueller2005d} vorgeschlagen 
-- einen leeren Kopf im Vorfeld annehmen, nur wäre dieser dann ein spezielles leeres Element,
was nirgends sonst in der Grammatik gebraucht würde
und nur zur Erfassung der scheinbar mehrfachen Vorfeldbesetzung stipuliert würde.

Alternativ könnte man Spezialregeln formulieren, die das Material
im Vorfeld zu einer Konstituente kombinieren \citep[Kapitel~7.6]{Welke2019a-u}. Dabei stellt sich natürlich die Frage nach
der syntaktischen Kategorie dieser Konstituente. Nähme man an, dass es sich um eine verbale
Projektion handelt, müsste man kopflose Spezialregeln annehmen, da die Konstituenten vor dem finiten
Verb in (\mex{0}a,c) keine Verben sind und demzufolge nicht der Kopf einer verbalen Projektion sein
können.
\is{Verzweigung!flache|)}

\subsubsection{Binär verzweigende Strukturen und Linearisierungsdomänen}
\label{sec-Linearisierung}
\label{sec-Reape-Linearisierung}
\is{Linearisierung!-sdomäne|(}

\citet{Reape90a,Reape92a,Reape94a}\is{Verzweigung!binäre|(} hat mit seinen Arbeiten 
den Weg für Linearisierungsgrammatiken\is{Linearisierung!-sgrammatik} geebnet,
die binär verzweigende Strukturen, aber flache Linearisierungsdomänen annehmen. Ein Verb
befindet sich dann mit seinen Argumenten in derselben Linearisierungsdomäne und kann,
obwohl es nicht zum selben lokalen Baum gehört, den Linearisierungsbeschränkungen entsprechend
angeordnet werden. Solche Modelle wurden für das Deutsche von \citet{Kathol95a,Kathol2000a} und 
auch von mir vertreten \citep{Mueller95c,Mueller99a,Mueller2002b}. Wie ich jedoch gleich zeigen
werde, gibt es Daten, die sich nicht gut erklären lassen. 

Linearisierungsgrammatiken unterscheiden sich von anderen Grammatikmodellen dadurch, dass
sie diskontinuierliche Konstituenten\is{Konstituente!diskontinuierliche} zulassen, \dash, es können auch Konstituenten miteinander
kombiniert werden, die nicht nebeneinander stehen. Im folgenden soll kurz die formale Umsetzung
einer Linearisierungsgrammatik vorgestellt werden: Reape führt ein listenwertiges Merkmal
\dom für die Linearisierungsdomäne ein. Alle eingekreisten Elemente in Abbildung~\vref{abb-lin-domains}
werden in diese Liste eingesetzt.
\begin{figure}
\begin{forest}
sm edges
[{V[\comps \sliste{} ]}
  [{\ibox{1} NP[\type{nom}]},ellipse,draw
    [Aicke]]
  [{V[\comps \sliste{ \ibox{1} }]}
    [{\ibox{2} NP[\type{dat}]},ellipse,draw 
      [dem Affen, roof]]
    [{V[\comps \sliste{ \ibox{1}, \ibox{2} }]}
      [{\ibox{3} NP[\type{acc}]},ellipse,draw
        [den Stock,roof]]
      [{V[\comps \sliste{ \ibox{1}, \ibox{2}, \ibox{3} }]},ellipse,draw
        [gibt]]]]]
\end{forest}
\caption{\label{abb-lin-domains}Eingekreiste Knoten werden in eine Linearisierungsdomäne eingesetzt.}
\end{figure}


In \citew[\page162]{Mueller99a} habe ich vorgeschlagen,
Köpfe so zu repräsentieren, dass jeder Kopf in seiner Konstituentenstellungsdomäne eine Beschreibung
von sich selbst enthält. (\mex{1}) zeigt eine Beschreibung, die für alle Köpfe zutrifft:
\ea
\ms[word]{
phon   & \ibox{1} \\
synsem & \ibox{2} \\
dom & \liste{ \ms[word]{ phon   & \ibox{1} \\
                         synsem & \ibox{2} \\
                         dom    & \liste{}\\
                       }
            } \\
}
\z
\itdopt{zyklische Strukturen}
Dabei sind unter dem Merkmal \synsem\isfeat{synsem}
sowohl syntaktische als auch semantische Informationen zusammengefasst.
Zu \textsc{synsem} siehe auch Kapitel~\ref{Kapitel-Lokalität}.

In Kopf"=Argument- und Kopf"=Adjunkt"=Strukturen werden Adjunkt- und Komplementtöchter in diese Liste eingesetzt 
und relativ zum Kopf angeordnet. Das sieht formal wie folgt aus:\footnote{
  In \citew{Mueller2002b} gilt diese Beschränkung für Verbalkomplexe nicht.%
}
\ea
\type{headed"=phrase}\istype{headed"=phrase} \impl
\ms{
  head-dtr$|$dom  & \ibox{1} \\
  non-head-dtrs   & \ibox{2} \\
  dom  & \ibox{1} $\bigcirc$ \ibox{2} \\
}
\z
Das Symbol `$\bigcirc$'\is{$\bigcirc$}\is{Relation!$\bigcirc$}\isrel{shuffle}\label{rel-shuffle} steht für die Relation
\emph{shuffle}. Die \textit{shuffle\/}"=Relation besteht zwischen drei Listen
A, B und C, gdw.\ C alle Elemente von A und B enthält und die Reihenfolge
der Elemente von A und die Reihenfolge der Elemente in B in C erhalten ist. (\mex{1}) zeigt die
Verknüpfung zweier zweielementiger Listen:
\ea
$\phonliste{ a, b } \bigcirc \phonliste{ c, d } =
\begin{tabular}[t]{@{}l}
\phonliste{ a, b, c, d } $\vee$\\*[1mm]
\phonliste{ a, c, b, d } $\vee$\\*[1mm]
\phonliste{ a, c, d, b } $\vee$\\*[1mm]
\phonliste{ c, a, b, d } $\vee$\\*[1mm]
\phonliste{ c, a, d, b } $\vee$\\*[1mm]
\phonliste{ c, d, a, b }
\end{tabular}$
\z
Das Ergebnis ist eine Disjunktion von sechs Listen. In all diesen Listen steht \emph{a} immer vor \emph{b}
und \emph{c} immer vor \emph{d}, weil das in den verknüpften Listen \phonliste{ a, b } und
\phonliste{ c, d } so ist.
\emph{b} kann aber durchaus zwischen oder hinter \emph{c} und \emph{d} stehen.

Elemente in der \doml können frei angeordnet werden, solange LP"=Regeln nicht verletzt werden.
Die Linearisierungsdomänen sind Kopfdomänen, \dash, nur Elemente, die zum selben Kopf gehören,
können relativ zueinander umgestellt werden. Dadurch wird erfasst, dass diese Art Umstellung 
ein lokales Phänomen ist.

Die Elemente der Konstituentenstellungsdomäne sind entsprechend der Reihenfolge in der Äußerung angeordnet.
Der \phonw eines phrasalen Zeichens ergibt sich also aus der Verknüpfung der einzelnen \phonwe der
Elemente in der \doml:
\ea
\type{phrase} \impl
\ms{
 phon & \ibox{1} $\oplus$ \ldots{} $\oplus$ \ibox{n} \\
 dom  & \sliste{ [ phon  \ibox{1}], \ldots, [phon \ibox{n}] } }
\z
Die folgenden Abbildungen zeigen die Analysen für die wichtigsten Fälle:
\eal
\ex\label{bsp-Aicke-dem-Affen-den-Stock-gibt}
(weil) Aicke dem Affen den Stock gibt
\ex\label{bsp-Aicke-den-Stock-dem-Affen-gibt}
(weil) Aicke den Stock dem Affen gibt
\ex Gibt Aicke den Stock dem Affen?
\zl
Die Variabilität in der Stellung wird in diesem Ansatz nicht dadurch erreicht, dass ein beliebiges
Element aus der \compsl mit dem Kopf verbunden werden kann, sondern dadurch, dass das Element, das
mit dem Kopf kombiniert wird, nicht unbedingt neben dem Kopf stehen muss.
Abbildung~\vref{abb-linearisierung-kont} zeigt die Analyse von (\mex{0}b). Die Elemente,
die in Abbildung~\ref{abb-linearisierung-kont} kombiniert werden, grenzen aneinander. Die Analyse
gleicht also der, die wir mit den in Kapitel~\ref{Kapitel-komplementation} beschriebenen Mitteln auch
bekommen würden. Neu ist lediglich die \doml. Bei der Kombination von \emph{gibt} mit seinen Argumenten
werden diese in die \doml eingesetzt. Die Anordnung der Konstituenten entspricht der Abfolge
der Domänenelemente am obersten Knoten in Abbildung~\ref{abb-linearisierung-kont}.
\begin{figure}
\begin{forest}
sm edges
[V\feattab{\comps \sliste{},\\
           \textsc{dom} \phonliste{ Aicke, dem Affen, den Stock, gibt }}
  [{\ibox{1} NP[\type{nom}]}
    [Aicke,roof]]
  [V\feattab{\comps \sliste{ \ibox{1} },\\
             \textsc{dom} \phonliste{ dem Affen, den Stock, gibt }}
    [{\ibox{2} NP[\type{dat}]} 
      [dem Affen, roof]]
    [V\feattab{\comps \sliste{ \ibox{1}, \ibox{2} },\\
               \textsc{dom} \phonliste{ den Stock, gibt }}
      [{\ibox{3} NP[\type{acc}]}
        [den Stock,roof]]
      [V\feattab{\comps \sliste{ \ibox{1}, \ibox{2}, \ibox{3} },\\
        \textsc{dom} \phonliste{ gibt }}
        [gibt]]]]]
\end{forest}
\caption{\label{abb-linearisierung-kont}Linearisierungsanalyse mit einem Beispiel für kontinuierliche Konstituenten}
\end{figure}

%\noindent
Die Analyse von (\ref{bsp-Aicke-den-Stock-dem-Affen-gibt}) zeigt
Abbildung~\vref{abb-linearisierung-diskont}. Im Gegensatz zu
(\ref{bsp-Aicke-dem-Affen-den-Stock-gibt}) grenzt \emph{den Stock} in
(\ref{bsp-Aicke-den-Stock-dem-Affen-gibt}) nicht an \emph{gibt}. In der Analyse von
(\ref{bsp-Aicke-den-Stock-dem-Affen-gibt}) bilden \emph{den Stock} und \emph{gibt} eine
diskontinuierliche Konstituente. Zwischen den beiden Elementen ist noch Platz, was man daran sehen
kann, dass \emph{dem Affen} am nächsthöheren Knoten zwischen den beiden Elementen zu stehen kommt.  
%
\begin{figure}
\begin{forest}
sm edges
[V\feattab{\comps \sliste{},\\
           \textsc{dom} \phonliste{ Aicke, \blau{den Stock}, dem Affen, \blau{gibt} }}
  [{\ibox{1} NP[\type{nom}]}
    [Aicke,roof]]
  [V\feattab{\comps \sliste{ \ibox{1} },\\
             \textsc{dom} \phonliste{ \blau{den Stock}, dem Affen, \blau{gibt} }}
    [{\ibox{2} NP[\type{dat}]} 
      [dem Affen, roof]]
    [V\feattab{\comps \sliste{ \ibox{1}, \ibox{2} },\\
               \textsc{dom} \phonliste{ \blau{den Stock, gibt} }}
      [{\ibox{3} NP[\type{acc}]}
        [den Stock,roof]]
      [V\feattab{\comps \sliste{ \ibox{1}, \ibox{2}, \ibox{3} },\\
        \textsc{dom} \phonliste{ gibt }}
        [gibt]]]]]
\end{forest}

\caption{\label{abb-linearisierung-diskont}Linearisierungsanalyse mit einem Beispiel für diskontinuierliche Konstituenten}
\end{figure}

\noindent
Abbildung~\vref{abb-linearisierung-diskont-verb} zeigt schließlich die Analyse eines Satzes mit Verberststellung. Hier
wird das Verb als erstes Element der Linearisierungsdomäne angeordnet. Die entstehenden Konstituenten sind zum Teil wieder
diskontinuierlich.
\begin{figure}
\begin{forest}
sm edges
[V\feattab{\comps \sliste{},\\
           \textsc{dom} \phonliste{ \blau{gibt}, Aicke, dem Affen, \blau{den Stock} }}
  [{\ibox{1} NP[\type{nom}]}
    [Aicke,roof]]
  [V\feattab{\comps \sliste{ \ibox{1} },\\
             \textsc{dom} \phonliste{ \blau{gibt}, dem Affen, \blau{den Stock} }}
    [{\ibox{2} NP[\type{dat}]} 
      [dem Affen, roof]]
    [V\feattab{\comps \sliste{ \ibox{1}, \ibox{2} },\\
               \textsc{dom} \phonliste{ \blau{gibt, den Stock} }}
      [{\ibox{3} NP[\type{acc}]}
        [den Stock,roof]]
      [V\feattab{\comps \sliste{ \ibox{1}, \ibox{2}, \ibox{3} },\\
        \textsc{dom} \phonliste{ gibt }}
        [gibt]]]]]
\end{forest}

\caption{\label{abb-linearisierung-diskont-verb}Linearisierungsanalyse mit einem Beispiel für diskontinuierliche Konstituenten und Verbstellung}
\end{figure}

Man kann sich leicht selbst davon überzeugen, dass die Dominanzstrukturen für alle Sätze in (\mex{0}) identisch sind.
Die Analysen unterscheiden sich lediglich hinsichtlich der Anordnung der Elemente in den Stellungsdomänen.

Abbildung~\vref{abb-linear-surface} stellt die Analyse von (\mex{0}c) noch einmal anders dar. Die Konstituenten sind hier entsprechend
der Oberflächenreihenfolge angeordnet. Man sieht in dieser Abbildung recht deutlich, dass diskontinuierliche Konstituenten vorliegen.
%
\begin{figure}

\forestset{
%  default preamble'={}, % The default preamble makes a mess in this situation.
  % 
  % This style visually relates a complement to some comps node. The argument
  % should be the intended SISTER.
  funky complement/.style={
    %tier=NP, % Horizontally align NPs ... not needed because they are all of the same height.
    for children={tier=word}, % Align the NP content with "gibt".
    for nodewalk/.process=Ow{#1.id}{id=##1}{% This monstrosity converts a relative node name (e.g. "!r1") to the id of that node.
      edge label={node[pos=0.5,left]{H}}, % Put "H" on the edge from the sister comps.
    },
    % Draw the funky edge from a complement to its visual parent. The line is
    % broken at the north of the sister comps, and also marked with "C" there.
    edge path'={(#1u.south) -- (.north |- {#1.north}) node[above]{C} -- (.north)},
  },
}
\begin{forest}
  % sn edges,   % note the s*n*, not s*m* // Can live without, as we draw the funky edges manually.
  for tree={l sep*=2}, % Let this tree be taller.
delay={where tier={word}{inner xsep=0}{}}, % unclear why it has to happen here with delay.
  [V\feattab{\comps \sliste{},\\
      \textsc{dom} \phonliste{ gibt, Aicke, dem Affen, den Stock }}, s sep+=7ex
    [V\feattab{\comps \sliste{ \ibox{1} },\\
        \textsc{dom} \phonliste{ gibt, dem Affen, den Stock }},
      [V\feattab{\comps \sliste{ \ibox{1}, \ibox{2} },\\
          \textsc{dom} \phonliste{ gibt, den Stock }}
        [V\feattab{\comps \sliste{ \ibox{1}, \ibox{2}, \ibox{3} },\\
            \textsc{dom} \phonliste{ gibt }},
          [gibt]]]]
    [,% This empty node holds all the complements. The idea is to put it next
      % to the comps node that should determine the x (s) position of the
      % complements (so I put it next to the second comps node). How does this
      % work? On its own, Forest squashes the complements close to "gibt", but
      % then the funky edge from NP [1] overlaps the second comps node. But
      % then we adjust the "s" of this empty complement holder (in "before
      %   computing xy") to avoid that.
      no edge, % Phantom does not work here, because we need the empty
               % complement holder to be physically present.
      calign=first, % Put this empty node directly above the leftmost NP ...
      before computing xy={% ... and set its position manually:
        %s/.option={!s.max x}, % to the (right) width of its sibling (the second comps)
        %s+/.option=!u.s sep,  % plus a bit more
      },
      tier=lowest comps, % Vertically align it to the lowest comps node (gibt).
      for parent={calign=first}, % The sister comps (Stock, Affe) should be
                                 % directly below its parent.
      for sibling={where n children=0{% This finds "gibt".
          for parent={tier=lowest comps}, % Mark the lowest comps for alignment with the empty complement holder.
          tier=word, % Align "gibt" with the content of NPs.
          %inner xsep=0, % Remove space otherwise the triangles get too big.
          % Does not work here, has to be done with delay at the top. St. Mü. 21.06.2024
        }{}},
      [{\ibox{1} NP[\type{nom}]}, funky complement=!r1 % the first child of the root
        [Aicke,roof]]
      [{\ibox{2} NP[\type{dat}]}, funky complement=!r11  % the first child of the first child of the root
        [dem Affen,roof]]
      [{\ibox{3} NP[\type{acc}]}, funky complement=!r111
        [den Stock,roof]]]
  ]
\end{forest}
\caption{\label{abb-linear-surface}Linearisierungsanalyse mit Konstituenten in Oberflächenreihenfolge}
\end{figure}

Nach dieser kurzen Vorstellung der Linearisierungsansätze von Reape, Kathol und mir sollen
jetzt die Probleme dieser Ansätze besprochen werden: Diese Ansätze haben denselben Nachteil
wie die Ansätze, die von flachen Strukturen ausgehen: Man kann nicht motivieren,
dass mehrere Konstituenten im Vorfeld eine gemeinsame Konstituente bilden.

\is{Vorfeldbesetzung|(}%
Außerdem haben Linearisierungsanalysen mit binär verzweigender Struktur den Nachteil,
dass man nicht ohne weiteres erklären kann,
wieso sowohl Dativobjekte als auch Akkusativobjekte mit dem Verb im Vorfeld stehen können.
Die Beispiele in (\mex{1}) zeigen, dass mit demselben Verb verschiedene Voranstellungen möglich sind,
je nach dem wie man die Argumentstellen belegt:
%% \NOTE{JB: findet Beispiele nicht schön, fragt nach
%%   Korpusbelegen, FB: find sie gut.}
\eal
\label{bsp-acc-dat-pvp-zwei}
\ex Märchen erzählen sollte man den Wählern nicht.
\ex Den Wählern erzählen sollte man diese Geschichte nicht.
\zl
\emph{den Wählern erzählen} und \emph{Märchen erzählen} bilden eine Konstituente (zu den Details
siehe Abschnitt~\ref{sec-pvp}). In Linearisierungsgrammatiken muss man die Argumente eines Kopfes
in einer festen Reihenfolge sättigen, da die Sättigungsreihenfolge von der Oberflächenreihenfolge unabhängig ist. Ließe
man beliebige Sättigungsreihenfolgen zu, bekäme man unechte Mehrdeutigkeiten\is{Mehrdeutigkeit!unechte}.
Nimmt man für \emph{erzählen} eine \compsl der Form $\langle$~NP[\textit{nom}], NP[\textit{dat}],
NP[\textit{acc}]~$\rangle$ an, dann kann man nur (\mex{0}a) analysieren, (\mex{0}b) bleibt unanalysierbar,
da \emph{den Wählern} erst mit \emph{erzählen} kombiniert werden kann, wenn die Kombination
mit dem Akkusativobjekt erfolgt ist.

\citet[\page242]{Kathol2000a} schlägt zur Lösung dieses Problems vor,
für die Objekte keine Reihenfolge in der Valenzliste festzulegen. Damit sind
die Sätze in (\mex{0}) zwar analysierbar, aber ein Satz wie (\mex{1}) hätte
dann zwei Analysen:
\ea
dass er den Wählern Märchen erzählt
\z
Da die Reihenfolge der Konstituenten im Satz in Linearisierungsgrammatiken
von der Reihenfolge der Elemente in der Valenzliste (hier \comps bei Kathol \subcat) unabhängig ist, kann man
für jede Abfolge in der \compsl jede Oberflächenreihenfolge ableiten und bekommt
somit für alle Sätze mit beiden Objekten im Mittelfeld unerwünschte unechte Mehrdeutigkeiten.

Für den hier vorgestellten Ansatz sind die Sätze in (\ref{bsp-acc-dat-pvp-zwei}) unproblematisch,
da das Kopf"=Argument"=Schema auf S.\,\pageref{schema-Kopf-Komplementschema-prel2} die Kombination von Argumenten
mit ihrem Kopf in beliebiger Reihenfolge zulässt.
\is{Vorfeldbesetzung|)}%
\is{Linearisierung!-sdomäne|)}%
\is{Verzweigung!binäre|)}


\subsubsection{Variable Verzweigung}
\label{crysmann}

In \citew[Kapitel~11.5.2]{Mueller99a} und \citew{Mueller2004b} habe ich darauf hingewiesen,
dass die maschinelle Bottom"=Up"=Verarbeitung von Grammatiken, die leere verbale Köpfe enthalten, sehr aufwendig
ist, da beliebige Phrasen mit den leeren verbalen Köpfen verbunden werden können, weil
Valenz und semantischer Beitrag der Verbspur so lange unbekannt sind, bis ihre Projektion
mit dem Verb in Erststellung kombiniert wird. 
%Auch ist ohne zusätzliche Hilfsmittel wie
%statistische Verfahren die Position der Verbspur nicht vorherzusagen, so dass
%
Berthold Crysmann hat die von mir im \verbmobil-Projekt
entwickelte Grammatik \citep{MK2000a} so verändert, dass sie sich effektiver verarbeiten lässt \citep{Crysmann2003b}.
Er hat dazu die unär verzweigenden Grammatikregeln, die der Verbspur entsprechen, entfernt und statt einer
Analyse mit uniformer Rechtsverzweigung eine Analyse mit Linksverzweigung bei unbesetzter
rechter Satzklammer und mit Rechtsverzweigung bei besetzter Satzklammer implementiert.
Für die beiden Sätze in (\mex{1}) gibt es also unterschiedliche Verzweigungen:
\eal
\ex {}[[[Gibt Aicke] dem Affen] den Stock]?
\ex {}[Hat [Aicke [dem Affen [den Stock gegeben]]]]?
\zl
In Crysmanns Analyse gibt es somit Verbbewegung, wenn ein Verbalkomplex vorliegt, und es
gibt keine Verbbewegung, wenn die rechte Satzklammer nicht besetzt ist.
Zu ähnlichen Vorschlägen siehe auch \citew*[\page225]{KW91a} und \citew*[\page 194, 201]{SRTD96a}.
Auf diese Weise hat man zwar die Verarbeitungsprobleme, die ein leerer verbaler Kopf mit sich
bringt, beseitigt, aber man hat auch keine Möglichkeit mehr, die scheinbar mehrfache
Vorfeldbesetzung\is{Vorfeldbesetzung} mit Hilfe eines leeren verbalen Kopfes\is{leere Kategorie} zu erklären.

Crysmann (in der Diskussion zu seinem Vortrag auf dem Workshop \emph{Large"=Scale Grammar Development and Grammar Engineering} 
2006 in Haifa) hat vorgeschlagen, die Tatsache, dass die Elemente, die bei der scheinbar mehrfachen Vorfeldbesetzung
vorangestellt werden, vom selben Verb abhängen müssen \citep[\page67]{Fanselow93a}, dadurch zu erfassen, dass bei der Einführung
der Fernabhängigkeiten offen gelassen wird, ob sich ein oder mehrere Elemente in \slasch befinden
(zu \slasch und der Behandlung der Fernabhängigkeiten siehe Kapitel~\ref{Kapitel-nla}).
Für die mehrfache Vorfeldbesetzung würde man annehmen, dass es mehrere Elemente in \slasch gibt,
die jeweils zum selben Kopf in Beziehung stehen. Die Fernabhängigkeiten können dann entweder mit einem
flach verzweigenden Schema gemeinsam oder aber eins nach dem anderen abgebunden werden.
Die Ausarbeitung einer solchen Analyse muss folgende Fakten erklären:
\begin{enumerate}
\item Sowohl Adjunkte als auch Argumente können in Sätzen mit scheinbar mehrfacher Vorfeldbesetzung auf"|treten.
      Man vergleiche die Datensammlung in \citew{Mueller2003b}, ein Beispiel sei hier gegeben (von S.\,36):
      \ea
      {}[Gezielt] [Mitglieder] [im Seniorenbereich] wollen die Kendoka allerdings nicht
      werben.\label{bsp-gezielt-mitglieder}\footnote{
        taz, 07.07.1999, S.\,18.
      }
      \z
\item Die Reihenfolge der Elemente vor dem finiten Verb entspricht für gewöhnlich der Reihenfolge, 
      die diese bei unmarkierter Stellung im Mittelfeld haben würden 
      \parencites[\page412--413]{Eisenberg94a}[Abschnitt~2.10]{Mueller2005d}.
      Die Beispiele in (\mex{1}) und (\mex{2}) sollen das verdeutlichen: (\mex{1}a) ist ein Beleg für mehrfache Vorfeldbesetzung
      und (\mex{1}b) die Variante mit Umstellung der Konstituenten innerhalb des Vorfelds.
\judgewidth{??}
\eal
\ex[]{ {}[Alle Träume] [gleichzeitig] lassen sich nur selten 
      verwirklichen.\label{bsp-alle-traeume-gleichzeitig}\footnote{
        Broschüre der Berliner Sparkasse, 1/1999.
        }
}
\ex[?*]{
Gleichzeitig alle Träume lassen sich nur selten verwirklichen.
}
\zl
(\mex{0}b) ist markiert und genauso verhält es sich mit der Abfolge in (\mex{1}b) im Vergleich zu (\mex{1}a):
\eal\NOTE{FB: findet b toll.}
\ex[]{
weil sich nur selten alle Träume gleichzeitig verwirklichen lassen
}
\ex[??]{
weil sich nur selten gleichzeitig alle Träume verwirklichen lassen
}
\zl
In den natürlich vorkommenden Belegen aus dem IDS"=Korpus und anderen Korpora, die
\citet{Bildhauer2011a} gefunden hat, lag jeweils die präferierte Mittelfeldstellung vor.
\item In Fällen, in denen mehr als ein Bestandteil eines idiomatischen Ausdrucks vorangestellt wurde,
      kann mitunter einer der Idiombestandteile nicht in Isolation vorangestellt werden, ohne dass die
      Äußerung ungrammatisch würde bzw.\ die idiomatische Lesart verlöre \citep[Abschnitt~2.9]{Mueller2005d}.
\eal
\ex {}[Öl] [ins Feuer] goß\iw{gießen!Öl ins Feuer $\sim$} gestern das Rote-Khmer-Radio:
      \ldots\footnote{
        taz, 18.06.1997, S.\,8.
}
\ex\iw{setzen!das Tüpfel aufs i $\sim$}
{}[Das Tüpfel] [aufs i] setzte der Bürgermeister von Miami, als er am Samstagmorgen von einer schändlichen 
Attacke der US-Regierung sprach.\footnote{
        taz, 25.04.2000, S.\,3. %taz Nr. 6126 vom 25.4.2000 Seite 3 im Original steht `setze'
    }
\ex {}[Ihr Fett] [weg] bekamen natürlich auch alte und neue Regierung [\ldots]\footnote{
        Mannheimer Morgen, 10.03.1999, Lokales; SPD setzt auf den "`Doppel-Baaß"'. %M99/903.16159 Mannheimer Morgen, 10.03.1999, Lokales; SPD setzt auf den "Doppel-Baaß"
      }
\zl
\eal
\ex[*]{
Ins Feuer goß gestern das Rote-Khmer-Radio Öl.
}
\ex[*]{
Aufs i setzte der Bürgermeister von Miami das Tüpfel, als er am Samstagmorgen von einer schändlichen 
Attacke der US-Regierung sprach.
}
\ex[*]{
Weg bekamen natürlich auch alte und neue Regierung ihr Fett.
}
\zl
\end{enumerate}
Diese Fakten sind für eine Analyse mit mehreren Elementen in \slasch aus folgenden Gründen problematisch:
Der semantische Beitrag einer Konstituente muss berücksichtigt werden. In der HPSG"=Theorie
wird das bei der Einführung der Fernabhängigkeit gemacht. Deshalb müsste die Tatsache, dass
sowohl Adjunkte als auch Argumente in den verschiedensten Kombinationen im Vorfeld stehen
können, bei der Einführung der Fernabhängigkeiten berücksichtigt werden. Werden Adjunkte
syntaktisch eingeführt, bedeutet das, dass der \modw eines zu extrahierenden Adjunkts mit dem
\synsemw des Kopfes, den das Adjunkt modifiziert, identifiziert werden muss. Bei Argumenten
ist dies natürlich nicht der Fall.\footnote{
  Bei einer Analyse, die Adjunkte als Argumente behandelt \citep*{BMS2001a}, entfällt dieses Problem. Allerdings hat
  eine solche Analyse Probleme mit bestimmten Skopusverhältnissen bei koordinierten VPen.
  Siehe \citew{Cipollone2001a,Levine2003a}, \citew[Kapitel~3.6.1]{LH2006a}. Siehe aber auch \citew{Chaves2009a}.%
}
Das heißt, man kann nicht ohne weiteres, wie das von
Crysmann vorgeschlagen wurde, annehmen, dass die Anzahl der extrahierten Elemente unterspezifiziert
bleibt. Es ist sicher möglich, eine unterspezifizierte Einführung von Fernabhängigkeiten
zu entwickeln, allerdings ist das dann eine Spezialbehandlung, während bei Verwendung
eines leeren Kopfes, wie er für die Verbbewegung verwendet wurde, keine zusätzlichen Annahmen
nötig sind. Adjunkte im Vorfeld werden ganz normal über das Kopf"=Adjunkt"=Schema mit dem leeren
verbalen Kopf verbunden und Argumente werden ebenfalls unspektakulär über das Kopf"=Argument"=Schema
mit dem leeren verbalen Kopf kombiniert \citep[Abschnitt~4]{Mueller2005d}.

Zum zweiten Punkt ist zu sagen, dass eine Analyse ohne einen verbalen Kopf im Vorfeld nicht erklären
kann, warum das nichtverbale Material im Vorfeld denselben Linearisierungsgesetzmäßigkeiten 
wie im Mittelfeld unterliegt.\footnote{
  Siehe auch Kapitel~\ref{sec-disc-entry} (insbesondere die Diskussion der Beispiele (\ref{bsp-partikel+anderes+material-topo-felder})
  auf Seite~\pageref{bsp-partikel+anderes+material-topo-felder}) zu Verbpartikeln, die mit anderem Material zusammen
  innerhalb eines komplexen Vorfelds stehen können. Die Partikel steht dann innerhalb des Vorfelds in der rechten Satzklammer
  und die anderen Konstituenten im Mittelfeld bzw.\ im Nachfeld.%
} Wollte man die Abfolge der \textsc{slash}"=Elemente bei der Einführung der Fernabhängigkeiten
regeln, so müsste man dort auf alle linearisierungsrelevante Information Bezug nehmen und \zb
sagen, dass Pronomina in der \slashl vor nichtpronominalen Elementen stehen müssen (siehe auch Kapitel~\ref{sec-lex-intro-udc}
zur lexikalischen Einführung von Fernabhängigkeiten). Alternativ
könnte man diese Beschränkungen bei der Abbindung der Fernabhängigkeiten überprüfen. In jedem Fall
wären gesonderte Mechanismen stipuliert worden, die von den Mechanismen, die für die Anordnung
im Mittelfeld verwendet werden, verschieden sind.

Bei der Annahme eines leeren verbalen Kopfes kann man die Verhältnisse hingegen leicht erklären:
Innerhalb des komplexen Vorfelds gibt es wieder ein Mittelfeld.
Die Umordnung im Mittelfeld ist mit bestimmten pragmatischen Effekten verbunden und die Voranstellung
von komplexen Verbalprojektionen ebenso. Wenn beides zusammen auf"|tritt, gibt es Konflikte bei der
Erfüllung der pragmatischen Restriktionen, weshalb die Beispiele mit Umstellungen innerhalb eines
komplexen Vorfelds markierter sind.

Auch der dritte Punkt lässt sich in einer \textsc{slash}"=basierten Analyse nicht ohne weiteres erklären:
Hier müssten Beschränkungen bei der Einführung bzw.\ Abbindung von Idiombestandteilen
in \slasch formuliert werden. So müsste man \zb irgendwie sicherstellen, dass \emph{ins Feuer} nur
dann in \slasch aufgenommen werden darf, wenn auch \emph{Öl} in \slasch ist.
Bei der Analyse mit der Verbspur ist das hingegen nicht nötig: In einer Theorie, die davon ausgeht, dass
\emph{Öl ins Feuer gießen} einen Komplex nach der Art der Verbalkomplexe bildet, die in
Kapitel~\ref{Kapitel-Verbalkomplex} besprochen werden, folgt automatisch, dass \emph{ins Feuer} nicht
aus der Mitte des Komplexes vorangestellt werden kann. Ob das Verb dabei als Spur oder overt neben
\emph{Feuer} realisiert wird, ist dabei unerheblich. Siehe Kapitel~\ref{sec-pvp}.

%% In der Diskussion von (\ref{bsp-absichtlich-nicht-anal}) haben wir gesehen, dass linksstehende
%% Elemente Skopus über rechtsstehende Elemente haben. Dabei werden Verberstsätze wie (\ref{bsp-absichtlich-nicht-anal-v1}) 
%% so interpretiert, als stünde das Verb noch am Ende. Ansätze mit variabler Verzweigung können das nicht erfassen.
%% \citet{Crysmann2003b} argumentiert, dass das nicht allgemein gilt, und dass die Skopusregel
%% deshalb als extragrammatische Präferenzregel behandelt werden sollte. Er gibt das folgende
%% Beispiel an:
%% \ea
%% Da muss es schon erhebliche Probleme mit der Ausrüstung gegeben haben, da wegen schlechten Wetters ein Reinhold Messmer niemals aufgäbe.
%% \z
%% In (\mex{0}) hat das \emph{niemals} Skopus über \emph{wegen des schlechten Wetters} obwohl
%% es rechts der PP steht. Bei diesem Beispiel handelt es sich jedoch wahrscheinlich um eine Fokusumstellung.
%% Diese Art Umstellung hat andere Eigenschaften als die bisher behandelten Mittelfeldumstellungen.
%% Sie ähnelt Voranstellungen, wie wir sie im nächsten Kapitel kennenlernen werden.
%% Die abweichende Interpretationsmöglichkeit ist also auch in dem hier vertretenen Ansatz erklärbar.


\subsection{Andere Theorien}
\label{sec-alternativen-konstituentenreihenfolge-gb}

Im Abschnitt~\ref{sec-mf} wurde eine Möglichkeit für die Analyse der Stellung von Konstituenten im Mittelfeld vorgestellt.
In der \gbt und ihren Nachfolgern wird die Umstellung von Konstituenten mitunter mit Bewegung modelliert.
Konstituenten werden in einer zugrundeliegenden Struktur in einer bestimmten Abfolge lizenziert
und dann über Umstellungen in die endgültige Position gebracht. Welcher Art die Zielpositionen sind,
ist von Analyse zu Analyse verschieden. In den folgenden Abschnitten sollen einige GB"=Varianten
besprochen werden. Eine LFG"=Analyse der Abfolgevarianten im Mittelfeld wird erst im nächsten
Kapitel besprochen, da sie mit der Vorfeldbesetzung interagiert.

\subsubsection{Funktionale Kategorien}

In den folgenden beiden Teilabschnitten diskutiere ich erst Ansätze, die die funktionalen Projektionen
AgrS, AgrO und AgrIO benutzen und dann solche, die spezielle funktionale Projektionen 
mit Informationsstrukturbezug verwenden.

\subsubsubsection{AgrS, AgrO und AgrIO}

\is{Kategorie!funktionale|(}%
Als Beispiel sei in Abbildung~\vref{fig-konstituentenstellung-meinunger} eine Struktur von \citet{Meinunger2000a} angegeben.\footnote{
  Siehe auch \citew{Haftka96a}.%
}

\begin{figure}
\oneline{%
\begin{forest}
% if we are on the word tier, set the inner xsep to 0. This gets the size of the roof right.
% without this the roof would overlap. St. Mü. 21.06.2024
delay={where tier={word}{inner xsep=0}{}}
[AgrSP
  [SU [die Firma Müller,roof,tier=word]]
  [AgrS$'$ 
    [AgrIOP
      [IO [meinem Onkel, roof,tier=word]]
      [AgrIO$'$
        [AgrOP
          [DO [diese Möbel, roof,tier=word]]
          [AgrO$'$ 
            [VP, s sep+=7pt
              [Adv [erst gestern, roof,tier=word]]
              [VP
                [t\sub{SU}]
                [V$'$
                  [t\sub{IO}]
                  [VP
                    [t\sub{DO}]
                    [V [zugestellt,tier=word]]]]]]
             [AgrO$^0$]]]
         [AgrIO$^0$]]]
       [AgrS$^0$ [hat,tier=word]]]]
\end{forest}}
% bei ihm ist noch ein daß im Bild
\caption{\label{fig-konstituentenstellung-meinunger}Analyse für \emph{dass die Firma Müller meinem Onkel diese Möbel erst gestern zugestellt hat} 
nach \citew[\page101]{Meinunger2000a}}
\end{figure}
Das Subjekt (SU), das indirekte Objekt (IO) und das direkte Objekt (DO) werden als Konstituenten
einer komplex strukturierten VP erzeugt und dann in spezielle Positionen (Spezifikatorpositionen)
funktionaler Projektionen verschoben. Die Stellen, an denen die Konstituenten eigentlich stehen würden,
sind durch Spuren (t\sub{SU}, t\sub{IO}, t\sub{DO}) markiert. AgrS, AgrIO und AgrDO sind dabei leere
Kategorien,\is{leere Kategorie} die für die Analyse von Kongruenz eine Rolle spielen. AgrIO und
AgrDO wurden ursprünglich für Sprachen mit Objektkongruenz\is{Kongruenz!Objekt-} eingeführt, ein
Phänomen, das es im Deutschen nicht gibt.


In der HPSG wurde in den letzten Jahren versucht, ohne leere Elemente auszukommen. Meiner Meinung
nach ist dies jedoch nicht möglich, wenn man die Zusammenhänge auf einsichtsvolle Weise erfassen
will. Eine genauere Diskussion hierzu findet man in \citew{Mueller2004e}. Allerdings ist man immer
noch bestrebt, so wenig wie möglich leere Elemente zu verwenden, \dash, man versucht die Stipulation
unsichtbarer Einheiten zu vermeiden, wo immer es geht.
Kongruenz wird in der HPSG deshalb über die Identität von Merkmalen erzwungen (siehe
Kapitel~\ref{Kapitel-kongruenz}), leere Agreement"=Köpfe und Bewegungen in bestimmte Baumpositionen
spielen bei der Beschreibung keine Rolle.  Man modelliert vielmehr direkt beobachtbare Eigenschaften
linguistischer Zeichen: In (\mex{1}a) haben die NP und das Verb die Eigenschaft, Plural zu sein, und
in (\mex{1}b) sind sie Singular.

\eal
\ex Die Kinder schlafen.
\ex Das Kind schläft.
\zl
Genauso können informationsstrukturelle\is{Informationsstruktur} Eigenschaften direkt modelliert
werden, und man muss dazu keine komplexen Strukturen annehmen. Zur Behandlung der
Informationsstruktur im Rahmen der HPSG siehe \citew{EV96a}, \citew{deKuthy2002a},
\citew{Bildhauer2008a}, \citew{Paggio2009a-u} 
und \citew{Song2017a-u}. \citet{DeKuthyInformationStructureHandbook} gibt einen Überblick über die Behandlung der
Informationsstruktur im Rahmen der HPSG.%


\subsubsubsection{TopP, FocP und KontrP}

\mbox{}\citet{Frey2004a}\NOTE{Seitenzahl}\is{Topik|(}\is{Fokus|(}\is{Informationsstruktur|(} nimmt eine KontrP (Kontrastphrase) und 
%\citet[\page165]{Frey2000a-u} 
\citet{Frey2004b-u}
eine TopP (Topikphrase) an (siehe auch \citew{Rizzi97a-u} für TopP und FocP (Fokusphrase) im Italienischen und 
\citew{Haftka95a}, \citew[\page19]{Abraham2003a} für Analysen mit TopP und/""oder FocP für das Deutsche).
Konstituenten müssen je nach ihrem informationsstrukturellen Status in die Spezifikatorpositionen
dieser funktionalen Köpfe bewegt werden. \citet{Fanselow2003b} hat gezeigt, dass solche bewegungsbasierten
Theorien für die Anordnung von Elementen im Mittelfeld nicht mit gegenwärtigen Annahmen innerhalb
des Minimalistischen Programms\is{Minimalistisches Programm} (aus dem diese Theorien stammen) kompatibel sind. Der Grund ist,
dass manche Umstellungen erfolgen, um anderen Elementen Platz zu machen. So wird zum Beispiel
durch die Anordnung in (\mex{1}b) erreicht, dass nur \emph{verhaftete} oder auch nur \emph{gestern}
fokussiert sein kann.\NOTE{FB: Das ist komisch. Damit in (\mex{1}b) nur \emph{gestern} fokussiert
  sein kann, braucht \emph{gestern} die Hauptbetonung. Aber das geht doch auch in (\mex{1}a). Und
  für \emph{verhaftete} gilt (denke ich) das gleiche.}
\eal
\ex dass die Polizei gestern Linguisten verhaftete
\ex dass die Polizei Linguisten gestern verhaftete
\zl
Fanselow formuliert die Generalisierung in Bezug auf Umstellungen so: Ein direktes Objekt wird
umgestellt, wenn die mit einem Satz verbundene Informationsstruktur verlangt, dass entweder
eine andere Konstituente im Fokus ist oder dass das Objekt nicht Teil des Fokus ist. Im Deutschen
kann man Teilfokussierungen auch mit besonderer Intonation erreichen, in Sprachen wie dem Spanischen
ist das jedoch nicht möglich.

Man kann also nicht annehmen, dass Elemente in eine bestimmte Baumposition bewegt werden müssen,
da dort ein Merkmal überprüft werden muss. Das ist jedoch eine Voraussetzung für Bewegung in
der gegenwärtigen Minimalistischen Theorie. \textcites[Abschnitt~4]{Fanselow2003b}[\page8]{Fanselow2006a} 
hat außerdem gezeigt, dass sich die Abfolgebeschränkungen,
die man für Topik und Fokus und Satzadverbiale
%\NOTE{FB: vielleicht erklären} 
feststellen kann, mit einer Theorie erfassen lassen,
die zu der hier vorgestellten parallel ist: Argumente können eins nach dem anderen mit ihrem Kopf
kombiniert (in Minimalistischer Terminologie: \emph{gemerget}\is{Merge} werden) und Adjunkte können an jede
Projektionsstufe angeschlossen werden. Die Stellung der Satzadverbien direkt vor dem fokussierten
Teilbereich des Satzes wird semantisch erklärt: Da Satzadverbien sich wie fokussensitive Operatoren
verhalten, müssen sie direkt vor den Elementen stehen, auf die sie sich beziehen. Daraus folgt,
dass Elemente, die nicht zum Fokus gehören (Topiks), vor dem Satzadverb stehen müssen. Eine besondere
Topikposition ist zur Beschreibung lokaler Umstellungen im Mittelfeld jedoch nicht nötig.
\is{Topik|)}\is{Fokus|)}%
\is{Kategorie!funktionale|)}\is{Informationsstruktur|)}

\subsubsection{Quantorenskopus}
\label{sec-Scrambling-Skopus}

Außer\is{Skopus|(}
dem Ziel, Eigenschaften von Objekten möglichst direkt zu modellieren, gibt es aber noch wichtigere
Gründe, keine Bewegungsanalyse für die relativ freie Abfolge von Konstituenten im Mittelfeld anzunehmen:
Lange Zeit wurde gegen Ansätze, die die Konstituentenstellung im \mf ohne Bewegung erklären wollen,
mit Skopus"=Daten argumentiert. Sätze wie (\mex{1}b) sind hinsichtlich des Quantorenskopus ambig.
\eal
\ex Es ist nicht der Fall, dass er mindestens einem Verleger fast jedes Gedicht anbot.
\ex Es ist nicht der Fall, dass er fast jedes Gedicht$_i$ mindestens einem Verleger \_$_i$ anbot.
\zl
Es wurde behauptet, dass man zur Erklärung der Ambiguität verschiedene Strukturen benötigt, nämlich
zum einen die direkt beobachtbare und zum anderen solche, die der Rückübertragung einer oder
mehrerer umgestellter Phrasen in die Position der Spur entsprechen \citep{Frey93a}.
Nun hat sich aber herausgestellt, dass Ansätze, die Spuren annehmen,
problematisch sind, denn sie sagen für Sätze, in denen es mehrere Spuren gibt, 
Lesarten voraus, die nicht wirklich vorhanden sind (siehe \citew[\page 146]{Kiss2001a} und \citew[Abschnitt~2.6]{Fanselow2001a}).
So könnte \zb in (\mex{1}) \emph{mindestens einem Verleger} an der Stelle von \_$_i$ interpretiert werden,
was dann zur Folge hätte, dass man eine Lesart bekommt, in der
\emph{fast jedes Gedicht} Skopus über \emph{mindestens einem Verleger} hat.
\ea
Ich glaube, dass mindestens einem Verleger$_i$ fast jedes Gedicht$_j$ nur dieser Dichter \_$_i$ \_$_j$ angeboten hat.
\z
Eine solche Lesart gibt es aber nicht.%
\itdopt{Salzmann, Webelhuth}
%
\is{Skopus|)}

\subsubsection{Freezing}
\is{Freezing@\emph{Freezing}|(}

In \gb nimmt man an, dass aus bewegten Phrasen nichts mehr extrahiert, \dash \zb ins Vorfeld gestellt werden kann.
Das wurde als Test für den Status der Umstellungen im \mf benutzt. Zum Beispiel \citet{Diesing92a} behauptet,
dass der Inselstatus\is{Extraktion!sinsel} von umgestellten Phrasen für eine Bewegungsanalyse spricht. Wie aber
das folgende Beispiel aus \citew[\page101]{Mueller99a} zeigt, ist die Behauptung empirisch nicht haltbar:
\eal
\ex {}[Zum Gartenvereinsvorsitzenden]$_i$ hätte er [das Talent \_$_i$].\iw{Talent}
\ex {}[Zum Gartenvereinsvorsitzenden]$_i$ hätte [das Talent \_$_i$] wohl nur dieser Mann.\label{bsp-gartenvereinsvorsitzender}
\zl
Siehe auch \citew[\page187--192]{Fanselow91a} und \citew[Abschnitt~2.3]{Fanselow2001a}.%
\is{Freezing@\emph{Freezing}|)}

%\section*{Kontrollfragen}

\questions{
\begin{enumerate}
\item Welche Konstituentenstellungsphänomene kennen Sie?
\item Warum wurden Kopf"=Spezifikator"=Strukturen eingeführt?
\end{enumerate}
}

%\section*{Übungsaufgaben}

\exercises{
\begin{enumerate}
\item Skizzieren Sie die Analysebäume für folgende Sätze und erläutern Sie,
      wie die Unterschiede zwischen den Sätzen von der in diesem Kapitel
      vorgestellten Analyse erfasst werden:
\eal
\ex dass Kurt-Martin drei Mark dem Clown gibt
\ex dass dem Clown Kurt-Martin drei Mark gibt 
\ex Gibt Kurt-Martin dem Clown drei Mark?
\zl

\item Laden Sie die zu diesem Kapitel gehörende Grammatik (siehe Übung~\ref{uebung-grammix-kapitel4} auf Seite~\pageref{uebung-grammix-kapitel4}).
Im Fenster, in dem die Grammatik geladen wird, erscheint zum Schluss eine Liste von Beispielen.
Geben Sie diese Beispiele nach dem Prompt ein und wiederholen Sie die in diesem Kapitel besprochenen
Aspekte.

\item (Zusatzaufgabe) Laden Sie das Babel"=System, das auf der Grammix"=CD bzw.\ in der Grammix-Virtual-Machine enthalten ist. 
Nach Beendigung des Ladevorgangs erscheint ein Prompt (\texttt{>>>}). Geben Sie den folgenden Satz
ein:
\ea
Er glaubt, dass Aicke dem Affen den Stock gibt.
\z
Sehen Sie sich im Chart-Display die Kanten für \emph{Aicke dem Affen den Stock gibt} an. Mit der
rechten Maustaste bekommen Sie im Chart"=Display ein Menü, in dem es einen Eintrag gibt, mit dem man
nur die Kanten anzeigen lassen kann, die zu einer Analyse der vollständigen Eingabe beitragen. Wenn
sie diesen Menüpunkt anklicken, werden Sie sehen, dass es zwei Kanten gibt, die \emph{dem Affen den Stock
gibt} überspannen. Die untere der beiden ist diskontinuierlich, was man sehen kann, wenn man
die Kante anklickt und den Menüpunkt "`Zeige Kinder"' auswählt. Experimentieren Sie mit anderen Sätzen
aus dem Abschnitt~\ref{sec-Linearisierung}.
\end{enumerate}
}

%\section*{Literaturhinweise}

\furtherreading{
\citew{Mueller2004b} diskutiert die Möglichkeiten, die es im Rahmen der HPSG für die Analyse der
relativ freien Konstituentenstellung gibt. Teile dieser Diskussion sind auch hier im
Alternativenabschnitt enthalten.

Die Analyse des Deutschen als Verbletztsprache ist relativ alt \parencites{Bach62a}[\page34]{Bierwisch63a}{Reis74a}[Kapitel~1]{Thiersch78a}. Bierwisch
schreibt die Annahme einer zugrundeliegenden Verbletztstellung \citet{Fourquet57a} zu. Eine
Übersetzung des von Bierwisch zitierten französischen Manuskripts kann man in
\citew[\page117--135]{Fourquet70a} finden.

Auf Seite~\pageref{page-verbletzt} habe ich Argumente für die Annahme aufgeführt, dass die Stellung
mit dem Verb in Letztposition die Grundstellung ist. Die Linearisierungsanalysen, die in
\citew{Reape94a}, \citew{Kathol2001a} und \citew{Mueller99a,Mueller2002b,Mueller2004b} vertreten wurden, sind mit
diesen Beobachtungen aber auch kompatibel. Einzig und allein das Phänomen der scheinbar mehrfachen
Vorfeldbesetzung \citep{Mueller2005d,MuellerGS} stellt wirklich harte Evidenz gegen linearisierungsbasierte Analysen dar.
}