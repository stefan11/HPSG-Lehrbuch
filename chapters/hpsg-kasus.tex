%% -*- coding:utf-8 -*-
%%%%%%%%%%%%%%%%%%%%%%%%%%%%%%%%%%%%%%%%%%%%%%%%%%%%%%%%%
%%   $RCSfile: hpsg-kasus.tex,v $
%%  $Revision: 1.15 $
%%      $Date: 2008/09/30 09:14:41 $
%%     Author: Stefan Mueller (CL Uni-Bremen)
%%    Purpose: 
%%   Language: LaTeX
%%%%%%%%%%%%%%%%%%%%%%%%%%%%%%%%%%%%%%%%%%%%%%%%%%%%%%%%%



\chapter{Kasus}
\label{chap-kasus}

In diesem Kapitel werden verschiedene Arten von Kasus vorgestellt, und es wird gezeigt,
wie sich die Kasusvergabe in Abhängigkeit vom syntaktischen Kontext regeln läßt.
Die Formulierung allgemeiner Prinzipien für die Kasusvergabe ist sehr sinnvoll,
da sie es ermöglicht, im Lexikon Information über Kasus unterspezifiziert zu lassen.
Somit braucht man für die Sätze in (\mex{1}) nur einen Lexikoneintrag für das Verb \emph{lesen}:
\eal
\ex Er möchte das Buch lesen.
\ex Ich sah ihn das Buch lesen.
\zl
In (\mex{0}a) hat das Subjekt von \emph{lesen} Nominativ, in (\mex{0}b) dagegen Akkusativ.
Welchen Kasus das Subjekt von \emph{lesen} bekommt, hängt vom Kontext ab, in dem das
Verb \emph{lesen} verwendet wird, und wird über ein allgemeines Prinzip geregelt. Damit
ein solches Prinzip formuliert werden kann, muß klar sein, was für Arten von Kasus es
(im Deutschen) gibt. Verschiedene Kasusarten werden im folgenden Abschnitt vorgestellt.

\section{Das Phänomen}

Im folgenden Abschnitt wird die Unterscheidung zwischen lexikalischem und strukturellem
Kasus eingeführt. Abschnitt~\ref{sec-sem-kasus} beschäftigt sich mit dem sogenannten
semantischen Kasus, bestimmten Kasusformen, die durch die Verwendung eines Adjunkts
mit einer bestimmten Bedeutung erzwungen sind. 
%Kongruenzkasus wird im Abschnitt~\ref{sec-kongruenzkasus} besprochen.

\subsection{Der Kasus von Argumenten: Struktureller und lexikalischer Kasus}
\label{sec-struk-lex-kas}
\label{sec-struc-lex-kas}
\is{Kasus!struktureller|(}%
\is{Kasus!lexikalischer|(}

Es gibt Prädikate, deren Argumente Kasus haben, der von der syntaktischen Umgebung,
in der das Prädikat realisiert wird, abhängt. Man bezeichnet solche Argumente
als Argumente mit strukturellem Kasus. Bei kasusmarkierten Argumenten,
die keinen strukturellen Kasus haben, spricht man von lexikalischem Kasus.

\is{Verb!AcI-|uu}
Beispiele für strukturellen Kasus sind:\footnote{
        Vergleiche \citew*[\page 200]{HM94a}\iaf{Heinz}\iaf{Matiasek}.

        Bei (\mex{1}b) handelt es sich um eine AcI"=Konstruktion. AcI heißt Akkusativ mit
        Infinitiv. Das logische Subjekt des eingebetteten Verbs (im Beispiel \emph{kommen})
        wird zum Akkusativobjekt des Matrixverbs (im Beispiel \emph{lassen}).
        Beispiele für AcI-Verben sind Wahrnehmungsverben\is{Verb!Wahrnehmungs-} wie \emph{hören} und 
        \emph{sehen} sowie \emph{lassen}. Siehe auch Kapitel~\ref{sec-aci}.
}
\eal
\ex Der Installateur kommt.
\ex Der Mann läßt\iw{lassen} den Installateur kommen.
\ex das Kommen des Installateurs
\zl
Im ersten Satz wird dem Subjekt Nominativ\is{Kasus!Nominativ} zugewiesen, wogegen \emph{Installateur} im
zweiten Satz im Akkusativ\is{Kasus!Akkusativ} steht und im dritten in Verbindung mit der Nominalisierung\is{Nominalisierung}
im Genitiv\is{Kasus!Genitiv}. Der Akkusativ von Objekten ist gewöhnlich auch ein struktureller Kasus. 
Bei Passivierung\is{Passiv} wird er zum Nominativ:
\eal
\ex Karl schlägt den Hund.
\ex Der Hund wird geschlagen.
\zl
Im Gegensatz zum Akkusativ ist der von einem Verb abhängige Genitiv\is{Kasus!Genitiv} ein lexikalischer Kasus:
Bei Passivierung ändert sich der Kasus eines Genitivobjekts nicht.
\eal
\ex Wir gedenken der Opfer.\iw{gedenken}
\ex Der Opfer wird gedacht.
\zl
In (\mex{0}b) handelt es sich um ein sogenanntes unpersönliches Passiv\is{Passiv!unpersönliches},
\dash, im Gegensatz zum Beispiel (\mex{-1}b), in dem das Akkusativobjekt zum Subjekt wird, 
gibt es in (\mex{0}b) kein Subjekt.

Genauso gibt es keine Veränderungen bei Dativobjekten\is{Kasus!Dativ|(}:
\eal
\ex Der Mann hat ihm geholfen.
\ex Ihm wird geholfen.
\zl
Es wird kontrovers diskutiert, ob einige oder alle Dative in verbalen Umgebungen als
strukturelle Kasus behandelt werden sollten \citep{denBesten85,denBesten85b,Fanselow87a,Fanselow2000b,%
Fanselow2003b,% 206
Czepluch88,Wegener90,Sternefeld95a,% 77
Stechow96a,Wunderlich97a,Wunderlich97c,Ryu97a,Molnarfi98a,Gunkel2003b}
% Welke2009a:530
%Wunderlich97a:48,
%Wunderlich97c:107,Ryu97a:199,denBesten85:26
%----------------------------------------------------------------------------------------------------
oder ob alle Dative lexikalischen Kasus bekommen \citep{Haider85b,Haider86,Haider93a,%
HR2003a,% S. 222
HM94a,%
Scherpenisse86a,%S.\,97
Pollard94a,Mueller99a,Mueller2001a,Meurers99b,VS98a,Abraham95a-u,%
McIntyre2006a}. %S. 187
% MdK2001a sagen explizit nichts zum Dativ
%
% Abraham86a:12 Dativ von helfen = lexikalisch
% Abraham93a:176 behauptet, daß freie Dative nicht strukturell sein kann
% Abraham95a: 236, 2005:219, 235
% Wegener90:98 behauptet nicht von allen Dativen, daß sie strukturell sind (zB grauen)
%
Das sogenannte Dativpassiv\is{Passiv!Dativ-}, das mit Verben wie
\word{bekommen}, \word{erhalten} und \word{kriegen} gebildet werden kann,
wird als Evidenz für den Dativ als strukturellen Kasus gesehen.
In (\mex{1}b) steht das Dativargument von \emph{schenken} im Nominativ:
\eal
\ex Der Mann  hat   den Ball dem Jungen geschenkt.
\ex Der Junge bekam den Ball            geschenkt.
\zl
Manche derjenigen, die den Dativ zu den lexikalischen Kasus zählen, nehmen
einen besonderen Prozeß an, der im Zusammenhang mit dem Dativ"=Passiv
eine Dativ"=NP in eine NP mit strukturellem Kasus umwandelt 
(\citealp[Abschnitt~4.1]{Haider86}; \citealp[\page 228]{HM94a}; \citealp[\page 298]{Mueller99a}).
\citet{Gunkel2003b} kritisiert diese Ansätze zu Recht, denn wenn man all die
Kasus zu den strukturellen Kasus zählen will, die sich in Abhängigkeit von
ihrer syntaktischen Umgebung ändern können, dann muß man Argumentdative wie die
in (\mex{0}a) zu den strukturellen Kasus zählen.

Ich werde trotzdem den Dativ zu den lexikalischen Kasus zählen und möchte dies im folgenden
begründen: \citet[\page 20]{Haider86} weist darauf hin, dass Beispiele wie die folgenden Evidenz
für eine Behandlung des Dativs als lexikalischen Kasus darstellen:
\eal
\ex[]{\iw{streicheln}
Er       streichelt den Hund.
}
\ex[]{
Der Hund wird gestreichelt.
}
\ex[]{
sein Streicheln des    Hundes
}
\ex[]{\label{bsp-er-hilft-den-kindern}\iw{helfen}
Er hilft den Kindern.
}
\ex[]{
Den Kindern wird geholfen.
}
\ex[]{
das Helfen der Kinder
}\label{das-helfen-der-Kinder}
\ex[*]{
sein Helfen der Kinder
}\label{sein-helfen-der-Kinder}
\zl
Der Akkusativ in (\mex{0}a) ist strukturell und kann bei Passivierung als Nominativ
und nach einer Nominalisierung als Genitiv realisiert werden.
Der Dativ in (\ref{bsp-er-hilft-den-kindern}) kann dagegen in Nominalisierungen nicht zum Genitiv werden,
sondern nur in einer komplexen Nominalisierung pränominal realisiert werden:
% Genauso Genitiv:
% Opfergedenken mit armenischer Hymne, taz, 26.04.2004, S.\,10
\ea
das Den-Kindern-Helfen
\z
Die Genitiv"=NP \emph{der Kinder} in (\ref{das-helfen-der-Kinder}) bezieht sich auf das Agens
von \emph{helfen}. Das Agens hat strukturellen Kasus und kann deshalb auch als Genitiv in
nominalen Umgebungen auf"|treten. Wenn das Subjekt wie in (\ref{sein-helfen-der-Kinder})
durch ein Possessivum ausgedrückt wird, wird die Phrase ungrammatisch. Die Verhältnisse in
(\mex{-1}) sind erklärt, wenn man annimmt, daß in Nominalisierungen nur Elemente mit strukturellem
Kasus realisiert werden können und daß der Dativ ein lexikalischer Kasus ist.
Die Daten in (\mex{-1})
werden von denjenigen, die annehmen, daß der Dativ ein struktureller Kasus ist, oft
nicht besprochen.

Ein weiteres Problem, das sich ergibt, wenn man den Dativ zu den strukturellen Kasus
zählt, ist, daß man die Dativobjekte zweistelliger Verben\NOTE{JB: Was sind zweistellige Verben? Versteht diesen Abschnitt nicht.}
nicht von Akkusativobjekten
unterscheiden kann, wenn man als einzige Information über den Kasus eines Objektes
die Information hat, daß es strukturellen Kasus hat. Man betrachte \zb \emph{helfen} und
\emph{unterstützen} in (\mex{1}).
\eal
\ex Er hilft ihm.
\ex Er unterstützt ihn.
\zl
Beide Verben haben ein Subjekt mit strukturellem Kasus. Wären nun sowohl der Dativ als auch der
Akkusativ strukturelle Kasus, würden sich die Valenzanforderungen der beiden Verben nicht
unterscheiden, was nicht adäquat ist.\footnote{
  \citet{Gunkel2003b} führt zwei verschiedene strukturelle Kasus ein, so daß man die Verben in
  (\mex{0}) wieder unterscheiden kann. Zu einer Diskussion seines Ansatzes siehe
  Abschnitt~\ref{sec-alternativen-struc-dativ}.%
}

Bei ditransitiven Verben kann man sagen, daß das erste Argument Nominativ, das zweite
%% \NOTE{Iwanowa:
%%   Die Diskussion um lexikalischen und strukturellen Kasus, und die kurzen Argumentationen der
%%   einzelnen Verfechter sind sehr verwirrend und tragen wenig zum Verständnis der Kasus bei.} 
Akkusativ und das dritte Dativ bekommt, aber bei zweistelligen Verben kann man nicht aus allgemeinen
Prinzipien ableiten, wann Akkusativ und wann Dativ auf"|treten muß.
Von Stechow und Sternefeld \citeyearpar[\page 435]{SS88a} und
\citet[\page 187]{Stechow90a} und Autoren, die die Unterscheidung zwischen
strukturellem und lexikalischem Kasus aus semantischer Sicht\is{Linking} machen,\footnote{
  \citealp[\page 12]{Kaufmann95a}; \citealp[\page 21--26]{Stiebels96a}; \citealp[\page 313]{Olsen97c};
  \citealp[\page 57, S.\,129]{Rapp97a}. Siehe jedoch \citew[\page 51]{Wunderlich97a}.%
% Wunderlich wird immer in Manuscriptfassung zitiert. In der endgültigen Version ist es dann anders.
}
nehmen deshalb an,
daß der Dativ zweistelliger Verben ein lexikalischer Kasus ist.
%(\citealp[\page 22]{Stiebels96a}\ia{Stiebels})
Das sagt voraus, daß das Dativpassiv mit solchen Verben nicht möglich ist.
Es ist richtig, daß das Dativpassiv mit zweistelligen Verben selten ist \citep{HW95a},
aber es ist entgegen den Vorhersagen nicht ausgeschlossen.  \citet[\page 75]{Wegener90} erklärt die
Seltenheit mit der niedrigen Frequenz zweistelliger unakkusativer\is{Verb!unakkusativisches}\footnote{ 
  Zur Unterscheidung zwischen unergativen und unakkusativischen Verben 
  siehe Kapitel~\ref{sec-unakkusativitaet}. Die unergativen Verben erlauben kein Passiv
  oder wenn, dann nur unter sehr eingeschränkten Bedingungen.%
}
Verben mit Dativobjekt.
Wegener (\citeyear[\page 134]{Wegener85a}; \citeyear[\page 75]{Wegener90}\ia{Wegener}) 
gibt die Beispiele in (\mex{1}) an:\footnote{
        Siehe auch \citew[\page 161--162]{Fanselow87a} und \citew[\page 143]{Eisenberg94a}.
}
\eal
\label{ex-er-kriegte-geholfen}
\ex\iw{helfen}\iw{gratulieren}\iw{applaudieren}
Er kriegte von vielen geholfen / gratuliert / applaudiert.
\ex\iw{danken} 
Man kriegt täglich gedankt.
\zl

\noindent
Die Beispiele in (\mex{1}) sind Korpusbelege:
\eal
\ex\label{ex-kriegen-fr} "`Da kriege ich geholfen."'\footnote{
Frankfurter Rundschau, 26.06.1998, S.\,7.%
}
\ex\label{bsp-bekommen-feldpost}
% auch nach applaudiert geholfen + bekommen und kriegen gesucht 21.09.2003
Heute morgen bekam ich sogar schon gratuliert.\footnote{%
Brief von Irene G.\ an Ernst G.\ vom 10.04.1943, Feldpost-Archive mkb-fp-0270.}
%Branich IG-Vorsitzender Friedel Schönel meinte deshalb, 
\ex\label{ex-klaerle} 
"`Klärle"' hätte es wirklich mehr als verdient, auch mal zu einem "`unrunden"' Geburtstag gratuliert zu bekommen.\footnote{
Mannheimer Morgen, 28.07.1999, Lokales; "`Klärle"' feiert heute Geburtstag.%
}
\ex\label{bsp-elvis}
Mit dem alten Titel von Elvis Presley "`I can't help falling in love"' bekam Kassier Markus Reiß zum Geburtstag gratuliert, [\ldots]\footnote{
%der dann noch viel später bekannte: "Ich hab' immer noch Gänsehaut, das war der schönste Teil meines Geburtstages." Doch auch die anderen Abteilungen des Bürgervereins können auf ein erfolgreiches Jahr 1998 zurückblicken.
Mannheimer Morgen, 21.04.1999, Lokales; Motor des gesellschaftlichen Lebens.%
}
\zl
Die Beispiele (\ref{ex-kriegen-fr}), (\ref{ex-klaerle}) und (\ref{bsp-elvis}) 
sind aus folgendem Grund besonders interessant:\label{frequenz-von-korpusbelegen}
\citet{HW95a} stellen fest, daß es im Korpus\is{Korpus} des Instituts für Deutsche Sprache in Mannheim keine
Belege für das Dativpassiv mit zweistelligen Verben gibt, stufen dieses Muster aber als grammatisch
möglich ein. Eine Sichtweise, die auch von Wegener, Fanselow und Eisenberg geteilt wird (siehe (\ref{ex-er-kriegte-geholfen})).
1999 finden sich dann plötzlich Sätze aus der Frankfurter Rundschau und dem Mannheimer Morgen im IDS"=Korpus, die genau dem Muster 
entsprechen. Man kann also aus der Nichtexistenz von Daten im Korpus nicht unbedingt etwas schließen. Man kann sich
Gedanken darüber machen, warum ein Muster relativ selten ist und vielleicht auch strukturelle
Gründe dafür finden, das Korpus aber als alleiniges Mittel zur Rechtfertigung negativer
Aussagen zu verwenden, ist gefährlich. Negative Aussagen sollten immer noch durch Experimente abgesichert
werden. (\ref{bsp-bekommen-feldpost}) zeigt übrigens, daß es das \emph{bekommen}"=Passiv mit \emph{gratulieren} 
schon 1943 gab. Hätte das IDS"=Korpus diese Daten 1995 enthalten, hätte man auch 1995 schon etwas finden können.

Nach der Besprechung der strukturellen Kasus Nominativ und Akkusativ haben wir bereits den Genitiv und den
Dativ von Verbargumenten bei den lexikalischen Kasus eingeordnet. Neben den bisher besprochenen
strukturellen Akkusativen gibt es aber auch lexikalische:
%\eal
\ea Ihn dürstet.\iw{dürsten}
%\ex Die Mutter lehrte ihren Sohn einen neuen Kanon.\label{bsp-die-mutter-lehrte}\iw{lehren}
\z
% Die Beurteilungen sind zu schwankend. Wunderlich 2002 nimmt OT-Ansatz.
%% In (\mex{0}) haben \emph{ihn} und \emph{ihren Sohn} nicht die Möglichkeit,
%% ihren Kasus (\zb durch Passivierung) zu ändern.\footnote{
%%   Merkwürdigerweise ist (i) für einige Sprecher akzeptabel:
%% \ea
%% Ihr Sohn bekam einen neuen Kanon gelehrt.
%% \z
%% Interessant hieran ist, daß es sich um ein Dativ"=Passiv handelt. Der Aktivsatz zu (i) muß
%% also (ii) sein und nicht (\mex{0}b).
%% \ea[?*]{
%% Jemand lehrte ihrem Sohn einen neuen Kanon.
%% }
%% \z
%% Für die Analyse von (i) müßte man wohl einen eigenen Lexikoneintrag für \emph{lehren} mit
%% Dativobjekt annehmen.
%% }
%% \emph{ein neuer Kanon} kann Subjekt in einem Passivsatz werden, \emph{ihren Sohn} nie.
In (\mex{0}) hat \emph{ihn} nicht die Möglichkeit, seinen Kasus (\zb durch Passivierung) zu ändern,
weshalb es sinnvoll ist, diese Akkusative zu den lexikalischen Akkusativen zu zählen.

\is{Pr"aposition|(}%
Haider argumentiert dafür, daß auch die Nominalphrasen
in Präpositionalobjekten vom Verb strukturellen Kasus zugewiesen bekommen
(bei Haider kommt hier nur der Akkusativ in Frage, da er den Dativ als lexikalischen Kasus behandelt).
Modifizierende Präpositionen sollen dagegen ihren nominalen Komplementen selbst Kasus zuweisen.
In Abhängigkeit von der jeweiligen Präposition ist der Kasus dann Genitiv, Dativ bzw.\
Akkusativ oder hängt -- wie in (\mex{1}) -- von der Bedeutung der Präposition ab.
\eal
\ex Er läuft in diesen Wald.
\ex Er läuft in diesem Wald.
\zl
Eine strukturelle Vergabe von Kasus an Präpositionalobjekte wäre in Haiders
System nur gerechtfertigt, wenn es keine Präpositionalobjekte mit Dativ"=Objekt gäbe.
\citet[\page 78]{Eisenberg94a} gibt aber die Beispiele in (\mex{1}), und (\mex{1}a) zeigt,
daß es durchaus Komplementpräpositionalphrasen gibt, deren NP nicht im Akkusativ steht.
\eal
\ex Sie hängt an\iw{hängen an} ihrer elektrischen Eisenbahn.
\ex Sie denkt an\iw{denken an} ihre Vergangenheit.
\zl
Nimmt man an, daß der Dativ auch ein struktureller Kasus ist, könnte man zwar
trotzdem davon ausgehen, daß Präpositionalobjekte strukturellen Kasus tragen, doch
entfällt hier die Motivation über das (Dativ-)Passiv, denn die Präpositionalobjekte
können im Deutschen nie zum Subjekt werden, und der Kasus der NP ändert sich in
Präpositionalobjekten auch nicht.

Ich behandle also Präpositionen einheitlich: Eine Präposition weist unabhängig von ihrer
Verwendung als Modifikator oder Komplement der von ihr selegierten Nominalphrase
lexikalischen Kasus zu.
%
\is{Pr"aposition|)}%

Genau wie der Kasus von NPen in Präpositionalphrasen sich nicht ändern kann, kann
sich der Kasus von Objekten von Adjektiven nicht ändern.
Adjektive\is{Adjektiv} können Genitiv und Dativ zuweisen:
\eal
\ex Er war sich dessen sicher.\iw{sicher}
\ex Sie ist ihm treu.\iw{treu}
\zl
Die Zuweisung von Akkusativ ist ebenfalls möglich, wie die folgenden Beispiele zeigen:
\eal
\ex Das ist diesen Preis nicht wert.
\ex Der Student ist das Leben im Wohnheim nicht gewohnt.\iw{gewohnt}\footnote{
        \citew*[\page 312]{HB72a}.
      }
\ex Du bist mir eine Erklärung schuldig.\footnote{
        \citew*[\page 620]{HFM81}.
      }
\zl
Der Akkusativ ist bei Komplementen von Adjektiven aber selten \citep*[\page 98, Fn.\,3]{Haider85b}.

Im Gegensatz zum Kasus der Objekte hängt der Kasus der Adjektivsubjekte von der syntaktischen
Umgebung ab:\footnote{
  Siehe auch \citew[\page 84]{Wunderlich84} zur Diskussion ähnlicher Beispiele.%
}
\eal
\ex Der Mond wurde kleiner.\iw{klein}
\ex Er sah\iw{sehen} den Mond kleiner werden.
\zl


\subsection{Semantische Kasus}
\label{sec-sem-kasus}
\is{Kasus!semantischer|(}

Nominalphrasen können sowohl als Komplement als auch als Adjunkt
auf"|treten. \citet*{Haider85b} gibt folgende 
Beispiele für adverbiale Nominalphrasen:\is{adverbialer Akkusativ}
\eal
\ex Sie hörten \emph{den ganzen Tag}\iw{Tag} dieselbe Schallplatte.\label{bsp-den-ganzen-tag-dieselbe-schallplatte}
\ex Laßt \emph{mir} den Hund in Ruhe!\label{bsp-lasst-mir}
\ex \emph{Eines Tages}\iw{Tag} erschien ein Fremder.
\zl
% Er kam eines Morgens und bleib den ganzen Tag. Wegener85b:108
Haider rechnet auch Dative, die auf denjenigen referieren, zu dessen Gunsten (\ref{bsp-er-goss-ihr})
bzw.\ Ungunsten (\mex{1}b) etwas geschieht (\textit{dativus commodi\/}\is{Dativ!commodi@{\it
    commodi\/}} und \textit{dativus incommodi\/}\is{Dativ!incommodi@\textit{incommodi\/}})\footnote{\label{fn-dativ-commodi}%
        Die Unterscheidung zwischen \textit{dativus commodi\/} und \textit{dativus incommodi\/}
        ist aus syntaktischer Sicht wenig sinnvoll, wie \citet*[\page 100]{Wegener85b}\iaf{Wegener}
        gezeigt hat.
        Ob es sich um einen \textit{dativus commodi\/} oder um einen \textit{dativus incommodi\/} handelt,
        hängt von außersprachlichen Faktoren ab.
        \eal
        \ex Der kleine Junge / der berühmte Maler bemalt ihr den Tisch.
        \ex Er öffnet ihr die Bluse.
        \zl
        Selbst (\ref{bsp-er-zuendet-ihr}) kann als \textit{dativus commodi\/} verstanden werden, wenn
        er ihr bei einem Versicherungsbetrug hilft.%
}, zu den adverbialen Nominalphrasen.
\eal
\ex Er goß \emph{ihr} die Blumen.\label{bsp-er-goss-ihr}
\ex Er zündete \emph{ihr} das Haus an.\label{bsp-er-zuendet-ihr}
\zl
\citet*{Wegener85b}\ia{Wegener} hat jedoch gezeigt, daß diese Dative Komplementstatus haben.
Nur der ethische Dativ\NOTE{FB: ethischen Dativ erklären} (\ref{bsp-lasst-mir})\is{Dativ!ethischer} und der Urteilsdativ (\textit{Dativ iudicantis}\is{Dativ!iudicantis@\textit{iudicantis}}) (\mex{1})
sind als Adjunkte und als wirklich "`freie Dative"'\is{Dativ!freier} einzuordnen.\footnote{
        Die Beispiele in (\mex{1}) sind von \citet*[\page 53]{Wegener85b}.
}
\eal
\ex Das ist \emph{mir} zu\iw{zu!Grad} schwer.
\ex Das ist \emph{dem Kind} zu langweilig / nicht interessant genug.\iw{genug!Grad}
\ex Du läufst \emph{der Oma} zu\iw{zu!Grad} schnell.
\ex Das Wasser ist \emph{dem Baby} warm genug.\iw{genug!Grad}
\zl

\noindent
Haider zeigt, daß die Annahme, daß in (\ref{bsp-den-ganzen-tag-dieselbe-schallplatte}) beide
Nominalphrasen vom Verb den Akkusativ zugewiesen bekommen,
nicht sinnvoll ist, da Zeitangaben wie \emph{den ganzen Tag} auch in adjektivischen und
nominalen Umgebungen vorkommen.
% zitiert Toman83
\eal
\ex die Ereignisse \emph{letzten Sommer}\iw{Sommer}
\ex der Flirt \emph{vorigen Dienstag}\iw{Dienstag}
\ex die \emph{diesen Sommer} sehr günstige Witterung
\ex die \emph{diesen Sommer} sehr teuren Urlaubsreisen
\zl
Da Elemente mit strukturellem Kasus in Nominalumgebungen
im Genitiv stehen müssen, kann es sich in (\mex{0}a,b)
nicht um die Zuweisung eines strukturellen Kasus handeln.
%
Die Kasus in (\mex{0}) werden nicht aufgrund ihres Vorkommens in einer bestimmten
syntaktischen Struktur zugewiesen, sondern sind vielmehr durch die Bedeutung des Nomens bestimmt.

%\citep*{ZMT85a} -> semantische Kasusmarkierung
Der freie Akkusativ kommt bei Maß"-angaben\is{Maßangaben} (Zeitdauer und Zeitpunkt)
vor (\mex{1}) und Genitiv bei Lokalangaben oder Zeitangaben (\mex{2}).
\eal
\ex Sie studierte \emph{den ganzen Abend}.\iw{Abend}
\ex \emph{Nächsten Monat}\iw{Monat} kommen wir.
\zl
\eal
\ex Ein Mann kam \emph{des Weges}.\iw{Weg}
\ex \emph{Eines Tages}\iw{Tag} sah ich sie wieder.
\zl

Interessanterweise können auch komplexere Nominalphrasen als Adjunkte verwendet werden:\footnote{
  Siehe hierzu auch \citew{Flickinger2008a}.
}
\ea
\label{bsp-den-groessten-Teil-der-Woche}
Er arbeitet \emph{den größten Teil der Woche} zu hause.
\z
Beispiele wie (\mex{0}) zeigen, dass das Nomen, das den Zeitausdruck beisteuert, nicht
notwendigerweise im Akkusativ stehen muss. Es kann wie in (\mex{0}) Teil einer größeren
Nominalphrase im Akkusativ sein, die dann als Ganzes als Modifikator verwendet werden kann.
% reicht nicht \enlargethispage{2\baselineskip}
%
\is{Kasus!semantischer|)}



%% \subsection{Kongruenzkasus}
%% \label{sec-kongruenzkasus}
%% \is{Kasus!Kongruenz-|(}

%% Interessant sind Verben wie \word{nennen}, die mit zwei Akkusativen
%% auf"|treten:\footnote{
%%         Die Beispiele in (\mex{1}) und (\mex{2}) sind von \citet[\page 548]{Flaeming81a}\iaf{Fläming}.
%% }
%% % Haider85:99 mit Verräter
%% \ea
%% Er nannte ihn einen Experten.
%% \z
%% Hier scheint ein Verb vorzuliegen, das wie \emph{lehren} mit zwei Akkusativen vorkommt.
%% Sieht man sich die Passivvariante des Satzes in (\mex{0}) an, stellt man jedoch fest,
%% daß es sich beim Kasus von \emph{einen Experten} nicht um einen normalen Akkusativ handeln
%% kann:
%% \ea
%% Er wurde ein Experte genannt.
%% \z
%% Lägen in (\mex{-1}) zwei normale Akkusative vor, würde sich nur der Kasus eines Akkusativs
%% bei der Passivierung ändern, es ändern sich aber beide Kasus. Man nennt den Kasus von \emph{einen Experten}
%% auch Kongruenzkasus. Die prädikative Phrase \emph{einen Experten} stimmt in ihrem Kasus mit dem
%% Element, über das prädiziert wird, überein. 

%% Ähnliche Effekte kann man in Sätzen mit den Präpositionen\is{Pr"aposition} \emph{als}\iw{als|(} und \emph{wie}\iw{wie|(}
%% beobachten.\footnote{
%%         Das Handwörterbuch der deutschen Gegenwartssprache \citep{Kempcke84}\ia{Kempcke} ordnet \emph{als}
%%         und \emph{wie} als nebenordnende Konjunktionen\is{Konjunktion} ein.
%% %Wunderlich84:73; Fanselow86:361, Fanselow91a:87 -> Präposition
%%         Zu einer Kritik an der Bezeichnung "`Präposition"' siehe auch 
%%         \citew[\page 173, Fußnote 4, S.\,204--205]{Heringer73a}.
%%         Heringer nennt diese Elemente Identifikationstranslative\is{Identifikationstranslativ}.
%% %
%% %        Oft wird behauptet, daß Präpositionen keinen Nominativ regieren können. (i)
%% %        scheint zu belegen, daß \emph{als} den Nominativ regieren kann, denn
%% %        in (ia) und (ic) gibt es kein Element, auf das sich \emph{künftiger Ministerpräsident}
%% %        bezieht. \emph{Man} scheidet aus semantischen Gründen aus.
%% %\fneenumsentence{
%% %\ex Der Mann, dem gute Chancen als künftiger Ministerpräsident eingeräumt werden, \ldots%\footnote{
%% %        (taz, 19./20.09.1998, S.\,3)
%% %%}
%% %% zieht dieser Tage über Wochenmärkte und verteilt rote Kärtchen mit dem Schriftzug:
%% %% "`Zeigt den Rechten die rote Karte. Besonders im Osten. SPD."'
%% %\ex der Mann, dem als künftigem Ministerpräsidenten gute Chancen eingeräumt werden
%% %\ex Man räumt ihm Chancen als künftiger Ministerpräsident ein.
%% %\ex Man räumt ihm als künftigem Ministerpräsidenten Chancen ein.
%% %}
%% }
%% \eal
%% \label{bsp-gelten-als}
%% \ex Er gilt als großer Künstler.\footnote{
%%         \citew[\page 203--204]{Heringer73a}.
%%       }
%% \ex Man läßt ihn als großen Künstler gelten\iw{gelten als}.
%% \zl
%% \eal
%% \label{bsp-ansehen-als}
%% \ex Ich sehe ihn als meinen Freund an.\iw{ansehen}\footnote{
%%         \citew*[\page 154]{SS88a}.
%% }
%% \ex Er wird als mein Freund angesehen.
%% \zl
%% In (\mex{-1}) und (\mex{0}) sind die Präpositionalphrasen direkte Komplemente der Verben \emph{gelten}
%% bzw.\ \emph{ansehen}.
%% Treten solche Verben als adjektivisches Partizip auf, so kongruiert die \emph{als}"=Phrase
%% mit dem nicht ausgedrückten Subjekt, das im Nominativ steht (siehe auch Abschnitt~\ref{sec-kasus-nicht-realisierter-subj}).
%% \ea
%% eine als Hebel wirkende\iw{wirken als} Stange\footnote{
%%         \citet[\page 212]{Heringer73a}\iafwrong{Heringer} erklärt diesen Nominativ mit der Tendenz, prädikative
%%         nominale Komplemente nicht mehr zu deklinieren. Diese Beobachtung ist im allgemeinen richtig
%%         (siehe lexikalischer Nominativ in Kopulakonstruktionen), doch für die diskutierte Verbklasse
%%         falsch.
%% }
%% \z
%% Der Kasus des modifizierten Nomens ist unabhängig vom Kasus des Subjekts innerhalb der Partizipialkonstruktion.

%% Während die \emph{wie}"=Präpositionalphrasen\iw{wie} in (\mex{1}a--b) als Argumente zu analysieren sind,
%% handelt es sich bei denen in (\mex{1}c--d) um modifizierende.\footnote{
%%         Die Beispiele (\ref{bsp-verhalten-behandeln-helfen-wie}) sind von \citet*[\page 75]{Wunderlich84}\iaf{Wunderlich}.
%% } 
%% \eal
%% \label{bsp-verhalten-behandeln-helfen-wie}
%% \ex Sie verhielt\iw{verhalten} sich wie ihr Vater.
%% \ex Ich behandelte\iw{behandeln} ihn wie meinen Bruder.
%% \ex Ich half\iw{helfen} ihm wie einem Freund.
%% \ex Ich erinnerte\iw{erinnern} mich dessen wie eines fernen Alptraums.
%% \zl
%% %% Die modifizierenden Präpositionalphrasen
%% %% verhalten sich syntaktisch genauso wie 
%% %% die {\em ein- nach d- ander-\/}-Ad\-ver\-bial\-phra\-sen:\iw{ein- nach d- ander-} 
%% %% Sie modifizieren ein Verb und stimmen mit einem Argument des Verbs in einem bestimmten Merkmal überein.
%% \iw{wie|)}

%% \begin{comment}
%% Interessanterweise scheint es auch Verben bzw.\ Funktionsverbgefüge\is{Funktionsverbgefüge}
%% zu geben, in denen der Kasus der \emph{als}-Phrase feststeht.
%% \end{comment}

%% %% Auch bei diesen Konstruktionen muß -- genau wie bei den Kopulakonstruktionen\is{Kopula} in 
%% %% (\ref{bsp-kopula-kein-agreement}) -- das Element in der Präpositionalphrase bzw.\ im zweiten Objekt
%% %% nicht in Genus und Numerus mit dem ersten Objekt übereinstimmen, wie das \zb 
%% %% von \citet[\page 154]{SS88a}\ia{von Stechow}\ia{Sternefeld} behauptet wird.
%% %% % auch Admoni66:114, Heringer73a:205,Fn2 Schmidt65:129  ??
%% %% \eal
%% %% \ex Er empfand\iw{empfinden} diese Anschuldigungen als große Beleidigung.
%% %% \ex Er nannte\iw{nennen} diese Behauptungen einen Schmarrn.
%% %% \ex Er nannte diese Frau ein Genie.
%% %% \zl%

%% \begin{comment}
%% Zum Lexikoneintrag in (\ref{le-nennen}) ist noch anzumerken, daß explizit angenommen wird, daß \emph{ihn}
%% und \emph{einen Lügner} Komplemente von \emph{nennen} sind. Das unterscheidet die Analyse
%% von solchen, die mitunter für englische Sätze wie (\mex{1}) angenommen werden.
%% \ea
%% He considered him a fool.
%% \z
%% So nimmt \zb \citet[\page 14]{Rothstein}\ia{Rotstein} an, daß \emph{him a fool} einen
%% sogenannten {\em Small Clause}\is{Small Clause@\textit{Small Clause}} bildet.
%% Würde man das aufs Deutsche übertragen, wäre nicht zu erklären, warum in
%% (\mex{1}) \emph{einen Lügner} zwischen Elementen des Satzes \emph{ihn einen Lügner}
%% stehen kann.
%% \ea
%% Er nannte ihn mit Nachdruck einen Lügner.
%% \z
%% Auch die Passivdaten wären nicht ohne weiteres erklärbar.
%% \end{comment}

%% Man könnte dieses Verhältnis gut erfassen, wenn man annehmen würde, daß prädikative
%% Phrasen immer mit dem Element, über das sie prädizieren, im Kasus übereinstimmen.
%% Dies würde sofort auch Beispiele wie das in (\mex{1}) erklären:
%% %% , wenn man davon ausgeht,
%% %% daß das nicht realisierte Subjekt von Infinitiven im Nominativ steht.
%% \ea
%% \label{bsp-lex-nom}
%% Er wird ein großer Linguist.\label{bsp-er-wird-ein-grosser}\footnote{
%% %        (\ref{bsp-er-wird-ein-grosser}) und (\ref{bsp-lass-ihn-einen-grossen}) sind von 
%% \citew[\page 205]{Heringer73a}.
%% %.
%%       }
%% %% \ex Er beschloß, ein Linguist zu werden.\footnote{
%% %%         \citep*[\page 216]{Oppenrieder91a}\iaf{Oppenrieder}
%% %%         }\label{bsp-er-beschloss}
%% % Er brauchte nicht einmal sechs Minuten, um gegen Trevor Berbick der jüngste Schwergewichtsweltmeister
%% % aller Zeiten zu werden. taz, 08./09.06.2002, taz-mag, S.\,III
%% \z
%% Leider ist die Sache mit den Kopulakonstruktionen\is{Kopula} nicht so einfach, denn wenn
%% die beiden Nominalphrasen immer im Kasus übereinstimmen müßten, dann würde
%% man erwarten, daß in AcI"=Konstruktionen (siehe Kapitel~\ref{sec-aci}) beide Nominalphrasen im Akkusativ stehen.
%% Wie die Beispiele in (\mex{1}) zeigen, ist dem nicht so:\footnote{
%%         Die Sätze in (i) entsprechen dem, was man bei Kongruenzkasus erwarten würde.
%%         Man muß diese Konstruktion aber zu den idiomatischen\is{Idiom} Wendungen zählen.
%%         \eal
%%         \ex[]{
%%         Er läßt den lieben Gott 'n frommen Mann sein.\iw{lassen}
%%         }
%%         \ex[*]{
%%         Er läßt den lieben Gott 'n frommer Mann sein.
%%         }
%%         \zl
%%         Der \citet*[{\S}\,1473]{Duden73}\iaf{Duden} bezeichnet solche Konstruktionen mit Akkusativ als veraltet.
%%         Der \citet*[{\S}\,1259]{Duden95} behauptet darüber hinaus, daß der Akkusativ in Sätzen wie (\mex{1})
%%         nach dem Muster von (i) in der Schweiz üblich ist. Für diese Varianten des Deutschen ist
%%         dann der Kasus des Prädikats als Kongruenzkasus zu behandeln.
%% %
%% % Adam P und Fanselow2000b:4
%% % Fanselow zitiert Yip et al: Es gibt keinen lexikalischen Nominativ
%% %
%% %        Folgt man der Analyse aus Abschnitt~\ref{sec-kongruenzkasus}, liegt in (\ref{bsp-lex-nom})
%% %        lexikalischer Nominativ vor.

%%         Zu erwähnen sind in diesem Zusammenhang auch noch Sätze wie (ii), die nach \citet[\page 174]{Plank85a}\iadata{Plank}
%%         im Niederdeutschen\il{Niederdeutsch} grammatisch sein sollen:
%%         \ea
%%         Der Kanzler ist einen starken Mann.
%%         \zlast
%% }
%% \eal
%% \label{bsp-aci-kopula}
%% %\ex Laß ihn einen großen Linguisten werden.\label{bsp-lass-ihn-einen-grossen}
%% \ex Laß\iw{lassen|(} den wüsten Kerl [\ldots] meinetwegen ihr Komplize sein.\footnote{
%%         (\ref{bsp-lass-den-wuesten-kerl}) und (\ref{bsp-lass-mich}) sind aus dem \citet*[{\S}\,6925]{Duden66}.\iaf{Duden} %\citet*[{\S}\,1473]{Duden73}.\iaf{Duden}
%%         Die Quellen finden sich dort.
%%       }\label{bsp-lass-den-wuesten-kerl}
%% \ex Laß mich dein treuer Herold sein.\label{bsp-lass-mich}
%% \ex Baby, laß\iw{lassen|)} mich dein Tanzpartner sein.\footnote{
%%         Funny van Dannen, Benno-Ohnesorg-Theater, Berlin, Volksbühne, 11.10.1995
%%         }
%% \zl
%% \NOTE{Quelle Funny van Dannen genau mit CD}
%% Für Kopulakonstruktionen muß man wohl wie \citet[\page 54]{Thiersch78a} davon ausgehen,
%% daß der Nominativ des Nicht"=Subjekts ein lexikalischer Kasus ist.
\is{Kasus!struktureller|)}%
\is{Kasus!lexikalischer|)}%
%\is{Kasus!Kongruenz-|)}
\ia{Haider|)}%


\subsection{Der Kasus nicht ausgedrückter Subjekte}
\label{sec-kasus-nicht-realisierter-subj}

\citet*[Kapitel~6]{Hoehle83a}
hat gezeigt, wie man den Kasus nicht an der Oberfläche
auf"|tretender Elemente bestimmen kann. Mit der Phrase {\em ein- nach d- ander-\/}\iw{ein- nach d- ander-|(uu} kann
man sich auf mehrzahlige Konstituenten beziehen. Dabei muß \emph{ein- nach d- ander-}
in Kasus und Genus mit der Bezugsphrase übereinstimmen.
(\mex{1}) zeigt einfache Sätze, in denen \emph{ein- nach d- ander-} sich auf Subjekte bzw.\
Objekte bezieht:\footnote{
        \fromto{\mex{1}}{\mex{3}} sind von Höhle.
}
\eal
\label{bsp-tueren-hoehle}
\ex Die Türen sind eine nach der anderen kaputtgegangen.
\ex Einer nach dem anderen haben wir die Burschen runtergeputzt.
\ex Einen nach dem anderen haben wir die Burschen runtergeputzt.
\ex Ich ließ die Burschen einen nach dem anderen einsteigen.
\ex Uns wurde einer nach der anderen der Stuhl vor die Tür gesetzt.
\zl
In (\mex{1}) bezieht sich \emph{ein- nach d- ander-} auf Dativ- bzw.\ Akkusativobjekte
eingebetteter Infinitive mit \emph{zu}:
\is{Inkoh"arenz}
\eal
\ex Er hat uns gedroht, die Burschen demnächst einen nach dem anderen wegzuschicken.
\ex Er hat angekündigt, uns dann einer nach der anderen den Stuhl vor die Tür zu setzen.
\ex Es ist nötig, die Fenster, sobald es geht, eins nach dem anderen auszutauschen.
\zl
In (\mex{1}) befindet sich \emph{ein- nach d- ander-} ebenfalls innerhalb der Infinitiv"=VP:
\eal
\label{bsp-nominativ-inkoh}
\ex Ich habe den Burschen geraten, im Abstand von wenigen Tagen einer nach dem anderen
      zu kündigen.\label{bsp-nominativ-inkoh-geraten}
\ex Die Türen sind viel zu wertvoll, um eine nach der anderen verheizt zu werden.
\ex Wir sind es leid, eine nach der anderen den Stuhl vor die Tür gesetzt zu kriegen.
\ex Es wäre fatal für die Sklavenjäger, unter Kannibalen zu fallen und einer nach dem
      anderen verspeist zu werden.
\zl
{\em ein- nach d- ander-\/} ist in keinem der Sätze in (\mex{0}) das Subjekt,
%\NOTE{FB: Vielleicht sagen, was es      sonst ist: Adjunkt}
da dieses in dieser
Form Infinitivkonstruktion nie realisiert wird. {\em Ein- nach d- ander-\/} bezieht sich
jedoch auf das Subjekt des \zuis. Daraus daß {\em ein- nach d- ander-\/} in (\mex{0}) im Nominativ
steht, kann man schließen, daß das nicht realisierte Subjekt ebenfalls im Nominativ
stehen muß.
\ia{Höhle|)}

Dasselbe gilt für nicht realisierte Subjekte in Umgebungen adjektivischer Partizipien.
\eal
\label{bsp-nominativ-adj}
\ex die eines nach dem anderen einschlafenden Kinder
\ex die einer nach dem anderen durchstartenden Halbstarken
\ex die eine nach der anderen loskichernden Frauen
\zl
Die Wörter \emph{Kinder}, \emph{Halbstarken} und \emph{Frauen} sind in (\mex{0}) keine syntaktischen
Argumente der Partizipien. Sie sind vielmehr der Kopf der Nominalphrase und werden von den Partizipien
modifiziert. Daß die Nomina nicht die Subjekte der Partizipien sein können, kann man daran
feststellen, daß sie in Kasus stehen können, die als Kasus von Subjekten im Deutschen nicht möglich sind:
\ea
Sie helfen den einer nach dem anderen durchstartenden Männern.
\z
In (\mex{0}) steht \emph{den \ldots{} Männern} im Dativ. Der Dativ wurde von uns zu den lexikalischen
Kasus gezählt, Subjekte haben aber immer strukturellen Kasus. Zu Subjektstests und Dativen im Deutschen
siehe auch Seite~\pageref{page-dativsubjekte}.
Das Subjekt der adjektivischen Partizipien ist also nicht als syntaktisches Argument des
Adjektivs ausgedrückt.

Man muß deshalb sicherstellen, daß auch nicht realisierte Subjekte Kasus zugewiesen bekommen.
Würde man diesen Kasus unterspezifiziert lassen, würden Sätze wie (\mex{1}) falsch analysiert werden.
\ea[\#]{
Ich habe den Burschen geraten, im Abstand von wenigen Tagen einen nach dem anderen zu kündigen.
}
\z
In der zulässigen Lesart von (\mex{0}) ist die Phrase 
\emph{einen nach dem anderen} das Objekt von \emph{kündigen} und kann
sich nicht auf das Subjekt des Infinitivs, das referenzidentisch
mit \emph{den Burschen} ist, beziehen.
\iw{ein- nach d- ander-|)uu}



\section{Die Analyse}
\is{Kasus!struktureller|(}\is{Kasus!lexikalischer|(}

In den folgenden beiden Abschnitten werden die Vergabe von Kasus an Argumente und 
die Restriktionen in bezug auf semantische Kasus erklärt. 
%% Auf die Analyse
%% der Kongruenzkasus wird in diesem Buch nicht eingegangen, da Kopulakonstruktionen
%% und Prädikationen nach dem Muster \emph{jemanden für etwas halten} nicht besprochen
%% werden.


\subsection{Kasus von Argumenten}

Ich gehe im folgenden davon aus, daß der Dativ immer ein lexikalischer Kasus ist. 
Ein ditransitives Verb wie \word{geben} hat dann einen \subcatw \sliste{ NP[\str], NP[\str], NP[\ldat] },
wobei \textit{str\/}\istype{str} für strukturellen Kasus und \type{ldat} für lexikalischen Dativ steht.
Die Zuweisung struktureller Kasus wird durch das folgende Prinzip geregelt (\citealp{Prze99}; 
\citealp{Meurers99b}; \citealp[Kapitel~10.4.1.4]{Meurers2000b}\ia{Meurers}; \citealp{MdK2001a}):\footnote{
  Eine korrekte Formalisierung wurde von Detmar Meurers
  vorgeschlagen (\citealp{Meurers99b}; \citealp{MdK2001a}; \citealp[Kapitel~10.4.1.4]{Meurers2000b}).
  Diese ist recht komplex und wird deshalb hier in einem Anhang diskutiert (siehe Abschnitt~\ref{kasus-anhang}).
  Zu \prz{}s Ansatz siehe Abschnitt~\ref{kasus-adamp}.%
}
\begin{prinzip-break}[Kasusprinzip]\is{Prinzip!Kasus-}
\label{case-p}
\begin{itemize}
\item In einer Liste, die sowohl das Subjekt als auch die Komplemente eines verbalen Kopfes
      enthält, bekommt das am wenigsten oblique Element mit strukturellem Kasus 
      Nominativ\is{Kasus!Nominativ}, es sei denn es wird von einem übergeordneten Kopf angehoben.
\item Alle anderen nicht angehobenen Elemente der Liste, die strukturellen Kasus tragen, bekommen Akkusativ\is{Kasus!Akkusativ}.
\item In nominalen Umgebungen wird Elementen mit strukturellem Kasus Genitiv\is{Kasus!Genitiv} zugewiesen.
\end{itemize}
\end{prinzip-break}
Die Einschränkung in bezug auf die Anhebung von Elementen dient zur korrekten Erfassung der
Kasuszuweisung in bestimmten Sätzen mit mehreren Verben (\zb AcI"=Konstruktionen, siehe
Kapitel~\ref{chap-anhebung} und S.\,\pageref{page-aci-kasus}).
%% Die Formulierung des Prinzips wurde wesentlich durch die Arbeiten
%% von Adam Przepi{\'o}rkowski und Detmar Meurers beeinflußt (\citealp{Prze99}; 
%% \citealp{Meurers99b}; \citealp[Kapitel~10.4.1.4]{Meurers2000b}\ia{Meurers}; \citealp{MdK2001a}).
%% Die Unterschiede zu den von diesen Autoren vorgeschlagenen Prinzipien werden im
%% Abschnitt~\ref{sec-kasus-alternativen} diskutiert.

Das Wirken des Kasusprinzips soll an den Beispielen in (\mex{1}) erklärt werden. Die \subcatlen der
jeweiligen Verben sind entsprechend der Obliqueness der Argumente geordnet (siehe
Seite~\pageref{page-obliquen-h}), \dash, das Subjekt steht an erster Stelle. Wenn es ein
Akkusativobjekt gibt, steht es an zweiter Stelle. Gibt es zusätzlich zum Akkusativobjekt ein
Dativobjekt folgt dieses dem Akkusativobjekt:
\ea
\begin{tabular}[t]{@{}l@{~}l@{~}l}
a. & \emph{schläft}:     & \subcat \sliste{ NP[\type{str}]$_j$ }\\[2mm]
b. & \emph{unterstützt}: & \subcat \sliste{ NP[\type{str}]$_j$, NP[\type{str}]$_k$ }\\[2mm]
c. & \emph{hilft}:       & \subcat \sliste{ NP[\type{str}]$_j$, NP[\type{ldat}]$_k$ }\\[2mm]
d. & \emph{schenkt}:     & \subcat \sliste{ NP[\type{str}]$_j$, NP[\type{str}]$_k$, NP[\type{ldat}]$_l$ }\\
\end{tabular}
\z
Die \subcatl von \emph{schläft} enthält genau ein Element: eine NP mit strukturellem Kasus.
Da diese NP das erste Element der \subcatl ist, bekommt sie Nominativ. Genauso bekommt die erste NP
in der \subcatl von \emph{unterstützt} Nominativ, die zweite bekommt allerdings Akkusativ. Bei
\emph{hilft} bekommt das erste Element Nominativ, das zweite ist lexikalisch als Dativ markiert. Bei
\emph{schenkt} bekommen die ersten beiden Elemente wieder jeweils Nominativ und Akkusativ, das dritte
Element ist wie das Objekt von \emph{hilft} lexikalisch als Dativ markiert.

Bei Passivierung der Verben in (\mex{0}) ergeben sich die folgenden \subcat"=Listen:
\ea
\begin{tabular}[t]{@{}l@{~}l@{~}l}
a. & \emph{geschlafen wird}:  & \subcat \sliste{ }\\[2mm]
b. & \emph{unterstützt wird}: & \subcat \sliste{ NP[\type{str}]$_k$ }\\[2mm]
c. & \emph{geholfen wird}:    & \subcat \sliste{ NP[\type{ldat}]$_k$ }\\[2mm]
d. & \emph{geschenkt wird}:   & \subcat \sliste{ NP[\type{str}]$_k$, NP[\type{ldat}]$_l$ }\\
\end{tabular}
\z
In (\mex{0}) steht jetzt eine andere NP an erster Stelle. Wenn diese NP strukturellen Kasus hat,
bekommt sie Nominativ, wenn das wie in (\mex{0}c) nicht der Fall ist, bleibt der Kasus wie er
ist, nämlich lexikalisch spezifiziert.

Für das Dativpassiv muß man ein bißchen zaubern: Bei der Kombination von \emph{geholfen} und
\emph{bekommen} bzw.\ von \emph{geschenkt} und \emph{bekommen} wird das Dativargument von 
\emph{geholfen} bzw.\ von \emph{geschenkt} zum ersten Argument gemacht und der lexikalische
Dativ beim eingebetteten Verb wird zu einem strukturellen Kasus bei der Kombination mit dem Passiv"=Hilfsverb:
\ea
\begin{tabular}[t]{@{}l@{~}l@{~}l}
a. & \emph{geholfen bekommt}:    & \subcat \sliste{ NP[\type{str}]$_k$ }\\[2mm]
b. & \emph{geschenkt bekommt}:   & \subcat \sliste{ NP[\type{str}]$_l$, NP[\type{str}]$_k$ }\\
\end{tabular}
\z
Wie diese Umordnung genau passiert, wird im Kapitel~\ref{sec:dat-pass} ausführlicher erklärt. Hier ist
nur wichtig, wie die Kasusvergabe funktioniert: Dadurch daß das Dativargument bei Kombination mit
dem Passivhilfsverb strukturellen Kasus hat und an erster Stelle in der Valenzliste von
\emph{geholfen bekommen} bzw.\ von \emph{geschenkt bekommen} steht, kriegt es Nominativ. Bei
\emph{geschenkt bekommen} bekommt das zweite Element (das direkte Objekt) Akkusativ. 

Die Umwandlung eines lexikalischen in einen strukturellen Kasus ist unschön,
aber es scheint zur Zeit keine bessere Alternative zu geben. Einen Vorschlag diskutiere ich in
Abschnitt~\ref{sec-alternativen-struc-dativ}.


Für die Erklärung der Kasusvergabe in AcI"=Konstruktionen\label{page-aci-kasus}\is{Verb!AcI-} müssen wir etwas vorgreifen:
Bei der Analyse der AcI"=Konstruktion findet eine Argumentkomposition statt, \dash, die Argumente
des eingebetteten Verbs werden zu Argumenten des AcI"=Verbs (\zb \emph{sehen}, \emph{lassen}).
Man sagt, daß Verben wie \emph{sehen} und \emph{lassen} die Argumente der unter sie eingebetteten
Verben anheben. Solche Verben werden deshalb auch Anhebungsverben\is{Verb!Anhebungs-} (\emph{raising verbs}) genannt.
Entsprechende Valenzlisten sind in (\mex{1}) zu sehen:
\ea
\begin{tabular}[t]{@{}l@{~}l@{~}l}
a. & \emph{schlafen läßt}:     & \subcat \sliste{ NP[\str]$_i$, NP[\type{str}]$_j$ }\\[2mm]
b. & \emph{unterstützen läßt}: & \subcat \sliste{ NP[\str]$_i$, NP[\type{str}]$_j$, NP[\type{str}]$_k$ }\\[2mm]
c. & \emph{helfen läßt}:       & \subcat \sliste{ NP[\str]$_i$, NP[\type{str}]$_j$, NP[\type{ldat}]$_k$ }\\[2mm]
d. & \emph{schenken läßt}:     & \subcat \sliste{ NP[\str]$_i$, NP[\type{str}]$_j$, NP[\type{str}]$_k$, NP[\type{ldat}]$_l$ }\\
\end{tabular}
\z
NP[\str]$_i$ steht hierbei jeweils für das Subjekt des AcI-Verbs. 
NP[\type{str}]$_j$, NP[\type{str}]$_k$ bzw.\ NP[\type{ldat}]$_l$ sind die Argumente des eingebetteten
Verbs. Für die Kasusvergabe sind nur die Valenzlisten in (\mex{0}) maßgeblich. Die Argumente in den Valenzlisten der
eigentlichen Verben spielen für die Kasusvergabe keine Rolle, da das Kasusprinzip die Kasuszuweisung ausschließt,
wenn ein Element angehoben wird. Damit sind nur die Verhältnisse in (\mex{0}) für die Kasuszuweisung interessant,
das erste Element in den Listen in (\mex{0}) bekommt immer Nominativ, die restlichen Elemente mit strukturellem
Kasus bekommen Akkusativ. Die logischen Subjekte der eingebetteten Verben werden also im Akkusativ realisiert.

Die Kasuszuweisung an das Subjekt von Adjektiven\is{Adjektiv} funktioniert analog. Die Kopula\is{Kopula} wird mit dem Adjektiv
verbunden, und es entsteht eine Valenzliste, die die Argumente des Adjektivs enthält (\mex{1}a). Wird
ein solcher Komplex noch unter ein AcI"=Verb wie \emph{sehen} eingebettet, erhält man die Liste in (\mex{1}b):
\ea
\begin{tabular}[t]{@{}l@{~}l@{~}l}
a. & \emph{kleiner werden}:     & \subcat \sliste{ NP[\str]$_j$ }\\[2mm]
b. & \emph{kleiner werden sah}: & \subcat \sliste{ NP[\str]$_i$, NP[\type{str}]$_j$ }\\
\end{tabular}
\z
Die Kasuszuweisung funktioniert analog zu den bereits diskutierten Fällen. In den verbalen Umgebungen
der Kopula bzw.\ des AcI"=Verbs bekommen die NPen mit strukturellem Kasus Nominativ bzw.\ Akkusativ.%
\is{Kasus!struktureller|)}\is{Kasus!lexikalischer|)}


\subsection{Semantischer Kasus}

\is{Kasus!semantischer|(}%
Der Kasus von NPen wie \emph{den ganzen Tag} in (\mex{1}) ist von der syntaktischen Umgebung unabhängig.
\eal
\ex Sie arbeiten den ganzen Tag.
\ex Den ganzen Tag wird gearbeitet, [\ldots].\footnote{
  \url{http://www.philo-forum.de/philoforum/viewtopic.html?p=146060}. \urlchecked{12}{05}{2005}.
}
\zl
Daß die NP im Akkusativ steht, hängt mit ihrer Funktion zusammen. Die NP \emph{den ganzen Tag} in
der Verwendung in (\mex{0}) unterscheidet sich von den NPen mit dem Kopf \emph{Tag}, wie
man sie für die Analyse der Sätze in (\mex{1}) braucht:
\eal
\ex Ich liebe diesen Tag.
\ex Dieser Tag gefällt mir.
\zl
In (\mex{0}) liegen ganz gewöhnliche Argumente vor, in (\mex{-1}) dagegen ein Adjunkt. Adjunkte
unterscheiden sich von Argumenten durch ihren \modw: Bei Argumenten ist der \modw \type{none},
bei Adjunkten ist der \modw eine Merkmalstruktur vom Typ \type{synsem}. Man kann jedoch nicht
einfach annehmen, daß es für Nomina wie \emph{Tag} neben dem Argument"=Lexikoneintrag noch einen für
die Funktion als Adjunkt gibt, denn wie (\ref{bsp-den-groessten-Teil-der-Woche}) -- hier als
(\mex{1}) wiederholt -- zeigt, kann der Zeitausdruck unter ein anderes Nomen eingebettet sein.
\ea
\label{bsp-den-groessten-Teil-der-Woche-zwei}
Er arbeitet \emph{den größten Teil der Woche} zu hause.
\z
In (\mex{0}) modifiziert nicht \emph{Woche} bzw.\ \emph{der Woche} das Verb, sondern nur die gesamte
Nominalphrase \emph{den größten Teil der Woche}. 

Statt verschiedene Lexikoneinträge für Zeitausdrücke anzunehmen, verwende ich deshalb ein
Dominanzschema, das eine Akkusativ"=NP, die einen Zeitabschnitt beschreibt, zu einer NP macht, die
einen verbalen Ausdruck modifizieren kann, \dash zu einer NP mit entsprechendem \modw.\footnote{
  Ein solches Schema ist nicht außergewöhnlich. Semantiker verwenden oft solche Regeln für
  sogenanntes \mbox{Type}"=Shifting \citep{Partee86b-u}. Zum Beispiel werden referentielle Nominalgruppen
  wie in (i) so zu prädikativen Nominalphrasen wie in (ii):
  \eal
  \ex Ein Lehrer lacht.
  \ex Er ist ein Lehrer.
  \zl
  Zu einer HPSG"=Analyse siehe \citew{MuellerPredication,MuellerCopula}.
} Da das Schema
nur auf NPen im Akkusativ angewendet wird, ist sichergestellt, daß Sätze wie (\mex{1}) nicht
analysiert werden: 
\eal 
\ex[*]{ 
Er arbeitet der ganze Tag.  
}
\ex[*]{
weil der ganze Tag gearbeitet wurde
}
\zl
%%
%% \eas
%% semantischer Kasus für Akkusative der Zeit:\\
%% \onems{
%% phon  \phonliste{ Tag }\\
%% synsem$|$loc \onems{ cat  \ms{ head & \ms[noun]{
%%                                       case & acc\\
%%                                       mod  & \onems{ loc \ms{ cat  & \ms{ head & verbal\\
%%                                                                         }\\
%%                                                               cont & \ibox{1}\\
%%                                                             }\\
%%                                                    }\\
%%                                       }\\
%%                                subcat & \sliste{ Det }\\
%%                              }\\
%%                     cont \ms[and]{
%%                            arg1 & \ibox{1}\\
%%                            arg2 & \textrm{duration(\ibox{1},\ibox{2})}\\
%%                            }\\
%%                     context$|$inds \liste{ \ms{ ind & \ibox{2} \ms{ per & 3\\
%%                                                                       num & sg\\
%%                                                                       gen & mas\\
%%                                                                     } \\
%%                                                   restr & \liste{ \ms[tag]{
%%                                                                   inst & \ibox{2} \\
%%                                                                   } }\\
%%                                                 }  }\\
%%                   }\\
%% } 
%% \zs
\is{Kasus!semantischer|)}
%\section{Kongruenzkasus}




\section{Alternativen}
\label{sec-kasus-alternativen}

In den Abschnitten~\ref{sec-alternativen-struc-dativ}--\ref{kasus-adamp} diskutiere ich verschiedene HPSG"=Ansätze,
und Abschnitt~\ref{sec-kasus-fin-verb-nullkasus} beschäftigt sich mit der Kasuszuweisung in der GB"=Theorie.

\subsection{Struktureller Dativ}
\label{sec-alternativen-struc-dativ}

\citet[\page 76]{Gunkel2003b} kritisiert andere HPSG"=Analysen, die Vorschlägen von Haider folgend
den Dativ als lexikalischen Kasus behandelt haben. Er entwickelt eine Typhierarchie,
in die er zwei spezielle Typen \type{scase1} und \type{scase2} aufnimmt (S.\,96). Dabei steht \type{scase1}
für die strukturellen Kasus, die in verbalen Umgebungen als Nominativ oder Akkusativ realisiert
werden können, und \type{scase2} steht für die Kasus, die in verbalen Umgebungen als Nominativ (im Dativpassiv)
oder Dativ realisiert werden können. Auf die Kasusvergabe in nominalen Umgebungen geht Gunkel
nicht ein. Integriert man den Genitiv in die Typhierarchie, bekommt man etwas
wie Abbildung~\vref{abb-case}. Statt Gunkels \type{scase1} und \type{scase2} habe ich
die etwas aussagekräftigeren Bezeichnungen \type{structural\_nga} und \type{structural\_nd}
gewählt.\istype{snom}\istype{sgen}\istype{sdat}\istype{sacc}\istype{lnom}\istype{lgen}\istype{ldat}\istype{lacc}
\begin{figure}[htbp]
\centerline{\begin{pspicture}(-0.2,1)(10.4,7.2)
%\psgrid
\rput[B](5,7){\rnode{case}{\textit{case}}}
%
\rput[B](2.5,5){\rnode{structural}{\textit{structural}}}
%
\rput[B](1,3){\rnode{structuralnga}{\textit{structural\_nga}}}
\rput[B](4,3){\rnode{structuralnd}{\textit{structural\_nd}}}
\rput[B](6,3){\rnode{nom}{\textit{nom}}}
\rput[B](7,3){\rnode{gen}{\textit{gen}}}
\rput[B](8,3){\rnode{dat}{\textit{dat}}}
\rput[B](9,3){\rnode{acc}{\textit{acc}}}
%\rput[B](11,3){\rnode{lexical}{\textit{lexical}}}
%
\rput[B](2,1){\rnode{snom}{\textit{snom}}}
\rput[B](3,1){\rnode{sgen}{\textit{sgen}}}
\rput[B](4,1){\rnode{sdat}{\textit{sdat}}}
\rput[B](5,1){\rnode{sacc}{\textit{sacc}}}
%
\rput[B](7,1){\rnode{lnom}{\textit{lnom}}}
\rput[B](8,1){\rnode{lgen}{\textit{lgen}}}
\rput[B](9,1){\rnode{ldat}{\textit{ldat}}}
\rput[B](10,1){\rnode{lacc}{\textit{lacc}}}
%
\psset{angleA=-90,angleB=90,arm=0pt}
%
\ncdiag{case}{structural}\ncdiag{case}{lexical}
\ncdiag{structural}{structuralnga}\ncdiag{structural}{structuralnd}
\ncdiag{case}{nom}\ncdiag{case}{acc}\ncdiag{case}{gen}\ncdiag{case}{dat}
\ncdiag{structuralnga}{snom}\ncdiag{structuralnga}{sacc}\ncdiag{structuralnga}{sgen}
\ncdiag{structuralnd}{snom}\ncdiag{structuralnd}{sdat}
\ncdiag{nom}{snom}
\ncdiag{gen}{sgen}
\ncdiag{dat}{sdat}
\ncdiag{acc}{sacc}
%\ncdiag{lexical}{lnom}\ncdiag{lexical}{lgen}\ncdiag{lexical}{ldat}\ncdiag{lexical}{lacc}
\ncdiag{nom}{lnom}
\ncdiag{gen}{lgen}
\ncdiag{dat}{ldat}
\ncdiag{acc}{lacc}
%
\end{pspicture}}
\caption{\label{abb-case}Typhierarchie für die Subtypen von \type{case} nach \citew{Gunkel2003b} erweitert um \type{sgen}}
\end{figure}
Wie das Nominalisierungsbeispiel in (\ref{sein-helfen-der-Kinder}) -- hier als (\mex{1}) wiederholt -- zeigt,
können Dativargumente nie im Genitiv realisiert werden. 
\ea[*]{
sein Helfen der Kinder
}
\z
In Abbildung~\ref{abb-case} 
hat der Typ \type{structural\_nd} deshalb nur die Untertypen \type{snom} und \type{sdat},
nicht aber \type{sgen}. \type{structural\_nga} hat dagegen \type{snom}, \type{sgen} und \type{sacc}
als Untertypen.

Die Valenzrepräsentationen für einige typische Verben sind in (\mex{1}) zu sehen:
\ea
\begin{tabular}[t]{@{}l@{~}l@{~}l@{}}
a. & \emph{schlafen}:     & \subcat \sliste{ NP[\type{str\_nga}] }\\
b. & \emph{unterstützen}: & \subcat \sliste{ NP[\type{str\_nga}], NP[\type{str\_nga}] }\\
c. & \emph{helfen}:       & \subcat \sliste{ NP[\type{str\_nga}], NP[\type{str\_nd}] }\\
d. & \emph{schenken}:     & \subcat \sliste{ NP[\type{str\_nga}], NP[\type{str\_nga}], NP[\type{str\_nd}] }\\
\end{tabular}
\z
Gunkels Kasusprinzip (Kapitel~2.3.12) sagt dann, daß in verbalen Umgebungen das erste Argument Nominativ bekommt
und alle anderen Argumente nicht Nominativ ($\neg nom$) sind. Wenn man den Genitiv in die Hierarchie aufnimmt, bleiben
für den Kasus des Objekts von \emph{unterstützen} dann immer noch \type{sgen} und \type{sacc}. Man müßte
also bei der Verwendung der Typhierarchie in Abbildung~\ref{abb-case} auch den Genitiv ausschließen,
das kann man wie folgt tun: Man sagt, daß alle Argumente, die in verbalen Umgebungen nicht an erster Stelle
stehen, weder Nominativ noch Genitiv sein dürfen ($\neg nom \wedge \neg gen$). Das ist aber äquivalent
mit $dat \vee acc$ bzw.\ mit einem in die obige Hierarchie integrierten Typ
\type{structural\_da}. Macht man sich klar, daß ein Typ in einer Hierarchie für eine Disjunktion 
all seiner maximal spezifischen Untertypen steht, so wird auch klar, daß die Kodierung in
Abbildung~\ref{abb-case} nichts anderes ist als eine disjunktive Aufzählung von Möglichkeiten.
\type{structural\_nga} steht für $snom \vee sgen \vee sacc$. Damit stehen die Lexikoneinträge in (\mex{0})
aber für folgendes:
\ea
\begin{tabular}[t]{@{}l@{~}l@{~}l@{}}
a. & \emph{schlafen}:     & \subcat \sliste{ NP[$snom \vee sacc \vee sgen$] }\\
b. & \emph{unterstützen}: & \subcat \sliste{ NP[$snom \vee sacc \vee sgen$], NP[$snom \vee sacc \vee sgen$] }\\
c. & \emph{helfen}:       & \subcat \sliste{ NP[$snom \vee sacc \vee sgen$], NP[$snom \vee sdat$] }\\
d. & \emph{schenken}:     & \subcat $\langle$ \begin{tabular}[t]{@{}l@{}}
                                              \textsc{NP[$snom \vee sacc \vee sgen$], NP[$snom \vee sacc \vee sgen$],}\\
                                              \textsc{NP[$snom \vee sdat$]} $\rangle$\\
                                              \end{tabular}
\end{tabular}
\z
Damit die Analyse funktioniert, muß man also in Lexikoneinträgen Möglichkeiten disjunktiv spezifizieren (bzw.\
entsprechend unterspezifizieren) und im Kasusprinzip auch, \dash, man zählt entweder
alle Möglichkeiten einfach auf oder kodiert sie in Typen. 
% Wieso eigentlich??
%Die Abbildung~\ref{abb-case} müßte noch um einen Typ für $dat \vee acc$ erweitert werden. 
Ich ziehe das sehr viel einfachere\footnote{
  Die formale Umsetzung ist nicht einfach. Das liegt aber an der Einschränkung in bezug auf Anhebung.
  Eine solche Einschränkung braucht Gunkel auch.%
}, im vorigen Abschnitt formulierte Prinzip
vor. Der Preis, den man dafür zahlen muß, ist ein spezieller
Lexikoneintrag für die Hilfsverben, die beim Dativpassiv\is{Passiv!Dativ-} vorkommen. In diesem Lexikoneintrag
wird der lexikalische Dativ zum strukturellen Nominativ gemacht. Wie das genau funktioniert,
wird im Kapitel~\ref{chap-passiv} erklärt.

Ein weiterer Nachteil des Gunkelschen Ansatzes ist, daß nicht ohne weiteres erklärt werden kann,
wieso Partizipien wie \emph{geholfen} nicht wie in (\mex{1}) pränominal verwendet werden können.
% Auch Abraham2005:236
%   das dem Mann von der Frau gezeigte Etwas
% * der Etwas von der Frau gezeigte Mann
\ea[*]{
der geholfene Mann
}
\z
Mit der Passivanalyse, die in Kapitel~\ref{chap-passiv} vorgestellt wird, würde man die
folgende Valenzliste für \emph{geholfen} bekommen:
\ea
\emph{geholfen}: \subcat \sliste{ NP[\type{str\_nd}] }
\z
An der ersten Stelle der Valenzliste stünde ein struktureller Kasus. Die Lexikonregel für die
Ableitung von adjektivischen Partizipien in (\ref{lr-adjective-formation-da-approach}) 
auf Seite~\pageref{lr-adjective-formation-da-approach} ist für den im vorigen Abschnitt vorgestellten
Ansatz sehr einfach: Sie besagt, daß eine Adjektivform abgeleitet werden kann, wenn das
erste Element der \subcatl des Partizips strukturellen Kasus hat. Dieses Element mit strukturellem
Kasus wird dann zum Subjekt des Adjektivs, \dash zu dem Element, über das prädiziert wird.
Das entspricht dem modifizierten Nomen. Das Objekt von \emph{unterstützt} wird somit zu dem
Element, das dem modifizierten Nomen entspricht. Da der Kasus des Objekts von \emph{helfen}
aber lexikalisch ist, wird ein Wort wie \noword{geholfene} überhaupt nicht lizenziert. Mit einer
\subcatl wie in (\mex{0}) müßte man dagegen in der Lexikonregel für die Adjektivderivation
noch festlegen, daß das Objekt des Verbs, das dann zum Subjekt des Adjektivs wird,
kein Dativobjekt sein darf. Dies ist jedoch nicht ohne weiteres möglich, denn \type{str\_nd}
steht für $\nom \vee dat$, und wenn man so etwas wie $\neg dat$ sagt, bleibt immer noch \type{nom}
übrig. Man kann auch nicht verlangen, daß das Objekt nicht vom Typ \type{str\_nd} sein darf, denn
das würde den Nominativ ausschließen, und Subjekte von adjektivischen Partizipien haben strukturellen Nominativ,
wie die Beispiele in (\ref{bsp-nominativ-adj}) für adjektivische Partizipien gezeigt haben.
Was man zur Lösung dieses Problems braucht und was Gunkel auch vorschlägt, ist ein listenwertiges\footnote{
  Gunkel verwendet Mengen.%
}
Merkmal (\textsc{ia} = \emph{internal argument}), das eine Liste mit dem Element mit Akkusativobjekteigenschaften zum Wert hat. Die
Adjektivlexikonregel kann dann dieses Element zum Subjekt des Adjektivs machen. Bei unakkusativen
Verben ist das logische Subjekt in \textsc{ia}, bei unergativen Verben ist es das Akkusativobjekt \citep[\page 97]{Gunkel2003b}.
Verben, die nur ein Subjekt und ein Dativobjekt regieren, haben die leere Liste als Wert von \textsc{ia}.
Wie ich in Kapitel~\ref{chap-passiv} zeigen werde, braucht man ein Merkmal zur Auszeichnung des
Akkusativobjekts nicht, wenn man annimmt, daß der Dativ ein lexikalischer Kasus ist.

\subsection{Zuweisung in Abhängigkeit von phrasaler Konfiguration}

\mbox{}\citew[\page 209--210]{HM94a}, \citew[\page 280--281]{Mueller99a} und \citew[\page 112]{Gunkel2003b}
schlagen Kasusprinzipien vor, die Kasus innerhalb von Kopf"=Argumentstrukturen, \dash in bestimmten
phrasalen Konfigurationen, zuweisen. So sorgt \zb die folgende Implikation dafür,
daß wenn es an der ersten Stelle der \subcatl der Kopf"|tochter eine NP mit strukturellem
Kasus gibt, diese NP Nominativ bekommt:\footnote{
  Im 99er Buch gehe ich davon aus, daß die Argumente vom Ende der \subcatl vor denen am Anfang der
  Liste gesättigt werden. Das Kopf"=Argument"=Schema entspricht also dem Schema auf
  Seite~\pageref{schema-bin-prel} mit \emph{append}   und nicht dem Schema auf
  Seite~\pageref{schema-bin-prel2} mit \emph{delete}. Deshalb steht das Subjekt auch bei
  Verbprojektionen immer an erster Stelle der \subcatl. Es kann nicht passieren, daß das Subjekt
  gesättigt wird und dann eine Objektsnominalphrase mit strukturellem Kasus an erster Stelle der \subcatl steht.%
}

\eas
Ausschnitt aus dem Kasusprinzip von \citew[\page 280]{Mueller99a}\\\istype{head"=argument"=structure}
\onems[head"=argument"=structure]{ 
     synsem  \ms{ loc$|$cat$|$head & \ms[verb]{ vform & fin \\ } \\ 
                 } \\
     h-dtr$|$synsem$|$loc$|$cat$|$subcat \sliste{ NP[\textit{str\/}] } $\oplus$ \ibox{1} \\
    } ~\impl \\ \\
\mbox{}\hspace{3cm}\ms{ h-dtr$|$synsem$|$loc$|$cat$|$subcat & \sliste{ NP[\textit{snom\/}] } $\oplus$ \ibox{1}  \\
   }
\zs

\noindent
Das Problem an dieser Art Kasuszuweisung ist, daß phrasale Strukturen vorliegen müssen,
damit Kasus zugewiesen werden kann. Die Konsequenz ist, daß auch intransitive Verben wie \emph{schlafen}
in Infinitivkonstruktionen wie (\mex{1}) in Kopf"=Argument"=Strukturen oder ähnlichen Strukturen
projiziert werden müssen, damit das Subjekt Kasus erhält.
\ea
weil er nicht versucht hat zu schlafen
\z
Eine solche unäre Projektion wurde von \citet*[\page 32, Fn. 32]{ps2} für das Englische
angenommen, würde aber zu vielfältigen Interaktionen mit der hier für das Deutsche
entwickelten Grammatik führen. Zum Beispiel bekommt man unechte Mehrdeutigkeiten,
wenn man annimmt, daß die Information über den phrasalen Status eines Komplements 
nicht Teil der Information ist, die in nichtlokalen Abhängigkeiten durch Strukturteilung
identifiziert wird (wenn der \lexw\isfeat{lex} Teil von \synsem und nicht von \HPSGloc ist).\footnote{
  Zu \textsc{lex} siehe Kapitel~\ref{chap-verbalkomplex}.%
}
\ea
Schlafen will Maria.
\z
Bei einer Analyse für (\mex{0}), wie sie im Kapitel~\ref{sec-pvp} vorgestellt wird, ergeben
sich dann zwei Lesarten: eine, in der {\em schlafen\/} direkt Füller für {\em will Maria\/}
ist, und eine, in der {\em schlafen\/} vorher projiziert wird (zu diesem und anderen Problemen siehe
\citew[Kapitel~15.8]{Mueller99a}).

Gunkel schlägt eine andere Behandlung der Voranstellung von Teilphrasen vor, die
zwar dieses Problem nicht, aber dafür andere, schwerwiegendere hat. Gunkels Analyse
für die Voranstellung von Phrasen wird in Kapitel~\ref{sec-pvp-vdrei} genauer diskutiert.






\subsection{Przepi{\'o}rkowski: 1999b}
\label{kasus-adamp}

Für das Verständnis der beiden folgenden Abschnitte wird die Lektüre
der Kapitel~\ref{chap-verbalkomplex}--\ref{chap-passiv} empfohlen.

\citet{Prze99} schlägt vor, Kasus in bezug auf die \argstl zuzuweisen. \argst\isfeat{arg-st} ist eine
Liste, die sämtliche Argumente eines Kopfes enthält. Die Verwendung einer
Liste mit allen Argumenten wurde von \citet[Kapitel~9]{ps2} eingeführt. Pollard
und Sag schlagen vor, Fernabhängigkeiten\is{Extraktion} über Lexikonregeln einzuführen.
Extrahierte Elemente sind dann nicht mehr in den Valenzlisten enthalten.
Da extrahierte Elemente aber auch für andere Teilbereiche der Theorie -- wie \zb
die Bindungstheorie\is{Bindungstheorie} -- wichtig sind und da Eigenschaften wie
relative Obliqueness\is{Obliqueness} durch die Stellung in einer Liste kodiert sind, wird für die
Kodierung dieser Eigenschaften eine weitere Liste, die \argstl, verwendet.

\prz schlägt vor, die Repräsentation von Argumenten in Valenzlisten
weiter zu strukturieren. Zusätzlich zur \type{synsem}"=Information wird
ein Boolesches Merkmal (\textsc{realized}\isfeat{realized}\label{page-realized}) verwendet,
das Auskunft darüber gibt, ob ein Argument lokal realisiert 
oder angehoben wurde. Elemente, die mit ihrem Kopf über das Kopf"=Argument"=Schema
verbunden werden, werden als \textsc{realized}+ gekennzeichnet. Dasselbe gilt
für Argumente, die extrahiert oder als Klitikon\is{Klitisierung}
morphologisch realisiert werden.
Argumente, die von übergeordneten Köpfen angezogen werden, werden als
\textsc{realized}$-$ markiert. Für die Kasuszuweisung ist dann jeweils nur
die Umgebung relevant, in der ein Element realisiert wird.
\citet[\page 238]{Prze99} gibt folgendes Kasusprinzip an:
\eal
\label{case-prz}
\ex \ms[cat]{
    head & verb\\
    arg-st & \liste{ \ms{ argument & \textrm{NP[\str]}\\
                          realized & +\\
                        } } $\oplus$ \ibox{2}\\ } \impl \\
    \mbox{}\hspace{3cm}\ms{ arg-st & \liste{ \ms{ argument & \textrm{NP[\type{snom}]}\\ } } $\oplus$ \ibox{2} }
\ex \ms[cat]{
    head & verb\\
    arg-st & \ibox{1} \type{ne\_list} $\oplus$ \liste{ \ms{ argument & \textrm{NP[\str]}\\
                          realized & +\\
                        } } $\oplus$ \ibox{2}\\ } \impl \\
    \mbox{}\hspace{3cm}\ms{ arg-st & \ibox{1} $\oplus$ \liste{ \ms{ argument & \textrm{NP[\type{sacc}]}\\ } } $\oplus$ \ibox{2} }
\zl
Diese Formalisierung erfaßt die Kerndaten in (\mex{1}) und (\mex{2}) korrekt:
\eal
\ex Er repariert den Wagen.
\ex Sie läßt ihn den Wagen reparieren.
\zl
\eal
\ex weil er den Wagen zu reparieren versuchte
\ex weil der Wagen zu reparieren versucht wurde
\zl
In (\mex{-1}a) werden die Argumente von \emph{repariert} direkt realisiert.
Sie bekommen deshalb entsprechend der Implikationen in (\mex{-2}) Nominativ
bzw.\ Akkusativ zugewiesen. In (\mex{-1}b) dagegen werden die Argumente von
\emph{reparieren} nicht als Argumente des Hauptverbs, sondern als Argumente
von \emph{reparieren läßt} realisiert. Dies führt dazu, daß das Subjekt
von \emph{reparieren} Akkusativ bekommt, da es an der zweiten Stelle
der Argumentstruktur des Komplexes \emph{reparieren läßt} steht.

Die Erklärung des sogenannten Fernpassivs\is{Passiv!Fern-} in (\mex{0}b) verläuft analog:
In (\mex{0}a) wird dem Objekt von \emph{reparieren} als Argument von \emph{reparieren} Kasus
zugewiesen, in (\mex{0}b) dagegen wird das Objekt angehoben, und da
das Subjekt von \emph{versucht} durch die Passivierung unterdrückt wird,
wird das Objekt von \emph{zu reparieren} zum ersten Element der
Argumentstruktur von \emph{zu reparieren versucht wurde} und erhält somit
Nominativ.

Bestimmte schwierige Fälle von Fernpassiv lassen sich mit den Implikationen
in (\mex{-2}) aber nicht erfassen. Sätze wie die
in (\mex{1}) werden im Kapitel~\ref{sec-remote-passive-phen}
noch ausführlich diskutiert:
\ea\label{erfolg-auszukosten-erlaubt-kasus}
\iw{auskosten}
Der Erfolg        wurde uns      nicht auszukosten erlaubt.\footnote{
        \citew[\page 110]{Haider86c}.%
}
\z
In (\mex{0}) wurde das Objektkontrollverb \emph{erlauben} passiviert,
und das Objekt von \emph{auskosten} wird zum Subjekt der gesamten Konstruktion.
Der Verbalkomplex \emph{auszukosten erlaubt wurde} hat die Valenz- bzw.\
Argumentstrukturliste in (\mex{1}):
\ea
\emph{auszukosten erlaubt wurde}:   \subcat \sliste{ NP[\type{ldat}], NP[\type{str}] }
\z
Die Implikation in (\ref{case-prz}) würden der NP[\str] Akkusativ zuweisen.
Das läßt sich reparieren, indem man verlangt, daß in der ersten Implikation
keine Nominalphrase mit strukturellem Kasus vor der NP[\str] steht, die dann
Nominativ bekommt, \dash, entweder steht vor dieser NP nichts, eine NP mit lexikalischem
Kasus oder etwas, das keine NP ist. Analog muß die zweite Implikation 
so geändert werden, daß beliebiges Material vor der NP[\str] stehen kann,
vorausgesetzt dieses enthält eine weitere NP mit strukturellem Kasus.

\prz kritisiert Heinz und Matiaseks Bezugnahme auf die Konstituentenstruktur
als nicht angemessen, da Kasusvergabe ein lokales Phänomen ist und eine
Analyse deshalb ohne Bezug auf bestimmte Konfigurationen auskommen sollte. \prz{}s Analyse ist aber
mit der normalerweise angenommenen Analyse der Voranstellung von Teilverbalphrasen nicht verträglich, wenn er nicht
selbst Restriktionen in bezug auf Lokalität\is{Lokalität} aufgibt. Das soll im folgenden
etwas ausführlicher erläutert werden: \argst ist ein Merkmal, das nur für
Wörter zulässig ist \citep[\page 236]{Prze99}. Betrachtet man nun Sätze wie
(\mex{1}), sieht man, daß das Vorfeld von einer komplexen Projektion
besetzt wird.
\eal
\label{bsp-der-aufsatz-gelesen}
\ex {}[Der Aufsatz gelesen] wurde am Wochenende.
\ex {}[Den Aufsatz gelesen] hat er am Wochenende.
\zl
Innerhalb dieser Projektion wird ein Argument des Verbs realisiert,
aber welchen Kasus es tragen muß, hängt vom Rest des Satzes ab.
Zu weiteren Daten nach dem Muster von (\mex{0}a)
siehe auch (\ref{bsp-subjekt-im-vf-passiv}) auf Seite~\pageref{bsp-subjekt-im-vf-passiv}.


Die Kombination der Hilfsverben (bzw.\ der entsprechenden Verbspur)
mit der Projektion im Vorfeld kann nicht auf die Argumentstruktur von
\emph{gelesen} Bezug nehmen, da diese nicht Bestandteil der Projektion
\emph{der/den Aufsatz gelesen} ist. Sie ist nur bei \emph{gelesen} selbst repräsentiert.

Man könnte dieses Problem zu lösen versuchen, indem man die Argumentstruktur
projiziert, so daß sie auch an phrasalen Knoten präsent ist. Dann könnten Perfekt
und Passiv"=Hilfsverb auf die Argumentstruktur von \emph{gelesen} zugreifen.
Für die korrekte Kasuszuweisung in (\mex{0}) muß man aber zusätzlich
in der Argumentstruktur von \emph{gelesen} markieren, daß das Subjekt
in (\mex{0}a) blockiert, also für die Kasuszuweisung irrelevant ist.
Diese argumentstrukturbezogene Lösung läßt sich aber nicht auf Beispiele
wie (\mex{1}) anwenden. In (\mex{1}) muß das Subjekt des Verbs im Vorfeld
als Akkusativ realisiert werden.
\eal
\label{bsp-den-saenger-jodeln}
\ex[?]{
{}[Den Sänger jodeln]\iw{jodeln} läßt\iw{lassen} der König.\footnote{
	\citew[\page 57]{Oppenrieder91a}.
	}
}
\ex[*]{
Der Sänger jodeln\iw{jodeln} läßt\iw{lassen} der König.
}
\ex[]{
{}[Den Mechaniker das Auto reparieren] ließ der Lehrer schon oft.\footnote{
  \citew[\page 32]{Grewendorf94a}.
}
}
\ex[*]{
Der Mechaniker das Auto reparieren ließ der Lehrer schon oft.
}
\zl
Selbst wenn die Argumentstruktur von \emph{jodeln} bzw.\ \emph{reparieren}
für \emph{lassen} zugänglich ist, nützt das nichts, denn in \prz{}s
Ansatz ist für die Kasusvergabe entscheidend, wo ein Element realisiert wird.

\citet{Prze99b} integriert deshalb Meurers Ansatz der Kasuszuweisung (\citeyear{Meurers99b}; \citeyear[Kapitel~10.4.1.4]{Meurers2000b})
in seine Theorie und führt zusätzlich zum \textsc{realized}"=Merkmal
das Merkmal \textsc{raised}\isfeat{raised} ein, das den Wert + hat, wenn ein
Argument angehoben wird, und den Wert $-$, wenn das nicht der Fall ist.
Der \textsc{raised}"=Wert eines Arguments wird über die folgende Beschränkung
festgelegt:
\ea
\label{raised-wert}
\type{unembedded-sign}\istype{unembedded-sign} \impl\\
\mbox{}\hspace{3ex}($\forall$\ibox{0},\ibox{1},\ibox{2},\ibox{3} (\ibox{0} \ms[cat]{
                                                           head   & \ibox{1}\\
                                                           arg-st & \ibox{2}\\
                                                      } $\wedge$ member(\ibox{3}[\textsc{arg} \ibox{4} ], \ibox{2})) \impl\\
\mbox{}\hspace{6ex}(\ibox{3} [\textsc{raised} +] $\leftrightarrow$\\
\mbox{}\hspace{9ex}$\exists$[\argst \ibox{5} ]\\
\mbox{}\hspace{9ex}(member([\textsc{arg} \ibox{4} ],\ibox{5}) $\wedge$ member([\textsc{arg$|$loc$|$cat$|$head} \ibox{1} ], \ibox{5}))))
\z
Übersetzt man diesen Ausdruck in eine besser lesbare Form, erhält man:
\ea
In einem nicht eingebetteten Zeichen (\dash in einer Äußerung)\\
\mbox{}\hspace{3ex}gilt für alle \type{category}"=Objekte in diesem Zeichen, die den \headw \iboxt{1} und\\
\mbox{}\hspace{3ex}die \argst \iboxt{2} haben, wobei \iboxt{3} ein Element von \iboxt{2} mit dem
     \textsc{argument}"=Wert \iboxt{4}\\
\mbox{}\hspace{3ex}ist,\\
\mbox{}\hspace{3ex}daß dieses Element \iboxt{3} \textsc{raised}+ ist gdw.\ \\
\mbox{}\hspace{6ex}es eine \argstl gibt, die ein Element mit demselben [\textsc{arg} \ibox{4} ] und\\
\mbox{}\hspace{6ex}außerdem noch ein Element mit demselben \headw wie \ibox{0}, nämlich \iboxt{1},\\
\mbox{}\hspace{6ex}enthält.
\z
Das heißt, man sucht nach einer Argumentstrukturliste, die zu einem Matrixverb gehört, das
\iboxt{0} einbettet. Ein Argument von \iboxt{0}, nämlich \iboxt{3}, kommt in beiden Argumentstrukturlisten
vor und wird deshalb in der Argumentstrukturliste des eingebetteten Kopfes als \textsc{raised}+ markiert.

Das Kasusprinzip wird dann unter Bezugnahme auf die \textsc{raised}"=Werte wie folgt neu formuliert:
\eal
\label{case-prz-zwei}
\ex \begin{tabular}[t]{@{}l@{}}
    \ms[cat]{
    head & verb\\
    arg-st & \liste{ \ms{ argument & \textrm{NP[\str]}\\
                          raised & $-$\\
                        } } $\oplus$ \ibox{2}\\ } \impl \\
    \mbox{}\hspace{3cm}\ms{ arg-st & \sliste{ [ \textsc{argument}~~  \textrm{NP[\type{snom}]} ] } $\oplus$
    \ibox{2} }
    \end{tabular}
\ex \begin{tabular}[t]{@{}l@{}}
    \ms[cat]{
    head & verb\\
    arg-st & \ibox{1} \type{ne\_list} $\oplus$ \liste{ \ms{ argument & \textrm{NP[\str]}\\
                          raised & $-$\\
                        } } $\oplus$ \ibox{2}\\ } \impl \\
    \mbox{}\hspace{3cm}\ms{ arg-st & \ibox{1} $\oplus$ \sliste{ [ \textsc{argument}~~ \textrm{ NP[\type{sacc}]} ] } $\oplus$ \ibox{2} }
    \end{tabular}
\zl
Diesen Ansatz kann man aus zweierlei Gründen kritisieren: Zum einen ist die globale Beschränkung
unschön, die nur für eine gesamte Äußerung festlegt, welche Elemente angehoben wurden und welche nicht.
Die Theorie würde Äußerungen wie (\mex{1}) also nur mit Bezug auf eine vollständige Struktur zurückweisen.
\eal
\ex[*]{
Der Mann, den ihn kennt, lacht.
}
\ex[*]{
Er kommt herüber, um der Mann zu begrüßen.
}
\zl
In beiden Fällen ist jedoch klar, daß Kasus innerhalb des Relativsatzes bzw.\
innerhalb des Infinitivs zugewiesen wird, denn aus diesen Bereichen kann nichts angehoben werden. 
Siehe Kapitel~\ref{kasus-anhang} zu einem anderen Vorschlag zur Bestimmung des \textsc{raised}"=Wertes.

Das zweite Problem mit dem revidierten Ansatz ist dasselbe, das bereits diskutiert
wurde: Die Argumentstruktur wird bei \prz nicht projiziert. Deshalb ist
unklar, wie in (\ref{bsp-der-aufsatz-gelesen}) bzw.\ (\ref{bsp-den-saenger-jodeln})
die Information über die Argumentstruktur des Verbs im Vorfeld zum
Matrixverb gelangen soll. Man muß also die Argumentstruktur bzw.\ eine entsprechende
Repräsentation projizieren. Ein entsprechender Vorschlag wurde von Detmar Meurers gemacht.
Dieser kann erst im Kapitel~\ref{kasus-anhang} vorgestellt werden, da erst dann 
die Verbalkomplexbildung und das Fernpassiv besprochen wurden.




\subsection{Nominativzuweisung durch das finite Verb und Nullkasus}
\label{sec-kasus-fin-verb-nullkasus}

%Abney87a:16
%Haider hat noch PRO und PRO ist immer Nominativ  ???
% Grewendorf88a:161 -> PRO hat keinen Kasus
Ansätze,\is{Kasus!Null-|(} die davon ausgehen, daß der Nominativ nur vom finiten Verb\footnote{
  Mitunter wird angenommen, daß Nominativ nur von funktionalen Köpfen wie INFL (Inflection) oder T (Tense) zugewiesen
  wird. Aus den Texten wird dann nicht ganz klar, ob die Autoren annehmen, daß INFL bzw.\ T nur für
  die Projektionen finiter Verben eine Rolle spielen, oder ob sie davon ausgehen, daß auch
  Infinitive unter T eingebettet sind. Siehe \citet[\page 58]{Wurmbrand2003a} für einen Vorschlag der Einbettung
  der Infinitive mit \emph{zu} unter T.%
} zugewiesen wird
% Chomsky / Lasnik im Handbuch Syntax, Chomsky 1995:
% PRO has Null Case and therefore can only appear in infinitive clauses where the verb lacks Tense,
% and thus has no Nom Case. (Karimi2005a:102)
%
(\citealp[\page 50]{Chomsky93a}, \citealp[\page 26]{Haider84b}, \citealp[\page 73]{FF87a},
% Wunderlich läßt aber offen, ob es INFL oder das Verb ist, das Kasus zuweist.
%\citealp[\page 286]{Wunderlich87c},
\citealp[\page 183]{Bierwisch90a}, \citealp{Molnarfi98a},
\citealp*[\page 475]{BBM2001a}, \citealp[\page 88]{Eroms2000a}, \citealp[\page 21, \page 45, \page 295]{Abraham2005a}), 
können die Daten in (\ref{bsp-nominativ-inkoh}) und (\ref{bsp-nominativ-adj}) nicht erklären.
In (\ref{bsp-nominativ-inkoh}) wird das nicht realisierte Subjekt der Infinitivverbphrase
zwar vom finiten Verb kontrolliert,\is{Objekt!-kontrolle} der Kasus wird jedoch nicht durch das finite Verb
bestimmt. So ist \zb in (\ref{bsp-nominativ-inkoh-geraten}) -- hier als (\mex{1}) wiederholt -- das Dativobjekt 
des finiten Verbs (\emph{raten}) koreferent mit dem Subjekt von \emph{kündigen}. 
\ea
Ich habe den Burschen geraten, im Abstand von wenigen Tagen einer nach dem anderen zu kündigen.
\z
Der Kasus der beiden Elemente ist jedoch verschieden. Das heißt, daß der Kasus in (\mex{0})
innerhalb der Infinitivumgebung zugewiesen werden muß. In der GB-Theorie wird meist davon
ausgegangen, daß das Subjekt von kontrollierten Infinitiven nicht regiert ist, also auch
keinen Kasus bekommen kann. Es wird gesagt, daß solche Subjekte Null"=Kasus bzw.\ keinen Kasus
haben (\citealp[\page 161]{Grewendorf88a}; \citealp[\page 42]{Frey93a}).
Wenn das Subjekt von \emph{kündigen} Null"=Kasus hat,
dann braucht man einen speziellen Mechanismus, der sicherstellt,
daß {\em ein- nach d- ander-\/} immer im Nominativ steht, wenn es sich auf ein nicht 
ausgedrücktes Subjekt bezieht. Die hier vertretene Analyse scheint einfacher,
da nicht angehobene Subjekte immer Nominativ bekommen und somit
die Kongruenz der \emph{ein- nach d- ander}"=Phrase erwartet ist.
%\ifthenelse{\boolean{draft}}{
%}{}
\is{Kasus!Null-|)}




\section*{Kontrollfragen}


\begin{enumerate}
\item Welche Arten von Kasus unterscheidet man?
\item Warum wird bei der Passivierung von (\mex{1}a) der Akkusativ nicht zum Nominativ?
      \eal
      \ex[]{
      weil er den ganzen Tag arbeitete
      }
      \ex[*]{
      weil der ganze Tag gearbeitet wurde
      }
      \ex[]{
      weil den ganzen Tag gearbeitet wurde
      }
      \zl
\end{enumerate}

\section*{Übungsaufgaben}

\begin{enumerate}
\item Welche der NPen in den folgenden Sätzen haben strukturellen, welche lexikalischen Kasus?
\eal
\ex Der Junge lacht.  
\ex Mich friert.    
\ex Er zerstört das Auto.
\ex Das dauert ein ganzes Jahr.
\ex Er hat nur einen Tag dafür gebraucht.
\ex Er denkt an den morgigen Tag.
\zl

\item Laden Sie die zu diesem Kapitel gehörende Grammatik von der Grammix"=CD
(siehe Übung~\ref{uebung-grammix-kapitel4} auf Seite~\pageref{uebung-grammix-kapitel4}).
Im Fenster, in dem die Grammatik geladen wird, erscheint zum Schluß eine Liste von Beispielen.
Geben Sie diese Beispiele nach dem Prompt ein und wiederholen Sie die in diesem Kapitel besprochenen
Aspekte.

\end{enumerate}



\section*{Literaturhinweise}

Kasus werden von Wissenschaftlern aus dem angloamerikanischen Raum oft vernachlässigt, da das
Englische in bezug auf Kasus nicht so interessant zu sein scheint. \citet{ps,ps2} machen zum
Beispiel keinen Unterschied zwischen strukturellen und lexikalischen Kasus. In HPSG"=Grammatiken für
das Deutsche wurden dagegen schon in den ersten Arbeiten zum Deutschen Haiders Ansätze
\citeyearpar{Haider85b} aus der \gbt übernommen und wurde struktureller Kasus in Abhängigkeit von der
syntaktischen Konfiguration zugewiesen \citep[\page 209--210]{HM94a}. Es hat sich jedoch gezeigt,
daß eine rein strukturelle Kasuszuweisung ungeeignet ist. \citet{Meurers99a} und \citet{Prze99b}
haben deshalb Analysen entwickelt, die für die Kasuszuweisung auf die Argumentstruktur bzw.\
Valenzinformation eines Kopfes Bezug nehmen. Siehe auch Kapitel~\ref{kasus-anhang}. 
Das hier vorgestellte Kasusprinzip ähnelt sehr stark dem von \citet*{YMJ87} vorgeschlagenen und kann
damit auch die Kasussysteme verschiedener Sprachen erklären, die von den genannten Autoren besprochen
wurden, insbesondere auch das komplizierte Kasussystem des Isländischen\il{Isländisch}.
Ein wesentlicher Unterschied zwischen dem hier angegebenen Kasusprinzip und dem von
Yip, Maling und Jackendoff ist, daß das Prinzip~\ref{case-p} wegen der Anhebungseinschränkung
monoton\is{Monotonie} ist, \dash, Kasus, die einmal zugewiesen wurden, werden nicht von einem übergeordneten Prädikat überschrieben.

