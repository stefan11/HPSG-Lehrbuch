%% -*- coding:utf-8 -*-
%%%%%%%%%%%%%%%%%%%%%%%%%%%%%%%%%%%%%%%%%%%%%%%%%%%%%%%%%
%%   $RCSfile: hpsg-rs.tex,v $
%%  $Revision: 1.16 $
%%      $Date: 2008/09/30 09:14:41 $
%%     Author: Stefan Mueller (CL Uni-Bremen)
%%    Purpose: 
%%   Language: LaTeX
%%%%%%%%%%%%%%%%%%%%%%%%%%%%%%%%%%%%%%%%%%%%%%%%%%%%%%%%%

\chapter{Relativsätze}
\label{chap-rs}
\is{Relativsatz|(}

{\exewidth{(9)}%
Nach der Behandlung der Vorfeldbesetzung im vorigen Kapitel widmen wir uns jetzt
der Analyse der Relativsätze. Im Abschnitt~\ref{sec-rs-phenomena} werden die syntaktischen
Eigenschaften von Relativsätzen diskutiert, und in Abschnitt~\ref{sec-rs-anal}
wird die Analyse vorgestellt.

\section{Das Phänomen}
\label{sec-rs-phenomena}

Relativsätze bestehen aus einer Phrase, die ein Relativpronomen enthält, die ich im folgenden
\emph{Relativphrase}\is{Relativphrase} nennen werde, und einem sich daran anschließenden finiten Satz mit dem finiten Verb in 
Endstellung, aus dem die Relativphrase vorangestellt wurde. Beispiele verschiedenster Art zeigt (\mex{1}):
\eal
\label{bsp-relativsaetze}
\ex Der Mann, [\emph{der}] Aicke küßt, liebt sie. \label{r1}
\ex Der Mann, [\emph{den}] Aicke küßt, liebt sie. \label{r1b}
\ex Der Mann, [\emph{dem}] Aicke zuhört, liebt sie.\label{r1c}
\ex Der Mann, [von \emph{dem}] Aicke geküßt wird, liebt sie. \label{r2}
\ex Die Stadt, [in \emph{der}] Karl arbeitet, ist attraktiv. \label{r3}
\ex Ich machte Änderungen, [\emph{deren}\iw{deren!Relativpronomen} Tragweite] mir nicht bewußt war.\label{r4}
\ex es hätte die FDP zerrissen und Kandidat Scharping das Signal gebracht, [\emph{dessen} entbehrend] er schließlich scheiterte.\footnote{
  taz, 20.10.1998, S.\,1.}
\ex Das ist ein Umstand, [\emph{den} zu berücksichtigen] meist vergessen wird.\label{r5}\footnote{
  \citet[\page79]{Bech55a} gibt ein nahezu identisches Beispiel.%
}
\ex Die Nato befindet sich in einem Zustand, [\emph{den} zu verhindern] sie eigentlich gegründet wurde.\footnote{
  Martin Schulze, Bericht aus Bonn, ARD, 23.04.1999.}
\zl
Die Relativphrase kann ein Subjekt, (Akk/Dat/Gen/PP) Objekt, Adjunkt oder VP"=Komplement sein.
\label{page-rattenfaenger}%
Die Relativphrase kann komplex sein (VP, PP, NP) oder nur aus einem Pronomen bestehen.
Bei komplexen NPen ist das Relativwort ein Possesivum\is{Pronomen!Possessiv-}. 
Beispiele wie (\mex{0}d--i) werden nach \citet*[\page 108]{Ross67} in Anlehnung an die Geschichte vom 
\href{http://www.hameln.de/tourismus/rattenfaenger/}{Rat\-ten\-fän\-ger von Hameln} 
auch Rattenfängerkonstruktionen\is{Rattenfängerkonstruktion} genannt,
weil das Relativpronomen das Material in der jeweils vorangestellten Konstituente "`mitzieht"', \dash
zusammen mit anderem Material vorangestellt wird.

Das Relativwort\is{Kongruenz!Relativwort"=Nomen} muss mit seinem Bezugsnomen in Numerus und Genus übereinstimmen. Die Kasus von
Bezugsnomen und Relativpronomen können allerdings verschieden sein. Der Kasus des Relativpronomens
wird nur von dessen jeweiligem Kopf, meist dem Verb im Relativsatz bestimmt.

In (\mex{0}) handelt es sich bei den Relativpronomina\is{Pronomen!Relativ-|(} um sogenannte \emph{d}"=Pronomina, aber
auch \emph{w}"=Pronomina kommen als Relativum vor, wie die Beispiele in (\mex{1}) zeigen.
\eal
\label{bsp-rs-w-pron}
% weiterführender RS
% \ex Ich komme eben aus der Stadt, [\emph{wo}] ich Zeuge eines Unglücks gewesen bin.\footnote{
% 	\citep*[\page672]{Duden84}
% 	}\label{bsp-wo-ich-zeuge}
\ex Studien haben gezeigt, daß mehr Unfälle in Städten passieren, [\emph{wo}\iw{wo!Relativpronomen}] 
      die Zebrastreifen abgebaut werden, weil die Autofahrer unaufmerksam werden.\footnote{
        taz berlin, 03.11.1997, S.\,23.
        }
\ex Zufällig war ich in dem Augenblick zugegen, [\emph{wo}] der Steppenwolf 
      zum erstenmal unser Haus betrat und bei meiner Tante sich einmietete.\footnote{
                Herman Hesse, \emph{Der Steppenwolf}. Berlin und Weimar: Auf"|bau-Verlag. 1986, S.\,6.
	}
\ex Tage, [an \emph{welchen}] selbst die Frage, ob es nicht an der Zeit sei, dem Beispiele
      Adalbert Stifters zu folgen und beim Rasieren zu verunglücken, ohne Auf"|regung
      oder Angstgefühle sachlich und ruhig erwogen wird,\footnote{
		ebenda, S.\,27.
	}
\ex War das, [\emph{worum}\iw{worum!Relativpronomen}] wir Narren uns mühten, schon immer vielleicht nur ein Phantom gewesen?\footnote{
		ebenda, S.\,39.
	}

\ex Den Hass gegen Bill Gates' Firma schürten auch Meldungen, [\emph{wonach}]\iw{wonach!Relativpronomen} die USA die Angriffssoftware
      taiwanischer Kampfjets auf Microsoft"=Basis "`ziel\-an\-ge\-paßt"' hätten.\footnote{
        taz, 12.01.2000, S.\,9.
      }
\ex Dort vielleicht war das, [\emph{was}\iw{was!Relativpronomen}] ich begehrte, dort vielleicht würde meine Musik gespielt.\footnote{
                Herman Hesse, \emph{Der Steppenwolf}. Berlin und Weimar: Auf"|bau-Verlag. 1986, S.\,40.
	}\label{bsp-meine-musik}
\ex {}[\ldots], das ist nun wieder eine Frage, [über \emph{welche}\iw{welche}] müßige Leute nach Belieben brüten
	mögen.\footnote{
                ebenda, Tractat vom Steppenwolf, S.\,6.
	} 
\ex {}[\ldots], heute gibt es nichts, [\emph{was}] der kritischen Betrachtung wert wäre oder
      [\emph{wor\-über}] sich auf"|zuregen lohnte.\footnote{
	taz, 14.11.1996, S.\,13.
      }
% GRI/SAG.00391 Die Langobarden und Aßipiter, S. 359
% Und um dies glaubhafter zu machen, stellten sie ihre Zelte weit auseinander und zündeten viele Feuer im Lager an. Die Aßipiter gerieten dadurch in Furcht und wagten nun den Krieg, womit sie gedroht hatten, nicht mehr zu führen.
\zl
\is{Pronomen!Relativ-|)}

\noindent
Es gibt verschiedene Arten von Relativsätzen: Sie können wie in (\mex{1}) Nomina modifizieren,
sie können sich als weiterführende Relativsätze\is{Relativsatz!weiterführender} auf einen ganzen Satz beziehen (\mex{2}), oder sie
können direkt als Argument (\mex{3}) oder Adjunkt (\mex{4}) eines Kopfes
frei\is{Relativsatz!freier} auf"|treten.
\ea
der Mann, der schläft
\z

\ea
Anna hat die Schachpartie gewonnen, was Peter ärgerte.
\z                                              

\eal
\label{bsp-frei-rs-subj}
\ex Wer das schriftliche Produkt eines Verwaltungsbeamten 
      als "`mittleren Schwachsinn"' bezeichnet,
      muß mit 2.400 Mark Geldstrafe rechnen.\footnote{
        taz, 30.11.1995, S.\,20.}
\ex Macht kaputt, [\emph{was}] euch kaputtmacht!\footnote{
        Ton, Steine, Scherben, \emph{Warum geht es mir so dreckig?}, erschienen bei Indigo, David Volksmund Prod.\ als LP und CD, 1971.
      }
\zl
\eal
\label{bsp-frei-rs-mod}
\ex Wo das Rauchen derartig stigmatisiert ist wie von Köppl geplant, 
      kann man sich leicht als Rebell fühlen, bloß weil man raucht.\footnote{
	taz, 15.11.1996, S.\,10.
}\label{bsp-rauchen-stigmatisiert}
\ex Wo noch bis zum Dezember vergangenen Jahres die "`Projekte am Kollwitzplatz"' und
      "`Netzwerk Spielkultur"' ihren Sitz hatten, prangt heute das Schild "`Zu vermieten"'.\footnote{
        taz berlin, 27.07.1997, S.\,23.
        }
\zl
}%exewidth

\noindent
Weiterführende und freie Relativsätze können hier nicht besprochen werden. Der interessierte Leser
sei auf \citew{Holler2003a} bzw.\ auf \citew{Bausewein90} und \citew{Mueller99a,Mueller99b} verwiesen.

Wie bereits im vorigen Kapitel gezeigt wurde (siehe Seite~\pageref{bsp-nla-rs}), 
kann man die Abfolgen in Relativsätzen nicht durch
eine lokale Umordnung der Relativphrase erklären. Beispiele wie die in (\mex{1}) zeigen
klar, dass hier eine Fernabhängigkeit vorliegt:
\ea
das Thema, [über das]$_i$ er Peter gebeten\iw{bitten} hat, [\sub{VP} [einen Vortrag\iw{Vortrag} \_$_i$] zu halten],
\z
\ea
Wollen wir mal da hingehen, wo$_i$\iw{wo!Relativpronomen} Jochen gesagt hat, [dass es \_$_i$ so gut schmeckt]?
\z
Hier besteht eine Analogie zur Vorfeldbesetzung in (\mex{1}).
\eal
\ex Über dieses Thema hat er Peter gebeten, einen Vortrag zu halten.
\ex Wo hat Jochen gesagt, dass es so gut schmeckt?
\zl



\section{Die Analyse}
\label{sec-rs-anal}

Dieser Abschnitt ist in zwei Teile geteilt: Zuerst beschäftigen wir uns
mit der Syntax der Relativsätze. Im Abschnitt~\ref{sec-rs-sem-anal} wird dann ihre Semantik
behandelt.

\subsection{Die Syntax von Relativsätzen}
\label{sec-syntax-rs}
\label{lp-rs}% Sections zusammengeschmissen

Nach der Lektüre des vorangegangenen Kapitels ist klar, wie Fernabhängigkeiten wie \zb die
Vorfeldbesetzung und die Voranstellung in Relativsätzen modelliert werden können. In Relativsätzen gibt
es jedoch noch eine zweite Fernabhängigkeit: Relativpronomina können tief in der Relativphrase
eingebettet sein. Die Information darüber, dass eine Phrase ein Relativpronomen enthält, muss
am obersten Knoten der betreffenden Phrase vorhanden sein, denn sonst wäre es nicht möglich zu erklären,
wieso die Sätze in (\mex{1}b,c) keine Relativsätze sind und höchstens als Einschübe zu verstehen sind.
\eal
\ex[]{
Der Mann, der schläft, schnarcht.
}
\ex[*]{
Der Mann, er schläft, schnarcht.
}
\ex[*]{
Der Mann, der Mann schläft, schnarcht.
}
\zl
In (\mex{0}b) und (\mex{0}c) steht \emph{er} bzw.\ \emph{der Mann} an der Stelle des
Relativpronomens, und weder \emph{er} noch \emph{der Mann} enthält ein Relativpronomen.

Außerdem muss man die Übereinstimmung des Relativpronomens mit seinem Bezugswort in Numerus und Genus sicherstellen.
\eal
\ex Der Mann$_i$, [von \emph{dem}$_i$] Aicke geküßt wird, liebt sie.
\ex Die Stadt$_i$, [in \emph{der}$_i$] Karl arbeitet, ist attraktiv.
\ex Änderungen$_i$, [\emph{deren}$_i$ Tragweite] mir nicht bewußt war
\ex das Signal$_i$, [\emph{dessen}$_i$ entbehrend] er schließlich scheiterte
\ex ein Umstand$_i$, [\emph{den}$_i$ zu berücksichtigen] meist vergessen wird
\zl
Die Kongruenz und Koreferenz von Bezugsnomen und Relativpronomen läßt sich einfach durch eine Koindizierung\is{Koindizierung}, 
eine Strukturteilung der \textsc{index}"=Werte ausdrücken. Dazu wird der referentielle Index
des Relativwortes nach oben gereicht, so wie das für die Vorfeldbesetzung mit
\textsc{local}"=Werten gemacht wurde (vergleiche die Extraktionsspur in (\ref{le-extraktionsspur}) auf Seite~\pageref{le-extraktionsspur}).
Die entsprechende Fernabhängigkeit beginnt jedoch in einem phonologisch gefüllten Lexikoneintrag:
\ea\is{Pronomen!Relativ-}
\ms[word]{ phon & \phonliste{ dem } \\
    loc & \ms{ cat & \ms{ head & \ms[noun]{
                                                  cas & dat\/ \\ 
                                                } \\
                                         spr    & \liste{} \\
                                         comps & \liste{} \\
                                       } \\
                               cont & \ms{ ind & \ibox{1} \ms{ per & 3 \\
                                                               num & sg \\
                                                               gen & mas $\vee$ neu \\
                                                             } \\
                                         } \\
                             } \\
                     nonloc & \ms{ %que   & \liste{} \\
                                                rel   & \sliste{ \ibox{1} } \\
                                                slash & \liste{} \\ 
                                                %extra & \liste{} \\
                                  } \\
   }
\z
In der Extraktionsspur wird der \localw mit dem \slashel identifiziert,
in den Einträgen für Relativpronomina gibt es eine Strukturteilung zwischen dem Index des Pronomens
und dem Element von \textsc{rel}.

Abbildung~\vref{abb-rel-percolation} zeigt das Weiterreichen der Information über
den referentiellen Index eines Relativpronomens in einer komplexen Präpositionalphrase.
\begin{figure}
\centering
\begin{forest}
sm edges
[{PP[\textsc{rel} \sliste{ \ibox{1} }]}
  [P [von]]
  [{NP[\textsc{rel} \sliste{ \ibox{1} }]}
    [{Det[\textsc{rel} \sliste{ \ibox{1} }] } [dessen] ]
    [N [Schwester] ] ] ]
\end{forest}
\caption{\label{abb-rel-percolation}Weiterreichen des \textsc{rel}-Wertes in Relativphrasen}
\end{figure}

\noindent
Bei der Analyse des Relativsatzes in (\mex{1}) muss die Fernabhängigkeit, die
im finiten Satz beginnt, abgebunden werden.
\ea
Mann, [von dessen Schwester]$_i$ [Aicke ein Bild \_$_i$ gemalt hat]
\z
Der Füller der Fernabhängigkeit ist die Relativphrase. Da der gesamte Relativsatz
Eigenschaften hat, die nicht mit denen von finiten Sätzen kompatibel sind, kann
das finite Verb nicht der Kopf des Relativsatzes sein (mehr dazu im nächsten Abschnitt).
Ich gehe deshalb davon aus, dass sowohl die Relativphrase als auch der finite Satz, aus
dem sie extrahiert wurde, Nicht"=Kopf"|töchter sind. Abbildung~\vref{abb-mann-von-dessen-schwester-syn}
zeigt die syntaktischen Aspekte der Analyse im Überblick.
\begin{figure}
\centering
\begin{forest}
sm edges
[{RS[\textsc{rel} \eliste, \textsc{slash} \eliste] }
   [{PP[\textsc{loc} \ibox{2}, \textsc{rel} \sliste{ \ibox{1} }]}
     [P [von] ]
     [{NP[\textsc{rel} \sliste{ \ibox{1} }]}
        [{Det[\textsc{rel} \sliste{ \ibox{1} }] } [dessen] ]
        [N [Schwester] ] ] ]
   [{S[\type{fin}, \textsc{slash} \sliste{ \ibox{2} }] } 
      [Aicke ein Bild \_ gemalt hat, roof] ] ]
\end{forest}
\caption{\label{abb-mann-von-dessen-schwester-syn}Perkolation und Abbindung der \textsc{rel}- und \slashwe}
\end{figure}
Der \relw wird vom Relativpronomen bis zum obersten Knoten der Relativphrase hochgereicht. Im Relativsatzschema\is{Schema!Relativsatz-}
wird überprüft, ob die erste Nicht"=Kopf"|tochter einen gefüllten \relw enthält, \dash ob es in der ersten
Phrase ein Relativpronomen gibt. Der \relw der Mutter ist die leere Liste, da der zum Relativpronomen gehörige
Index nicht aus dem Relativsatz hinaus weitergereicht wird. Die Abbindung des \textsc{slash}"=Wertes aus dem
finiten Satz funktioniert parallel zur Vorfeldbesetzung: Der \slashw des Satzes wird mit dem \localw
der Relativphrase identifiziert, der \slashw des Mutterknotens ist die leere Liste, da die Fernabhängigkeit
durch die Relativphrase abgebunden wird. Schema~\vref{rs-schema-struk} zeigt einen Auszug des Schemas, das die Struktur in 
Abbildung~\ref{abb-mann-von-dessen-schwester-syn} lizenziert.
%\begin{figure}
\begin{samepage}
\begin{schema}[Relativsatzschema (strukturelle Aspekte)]
\label{rs-schema-struk}
\type{relative-clause}\istype{relative"=clause} \impl\\
%\resizebox{\linewidth}{!}{
\onems{ loc$|$cat \ms{ head & relativizer \\
                       spr  & \eliste\\
                       comps & \liste{ } \\
                     } \\
                 nonloc \ms{  rel   & \liste{} \\
                              slash & \liste{} \\
                           } \\[6mm]
   nh-dtrs \liste{ \begin{tabular}{@{}l@{}}
                            \ms{ loc    & \ibox{1} \\
                                               nonloc & \ms{%  que   & \liste{} \\
                                                              rel   & \sliste{ [] } \\
                                                              slash & \eliste \\
                                                           } \\
                                   }, %\\
                            \onems{ loc$|$cat \onems{ head \ms[verb]{ initial & $-$ \\
                                                                      vform   & fin \\
                                                                      dsl     & none\\
                                                                                } \\
                                                               comps \liste{} \\
                                                                     } \\
                                                 nonloc \ms
\end{schema}
\end{samepage}
%\vspace{-\baselineskip}\end{figure}
%
Die erste Nicht"=Kopf"|tochter muss ein Element in \rel enthalten. Der \localw der ersten
Nicht"=Kopf"|tochter \iboxb{1} entspricht dem \slashw, der zweiten Nicht"=Kopf"|tochter, einer
vollständig gesättigten (\comps \eliste) Projektion eines finiten Verbs in Letztstellung. Der \dslw
\emph{none} schließt die Verwendung der Verbspur in Relativsätzen aus. Ohne diese Restriktion würde
die Grammatik die ungrammatische Folge in (\mex{1}a) lizenzieren, die aus (\mex{1}b) entsteht, wenn
man statt des Verbs eine Verbspur einsetzt:
\eal
\ex[*]{
Der Mann, den Aicke, schläft.
}
\ex[]{
Der Mann, den Aicke kennt, schläft.
}
\zl

\noindent
Die Relativphrase bindet die Fernabhängigkeit aus dem finiten Satz in Verbletztstellung ab,
weshalb der \slashw der Mutter die leere Liste ist. Genauso wird die Information über das
Relativpronomen innerhalb des Relativsatzes abgebunden: Der \relw der Gesamtstruktur ist die leere Liste. 

Das Ergebnis der Kombination ist ein Relativsatz und hat deshalb den Kopfwert
\type{relativizer}.\label{Erklaerung-relativizer} Relativsätze unterscheiden sich von normalen Sätzen dadurch, dass sie Nomina
modifizieren können, \dash, sie haben einen gefüllten \modw. Würde man Relativsätze einfach als eine
Projektion des finiten Verbs (mit entsprechendem \modw beim Verb) analysieren, müßten finite Verben,
auch wenn sie nicht in Relativsätzen verwendet werden, Nomina modifizieren können. (\mex{1}) sollte
also grammatisch sein, was aber nicht der Fall ist.
\ea[*]{
der Mann, Aicke den Mann kennt
}
\z
Die Modifikation setzt die Relativsatzsyntax voraus, \dash, es muss eine Relativphrase geben und einen
finiten Satz mit Verbletztstellung, aus dem diese Relativphrase vorangestellt wurde. Einfache
Projektionen von Verben wie \emph{Aicke den Mann kennt} können nicht modifizieren.


%\subsection{Linearisierungsregeln für Relativsätze}
%\label{lp-rs}

Bisher\is{Linearisierung!-sregel|(} wurden noch keine Beschränkungen
für die Abfolge der Nicht"=Kopf"|töchter in Relativsätzen angegeben. Natürlich muss die das
Relativpronomen enthaltende Phrase immer vor dem finiten Satz stehen, aus dem sie bewegt wurde. Eine
andere Abfolge ist ungrammatisch, wie das folgende Beispiel zeigt:
\ea[*]{
\label{bsp-lp-rs}
Mann, [\sub{S} Aicke \_$_i$ liebt] den$_i$
}
\z
Außerdem kann die Relativphrase nicht innerhalb einer komplexen
Nominalphrase rechts des Kopfnomens stehen:\footnote{
  Hierin unterscheiden sich Relativsätze von Interrogativsätzen\is{Interrogativsatz},
  denn bei letzteren sind solche Abfolgen möglich:
\ea
Viele Angehörige wußten nicht, an Bord welcher Maschine ihre Angehörigen waren.
  (Tagesschau, 03.01.2004, 20:00)
\zlast
}
\eal
\ex[*]{
Der Mann, [ein Bild von dessen Schwester] Aicke malt, schläft.
}
\ex[*]{
         der Vortrag, die Verfasserin dessen uns sehr attraktiv erscheint\footnote{
           \citew[\page211]{Fanselow87a}.
         }
}
\zl
In (\mex{0}a) gibt es eine nach vorn bewegte Konstituente, die ein Relativpronomen enthält 
(\emph{ein Bild von dessen Schwester}), und dennoch ist der Satz ungrammatisch.
Konstituenten, die ein Relativpronomen enthalten, stehen immer vor anderen Konstituenten.
Eine Ausnahme bilden Präpositionen\is{Pr"aposition} 
und koordinierende Konjunktionen.\is{Koordination}
\eal
\ex der Stuhl, [auf dem] Aicke sitzt
\ex der Moment, [auf den] ich gewartet habe
\zl
\ea
der Mann, [dessen Frau [und dessen Tochter]] ich kenne,
\z
Das wird von der folgenden Linearisierungsregel korrekt erfaßt.
\footnote{
        \citet*{Riemsdijk85} 
        formuliert eine ähnliche Regel. Zu einer Präzisierung dieser
        Regel in Bezug auf Koordinationsstrukturen siehe \citew[\page151]{Mueller99a}.%
}
%% \footnote{
%%         \citet*[\page150]{Wunderlich80}\ia{Wunderlich|fn{\thefootnote}} 
%%         gibt folgendes Beispiel, das der LP-Regel (\ref{lp-rel}) widerspricht.
%%         \ea
%%         ungefähr zu der Zeit, drei Stunden vor der wir im Kino gewesen sind,
%%         \z
%%         Ich finde diesen Satz nicht akzeptabel. Allerdings gibt es Sätze wie (ii),
%%         die Gegenbeispiele zu sein scheinen.
%%         \eal
%%         \ex Die Dativ-NP bezeichnet die Größe, in bezug auf welche der Vergleich
%%         gilt, \ldots{} (Im Haupttext von \citep*[\page54]{Wegener85b}) %auch S. 56
%%         \ex \ldots{} eröffnet eine Möglichkeit, in bezug auf die das Repräsentationsformat
%%               den Rahmen des $\lambda$"=Kalküls verläßt. (Im Haupttext von \citep*[\page54]{Kaufmann95a})
%% %auch S. 142, 191
%%         \ex Wie oben aber schon angesprochen, gibt es andere Optionen in der
%%         Universalgrammatik, die das Auftreten von expletiven Elementen unnötig machen können,
%%         etwa den PRO-DROP-Parameter, mit Hilfe dessen qua Inversion die Subjekts-NP in den Rektionsbereich
%%         des Verbs gelangt. (Im Haupttext von \citep[\page216]{Fanselow87a})
%%         \ex Wir haben in diesem Fall also zwei regierende Kategorien anzunehmen,
%%         relativ zu denen jeweils Pronominalisierung bzw.\ Refelxivierung mit den Mitteln
%%         der Bindungstheorie erklärt ist. (Im Haupttext von \citep[\page187]{Grewendorf83a})
%%         \ex Es gibt jedoch ein syntaktisches Kriterium, auf Grund dessen die
%%         beiden Sätze unterschieden werden können. (Im Haupttext von \citew[\page41]{Steinitz69a})
%%         \zl
%%         Eventuell kann man \emph{in Bezug auf} als feste Wendung, die
%%         Wortcharakter hat, einordnen. Bei Phrasen wie \emph{mit Hilfe} ist ja eine Zusammenschreibung
%%         und eine Einordnung als Präposition möglich.
%%
%%         Fanselow macht sich übrigens auf S.\,211 Gedanken über \textit{Pied Piping}
%%         und gibt die (ii.b) ziemlich ähnliche Phrase (iii) an.
%%         \ea[*]{
%%         der Vortrag, die Verfasserin dessen uns sehr attraktiv erscheint
%%         }
%%         \zlast
%% }
\ea
\label{lp-rel}
\textsc{rel} \sliste{ \type{index} } $< \neg$ P
\z
Diese Regel besagt, dass jede Tochter eines Zeichens mit nichtleerer 
\textsc{rel}"=Liste vor allen anderen Töchtern steht, wobei Präpositionen ausgenommen sind. 

Man beachte, dass die LP"=Regel in (\ref{lp-rel}) für das
Englische\il{Englisch} nicht gilt.
\ea
Here's the minister [[in [the middle [of [whose seremon]]]] 
the dog barked].\footnote{ 
	\citew*[\page212]{ps2}.
}
\z
Diesen Satz kann man nicht mit (\mex{1}) übersetzen.\NOTE{JB: find ich aber fast okay}
\ea[*]{
Das ist der Pfarrer, in der Mitte von dessen Predigt der Hund bellte.
}
\z
Eine Übersetzung wie (\mex{1}) wäre wohl angebrachter.
\ea
Das ist der Pfarrer, der die Predigt hielt, in deren\iw{deren!Relativpronomen} Mitte der Hund bellte.
\z
\is{Linearisierung!-sregel|)}


\subsection{Die Semantik von Relativsätzen}
\label{sec-rs-sem-anal}

Die hier betrachteten Relativsätze (\mex{1}a) verhalten sich wie pränominale
Adjektive (\mex{1}b) oder PPen, die Nomina modifizieren (\mex{1}c).
\eal
\ex die Frau, die alle kennen
\ex die schöne Frau
\ex die Frau im Cafe
\zl
Genau wie diese selegieren sie ein \nbar über das \textsc{mod}"=Merkmal und 
integrieren den semantischen Beitrag des Nomens in ihre Restriktionsliste (siehe Abschnitt~\ref{sem-adj}).
Das heißt, die Relativsätze verhalten sich anders als normale finite Sätze,
denn normale Verbletztsätze können keine Nomina modifizieren:
\eal
\ex[*]{
die Frau, den Bruder alle kennen
}
\ex[*]{
die Frau, alle deren Bruder kennen
}
\zl
Die Modifikation ist nur genau dann möglich, wenn die spezielle Relativsatzsyntax vorliegt.
Enthält der Satz kein Relativpronomen oder ist dieses irgendwo anders als in der ersten
Phrase, werden die Äußerungen ungrammatisch.

Es gibt mehrere Möglichkeiten, mit dieser Situation umzugehen: Man kann
einen phonologisch leeren Kopf annehmen, der einen Satz als Komplement nimmt 
und selbst ein Modifikator ist, der dann nach der Kombination mit dem Satz
die \nbar modifizieren kann. Diesen Weg sind \citet[Kapitel~5]{ps2} gegangen.\footnote{
  Diese Analyse für das Deutsche findet man neben der hier vorgestellten in \citew{Mueller99a,Mueller99b}.
  \citet{Sag97a} schlägt eine weitere Möglichkeit vor: Der \modw von Verben ist
  bei ihm unterspezifiziert. Verben können also in Relativsatzkonstruktionen
  als Kopf auf"|treten. Fälle wie (\mex{0}) muss er anders ausschließen. Siehe Abschnitt~\ref{sec-konstruktionsbasierte-rs}.%
}
Eine Alternative ist, ein Schema zu verwenden, das den Satz
zu einem Modifikator projiziert. In einem solchen Schema gibt es dann keinen
Kopf. Der leere Kopf ist quasi direkt in die Regel integriert. Dieser Ansatz entspricht dem aus Abschnitt~\ref{sec-syntax-rs}.
Das folgende
Schema zeigt die semantischen Aspekte des Relativsatzschemas:


\begin{schema}[Relativsatzschema (semantische Aspekte und Kongruenz)]
\label{rs-schema-sem}\is{Schema!Relativsatz-}
%\medskip
\samepage
\type{relative-clause} \impl\\
%\resizebox{\linewidth}{!}{%
\onems{ loc \ms{ cat & \ms{ head & \onems[relativizer]{ mod \textrm{$\overline{\mbox{\textrm{N}}}$:} \ms{ ind   & \ibox{1} \\
                                                                                           restr & \ibox{2}  \\
                                                                                         } \\
                                                              } \\
                                     } \\
                           cont  & \ms{ ind   & \ibox{1}  \\
                                        restr & \sliste{ \ibox{3} } $\oplus$ \ibox{2} \\
                                     } \\
               }\\[6mm]
   nh-dtrs \liste{ [ nonloc$|$rel   \sliste{ \ibox{1} } ], [ loc$|$cont \ibox{3} ] } \\[2mm]
}%}
\end{schema}

\noindent
Dabei gleicht das, was unter \textsc{loc} steht, dem, was wir schon im Abschnitt~\ref{sem-adj}
kennengelernt haben. Man vergleiche den \locw mit dem Lexikoneintrag für das
Adjektiv \emph{interessantes} in (\ref{le-interessantes-sem}) auf
Seite~\pageref{le-interessantes-sem}. Der \locw von \emph{interessantes} ist der Übersichtlichkeit
halber auch hier in (\mex{1}) angegeben:

\ea
\locw für \emph{interessantes}:\\
\label{le-interessantes-zwei}%
\ms{ 
   cat & \ms{ head & \ms[adj]
                      { %prd & $-$ \\
                        mod &  \textrm{$\overline{\mbox{\textrm{N}}}$:} \ms{ ind   & \ibox{1} \\
                                                                     restr & \ibox{2} \\
                                                                    } \\
                      } \\
               comps & \liste{} \\
             } \\
   cont & \ms{ ind & \ibox{1} \ms{ per & 3 \\
                                 num & sg \\
                                 gen & neu \\
                               } \\
                     restr & \liste{ \ms[interessant]{ 
                                theme & \ibox{1} \\ 
                               }} $\oplus$ \ibox{2}  \\
              } \\
}
\z
Das Adjektiv hat aufgrund seiner Flexion einen bestimmten Genus"=Wert, der sich unter
\textsc{ind} widerspiegelt. Bei Relativsätzen sind die Numerus"= und Genuswerte durch das Relativpronomen
bestimmt. Die entsprechende Information wird im Baum nach oben gereicht, und durch die Strukturteilung
des \textsc{rel}"=Wertes (\iboxt{1} im Schema~\ref{rs-schema-sem}) mit dem \textsc{ind}"=Wert wird dann sichergestellt, dass der gesamte Relativsatz
einen referentiellen Index hat, der dem des Relativpronomens entspricht. Dadurch dass der semantische
Index des modifizierten Nomens ebenfalls mit dem semantischen Index des Relativsatzes identifiziert wird,
ist sichergestellt, dass das Relativpronomen mit dem Nomen, auf das sich der Relativsatz bezieht, in
Numerus und Genus kongruiert.

Adjektive wie \emph{interessantes} steuern in ihrem Lexikoneintrag eine Relation zur Gesamtbedeutung der Phrase,
in der sie dann verwendet werden, bei. In Relativsätzen entspricht diese Relation der Relation, die
vom Verb im Relativsatz kommt, also der \iboxt{3} im Schema~\ref{rs-schema-sem}.
Für einen einfachen Relativsatz wie den in (\mex{1}) bekommt man somit eine Struktur wie (\mex{2}):
\ea
das Kind, das lacht
\z
\ea
\ms{ phon & \phonliste{ das, lacht }\\
   cat & \ms{ head & \ms[relativizer]
                      { %prd & $-$ \\
                        mod &  \textrm{$\overline{\mbox{\textrm{N}}}$:} \ms{ ind   & \ibox{1} \\
                                                                     restr & \ibox{2} \\
                                                                    } \\
                      } \\
               comps & \liste{} \\
             } \\
   cont & \ms{ ind & \ibox{1} \ms{ per & 3 \\
                                 num & sg \\
                                 gen & neu \\
                               } \\
                     restr & \liste{ \ms[lachen]{ 
                                agens & \ibox{1} \\ 
                               }} $\oplus$ \ibox{2}  \\
              } \\
}
\z
Die Bedeutungskombination der gesamten Nominalphrase in (\mex{-1}) erfolgt dann wie in
Abschnitt~\ref{sem-adj} beschrieben: Da Relativsätze Adjunkte sind, werden sie in Kopf"=Adjunkt"=Strukturen
mit einer entsprechenden \nbar kombiniert. In (\mex{-1}) ist das das Nomen \emph{Kind}. Der \modw des Relativsatzes
wird mit der Kopf"|tochter identifiziert. Die Restriktionsliste von
\emph{Kind} \iboxb{2} wird, wie in (\mex{0}) zu sehen ist, mit der vom Relativsatz beigesteuerten vereinigt.
Da der semantische Beitrag in Kopf"=Adjunkt"=Strukturen von der Adjunkttochter übernommen wird,
enthält der Gesamtbeitrag dann \sliste{ \textrm{kind\iboxb{1}, lachen\iboxb{1}} }.

Die Analyse unseres Standardbeispiels in (\mex{1}) mit Modifikation der \nbar
zeigt Abbildung~\vref{abb-Kind-von-dessen-Schwester}.
\ea
Kind, von dessen Schwester Aicke ein Bild gemalt hat
\z
\begin{figure}
\centerline{%
\begin{forest}
sm edges
[\nbar{}\ind{1}
  [\ibox{2}\nbar{}\ind{1} [Kind]] 
  [{RS[\textsc{mod} \ibox{2}, \textsc{rel} \eliste, \textsc{slash} \eliste] }
    [{PP[\textsc{loc} \ibox{3}, \textsc{rel} \sliste{ \ibox{1} }]}
      [P [von] ]
      [{NP[\textsc{rel} \sliste{ \ibox{1} }]}
         [{Det[\textsc{rel} \sliste{ \ibox{1} }] } [dessen] ]
         [N [Schwester] ] ] ]
    [{S[\type{fin}, \textsc{slash} \sliste{ \ibox{3} }] } 
       [Aicke ein Bild \_ gemalt hat, roof] ] ] ]
\end{forest}}
\caption{\label{abb-Kind-von-dessen-Schwester}Perkolation und Abbindung der \textsc{rel}- und \slashwe}
\end{figure}


\noindent
Schema~\ref{rs-schema}\vpageref{rs-schema} enthält sowohl die bisher diskutierten syntaktischen als auch die semantischen
Beschränkungen.

\begin{figure}
\begin{schema}[Relativsatzschema]
\label{rs-schema}\is{Schema!Relativsatz-}
\medskip\samepage
\type{relative-clause} \impl\\
\oneline{%
\onems{ loc \ms{ cat & \ms{ head & \onems[relativizer]{ mod \textrm{$\overline{\mbox{\textrm{N}}}$:} \ms{ ind   & \ibox{1} \\
                                                                                           restr & \ibox{2}  \\
                                                                                         } \\
                                                              } \\
                                     comps & \liste{ } \\
                                     } \\
                           cont  & \ms{ ind   & \ibox{1}  \\
                                        restr & \sliste{ \ibox{3} } $\oplus$ \ibox{2} \\
                                     } \\
                              } \\
                 nonloc \ms{  rel   & \liste{} \\
                              slash & \liste{} \\
                           } \\
   nh-dtrs \liste{ \begin{tabular}{@{}l@{}}
                            \ms{ loc    & \ibox{4} \\
                                               nonloc & \ms{%  que   & \liste{} \\
                                                              rel   & \sliste{ \ibox{1} } \\
                                                              slash & \eliste \\
                                                           } \\
                                   }, %\\
                            \onems{ loc  \onems{ cat \onems{ head \ms[verb]{ initial & $-$ \\
                                                                                  vform   & fin \\
                                                                                } \\
                                                               comps \liste{} \\
                                                               % vcomp none is nicht nötig, ergibt sich aus anderen Schemata
                                                                     } \\
                                                            cont \ibox{3} \\
                                                          } \\
                                                 nonloc \ms{% que   & \liste{} \\
                                                             rel   & \liste{} \\
                                                             slash & \sliste{ \ibox{4} } \\
                                                             } \\
                                               } \\
                            \end{tabular}
                         } \\
}}
\end{schema}
\vspace{-\baselineskip}\end{figure}


%\noindent
\type{relative-clause} ist kein Untertyp von \type{headed-phrase},
das Valenzprinzip und das Semantikprinzip gelten deshalb nicht. Die entsprechende Information
ist im Schema daher explizit spezifiziert.
Einen Überblick über die aktuelle Typhierarchie unter \type{sign} gibt Abbildung~\vref{abb-sign-relcl}.
%
\begin{figure}
\centering
\scalebox{0.8}{%
\begin{sideways}%
\begin{forest}
type hierarchy,
 % for tree={
 %   calign=fixed angles,
 %   calign angle=50
 % } 
[phrase
    [non-headed-phrase, l sep*=2
      [relative-clause]]
    [headed-phrase, l sep*=2, for children={l sep*=4}
      [head-non-adjunct-phrase, name=non-adjunct, 
        [head-adjunct-phrase,no edge, name=adjunct]]
      [head-non-argument-phrase, name=non-arg
        [head-argument-phrase,no edge, name=argument]]
      [head-non-filler-phrase, name= non-filler
        [head-filler-phrase, no edge, name=filler]]
      [head-non-specifier-phrase, name=non-specifier
        [head-specifier-phrase, no edge, name=specifier]]]]
\draw (non-adjunct.south) to (argument.north);
\draw (non-adjunct.south) to (filler.north);
\draw (non-adjunct.south) to (specifier.north);
\draw (non-arg.south) to (adjunct.north);
\draw (non-arg.south) to (filler.north);
\draw (non-arg.south) to (specifier.north);
\draw (non-filler.south) to (adjunct.north);
\draw (non-filler.south) to (argument.north);
\draw (non-filler.south) to (specifier.north);
\draw (non-specifier.south) to (adjunct.north);
\draw (non-specifier.south) to (argument.north);
\draw (non-specifier.south) to (filler.north);
\end{forest}
\end{sideways}}
\caption{\label{abb-sign-relcl}Typhierarchie für \type{phrase}}
\end{figure}




\section{Alternativen}

In den beiden folgenden Abschnitten wird eine alternative HPSG"=Analyse und ein
Kategorialgrammatik"=Ansatz diskutiert.

\subsection{Eine konstruktionsbasierte Relativsatzanalyse}
\label{sec-konstruktionsbasierte-rs}


Die klassische Relativsatzanalyse in \citew[Kapitel~5]{ps2} geht davon aus, dass ein leerer
Komplementierer mit einer Relativphrase und einem Satz, aus dem die Relativphrase extrahiert wurde,
kombiniert wird. Für den Relativsatz \emph{to whom Kim gave a book} nehmen Pollard und Sag folgende Struktur an:
\ea
{}[\sub{RP} [\sub{PP} to whom]$_i$ [\sub{R'} e [\sub{S/PP} Kim gave a book \_$_i$]]]
\z
e ist dabei ein leerer Kopf der Kategorie Relativierer\is{Relativierer} (\emph{relativizer}). Er wird zuerst mit
\emph{Kim gave a book} kombiniert und bildet eine R'"=Projektion. Die R'"=Projektion selegiert die
Relativphrase und bildet mit ihr zusammen dann eine vollständige R"=Projektion. Diese Analyse
entspricht der \xbart: Es gibt einen Kopf in jeder 
der Teilstrukturen. Der Kopf selegiert entsprechende Argumente und bestimmt den Bedeutungsbeitrag
der Gesamtkonstruktion. Die im Abschnitt~\ref{sec-rs-anal} vorgestellte Analyse unterscheidet sich
nur darin in der von Pollard und Sag, dass der Effekt des leeren Kopfes in eine Grammatikregel
integriert wurde. \citet*[\page153, Lemma~4.1]{BHPS61a} haben gezeigt, dass man
Phrasenstrukturgrammatiken mit leeren Elementen in solche ohne leere Elemente umwandeln
kann. Dieselbe Technik, die sie vorgeschlagen haben, wurde auch hier für die Relativsatzanalyse
verwendet: Statt einen leeren Kopf mit zwei Argumenten zu kombinieren, verwendet man eine
Grammatikregel, die die beiden Elemente direkt kombiniert und das Ergebnis, das die Kombination des
leeren Kopfes mit seinen zwei Argumenten hätte, direkt am Mutterknoten repräsentiert.

Einen anderen Weg geht \citet{Sag97a}. Er schlägt zwar ebenfalls Grammatikregeln statt leerer Köpfe
für die Analyse von Relativsätzen vor, geht aber davon aus, dass in Relativsätzen das Verb der Kopf
ist. Aus dieser Annahme ergeben sich zwei Probleme: Das erste wurde bereits auf
Seite~\pageref{Erklaerung-relativizer} angesprochen. Wenn das Verb der Kopf des Relativsatzes ist,
dann muss das Verb einen spezifizierten \modw haben (können), und man muss sicherstellen, dass
Verbalprojektionen, die nicht in Relativsätzen vorkommen, nicht \nbar{}s modifizieren können. Das
zweite Problem besteht darin, dass man für die Kombination eines Relativsatzes mit einem \nbar eine
nominale Semantik braucht. Verbalprojektionen haben aber eine verbale Semantik. Sag löst das erste
Problem, indem er für den Typ \type{clause} die als Default\is{Default} spezifizierte Beschränkung
einführt, dass der \modw \type{none} ist (S.\,480). Der \modw von Verben ist damit nicht mehr eine
lexikalische Eigenschaft der Verben, sondern wird je nach Verwendungsweise des Verbs
festgelegt. Damit die erwünschten Resultate erzielt werden, muss man bei einem solchen Vorgehen also
mindestens zwei Typen von Sätzen unterscheiden: Relativsätze und Nicht"=Relativsätze. Bei
Relativsätzen wird der Default"=Wert überschrieben, bei Nicht"=Relativsätzen bleibt er erhalten. Sag geht von
mindestens vier Untertypen von \type{clause} aus: \type{decl-cl}, \type{inter-cl}, \type{imp-cl} und
\type{rel-cl}. 
%% Für diese Typen legt er die folgenden Beschränkungen fest:
%% \ea
%% \begin{tabular}[t]{@{}ll@{}}
%% Typ      & Beschränkung\\
%% \type{decl-cl}  & [ \textsc{content} \type{proposition}] \\
%% \type{inter-cl} & [ \textsc{content} \type{question}]    \\
%% \type{imp-cl}   & [ \textsc{content} \type{directive}]   \\
%% \type{rel-cl}   & \ms{ head    & \ms{ mc & -\\
%%                                inv & -\\
%%                                mod & \textrm{[}{\textsc{head} noun \textrm{]}}\\
%%                              } \\
%%                 content & proposition \\
%%               }\\
%% \end{tabular}
%% \z
%% Diese Typen haben gemeinsame Untertypen mit \type{head-adj-ph}, \type{hd-fill}, \type{hd-comp-ph},
%% \type{fin-hd-subj-ph}, \type{hd-spr-ph} und \type{non-hd-ph}, für die ähnliche Beschränkungen
%% gelten wie für die in diesem Buch verwendeten Typen \type{head-adjunct-phrase},
%% \type{head-argument-phrase}, \type{head-specifier-phrase} und
%% \type{head-filler-phrase}. Daraus ergibt sich folgendes Problem: Da \zb Deklarativsätze vom Typ
%% \type{hd-fill} sind, gilt für sie das Semantikprinzip, das dafür sorgt, dass der \contw des Satzes
%% mit dem \contw der Kopftochter übereinstimmt. Da der \contw des Verbs auch in Teilstrukturen für
%% VPen projiziert wird, ist der \contw des Deklarativsatzes identisch mit dem \contw des Verbs. Dieser
%% Typ ist aber ein Typ, der die Argumentrollen des Verbs einführt (\zb \type{geben}) und nicht ein Typ
%% wie \type{proposition}. \citet{SWB2003a} verwenden deshalb ein spezielles Merkmal innerhalb von
%% \cont, das sie \textsc{mode} nennen. Der Wert gibt Auskunft über den Satzmodus.

Die Lösung des zweiten Problems ist sehr unschön: Da die Relativsätze eine verbale Semantik haben,
muss die Verrechnung dieser Semantik mit der des modifizierten Nomens außerhalb des Relativsatzes
passieren. \citet[\page475]{Sag97a} schlägt deshalb ein spezielles Schema vor, das eine \nbar mit
einem Relativsatz kombiniert. Diese Analyse widerspricht den Grundannahmen in der HPSG, die auf
\citet{Saussure16a-de} zurückgehen: Sprachliche Zeichen sollen Form"=Bedeutungspaare sein. So ist ein
Relativsatz etwas, das einen bestimmten syntaktischen Aufbau hat, der einer Relativsatzbedeutung
entspricht. In der Sagschen Analyse ist das nicht widergespiegelt, denn dort hat der Relativsatz die
dem Verb entsprechende Bedeutung, die Relativsatzbedeutung kommt erst bei Kombination mit einem
Nomen hinzu.

Zusätzlich zur \nbar"=Relativsatz"=Spezialregel, die nur für die Analyse von
Nominalstrukturen mit Relativsatz gebraucht wird, gibt es noch das allgemeine
Kopf"=Adjunkt"=Schema (\type{simple-hd-adj-ph}). Die im Abschnitt~\ref{sec-rs-anal} vorgestellte
Analyse kommt dagegen mit einem sehr allgemeinen Kopf"=Adjunkt"=Schema aus.





\subsection{Das Relativpronomen als Kopf}


\citet{Steedman89a}\is{Pronomen!Relativ-|(} stellt in seinem Artikel die Kategorialgrammatik\is{Kategorialgrammatik (CG)|(} vor.
Er diskutiert englische Relativsätze wie (\mex{1}) und gibt den
Lexikoneintrag in (\mex{2}) für Relativpronomina wie \emph{which} und \emph{who}(\emph{m})
(S.\,217):\footnote{
  Siehe auch \citew[\page 614]{SB2006a-u}.
}
\ea
apples which Harry eats
\z
\ea
(N$\backslash$N)/(S/NP)
\z
Die Symbole `/' und `$\backslash$' stehen für Kombination nach links bzw.\
rechts, \dash, S/NP steht für etwas, das einen Satz ergibt, wenn es mit einer NP,
die sich rechts von ihm befindet, kombiniert wird. In (\mex{0}) steht
S/NP für \emph{Harry eats}, \dash die Verbprojektion, der ein Objekt fehlt.
Der Eintrag für das Relativpronomen bedeutet folgendes: Das Relativpronomen
verlangt rechts von sich eine Wortgruppe der Kategorie S/NP (den Satz mit einer
fehlenden NP), und wenn es mit dieser Wortgruppe kombiniert worden ist, ist das
Ergebnis N$\backslash$N. N$\backslash$N steht für eine Kategorie, die sich
nach links mit einem Nomen verbindet und ein Nomen ergibt. 
%X/X bzw.\ X$\backslash$X
%sind die allgemeinen Muster für Modifikatoren in der Kategorialgrammatik.
Der Aspekt der Analyse, der hier interessiert, ist, dass das Relativpronomen
die externen Eigenschaften des gesamten Relativsatzes bestimmt. Das Relativpronomen
ist ein Funktor, der einen Satz mit fehlendem Objekt selegiert und festlegt,
dass das Resultat der Kombination ein Nomen modifizieren kann.

Ähnliche Analysen wurden im Rahmen der HPSG für sogenannte freie Relativsätze\is{Relativsatz!freier}
vorgeschlagen. Freie Relativsätze sind Relativsätze, die im Satz ohne Bezugsnomen
stehen und direkt als Argument oder Adjunkt zu einem höheren Verb fungieren.
Der Satz in (\mex{1}) ist ein Beispiel für eine Konstruktion mit einem freien Relativsatz.
\ea
Wer die Hausaufgaben sofort abgibt, kann die Ferien genießen.
\z
In (\mex{0}) gibt es für den Relativsatz \emph{wer die Hausaufgaben sofort abgibt} kein
sichtbares Bezugswort. Man kann sich überlegen, dass der Satz einem Satz wie (\mex{1})
mit sichtbarem Bezugswort entspricht:
\ea
Jeder, der die Hausaufgaben sofort abgibt, kann die Ferien genießen.
\z
Eine Analyse, die für das Deutsche \citep{Kubota2002a} und das Englische \citep{WK03a}
% und das Französische 
vorgeschlagen wurde, geht davon aus, dass in Sätzen wie (\mex{-1}) das Relativpronomen der Kopf ist.
Damit wäre ohne weitere Annahmen \emph{wer die Hausaufgaben sofort abgibt} eine NP,
und es wäre erklärt, warum dieser Relativsatz an Stelle einer Argument"=NP stehen kann.

Das Problem, das alle Ansätze haben, die davon ausgehen, dass das Relativpronomen der
Kopf/""Funktor ist, ist jedoch, dass Relativpronomina tief eingebettet sein können.
(\mex{1}) zeigt englische\il{Englisch} Beispiele von \citet[\page212]{ps2} bzw.\ von \citet[\page 109]{Ross67}\nocite{Ross86a-u}:
\eal
\ex Here's the minister [[in [the middle [of [whose sermon]]]] the dog barked].
\ex Reports [the height of the lettering on the covers of which] the government prescribes should be abolished.
\zl
In (\mex{0}a) ist das Relativpronomen der Determinator von \emph{sermon}.
Je nach Analyse ist \emph{whose} der Kopf der Phrase \emph{whose sermon},
diese NP allerdings ist unter \emph{of} eingebettet und die Phrase \emph{of whose sermon}
ist von \emph{middle} abhängig. Die gesamte NP \emph{the middle of whose sermon}
ist ein Komplement der Präposition \emph{in}. Wollte man behaupten, dass \emph{whose}
der Kopf des Relativsatzes in (\mex{0}a) ist, müßte man schon einige Handstände (oder Kopfstände?)
machen. In (\mex{0}b) ist das Relativpronomen noch tiefer eingebettet.
In der Kategorialgrammatik gibt es neben der Funktionalapplikation, die
für die Kombination von Konstituenten verwendet wird, noch die Typanhebung\is{Typanhebung} von
Konstituenten. Mit solchen Typanhebungen ist es dann möglich, Wortfolgen als
Konstituenten zu analysieren, die normalerweise nicht als Konstituenten betrachtet
werden, aber zum Beispiel für die einfache Analyse von Koordinationsstrukturen\is{Koordination}
gebraucht werden \citep{Steedman89a}. \citet[\page 204]{Morrill95a} diskutiert für das
Relativpronomen im Relativsatz in (\mex{1}a) den Lexikoneintrag in (\mex{1}b):\footnote{
  Morrill verwendet eine andere Notation. Ich habe sie an die von Steedman angepaßt.
% Schreibt (PP/NP)$\backslash$(N$\backslash$N)/(S/NP), meint aber wohl
%          (PP/NP)$\backslash$(N$\backslash$N)/(S/PP)
}
\eal
\ex about which John talked
\ex (PP/NP)$\backslash$(N$\backslash$N)/(S/PP)
\zl
In diesem Lexikoneintrag verlangt das \emph{which} links von sich etwas, das eine Nominalphrase
braucht, um eine vollständige Präpositionalphrase zu ergeben, \dash, \emph{wich} selegiert die
Präposition. Morrill stellt fest, dass zusätzliche Einträge für Fälle angenommen werden müssen, in
denen das Relativpronomen in der Mitte der vorangestellten Phrase steht.
\ea
the contract the loss of which after so much wrangling John would finally have to pay for
\z
Morrill stellt fest, dass man alle Fälle durch zusätzliche lexikalische Stipulationen behandeln
kann. Er schlägt stattdessen zusätzliche Arten der Kombination von Funktoren und Argumenten vor,
die einen Funktor B $\uparrow$ A sein Argument A umschließen lassen und B ergeben bzw.\ einen Funktor A $\downarrow$
B in sein Argument A einfügen, um dann B zu ergeben (S.\,190). Mit diesen zusätzlichen Operationen
braucht er dann noch die beiden Lexikoneinträge in (\mex{1}), um die diversen
Rattenfängerkonstruktionen ableiten zu können:
\eal
\ex (NP $\uparrow$ NP) $\downarrow$ (N$\backslash$N)/(S/NP)
\ex (PP $\uparrow$ NP) $\downarrow$ (N$\backslash$N)/(S/PP)
\zl
Morrill ist es auf diese Weise gelungen, die Anzahl der Lexikoneinträge für \emph{which} zu
reduzieren, trotzdem bleibt der Fakt, dass er die Kategorien, die in Rattenfängerkonstruktionen
vorkommen können, im Lexikoneintrag des Relativpronomens erwähnen muss. 
%% Im Deutschen ist die
%% Rattenfängerkonstruktion bei Präpositionalphrasen, Adjektivphrasen und auch Verbphrasen möglich.
Auch geht die Einsicht, dass es sich bei Relativsätzen um eine Phrase mit einem Relativpronomen
+ Satz, dem die Relativphrase fehlt, handelt, in solchen Analysen verloren.

Für die Behandlung freier Relativsätze wie (\mex{1}) schlägt \citet[\page 164]{Kubota2002a} im
Rahmen der HPSG vor, das Relativpronomen als Kopf zu betrachten, wobei das Relativpronomen eine Präposition und einen finiten
Satz mit extrahierter PP selegiert.
\ea
Aus wem noch etwas herausgequetscht werden kann, ist sozial dazu verpflichtet, es abzuliefern; \ldots\footnote{
        Wiglaf Droste, taz, 01.08.1997, S.\,16.
      }
\z
Diese Analyse kann nicht erfassen, wieso das Relativpronomen den Kasus hat, den die Präposition
\emph{aus} verlangt (Dativ) und wieso der gesamte freie Relativsatz trotzdem den Platz des
Nominativarguments von \emph{verpflichtet} einnehmen kann. Man könnte die Analyse technisch retten,
wenn man annehmen würde, dass es sich bei \emph{wem} um ein Pronomen mit dem Kasus Nominativ handelt,
das eine Präposition selegiert, die ein Element im Dativ verlangt. Da es so etwas aber an keiner
anderen Stelle in der deutschen Grammatik gibt, ist eine solche Lösung abzulehnen. Alternative
Lösungen, die \emph{aus wem noch etwas herausgequetscht werden kann} als vollständigen Relativsatz
analysieren, der dann die Nominativstelle einnehmen kann, sind Kubotas Ansatz vorzuziehen. Siehe
\citew{Mueller99a,Mueller99b} für eine entsprechende Analyse im Rahmen der HPSG.%
\is{Relativsatz|)}
\is{Kategorialgrammatik (CG)|)}
\is{Pronomen!Relativ-|)}

%\section*{Kontrollfragen}

\questions{
\begin{enumerate}
\item Wie sind Relativsätze aufgebaut?
\end{enumerate}
}

%\section*{Übungsaufgaben}

\exercises{
\begin{enumerate}
\item Welche Relativpronomina kennen Sie?
\item Skizzieren Sie die syntaktsichen Aspekte der Analyse für die folgende Phrase:
\ea
die Blume, die allen gefällt
\z
Gehen Sie auf die \slashwe und die \relwe ein.

\item Laden Sie die zu diesem Kapitel gehörende Grammatik von der Grammix"=CD
(siehe Übung~\ref{uebung-grammix-kapitel4} auf Seite~\pageref{uebung-grammix-kapitel4}).
Im Fenster, in dem die Grammatik geladen wird, erscheint zum Schluß eine Liste von Beispielen.
Geben Sie diese Beispiele nach dem Prompt ein und wiederholen Sie die in diesem Kapitel besprochenen
Aspekte.

\end{enumerate}
}

\furtherreading{
\citet{Arnold:Godard:2021a} beschäftigen sich mit Analysen von verschiedenen Relativsatzkonstruktionen im
Englischen, Französischen, Japanischen und Koreanischen. Sie besprechen auch so genannte freie
Relativsätze. Zu diesen siehe auch \citew{Mueller99b}.
}