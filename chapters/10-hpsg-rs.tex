%% -*- coding:utf-8 -*-
%%%%%%%%%%%%%%%%%%%%%%%%%%%%%%%%%%%%%%%%%%%%%%%%%%%%%%%%%
%%   $RCSfile: hpsg-rs.tex,v $
%%  $Revision: 1.16 $
%%      $Date: 2008/09/30 09:14:41 $
%%     Author: Stefan Mueller (CL Uni-Bremen)
%%    Purpose: 
%%   Language: LaTeX
%%%%%%%%%%%%%%%%%%%%%%%%%%%%%%%%%%%%%%%%%%%%%%%%%%%%%%%%%

\chapter{Relativsätze}
\label{Kapitel-rs}
\is{Relativsatz|(}

{\exewidth{(9)}%
Nach der Behandlung der Vorfeldbesetzung im vorigen Kapitel wird jetzt
die Analyse der Relativsätze besprochen. Im Abschnitt~\ref{sec-rs-phenomena} werden die syntaktischen
Eigenschaften von Relativsätzen diskutiert, und in Abschnitt~\ref{sec-rs-anal}
wird die Analyse vorgestellt.

\section{Das Phänomen}
\label{sec-rs-phenomena}

Relativsätze bestehen aus einer Phrase, die ein Relativpronomen enthält, die ich im folgenden
\emph{Relativphrase}\is{Relativphrase} nennen werde, und einem sich daran anschließenden finiten Satz mit dem finiten Verb in 
Endstellung, aus dem die Relativphrase vorangestellt wurde. Beispiele verschiedenster Art zeigt (\mex{1}):
\eal
\label{bsp-relativsaetze}
\ex der Affe, [\emph{der}] Aicke kennt \label{r1}
\ex der Affe, [\emph{den}] Aicke kennt \label{r1b}
\ex der Affe, [\emph{dem}] Aicke Futter gibt \label{r1c}
\ex der Affe, [von \emph{dem}] Aicke gehört hat \label{r2}
\ex Die Stadt, [in \emph{der}] Karl arbeitet, ist attraktiv. \label{r3}
\ex Ich machte Änderungen, [\emph{deren}\iw{deren!Relativpronomen} Tragweite] mir nicht bewußt war.\label{r4}
\ex es hätte die FDP zerrissen und Kandidat Scharping das Signal gebracht, [\emph{dessen} entbehrend] er schließlich scheiterte.\footnote{
  taz, 20.10.1998, S.\,1.}
\ex Das ist ein Umstand, [\emph{den} zu berücksichtigen] meist vergessen wird.\label{r5}\footnote{
  \citet[\page79]{Bech55a} gibt ein nahezu identisches Beispiel.%
}
\ex Die Nato befindet sich in einem Zustand, [\emph{den} zu verhindern] sie eigentlich gegründet wurde.\footnote{
  Martin Schulze, Bericht aus Bonn, ARD, 23.04.1999.}
\zl
Die Relativphrase kann ein Subjekt, (Akk/Dat/Gen/PP) Objekt, Adjunkt oder VP"=Komplement sein.
\label{page-rattenfaenger}%
Die Relativphrase kann komplex sein (VP, PP, NP) oder nur aus einem Pronomen bestehen.
Bei komplexen NPen ist das Relativwort ein Possessivum\is{Pronomen!Possessiv-}. 
Beispiele wie (\mex{0}d--i) werden nach \citet*[\page 108]{Ross67} in Anlehnung an die Geschichte vom 
\href{http://www.hameln.de/tourismus/rattenfaenger/}{Rat\-ten\-fän\-ger von Hameln} 
auch Rattenfängerkonstruktionen\is{Rattenfängerkonstruktion} genannt,
weil das Relativpronomen das Material in der jeweils vorangestellten Konstituente "`mitzieht"', \dash
zusammen mit anderem Material vorangestellt wird.

Das Relativwort\is{Kongruenz!Relativwort"=Nomen} muss mit seinem Bezugsnomen in Numerus und Genus übereinstimmen. Die Kasus von
Bezugsnomen und Relativpronomen können allerdings verschieden sein. Der Kasus des Relativpronomens
wird nur von dessen jeweiligem Kopf, meist dem Verb im Relativsatz bestimmt.

In (\mex{0}) handelt es sich bei den Relativpronomina\is{Pronomen!Relativ-|(} um sogenannte \emph{d}"=Pronomina, aber
auch \emph{w}"=Pronomina kommen als Relativum vor, wie die Beispiele in (\mex{1}) zeigen.
\eal
\label{bsp-rs-w-pron}
% weiterführender RS
% \ex Ich komme eben aus der Stadt, [\emph{wo}] ich Zeuge eines Unglücks gewesen bin.\footnote{
% 	\citep*[\page672]{Duden84}
% 	}\label{bsp-wo-ich-zeuge}
\ex Studien haben gezeigt, daß mehr Unfälle in Städten passieren, [\hspace{-.2ex}\emph{wo}\iw{wo!Relativpronomen}] 
      die Zebrastreifen abgebaut werden, weil die Autofahrer unaufmerksam werden.\footnote{
        taz berlin, 03.11.1997, S.\,23.
        }
\ex Zufällig war ich in dem Augenblick zugegen, [\hspace{-.2ex}\emph{wo}] der Steppenwolf 
      zum erstenmal unser Haus betrat und bei meiner Tante sich einmietete.\footnote{
                Herman Hesse, \emph{Der Steppenwolf}. Berlin und Weimar: Auf"|bau-Verlag. 1986, S.\,6.
	}
\ex Tage, [an \emph{welchen}] selbst die Frage, ob es nicht an der Zeit sei, dem Beispiele
      Adalbert Stifters zu folgen und beim Rasieren zu verunglücken, ohne Auf"|regung
      oder Angstgefühle sachlich und ruhig erwogen wird,\footnote{
		Herman Hesse, \emph{Der Steppenwolf}. Berlin und Weimar: Auf"|bau-Verlag. 1986, S.\,27.
	}
\ex War das, [\hspace{-.2ex}\emph{worum}\iw{worum!Relativpronomen}] wir Narren uns mühten, schon immer vielleicht nur ein Phantom gewesen?\footnote{
		Herman Hesse, \emph{Der Steppenwolf}. Berlin und Weimar: Auf"|bau-Verlag. 1986, S.\,39.
	}

\ex Den Hass gegen Bill Gates' Firma schürten auch Meldungen, [\hspace{-.2ex}\emph{wonach}]\iw{wonach!Relativpronomen} die USA die Angriffssoftware
      taiwanischer Kampfjets auf Microsoft"=Basis "`ziel\-an\-ge\-paßt"' hätten.\footnote{
        taz, 12.01.2000, S.\,9.
      }
\ex Dort vielleicht war das, [\hspace{-.2ex}\emph{was}\iw{was!Relativpronomen}] ich begehrte, dort vielleicht würde meine Musik gespielt.\footnote{
                Herman Hesse, \emph{Der Steppenwolf}. Berlin und Weimar: Auf"|bau-Verlag. 1986, S.\,40.
	}\label{bsp-meine-musik}
\ex {}[\ldots], das ist nun wieder eine Frage, [über \emph{welche}\iw{welche}] müßige Leute nach Belieben brüten
	mögen.\footnote{
                Herman Hesse, \emph{Der Steppenwolf}. Berlin und Weimar: Auf"|bau-Verlag. 1986, Tractat vom Steppenwolf, S.\,6.
	} 
\ex {}[\ldots], heute gibt es nichts, [\hspace{-.2ex}\emph{was}] der kritischen Betrachtung wert wäre oder
      \mbox{[\hspace{-.2ex}\emph{worüber}]} sich auf"|zuregen lohnte.\footnote{
	taz, 14.11.1996, S.\,13.
      }
% GRI/SAG.00391 Die Langobarden und Aßipiter, S. 359
% Und um dies glaubhafter zu machen, stellten sie ihre Zelte weit auseinander und zündeten viele Feuer im Lager an. Die Aßipiter gerieten dadurch in Furcht und wagten nun den Krieg, womit sie gedroht hatten, nicht mehr zu führen.
\zl
\is{Pronomen!Relativ-|)}

\noindent
Es gibt verschiedene Arten von Relativsätzen: Sie können wie in (\mex{1}) Nomina modifizieren,
sie können sich als weiterführende Relativsätze\is{Relativsatz!weiterführender} auf einen ganzen Satz beziehen (\mex{2}), oder sie
können direkt als Argument (\mex{3}) oder Adjunkt (\mex{4}) eines Kopfes
frei\is{Relativsatz!freier} auf"|treten.
\ea
das Kind, [das im Zelt schläft]
\z

\ea
Conny hat die Schachpartie gewonnen, [was Aicke ärgerte].
\z                                              

\eal
\label{bsp-frei-rs-subj}
\ex {}[Wer das schriftliche Produkt eines Verwaltungsbeamten als "`mittleren Schwachsinn"' bezeichnet],
      muß mit 2.400 Mark Geldstrafe rechnen.\footnote{
        taz, 30.11.1995, S.\,20.}
\ex Macht kaputt, [was euch kaputtmacht]!\footnote{
        Ton, Steine, Scherben, \emph{Warum geht es mir so dreckig?}, erschienen bei Indigo, David Volksmund Prod.\ als LP und CD, 1971.
      }
\zl
\eal
\label{bsp-frei-rs-mod}
\ex {}[Wo das Rauchen derartig stigmatisiert ist wie von Köppl geplant], 
      kann man sich leicht als Rebell fühlen, bloß weil man raucht.\footnote{
	taz, 15.11.1996, S.\,10.
}\label{bsp-rauchen-stigmatisiert}
\ex {}[Wo noch bis zum Dezember vergangenen Jahres die "`Projekte am Kollwitzplatz"' und
      "`Netzwerk Spielkultur"' ihren Sitz hatten], prangt heute das Schild "`Zu vermieten"'.\footnote{
        taz berlin, 27.07.1997, S.\,23.
        }
\zl
}%exewidth

\noindent
Weiterführende und freie Relativsätze können hier nicht besprochen werden. Der interessierte Leser
sei auf \citew{Holler2003a} bzw.\ auf \citew{Bausewein90} und \citew{Mueller99a,Mueller99b} verwiesen.

Wie bereits im vorigen Kapitel gezeigt wurde (siehe Seite~\pageref{bsp-nla-rs}), 
kann man die Abfolgen in Relativsätzen nicht durch
eine lokale Umordnung der Relativphrase erklären. Beispiele wie die in (\mex{1}) zeigen
klar, dass hier eine Fernabhängigkeit vorliegt:
\ea
das Thema, [über das]$_i$ Aicke Kim gebeten\iw{bitten} hat, [\sub{VP} [einen Vortrag\iw{Vortrag} \_$_i$] zu halten],
\z
\ea
Wollen wir mal da hingehen, wo$_i$\iw{wo!Relativpronomen} Jochen gesagt hat, [dass es \_$_i$ so gut schmeckt]?
\z
Hier besteht eine Analogie zur Vorfeldbesetzung in (\mex{1}).
\eal
\ex Über dieses Thema hat Aicke Kim gebeten, einen Vortrag zu halten.
\ex Wo hat Jochen gesagt, dass es so gut schmeckt?
\zl



\section{Die Analyse}
\label{sec-rs-anal}

Dieser Abschnitt ist in zwei Teile geteilt: Zuerst beschäftigen wir uns
mit der Syntax der Relativsätze. Im Abschnitt~\ref{Abschnitt-rs-sem-Analyse} wird dann ihre Semantik
behandelt.

\subsection{Die Syntax von Relativsätzen}
\label{sec-syntax-rs}
\label{lp-rs}% Sections zusammengeschmissen

Nach der Lektüre des vorangegangenen Kapitels ist klar, wie Fernabhängigkeiten wie \zb die
Vorfeldbesetzung und die Voranstellung in Relativsätzen modelliert werden können. In Relativsätzen gibt
es jedoch noch eine zweite Fernabhängigkeit: Relativpronomina können tief in der Relativphrase
eingebettet sein. Die Information darüber, dass eine Phrase ein Relativpronomen enthält, muss
am obersten Knoten der betreffenden Phrase vorhanden sein, denn sonst wäre es nicht möglich zu erklären,
wieso die Sätze in (\mex{1}b,c) keine Relativsätze sind und höchstens als Einschübe zu verstehen sind.
\eal
\ex[]{
Das Kind, das spielt, lacht.
}
\ex[*]{
Das Kind, es spielt, lacht.
}
\ex[*]{
Das Kind, das Kind spielt, lacht.
}
\zl
In (\mex{0}b) und (\mex{0}c) steht \emph{es} bzw.\ \emph{das Kind} an der Stelle des
Relativpronomens, und weder \emph{es} noch \emph{das Kind} enthält ein Relativpronomen.

Außerdem muss man die Übereinstimmung des Relativpronomens mit seinem Bezugswort in Numerus und Genus sicherstellen.
\eal
\ex Der Roman$_i$, [von \emph{dem}$_i$] Aicke begeistert war, wird nicht mehr verkauft.
\ex Die Stadt$_i$, [in \emph{der}$_i$] Karl arbeitet, ist attraktiv.
\ex Änderungen$_i$, [\emph{deren}$_i$ Tragweite] mir nicht bewusst war
\ex das Signal$_i$, [\emph{dessen}$_i$ entbehrend] er schließlich scheiterte
\ex ein Umstand$_i$, [\emph{den}$_i$ zu berücksichtigen] meist vergessen wird
\ex Umstände$_i$, [\emph{die}$_i$ zu berücksichtigen] meist vergessen wird
\zl
Die Kongruenz und Koreferenz von Bezugsnomen und Relativpronomen lässt sich einfach durch eine Koindizierung\is{Koindizierung}, 
eine Strukturteilung der \textsc{index}"=Werte ausdrücken. Dazu wird der referentielle Index
des Relativwortes nach oben gereicht, so wie das für die Vorfeldbesetzung mit
\textsc{local}"=Werten gemacht wurde (vergleiche die Extraktionsspur in (\ref{le-Extraktionsspur}) auf Seite~\pageref{le-Extraktionsspur}).
Die entsprechende Fernabhängigkeit beginnt jedoch in einem phonologisch gefüllten Lexikoneintrag:
\ea\is{Pronomen!Relativ-}
\ms[word]{ phon & \phonliste{ dem } \\
    loc & \ms{ cat & \ms{ head & \ms[noun]{
                                                  cas & dat\/ \\ 
                                                } \\
                                         spr    & \liste{} \\
                                         comps & \liste{} \\
                                       } \\
                               cont & \ms{ ind & \ibox{1} \ms{ per & 3 \\
                                                               num & sg \\
                                                               gen & mas $\vee$ neu \\
                                                             } \\
                                         } \\
                             } \\
                     nonloc & \ms{ %que   & \liste{} \\
                                                rel   & \sliste{ [ind \ibox{1} ] } \\
                                                slash & \liste{} \\ 
                                                %extra & \liste{} \\
                                  } \\
   }
\z
In der Extraktionsspur wird der \localw mit dem \slashel identifiziert,
in den Einträgen für Relativpronomina gibt es eine Strukturteilung zwischen dem Index des Pronomens
und dem Index des Elements in der \rell.\footnote{%
  Im Abschnitt~\ref{Abschnitt-Semantik-possessive-Relativpronomina} wurde erklärt, warum sowohl der
  Index als auch der \ltopw nach oben gereicht werden müssen.
}

Abbildung~\vref{abb-rel-percolation} zeigt das Weiterreichen der Information über
den referentiellen Index eines Relativpronomens in einer komplexen Präpositionalphrase.

\begin{figure}
\centering
\begin{forest}
sm edges
[{PP[\textsc{rel} \sliste{ \ibox{1} }]}
  [P [von]]
  [{NP[\textsc{rel} \sliste{ \ibox{1} }]}
    [{Det[\textsc{rel} \sliste{ \ibox{1} }] } [dessen] ]
    [N [Fahrrad] ] ] ]
\end{forest}
\caption{\label{abb-rel-percolation}Weiterreichen des \textsc{rel}-Wertes in Relativphrasen}
\end{figure}

Bei der Analyse des Relativsatzes in (\mex{1}) muss die Fernabhängigkeit, die
im finiten Satz beginnt, abgebunden werden.
\ea
Kind, [von dessen Fahrrad]$_i$ [sie ein Bild \_$_i$ gemacht hat]
\z
Der Füller der Fernabhängigkeit ist die Relativphrase. Da der gesamte Relativsatz
Eigenschaften hat, die nicht mit denen von finiten Sätzen kompatibel sind, kann
das finite Verb nicht der Kopf des Relativsatzes sein (mehr dazu im nächsten Abschnitt).
Ich gehe deshalb davon aus, dass es in Relativsätzen keine Kopf"|tochter gibt. Beide Töchter sind
einfach Elemente der \dtrsl. Abbildung~\vref{abb-von dessen Fahrrad-syn}
zeigt die syntaktischen Aspekte der Analyse im Überblick.
\begin{figure}
\centering
\begin{forest}
sm edges
[{RS[\textsc{rel} \eliste, \textsc{slash} \eliste] }
   [{PP[\textsc{loc} \ibox{2}, \textsc{rel} \sliste{ \ibox{1} }]}
     [P [von] ]
     [{NP[\textsc{rel} \sliste{ \ibox{1} }]}
        [{Det[\textsc{rel} \sliste{ \ibox{1} }] } [dessen] ]
        [N [Fahrrad] ] ] ]
   [{S[\type{fin}, \textsc{slash} \sliste{ \ibox{2} }] } 
      [sie ein Bild \_ gemacht hat, roof] ] ]
\end{forest}
\caption{Perkolation und Abbindung der \textsc{rel}- und \slashwe}\label{abb-von dessen Fahrrad-syn}
\end{figure}
Der \relw wird vom Relativpronomen bis zum obersten Knoten der Relativphrase hochgereicht. Im Relativsatzschema\is{Schema!Relativsatz-}
wird überprüft, ob die erste Tochter einen gefüllten \relw enthält, \dash ob es in der ersten
Phrase ein Relativpronomen gibt. Der \relw der Mutter ist die leere Liste, da der zum Relativpronomen gehörige
Index nicht aus dem Relativsatz hinaus weitergereicht wird. Die Abbindung des \textsc{slash}"=Wertes aus dem
finiten Satz funktioniert parallel zur Vorfeldbesetzung: Der \slashw des Satzes wird mit dem \localw
der Relativphrase identifiziert, der \slashw des Mutterknotens ist die leere Liste, da die Fernabhängigkeit
durch die Relativphrase abgebunden wird. Schema~\vref{rs-schema-struk} zeigt einen Auszug des Schemas, das die Struktur in 
Abbildung~\ref{abb-von dessen Fahrrad-syn} lizenziert.
%\begin{figure}
\begin{samepage}
\begin{schema}[Relativsatzschema (strukturelle Aspekte)]
\label{rs-schema-struk}
\type{relative-clause}\istype{relative"=clause} \impl\\
%\resizebox{\linewidth}{!}{
\onems{ loc$|$cat \ms{ head & relativizer \\
                       spr  & \eliste\\
                       comps & \liste{ } \\
                     } \\
                 nonloc \ms{  rel   & \liste{} \\
                              slash & \liste{} \\
                           } \\[-5\baselineskip]
   dtrs \liste{ \begin{tabular}{@{}l@{}}
                            \ms{ loc    & \ibox{1} \\
                                               nonloc & \ms{%  que   & \liste{} \\
                                                              rel   & \sliste{ [] } \\
                                                              slash & \eliste \\
                                                           } \\
                                   }, %\\
                            \onems{ loc$|$cat \onems{ head \ms[verb]{ initial & $-$ \\
                                                                      vform   & fin \\
                                                                      dsl     & none\\
                                                                                } \\
                                                               comps \liste{} \\
                                                                     } \\
                                                 nonloc \ms
\end{schema}
\end{samepage}
%\vspace{-\baselineskip}\end{figure}
%
Die erste Tochter muss ein Element in \rel enthalten. Der \localw der ersten Tochter \iboxb{1}
entspricht dem \slashw der zweiten Tochter, einer 
vollständig gesättigten (\comps \eliste) Projektion eines finiten Verbs in Letztstellung. Der \dslw
\emph{none} schließt die Verwendung der Verbspur in Relativsätzen aus. Ohne diese Restriktion würde
die Grammatik die ungrammatische Folge in (\mex{1}a) lizenzieren, die aus (\mex{1}b) entsteht, wenn
man statt des Verbs eine Verbspur einsetzt:
\eal
\ex[*]{
Der Affe, den Aicke \_, schmatzt.
}
\ex[]{
Der Affe, den Aicke kennt, schmatzt.
}
\zl

\noindent
Die Relativphrase bindet die Fernabhängigkeit aus dem finiten Satz in Verbletztstellung ab,
weshalb der \slashw der Mutter die leere Liste ist. Genauso wird die Information über das
Relativpronomen innerhalb des Relativsatzes abgebunden: Der \relw der Gesamtstruktur ist die leere Liste. 

Das Relativsatzschema entspricht im Wesentlichen
dem Filler"=Head"=Schema, das auf S.\,\pageref{hf-schemaa} vorgestellt wurde. Es unterscheidet sich lediglich im
\relw der Füller-Tochter, in der Kategorie der Mutter und im \iniw der zweiten Tochter: Während V2"=Sätze mit einer
Verbprojektion in Initialstellung (\textsc{ini}$+$) gebildet werden, ist die zweite Tochter im
Relativsatz eine Verbalprojektion mit Verb in Letztstellung (\textsc{ini}$-$). Diese Ähnlichkeit der
beiden Satztypen kann man durch einen gemeinsamen Supertyp für Relativsätze und Verbzweitsätze mit
Verb in Initialstellung erfassen.


Das Ergebnis der Kombination von Relativphrase und Verbletztsatz mit entsprechender Lücke ist ein Relativsatz und hat deshalb den Kopfwert
\type{relativizer}.\label{Erklaerung-relativizer} Relativsätze unterscheiden sich von normalen Sätzen dadurch, dass sie Nomina
modifizieren können, \dash, sie haben einen gefüllten \modw. Würde man Relativsätze einfach als eine
Projektion des finiten Verbs (mit entsprechendem \modw beim Verb) analysieren, müssten finite Verben,
auch wenn sie nicht in Relativsätzen verwendet werden, Nomina modifizieren können. (\mex{1}) sollte
also grammatisch sein, was aber nicht der Fall ist.
\ea[*]{
der Affe, Aicke den Affen kennt
}
\z
Die Modifikation setzt die Relativsatzsyntax voraus, \dash, es muss eine Relativphrase geben und einen
finiten Satz mit Verbletztstellung, aus dem diese Relativphrase vorangestellt wurde. Einfache
Projektionen von Verben wie \emph{Aicke den Affen kennt} können nicht modifizieren.


%\subsection{Linearisierungsregeln für Relativsätze}
%\label{lp-rs}

Bisher\is{Linearisierung!-sregel|(} wurden noch keine Beschränkungen
für die Abfolge der Nicht"=Kopf"|töchter in Relativsätzen angegeben. Natürlich muss die das
Relativpronomen enthaltende Phrase immer vor dem finiten Satz stehen, aus dem sie bewegt wurde. Eine
andere Abfolge ist ungrammatisch, wie das folgende Beispiel zeigt:
\ea[*]{
\label{bsp-lp-rs}
Affe, [\sub{S} Aicke \_$_i$ kennt] den$_i$
}
\z
Außerdem kann die Relativphrase nicht innerhalb einer komplexen
Nominalphrase rechts des Kopfnomens stehen:\footnote{
  Hierin unterscheiden sich Relativsätze von Interrogativsätzen\is{Interrogativsatz},
  denn bei letzteren sind solche Abfolgen möglich:
\ea
Viele Angehörige wußten nicht, an Bord welcher Maschine ihre Angehörigen waren.
  (Tagesschau, 03.01.2004, 20:00)
\z
}
\eal
\ex[*]{
Das Kind, [ein Bild von dessen Fahrrad] Aicke macht, schläft.
}
\ex[*]{
         der Vortrag, die Verfasserin dessen uns sehr attraktiv erscheint\footnote{
           \citew[\page211]{Fanselow87a}.
         }
}
\zl
In (\mex{0}a) gibt es eine nach vorn bewegte Konstituente, die ein Relativpronomen enthält 
(\emph{ein Bild von dessen Fahrrad}), und dennoch ist der Satz ungrammatisch.
Konstituenten, die ein Relativpronomen enthalten, stehen immer vor anderen Konstituenten.
Eine Ausnahme bilden Präpositionen\is{Präposition} 
und koordinierende Konjunktionen.\is{Koordination}
\eal
\ex der Stuhl, [auf dem] Aicke sitzt
\ex der Moment, [auf den] ich gewartet habe
\zl
\ea
das Kind, [dessen Schwester [und dessen Bruder]] ich kenne,
\z
Das wird von der folgenden Linearisierungsregel korrekt erfasst.\footnote{
        \citet*{Riemsdijk85} 
        formuliert eine ähnliche Regel. Zu einer Präzisierung dieser
        Regel in Bezug auf Koordinationsstrukturen siehe \citew[\page151]{Mueller99a}.%
}
%% \footnote{
%%         \citet*[\page150]{Wunderlich80}\ia{Wunderlich|fn{\thefootnote}} 
%%         gibt folgendes Beispiel, das der LP-Regel (\ref{lp-rel}) widerspricht.
%%         \ea
%%         ungefähr zu der Zeit, drei Stunden vor der wir im Kino gewesen sind,
%%         \z
%%         Ich finde diesen Satz nicht akzeptabel. Allerdings gibt es Sätze wie (ii),
%%         die Gegenbeispiele zu sein scheinen.
%%         \eal
%%         \ex Die Dativ-NP bezeichnet die Größe, in bezug auf welche der Vergleich
%%         gilt, \ldots{} (Im Haupttext von \citep*[\page54]{Wegener85b}) %auch S. 56
%%         \ex \ldots{} eröffnet eine Möglichkeit, in bezug auf die das Repräsentationsformat
%%               den Rahmen des $\lambda$"=Kalküls verläßt. (Im Haupttext von \citep*[\page54]{Kaufmann95a})
%% %auch S. 142, 191
%%         \ex Wie oben aber schon angesprochen, gibt es andere Optionen in der
%%         Universalgrammatik, die das Auftreten von expletiven Elementen unnötig machen können,
%%         etwa den PRO-DROP-Parameter, mit Hilfe dessen qua Inversion die Subjekts-NP in den Rektionsbereich
%%         des Verbs gelangt. (Im Haupttext von \citep[\page216]{Fanselow87a})
%%         \ex Wir haben in diesem Fall also zwei regierende Kategorien anzunehmen,
%%         relativ zu denen jeweils Pronominalisierung bzw.\ Refelxivierung mit den Mitteln
%%         der Bindungstheorie erklärt ist. (Im Haupttext von \citep[\page187]{Grewendorf83a})
%%         \ex Es gibt jedoch ein syntaktisches Kriterium, auf Grund dessen die
%%         beiden Sätze unterschieden werden können. (Im Haupttext von \citew[\page41]{Steinitz69a})
%%         \zl
%%         Eventuell kann man \emph{in Bezug auf} als feste Wendung, die
%%         Wortcharakter hat, einordnen. Bei Phrasen wie \emph{mit Hilfe} ist ja eine Zusammenschreibung
%%         und eine Einordnung als Präposition möglich.
%%
%%         Fanselow macht sich übrigens auf S.\,211 Gedanken über \textit{Pied Piping}
%%         und gibt die (ii.b) ziemlich ähnliche Phrase (iii) an.
%%         \ea[*]{
%%         der Vortrag, die Verfasserin dessen uns sehr attraktiv erscheint
%%         }
%%         \zlast
%% }
\ea
\label{lp-rel}
{}[\textsc{rel} \sliste{ [] }] $< \neg$ P
\z
Diese Regel besagt, dass jede Tochter eines Zeichens mit nicht-leerer 
\textsc{rel}"=Liste vor allen anderen Töchtern steht, wobei Präpositionen ausgenommen sind. 

Man beachte, dass die LP"=Regel in (\ref{lp-rel}) für das
Englische\il{Englisch} nicht gilt.
\ea
Here's the minister [[in [the middle [of [whose seremon]]]] 
the dog barked].\footnote{ 
	\citew*[\page212]{ps2}.
}
\z
Diesen Satz kann man nicht mit (\mex{1}) übersetzen.\NOTE{JB: find ich aber fast okay}
\ea[*]{
Das ist der Pfarrer, in der Mitte von dessen Predigt der Hund bellte.
}
\z
Eine Übersetzung wie (\mex{1}) wäre wohl angebrachter.
\ea
Das ist der Pfarrer, der die Predigt hielt, in deren\iw{deren!Relativpronomen} Mitte der Hund bellte.
\z
\is{Linearisierung!-sregel|)}


\subsection{Die Semantik von Relativsätzen}
\label{Abschnitt-rs-sem-Analyse}

Die hier betrachteten Relativsätze (\mex{1}a) verhalten sich wie pränominale
Adjektive (\mex{1}b) oder PPen, die Nomina modifizieren (\mex{1}c).
\eal
\ex der Roman, den alle kennen
\ex der interessante Roman
\ex der Roman aus Japan
\zl
Genau wie diese selegieren sie ein \nbar über das \textsc{mod}"=Merkmal und teilen den \ltopw ihrer
Hauptrelation mit dem der modifizierten \nbar (siehe Abschnitt~\ref{Abschnitt-Semantik-Adjektivmodifikation-mutmaßlich}).
Das heißt, die Relativsätze verhalten sich anders als normale finite Sätze,
denn normale Verbletztsätze können keine Nomina modifizieren:
\eal
\ex[*]{
der Roman, [den Autor alle kennen]
}
\ex[*]{
der Roman, [alle dessen Autor kennen]
}
\zl
\itdopt{doppelt?}
Die Modifikation ist nur genau dann möglich, wenn die spezielle Relativsatzsyntax vorliegt.
Enthält der Satz kein Relativpronomen oder ist dieses irgendwo anders als in der ersten
Phrase, werden die Äußerungen ungrammatisch.

Es gibt mehrere Möglichkeiten, mit dieser Situation umzugehen: Man kann
einen phonologisch leeren Kopf annehmen, der einen Satz als Komplement und eine Relativphrase als Spezifikator nimmt 
und selbst ein Modifikator ist, der dann nach der Kombination mit dem Satz und der Relativphrase
die \nbar modifizieren kann. Diesen Weg sind \citet[Kapitel~5]{ps2} gegangen.\footnote{
  Diese Analyse für das Deutsche findet man neben der hier vorgestellten Variante noch mit Situationssemantik
  in Müller (\citeyear[Kapitel~10.3]{Mueller99a}, \citeyear[Abschnitt~2.7]{Mueller99b}).
  \citet{Sag97a} schlägt eine weitere Möglichkeit vor: Der \modw von Verben ist
  bei ihm unterspezifiziert. Verben können also in Relativsatzkonstruktionen
  als Kopf auf"|treten. Fälle wie (\mex{0}) muss er anders ausschließen. Siehe Abschnitt~\ref{sec-konstruktionsbasierte-rs}.%
}
Eine Alternative ist, ein Schema zu verwenden, das den Satz mit einer Relativphrase zu einem
Modifikator kombiniert. In einem solchen Schema gibt es dann keinen
Kopf. Der leere Kopf ist quasi direkt in die Regel integriert. Dieser Ansatz entspricht dem aus Abschnitt~\ref{sec-syntax-rs}.
Das folgende Schema zeigt einige semantischen Aspekte des Relativsatzschemas:


\begin{schema}[Relativsatzschema (semantische Aspekte und Kongruenz)]
\label{rs-schema-sem}\is{Schema!Relativsatz-}
\type{relative-clause} \impl\\
%\resizebox{\linewidth}{!}
\end{schema}

\noindent
% Dabei gleicht das, was unter \textsc{loc} steht, dem, was wir schon im Abschnitt~\ref{sem-adj}
% kennengelernt haben. Man vergleiche den \locw mit dem Lexikoneintrag für das
% Adjektiv \emph{interessantes} in (\ref{le-interessantes-sem}) auf
% Seite~\pageref{le-interessantes-sem}. Der \locw von \emph{interessantes} ist der Übersichtlichkeit
% halber auch hier in (\mex{1}) angegeben:
%
Bei Relativsätzen müssen die Numerus"= und Genuswerte des modifizierten Nomens mit denen des Relativpronomens
übereinstimmen. Die entsprechende Information wird im Baum vom Relativpronomen nach oben gereicht, und durch die Strukturteilung
des Indexes innerhalb des \textsc{rel}"=Wertes (\iboxt{1} im Schema~\ref{rs-schema-sem}) mit dem \textsc{ind}"=Wert der
modifizierten \nbar wird dann sichergestellt, dass die modifizierte \nbar einen referentiellen Index
hat, der dem des Relativpronomens entspricht. Dadurch dass der semantische 
Index des modifizierten Nomens \iboxb{1} mit dem semantischen Index der Relativphrase identifiziert wird,
ist sichergestellt, dass das Relativpronomen mit dem Nomen, auf das sich der Relativsatz bezieht, in
Numerus und Genus kongruiert. Der \ltopw des Satzteils des Relativsatzes wird mit dem \ltopw des
gesamten Relativsatzes \iboxb{2} geteilt. Da der \ltopw zwischen Kopftochter und Adjunkttochter bei
nicht skopalen Modifikatoren geteilt wird
(S.\,\pageref{Semantikprinzip-nicht-skopaleTochter-LTOP}), haben die Hauptprädikation im Satz und
das modifizierte Nomen dieselbe Handle. Der \indw des Relativsatzes ist ebenfalls Teil von \cont und
mit der Ereignisvariable des finiten Verbs identisch. Der Wert ist Prinzip egal, weil der
Relativsatz eine Maximalprojektion ist, die nicht erweitert und auch nicht 
als Argument genommen wird. Der Wert ist dennoch im Relativsatzschema festgelegt, damit die
Parallelität zu anderen Strukturen gewährleistet ist.

Adjektive wie \emph{interessantes} steuern in ihrem Lexikoneintrag eine Relation zur Gesamtbedeutung der Phrase,
in der sie dann verwendet werden, bei. In Relativsätzen entspricht diese Relation der Relation, die
vom Verb im Relativsatz kommt. Diese wird als Element von \rels nach oben gereicht. Da der \ltopw
der Satztochter mit dem \ltopw des modifizierten Nomens identifiziert wird, sind Nomen und
Hauptrelation des Relativsatzes miteinander verknüpft. Für einen einfachen Relativsatz wie den in
(\mex{1}) bekommt man somit eine Struktur wie (\mex{2}):
\ea
Das Kind, das spielt, lacht.
\z
\ea
\ms{ phon & \phonliste{ das, spielt }\\
   loc & \ms{ cat & \ms{ head & \ms[relativizer]
                                { %prd & $-$ \\
                                   mod &  \upshape \nbar\ind{1} \\
                                } \\
                         spr   & \eliste\\
                         comps & \eliste\\
                        } \\
              cont & \ms{ ind & \ibox{3} \\
                          ltop & \ibox{2}\\
                      } } \\
   nonloc & \ms{ rels & \eliste\\
                 slash & \eliste}\\
rels & \liste{\ms[spielen]{ 
                lbl  & \ibox{2} \\
                arg0 & \ibox{3} event\\
                arg1 & \ibox{1} \\ 
                }}   \\
}
\z
Die Bedeutungskombination der gesamten Nominalphrase in (\mex{-1}) erfolgt dann wie in
Abschnitt~\ref{Semantik-Adjektivmodifikation} bzw.~\ref{Abschnitt-Semantik-Adjektivmodifikation-mutmaßlich} beschrieben: Da Relativsätze Adjunkte sind, werden sie in Kopf"=Adjunkt"=Strukturen
mit einer entsprechenden \nbar kombiniert. In (\mex{-1}) ist das das Nomen \emph{Kind}. Der \modw des Relativsatzes
wird mit der Kopf"|tochter identifiziert. Dadurch wird der Index des modifizierten Nomens mit dem
des Relativwortes geteilt. Der \ltopw des modifizierten Nomens wird mit dem des
Relativsatzes identifiziert, da nicht-skopale Modifikation vorliegt. Es ergibt sich also folgende MRS für (\mex{-1}):
\ea
\textmrs{ h0, e2, \{ h1:def\_q(x, h2, h3), h4:kind(x), h4:spielen(e1, x), h5:lachen(e2, x) \},\\
\hphantom{\textlangle~h0, e2,~}\{ h2 \qeq h4 \}  }
\z
Abbildung~\ref{Abbildung-Das Kind, das spielt, lacht} zeigt das im Überblick.
\begin{figure}
\oneline{%
\begin{forest}
sm edges, for tree={l sep+=4ex}
[V\ms{ rel   & \eliste\\
       slash & \eliste\\
       ltop  & h5\\
       rels  & \nliste{ h1:def\_q(x, h2, h3), h4:k(x), h4:sp(e1, x), h5:l(e2, x) }\\
       hcons & \nliste{ h2 \qeq h4 }}, s sep+=2em
  [NP\ibox{1}\ms{ rel   & \eliste\\
                  slash & \eliste\\
                  ltop  & h4\\
                  rels  & \nliste{ h1:def\_q(x, h2, h3), h4:k(x), h4:sp(e1, x) }\\
                  hcons & \nliste{ h2 \qeq h4 } }
    [Det\ms{ rel   & \eliste\\
             slash & \eliste\\
             rels  & \nliste{ h1:def\_q(x, h2, h3) } \\
             hcons & \nliste{ h2 \qeq h4 } } [das]]
    [N\ms{ rel & \eliste\\
           slash & \eliste\\
           ltop  & h4\\
           rels  & \nliste{ h4:k(x), h4:sp(e1, x) }\\
           hcons & \eliste\\} 
      [\ibox{2}\,N\ms{ rel   & \eliste\\
             slash & \eliste\\
             ltop  & h4\\
             rels  & \nliste{ h4:k(x), h4:sp(e1, x) }\\
             hcons & \eliste\\} [Kind]]
      [RC \ms{ mod   & \ibox{2}\\
               rel   & \eliste\\
               slash & \eliste\\
               ltop  & h4 = h6\\
               rels  & \nliste{ h6:sp(e, x) }\\
               hcons & \eliste\\}
        [NP\ibox{3}\ms{ rel   & \nliste{ [\textsc{ind} x] }\\
                        slash & \eliste\\
                        rels  & \eliste\\
                        hcons & \eliste\\} [das]]
        [V\ms{ rel   & \eliste\\
               slash & \nliste{ \ibox{3} }\\
               ltop  & h6\\
               rels  & \nliste{ h6:sp(e1, x) }\\
               hcons & \eliste\\}
          [NP\ibox{3}\ms{ rel   & \eliste\\
                          slash & \nliste{ \ibox{3} }\\
                          rels  & \eliste\\
                          hcons & \eliste\\}P [\trace]]
          [V\ms{ rel   & \eliste\\
                 slash & \eliste\\
                 ltop  & h6\\
                 rels  & \nliste{ h6:sp(e1, x) }\\
                 hcons & \eliste\\} [spielt]]]]]]
  [V\ms{ rel   & \eliste\\
         slash & \nliste{ \ibox{1} }\\
         ltop  & h5\\
         rels  & \nliste{ h5:l(e2, x) }\\
         hcons & \eliste\\} [lacht,roof]]]
\end{forest}}
\caption{Analyse von \emph{Das Kind, das spielt, lacht.}}\label{Abbildung-Das Kind, das spielt, lacht}
\end{figure}
In der Abbildung habe ich die Handels, die von verschiedenen Relationen kommen, erst einmal nicht
identifiziert. h4 und h6 sind also verschieden und ich habe die durch das Semantikprinzip (siehe
S.\,\pageref{ex-scopal-ltop}) erzwungene Identifikation von \nbar"=Handle und Relativsatz"=Handle durch
h4 = h6 ausgedrückt. Weiter oben im Baum haben sowohl \relation{kind} als auch \relation{spielen} das Handle h4.

Damit ist die Relativsatzanalyse fast vollständig erklärt. Es bleibt ein interessantes Problem: das
Problem mit possessiven Relativpronomen. 

Die MRS für unser komplexestes Beispiel in (\mex{1}a) zeigt (\mex{1}b):
\eal
\ex\label{ex Kind, von dessen Fahrrad sie ein Bild macht}
Kind, von dessen Fahrrad sie ein Bild macht
\ex \textmrs{ h0, x, \{ h1:kind(x), h2:def\_q(y, h3, h4), h1:besitzen(e1, x, y), h5:fahrrad(y),\\
    \hphantom{\textlangle~h0, x,~\}~}h6:exist\_q(z, h7, h8), h9:bild(z, y), h1:machen(e2, i1, z) \},\\
\hphantom{\textlangle~h0, x,~}\{ h3 \qeq h5, h7 \qeq h9  \}  }
\zl
Im Kapitel~\ref{sec-Wohlgeformtheitsbedingungen für MRSen} wurde bereits erklärt, warum die
Possessivrelation \relation{besitzen} dasselbe Label haben muss, wie der restliche Satz (\emph{sie
  ein Bild macht}) und nicht etwa dasselbe Label wie die vom Nomen eingeführte Relation
(\relation{fahrrad}), zu dem das possessive Relativpronomen gehört. Würde man das possessive
Relativpronomen parallel zum Possessivpronomen auf S.~\pageref{le-Possessivpronomen} behandeln, so
würde die \relation{besitzen}-Relation mit dem Nomen \emph{Fahrrad} das 
Handle teilen, was zu einer falschen Semantik führen würde. Das Problem kann man lösen, indem man
den \ltopw des possessiven Relativpronomens gemeinsam mit dem Index nach oben gibt und dann im
Relativsatzschema mit dem \ltopw des Satzes, aus dem die Relativphrase extrahiert wurde,
identifiziert. Diese Analyse habe ich in der ERG-Grammatik für das Englische \citep{CF2000a-u} gefunden und für das
Deutsche übernommen.\footnote{%
Die ERG-Grammatik unterscheidet sich von der hier vorgestellten Grammatik dadurch, dass \rels Teil
des \contwes ist. Zum \contw siehe Abschnitt~\ref{Abschnitt MRS und Fernabhängigkeiten}.
}
Der Lexikoneintrag für das possessive Relativpronomen \emph{dessen} hat die Form in (\mex{1}):
\ea
\label{le-dessen}%
Lexikoneintrag für das possessive Relativpronomen \emph{dessen}:\\
% dessen Frau / Mann / Kind
% dessen Affe stinkt
% dessen Affe wir gedenken
% dessen Affen wir helfen
% dessen Affen wir kennen
\ms{ 
  loc & \ms{ cat & \ms{ head  & \onems[det]{
                                spec|loc|cont \ms{ ind  & \ibox{1}\\
                                                   ltop & \ibox{2}}}\\
                        spr   & \eliste\\
                        comps & \eliste \\
                       } \\
             cont &  \ibox{3} \ms{
                      ltop & \ibox{4}\\
                      ind  & \ibox{5} \ms{
                                       per & 3\\
                                       num & sg\\
                                       }}}\\
  nonloc & \ms{ rel   & \sliste{ \ibox{3} }\\ 
                slash & \eliste\\
              }\\ 
  rels & \liste{ \ms[def\_q]{
                 arg0 & \ibox{1}\\ 
                 rstr & \ibox{6}\\ 
%                      body &\\
                 }, \ms[besitzen\_rel]{
                    lbl  & \ibox{4}\\ 
%                           arg0 & event\\
                    arg1 & \ibox{5}\\
                    arg2 & \ibox{1}
                 } }\\
  hcons & \liste{ \ms[qeq]{
                  harg & \ibox{6}\\
                  larg & \ibox{2}\\
                  }
              }
}
\z
Das Possessivpronomen \emph{dessen} führt einen definiten Quantor für das Nomen ein, dessen Artikel
es ist. Die Variable -- der Wert von \argzero{} -- ist \ibox{1}, auch der Index des Nomens. Die Restriktion des Quantors ist
\ibox{6} und \ibox{6} ist \qeq mit \ibox{2}, dem \ltopw des Nomens. Ansonsten bringt das
Possessivpronomen noch einen referentiellen Index mit \iboxb{5}, der auch der bzw.\ die Besitzende
ist. Dieser Index wird mit dem Index des modifizierten Nomens identifiziert und zu diesem Zwecke als
Bestandteil von \textsc{rel} nach oben gegeben. Das zweite Argument von \relation{besitzen}
\iboxb{1} ist der, die oder das Besessene, das Nomen, dessen Determinator das Possessivum ist.


Schema~\ref{rs-schema} enthält sowohl die bisher diskutierten syntaktischen als auch die semantischen
Beschränkungen.

%\begin{figure}
\begin{schema}[Relativsatzschema]
\label{rs-schema}\is{Schema!Relativsatz-}
\type{relative-clause} \impl\\
\oneline{%
\onems{ loc \ms{ cat & \ms{ head & \onems[relativizer]{ mod \upshape \nbar\ind{1} \\
                                                              } \\
                                     spr   & \eliste\\
                                     comps & \eliste\\
                                     } \\
                           cont  & \ibox{2} \ms{ ltop &  \ibox{3}  \\
                                               } \\
                              } \\
                 nonloc \ms{  rel   & \liste{} \\
                              slash & \liste{} \\
                           } \\
   dtrs \liste{ \begin{tabular}{@{}l@{}}
                            \ms{ loc    & \ibox{4} \\
                                 nonloc & \ms{%  que   & \liste{} \\
                                                              rel   & \sliste{ \ms{ ind  & \ibox{1}\\
                                                                                    ltop & \ibox{3}
                                                                                           } } \\
                                                              slash & \eliste \\
                                                           } \\
                                   }, %\\
                            \onems{ loc  \onems{ cat \onems{ head \ms[verb]{ initial & $-$ \\
                                                                                  vform   & fin \\
                                                                                } \\
                                                               comps \liste{} \\
                                                               % vcomp none is nicht nötig, ergibt sich aus anderen Schemata
                                                                     } \\
                                                            cont \ibox{2} \\
                                                          } \\
                                                 nonloc \ms{% que   & \liste{} \\
                                                             rel   & \liste{} \\
                                                             slash & \sliste{ \ibox{4} } \\
                                                             } \\
                                               } \\
                            \end{tabular}
                         } \\
}}
\end{schema}
%\vspace{-\baselineskip}\end{figure}
\ibox{1} ist der referentielle Index des modifizierten Nomens. Dieser ist identisch mit dem Index
des Relativpronomens (innerhalb von \textsc{rel} in der ersten Tochter). \ibox{2} ist der
semantische Beitrag des gesamten Relativsatzes und auch der rechten
Tochter des Relativsatzes, also dem Teilsatz, aus dem die Relativphrase extrahiert ist. Der \ltopw
innerhalb von \ibox{2} ist \ibox{3}. 

Die Analyse unseres Standardbeispiels in (\ref{ex Kind, von dessen Fahrrad sie ein Bild macht}) mit
Modifikation der \nbar zeigt Abbildung~\vref{abb-Kind-von-dessen-Fahrrad}.
\begin{figure}
\centerline{%
\begin{forest}
sm edges
[\nbar{}\ind{1}
  [\ibox{2}\nbar\ind{1} [Kind]] 
  [{RS[\textsc{mod} \ibox{2}, \textsc{rel} \eliste, \textsc{slash} \eliste] }
    [{PP[\textsc{loc} \ibox{3}, \textsc{rel} \sliste{ \ibox{4} }]}
      [P [von] ]
      [{NP[\textsc{rel} \sliste{ \ibox{4} }]}
         [{Det[\textsc{rel} \sliste{ \ibox{4} \ms{ ltop & \ibox{5}\\
                                          ind  & \ibox{1} } }] } [dessen] ]
         [N [Fahrrad] ] ] ]
    [{S[\type{fin}, \ltop \ibox{5}, \textsc{slash} \sliste{ \ibox{3} }] } 
       [sie ein Bild \_ macht, roof] ] ] ]
\end{forest}}
\caption{Perkolation und Abbindung der \textsc{rel}- und
  \slashwe, Identifikation des \ltop- und Index-Wertes}\label{abb-Kind-von-dessen-Fahrrad}
\end{figure}
Der \ltopw von \emph{dessen}, also das Label von \relation{besitzen} \iboxb{5}, und der Index des
Relativpronomens \iboxb{1} werden über \rel nach oben gegeben. Im Relativsatz wird der \ltopw von
\emph{dessen} mit dem \ltopw des finiten Satzes identifiziert. Der \localw der PP entspricht der
Lücke im finiten Satz \iboxb{3}. \rel und \slasch sind auf der Ebene des Relativsatzes abgebunden,
weshalb nichts weiter nach oben gereicht wird: \rel und \slasch des Relativsatzes sind die leere Liste. Der Relativsatz modifiziert eine \nbar. Über \ibox{2} kann der
Relativsatz auf alle Eigenschaften der \nbar zugreifen und auch die Identität des referentiellen
Indexes der modifizierten \nbar mit dem Index des Relativpronomens erzwingen. 

Die detaillierte Version der Analyse, die auch noch \rels- und \hcons-Werte enthält, zeigt Abbildung~\vref{abb-Kind-von-dessen-Fahrrad-RELS}.
\begin{figure}
\begin{sideways}
\resizebox{\textheight-3\baselineskip}{!}{%
\begin{forest}
sm edges
[\nbar{}\ind{1}\ms{ ltop  & h1 = h5\\
                    rels  & $\langle$ h1:kind(x), h2:def(y, h3, h4), h1:besitzen(e1, x, y),\\
                          & \hphantom{$\langle$~}h6:fahrrad(y), h7:exist(z, h8, h9), h10:bild(z, y), h1:machen(e2, i1, z) $\rangle$\\
                    hcons & \nliste{ h3 \qeq h6, h8 \qeq h10 } }
  [\ibox{2}\nbar{}\ms{ ind   & x\\
                       ltop  & h1\\
                       rels  & \nliste{ h1:k(x) }\\
                       hcons & \eliste} [Kind]] 
  [RS\ms{ mod   & \ibox{2}\\
          rel   & \eliste\\
          slash & \eliste\\
          ltop  & h5 = h11\\
          rels  & \nliste{ h2:def(y, h3, h4), h5:be(e1, x, y), h6:f(y), h7:exist(z, h8, h9), h10:bi(z, y), h5:m(e2, i1, z) }\\
          hcons & \nliste{ h3 \qeq h6, h8 \qeq h10 } }
    [PP\ibox{3}\ms{ind   & y\\
                   rel   & \sliste{ \ibox{4} }\\
                   rels  & \nliste{ h2:def(y, h3, h4), h5:be(e1, x, y), h6:f(y) }\\
                   hcons & \nliste{ h3 \qeq h6 }}
      [P\ms{ ind   & y\\
             rels  & \eliste\\
             hcons & \eliste} [von] ]
      [NP\ms{ind   & y\\
             rel & \sliste{ \ibox{4} }\\
             rels  & \nliste{ h2:def(y, h3, h4), h5:be(e1, x, y), h6:f(y) }\\
             hcons & \nliste{ h3 \qeq h6 }}
         [Det\ms{rel & \sliste{ \ibox{4} \ms{ ltop & h5\\
                                              ind  & x } }\\
                 rels  & \normalfont\upshape $\langle$ h2:def(y, h3, h4), \\
                       & \normalfont\upshape \hphantom{$\langle$~}h5:be(e1, x, y) $\rangle$\\
                 hcons & \nliste{ h3 \qeq h6 }} [dessen] ]
         [N\ms{ ind   & y\\
                rels  & \nliste{ h6:f(y) } \\
                hcons & \eliste} [Fahrrad] ] ] ]
    [V\ms{ltop  & h11\\
          slash & \sliste{ \ibox{3} }\\
          rels  & \nliste{ h7:exist(z, h8, h9), h10:bi(z, y), h11:m(e2, i1, z) }\\
          hcons & \nliste{ h8 \qeq h10 } } 
      [NP\ms{ ind   & i1\\
              slash & \sliste{ \ibox{3} }\\              
              rels  & \eliste\\
              hcons & \eliste} [sie]]
      [V\ms{ ltop  & h11\\
             slash & \nliste{ \ibox{3} }\\
             rels  & \nliste{ h7:exist(z, h8, h9), h10:bi(z, y), h11:m(e2, i1, z) }\\
             hcons & \nliste{ h8 \qeq h10 }}
        [NP\ms{ ind   & z\\
                slash & \nliste{ \ibox{3} }\\
                rels  & \nliste{ h7:exist(z, h8, h9), h10:bi(z, y) }\\
                hcons & \nliste{ h8 \qeq h10 }} 
          [Det\ms{ rels  & \nliste{ h7:exist(z, h8, h9)}\\
                   hcons & \nliste{ h8 \qeq h10 }} [ein]]
          [N\ms{ ind   & z\\
                 slash & \nliste{ \ibox{3} }\\
                 rels  & \nliste{ h10:bi(z, y) }\\
                 hcons & \eliste}
            [N\ms{ ind   & z\\
                   rels  & \nliste{ h10:bi(z, y) }\\
                   hcons & \eliste} [Bild]]
            [PP\ibox{3}\ms{ ind   & y\\
                            slash & \sliste{ \ibox{3} }\\
                            rels  & \eliste\\
                            hcons & \eliste} [\trace]]]]
        [V\ms{ ltop  & h11\\
               rels  & \nliste{ h11:m(e2, i1, z) }\\
               hcons & \eliste} [macht]]]]]]
\end{forest}
}
\end{sideways}
\caption{\label{abb-Kind-von-dessen-Fahrrad-RELS}Perkolation und Abbindung der \textsc{rel}- und
  \slashwe, Identifikation des \ltop- und Index-Wertes}
\end{figure}
Wie Abbildung~\ref{Abbildung-Das Kind, das spielt, lacht} auch, wird die Analyse in
Abbildung~\ref{abb-Kind-von-dessen-Fahrrad} von unten 
nach oben erklärt, \dash es werden \ltopwe nach oben gereicht und dann an übergeordneten Knoten mit
anderen \ltopwen identifiziert. An Knoten, die noch weiter oben sind, wird dann das linke Handle
weiter benutzt und alle Relationen, die als Label das rechte Handle hatten, werden ebenfalls mit dem
linken Handle versehen. Zum Beispiel steht bei \ltop für \emph{von dessen Fahrrad sie ein Bild
  macht} h5 = h11. Im Baum für \emph{von dessen Fahrrad} gibt es Relationen mit Label h5, im Baum
für \emph{sie ein Bild macht} Relationen mit h11. In der Liste der Relationen für die gesamte Phrase
haben die jeweiligen Relationen alle das Label h5. Bei der Bildung der Phrase \emph{Kind, von dessen
  Fahrrad sie ein Bild macht} wird h1 mit h5 gleichgesetzt und am Gesamtknoten haben alle
Relationen, die mal h1, h5 oder h11 als Label hatten, das Label h1. Streng genommen ist das nicht
korrekt. Überall im Baum müsste h1 stehen, aber wenn man das so machen würde, wäre überhaupt nicht
mehr nachvollziehbar, wo welche Information herkommt, und an welcher Stelle in der Analyse sie
geteilt worden ist.

Bevor alternative Analysen besprochen werden, muss noch ein Detail geklärt werden:
\type{relative-clause} ist kein Untertyp von \type{headed-phrase}, das Valenzprinzip und das
Semantikprinzip gelten deshalb nicht. Die entsprechende Information (Valenzmerkmale und \contw) ist
im Relativsatzschema daher explizit spezifiziert. Einen Überblick über die aktuelle Typhierarchie
unter \type{phrase} gibt Abbildung~\vref{abb-sign-relcl}. 
%
\begin{figure}
\centering
\oneline{%
%\scalebox{0.8}{%
%\begin{sideways}%
\begin{forest}
type hierarchy,
 % for tree={
 %   calign=fixed angles,
 %   calign angle=50
 % } 
[phrase
    %[non-headed-phrase, l sep*=2
    [relative-clause]%]
    [headed-phrase, l sep*=2, for children={l sep*=4}
      [head-non-adjunct-phrase, name=non-adjunct, 
        [head-adjunct-phrase,no edge, name=adjunct]]
      [head-non-argument-phrase, name=non-arg
        [head-argument-phrase,no edge, name=argument]]
      [head-non-filler-phrase, name= non-filler
        [head-filler-phrase, no edge, name=filler]]
      [head-non-specifier-phrase, name=non-specifier
        [head-specifier-phrase, no edge, name=specifier]]]]
\draw (non-adjunct.south) to (argument.north);
\draw (non-adjunct.south) to (filler.north);
\draw (non-adjunct.south) to (specifier.north);
\draw (non-arg.south) to (adjunct.north);
\draw (non-arg.south) to (filler.north);
\draw (non-arg.south) to (specifier.north);
\draw (non-filler.south) to (adjunct.north);
\draw (non-filler.south) to (argument.north);
\draw (non-filler.south) to (specifier.north);
\draw (non-specifier.south) to (adjunct.north);
\draw (non-specifier.south) to (argument.north);
\draw (non-specifier.south) to (filler.north);
\end{forest}
%\end{sideways}
}
\caption{\label{abb-sign-relcl}Typhierarchie für \type{phrase}}
\end{figure}




\section{Alternativen}

In den beiden folgenden Abschnitten wird eine alternative HPSG"=Analyse und ein
Kategorialgrammatik"=Ansatz diskutiert.

\subsection{Eine konstruktionsbasierte Relativsatzanalyse}
\label{sec-konstruktionsbasierte-rs}


Die klassische Relativsatzanalyse in \citew[Kapitel~5]{ps2} geht davon aus, dass ein leerer
Komplementierer mit einer Relativphrase und einem Satz, aus dem die Relativphrase extrahiert wurde,
kombiniert wird. Für den Relativsatz \emph{to whom Kim gave a book} nehmen Pollard und Sag folgende Struktur an:
\ea
{}[\sub{RP} [\sub{PP} to whom]$_i$ [\sub{R$'$} e [\sub{S/PP} Kim gave a book \_$_i$]]]
\z
e ist dabei ein leerer Kopf der Kategorie Relativierer\is{Relativierer} (\emph{relativizer}). Er wird zuerst mit
\emph{Kim gave a book} kombiniert und bildet eine R$'$"=Projektion. Die R$'$"=Projektion selegiert die
Relativphrase und bildet mit ihr zusammen dann eine vollständige R"=Projektion. Diese Analyse
entspricht der \xbart: Es gibt einen Kopf in jeder 
der Teilstrukturen. Der Kopf selegiert entsprechende Argumente und bestimmt den Bedeutungsbeitrag
der Gesamtkonstruktion. Die im Abschnitt~\ref{sec-rs-anal} vorgestellte Analyse unterscheidet sich
nur darin in der von Pollard und Sag, dass der Effekt des leeren Kopfes in eine Grammatikregel
integriert wurde. \citet*[\page153, Lemma~4.1]{BHPS61a} haben gezeigt, dass man
Phrasenstrukturgrammatiken mit leeren Elementen in solche ohne leere Elemente umwandeln
kann. Dieselbe Technik, die sie vorgeschlagen haben, wurde auch hier für die Relativsatzanalyse
verwendet: Statt einen leeren Kopf mit zwei Argumenten zu kombinieren, verwendet man eine
Grammatikregel, die die beiden Elemente direkt kombiniert und das Ergebnis, das die Kombination des
leeren Kopfes mit seinen zwei Argumenten hätte, direkt am Mutterknoten repräsentiert.

Einen anderen Weg geht \citet{Sag97a}. Er schlägt zwar ebenfalls Grammatikregeln statt leerer Köpfe
für die Analyse von Relativsätzen vor, geht aber davon aus, dass in Relativsätzen das Verb der Kopf
ist. Aus dieser Annahme ergeben sich zwei Probleme: Das erste wurde bereits auf
Seite~\pageref{Erklaerung-relativizer} angesprochen. Wenn das Verb der Kopf des Relativsatzes ist,
dann muss das Verb einen spezifizierten \modw haben (können), und man muss sicherstellen, dass
Verbalprojektionen, die nicht in Relativsätzen vorkommen, nicht \nbar{}s modifizieren können. Das
zweite Problem besteht darin, dass man für die Kombination eines Relativsatzes mit einem \nbar in
der von Sag verwendeten \isi{Situationssemantik} eine nominale Semantik beim Adjunkt
braucht \parencites[Abschnitt~1.9, 213]{ps2}. Verbalprojektionen haben aber eine verbale Semantik. Sag löst das erste 
Problem, indem er für den Typ \type{clause} die als Default\is{Default} spezifizierte Beschränkung
einführt, dass der \modw \type{none} ist (S.\,480). Der \modw von Verben ist damit nicht mehr eine
lexikalische Eigenschaft der Verben, sondern wird je nach Verwendungsweise des Verbs
festgelegt. Damit die erwünschten Resultate erzielt werden, muss man bei einem solchen Vorgehen also
mindestens zwei Typen von Sätzen unterscheiden: Relativsätze und Nicht"=Relativsätze. Bei
Relativsätzen wird der Default"=Wert überschrieben, bei Nicht"=Relativsätzen bleibt er erhalten. Sag geht von
mindestens vier Untertypen von \type{clause} aus: \type{decl-cl}, \type{inter-cl}, \type{imp-cl} und
\type{rel-cl}. Man könnte sich überlegen, ob man Ähnliches auch für die hier vorliegende Grammatik
des Deutschen annehmen möchte, aber es scheint nicht sinnvoll, der Wortfolge in (\mex{1}) einen
Satztyp zuzuweisen.
\ea
der Affe den Stock greift
\z
(\mex{0}) enthält das transitive Verb \emph{greifen} mit allen benötigten Argumenten. Es handelt
sich also um eine vollständige Verbalprojektion. Allerdings ist diese ohne nebensatzeinleitende
Konjunktion nicht in größeren Kontexten verwendbar. Wenn der \modw von (\mex{0}) nicht spezifiziert
wäre, könnte man (\mex{0}) als Adjunkt verwenden. Und zwar mit beliebigen Köpfen mit beliebigen
Sättigungsgraden. Das gilt es zu vermeiden. Die einzige Möglichkeit, die ich sehe, ist die
lexikalische Spezifikation des \modwes von Verben.
%% Für diese Typen legt er die folgenden Beschränkungen fest:
%% \ea
%% \begin{tabular}[t]{@{}ll@{}}
%% Typ      & Beschränkung\\
%% \type{decl-cl}  & [ \textsc{content} \type{proposition}] \\
%% \type{inter-cl} & [ \textsc{content} \type{question}]    \\
%% \type{imp-cl}   & [ \textsc{content} \type{directive}]   \\
%% \type{rel-cl}   & \ms{ head    & \ms{ mc & -\\
%%                                inv & -\\
%%                                mod & \textrm{[}{\textsc{head} noun \textrm{]}}\\
%%                              } \\
%%                 content & proposition \\
%%               }\\
%% \end{tabular}
%% \z
%% Diese Typen haben gemeinsame Untertypen mit \type{head-adj-ph}, \type{hd-fill}, \type{hd-comp-ph},
%% \type{fin-hd-subj-ph}, \type{hd-spr-ph} und \type{non-hd-ph}, für die ähnliche Beschränkungen
%% gelten wie für die in diesem Buch verwendeten Typen \type{head-adjunct-phrase},
%% \type{head-argument-phrase}, \type{head-specifier-phrase} und
%% \type{head-filler-phrase}. Daraus ergibt sich folgendes Problem: Da \zb Deklarativsätze vom Typ
%% \type{hd-fill} sind, gilt für sie das Semantikprinzip, das dafür sorgt, dass der \contw des Satzes
%% mit dem \contw der Kopftochter übereinstimmt. Da der \contw des Verbs auch in Teilstrukturen für
%% VPen projiziert wird, ist der \contw des Deklarativsatzes identisch mit dem \contw des Verbs. Dieser
%% Typ ist aber ein Typ, der die Argumentrollen des Verbs einführt (\zb \type{geben}) und nicht ein Typ
%% wie \type{proposition}. \citet{SWB2003a} verwenden deshalb ein spezielles Merkmal innerhalb von
%% \cont, das sie \textsc{mode} nennen. Der Wert gibt Auskunft über den Satzmodus.

Die Lösung des zweiten Problems ist sehr unschön: Da die Relativsätze eine verbale Semantik haben,
muss die Verrechnung dieser Semantik mit der des modifizierten Nomens außerhalb des Relativsatzes
passieren. \citet[\page475]{Sag97a} schlägt deshalb ein spezielles Schema vor, das eine \nbar mit
einem Relativsatz kombiniert. Diese Analyse widerspricht den Grundannahmen in der HPSG, die auf
\citet{Saussure16a-de} zurückgehen: Sprachliche Zeichen sollen Form"=Bedeutungspaare sein. So ist ein
Relativsatz etwas, das einen bestimmten syntaktischen Aufbau hat, der einer Relativsatzbedeutung
entspricht. In der Sagschen Analyse ist das nicht widergespiegelt, denn dort hat der Relativsatz die
dem Verb entsprechende Bedeutung, die Relativsatzbedeutung kommt erst bei Kombination mit einem
Nomen hinzu.

Zusätzlich zur \nbar\hyp Relativsatz"=Spezialregel, die nur für die Analyse von
Nominalstrukturen mit Relativsatz gebraucht wird, gibt es noch das allgemeine
Kopf"=Adjunkt"=Schema (\type{simple-hd-adj-ph}). Die in früheren Auflagen dieses Buches vorgestellte
Analyse kommt dagegen mit einem sehr allgemeinen Kopf"=Adjunkt"=Schema aus. Insgesamt entfällt das
Problem, wenn man MRS verwendet, weil dann die Konjunktion von Relationen nicht durch gemeinsame
Mitgliedschaft von Relationen in einer Menge entsteht, sondern durch die Identifikation von
Handles. In der hier vorgestellten Analyse ist der Bedeutungsbeitrag des Relativsatzes identisch mit
der einer Verbalprojektion.


\subsection{Das Relativpronomen als Kopf}


\citet{Steedman89a}\is{Pronomen!Relativ-|(} stellt in seinem Artikel die Kategorialgrammatik\is{Kategorialgrammatik (CG)|(} vor.
Er diskutiert englische Relativsätze wie (\mex{1}) und gibt den
Lexikoneintrag in (\mex{2}) für Relativpronomina wie \emph{which} und \emph{who}(\emph{m})
(S.\,217):\footnote{
  Siehe auch \citew[\page 614]{SB2006a-u}.
}
\ea
apples which Harry eats
\z
\ea
(N$\backslash$N)/(S/NP)
\z
Die Symbole `/' und `$\backslash$' stehen für Kombination nach links bzw.\
rechts, \dash, S/NP steht für etwas, das einen Satz ergibt, wenn es mit einer NP,
die sich rechts von ihm befindet, kombiniert wird. In (\mex{0}) steht
S/NP für \emph{Harry eats}, \dash die Verbprojektion, der ein Objekt fehlt.
Der Eintrag für das Relativpronomen bedeutet folgendes: Das Relativpronomen
verlangt rechts von sich eine Wortgruppe der Kategorie S/NP (den Satz mit einer
fehlenden NP), und wenn es mit dieser Wortgruppe kombiniert worden ist, ist das
Ergebnis N$\backslash$N. N$\backslash$N steht für eine Kategorie, die sich
nach links mit einem Nomen verbindet und ein Nomen ergibt. 
%X/X bzw.\ X$\backslash$X
%sind die allgemeinen Muster für Modifikatoren in der Kategorialgrammatik.
Der Aspekt der Analyse, der hier interessiert, ist, dass das Relativpronomen
die externen Eigenschaften des gesamten Relativsatzes bestimmt. Das Relativpronomen
ist ein Funktor, der einen Satz mit fehlendem Objekt selegiert und festlegt,
dass das Resultat der Kombination ein Nomen modifizieren kann.

Ähnliche Analysen wurden im Rahmen der HPSG für sogenannte freie Relativsätze\is{Relativsatz!freier}
vorgeschlagen. Freie Relativsätze sind Relativsätze, die im Satz ohne Bezugsnomen
stehen und direkt als Argument oder Adjunkt zu einem höheren Verb fungieren.
Der Satz in (\mex{1}) ist ein Beispiel für eine Konstruktion mit einem freien Relativsatz.
\ea
Wer die Hausaufgaben sofort abgibt, kann die Ferien genießen.
\z
In (\mex{0}) gibt es für den Relativsatz \emph{wer die Hausaufgaben sofort abgibt} kein
sichtbares Bezugswort. Man kann sich überlegen, dass der Satz einem Satz wie (\mex{1})
mit sichtbarem Bezugswort entspricht:
\ea
Jeder, der die Hausaufgaben sofort abgibt, kann die Ferien genießen.
\z
Eine Analyse, die für das Deutsche \citep{Kubota2003a-u} und das Englische \citep{WK03a}
% und das Französische 
vorgeschlagen wurde, geht davon aus, dass in Sätzen wie (\mex{-1}) das Relativpronomen der Kopf ist.
Damit wäre ohne weitere Annahmen \emph{wer die Hausaufgaben sofort abgibt} eine NP,
und es wäre erklärt, warum dieser Relativsatz an Stelle einer Argument"=NP stehen kann.

Das Problem, das alle Ansätze haben, die davon ausgehen, dass das Relativpronomen der
Kopf/""Funktor ist, ist jedoch, dass Relativpronomina tief eingebettet sein können.
(\mex{1}) zeigt englische\il{Englisch} Beispiele von \citet[\page212]{ps2} bzw.\ von \citet[\page 109]{Ross67}\nocite{Ross86a-u}:
\eal
\ex Here's the minister [[in [the middle [of [whose sermon]]]] the dog barked].
\ex Reports [the height of the lettering on the covers of which] the government prescribes should be abolished.
\zl
In (\mex{0}a) ist das Relativpronomen der Determinator von \emph{sermon}.
Je nach Analyse ist \emph{whose} der Kopf der Phrase \emph{whose sermon},
diese NP allerdings ist unter \emph{of} eingebettet und die Phrase \emph{of whose sermon}
ist von \emph{middle} abhängig. Die gesamte NP \emph{the middle of whose sermon}
ist ein Komplement der Präposition \emph{in}. Wollte man behaupten, dass \emph{whose}
der Kopf des Relativsatzes in (\mex{0}a) ist, müsste man schon einige Handstände (oder Kopfstände?)
machen. In (\mex{0}b) ist das Relativpronomen noch tiefer eingebettet.
In der Kategorialgrammatik gibt es neben der Funktionalapplikation, die
für die Kombination von Konstituenten verwendet wird, noch die Typanhebung\is{Typanhebung} von
Konstituenten. Mit solchen Typanhebungen ist es dann möglich, Wortfolgen als
Konstituenten zu analysieren, die normalerweise nicht als Konstituenten betrachtet
werden, aber zum Beispiel für die einfache Analyse von Koordinationsstrukturen\is{Koordination}
gebraucht werden \citep{Steedman89a}. \citet[\page 204]{Morrill95a} diskutiert für das
Relativpronomen im Relativsatz in (\mex{1}a) den Lexikoneintrag in (\mex{1}b):\footnote{
  Morrill verwendet eine andere Notation. Ich habe sie an die von Steedman angepasst.
% Schreibt (PP/NP)$\backslash$(N$\backslash$N)/(S/NP), meint aber wohl
%          (PP/NP)$\backslash$(N$\backslash$N)/(S/PP)
}
\eal
\ex about which John talked
\ex (PP/NP)$\backslash$(N$\backslash$N)/(S/PP)
\zl
In diesem Lexikoneintrag verlangt das \emph{which} links von sich etwas, das eine Nominalphrase
braucht, um eine vollständige Präpositionalphrase zu ergeben (PP/NP), \dash, \emph{wich} selegiert die
Präposition. Morrill stellt fest, dass zusätzliche Einträge für Fälle angenommen werden müssen, in
denen das Relativpronomen in der Mitte der vorangestellten Phrase steht.
\ea
the contract the loss of which after so much wrangling John would finally have to pay for\footnote{
\citew[\page 203]{Morrill95a}
}
\z
Morrill stellt fest, dass man alle Fälle durch zusätzliche lexikalische Stipulationen behandeln
kann. Er schlägt stattdessen zusätzliche Arten der Kombination von Funktoren und Argumenten vor,
die einen Funktor B $\uparrow$ A sein Argument A umschließen lassen und B ergeben bzw.\ einen Funktor A $\downarrow$
B in sein Argument A einfügen, um dann B zu ergeben (S.\,190). Mit diesen zusätzlichen Operationen
braucht er dann noch die beiden Lexikoneinträge in (\mex{1}), um die diversen
Rattenfängerkonstruktionen ableiten zu können:
\eal
\ex (NP $\uparrow$ NP) $\downarrow$ (N$\backslash$N)/(S/NP)
\ex (PP $\uparrow$ NP) $\downarrow$ (N$\backslash$N)/(S/PP)
\zl
Morrill ist es auf diese Weise gelungen, die Anzahl der Lexikoneinträge für \emph{which} zu
reduzieren, trotzdem bleibt der Fakt, dass er die Kategorien, die in Rattenfängerkonstruktionen
vorkommen können, im Lexikoneintrag des Relativpronomens erwähnen muss. 
%% Im Deutschen ist die
%% Rattenfängerkonstruktion bei Präpositionalphrasen, Adjektivphrasen und auch Verbphrasen möglich.
Auch geht die Einsicht, dass es sich bei Relativsätzen um eine Phrase mit einem Relativpronomen
+ Satz handelt, dem die Relativphrase fehlt, in solchen Analysen verloren.

Für die Behandlung freier Relativsätze wie (\mex{1}) schlägt \citet[\page 151, 164]{Kubota2003a-u} im
Rahmen der HPSG vor, das Relativpronomen als Kopf zu betrachten, wobei das Relativpronomen eine Präposition und einen finiten
Satz mit extrahierter PP selegiert.
\ea
Aus wem noch etwas herausgequetscht werden kann, ist sozial dazu verpflichtet, es abzuliefern; \ldots\footnote{
        Wiglaf Droste, taz, 01.08.1997, S.\,16.
      }
\z
Diese Analyse kann nicht erfassen, wieso das Relativpronomen den Kasus hat, den die Präposition
\emph{aus} verlangt (Dativ) und wieso der gesamte freie Relativsatz trotzdem den Platz des
Nominativarguments von \emph{verpflichtet} einnehmen kann. Man könnte die Analyse technisch retten,
wenn man annehmen würde, dass es sich bei \emph{wem} um ein Pronomen mit dem Kasus Nominativ handelt,
das eine Präposition selegiert, die ein Element im Dativ verlangt. Da es so etwas aber an keiner
anderen Stelle in der deutschen Grammatik gibt, ist eine solche Lösung abzulehnen. Alternative
Lösungen, die \emph{aus wem noch etwas herausgequetscht werden kann} als vollständigen Relativsatz
analysieren, der dann die Nominativstelle einnehmen kann, sind Kubotas Ansatz vorzuziehen. Siehe
\citew{Mueller99a,Mueller99b} für eine entsprechende Analyse im Rahmen der HPSG.%
\is{Relativsatz|)}
\is{Kategorialgrammatik (CG)|)}
\is{Pronomen!Relativ-|)}

%\section*{Kontrollfragen}

\questions{
\begin{enumerate}
\item Wie sind Relativsätze aufgebaut?
\end{enumerate}
}

%\section*{Übungsaufgaben}

\exercises{
\begin{enumerate}
\item Welche Relativpronomina kennen Sie?
\item Skizzieren Sie die syntaktischen Aspekte der Analyse für die folgende Phrase:
\ea
die Blume, die allen gefällt
\z
Gehen Sie auf die \slashwe und die \relwe ein.

\item Laden Sie die zu diesem Kapitel gehörende Grammatik von der Grammix"=CD
(siehe Übung~\ref{uebung-grammix-kapitel4} auf Seite~\pageref{uebung-grammix-kapitel4}).
Im Fenster, in dem die Grammatik geladen wird, erscheint zum Schluss eine Liste von Beispielen.
Geben Sie diese Beispiele nach dem Prompt ein und wiederholen Sie die in diesem Kapitel besprochenen
Aspekte.

\end{enumerate}
}

\furtherreading{
\citet{Arnold:Godard:2021a} beschäftigen sich mit Analysen von verschiedenen Relativsatzkonstruktionen im
Englischen, Französischen, Japanischen und Koreanischen. Sie besprechen auch so genannte freie
Relativsätze. Zu diesen siehe auch \citew{Mueller99b}.
}