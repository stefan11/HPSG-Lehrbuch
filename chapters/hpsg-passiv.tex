%% -*- coding:utf-8 -*-
%%%%%%%%%%%%%%%%%%%%%%%%%%%%%%%%%%%%%%%%%%%%%%%%%%%%%%%%%
%%   $RCSfile: hpsg-passiv.tex,v $
%%  $Revision: 1.23 $
%%      $Date: 2008/09/30 09:14:41 $
%%     Author: Stefan Mueller (CL Uni-Bremen)
%%    Purpose: 
%%   Language: LaTeX
%%%%%%%%%%%%%%%%%%%%%%%%%%%%%%%%%%%%%%%%%%%%%%%%%%%%%%%%%


\chapter{Passiv}
\label{chap-passiv}


In diesem Kapitel untersuche ich verschiedene Passivvarianten.
Nach der Datendiskussion im Abschnitt~\ref{passive-data} entwickle ich
im Abschnitt~\ref{sec-passive-anal} eine Analyse, die mit nur einem Lexikoneintrag für das Perfekt- bzw.\
Passivpartizip auskommt. Die Realisierung der Argumente in Aktiv- und Passivumgebungen
hängt nur vom entsprechenden Hilfsverb ab. 
Das sogenannte Fernpassiv wird mit Bezugnahme auf den in den vorigen
Kapiteln vorgestellten Verbalkomplex erklärt.



\section{Das Phänomen}
\label{passive-data}

Der Satz in (\mex{1}b) ist ein Beispiel für einen Passivsatz. Bei (\mex{1}b) handelt
es sich um das sogenannte Vorgangspassiv\is{Passiv!Vorgangs-}, das mit \emph{werden}\iw{werden!Passiv}
gebildet wird.
\eal
\ex\label{ex-Karl-oeffnet-das-Fenster}
Karl öffnet das Fenster.
\ex Das Fenster wird geöffnet.
\zl
Das Passiv kann dazu benutzt werden, das logische Subjekt eines Verbs zu unterdrücken.
Der Wunsch, das Subjekt zu unterdrücken, kann verschiedene Ursachen haben:
Die Person oder das Ding, das dem logischen Subjekt entspricht, kann weniger wichtig
für die Aussage oder bereits durch den Kontext bekannt sein.
Das logische Subjekt kann dann auch durch eine Präpositionalphrase (meist mit \emph{von})
ausgedrückt werden. Die Realisierung durch eine \emph{von}"=PP macht eine Anordnung
möglich, die von der Anordnung\is{Serialisierung} des Subjekts im Aktivsatz verschieden ist.
Ein anderer Grund für die Benutzung des Passivs ist die Veränderung der Argumentstruktur,
die ein Akkusativobjekt zum Subjekt macht und dadurch eine Koordination\is{Koordination} des passivierten
Verbs mit anderen Prädikaten ermöglicht, 
die das zugrundeliegende Akkusativobjekt des passivierten Verbs als Subjekt haben. (\mex{1}) zeigt ein Beispiel für solch eine
Koordination.
% Paul1919a:62
\ea
Der Mann wurde von einem Betrunkenen angefahren und starb an den Folgen.
\z

\noindent
Beim Passiv unterscheidet man das sogenannte persönliche Passiv vom unpersönlichen Passiv.
Ein passivierbares Verb, das im Aktiv ein Akkusativobjekt\is{Objekt!Akkusativ-} verlangt, hat eine Passivform,
in der dieses Objekt im Nominativ realisiert wird:
\eal
\label{ex-agentive-passive}
\ex Die Frau liebt den Mann.
\ex Der Mann wird geliebt.
\zl
Diese Art des Passivs wird \emph{persönliches Passiv}\is{Passiv!persönliches|uu} genannt.
Im Gegensatz dazu gibt es im Deutschen Passivierungen von Verben, die kein
Akkusativobjekt selegieren. In solchen Konstruktionen wird das Subjekt
des aktiven Verbs genau wie beim persönlichen Passiv unterdrückt.
Da es kein Akkusativobjekt gibt, das zum Subjekt
werden könnte, gibt es im Passivsatz kein Nominativelement.

Die Sätze in (\mex{1}b,d) sind Beispiele für das sogenannte \emph{unpersönliche Passiv}\is{Passiv!unpersönliches|uu}:
\eal
\label{ex-agentive-passive-dat}\iw{helfen}
\ex Die Frau hilft dem Mann.
\ex\label{ex-dem-mann-wird-geholfen} 
Dem Mann wird geholfen.
\ex Hier tanzen alle.\iw{tanzen}
\ex\label{ex-hier-wird-getanzt}
Hier wird getanzt.
\zl
\emph{helfen} ist ein Verb, das einen Nominativ und einen Dativ regiert (\mex{0}a).
In Passivsätzen wird das Subjekt unterdrückt und das Dativobjekt wird ohne
irgendeine Veränderung realisiert (\mex{0}b). \emph{tanzen} ist ein intransitives Verb.
Im Passivsatz (\mex{0}d) ist überhaupt keine NP vorhanden. Die Sätze in (\mex{0}b) und (\mex{0}d)
sind subjektlose Konstruktionen\is{Verb!subjektloses}. 
Deutsch unterscheidet sich von Sprachen wie dem Isländischen\il{Isländisch}
dadurch, daß es keine Dativsubjekte\is{Subjekt}\label{page-dativsubjekte}
gibt \citep*{ZMT85a}\iaright{Zaenen}\iaright{Maling}\iaright{Thr{\'a}insson}.
Ein Test zur Bestimmung von Subjekten, der von \citet*[\page477]{ZMT85a} angewendet wird,
ist der Test auf Kontrollierbarkeit\is{Kontrolle} eines Elements. Im Kapitel~\ref{expl-pred-subj-constr} habe ich bereits
gezeigt, daß subjektlose Konstruktionen nicht unter Kontrollverben eingebettet werden können.
Das Beispiel (\mex{1}) verdeutlicht das:
\eal
\ex[]{
Der Student versucht zu tanzen.
}
\ex[*]{
Der Student versucht, getanzt zu werden.
}
\zl
(\mex{0}a) hat ein kontrollierbares Subjekt (das von \emph{zu tanzen}). In (\mex{0}b) dagegen wurde das
Subjekt durch die Passivierung unterdrückt, weshalb der Satz nicht wohlgeformt ist.

Genauso können Infinitive mit passivierten Verben, die einen Dativ und keinen Akkusativ regieren,
nicht unter Kontrollverben eingebettet werden, wie (\mex{1}) zeigt:
\ea[*]{
\label{ex-der-student-versucht-geholfen-zu-werden}
Der Student versucht, geholfen zu werden.
}
\z
Das zeigt, daß der Dativ in (\ref{ex-dem-mann-wird-geholfen}) ein Komplement und kein Subjekt ist.
Außerdem gibt es beim unpersönlichen Passiv keine Kongruenz\is{Kongruenz} zwischen finitem Verb und dem Dativargument
(\mex{1}), was ebenfalls Evidenz dafür ist, daß es sich beim Dativ nicht um ein Subjekt
handelt.
%Sigurðsson sgat, daß es da zumindest keine Kongruenz mit dem Nominativ gibt.
%\NOTE{JB: gibt es irgendwo sonst Kongruenz zwischen finitem Verb und Dativ? St.Mü.: Wie ist
%  das im Isländischen?}
% McClosky LSA07: Kongruenz ist immer mit dem Nominativ.
\eal
\ex weil den Männern geholfen wird
\ex weil die Männer geliebt werden
\zl


\noindent
Das folgende Beispiel ist eine Variante von (\ref{ex-Karl-oeffnet-das-Fenster}), die Zustandspassiv\is{Passiv!Zustands-} genannt und mit \sein\iw{sein!Passiv} gebildet wird.
\ea 
Das Fenster ist geöffnet.
\z
Es ist aber durchaus umstritten, ob in (\mex{0}) ein Passiv vorliegt,
oder ob es sich bei \emph{geöffnet} um ein vom Partizip abgeleitetes Adjektiv handelt, das wie
\emph{offen} in \emph{Das Fenster ist offen.} mit \sein verbunden wurde. Für einen Überblick und Argumente für die
Einordnung als Konstruktion mit der Kopula\is{Kopula} \sein und einem Adjektiv siehe
\citew{Maienborn2005a}. Das sogenannte Zustandspassiv wird im folgenden nicht explizit
diskutiert. Sollte sich eine Analyse als Adjektiv als empirisch korrekt herausstellen, so ist das
durch die Analyse der Adjektivierung im Abschnitt~\ref{sec-adj-formation} größtenteils abgedeckt. Zu einer
Analyse als Passivform siehe \citew[Kapitel~3.2.2]{Mueller2002b}.

% Note that while the accusative object in (\ref{ex-agentive-passive}a) is realized as nominative in (\ref{ex-agentive-passive}b),
% the dative in (\ref{ex-agentive-passive-dat}a) does not change when the verb is passivized, as in (\ref{ex-agentive-passive-dat}b).

Im folgenden Abschnitt diskutiere ich die Einteilung von Verben in unakkusativische und unergativische Verben,
die für die Passivbildung und die Bildung adjektivischer Partizipien wichtig ist. 
%% Diese Verbklassifikation ist auch im Zusammenhang mit Resultativkonstruktionen wichtig, die im Kapitel~\ref{chap:result}
%% diskutiert werden. 
Nach der Diskussion der Unakkusativ/""Unergativunterscheidung wende ich
mich verschiedenen Formen des Passivs und ähnlichen Konstruktionen zu: dem Vorgangs- und
%dem Zustandspassiv,
dem Dativpassiv, dem Passiv mit \emph{lassen} und modalen Infinitiven mit
\sein. Ein Spezialfall, das sogenannte Fernpassiv, wird ebenfalls behandelt. Das Fernpassiv
ist eine Passivkonstruktion, in der das Akkusativobjekt eines tief eingebetteten Verbs als
Nominativ realisiert wird. Die Möglichkeit, ein Fernpassiv zu bilden, hängt mit der Fähigkeit
des Matrixverbs zur Verbalkomplexbildung zusammen.

\subsection{Unakkusativität}
\label{sec-unakkusativitaet}\label{sec-unaccusativity}
\label{sec-phen-hilfsverbselektion}
\is{Verb!unakkusativisches|(}

% Er trifft mich vs. Er begegnet mir. ? semantischer Unterschied

Obwohl es prinzipiell möglich ist, intransitive Verben zu passivieren, wie
(\ref{ex-dem-mann-wird-geholfen}) und (\ref{ex-hier-wird-getanzt}) gezeigt haben,
gibt es bestimmte Verben, die sich nicht passivieren lassen.
So können die Verben \word{ankommen} und  \word{auf"|fallen} nicht passiviert werden,
wie die Beispiele in (\mex{1}b) und (\mex{1}d) zeigen:
\eal
\label{ex-passive-erg}
\ex[]{
Der Zug kam  an.
}
\ex[*]{
Dort wurde angekommen.
}
\ex[]{
Der Mann fiel ihr auf.
}
\ex[*]{
Ihr wurde aufgefallen.
}
\zl
\is{Passiv!adjektivisches Partizip|(}%
Verben aus dieser Klasse können als pränominale adjektivische Partizipien II vorkommen,
wie die folgenden Beispiele zeigen:
\eal
\label{ex-prenominal-erg}
\ex 
der angekommene Zug
% Duerscheid89a:115 sagt, das gebe es nicht.
\ex
dem Regime aufgefallene "`Vaterlandsverräter"'\footnote{
        Die Zeit, 26.04.1985, S.\,3.%
}
\ex Hat er Kontakte zu politisch negativ aufgefallenen Personen?\footnote{
   Klaus Kordon, \emph{Krokodil im Nacken}, Wildheim, Basel, Berlin: Beltz Verlag, 2002, S.\,595.%
}
\zl
In den Beispielen in (\mex{0}) wird die Subjektrolle des adjektivischen Partizips vom
modifizierten Nomen gefüllt. Das unterscheidet diese Verben von transitiven Verben,
bei denen die Objektrolle des Partizips vom modifizierten Nomen gefüllt wird:
\eal
\ex die geliebte Frau
\ex der geschlagene Hund
\zl
In (\mex{0}a) ist die Frau diejenige, die geliebt wird, und in (\mex{0}b)
wird der Hund geschlagen. Verben ohne Akkusativobjekt bilden normalerweise
keine adjektivischen Passivpartizipien:
\eal
\label{ex-prenominal-nerg}
\ex[*]{
der getanzte Mann\NOTE{JB: der getanzte Walzer, noch Fußnote?}
}
\ex[*]{
der (ihm) geholfene Mann
}
\zl

\noindent
Man hat festgestellt, daß Argumente bestimmter Verben, die in Aktivsätzen im Nominativ realisiert
werden, Objekteigenschaften haben. Die entsprechenden Verben werden \emph{unakkusativisch}
\citep{Perlmutter78}\ia{Perlmutter} bzw.\ \emph{ergativ}\is{Verb!ergatives} (siehe \zb \citew{Grewendorf89a} und dort
zitierte Publikationen\nocite{Pullum88a}) genannt. Beispiele sind die bereits diskutierten Verben
\emph{ankommen} und \emph{auffallen}. Intransitive Verben, die nicht unakkusativisch sind, werden
\emph{unergativisch}\is{Verb!unergativisches} genannt. Beispiele für unergativische Verben sind
\emph{tanzen} und \emph{helfen}. \citet{Grewendorf89a}\ia{Grewendorf} stellt
vierzehn Tests zur Unterscheidung unakkusativischer von unergativischen Verben vor.
\citet{Fanselow92}\ia{Fanselow} fügt diesen Tests sechs weitere hinzu.  Trotz dieser großen Anzahl
von Tests ist man sich auch heute nicht in allen Fällen einig, ob bestimmte Verben als
unakkusativisch oder als unergativisch einzuordnen sind.
% auch Abb94, Wegener90, RHL96a
\citet{Kaufmann95a}\iaright{Kaufmann}
zeigt, daß viele der angeblichen Unterschiede zwischen unakkusativischen und unergativischen Verben
auf eine Weise zu erklären sind, die nichts mit der Unakkusativ/""Unergativunterscheidung zu tun
hat. Siehe auch \citew[Kapitel~2]{Abraham2005a}.

Die Verhältnisse in \fromto{\ref{ex-passive-erg}}{\mex{0}} lassen sich unabhängig von
der exakten Definition von Unakkusativität einfach erklären, wenn man annimmt,
daß das Subjekt der Verben in (\ref{ex-passive-erg}) ein zugrundeliegendes Objekt ist.
% FB: ist okay.
%% \NOTE{Iwanowa: letzter Abs.: die Argumentation ist verwirrend und unverständlich. Die Zeilen 3 bis 6 scheinen alle das gleiche zu beschreiben, ohne in der Argumentation einen Schritt weiter zu gehen.}
Wenn Passiv als Unterdrückung des Subjekts gesehen wird, ist klar, warum die Passivierung dieser
Verben nicht möglich ist: Da sie kein Subjekt im relevanten Sinne besitzen, kann kein Subjekt
unterdrückt werden. Wenn die logischen Subjekte von \emph{ankommen} und \emph{auf"|fallen}
zugrundeliegende Objekte sind, können sie bei einer Passivierung nicht unterdrückt werden,
und die Passivierung ist deshalb ausgeschlossen.\footnote{
        Unter bestimmten Umständen sind auch unakkusativische Verben
        passivierbar. Siehe hierzu Abschnitt~\ref{sec-passive-unakkusativ}.
%        und \citew[\page290]{Mueller99a}\ia{Müller}.
}
Die Bildung von adjektivischen Partizipien ist möglich, wenn es ein Element mit
Akkusativobjekteigenschaften gibt. Da angenommen wird, daß die Subjekte
von \emph{ankommen} und \emph{auf"|fallen} zugrundeliegende Objekte sind,
ist die Wohlgeformtheit der Phrasen in (\ref{ex-prenominal-erg}) erklärt. Der Kontrast zwischen
Adjektivbildungen wie (\ref{ex-prenominal-erg}) und (\ref{ex-prenominal-nerg}) ist recht klar,
weshalb sich die Möglichkeit der Verwendung eines Verbs als attributives Partizip II gut als
Unakkusativitätstest eignet.%
\is{Passiv!adjektivisches Partizip|)}


Ein weiterer Test für die Unterscheidung von unakkusativischen und unergativischen
Verben ist ihr Verhalten in Resultativkonstruktionen:\is{Resultativkonstruktion|(}
Resultativkonstruktionen werden normalerweise mit einem mono"=valenten bzw.\
einem Verb mit nur einem realisierten Argument, einem zusätzlichen Akkusativelement
und einem zusätzlichen Prädikat (einem Adjektiv oder einer Präpositionalphrase)
gebildet (\citealp{Wunderlich97c}; \citealp[Kapitel~5]{Mueller2002b}):
\eal
\ex  Heute verzichten die Hooligans vor und beim Fußballspiel auf Alkohol und trinken\iw{trinken} 
      erst nach dem Spiel \emph{ganze Kneipen} leer\iw{leer}.\label{ex-hooligans-trinken-kneipen-leer}\footnote{
        Mannheimer Morgen, 16.07.1998, Politik; Kanther sagt Hooligans den Kampf an
%       M98/807.58811 Mannheimer Morgen, 16.07.1998, Politik; Kanther sagt Hooligans den Kampf an
}
\ex\label{ex-hintern-platt} "`Als ich anfing, wollte ich mir eigentlich \emph{den Hintern} nicht so plattsitzen 
wie die älteren Grufties"', sagt Pape.\footnote{
        taz-Bremen, 07.02.1997, S.\,21.% Nr. 5267 Seite 21 vom 02.07.1997 
}
\ex Erinnern Sie sich an A Fish Called Wanda, wo genußvoll \emph{ein Hündchen nach dem anderen} plattgefahren wurde?\label{ex-huendchen-platt-fahren}\footnote{
        taz-Bremen, 03.03.1990, S.\,27.% Nr. 3048 Seite 27 vom 03.03.1990
        }
\ex\iw{sturmreif}\iw{schießen}
Ihre Artillerie hatte von  den umliegenden Bergen    \emph{die Stadt} sturmreif  geschossen.\label{ex-hatte-die-stadt-sturmreif-geschossen}\footnote{
taz, 15.07.1995, S.\,11.% T950715.92 TAZ Nr. 4670 Seite 11 vom 15.07.1995
}
\zl
Das Resultativprädikat (\emph{leer}, \emph{platt}, \emph{sturmreif}) sagt etwas über den Resultatszustand aus,
der durch das vom Verb beschriebene Ereignis bewirkt wird.
Wie die Beispiele zeigen, muß das Akkusativelement in diesen Konstruktionen nicht unbedingt
ein Argument des Verbs sein: Die Kneipen in (\ref{ex-hooligans-trinken-kneipen-leer}) 
sind nicht das Objekt von \emph{trinken}, und genausowenig ist der Hintern in (\ref{ex-hintern-platt}) das Objekt von
\emph{sitzen} und sind die Hunde in (\ref{ex-huendchen-platt-fahren}) das Objekt von \emph{fahren}. Städte können nicht das Objekt
von \emph{schießen} sein, statt \emph{schießen} müßte man das Verb \emph{beschießen} verwenden, wenn \emph{die Stadt}
Objekt sein soll. Sind die Verben in der Resultativkonstruktion unergativisch, dann prädiziert
das Resultativprädikat immer über das Akkusativelement (in (\mex{0}) kursiv gesetzt). Bei unakkusativischen Verben
prädiziert das Resultativum dagegen über das Subjekt des Verbs:
%% \eal
%% \label{ex-result-unacc}
%% %\item * Der Toast verbrannte schwarz.\footnote{
%% %        \citew[p.\,461]{Wunderlich95a}\iaright{Wunderlich}
%% %        }
%% \ex\iw{schmelzen}
%% Die Butter schmilzt zu einer Pfütze.
%% \ex\iw{erstarren}
%% Sein Gesicht erstarrt zu einer Maske.
%% \ex\iw{zerfallen}
%% Die Vase zerfällt in Stücke.
%% \ex\iw{frieren}
%% Die Milch friert zu einem Block.
%% \zl
\eal
\label{ex-unaccusative-result-zu-pp}
\ex\label{ex-fror-zu-eis} %Die Seewiese und ihre Geschichte +d+u Wo heute der "Taumler" taumelt und Würstchen mit Senf über die Theken gehen, war mal ein See. Er
%umgab Friedberg vom Westen her 
% the lake.meadow and her story where today the
% staggerer staggers and sausages with mustard over the
% counters go was once a lake. He surrounded Friedberg
% from.the west (to)
{}[\ldots] und im Winter fror \emph{sein Wasser} zu Eis.\footnote{
        Frankfurter Rundschau, 16.09.1999, S.\,3.%
}
\ex  
Dann erzählt Juliane Lumumba von den Tonbändern im Archiv, \emph{die} wegen fehlender Klimaanlage in der tropischen Hitze
     zu einer schwarzen Masse schmolzen.\footnote{
         Frankfurter Rundschau, 05.08.1997, S.\,3.%
}
\ex 
Dann ging mal das Schreibpapier aus oder \emph{die bestellte Ladung Kerzen} war zu Wachs geschmolzen, 
     ehe sie den Hafen erreicht hatte.\footnote{
Frankfurter Rundschau, 28.02.1998, S.\,8.%
}
\ex 
In einer derartigen Gesamtrechnung schmilzt \emph{manche Steuerfussdifferenz} zu einer Lappalie.\footnote{
Züricher Tagesanzeiger, 04.01.1997, S.\,1.%
}
\zl
Diese Daten sind wieder erklärt, wenn man annimmt, daß das Resultativprädikat
in Resultativkonstruktionen immer über das Element prädiziert, das Objekteigenschaften
hat. Das heißt, die Subjekte in (\mex{0}) sind keine normalen Subjekte, sondern
zugrundeliegend eigentlich Objekte.\is{Resultativkonstruktion|)}
%%
%% Im Deutschen sind Resultativkonstruktionen mit unakkusativischen Verben 
%% \zb im Vergleich mit dem Englischen relativ eingeschränkt. Resultativkonstruktionen
%% mit unakkusativischen Verben und Adjektiven sind oft schlecht, sie sind
%% aber prinzipiell möglich, wie die Beispiele in (\mex{1}) zeigen:
%% \ea\iw{hart}\iw{frieren}\label{ex-schneeheide}
%% Solange [\ldots], deckt man die Erde zwischen den Pflanzen in jedem Herbst mehrere Zentimeter mit humosem Material ab.
%% Darunter friert die Erde weniger hart, 
%% und im Sommer bleibt sie selbst am sonnigen und warmen Platz so feucht und kühl, wie Schneeheide es liebt.\footnote{
%% Mannheimer Morgen, 02.01.1999, Ratgeber; Eine robuste, kleine Winterschönheit.%
%% }
%% \z
%% \citet[\page22]{KW98a} haben festgestellt, daß Adjektivprädikate besser werden, 

%Resultativkonstruktionen werden im Kapitel~\ref{chap:result} genauer diskutiert.%


Bevor ich mich in den nächsten Abschnitten verschiedenen Passivformen zuwende,
möchte ich die Hilfsverbselektion\is{Hilfsverb!-selektion} diskutieren, die in der Literatur auch als
eins der Kriterien zur Unterscheidung unakkusativischer und unergativischer Verben verwendet
wird.\footnote{
  Siehe auch \citew{Ryu97a} zur Hilfsverbselektion im besonderen und
  zu Unakkusativitätstests im allgemeinen.
}
Normalerweise bilden unakkusativische Verben das Perfekt\is{Perfekt} mit \seinp
und unergativische Verben bilden das Perfekt mit \habenp. Es ist jedoch nicht sinnvoll,
sowohl Passivierbarkeit als auch die Auxiliarwahl zu den definierenden Kriterien für
Unakkusativität zu zählen, da zum Beispiel Bewegungsverben\is{Verb!Bewegungs-}
das Perfekt mit \sein bilden, aber dennoch passivierbar sind, wie die folgenden Beispiele zeigen:
% HR2003a:217 sehen Auxiliarwahl als Ergativitätstest an
\begin{sloppypar}
\eal
\ex\iw{fahren}
Gestern wurde allerdings noch gefahren, wenn auch erst mit Verzögerung.\footnote{
        taz, 25./26.07.1998, S.\,1, Bericht über die Tour de France.%%
}
\ex\iw{marschieren}
Aber es wurde damals ununterbrochen marschiert.\footnote{
        taz, berlin, 02.02.2000, S.\,19.%
}
\ex\iw{landen}
Im Norden kann nur gelandet werden.\footnote{
         taz, 05./06.02.2000, S.\,8.%
}
\ex\iw{schwimmen}
In allen anderen Gewässern Berlins und Brandenburgs kann gefahrlos geschwommen und geschluckt werden.\footnote{
         taz, 16./17.06.2001, S.\,30.%
}
%% Zustandspassiv
%% \ex
%% {}[\ldots] wenig später ist die letzte Bahn geschwommen.\footnote{
%%         taz berlin, 21./22.12.2002, S.\,27 Artikel über die Schließung des SEZ.%
%% }
\ex Er schaltete auf einen anderen Sender um [\ldots] Ich sah eine Weile
zu, aber es wurde überhaupt nicht geredet, nur weggerannt.\footnote{
        Jochen Schmidt, \emph{Müller haut uns raus.} München: Verlag C.\,H.\,Beck. 2002, S.\,140.%
}
\zl
\end{sloppypar}

\noindent
Die Verben in \fromto{\mex{1}}{\ref{ex-nachgekommen-werden}} sind homonym mit
Bewegungsverben, haben aber eine andere Bedeutung. Sie bilden wie die Bewegungsverben
das Perfekt mit \sein und sind passivierbar:
\eal
\ex\iw{verfahren}
Ich bin so verfahren, daß \ldots\footnote{
        \citew[\page764]{Duden-rs-91}.%
}
\ex 
"`Hier muß sensibel verfahren werden."'\footnote{
        taz berlin, 11.08.1998, S.\,17.%
}
\zl
\eal
%\ex Seit gestern ist man im Theater am Schlachthof in Neuss wieder zur Tagesordnung übergegangen.10.2.2005 taz NRW Spezial 110 Zeilen, PETER ORTMANN S. I
\ex Danach wäre man wieder zur Tagesordnung übergegangen: Abgase in die Luft blasen, schubkarrenweise FCKW in den Wald schütten, Ozonloch vergrößern.\footnote{
  taz, 29.01.2005, S.\,25.
}
\ex "`Am Montag kann auf keinen Fall einfach zur Tagesordnung übergegangen werden."'\footnote{
        taz, 29.02.2002, S.\,21.
}
\zl
\eal
\label{bsp-eingegangen-all}
%\ex Er ist einen Kompromiß eingegangen.
\ex\iw{eingehen}
"`Wir sind eine vertragliche Verpflichtung eingegangen, und zu dieser stehen wir"' [\ldots]\footnote{
        taz, 6.3.2002, S.\,9.%
}
%, erkla"rte Gu"nter Marquis, Pra"sident des Verbandes der Elektrizita"tswirtschaft (VDEW) gestern in Berlin.
%taz Nr. 6693 vom 6.3.2002, Seite 9, 84 Zeilen (TAZ-Bericht), NICK REIMER
\ex\label{bsp-eingegangen}
Für jeden Job, [\ldots] bei dem Verantwortung übernommen werden oder hin und wieder gar
    ein Kompromiss eingegangen werden muss, ist der ehemalige Finanzminister absolut ungeeignet.\footnote{
        taz, 28.05.2002, S.\,14.%
    }
% Es wird also hier mit der folgenden Hypothese angetreten [\ldots] WL2001a:9
% \ex "`Es ist unfair, wenn der Trainer von unseren Anhängern verbal angegangen wird."'\footnote{
%         Mannheimer Morgen, 23.11.1998.%
%      }
%
% auch: die eingegangene Verpflichtung  www.dse.de/za/lis/ruanda/seite2.htm - 16k - 27. Juni 2004
\zl
\eal
\label{bsp-angegangen-all}
\ex\iw{angehen}
"`Wären wir beim Ocean Race so gesegelt, wie wir die Kampagne um den America's Cup angegangen sind, 
hätten wir das Ziel nicht erreicht"', musste er eingestehen.\footnote{
        taz Hamburg, 11.6.2002, S.\, 24.% %, 99 Zeilen (TAZ-Bericht), OKE GO"TTLICH
}
%
\ex\label{bsp-angegangen}
Ob die finanziell
aufwendige Restaurierung nun tatsächlich angegangen wird oder ob die Wandmalereien lediglich fachgerecht konserviert werden, hat der
Heidelberger Gemeinderat demnächst zu entscheiden.\footnote{
        Mannheimer Morgen, 20.01.1989.%
}
\zl
\eal
\ex Wie "`Heise Online"' berichtet, ist Hormel bereits gegen die Software"=Marke Spam Arrest vorgegangen.\footnote{
  taz, 25.05.2005, S.\,20.%
}
\ex\iw{vorgehen}
"`Gegen Sozialhilfemissbrauch wird künftig konsequent vorgegangen"', sagt sie, als ob nicht schon der rot-grüne Senat 
     Sozialhilfeempfängern unangemeldete Kontrolleure ins Haus geschickt hätte.\footnote{
         taz Hamburg, 27.6.2002, S.\,21.%
    }%
%
\ex 
daß bei einem solchen Delikt gegen Autobahnpolizisten vorgegangen wird.\footnote{
        Frankfurter Rundschau, 12.09.1998, S.\,31.%
}
\zl
\eal
\label{ex-nachgekommen-werden}
%Der Rechnungshof bema"ngelte weiter, das Bundesfinanzministerium sei "seiner Verpflichtung zur Pru"fung der Lage des Unternehmens" nur unzureichend nachgekommen.taz Nr. 6776 vom 17.6.2002, Seite 8, 59 Zeilen (TAZ-Bericht), REM
\ex\iw{nachkommen}
Dem ist die taz nachgekommen.\footnote{
        taz, 08.06.2002, S.\,24.%
    }
\ex 
den finanziellen Verpflichtungen kann nicht nachgekommen werden\footnote{
        St. Galler Tagblatt, 30.09.1999.%
        }
\zl
\eal\iw{entgegentreten}
\ex Er ist dem entgegengetreten.
\ex Diesem Qualitätsdumping muss wirksam entgegengetreten werden.\footnote{
        taz, 02./03.08.2003, S.\,26.
}
\zl
% angehen mit sein HWG
Man beachte, daß sowohl in \pref{bsp-eingegangen} als auch in \pref{bsp-angegangen} ein persönliches
Passiv vorliegt. \emph{angehen} und \emph{eingehen} verlangen in Aktivsätzen einen Akkusativ.\footnote{
\citet[\page9]{Grewendorf89a} gibt die folgenden Beispiele für Verben, die eine Passivierung erlauben,
ihr Perfekt mit \sein bilden und ein Akkusativobjekt regieren:
\eal
\ex\iw{durchgehen}
Ich bin die Arbeit durchgegangen.
\ex 
Er ist den Bund fürs Leben eingegangen.
\ex\iw{ablaufen}
Er ist die ganze Stadt abgelaufen.
\ex\iw{fliehen}
Sie ist ihn geflohen.
\ex\iw{angehen}
Sie ist ihn angegangen.
\zl
Das letzte Beispiel ist ambig zwischen der Lesart von \emph{angehen},
die auch in (\ref{bsp-angegangen-all}) vorliegt, und der Lesart, die
körperliche Aggression ausdrückt.
Ein Beispiel mit \emph{durchgehen} findet man auch bei \citet[\page385]{Toman86a}.
Die Beispiele in (i) zeigen übrigens auch, daß die Generalisierung, wonach
Verben genau dann das Perfekt mit \emph{sein} bilden, wenn ihr Partizip 2 bezüglich des 
\hyperlink{externesArgument}{externen Arguments}\is{externes Argument} %(= Subjekt)
attributierbar ist \citep[\page110]{Gunkel2003b}, nicht ohne Ausnahme ist.%
}

Man kann jedoch auf Grundlage dieser Daten nicht einfach die Generalisierung formulieren,
daß alle Verben, die homonym mit Bewegungsverben sind, das Perfekt mit
\sein bilden, da es auch Beispiele wie (\mex{1}) gibt:
\ea
Ein Organisator im Bundesstaat Iowa hat dieses Problem umgangen.\footnote{
 Mannheimer Morgen, 30.05.1989.%
}
\z
Ich denke, daß es eher korrekt ist, davon auszugehen, daß Verben, die
ein Akkusativobjekt regieren, das Perfekt mit \haben bilden. 
\emph{angehen}, \emph{durchgehen} und \emph{eingehen} müssen dann als
Ausnahmen behandelt werden.%
\is{Hilfsverb!-selektion|)}%
\is{Verb!unakkusativisches|)}

Nachdem ich gezeigt habe, daß Passivierbarkeit und Hilfsverbselektion, was die
Unakkusativ/""Unergativunterscheidung angeht, unabhängig voneinander sind,
wende ich mich nun den verschiedenen Arten von Passiv zu.
%
% \subsection{Ergativity}
%
% It has been observed that dependents of certain verbs that have nominative case nevertheless behave
% like objects. Such verbs are called unakkusativisch\is{verb!unakkusativisch} \citep*{Perlmutter78}\ia{Perlmutter}
% or unakkusativisch\is{verb!ergative|(}\is{verb!unakkusativisch|(}. \citet*{Grewendorf89a}\ia{Grewendorf} provides fourteen tests
% for distinguishing unakkusativisch from unergativisch verbs. 
% \citet*{Fanselow92}\ia{Fanselow} adds another six. Despite this large number of tests 
% what has to be counted as an unakkusativisch verb is by no means an uncontroversial issue. \citet{Kaufmann95a}\iaright{Kaufmann}
% shows that many of the alleged differences between unakkusativisch and unergativisch verbs have to be explained
% by means that are not related to the proposed unakkusativisch/unergativisch distinction.

% One property of unakkusativisch verbs that is important for the present discussion is that they cannot be passivized.\footnote{
%         For a discussion of certain exceptional passivizations
%         of unakkusativisch verbs that have a special reading
%         see \citew[\page350]{Ruzicka89}\ia{Ruzicka@R\r{u}\v{z}i\v{c}ka}
%         and \citew[\page290]{Mueller99a}\ia{Müller}.
% }
% \eal
% \ex[]{\iw{ankommen}
% Karl kam  an.
% % }
% \ex[*]{
% Dort wurde angekommen.
% }
% \ex[]{\iw{auf"|fallen}
% Er fiel ihr auf.
% % }
% \ex[*]{
% Ihr wurde aufgefallen.
% }
% \zl
% Since this is also true for theme verbs\is{verb!theme}, \citet*[\page184]{Grewendorf89a}\ia{Grewendorf} does
% not count non"=passivizability as a defining test for the class of unakkusativisch verbs. \citet[\page90]{Wegener90}\iaright{Wegener}
% shows that theme verbs share a lot of the properties of unakkusativisch verbs and therefore it might be reasonable
% to regard the verbs that are usually referred to as unakkusativisch and the theme verbs as members of one class. I leave
% this question open here and concentrate on the clear cases in what follows.


\subsection{Vorgangspassiv}
\label{sec-agentive-pas}


% Kathol91a
Eine Anforderung an Subjekte in Passivkonstruktionen scheint zu sein, daß
die logischen Subjekte der passivierten Verben referentiell sein müssen. Das Beispiel
in (\mex{1}) zeigt, daß die Passivierung von expletiven\is{Pronomen!Expletiv-} Prädikaten nicht
möglich ist.
\ea[*]{\iw{regnen}
\label{ex-heute-wurde-geregnet}
Heute wurde geregnet.
}
\z
Die Beispiele in (\mex{1}) scheinen dem zu widersprechen:
\eal\label{ex-passiv-mit-regnen+praed}
\ex Naja, wirklich lange Zeit schaffen wir's den schweren Wolken auszuweichen und
werden nur ein bissl angeregnet.\footnote{
\url{http://xor.at/scand2k5-diary.shtml}. \urlchecked{09}{11}{2005}.
}
\ex {}[\ldots] denn egal wie schnell man läuft, man wird immer gleich
stark von der einen Seite angeregnet.\footnote{
  \url{http://www.swr3.de/fun/alltagsfragen/question.php?fid=65&mode=a}. \urlchecked{09}{11}{2005}.
}
\ex es ist nicht schlimm, wenn die Karte nassgeregnet wird.\footnote{
\url{http://www.agrar.de/pferde/forum/index.php?topic=2065.30}. \urlchecked{09}{11}{2005}.%
}
\ex Ich hoffe mal, daß wir dieses Jahr auch mal was vom Osterfeuer haben und nicht wieder nassgeregnet werden.\footnote{
  \url{http://www.jens-waltermann.de/wetten/1017219329.shtml}. \urlchecked{09}{11}{2005}.
}
\ex Nicht das wir etwas davon auf der Bühne abbekommen würden, 
    aber das kann man ja nun auch nicht mit ansehen, wie die Zuschauer draussen vollgeregnet werden.\footnote{
\url{http://www.crazycrackers.de/NEWS.htm}. \urlchecked{09}{11}{2005}.
}

\zl
In den Beispielen in (\mex{0}) wurde \emph{regnen} mit einem weiteren Prädikat verbunden, welches
selbst ein Argument in die Verbindung einbringt. Das Ergebnis der Kombination ist ein Verb,
das ein Expletivum verlangt, und das Argument, das von \emph{an}, \emph{naß} oder \emph{voll}
lizenziert wird. (\mex{1}) zeigt ein Beispiel für eine Verwendung eines solchen Verbs im Aktiv:
\eal
\ex da hats garnix gemacht, dass es uns auf dem hinweg nassgeregnet hat.\footnote{
 \url{http://www.jeg-board.de/index.php?name=PNphpBB2&file=viewtopic&t=52}. \urlchecked{09}{11}{2005}.%
}
\ex Den Beerengeist habe ich auch gebraucht, weil es uns bei der Besichtigung des Barfußpfads furchtbar nassgeregnet hat.\footnote{
 \url{http://f1.parsimony.net/forum994/messages/31207.htm}. \urlchecked{09}{11}{2005}.%
}
\ex Letzte Nacht hat es den Teich vollgeregnet,\footnote{%
\url{http://www.timmendorfer.de/pforum/showthread.php?id=160&eintrag=10}. \urlchecked{09}{11}{2005}.%
}
\zl
Es scheint nun so zu sein, daß in (\ref{ex-passiv-mit-regnen+praed}) Verben, die zusätzlich zu einem expletiven Subjekt noch
ein weiteres Argument verlangen, passiviert worden sind.

Die Sache liegt jedoch nicht so einfach, denn wie die \emph{von}"= bzw. \emph{durch}"=Phrasen in (\mex{1}) zeigen,
muß man für einige Sätze in jedem Fall ein referentielles Subjekt im Aktiv annehmen:
\eal
\label{bsp-regnen+resultativ-aktiv}
%% \ex wenn man dabei 2 kinder gleich mit durchschießt, und die anderen mit gehirn und blut vollgeregnet werden
%% http://www.ruhrpottforum.de/archive/29916/thread.html
\ex eins was von einer dicken schwarzen Wolke (ohne Regenschirm) nassgeregnet wird.\footnote{
  \url{http://forum.htpc-news.de/archive/index.php?t-431.html}. \urlchecked{09}{11}{2005}.
}
\ex denn unter beiden Kirchendächern kann jemand, der anfängt zu denken und aufhört
zu glauben, durch manchen unliebsamen Regen böse naßgeregnet werden.\footnote{
  \url{http://www.meinhard.privat.t-online.de/frauen/meine_suche.html}. \urlchecked{09}{11}{2005}.
}
\ex Nachdem Dario einen dort stehenden Baum volle Kanne abschoss und wir von einem kleinen
"`Blätterschauer"' vollgeregnet wurden,\footnote{
\url{http://www.hentscholin.de/archiv/08.11.02.txt}. \urlchecked{09}{11}{2005}.
}
\ex Wenn er von nem Kübel Kotze vollgeregnet wird\footnote{%
\href{http://forum.esistjuli.de/viewtopic.php?t=1247&postdays=0&postorder=asc&start=0&sid=4ca847b01bd6fc22d2c771fe4d186128}{http://forum.esistjuli.de/viewtopic.php?t=1247\&postdays=0\&postorder=asc\&start=0\&}""%
\href{http://forum.esistjuli.de/viewtopic.php?t=1247&postdays=0&postorder=asc&start=0&sid=4ca847b01bd6fc22d2c771fe4d186128}{sid=4ca847b01bd6fc22d2c771fe4d186128}. \urlchecked{09}{11}{2005}.%
}
\zl

\noindent
Die Beispiele in (\mex{1}) zeigen weitere Belege dafür, daß man Verben wie \emph{abregnen} mit einem referentiellen
Subjekt verwenden kann.
\eal
\ex Nachdem Hurrican Floyd seine unglaublichen Regenmassen über North Carolina abgeregnet hatte, überschwemmte der Tar-River das Umland.\footnote{
 taz, 25.09.1999, S.\,5.
}
\ex Bis dorthin ist die radioaktive Wolke auch nicht gekommen. 
    Sie sei künstlich über der Stadt Gomel abgeregnet worden, behaupten nicht wenige aus der Bürgerinitiative "`Kinder von Tschernobyl"'.\footnote{
taz, 12.11.1990, S.\,14.
}
\zl
Im Fall von (\mex{0}b) deutet das \emph{künstlich} darauf hin, daß es menschliche Verursacher gab, die
die Wolke abgeregnet haben.

Da man bei den Fällen in (\ref{ex-passiv-mit-regnen+praed}) das Aktiv"=Subjekt nicht sehen kann,
ist es durchaus denkbar, daß hier eine Variante von \emph{regnen} + \emph{an}/""\emph{ab}/""\emph{naß}/""\emph{voll}
passiviert wurde, die ein referentielles Subjekt hat.
Wir können also bei der Aussage bleiben, daß expletive Subjekte nicht durch Passivierung unterdrückt
werden können.


(\mex{1}) zeigt, daß Verben, die überhaupt kein Subjekt haben, ebenfalls nicht passiviert werden können:
% Jacobs94a:312 zitiert auch Zifonun92
\eal
\ex[]{\iw{grauen}
Dem Student graut (es) vor der Prüfung.
}
\ex[*]{
Dem Student wird (es) (vom Professor) vor der Prüfung gegraut.
}
\zl
Die Ungrammatikalität von (\ref{ex-heute-wurde-geregnet}) und (\mex{0}b) folgt
aus der Annahme, daß Passivierung die Unterdrückung einer Subjektrolle\label{ex-ps-demotion-subject-role}
ist \citep[\page307]{ps2}\iaright{Pollard}\iaright{Sag}. Da das Subjekt von \word{regnen}
keine semantische Rolle\is{semantische Rolle} hat und da \word{grauen} nur ein optionales Expletivum
als Subjekt hat und somit auch keine
Subjektrolle vergibt, können diese Verben nicht passiviert werden.
Es folgt auch, daß Subjektanhebungsverben\is{Anhebung} wie \zb \emph{scheinen} nicht passiviert
werden können:\label{subjekt-raising-kein-passiv}
% BJR89a:231 geben Beispiele für Passivierung von Anhebungsverben, unakkusativen und auch
% Doppelpassivierungen
% Litauisch, Türkisch, Sanskrit
\ea[*]{
weil dann zu schlafen geschienen wurde
}
\z
Die Beispiele in (\mex{1}) scheinen dem zu widersprechen:
\eal
\ex Nachdem angefangen worden ist, das teure, architektonisch umstrittene Gebäude nach und nach zu übergeben, 
    stellt man fest, daß die neue Bühnentechnik in keiner Phase des Einbaus auf ihre Eignung geprüft worden ist.\footnote{
Mannheimer Morgen, 21.04.1989, Feuilleton; Nichts geht mehr an der Bastille-Oper.%%M89/904.12899: Mannheimer Morgen, 21.04.1989, Feuilleton; Nichts geht mehr an der Bastille-Oper
        }
\ex Seine Kritik richtete sich daran aus, daß leider -- wie immer -- dann zuallererst am Personal angefangen wird zu sparen.\footnote{
       Mannheimer Morgen, 05.05.1989, Lokales; Den freien Samstag verteidigen.%% M89/905.14607: Mannheimer Morgen, 05.05.1989, Lokales; Den freien Samstag verteidigen
}
\zl
Aber wie bereits in Fußnote~\ref{fn-phase-verbs-control} auf S.\,\pageref{fn-phase-verbs-control} festgestellt
wurde, haben Verben wie \word{anfangen}\is{Verb!Phasen-} sowohl eine Anhebungs-\is{Verb!Anhebungs-} als auch
eine Kontrollversion\is{Verb!Kontroll-} (Siehe auch \citew{Perlmutter70}\iaright{Perlmutter} für das Englische\il{Englisch}
und \citew[\page658--659]{Suchsland87a} für das Deutsche). In (\mex{0}) liegt die Kontrollversion von \emph{anfangen} vor.
% VG wird auch von Hoekstra87a:155 behauptet
Die Daten in (\mex{0}) widersprechen also nicht der Behauptung, daß Anhebungsverben nicht passiviert
werden können. Sie zeigen jedoch klar, 
daß Vissers\aimention{Fredericus Theodorus Visser} Generalisierung\is{Vissers Generalisierung}\footnote{
        \citet[\page402]{Bresnan82c}\ia{Bresnan} zitiert Visser mit der folgenden Behauptung:
        Verben, deren Komplemente über das Subjekt des Verbs prädizieren, können nicht passiviert werden.

        \citet[\page304]{ps2}\ia{Pollard}\ia{Sag} geben als Vissers Generalisierung an,
        daß Subjektkontrollverben\il{Englisch} nicht passiviert werden können.
        Sie diskutieren (i):
        \eal
        \ex[]{
        Kim was persuaded to leave (by Dana).
        }
        \ex[*]{
        Kim was promised to leave (by Dana).
        }
        \zllast
}
für das Deutsche nicht zutrifft.
(\mex{1}) zeigt zusätzlich noch Passivbeispiele mit dem Subjektkontrollverb\is{Verb!Kontroll-}
        \emph{versprechen}:
%\begin{sloppypar}
\eal
% Es sei aber von beiden Seiten für nächstes Jahr versprochen worden, den Kinderfasching im RNZ ganz neu und vielleicht auch schöner
% und stilvoller zu feiern. 
% M99/902.07778 Mannheimer Morgen, 05.02.1999, Lokales; Kinderfasching im nächsten Jahr
\ex 
Wie oft schon wurde von der Stadtverwaltung versprochen, Abhilfe zu schaffen.\footnote{
        Mannheimer Morgen, 13.07.1999, Leserbriefe; Keine Abhilfe.%%M99/907.45513 Mannheimer Morgen, 13.07.1999, Leserbriefe; Keine Abhilfe
}
\ex Erneut wird versprochen, das auf eine Dekade angesetzte Investitionsprogramm mit einem Volumen von 630 Billionen Yen (10,5
Billionen DM) vorfristig zu erfüllen, [\ldots]\footnote{
    Süddeutsche Zeitung, 28.06.1995, S.\,28.% %U95/JUN.42005 Süddeutsche Zeitung, 028.06.1995, S. 28, Ressort: WIRTSCHAFT; Japans flaues Konjunkturprogramm
}
%wofür es allerdings noch sehr wenig Anzeichen gibt. 
%
% Nun sind solche Reformversprechen fast so alt wie die Stiftung Pro Helvetia selbst, geradezu rituell wird bei passender Gelegenheit
% versprochen, die unübersichtlichen Strukturen zu verschlanken, Leerläufe zu vermeiden, die Zusammenarbeit des Bundes mit anderen
% Förderern zu intensivieren und, wie sich die bundesrätliche Botschaft sehr höflich ausdrückt, "die langsamen und komplizierten
% Entscheidungswege von Pro Helvetia" zu straffen. 
% E99/MAI.12678 Züricher Tagesanzeiger, 14.05.1999, S. 5, Ressort: Schweiz; Mehr Geld, weniger Umstände
\zl
%\end{sloppypar}
In (\mex{0}a) wird das logische Subjekt durch eine PP ausgedrückt, und in (\mex{0}b) 
gibt es keine overte kontrollierende Phrase.
%
%Bresnan82c:404 -> das ist anaphoric control
%

%% Nach der Diskussion von Bedingungen für das Vorgangspassiv wende ich mich nun dem
%% Zustandspassiv zu und zeige, daß es für das Zustandspassiv ähnliche Beschränkungen gibt.

%% \subsection{Zustandspassiv}
%% \label{sec-passiv-state}
%% \is{Passiv!Zustands-|(}

%% Das Zustandspassiv drückt einen statischen Zustand aus, der das Resultat eines dynamischen 
%% Vorgangs ist.\footnote{
%%         Siehe \citet{Helbig87} für eine Diskussion von Beispielen, die oberflächlich
%%         dem Zustandspassiv gleichen, von Helbig aber nicht zum Zustandspassiv gezählt werden.
%%         \citet{Rapp97a} vertritt eine andere Ansicht.%
%% }
%% Wie \citet*[\page175]{HB72a}\ia{Helbig}\ia{Buscha} festgestellt haben, 
%% ist das Zustandspassiv nur dann möglich, wenn auch ein Vorgangspassiv möglich ist.
%% % Flaeming81:549 zitiert verschiedene Autoren
%% %
%% Die Umkehrung gilt jedoch nicht, wie \citet[\page776]{Juettner81}\ia{Jüttner},
%% \citet[\page261]{Zifonun92a}\ia{Zifonun} und \citet[\page145]{Eisenberg94a}\ia{Eisenberg}
%% gezeigt haben. Nicht jedes Verb, das passivierbar ist, erlaubt auch ein Zustandspassiv.
%% Die Bildung des Zustandspassivs ist ausgeschlossen, wenn das Subjekt
%% nicht in einen neuen Zustand versetzt wird. Das Zustandspassiv
%% kann also für Wahrnehmungsverben\is{Verb!Wahrnehmungs-} (\emph{riechen},\iw{riechen}
%% \emph{sehen},\iw{sehen} \emph{fühlen},\iw{fühlen} \emph{hören}\iw{hören}) 
%% und für andere Verben, die dieser Charakterisierung entsprechen, wie \zb \emph{loben},\iw{loben}
%% \emph{finden},\iw{finden}\NOTE{Aber die kleine Goma ist gefunden, und Nakata bringt sie 
%% jetzt zu den Koizumis zurück. Murakami Haruki, \emph{Kafka am Strand}, aus dem Japanischen von Ursula Gräfe, Köln: Dumont Literatur und Kunst Verlag, 2005, S.\,229}
%% \emph{verehren}\iw{verehren} und \emph{zeigen},\iw{zeigen}
%% nicht gebildet werden.\footnote{
%%         \citew[187--188]{Rapp97a}\iaright{Rapp} gibt eine formale Erklärung.
%% }$^,$\footnote{
%%   Die Beschränkung, daß das Objekt des Verbs in einen neuen Zustand gelangen muß,
%%   ist nur für nominale Objekte sinnvoll. (i) zeigt ein Beispiel für
%%   das Zustandspassiv mit Satzobjekt.
%%   \ea
%%   Dass das Unternehmen Marketingzahlen aufgemotzt hat, um mit der Imbisskette Burger King
%%   ins Geschäft zu kommen, ist bereits zugegeben. (taz, 28./29.02.2004, S.\,8)
%%   \zlast
%% }
%% Die Menge der Verben, die ein Zustandspassiv bilden, ist demzufolge eine echte Teilmenge
%% der passivfähigen Verben.

%% Das Zustandspassiv erlaubt, wie das Vorgangspassiv auch, persönliche (\mex{1}a,b)
%% und unpersönliche (\mex{1}c--f) Varianten:
%% \eal
%% \label{ex-stative-passive}
%% \ex\iw{öffnen}
%% Das Fenster ist geöffnet.
%% \ex\iw{zerzausen}
%% Seine dunkelbraunen Haare waren vom Wind zerzaust [\ldots]\label{ex-waren-vom-wind-zerzaust}\footnote{%, in seinen hellgrauen Augen schien es spöttisch aufzublitzen.
%%        Bolten, Y., {\em Komteß Silvia von Schönthal}. Hamburg, 1990, S.\,38, mit \cosmas gefunden.%GR1/TL1.02281 Bolten, Y.: Komteß Silvia von Schönthal. Hamburg, 1990, S. 38
%% }
%% % Brinker71:71
%% \ex\iw{denken} 
%% Nach diesem Plan ist nicht daran gedacht, große Mittel aufzuwenden.\footnote{
%%         \citew[\page41]{Hoehle78a}\ia{Höhle}.%
%% }
%% \ex\iw{servieren}
%% Es / jetzt ist serviert.\footnote{
%%         \citew[\page549]{Flaeming81a}\iadata{Flämig}.%
%% }
%% \ex\iw{reden}
%% Nun ist lange genug geredet.\footnote{
%%         \citew[\page224]{Wunderlich85}\iadata{Wunderlich}.%
%% }
%% \ex\iw{vorbeugen}
%% Der Energieverschwendung war durch dicke Isolierungen vorgebeugt.\footnote{
%%         \citew[\page41]{Hoehle78a}\ia{Höhle}.%
%% }
%% \ex\iw{helfen}
%% Dem Mann ist geholfen.
%% \zl
%% Beispiel (\ref{ex-waren-vom-wind-zerzaust}) zeigt, daß das Zustandspassiv
%% auch bei transitiven Verben mit unbelebtem\is{Belebtheit} Subjekt möglich ist.

%% Wie auch das Vorgangspassiv ist das Zustandspassiv nicht möglich, wenn
%% das Subjekt des eingebetteten Verbs ein Expletivum\is{Verb!expletives} ist:\footnote{
%%   Auch hier scheint es wieder Ausnahmen wie (i) zu geben:
%% \ea
%% Die Tische sind naß geregnet. \citep[\page131]{Mueller2002b}
%% \z
%% Zu einer Diskussion ähnlicher Fälle siehe Seite~\pageref{ex-passiv-mit-regnen+praed}.%
%% }
%% \ea[*]{\iw{regnen}
%% Ist heute geregnet?
%% }
%% \z
%% Das Zustandspassiv subjektloser Verben wie \emph{grauen} ist ebenfalls nicht möglich:
%% \ea[*]{
%% Dem Student ist (vom Professor) vor der Prüfung gegraut.
%% }
%% \z
%% %% The example in (\mex{1}) seems to be a remarkable exception.\is{predicate!resultative secondary}\is{verb!weather}\iw{sein!perfect}
%% %% \ea\iw{regnen}\iw{naß}
%% %% Die Tische sind naß geregnet.
%% %% \z
%% %% If one analyzes (\mex{0}) as a stative passive of the active sentence in (\mex{1}), an expletive subject\is{pronoun!expletive}
%% %% is suppressed in the stative passive.
%% %% \ea[?]{\label{ex-es-hat-die-tische-nass-geregnet}
%% %% Es hat die Tische naß geregnet.
%% %% }
%% %% \z
%% %% \citet[\page134]{Hoekstra88a}\ia{Hoekstra} und
%% %% Neeleman (\citeyear[\page133]{Neeleman94a}\iawrong{Neeleman}; \citeyear[\page227]{Neeleman95a}) 
%% %% diskutieren analoge Beispiele aus dem Niederländischen\il{Niederländisch}.
%% %% Siehe auch \citet[\page159, S.\,165]{Abraham93a} für eine Diskussion deutscher Daten.
%% %% Hoekstra, Neeleman und Abraham behandeln \emph{naß regnen} wegen der Auxiliarselektion\is{Auxiliarselektion}
%% %% als unakkusativisches Prädikat.
%% %% Damit wäre also (\mex{-1}) das Perfekt von (\mex{1}).
%% %% \ea
%% %% Die Tische regnen naß.
%% %% \z
%% %% Like in Dutch a perfect of such sentences cannot be formed with \haben\iw{haben!perfect} although
%% %% the perfect\is{perfect} of weather verbs without a resultative predicate must be formed with \haben.
%% %% %         Furthermore, an impersonal passive of (iv) is impossible as (v) shows:
%% %% %         \ea[*]{
%% %% %         Heute wurde naß geregnet.
%% %% %         }
%% %% %         \zlast

%% %% Diese Beispielklasse bleibt jedoch problematisch, da Sätze wie der folgende von Wunderlich (\citeyear[\page455]{Wunderlich95a}\iaright{Wunderlich}; \citeyear[\page118]{Wunderlich97c}) möglich zu sein scheinen:
%% %% \ea
%% %% Die Stühle wurden naß geregnet.
%% %% \z
%% %% If (\mex{0}) is the passive of \pref{ex-es-hat-die-tische-nass-geregnet}, one has to allow for passivization
%% %% of verbs with expletive subjectes. If, on the other hand, (\mex{0}) were the passive of (\mex{-1}), one would have
%% %% to allow passives of intransitive verbs that have an inanimate\is{animateness} subject.

%% \begin{comment}
%% Bevor wir uns anderen Passivvarianten zuwenden, soll noch kurz der Frage nachgegangen werden, ob es
%% sich beim Zustandspassiv eventuell gar nicht um ein Passiv handelt. So schlägt \citet{Abraham2005a}
%% vor, Sätze mit \emph{sein} + Partizip nicht zum Passiv zu rechnen, sondern sie einfach als
%% Kopulakonstruktionen\is{Kopula} zu analysieren, \dash als Verbindung aus der Kopula \sein und einem
%% Adjektiv. Dieser Vorschlag besticht durch seine Einfachkeit, es ist allerdings nicht zu sehen, wie
%% sich die Tatsache, daß das Subjekt des Hauptverbs genau wie bei anderen Passivvarianten auch durch
%% ein \emph{von}-PP ausgedrückt werden kann. Beispiele hierfür sind (\ref{ex-waren-vom-wind-zerzaust})
%% und (\mex{1}):
%% \eal
%% \ex Das Genozidverbot ist von der Staatengemeinschaft als völkerrechtliches Gewohnheitsrecht
%% anerkannt.\footnote{
%%   St. Galler Tagblatt, 28.01.1998; Zweimal Nein aus Innerrhoden  
%% }
%% \ex Pax Christi ist von den kroatischen und bosnischen Regierungen inzwischen als humanitäre
%% Hilfsorganisation anerkannt und arbeitet eng mit mit dem Flüchtlingswerk der Vereinten Nationen
%% zusammen.\footnote{
%% Frankfurter Rundschau, 10.05.1997, S. 3, Ressort: LOKAL-RUNDSCHAU; Pax Christi leistet praktische
%% Aufbauhilfe in Bosnien / Freiwillige Betreuer nach Schulung in den Flüchtlingslagern  
%% }
%% \ex Der MBA ist von der deutschen Kultusministerkonferenz anerkannt.\footnote{
%%   Frankfurter Rundschau, 18.04.1998, S. 22, Ressort: WIRTSCHAFTSSPIEGEL;  
%% }
%% \zl

%% IDS-Gramatik, Lenz94

%% (5) a. Die Zeichnung ist von einem Kind angefertigt.       Agensangabe
%%     b. * Die Zeichnung ist von einem Kind schön.
%% (6) a. Der Brief war mit roter Tinte geschrieben.    Instrumentalangabe
%%     b. * Der Brief war mit roter Tinte leserlich.
%% (7) a. Die Birnen waren in Rotwein gedünstet.              Lokalangabe
%%     b. * Die Birnen waren in Rotwein weich.
%% \end{comment}

%% \is{Passiv!Zustands-|)}

Nach der Diskussion des Vorgangs%- und des Zustands
passivs soll jetzt eine besondere
Passivvariante diskutiert werden: das Dativpassiv.



\subsection{Dativpassiv}
\label{sec:dat-pass}
\is{Passiv!Dativ-|(}

% Irgendwer hat mal behauptet, daß `lehren' ein Passiv erlaubt, daß dann aber
% keine Rektion von zwei Akkusativen, sondern von Akk und Dat vorliegt
%
% Die DSB-Funktionäre warfen Naiditsch und seiner Familie, in der auch
% die drei jüngeren Schwestern das Spiel sehr gut gelehrt bekamen,
% mehr Knüppel zwischen die Beine, als dass sie ihr größtes Talent im B-Kader förderten.
% taz, 12.08.2003, S.\,15

Eine besondere Variante des Passivs kann mit \emph{bekommen}\iw{bekommen!Passiv|(},
\emph{erhalten}\iw{erhalten!Passiv|(} und \emph{kriegen}\iw{kriegen!Passiv|(}
gebildet werden. In dieser Passivvariante wird eines der genannten Verben
mit einem Verb, das im Aktiv den Dativ regiert, kombiniert. Dieser Dativ wird dann
als Nominativ realisiert. Ein Beispiel ist in (\mex{1}b) zu sehen, wo
das Dativobjekt des Aktivsatzes (\emph{mir} in (\mex{1}a)) als Subjekt (\emph{ich})
realisiert wird.
\eal
\ex\iw{schenken}
Karl schenkt mir      ein Buch.
\ex\label{ex-ich-bekommen-ein-buch-geschenkt} 
Ich     bekomme ein Buch       geschenkt.
\zl
Daß die Bezeichnung Rezipientenpassiv, die man auch mitunter in der 
Literatur für das \emph{bekommen}"=Passiv findet,
% Grewendorf87a:125, Fanselow irgendwo
ungeeignet ist, zeigen die Sätze in (\mex{1}) und (\mex{2}).\footnote{
        Siehe hierzu auch \citew*[\page9, S.\,22]{Askedal84}\iaf{Askedal} 
        und \citew*[\page129]{Wegener85a}\iaf{Wegener}. 
        \citet*[\page371]{Eroms78}\iaf{Eroms} zitiert Fränkel\aimention{H. Fr{\"a}nkel} %1974
        mit (\mex{1}). %und Ebert 1978  mit (ib).
        Die Sätze in (\mex{2}) sind von \citet*[\page71]{Reis76a}\iaf{Reis}.
}
Die Sätze in (\mex{1}) und (\mex{2}) implizieren nicht, daß jemand etwas bekommt:
\ea\iw{ausschlagen}
\label{bsp-zaehne-ausgeschlagen}
Er bekam zwei Zähne ausgeschlagen.
\z
\eal
\label{bsp-bekommen-passiv-mit-akk}
\ex
Der Bub bekommt/kriegt das Spielzeug weggenommen.
%%"Wir kriegen die Pfarrer weggenommen", warnte eine andere. 
%R99/FEB.11401 Frankfurter Rundschau, 011.02.1999, S. 30, Ressort: N; Auch auf dem Dorf wird es immer weniger Pfarrer geben
\ex\iw{verbieten}
Der Mann bekommt/kriegt das Fahren  verboten.
\ex\iw{entziehen}
Der Betrunkene bekam/kriegte die Fahrerlaubnis     entzogen.
\zl
Die Bedeutung von \emph{bekommen} und \emph{kriegen} ist in diesen
Konstruktionen verblaßt. Es ist also nicht angebracht anzunehmen,
daß das Subjekt in solchen Konstruktionen ein Empfänger ist und
eine entsprechende thematische Rolle von \emph{bekommen}/""\emph{erhalten}/""\emph{kriegen}
bekommt. Auch die Forderung, daß das eingebettete Partizip dem
Dativ eine Rezipientenrolle zuweisen muß \citep[\page102]{Gunkel2003b},
ist nicht adäquat.
% Dowtys Rollen checken
Die Sätze in (\mex{1}) zeigen vielmehr, daß Verben, die in Dativpassivkonstruktionen
vorkommen, dem Dativ nicht unbedingt eine Rolle zuweisen müssen:
(\mex{1}a) ist eine spezielle Konstruktion, die auch als \emph{Caused"=Motion
Construction} bezeichnet wird. \emph{die Seife} ist kein logisches
Argument von \emph{waschen}. Wer oder was gewaschen wird, wird nicht gesagt,
die Information darüber ist in (\mex{1}a) nur indirekt erschließbar, da
von Augen die Rede ist. In (\mex{1}b) liegt ein Dativ vor, der theoretisch
zu zwei Dativklassen gehören kann: zu den sogenannten Pertinenzdativen\is{Dativ!Pertinenz-}
(Körperteildativen) oder zu den Dativ Commodi/Incommodi\is{Dativ!incommodi@\emph{incommodi}}\is{Dativ!commodi@\emph{commodi}},
die etwas über Nutznießer oder negativ Betroffene einer Handlung aussagen
(Siehe auch Fußnote~\ref{fn-dativ-commodi} auf Seite~\pageref{fn-dativ-commodi}). Beide Arten
Dativ sind als syntaktische, nicht jedoch als logische Argumente des Verbs einzustufen.
Bildet man nun einen zu (\mex{1}b) äquivalenten Satz mit Dativpassiv,
bekommt man (\mex{1}c).
\eal
\label{ex-bekommen-passive-und-waschen}
\ex Jemand wäscht     die Seife aus den Augen.
\ex Jemand wäscht ihm die Seife aus den Augen.
\ex\label{ex-er-bekam-die-seife-aus-den-augen-gewaschen}
Er bekam die Seife aus den Augen gewaschen.
\zl
\emph{gewaschen} weist aber keine Rezipienten"=Rolle zu.

(\mex{0}c) zeigt auch, daß es nicht richtig ist anzunehmen, daß sowohl \emph{bekommen} als auch das
eingebettete Verb eine Thema"=Rolle an das Akkusativobjekt zuweisen,
wie das \citet[\page23]{Haider86}, \citet[\page228]{HM94a}
und \citet[\page221]{Kathol2000a} tun: \emph{die Seife} ist kein logisches
Argument von \emph{gewaschen}.
Anstatt anzunehmen, daß \emph{bekommen}/""\emph{erhalten}/""\emph{kriegen} 
eine semantische Rolle zuweisen, behandle ich diese Verben also als echte Hilfsverben\is{Verb!Hilfs-}.

% stimmt nicht, weil es eine Lexikonregel geben könnte, die den Dativ nach der CM-Construction hinzufügt.
% \footnote{
%         Man beachte, daß Daten wie (\mex{0}) auch gegen Eisenbergs Aussage \citeyearpar[\page458]{Eisenberg99a}
%         sprechen, daß die semantische Rolle des Dativs ein prototypischer Rezipient ist. Da die
%        Caused"=Motion"=Construction nicht mit Verben möglich ist, die einen Dativ regieren,
%         kann \emph{ihm} in (\mex{0}b) nicht Argument
% }

% Fanselow diskutiert Idiom-Beispiele. Da könnten aber die Akkusative übertragene Lesarten haben
% und somit dennoch eine irgendwie geartete Rolle bekommen.

Ein Dativpassiv, bei dem eine komplexe Konstruktion mit freiem Dativ eingebettet ist,
die ebenfalls kein Akkusativobjekt enthält, zeigt (\mex{1}b):
\eal
\ex Jemand klopfte ihnen auf die Finger.
\ex\label{bsp-auf-die-finger-geklopft-bekommen}
%"Ich habe viele Erinnerungen, wie es damals in der Schule war - 
%daß es ganz vorne eine Eselsbank gab, auf der die sitzen mußten, die geschwätzt und gestört haben, 
daß wir noch nachsitzen mußten und auf die Finger geklopft bekamen\footnote{
        Frankfurter Rundschau, 03.06.1998, S.\,2.%
}\NOTE{Das kann nur analysiert werden, wenn die auf-PP ein Argument ist.}
\zl
In (\mex{0}b) handelt es sich um das Dativpassiv von (\mex{0}a).
%
% Irgendwie sind das doch Adjunkte. Die müßte man dann mit den Körperteildativen zusammen
% zu Argumenten machen ....
% Sie spuckte ihm ins Gesicht.
% Er bekam ins Gesicht gespuckt.

Andere Beispiele für das Dativpassiv von Verben, die keinen Akkusativ
zuweisen, wurden bereits auf Seite~\pageref{ex-er-kriegte-geholfen}
diskutiert und werden hier der Übersichtlichkeit halber
als (\ref{ex-er-kriegte-geholfen-zwei}) und (\mex{2}) wiederholt.\footnote{
        Die Beispiele in (\mex{1}) stammen von Wegener (\citeyear[\page134]{Wegener85a}; \citeyear[\page75]{Wegener90}\ia{Wegener}).
        Ähnliche Beispiele findet man auch bei \citew[\page161--162]{Fanselow87a} und
        \citew[\page143]{Eisenberg94a}. Siehe auch \citew[\page 203]{Abraham95a-u}.% 2005:189
}
\eal
\label{ex-er-kriegte-geholfen-zwei}
\ex\iw{helfen}\iw{gratulieren}\iw{applaudieren}
Er kriegte von vielen geholfen / gratuliert / applaudiert.
\ex\iw{danken} 
Man kriegt täglich gedankt.
\ex Ich möchte endlich einmal geholfen bekommen.
\zl
% Fanselow:
% \eal
% \ex Man kriegt täglich gedankt.
% %\ex\iw{vorleben} Wir wollen immer vorgelebt bekommen.
% \ex\iw{kondolieren}
% Man bekommt dann von Leuten kondoliert, die man gar nicht mehr gekannt hat.
% \zl
\eal
\ex
"`Da kriege ich geholfen."'\footnote{
Frankfurter Rundschau, 26.06.1998, S.\,7.%
}
\ex
% auch nach applaudiert geholfen + bekommen und kriegen gesucht 21.09.2003
Heute morgen bekam ich sogar schon gratuliert.\footnote{%
Brief von Irene G.\ an Ernst G.\ vom 10.04.1943, Feldpost-Archive mkb-fp-0270.}
\ex
%Branich IG-Vorsitzender Friedel Schönel meinte deshalb, 
"`Klärle"' hätte es wirklich mehr als verdient, auch mal zu einem "`unrunden"' Geburtstag gratuliert zu bekommen.\footnote{
Mannheimer Morgen, 28.07.1999, Lokales; "`Klärle"' feiert heute Geburtstag.%
}
\ex Mit dem alten Titel von Elvis Presley "`I can't help falling in love"' 
    bekam Kassier Markus Reiß zum Geburtstag gratuliert, [\ldots]\footnote{
Mannheimer Morgen, 21.04.1999, Lokales; Motor des gesellschaftlichen Lebens.%
}
\zl
% Allerdings kann man behaupten, daß der Satz das Objekt ist.
%http://board.gulli.com/thread/258695-excel-formel/ 07.01.2007
%Wenn ich solche Antworten lese kriege ich zuviel! Da bittet jemand um Hilfestellung für einfachste
%Funktionen und bekommt geantwortet, dass er ein Programm in Visual Basic schreiben soll...

\noindent
\citet{HW95a} merken an, daß solche Beispiele nicht besonders häufig sind.
\citet[\page75]{Wegener90} erklärt dies mit der geringen Frequenz bivalenter
Verben, die ein Dativobjekt verlangen und unergativisch sind. Zur Diskussion der Frequenz
solcher Daten siehe Seite~\pageref{frequenz-von-korpusbelegen}.
%%
%% \citet[\page19]{Wegener86a}
%% \eal
%% \ex[*]{
%% Er bekam zu der Stelle verholfen.
%% %bekommen && @verholfen 21.09.2003 nichts gefunden DWDS
%% }
%% \ex[*]{
%% Er kriegt zu der Reise geraten.
%% %bekommen && @geraten 21.09.2003 nichts gefunden DWDS
%% %kriegen && @geraten 21.09.2003 nichts gefunden DWDS
%% }
%% \zl
%%
%% Dative passives of ditransitive\is{verb!ditransitive} verbs with an optional accusative object that is not expressed
%% are easy to find. Some examples are given in (\mex{1}).
%% \eal
%% \ex\iw{kündigen|(}
%% auch er bekam im November gekündigt.\footnote{
%% BZK/W69.00593, WE 05.03.1969, S.\,10.% %, SONSTIGES, VERF.: -, AGT.: -
%% }
%% % \ex 
%% % Hier gehe es einerseits um eine Firma, die anderweitig gekündigt bekam und in ihrer Existenz bedroht sei 
%% % und andererseits um die Nachbarschaftsinteressen.\footnotemark%
%% % % here goes it on.the.one.hand about a company which from.elsewhere canceled got and in its existence threatened was and on.the.other.hand about the neighbor.interests\\
%% % \footnotetext{
%% % Mannheimer Morgen, 20.08.1998, Lokales; Grünes Licht für Skate-Anlage.%
%% % }
%% \ex
%% Sie werden mit ihrer Situation alleingelassen bis sie am Arbeitsplatz immer verhaltensauffälliger werden, 
%% schließlich ständig krank sind und dann möglicherweise gekündigt bekommen.\iw{kündigen|)}\footnote{
%%    Frankfurter Rundschau, 27.06.1998, S.\,3.%%, Ressort: N; Wenn Mitarbeiter trinken, wird nicht offen darüber gesprochen / Wildhof weiß Rat
%% }
%% \zl
%% \citet[\page45]{Hoehle78a} discusses the examples in (\mex{1}):
%% \eal
%% \ex\iw{vorlesen}
%% Karl kriegte jeden Abend vorgelesen.
%% \ex\iw{einschenken} 
%% Wir kriegten reichlich eingeschenkt.
%% \ex\iw{auf"|tun}
%% Wann kriegen wir aufgetan?
%% \zl


Die Tatsache, daß Beispiele wie \fromto{\ref{ex-er-kriegte-geholfen-zwei}}{\mex{0}} 
existieren, ist nicht überraschend, wenn man annimmt, daß
\emph{bekommen}/""\emph{erhalten}/""\emph{kriegen} nominalen Elementen keine Rolle
zuweisen. Würden sie einem Akkusativobjekt eine thematische Rolle zuweisen,
wären Sätze wie \fromto{\ref{ex-er-kriegte-geholfen-zwei}}{\mex{0}} ausgeschlossen.


Wie die folgenden Beispiele von \citet[\page104]{Leirbukt87} zeigen,
kann sich sowohl das logische Subjekt des eingebetteten Verbs (\mex{1}a)
als auch das Subjekt von \emph{bekommen} bzw.\ \emph{erhalten} (\mex{1}b) auf einen unbelebten Referenten beziehen:
\eal
\label{ex-zustrahlen-zuschreiben}
\ex\iw{zustrahlen}
{}[\ldots] während wir im     optischen Bereich von  der Sonne allein $10^8$mal    soviel  Energie zugestrahlt
     bekommen wie von  allen anderen Himmelskörpern zusammen [\ldots]\footnote{
        Stumpff, Karl, Hans-Heinrich Voigt (Hgg). 1972.
        \emph{Astronomie}. Frankfurt/M., Fischer Taschenbuch Verlag, S.\, 229.%%
}
\ex 
Beide Konstruktionen erhalten die gleiche Konstituentenstruktur zugeschrieben.\footnote{
        Das Beispiel stammt aus einem schwer zugänglichen Aufsatz von
        Leirbukt, 1977. Ich habe es nach \citew[\page23]{Askedal84} zitiert.%
}
%\ex {}[\ldots], daß Argumente ein kategoriales Merkmal des Regenten als speziellen Rektionsindex
%      zugewiesen erhalten.\footnote{
%        Im Haupttext von \citew[\page293]{Czepluch88}.
%      }
%\ex Perlmutter [\ldots] betrachtet die Klasse, zu der \emph{fallen} gehört, als Verben, die über ein
%      als Objekt klassifiziertes Argument verfügen, das aber keinen Akkusativ zugewiesen bekommt, und bezeichnet
%      sie daher als Unakkusativische.\footnote{
%        Im Haupttext von \citew[\page163]{Kaufmann95a}.%
%      }
\zl
% Reis85:149 macht das auch explizit: Der Wein bekam noch etwas Wasser beigemischt, wegen der Promille.
% gibt auch Beispiele mit Idiomen
Das Beispiel (\mex{0}b) zeigt, daß die Belebtheitsrestriktion\is{Belebtheit}, 
die \citet[\page76]{Reis76a}, \citet[\page18]{Wegener86a} 
% Wegener sagt, daß das insbesondere bei negativer REC-Lesart zutrifft.
und \citet[\page315]{Olsen97c} für die Subjekte von Dativpassivkonstruktionen formulieren,
empirisch nicht korrekt ist, da \emph{beide Konstruktionen} nicht belebt ist.\footnote{
        Allerdings ist der folgende, von Reis festgestellte Kontrast nicht zu leugnen:
        \eal
        \ex[]{
        Ich bekam/erhielt/kriegte die notwendige Unterstützung nicht versagt.
        }
        \ex[*]{
        Der Plan bekam/erhielt/kriegte die notwendige Unterstützung nicht versagt.
        }
        \zllast
}\NOTE{Mal nach Aktiv-Belegen mit unbelebtem Dativ suchen}
Solche Beispiele unterstützen Reis' Ansicht, daß \emph{bekommen}/""\emph{erhalten}/""\emph{kriegen} 
Hilfsverben sind, die keine Restriktionen für nichtverbale abhängige Elemente haben.


Nach diesen Betrachtungen zum Hilfsverbstatus von \emph{bekommen}/""\emph{erhalten}/""\emph{kriegen}
soll noch kurz erörtert werden, welche Verben das Dativpassiv (nicht) erlauben:
Unakkusativische Verben lassen kein Dativpassiv zu:
\eal
\label{bsp-kriegen-erg}
\ex[*]{\iw{auf"|fallen}
Ich bekomme (von Maria) auf"|gefallen.
}
\ex[*]{\iw{begegnen}
Sie kriegt begegnet.
}
\ex[*]{\iw{beitreten}
Die Gewerkschaft kriegt beigetreten.
}
\zl
Wie \citet*[\page72]{Reis76a}, \citet*[\page22]{Askedal84} und 
\citet{Leirbukt87} gezeigt haben, lassen nicht alle Verben,
die ein Passiv mit \emph{werden} erlauben, auch ein Dativpassiv zu:\footnote{
        Die Beispiele in (\mex{1}) stammen von \citet*[\page22]{Askedal84}.%
}
\eal
\ex[]{\iw{glauben}
Ihm wurde die Geschichte nicht mehr geglaubt.
}
\ex[*]{
Er bekam    /  erhielt  /  kriegte die Geschichte nicht mehr geglaubt.
}
\zl
Das heißt, die Menge der Verben, die ein Dativpassiv bilden können,
ist eine echte Teilmenge der Verben, die ein \emph{werden}"=Passiv bilden.
\iw{bekommen!Passiv|)}\iw{erhalten!Passiv|)}\iw{kriegen!Passiv|)}
\is{Passiv!Dativ-|)}

Bevor wir uns modalen Infinitiven und anderen passivähnlichen Konstruktionen
zuwenden, möchte ich noch das sogenannte Fernpassiv diskutieren,
das zu den interessantesten Phänomenen der deutschen Syntax gehört,
da es auf wunderbare Weise mit der bereits diskutierten Verbalkomplexbildung
interagiert.

\subsection{Das Fernpassiv}
\label{sec-remote-passive-phen}
\is{Passiv!Fern-|(}\is{Anhebung|(}

Normalerweise treten Objekte von eingebetteten Infinitiven nicht im Nominativ auf,
doch  \citet[\page175--176]{Hoehle78a} hat festgestellt, daß das in bestimmten Kontexten möglich ist.
Die folgenden Sätze sind Beispiele für das sogenannte Fernpassiv:
\eal
\ex\iw{versuchen|(}
daß er auch von mir zu überreden versucht wurde\footnote{
        \citew*[\page212]{Oppenrieder91a}\ia{Oppenrieder}.%
}
\ex\label{bsp-zu-reparieren-versucht-wurde}
weil    der Wagen oft zu reparieren versucht wurde\footnote{
  Siehe \citew[\page 176]{Hoehle78a} für ein ähnliches Beispiel.
}
\zl
Die Daten in (\mex{1}) sind Korpusbelege für dieses Muster:
\eal
\ex Dabei darf jedoch nicht vergessen werden, daß in der Bundesrepublik, wo ein Mittelweg zu gehen versucht wird, 
die Situation der Neuen Musik allgemein und die Stellung der Komponistinnen im besonderen noch recht unbefriedigend ist.\footnote{
Mannheimer Morgen, 26.09.1989, Feuilleton; Ist's gut, so unter sich zu bleiben?
}
\ex Noch ist es nicht so lange her, da ertönten gerade aus dem Thurgau jeweils die lautesten Töne, 
    wenn im Wallis oder am Genfersee im Umfeld einer Schuldenpolitik mit den unglaublichsten Tricks 
    der sportliche Abstieg zu verhindern versucht wurde.\footnote{
St.\ Galler Tagblatt, 09.02.1999, Ressort: TB-RSP; HCT und das Prinzip Hoffnung.%
}
\ex Die Auf- und Absteigenden erzeugen ungewollt einen Ton,
        der bewusst nicht als lästig zu eliminieren versucht wird, 
    sondern zum Eigenklang des Hauses gehören soll, so wünschen es sich die Architekten.\footnote{
Züricher Tagesanzeiger, 01.11.1997, S.\,61.%
}
\zl
Im IDS"=Korpus habe ich 2002 nur Fernpassive mit \emph{versuchen} gefunden. Susanne
Wurmbrand gibt jedoch die folgenden Beispiele mit \word{beginnen}, \word{vergessen} und \word{wagen},
die sie im World Wide Web gefunden hat \citep{Wurmbrand2003a}:
%% \footnote{
%%   In vielen ihrer Beispiele liegen Subjekte im Plural vor, die durch Kongruenz
%%   mit dem Passivhilfsverb als Subjekte identifizierbar sind. \citet{FF2004a}
%%   haben gezeigt, daß lokale Ambiguität bei der Verarbeitung durch den Menschen
%%   die Ursache dafür sein kann, daß eigentlich ungrammatische Strukturen als akzeptabel
%%   bzw.\ akzeptabler eingestuft werden. Je früher im Satz Beschränkungen verletzt
%%   werden, desto schlechter wird er bewertet. Dies könnte nun bei der Bewertung
%%   von Fernpassivdaten eine bessere Bewertung von Daten mit nicht eindeutig
%%   identifizierbarem Subjektskasus zur Folge haben. In (\mex{0}) und (\ref{bsp-wurde-zu-bauen-begonnen})
%%   sind die Subjekte maskulin/""singular, also eindeutig kasusmarkiert.
%%   Die anderen Beispiele in \fromto{\mex{1}}{\mex{2}} muß man wohl mit
%%   Vorsicht interpretieren.%
%% } 
\eal
%% \ex
%% Auch einen Jugendraum plant Herr Pfeiffer, dieser wurde bereits zu bauen begonnen.\footnote{
%%         \url{http://www.hollabrunn.noe.gv.at/mariathal/ortsvorsteher.html}, 28.07.2003.
%% }
\ex\label{bsp-wurde-zu-bauen-begonnen}%
der zweite Entwurf wurde zu bauen begonnen,\footnote{
\url{http://www.waclawek.com/projekte/john/johnlang.html}, 28.07.2003.
}
\zl
\eal
\ex Meist handelt es sich hier um Anordnungen, die zu stornieren vergessen wurden.\footnote{
        \url{http://www.rlp-irma.de/Dateien/Jahresabschluss2002.pdf}, 28.07.2003.
}
\ex Hiermit können Aufträge aus der HPC-Analytik gefiltert werden, die zu dru"cken vergessen worden sind 
    und daher noch nicht abgerechnet sind.\footnote{
        \url{http://www.iitslips.de/news.html}, 28.07.2003.
}
\zl
\eal
%\ex Ist plötzlich übervoll von Emotionen und längst begrabenen Träumen, die nicht zu leben gewagt wurden\footnote{
% nicht auffindbar
\ex NUR Leere, oder doch noch Hoffnung, weil aus Nichts wieder Gefühle entstehen,
    die so vorher nicht mal zu träumen gewagt wurden?\footnote{
        \url{http://www.ultimaquest.de/weisheiten_kapitel1.htm}, 28.07.2003.
}
\ex Dem Voodoozauber einer Verwünschung oder die gefaßte Entscheidung zu einer Trennung,
    die bis dato noch nicht auszusprechen gewagt wurden.\footnote{
        \url{http://www.wedding-no9.de/adventskalender/advent23_shawn_colvin.html}, 28.07.2003.
}
\zl
% Kasus bei PVP wie Haiders entziffern: Am leichtesten zu erklären fiel den 
% Experten dabei gestern der Kursverlust der Telekom, zu deren Schuldenproblem 
% eine neue Hiobsbotschaft kam.  (taz. 8./9. 9. 01 S. 9.)
%
In den Beispielen in (\mex{-3}) und (\mex{-2}) sind die Subjekte in den Passivsätzen
maskulin, weshalb ihr Kasus eindeutig als Nominativ zu identifizieren ist. In (\mex{-1}) und
(\mex{0}) ist das nicht der Fall, man kann aber durch die Kongruenz des Passivhilfsverbs
das Vorhandensein eines Subjekts erschließen: Wäre das \emph{die} ein Objekt,
läge ein unpersönliches Passiv vor, und das Passivhilfsverb müßte in der dritten Person
Singular stehen (zur Subjekt"=Verb"=Kongruenz siehe auch Seite~\pageref{bsp-kongruenz-passiv}).

In Fernpassivkonstruktionen wird das Objekt eines Verbs, das unter ein Passivpartizip eingebettet ist,
zum Subjekt des Satzes. Zum Beispiel verlangt das Verb \emph{reparieren} ein Subjekt und ein Akkusativobjekt.
In \pref{bsp-zu-reparieren-versucht-wurde} wird das Objekt im Nominativ realisiert.
Solche Realisierungen im Nominativ sind nur bei Vorliegen eines Verbalkomplexes, also bei sogenannten
kohärenten Konstruktionen möglich, wie die Beispiele in (\mex{1}) zeigen:
\eal
\ex[]{
weil oft versucht wurde, den Wagen zu reparieren
}
\ex[*]{
weil oft versucht wurde, der Wagen zu reparieren
}
\ex[]{
Den Wagen zu reparieren wurde oft versucht
}
\ex[*]{
Der Wagen zu reparieren wurde oft versucht
}
\zl
Die Daten in (\mex{0}) kann man erklären, wenn man annimmt, daß das Fernpassiv eine Passivierung
des Verbalkomplexes ist, \dash, wenn man dem Satz \pref{bsp-zu-reparieren-versucht-wurde}
die Struktur in (\mex {1}) zuweist:
\ea
weil    der Wagen     oft   [[zu reparieren versucht] wurde]
%
\z
In (\mex{-1}a,c) liegen keine Verbalkomplexe vor. Das Objekt von \emph{zu reparieren} ist Teil der VP
und bekommt deshalb Akkusativ. Die Passive in (\mex{-1}a,c) sind unpersönliche Passive\is{Passiv!unpersönliches}.
\iw{versuchen|)}

Das Fernpassiv ist nicht auf Subjektkontrollverben beschränkt. Komplexere Beispiele
können auch mit Objektkontrollverben wie \word{erlauben} gefunden werden:
\eal
\label{bsp-auskosten-fernpassiv}
\ex\iw{erlauben}
Keine Zeitung         wird ihr       zu lesen erlaubt.\footnote{
        Stefan Zweig. \emph{Marie Antoinette}. Leipzig: Insel-Verlag. 1932, S.\,515, 
        zitiert nach \citew[\page309]{Bech55a}\ia{Bech}. 
        Daß in diesem Beispiel ein Fernpassiv vorliegt, hat \citet[\page13]{Askedal88}\ia{Askedal} festgestellt.
}
\ex\iw{auskosten}
Der Erfolg        wurde uns      nicht auszukosten erlaubt.\footnote{
        \citew[\page110]{Haider86c}.%
}
\label{bsp-auskosten-fernpassiv-haider}
\zl

\noindent
Das Passiv der inkohärenten Konstruktion -- der Konstruktion, 
bei der kein Verbalkomplex vorliegt, -- ist ein unpersönliches Passiv:
\eas
Uns wurde erlaubt, den Erfolg auszukosten.
\zs
Aber wie die Beispiele in (\mex{-1}) zeigen, kann das Akkusativobjekt des eingebetteten Verbs
wie in den Beispielen mit \emph{versuchen} als Nominativ auf"|treten.
Die Generalisierung ist: In Passivkonstruktionen, in denen ein Verbalkomplex unter das Passivhilfsverb
eingebettet ist, wird das Subjekt unterdrückt, und von den verbleibenden Argumenten
wird das erste Argument mit strukturellem Kasus\is{Kasus!struktureller}
zum Subjekt und bekommt Nominativ.%
\is{Passiv!Fern-|)}\is{Anhebung|)}

Nach der Besprechung der Kernfälle des Passivs und der Interaktion zwischen Passiv und Verbalkomplexbildung
komme ich jetzt zu verschiedenen passivähnlichen Konstruktionen.

\subsection{Modale Infinitive}
\label{sec-modal-inf}

% Rapp97a:130 Heute ist zu Hause zu bleiben.

Außer\is{Infinitiv|see{Inkoh\"arenz, Koh\"arenz}}\iw{haben!modal|(}%
\iw{sein!modales|(}%
\is{Infinitiv!modaler|(}\is{Passiv!modaler Infinitiv|(}
in Perfektkonstruktionen kommen \haben und \sein auch noch in Verbindung mit \emph{zu}"=Infinitiven
vor. Die Sätze haben dann eine modale Bedeutung.
Die Realisierung der Argumente entspricht bei \zuinf und \haben einem Aktivsatz (\mex{1})
und bei \zuinf und \sein einem Passiv-Satz (\mex{2}):\footnote{
        Die Beispiele (\mex{1}) und (\mex{2}) sind von \citet*[\page72]{Bierwisch63a}.
}
\eal
\label{ex-habt-zu-erledigen}
\ex Ihr habt die Angelegenheit zu erledigen.
\ex Ihr müßt die Angelegenheit erledigen.
\zl

\eal
\label{ex-ist-zu-erledigen}
\ex Die Angelegenheit ist von euch zu erledigen.
% Widerlegt Demskes 1994:9 Behauptung, daß modale Passive ohne modifizierendes Adjektiv
% keine von der Präposition von regierte Agensphrase aufweisen können.
\ex Die Angelegenheit wird von euch erledigt.
\ex Die Angelegenheit muß von euch erledigt werden.
\zl


\noindent
Die modale Lesart kann beim \zuinf mit \sein Sätzen mit
\emph{können} (\mex{1}), \emph{dürfen} (\mex{2}), \emph{sollen} (\mex{3})
oder \emph{müssen} (\mex{4}) entsprechen \citep[Kapitel~2]{Gelhaus77}.
\ea
Die Tür ist für Hans leicht zu öffnen.
\z
\eal
\ex Auf Liebe und Gunst von uns Menschen ist ohnehin nicht sehr zu bauen.\footnote{
        \citew*[\page72]{Gelhaus77}.
}
\ex Ein wütender Straußenhahn ist nicht zu unterschätzen.\footnote{
        \citew*[\page69]{Gelhaus77}.% = nicht duerfen -> nicht hat Skopus ueber die modale Komponente
}
\zl
\ea
zum Schluß wußte niemand, wie das Erntefest zu feiern wäre\footnote{
        \citew*[\page74]{Gelhaus77}.
}
\z
\eal
\ex Selbstverständlich ist auch eine Verständigungsmöglichkeit durch Sprechfunk vorzusehen.\footnote{
        \citew*[\page56]{Gelhaus77}.
}
\ex Er wußte, daß er \ldots{} würde sprechen müssen, über Beerdigung, Verwaltungskram, der unweigerlich
    zu erledigen sein würde.\footnote{
        \citew*[\page56]{Gelhaus77}.
}

\zl
Das logische Subjekt kann mit \emph{von}, \emph{durch} oder \emph{für} angeschlossen werden.
\ea
Das Ziel wird für ihn nicht zu erreichen gewesen sein.\footnote{
         \citew*[\page72]{Bierwisch63a}.
}
\z
In der \emph{können}"=Lesart wird normalerweise die Präposition \emph{für} verwendet und in der
\emph{müssen}/""\emph{sollen}"=Lesart eine der Präpositionen \emph{von} und \emph{durch}.
%
% Demske94a:259 sagt, es gibt Verben, die immer nur eine Art P erlauben.
%
Nach \citet[\page225]{Demske94a} findet sich die \emph{von}"=PP ausschließlich bei Verben,
deren \hyperlink{externesArgument}{externes Argument}\is{externes Argument}
agentiv\is{Agentivität} interpretiert werden kann.

Im allgemeinen gibt es für jeden Aktivsatz einen Satz mit \zuinf und \haben und für jeden
Passivsatz einen Satz mit \zuinf und \sein \citep*[\page72]{Bierwisch63a}\ia{Bierwisch}.\footnote{
Die folgenden Beispiele von \citet[\page199]{Demske94a} scheinen dem zu widersprechen:
\eal
\ex[]{
Das Licht wird von der Folie reflektiert.
}
\ex[*]{
Das Licht ist von der Folie zu reflektieren.
}
\zllast
%Eventuell liegt die Unakzeptabilität jedoch daran, daß man sich schwer Kontext vorstellen
%kann, in denen (i.b) geäußert werden kann. 
%
%Ersetzt man den definiten Artikel durch \emph{einer}, wird der Satz aber besser.%
}
Man vergleiche die Sätze in (\ref{ex-habt-zu-erledigen}) und in (\ref{ex-ist-zu-erledigen}).
Die Umkehrung gilt jedoch nicht: Nicht alle Verben, die in modalen Infinitiven mit \emph{sein} vorkommen,
haben auch ein Passiv mit \emph{werden}. Das zeigen die folgenden
Beispiele, auf die \citet[\page53]{Hoehle78a}\ia{Höhle} hingewiesen hat:
\eal
\ex[]{\label{ex-ist-schwer-anzukommen}\iw{ankommen!gegen}
Dagegen ist schwer anzukommen.
}
\ex[*]{
Dagegen wurde angekommen.
}
\ex[]{\iw{bekommen!main verb}
Auch für dich ist etwas Brot zu bekommen.
}
\ex[*]{
Auch für dich wurde etwas Brot bekommen.
}
\ex[]{\iw{erfahren}
Der Zeitpunkt war schwer zu erfahren gewesen.
}
\ex[*]{
Der Zeitpunkt war erfahren worden.
}
\ex[]{\iw{erhalten!main verb}
Karten waren noch lange zu erhalten.
}
\ex[\#]{
Karten wurden erhalten.
}
\zl
\citet[\page17]{Haider86}\iaright{Haider} diskutiert die Beispiele in (\mex{1}), die ebenfalls
zeigen, daß es Modalkonstruktionen mit \sein gibt, die kein \emph{werden}"=Passiv haben:
\eal
\ex[*]{\iw{haben!Hauptverb}
Ein einziges Würstchen wird noch gehabt.
}
\ex[]{
Ein einziges Würstchen ist noch zu haben.
}
\ex[*]{\iw{gefallen}
Ihm wird leicht gefallen.
}
\ex[]{
Ihm ist leicht zu gefallen.
}
\zl
% \ex[*]{
% Südfrüchte werden gehabt.
% % }
% \ex[]{
% Südfrüchte waren entweder überteuert    oder gar    nicht zu haben.\footnotemark
% % \footnotetext{
%         Spiegel, 46/1999, S.\,200.%
%       }
% }
%
Haider erklärt den Unterschied, indem er annimmt, daß \sein 
weniger restriktiv als das Passivhilfsverb \emph{werden} ist: Während \emph{werden} ein
Verb verlangt, das seinem logischen Subjekt die Agens"=Rolle\is{semantische Rolle!Agens}
% Wunderlich87c:300 externes Argument des Verbs muss Agens sein, damit Passivierung mgl ist.
zuweist, verlangt \sein keine bestimmte Rollenzuweisung.\footnote{
        Eine ähnliche Beschränkung formuliert \citet[\page 375, \page 382]{Toman86a}. Er führt
        die Ungrammatikalität von Passivierungen des Verbs \emph{ankommen} auf
        die fehlende Agentivität des Subjekts zurück. Seiner Meinung nach ist 
        \emph{werden} für ein Partizip subkategorisiert, das das Merkmal [-stative] hat.%
%Abraham2005a:60 subjektdesigniertes Agensargument
% 96, 101,281,533
}$^,$\footnote{
        Man beachte, daß diese Erklärung einen weiten Agensbegriff verlangt. Zum Beispiel
        muß das Subjekt des Verbs \word{sehen} als Agens eingestuft werden.
        \eal
        \ex Er sah den Einbrecher.
        \ex Der Einbrecher wurde gesehen.
        \zl
% Bevor voreilig auf jüngere Geschwister gehört wird, heißt es also wirklich "`aufgepost!"'\footnote{
%        taz, 28./29.10.2000, S.\,5
%}
        Das Subjekt von \emph{sehen} wird jedoch oft als Experiencer\is{semantische Rolle!Experiencer} bezeichnet (\zb \citealp[\page
        190]{Devlin92}). \citet[\page65]{Dowty2000a} diskutiert englische\il{Englisch}
        Beispiele mit \emph{hear} und \emph{believe} und ordnet die im Passiv unterdrückten
        Argumente ebenfalls als Experiencer ein.%


\citet[\page 21]{Abraham2005a} bezeichnet das Subjekt von \emph{verlieren} als Source\is{semantische Rolle!Source}, aber auch ein
solches Subjekt kann durch Passivierung unterdrückt werden:
%% Weil durch das Schwitzen viel Flüssigkeit verloren wird, ist es wichtig, viel Tee, Mineralwasser, Fruchtsäfte und fettfreie Bouillon zu trinken.
%% St. Galler Tagblatt, 04.03.1999; Bewährtes aus Grossmutters «Natur-Apotheke»  
%% Die Beobachtung, dass im Bereich der Mittellinie der Ball oft schnell verloren wurde, mag als die eine generelle Erkenntnis des Herbstes gelten. 
%% St. Galler Tagblatt, 11.11.1997; Wenigstens ein Punkt für Herisau  
%% Nicht Diebe sind es, die Autofahrer in erster Linie zum Kauf neuer «Nummerntafeln» zwingen. Es sind meist die Fahrzeugbesitzer selbst, die sich Umtrieb und Kosten aufhalsen. 1999 wurden 154 gestohlene St. Galler Schilder ausgeschrieben, verloren wurden 723.
%% St. Galler Tagblatt, 08.08.2000; Schilderverlust ist nicht billig  
\ea
Und ob der Schlüssel verloren wurde, oder die Katze entlaufen ist, man wendet sich als erstes an die Polizei.
(Vorarlberger Nachrichten, 26.07.1997, S.\,A8)
\z

        Die Beispiele in (iii) zeigen, daß man auch unbelebte 
        Argumente wie \emph{die Grammatikalisierung} unter Agens fassen muß, wenn
        man das Passiv von der semantischen Rolle des Subjekts abhängig macht \citep[Kapitel~3.1.2]{Mueller2002b}.
        \eal
\ex Die Schneeflocken beeinflußten meine Entscheidung.
\ex Meine Entscheidung wurde durch die Schneeflocken beeinflußt.
\ex Die Grammatikalisierung überlagert sie.
\ex {} [\ldots] da    sie  von der Grammatikalisierung überlagert werden. (Im Haupttext von \citew[\page190]{Kaufmann95a})
\zl

        \citet[\page175]{HW95a} lassen explizit unbelebte Referenten als Agens zu.
        Siehe auch \citew[\page 129]{Marantz84a} zu möglichen semantischen Rollen des Subjekts in
        englischen Passivkonstruktionen.%
}

\noindent
Die Beispiele in (\mex{1}) zeigen, daß modale Infinitive mit \sein mit unakkusativischen
Verben nicht gebildet werden können, obwohl modale Konstruktionen mit \haben möglich sind.\footnote{
        Höhles Beispiel (\ref{ex-ist-schwer-anzukommen}) ist ein modaler Infinitiv mit
        \sein und einem unakkusativischen Verb. Ich habe keine Erklärung für die Grammatikalität dieses Beispiels.
}
\eal
\ex[]{\iw{entfallen}
die Gesetzesvorschrift selbst hat ersatzlos zu entfallen\footnote{
Die Zeit, 27.12.1985, S.\,4.%
}
}
\ex[*]{
Deshalb ist ersatzlos zu entfallen.
%`hence is to be dropped without replacement?
}
\ex[]{
%\NOTE{WS: (63) Was soll denn Satz c bedeuten? Der kommt mir irgendwie ungrammatisch vor.}%
Hat er zu gelingen, ist es wichtig, sich selber zu beobachten\footnote{
St.\ Galler Tagblatt, 23.10.1998, Ressort: TB-ARB; Wo bleibt die Paar-Beziehung?
}
}
\ex[*]{
Deshalb ist zu gelingen.
%`Hence, success is imperative? / `Hence, one must succeed?
}
\ex[]{ 
Das hat Ihnen diesmal zu gelingen.
}
\ex[*]{
Ihnen ist diesmal zu gelingen.
}
\ex[]{\iw{entfallen}
Solche wichtigen Sachen haben dir nicht wieder zu entfallen.
}
\ex[*]{
Dir ist leicht zu entfallen.
}
\zl
Weder die intransitiven unakkusativischen Verben in (\mex{0}a,c) noch die
unakkusativischen Verben mit Dativobjekt in (\mex{0}e,g) erlauben 
den modalen Infinitiv mit \sein.

Es gibt also kein klares Bild der Klasse der Verben, die modale Infinitive mit \sein erlauben:
Einige der Verben, die kein Vorgangspassiv erlauben, lassen aber modale Infinitive mit \sein
zu.%

Zuletzt soll noch festgehalten werden, daß das modale \sein nicht unter ein Passivhilfsverb
eingebettet werden kann:
\ea[*]{
\label{bsp-mod-inf+werden}
Dieser Wagen ist von ihnen bis morgen repariert zu werden.\footnote{
        \citew[\page2]{Wilder90a}.
}
}
\z
%
\iw{haben!modal|)}\iw{sein!modal|)}%
\is{Infinitiv!modaler|)}\is{Passiv!modaler Infinitiv|)}


\subsection{\emph{lassen}-Passiv}
\iw{lassen!Passiv|(}\is{Passiv!lassen@\emph{lassen}|(}

In diesem Abschnitt werden Passivformen mit \emph{lassen} diskutiert.
Der Satz in (\mex{1}a) entspricht einem persönlichem Passiv\is{Passiv!persönliches}, und
die in (\mex{1}b,c) entsprechen unpersönlichen Passiven\is{Passiv!unpersönliches}.\footnote{
        Die Beispiele in (\mex{1}b,c) stammen von \citet[\page19]{Reis76c}\iadata{Reis}.
}
\eal
\label{bsp-lassen-passiv}
\ex\iw{reparieren}
Er       läßt den Wagen     (von einem Fachmann) reparieren.
\ex\iw{helfen}
Der Vater        läßt der Mutter       (vom Sohn) helfen.
\ex\iw{gedenken}
Die Regierung        läßt der Toten      (vom Volke) gedenken.
\zl
Das logische Subjekt der Verben \emph{reparieren}, \emph{helfen} und 
\emph{gedenken} wird unterdrückt, kann aber mittels einer passivtypischen
PP realisiert werden.

Das Verb \emph{lassen} hat eine kausative\is{Verb!kausatives}\is{Kausativ!-konstruktion}
und eine permissive\is{Verb!permissives} Lesart:
\eal
\ex Der Mann läßt den Fachmann den Wagen reparieren.
\ex Die Mutter ließ das Schnitzel anbrennen.\label{bsp-schnitzel-anbrennen}\footnote{
        \citew[\page13]{Reis76c}\iadata{Reis}.%
      }
\ex Peter ließ es regnen.\label{bsp-er-liess-es-regnen}
\zl
In der kausativen Lesart verursacht/""bewirkt das Subjekt von \emph{lassen},
daß etwas geschieht, in der permissiven Lesart läßt das Subjekt zu, daß
etwas geschieht. Je nach Kontext ist in Aktivsätzen sowohl die kausative
als auch die permissive Lesart möglich.
(Zur Diskussion von (\ref{bsp-er-liess-es-regnen})
siehe auch Seite~\pageref{bsp-er-laesst-es-regnen}.)
%
In \emph{lassen}"=Passiv"=Konstruktionen hat \emph{lassen} normalerweise die 
kausative\is{Passiv!lassen@\emph{lassen}!kausativ} Lesart.
%und es wird mitunter behauptet, daß ein \emph{lassen}"=Passiv mit der permissiven Lesart nicht
%möglich ist (\citew[\page652]{Suchsland87a}; 
\citet[\page13]{Reis76c}\iaright{Reis} hat jedoch festgestellt, daß die
permissive\is{Passiv!lassen@\emph{lassen}!permissiv} Lesart ebenfalls möglich ist,
wenn das Subjekt des eingebetteten Verbs ein Reflexivum\is{Pronomen!Reflexiv-} ist.
\eal
\label{bsp-lassen-passiv-permissiv}
\ex 
Der Sänger ließ sich schließlich, um endlich seine Ruhe zu haben, von seinen Verehrerinnen abküssen.\footnote{
        \citew[\page13]{Reis76c}.%
}
\ex\iw{lesen!die Leviten $\sim$}
Gerhard Schröders Doppelgänger mußte sich in Abwesenheit des Originals die Leviten lesen lassen.\footnote{
Mannheimer Morgen, 05.03.1999, Politik; "`Derblecken"' auf dem Nockherberg.%% M99/903.14763 Mannheimer Morgen, 05.03.1999, Politik; "Derblecken" auf dem Nockherberg
}
\label{ex-leviten-lassen}
\ex 
sich vom Wind streicheln und sich von der feinen Gischt erfrischen zu lassen\label{ex-sich-vom-wind-streicheln-lassen}\footnote{
Mannheimer Morgen, 03.08.1998, Sport; "`Fun"' beim Sport: Mit Windsurfen fing alles an.% %M98/808.63751 Mannheimer Morgen, 03.08.1998, Sport; "Fun" beim Sport: Mit Windsurfen fing alles an
}
% \ex Prototyp des abstrakten Charakters der Freizeit ist das Verhalten jener, die sich in der Sonne braun
%     braten lassen, [\ldots]\footnotemark
% \footnotetext{
%         Adorno, quoted by taz, 15.06.2000, S.\,14.%
%     }
\zl
% \citet[\page655]{Suchsland87a} nimmt an, daß \emph{lassen} als AcI"=Verb und als Kontrollverb vorkommt.
% Nach Suchsland liegt die AcI"=Version vor, wenn das Subjekt des eingebetteten Verbs nicht realisiert wird,
% wird es dagegen wie in (\ref{bsp-lassen-passiv}) und (\ref{bsp-lassen-passiv-permissiv})
% nicht realisiert, handelt es sich um die Kontrollversion. Bei der Kontrollversion soll ausschließlich
% die kausative Lesart möglich sein. Wie die Daten von Reis zeigen, ist eine solche Unterscheidung nicht
% gerechtfertigt. Eine Analyse, die beide \emph{lassen}"=Versionen als Anhebungsverben analysiert, scheint
% den Verhältnissen eher gerecht zu werden.
% auf S. 660 diskutiert er dann auch Beispiele mit Reflexivum
%
% Gunkel2003b:232 sagt, daß der funktionale Druck zur Reflexivierung dafür verantwortlich ist
% Siehe auch S. 239

%% Man beachte, daß (\ref{ex-sich-vom-wind-streicheln-lassen}) ein weiteres Beispiel dafür ist, daß
%% das logische Subjekt des eingebetteten Verbs in Passivkonstruktionen unbelebt\is{Belebtheit} sein kann.
%%
Die Beispiele in (\mex{1}) zeigen, daß das \emph{lassen}"=Passiv nicht mit allen Verben möglich ist,
die ein Vorgangspassiv zulassen \citep[\page20]{Reis76c}\iaright{Reis}.
Daß das \emph{lassen}"=Passiv nur mit einer echten Teilmenge der Verben, die ein Vorgangspassiv bilden,
möglich ist, liegt wahrscheinlich an zusätzlichen semantischen Restriktionen des Verbs \emph{lassen}.
\eal
%%\ex Hans läßt von Peter glauben, schizophren zu sein.\footnote{
%%        \citew[\page144]{Grewendorf83a}.%
%%}
%\ex 
\ex[]{
Es wurde geglaubt, den Kindern nicht mehr helfen zu können.
}
\ex[*]{
Er ließ (von allen) glauben, den Kindern nicht mehr helfen zu können.
}
\zl
In (\mex{1}a) liegt eine permissive Lesart vor, die beim \emph{lassen}"=Passiv ausgeschlossen ist.
Deshalb ist die Einbettung von \emph{glauben} in \emph{lassen}"=Passiv"=Konstruktionen nicht möglich.
\eal
\ex[]{
Er ließ alle die Geschichte glauben.
}
\ex[*]{
Er ließ die Geschichte (von allen) glauben.
}
\zl

\noindent
Wie auch das Vorgangspassiv ist das \emph{lassen}"=Passiv mit expletiven Prädikaten nicht möglich:
\ea[*]{
Karl läßt regnen.
}
\z

\noindent
Die Einbettung von Vorgangspassiven unter \emph{lassen} ist nicht möglich,
wie \citet[\page3]{Wilder90a} festgestellt hat:
\eal
\ex[]{
Er läßt den Wagen von ihnen reparieren.
}
\ex[*]{
Er läßt den Wagen von ihnen repariert werden.\NOTE{JB findet den gut. St. Mü. vielleicht in
  permissiver Lesart.}
}
\zl
% Stimmt das auch für das permissive lassen?
%
\iw{lassen!Passiv|)}\is{Passiv!lassen@\emph{lassen}|)}


\subsection{Passivierung unakkusativischer Verben}
\label{sec-passive-unakkusativ}

Der Vollständigkeit halber sollen auch Beispiele wie die Sätze in (\mex{1}) nicht unerwähnt bleiben.
Diese Beispiele zeigen, daß auch als unakkusativisch einzuordnende Verben unter Umständen passivierbar sind:
% Flaeming81:551 zitiert Erben mit 'sterben'-beispiel
\eal
\ex[]{
Doch im Gleimtunnel wurde schon aus merkwürdigeren Gründen gestorben.\footnote{
              taz berlin, 04.10.1996, S.\,26.%
      }\iw{sterben}
}
\ex[]{
Wann darf gestorben\iw{sterben} werden?\footnote{
        Überschrift eines Artikels über Sterbehilfe, taz, taz-mag, 09./10.10.1999, S.\,VI.%
      }
}
\ex[]{
Gestorben wird immer, eingeäschert und beerdigt immer öfter anderswo. Eine Besichtigungsfahrt
ins Krematorium von Vyso\v{c}any in Tschechien -- und ein Einblick in unsere Zukunft als Leichen\footnote{
  taz, 29.04.2004, S.\,13. Zu vier weiteren Beispielen siehe taz, 25./26.11.2006 zum
  Friedhofssterben in Berlin.%
}
}
\ex[]{
Ein Berlin wie in den zwanziger Jahren, in dem gehurt, gesoffen und gescheitert\iw{scheitern} wird --
      und der Mann im Chefredaktionsbüro so aussieht, als würde er dabei ordentlich mittun.\footnote{
        Spiegel, 36/1999, S.\,36.%
      }
}
\ex[]{
Beim Anblick eines Exhibitionisten sei heftig geatmet, reihenweise in Ohnmacht gefallen und
mit zarter Stimme "`Schutzmann"' gerufen worden.\footnote{
        taz, 27.06.2003, S.\,20.%
}
}
\zl
%       Zu einer diesbezügliche Kritik an Pollards Ansatz siehe auch 
%        \citep*[\page250ff]{Kathol94a}\iaf{Kathol}.
Nach \citet*[\page350]{Ruzicka89}\ia{Ruzicka@R\r{u}\v{z}i\v{c}ka} werden mit solchen Äußerungen spezifische
pragmatische\is{Pragmatik} bzw.\ rhetorische Praktiken verfolgt. 
%zitiert Fillmore Kay OConner Language, 3/88, S. 506
Diese Äußerungen haben direktive Zwecke (\ref{ex-hier-wird-dageblieben}) oder 
ironisch"=fatalistische Nuancen (\ref{ex-wird-gestorben}).
\eal
\ex[?]{
Hier wird dageblieben\iw{dableiben} und nicht verschwunden.\iw{verschwinden}\label{ex-hier-wird-dageblieben}
}
\ex[]{
Hier wird nur gestorben.\label{ex-wird-gestorben}
}
\ex[?]{
Hier wird nicht angekommen,\iw{ankommen} sondern nur abgefahren.\iw{abfahren}\label{ex-wird-angekommen}
}
\zl
Sie können allgemeine (\ref{ex-wird-angekommen}) oder schicksalhafte (\ref{ex-wird-gestorben}) Bestimmung konstatieren, der
etwas unterworfen ist. 
\citet[\page205]{Wunderlich85}\ia{Wunderlich} zählt Sätze wie (\ref{ex-hier-wird-dageblieben})
ebenfalls zu den idiomatischen\is{Idiom} Wendungen für Verbote.
Wenn diese Passivierungen nur mit einer bestimmten Bedeutung auf"|treten,
ist es gerechtfertigt, diese Konstruktionen gesondert zu behandeln
und einen speziellen Lexikoneintrag für \emph{werden} zu verwenden, der die Einbettung
unakkusativischer Verben bei entsprechender Interpretation erlaubt.\footnote{
        Im folgenden Beispiel von \citet[\page119]{Faucher87}
        liegt keine solche abweichende Lesart vor.
        \ea
        Es konnte ihm nicht entgehen, daß ihm von Reiff und Duquesde,
        ganz besonders aber von Gruzynski mit einer vornehm
        ablehnenden Kühle begegnet wurde. (Th. Fontane, {\em L'Adultera},
        NTA-6, 1969, S.\,75)
        \z
        In (i) scheint jedoch trotz der Perfektbildung mit \emph{sein}
        ein nicht unakkusativisches Verb vorzuliegen, das homonym zum \emph{begegnen}
        mit der Bedeutung \emph{treffen} ist. Die Passivierung des letzteren
        ist auch wie erwartet ausgeschlossen.
        \eal
        \ex[*]{
        Ihm wurde (von Gruzynski) begegnet.
        }
        \ex[*]{
        Er bekam begegnet.
        }
        \zllast
}
Auf  diese Fälle gehe ich im Analyse"=Abschnitt nicht mehr ein.

%\ifthenelse{\boolean{draft}}{

\subsection{Agensausdrücke}

Die\is{Agensausdruck|(} Frage, ob Agensausdrücke wie die \emph{von}"=PP in (\mex{1}) als Argumente oder Adjunkte behandelt werden sollen,
wurde in der Literatur verschieden beantwortet: Als Argument behandeln 
die Präpositionalphrase \zb 
\citet[\page181]{Heringer73a}, 
\citet{Bresnan82a},
\citet{ps}, 
\citet{MS98a} und
\citet[Kapitel~15.3]{Mueller99a}.
%
%Gunkel2003b:65 zitiert 
\citet[\page161]{Hoehle78a}, \citet{Sadzinski87a}, \citet[\page174]{Stechow90a}, 
\citet[\page255]{Zifonun92a}, 
\NOTE{\citet[\page86]{Leiss92a},}
\citet[\page181]{Lieb92a},
\citet[\page740]{Wunderlich93a}, 
\NOTE{\citet[\page150]{Primus99a},}
\citet{Mueller2003e} und 
\citet[\page 65]{Gunkel2003b} 
behandeln Agensausdrücke als Adjunkte.
\ea
Er wurde von ihr geküßt.
\z

\noindent
Es gibt für beide Sichtweisen gute Argumente: Die Agensausdrücke füllen die Agens"=Rolle
eines Verbs. Die einfachste Art, diese Beziehung auszudrücken, ist anzunehmen, daß Agensausdrücke
von ihrem Verb abhängen. Die Syntax"=Semantik"=Verbindung kann dann im Lexikoneintrag für das
Verb oder im Lexikoneintrag für das Hilfsverb, das ja -- wenn man eine Verbalkomplexanalyse annimmt --
Zugriff auf die Argumente des Hauptverbs hat, stattfinden.

Will man eine Grammatik entwickeln, in der es nur einen Lexikoneintrag für das Partizip gibt,
so kann der Agensausdruck nicht vom Partizip selegiert sein, da in Perfektkonstruktionen
wie (\mex{1}a) keine \emph{von}"=PP vorkommt.\footnote{
  Man könnte natürlich behaupten, daß der Agens"=Ausdruck vom Partizip selegiert wird
  und daß das Perfekthilfsverb dann dafür sorgt, daß das Subjekt als NP und 
  nicht als eine \emph{von}"=PP realisiert wird. Eine solche Analyse ist jedoch abwegig,
  da es Verben gibt, die ein Perfekt mit \emph{haben} bilden, jedoch kein Passiv zulassen.
  \eal
  \ex[]{
    Ihm hat vor der Prüfung gegraut.
  }
  \ex[*]{
    Ihm wurde vor der Prüfung gegraut.
  }
  \zl
  Man brauchte also zwei verschiedene \emph{haben}.
}
\eal
\ex Er hat den Weltmeister geschlagen.
\ex Der Weltmeister wurde von ihm geschlagen.
\zl
Man müßte demzufolge annehmen, daß die \vonpp vom Passivhilfsverb selegiert wird. Davon bin ich
in \citew[\page288]{Mueller99a} ausgegangen. Wie ich aber selbst festgestellt habe, ist eine solche Selektion problematisch,
da die Agensausdrücke zusammen mit dem Partizip, aber ohne Hilfsverb im Vorfeld stehen können:
\eal
\label{von-pp-pvp}
%\item Von den Gangstern erstochen wurde Otto.\footnote{
%        \citew[\page146]{Olszok83}.
%}
\ex Von Grammatikern angeführt\iw{anführen} werden auch Fälle mit dem Partizip intransitiver Verben [\ldots]\footnote{
        Im Haupttext von \citew[\page28]{Askedal84} zu finden.
}
\ex Von der Inselregierung gefördert werden kleine Projekte, die sich langsam, aber stetig
        entwickeln.\footnote{
          taz, 11./12.08.2007
}
%% Zustandspassiv
%% \ex Von Riemsdijk entdeckt\iw{entdecken} sind nun Daten, die zeigen, daß es durchaus möglich ist, eine
%%       W-Phrase hinter \emph{glauben} zu haben.\footnote{
%%         Im Haupttext von \citew[\page66]{Fanselow87a}.
%%       }
%% \ex Durch grammatische Fakten belegen läßt sich nur das maskuline Genus von \emph{wer}
%%       bzw.\ das neutrale Genus von \emph{was}.\label{bsp-durch-fakten-belegen-laesst}\footnote{
%%         Im  Haupttext von \citew[\page77]{Pittner96a}.
%%       }
\zl
Diese Daten kann man nicht erklären, wenn man annehmen will, daß \emph{von Grammatikern angeführt} eine
Konstituente mit \emph{angeführt} als Kopf bildet, \emph{von Grammatikern} aber von \emph{werden} abhängt.\footnote{
  Man könnte sich auf eine scheinbar mehrfache Vorfeldbesetzung nach dem Muster von (\ref{bsp-mehr-vf})
  auf Seite~\pageref{bsp-mehr-vf} herausreden. Die Sätze in (\ref{von-pp-pvp}) sind jedoch
  von ihrer Intonation und ihrer Informationsstruktur\is{Informationsstruktur} her anders als die scheinbar mehrfache Vorfeldbesetzung.%
}
Es bleibt also nur die Möglichkeit, den Agensausdruck als Adjunkt (des Partizips) zu behandeln.

Selbst Autoren, die einen separaten Eintrag für das Partizip in Passivkonstruktionen annehmen, behandeln
die Agensausdrücke mitunter als Adjunkte. So argumentiert \zb \citet[\page255]{Zifonun92a}, daß die \emph{von}"=Phrase 
optional\is{Optionalität} und nicht verbspezifisch ist, und behandelt sie deshalb ebenfalls als Adjunkt.


\citet[Kapitel~7]{Hoehle78a} hat darauf hingewiesen, daß das Agens nicht nur durch
\emph{von}"=Phrasen ausgedrückt werden kann. Vielmehr werden allgemeine Inferenzmechanismen und Bezugnahme auf
Weltwissen\is{Weltwissen|(} benutzt, um das Agens zu ermitteln. Exemplarisch soll hier das folgende Beispiel von \citet[\page148]{Hoehle78a}
diskutiert werden:
\ea
Der Verletzte wurde zwischen zwei Sanitätern zum Krankenwagen gebracht.
\z
(\mex{0}) legt nahe, daß die Sanitäter den Verletzten zum Krankenwagen gebracht haben.
Beispiele wie (\mex{1}a) sind semantisch abweichend, da das Agens sowohl
durch die \emph{von}-PP als auch innerhalb der lokativen PP ausgedrückt zu sein scheint.
\eal
\label{ex-sanitaeter}
\ex[\#]{
Der Verletzte wurde von Karl zwischen zwei Sanitätern zum Krankenwagen gebracht.\NOTE{FB: Das
  Beispiel finde ich komisch. Für b muß man fast genauso lange überlegen, um einen passenden Kontext
  zu finden, wie für a.}
}
\ex[]{
Der Verletzte wurde von Karl zwischen zwei Ziegenböcken zum Krankenwagen gebracht.
}
\zl
Ich werde also im folgenden Abschnitt die Agensausdrücke als Adjunkte behandeln. Da
die \emph{von}-PPen, die beim Passiv als Agensausdruck auf"|treten, keine eigene Bedeutung
haben und ohnehin besonders zu behandeln sind, ist es auch gerechtfertigt, die Spezifikation
der PPen so vorzunehmen, daß die Verbindung zur Agensrolle des Hauptverbs hergestellt wird
und das Füllen der Agensrolle nicht Inferenzmechanismen und/""oder Weltwissen\is{Weltwissen|)}
überlassen werden muß.
\is{Agensausdruck|)}
 
%Nach der Diskussion verschiedener Passivformen stelle ich im folgenden
%Abschnitt die Analyse vor.
%}{}


\section{Die Analyse}
\label{sec-passive-anal}

In HPSG"=Grammatiken für das Englische wurden lexikonbasierte Analysen des Passivs vorgeschlagen,
die zwei Lexikoneinträge für die Partizipformen annehmen: einen für das Partizip Perfekt und einen
für das Partizip Passiv. Diese Einträge stehen über eine Lexikonregel zueinander in Beziehung
\citep[\page214--218]{ps}. Entsprechende Vorschläge sind auch aus anderen theoretischen Frameworks
bekannt (siehe Literaturhinweise am Ende dieses Kapitels). Eine Alternative zu solchen Ansätzen
wurde von \citet{Haider86} vorgeschlagen, der nur eine Repräsentation für das Partizip II annimmt.\footnote{
        Siehe auch \citew[\page37]{Bech55a} für eine frühe Anhebungsanalyse.%
}
Die Hilfsverben haben Zugriff auf die Argumente der eingebetteten Partizipien und legen fest,
welche der Argumente syntaktisch realisiert werden. Viele Autoren, die im Rahmen der HPSG arbeiten,
haben Haiders Vorschläge aufgegriffen \citep{Kathol91a,Kathol94a,HM94a,Lebeth94,Pollard94a,Ryu97a,Mueller99a,Mueller2002g,Gunkel2003b}.
Der Vorteil solcher Anhebungsanalysen\is{Anhebung} ist, daß ein einziger Eintrag für das Partizip II zur Analyse
des Perfekts\is{Perfekt} und des Passivs ausreicht. Das Hilfsverb für das Perfekt (\mex{1}a), 
Passiv (\mex{1}b) oder Dativpassiv\is{Passiv!Dativ-} (\mex{1}c) zieht die Argumente
des eingebetteten Partizips \emph{geschenkt} auf eine Weise an, die der jeweiligen Konstruktion
entspricht.
\eal
\label{bsp-kasus-geschenkt}
\ex Der Mann  hat   den Ball   dem Jungen geschenkt.
\ex Der Ball  wurde dem Jungen            geschenkt.
\ex Der Junge bekam den Ball geschenkt.
\zl
Beim Vorgangspassiv (\mex{0}b) wird das logische Subjekt des Hauptverbs unterdrückt und das
Akkusativobjekt wird als Nominativ realisiert. Beim Dativpassiv in (\mex{0}c) wird das Dativobjekt
zum Subjekt.
%% \footnote{
%%         \citet{Lebeth94b}\ia{Lebeth} assumes that the object is not promoted to subject, but is represented
%%         as object. See \citew[\page318]{Mueller99a}\ia{Müller} for some discussion of this approach.%
%% }

Bei Infinitiven im Futur (\mex{1}a), bei AcI"=Konstruktionen\is{Verb!AcI-} (\mex{1}b), 
beim Kausativpassiv\is{Passiv!Kausativ-}\is{Kausativ!-konstruktion} (\mex{1}c)
und bei der Medialkonstruktion\is{Medialkonstruktion} (\mex{1}d) gibt es ebenfalls
keine morphologischen Unterschiede, obwohl der Infinitiv in vielen verschiedenen 
syntaktischen Umgebungen mit unterschiedlichen Argumentrealisierungen verwendet wird:
\eal
\label{bsp-kasus-reparieren}
\ex weil    ein Mechaniker     den Wagen     reparieren wird
\ex weil    Karl       einen Mechaniker     den Wagen     reparieren läßt
\ex weil    Karl       den Wagen     (von einem Mechaniker) reparieren läßt
\ex weil    sich der Wagen     nicht reparieren läßt
\zl
In (\mex{0}a) übernimmt das Hilfsverb die Argumente des eingebetteten Verbs, ohne die Realisierung
der Argumente zu beeinflussen, \dash, die Argumente haben die Form, die sie in einem Satz mit finitem
Hauptverb auch hätten. In (\mex{0}b) wird das Subjekt
von \emph{reparieren} als Objekt von \emph{lassen} realisiert und bekommt Akkusativ.
Die Beispiele in (\mex{0}c,d) kann man ähnlich wie die in (\mex{-1}b)
als Objekt"=zu"=Objekt"=Anhebung
bzw.\ Objekt"=zu"=Subjekt"=Anhebung analysieren (\citealp[\page 191]{Bierwisch90a}; Gunkel \citeyear[\page151]{Gunkel99a}; \citeyear[Kapitel~4.4.2]{Gunkel2003b}).
Daß das logische Subjekt von \emph{reparieren} beim \emph{lassen}"=Passiv\is{Passiv!lassen@\emph{lassen}} in (\mex{0}c)
und in Medialkonstruktionen wie (\mex{0}d) unterdrückt wird, ist in den entsprechenden Lexikoneinträgen
von \emph{lassen} kodiert. 

Bevor die Details der Analyse der Sätze in (\mex{-1}) und (\mex{0}) erörtert werden, muß ich
noch eine Vorbemerkung zur Kennzeichnung unakkusativischer Verben machen: Wie in Abschnitt~\ref{sec-unakkusativitaet}
gezeigt wurde, haben die sogenannten unakkusativischen Verben Subjekte, die sich wie Objekte
verhalten. Aus verschiedenen Gründen, die wir gleich kennenlernen werden, möchte man das Element,
das Subjekteigenschaften hat, syntaktisch speziell hervorheben und gesondert behandeln.
\citet{Haider86}, der im Framework der \gb arbeitet, hat vorgeschlagen, das Argument des Verbs
mit Subjekteigenschaften -- das sogenannte designierte Argument -- in der Argumentstruktur
des Verbs speziell zu markieren.
%Haiders Vorschläge wurden von \citet{HM94a} und \citet{Lebeth94}
%teilweise in die HPSG übernommen. Eine Version von Heinz und Matiaseks Analyse findet man auch bei
%\citet{Gunkel99a}, der sich mit Kausativkonstruktion und dem \emph {lassen}"=Passiv beschäftigt.
Bei unergativischen und transitiven Verben entspricht das Subjekt dem designierten Argument, bei unakkusativischen Verben
gibt es kein designiertes Argument. Ich folge \citet{HM94a} und \citet{Lebeth94}, die ein listenwertiges
Merkmal \textsc{da}\isfeat{da} zur Repräsentation des designierten Arguments verwenden. Wenn es ein designiertes Argument gibt,
ist dieses sowohl Element der \subcatl als auch der \textsc{da}"=Liste.
Die folgende Aufzählung gibt einige prototypische Beispiele:
\ea
\label{da-representation-inf}
\begin{tabular}[t]{@{}l@{ }l@{\hspace{5ex}}l@{\hspace{5ex}}l@{}}
  &                               & \textsc{da}                          & \textsc{subcat}\\[2mm]
a.&ankommen (unakkusativisch):    & \sliste{}                         & \sliste{ NP[\type{str}] }\\[2mm]
b.&tanzen   (unergativisch):      & \sliste{ \ibox{1} } & \sliste{ \ibox{1} NP[\type{str}] }\\[2mm]
c.&auf"|fallen (unakkusativisch): & \sliste{}           & \sliste{ NP[\type{str}], NP[\type{ldat}] }\\[2mm]
d.&lieben      (transitiv):       & \sliste{ \ibox{1} } & \sliste{ \ibox{1} NP[\type{str}], NP[\type{str}] }\\[2mm]
e.&schenken    (ditransitiv):     & \sliste{ \ibox{1} } & \sliste{ \ibox{1} NP[\type{str}], NP[\type{str}], NP[\type{ldat}] }\\[2mm]
f.&helfen      (unergativisch):   & \sliste{ \ibox{1} } & \sliste{ \ibox{1} NP[\type{str}], NP[\type{ldat}] }\\[2mm]
g.&regnen      (unergativisch):   & \sliste{ \ibox{1} } & \sliste{ \ibox{1} NP[\type{str}] }\\
\end{tabular}
\z
Die unakkusativischen Verben \emph{ankommen} und \emph{auf"|fallen} haben die leere Liste als \textsc{da}"=Wert.
Die unergativischen und transitiven Verben haben ihr logisches Subjekt in der \dalist.
Man beachte, daß intransitive Verben wie \emph{ankommen} und \emph{tanzen} und Verben
wie \emph{auf"|fallen} und \emph{helfen}, die ein Dativobjekt verlangen,
nicht anhand der Argumente, die sie verlangen, unterschieden werden können.
\Dh \emph{ankommen} und \emph{tanzen} verlangen jeweils eine NP im Nominativ und \emph{auf"|fallen}
und \emph{helfen} verlangen eine Nominativ"=NP und eine Dativ"=NP.
Theorien, die sich nur auf Valenz beziehen, können also den Unterschied zwischen diesen
Verben in bezug auf Passivierbarkeit nicht erklären. 

Die Annahme eines designierten Arguments für \emph{regnen} muß noch erklärt werden:
Die im Abschnitt~\ref{sec-unakkusativitaet} diskutierten Tests greifen zum Teil nicht,
da Phrasen wie die in (\mex{1}) ohnehin ausgeschlossen sind, da das Subjekt ein Expletivum ist:
\eal
\ex[*]{
Heute wurde geregnet.
}
\ex[*]{
das geregnete Kind/es
}
\zl
Der Test mit den Resultativkonstruktionen läßt sich jedoch anwenden, und wie man sieht,
muß ein Resultativum über ein Objekt und nicht über das \emph{es} prädizieren (siehe auch die Beispiele in
(\ref{bsp-regnen+resultativ-aktiv})):
\ea
Es regnete die Stühle naß.\footnote{
  \citew[\page118]{Wunderlich97c}.
}
\z

\noindent
Für andere Verben, die ein Subjekt haben, das keine semantische Rolle zugewiesen bekommt (Anhebungsverben),
nehme ich an, daß der \daw immer die leere Liste ist.

%% geht nicht, wegen oben
%% Der Wert des \textsc{da}"=Merkmals sollte aus semantischen Eigenschaften des Verbs folgen. 
%% Wenn das erste Element der \subcatl ein Agens ist, \dash, wenn die zugehörige Relation
%% in der semantischen Repräsentation des Verbs eine Relation ist, die ein Agensargument hat,
%% dann wird dieses Element zum \textsc{da}"=Element. Leitet man den \daw auf diese Weise
%% aus semantischen Eigenschaften des Verbs ab, folgt, daß Verben wie \emph{grauen}
%% und \emph{regnen} als \daw die leere Liste haben. In Haiders ursprünglichem Vorschlag
%% korrelierte das Vorhandensein eines designierten Arguments mit der Auswahl des Hilfsverbs:
%% Verben mit designiertem Argument kommen mit \haben vor, Verben ohne designiertes Argument
%% mit \sein. Wetterverben und subjektlose Verben werden immer mit \haben kombiniert:
%% \eal
%% \ex weil ihm vor der Prüfung gegraut hat
%% \ex weil es geregnet/geschneit/gestürmt hat
%% \ex weil es geklingelt/geklopft/gestunken hat
%% \zl
%% Will man also den \daw von semantischen Eigenschaften des Verbs abhängig machen, muß
%% man auf die Hilfsverbauswahl in Abhängigkeit vom \daw verzichten. Wie im 
%% Abschnitt~\ref{sec-phen-hilfsverbselektion} gezeigt wurde, gibt es (teilweise systematische)
%% Ausnahmen im Bereich der Hilfsverbselektion, weshalb man dafür ein Merkmal (\textsc{auxf}\isfeat{auxf})
%% ansetzen sollte.


\subsection{Vorgangspassiv}
\label{sec-agentive-passive-da}

Wie schon erwähnt, hat Haider eine Analyse vorgeschlagen, die das designierte Argument
beim Partizip blockiert. Wird das Partizip in Passivkonstruktionen verwendet, bleibt
das designierte Argument blockiert, wird es in Perfektkonstruktionen verwendet,
deblockiert das Perfekthilfsverb das blockierte Element. Für die Lizenzierung
des Partizips nehme ich die folgende Lexikonregel an, die die Lexikoneinträge in (\mex{2}) 
lizenziert.
%\footnote{
%        Note that it is not necessary to assume a lexical rule.
%        An alternative is to assume that the argument blocking is done by the circumfix\is{circumfix}
%        \emph{ge}- -\emph{t}. Which approach is chosen depends
%        on general assumptions about inflection\is{inflection} and derivation. See Chapter~\ref{sec-morphology-hpsg}
%        for a general discussion. Note furthermore that \hm assume that the lexical rule relates
%        the participle to a bare infinitive. This is a view that is not adopted in this book. Instead,
%        I assume that both the base form and the participle form are related to the stem.

%        The following lexical rules do not contain specifications of the \phonv. The discussion
%        of inflection will be deferred to Chapter~\ref{sec-inflection-hpsg}.%
%}

\eas
Argumentblockierungslexikonregel für Partizipien:\\
\label{lr-da-reduction}%
\begin{tabular}[t]{@{}l@{}}
\ms[stem]{
synsem$|$loc$|$cat & \ms{
head   & \ms[verb]{
         da   & \ibox{1}\\
         }\\
subcat & \ibox{1} $\oplus$ \ibox{2}\\
}\\
} $\mapsto$\\
\ms[word]{
synsem$|$loc$|$cat & \ms{
head & \ms[verb]{
        vform & ppp\\
        subj & \ibox{1}\\
       }\\
subcat & \ibox{2}\\
}\\
}\is{Lexikonregel!Argumentblockierung}
\end{tabular}
\zs

\noindent
Diese Lexikonregel teilt die \subcatl des Eingabezeichens in zwei Teile: Den Teil,
der der \textsc{da}"=Liste entspricht \iboxb{1}, und einen Rest \iboxb{2}.
Nur der Rest wird zur \subcatl des Ausgabezeichens. 
Da ein eventuelles Element der \dalist nicht in der \subcatl des Ausgabezeichens enthalten ist,
kann es in Projektionen des Partizips auch nicht realisiert werden, denn die Dominanzschemata
kombinieren nur Köpfe mit Elementen aus der \subcat- bzw.\ \sprl des jeweiligen Kopfes (siehe
Kapitel~\ref{sec-subj-merkmal} zum Status des \subjms).
(\mex{1}) zeigt die Ausgabe der Regel für die Verben in \pref{da-representation-inf}.
Die \textsc{da}"=Liste wird mit der {\subj}"=Liste des Ausgabeverbs
identifiziert. 

%\begin{figure}
\ea
\begin{tabular}[t]{@{}l@{ }l@{\hspace{5ex}}l@{\hspace{5ex}}l@{}}
  &                               & \textsc{subj}                & \textsc{subcat}\\[2mm]
a.&angekommen  (unakkusativisch): & \sliste{}                 & \sliste{NP[\type{str}]}\\[2mm]
b.&getanzt     (unergativisch):   & \sliste{NP[\type{str}]}  & \sliste{}\\[2mm]
c.&aufgefallen (unakkusativisch): & \sliste{}                 & \sliste{NP[\type{str}], NP[\type{ldat}]}\\[2mm]
d.&geliebt     (transitiv):       & \sliste{NP[\type{str}]}  & \sliste{NP[\type{str}]}\\[2mm]
e.&geschenkt   (ditransitiv):     & \sliste{NP[\type{str}]}  & \sliste{NP[\type{str}], NP[\type{ldat}]}\\[2mm]
f.&geholfen    (unergativisch):   & \sliste{NP[\type{str}]}  & \sliste{NP[\type{ldat}]}\\[2mm]
g.&geregnet    (unergativisch):   & \sliste{NP[\type{str}]}  & \sliste{}\\
\end{tabular}
\z
%\vspace{-\baselineskip}\end{figure}

Die Lexikonregel in (\mex{-1}) erwähnt den \textsc{da}"=Wert in der Ausgabe der Lexikonregel nicht. Es ist aber wichtig,
daß der \textsc{da}"=Wert im Ausgabezeichen ebenfalls enthalten ist, da der \textsc{da}"=Wert dann für die
Analyse der Agensausdrücke von adjektivisch verwendeten Partizipien gebraucht wird
(siehe Abschnitt~\ref{sec-analyse-agensausdruecke}). Nach der Konvention,
die im Kapitel~\ref{sec-lr} erklärt wurde, sind Werte von Merkmalen, die im Output einer
Lexikonregel nicht erwähnt werden, identisch mit den Werten im Input der Lexikonregel.

Der Lexikoneintrag in (\mex{1}) zeigt das Passivhilfsverb:\footnote{
  Siehe auch \citew[\page 224]{HM94a} zu einem ähnlichen Eintrag.%
}

\eas
\label{le-werden-passive-da}
\mbox{\emph{werden} (Passivhilfsverb):}\\
\ms{
head$|$da & \liste{}\\
subcat    & \ibox{1} $\oplus$ \sliste{ V[\type{ppp}, \textsc{lex}+, \textsc{da} \sliste{ NP$_{ref}$}, \textsc{subcat} \ibox{1}] }\\
}
\zs

\noindent
Das Passivhilfsverb verlangt ein Partizip, das ein designiertes Argument (ein Element in der \dalist) hat.
Das schließt die Passivierung unakkusativischer Verben aus, da diese kein Element in \textsc{da} haben.
Da vom \textsc{da}"=Element verlangt wird, daß es referentiell ist (Der Index der NP ist vom Typ
\type{ref}\istype{ref}.), sind Passivierungen von Verben wie \emph{regnen}, die ein expletives Subjekt haben, ausgeschlossen.

Der Lexikoneintrag in (\mex{0}) kann sowohl das persönliche als auch das unpersönliche Passiv
erklären: Wenn \emph{wird} mit \emph{getanzt} oder \emph{geholfen} kombiniert wird,
bekommt man einen Verbalkomplex, der kein Argument selegiert (\emph{getanzt wird}), 
oder einen, der ein Dativobjekt verlangt (\emph{geholfen wird}). Da der Dativ ein lexikalischer
Kasus ist, selegiert der Verbalkomplex kein Element mit strukturellem Kasus, weshalb eine
subjektlose Konstruktion vorliegt, \dash eine Instanz des unpersönlichen Passivs.
Wenn wir \emph{geliebt} oder \emph{geschenkt} mit \emph{wird} kombinieren, bekommen
wir einen Verbalkomplex mit einer Valenzliste, die eine NP mit strukturellem Kasus enthält.
Dieses Element bekommt vom auf Seite~\pageref{case-p} angegebenen Kasusprinzip\is{Prinzip!Kasus-}
Nominativ zugewiesen, weshalb wir dann ein persönliches Passiv haben.

Nachdem ich gezeigt habe, wie das persönliche und das unpersönliche Passiv erklärt werden
kann, soll noch diskutiert werden, wodurch Doppelpassivierungen wie in (\mex{1}c) ausgeschlossen sind.
\eal
\label{ex-double-application-of-passive}
\ex[]{
weil er den Film liebt
}
\ex[]{
weil der Film geliebt wurde
}
\ex[*]{
weil geliebt worden wurde
}
\zl
Der Satz in (\mex{0}b) ist das persönliche Passiv von (\mex{0}a). Ohne Beschränkungen
für die Passivierung könnte man zu (\mex{0}b) ein unpersönliches Passiv bilden, das dann (\mex{0}c)
entspräche. (\mex{0}c) ist jedoch durch die Spezifikation des \dawes im Lexikoneintrag
des Passivhilfsverbs in (\ref{le-werden-passive-da}) ausgeschlossen. Der \textsc{da}"=Wert des Passivhilfsverbs
ist die leere Liste. Deshalb ist das Ergebnis der Kombination des Hilfsverbs mit dem Partizip parallel
zu unakkusativischen Simplexverben. Da die Einbettung unakkusativischer Verben unter das
Passivhilfsverb durch die Spezifikation der Valenz des Passivhilfsverbs ausgeschlossen ist,
kann auch der unakkusativische Verbalkomplex \emph{geliebt worden} nicht unter 
das Passivhilfsverb \emph{wurde} in (\mex{0}c) eingebettet werden.

Im Gegensatz zum Passivhilfsverb in \pref{le-werden-passive-da}
deblockiert das Perfekthilfsverb in (\mex{1}) das designierte Argument.
Es macht die Verknüpfung des \subjwes und der \subcatl des eingebetteten Partizips
zu seinem eigenen \subcatw.

\eas
\mbox{\haben (Perfekthilfsverb):}\\
\label{le-haben-perfekt}
\ms{
head$|$da & \eliste\\
subcat    & \ibox{1} $\oplus$ \ibox{2} $\oplus$ \sliste{ V[\type{ppp}, \textsc{lex}+, \textsc{subj} \ibox{1}, \textsc{subcat} \ibox{2}] }\\[2mm]
}
\zs

\noindent
Das blockierte designierte Argument -- das ja wegen der Lexikonregel (\ref{lr-da-reduction}) in der \subjl der Partizipform
repräsentiert ist -- wird also vom Hilfsverb wieder in die \subcatl aufgenommen.\footnote{
  Es mag hier verwunderlich erscheinen, daß auf den \subjw des eingebetteten Verbs und nicht
  auf den \daw Bezug genommen wird. Der Grund hierfür ist, daß man bei der Bezugnahme
  auf den \subjw über das Perfekthilfsverb \haben und das \haben in Konstruktionen mit modalem
  Infinitiv (siehe Abschnitt~\ref{sec-analysis-da-modal-inf}) generalisieren kann. Die \catwe der beiden Hilfsverben entsprechen
  (\ref{le-haben-perfekt}). Die jeweiligen Konstruktionen unterscheiden sich nur in der Semantik und in der Art, wie
  die Form des infiniten Verbs gebildet wird.%
}
Wenn der \subjw des eingebetteten Verbs die leere Liste ist, \dash, wenn ein subjektloses Verb unter \haben 
eingebettet wird, dann wird zur \subcatl des eingebetteten Verbs nichts hinzugefügt (\,\iboxt{1} = \sliste{}).
Da auch nichts blockiert wurde, ist das genau das, was erwünscht ist:
Im Perfekt werden alle Argumente realisiert. Das Hilfsverb ist
ein Anhebungsverb, \dash, im Lexikoneintrag sind keinerlei Restriktionen in bezug auf ein anzuhebendes
Subjekt formuliert. Somit können auch Verben mit expletivem\is{Verb!expletives} Subjekt 
und subjektlose Verben\is{Verb!subjektloses} unter \haben eingebettet
werden.
\eal
\ex\iw{regnen}
Es hat geregnet.
\ex\iw{grauen}\label{ex-hat-gegraut}
Dem Student hat vor der Prüfung gegraut.
\zl
Man beachte, daß (\mex{0}b) ausgeschlossen ist, wenn man verlangt, daß das
unter \haben eingebettete Verb seinem Subjekt eine semantische Rolle zuweist,
wie das \zb \citet[\page183]{Stechow90a} vorschlägt. Von Stechow kritisiert Haiders Ansatz
(Deblockade eines blockierten Arguments). Sein Versuch, die Beschränkungen mit Bezug
auf Rollenzuweisung zu formulieren, ist jedoch nicht äquivalent zu Haiders Ansatz. Bei Haider
gibt es die Möglichkeit, daß nichts blockiert wurde, also auch nichts deblockiert werden muß,\footnote{
        Haider (\citeyear[\page28]{Haider84a}; \citeyear[\page10]{Haider86}) macht die Auxiliarselektion allerdings explizit
% auch Haider/Rindler-Schjerve87a
        an dem Vorhandensein eines designierten Arguments fest. Verben mit designiertem
        Argument nehmen \haben, Verben ohne designiertes Argument nehmen \sein. Er muß
        also annehmen, daß \emph{grauen} ein leeres Subjekt hat, das das designierte
        Argument ist, da er sonst als Auxiliar \emph{sein} vorhersagen würde.%
}
von Stechows Beschränkung schließt dagegen sowohl (\mex{0}a) als auch (\mex{0}b) aus.
Für Wetterverben wird mitunter angenommen, daß sie dem Expletivum eine sogenannte
Quasi"=Argumentrolle\is{semantische Rolle!Quasi"=Theta"=Rolle} zuweisen. Bei \emph{grauen} müßte man
etwas Ähnliches annehmen, um von Stechows Analyse retten zu können. Die Konsequenz wäre dann, daß \grauen
sowohl einem leeren Element als auch einem Expletivum eine Rolle zuweisen könnte,
da auch Sätze wie (\mex{1}) möglich sind:
\ea
Dem Student hat es vor der Prüfung gegraut.
\z
Wetterverben\is{Verb!Wetter-} könnten aber nur dem Expletivum eine Rolle zuweisen, ein leeres
Element ist als Subjekt nicht erlaubt. Wie schon auf Seite~\pageref{page-syntaktisierung} gesagt,
denke ich, daß Analysen, die davon ausgehen, daß Wetterverben dem Expletivum eine Quasi"=Rolle
zuweisen, syntaktische Fakten unzulässigerweise in die Semantik verschieben. 

Das Perfekthilfsverb \sein ähnelt \haben. Es unterscheidet sich lediglich 
in der Deblo"ckierung des designierten Arguments des eingebetteten Partizips: Der \subjw
des eingebetteten Verbs wird nicht mit dem \subcatw verknüpft, nur die in \subcat repräsentierten
Argumente \iboxb{1} werden zu Argumenten des Hilfsverbs angehoben:
\eas
\mbox{\sein (Perfekthilfsverb):}\\
\label{le-sein-perfekt}
\ms{
head$|$da & \sliste{} \\
subcat    & \ibox{1} $\oplus$ \sliste{ V[\type{ppp}, \textsc{lex}+, \textsc{subcat} \ibox{1}] }\\[2mm]
}
\zs
Da bei der Partizipbildung mit unakkusativischen Verben wie \emph{angekommen} und \emph{aufgefallen}
nichts blockiert wurde, muß auch nichts deblockiert werden. Wie wir im
Abschnitt \ref{sec-analysis-da-modal-inf} sehen werden, ist der Eintrag für das \sein mit modalen
Infinitiven völlig parallel: Blockierte Argumente werden nicht deblockiert.


Man beachte, daß man mit diesem Lexikoneintrag die im Abschnitt~\ref{sec-unaccusativity} 
diskutierten Beispiele in \pref{bsp-eingegangen-all} und \pref{bsp-angegangen-all} --
hier als (\mex{1}) wiederholt -- nicht analysieren kann. 
\eal
\label{bsp-eingegangen-all-zwei}
%\ex Er ist einen Kompromiß eingegangen.
\ex\iw{eingehen}
"`Wir sind eine vertragliche Verpflichtung eingegangen, und zu dieser stehen wir"' [\ldots]\footnote{
        taz, 6.3.2002, S.\,9.%
}
%, erkla"rte Gu"nter Marquis, Pra"sident des Verbandes der Elektrizita"tswirtschaft (VDEW) gestern in Berlin.
%taz Nr. 6693 vom 6.3.2002, Seite 9, 84 Zeilen (TAZ-Bericht), NICK REIMER
\ex\label{bsp-eingegangen-zwei}
Für jeden Job, [\ldots] bei dem Verantwortung übernommen werden oder hin und wieder gar
    ein Kompromiss eingegangen werden muß, ist der ehemalige Finanzminister absolut ungeeignet.\footnote{
        taz, 28.05.2002, S.\,14.%
    }
% Es wird also hier mit der folgenden Hypothese angetreten [\ldots] WL2001a:9
% \ex "`Es ist unfair, wenn der Trainer von unseren Anhängern verbal angegangen wird."'\footnote{
%         Mannheimer Morgen, 23.11.1998.%
%      }
%
% auch: die eingegangene Verpflichtung  www.dse.de/za/lis/ruanda/seite2.htm - 16k - 27. Juni 2004
\zl
\eal
\label{bsp-angegangen-all-zwei}
\ex\iw{angehen}
"`Wären wir beim Ocean Race so gesegelt, wie wir die Kampagne um den America's Cup angegangen sind, 
hätten wir das Ziel nicht erreicht"', musste er eingestehen.\footnote{
        taz Hamburg, 11.6.2002, S.\, 24.% %, 99 Zeilen (TAZ-Bericht), OKE GO"TTLICH
}
%
\ex\label{bsp-angegangen-zwei}
Ob die finanziell
aufwendige Restaurierung nun tatsächlich angegangen wird oder ob die Wandmalereien lediglich fachgerecht konserviert werden, hat der
Heidelberger Gemeinderat demnächst zu entscheiden.\footnote{
        Mannheimer Morgen, 20.01.1989.%
}
\zl
Die a-Beispiele zeigen, daß das Perfekt mit \sein gebildet wird, und die b"=Beispiele zeigen, daß
die Verben eine Passivierung erlauben.
Zeichnet man das Subjekt von \word{angehen} und \word{eingehen} 
als designiertes Argument aus, sollte es beim Partizip blockiert sein und in der
Perfektkonstruktion deblockiert werden. Nur \haben deblockiert Argumente.
Entscheidet man sich dafür, kein Argument zum designierten Argument zu machen,
sollte das Passiv ausgeschlossen sein. Der einzige Ausweg aus diesem Dilemma
scheint die Stipulation eines zusätzlichen Eintrags für \sein zu sein,
der dem Eintrag von \haben in \pref{le-haben-perfekt} gleicht und der 
mit Ausnahmeverben wie \emph{angehen} und \emph{eingehen} kombiniert werden kann.

Interessanterweise kommt die hier vorgestellte Analyse mit den folgenden Daten zurecht:
\eal
\label{bsp-subjekt-im-vf-passiv}
\ex\iw{brechen}
Die Nase gebrochen wurde dem Boxer schon zum zweiten Mal.\footnote{
        \citew[\page386]{Luehr84a}.
}
\ex\iw{erschießen}
%\NOTE{WS: Das ist ein seeeeeehr merkwürdiger Satz, der zumindest ein \# verdient hat. FB findet das nicht}
Zwei Männer erschossen wurden    während des Wochenendes.\footnote{
        \citew*[\page210]{Webelhuth85a}\ia{Webelhuth}.%
        }
\label{bsp-subjekt-im-vf-passiv-letzt}
% Worden steht aber mit vorn.
% \ex\iw{bauen}
% Ein Haus gebaut worden ist gestern nicht.\footnote{
%         \citew[\page433]{Sternefeld85a}.
% }
%%% Ist Zustandspassiv
%% \ex\iw{zurückgewinnen}\iw{sein!stative passive}
%% Ein verkanntes Meisterwerk dem Musiktheater  zurückgewonnen ist da    nicht.\label{bsp-subjekt-im-vf-zp}\footnote{
%%         ECI Multilingual Corpus CD I, Frankfurter Rundschau Korpus, File ger03a01.eci 
%%         (FR Woche, die am 05.07.1992 endet).
%%         Ich danke Wojciech Skut\ia{Skut}\aimention{Wojciech Skut}
%%         der dieses Beispiel 1998 für mich im \negra"=Korpus gesucht hat und Thorsten Brants\ia{Brants}\aimention{Thorsten Brants}
%%         für das Auffinden der genauen Quelle.%
%%       }
% \ex Der Punkt erledigt ist niemals.\footnote{
%         Interview mit Voscherau, ARD, 20.05.1988, 22:45 Uhr, zitiert nach \citew[\page111]{Duerscheid89a}.
%     }
\zl
Das Subjekt von \emph{gebrochen} bzw.\ \emph{erschossen} %und \emph{zurückgewonnen}
wird durch die Argumentblockierungsregel blockiert. Das Objekt ist damit das erste
Element in der \subcatl der entsprechenden Verben. Das Passivhilfsverb deblockiert kein
Element, es verlangt lediglich, daß das eingebettete Verb ein referentielles designiertes Argument
hat, wo die anderen Argumente realisiert werden, wird vom Passivauxiliar nicht vorgeschrieben.
Das zugrundeliegende Objekt kann also mit dem Partizip im Vorfeld oder wie in (\mex{1})
im Mittelfeld realisiert werden:
\eal
\ex Erschossen wurden zwei Männer während des Wochenendes (nicht erstochen).
\ex Während des Wochenendes wurden zwei Männer erschossen.
\zl

\noindent
Das folgende unakkusativische Verb zeigt, daß diese Verben sich parallel zu den Passiven verhalten:
\ea
Kein Kraut gewachsen ist allerdings dagegen, dass zum Beispiel Suchmaschinendienste wie Google Ranking"=Eintragungen
nach hinten rutschen lassen, wenn [\ldots]\footnote{
c't, 9/2005, S.\,93.
}
\z
Da das Verb \emph{gewachsen} kein designiertes Argument hat, wird das Subjekt \emph{kein Kraut} nicht blockiert.
Es ist also das erste Element der \subcatl und kann als solches zusammen mit dem Partizip im Vorfeld stehen.
Das Perfekthilfsverb \emph{sein} deblockiert kein Argument, in (\mex{0}) werden lediglich die Argumente aus der \subcatl
des eingebetteten Verbs angezogen, die noch nicht im Vorfeld realisiert wurden.
% This approach has the advantage that participles always have passive valence properties.
% They may therefore be input to adjective formation lexical rules that produce adjectives
% which can be used in an analysis of (\ref{ex-adj-participle}). The adjective formation
% lexical rule does not have to license lexical items that have valence properties that are
% different from those of the participle, as would be necessary in Pollard's approach. Therefore
% the fact that passive changes valence properties is represented at one place in the grammar, namely
% the formation of the participle, and not distributed over different analyses of the phenomena
% of the formation of passive verbal complexes and adjective formation.


%% \subsection{Zustandspassiv}
%% \label{sec-stative-passive-da}


%% Das Zustandspassiv, das anhand der Beispiele in (\ref{ex-stative-passive}) diskutiert wurde,
%% kann man völlig parallel zum Vorgangspassiv analysieren.
%% Der in (\mex{1}) angegebene \catw des Hilfsverbs \sein für das Zustandspassiv ist deshalb
%% mit dem \catw des Hilfsverbs für das Vorgangspassiv in (\ref{le-werden-passive-da}) identisch:

%% \eas
%% \mbox{\stem{sei} (Hilfsverb für das Zustandspassiv):}\\
%% \label{le-sein-stative}
%% \ms{
%% da & \liste{}\\
%% subcat    & \ibox{1} $\oplus$ \liste{ V[\type{ppp}, \textsc{lex}+, \textsc{da} \sliste{ NP$_{ref}$}, \textsc{subcat} \ibox{1}] }\\[2mm]
%% }
%% \zs
%% Vom Perfekthilfsverb \sein unterscheidet sich der Lexikoneintrag darin, daß
%% er das Vorhandensein eines designierten Arguments verlangt. Damit ist ein Zustandspassiv
%% mit unergativischen Verben ausgeschlossen. Subjektlose Verben wie in (\mex{1})
%% können ebenfalls nicht unter das Zustandspassiv"=\sein eingebettet werden.
%% \ea[*]{\iw{grauen}
%% Dem Student ist (vom Professor) vor der Prüfung gegraut.
%% }
%% \z

%% \noindent
%% Die Spezifikation des \textsc{da}"=Elements als referentiell schließt die Passivierung
%% von expletiven Verben aus:
%% \ea[*]{\iw{regnen}
%% Ist heute geregnet?
%% }
%% \z

%% An dieser Stelle muß darauf hingewiesen werden, daß die Datendiskussion im Abschnitt~\ref{sec-unakkusativitaet}
%% sich ausschließlich auf die Realisierung von Argumenten bezieht.
%% Das heißt, es wurden Aussagen darüber gemacht, welche syntaktischen
%% Eigenschaften ein Argument haben muß, damit es als Subjekt realisiert werden kann,
%% damit es mit dem modifizierten Nomen identifiziert werden kann bzw.\ damit sich
%% ein Resultativprädikat darauf beziehen kann. Es wurde nicht gesagt, 
%% daß die Tatsache, daß ein Verb transitiv ist, auch bedeutet,
%% daß alle Passivformen mit diesem Verb bildbar sind. Es ist vielmehr so, daß es zum Beispiel
%% nicht für alle Verben, die ein Vorgangspassiv
%% erlauben, auch ein Zustandspassiv gibt (siehe Abschnitt~\ref{sec-passiv-state}).
%% Der Unterschied zwischen beiden Passivvarianten ist semantischer Natur: Beim
%% Zustandspassiv wird der Endzustand der Handlung besonders betont. Bei Tätigkeiten,
%% die keinen besonderen Endzustand beschreiben, ist die Bildung des Zustandspassivs schlecht.
%% Das heißt, eine Theorie, die die Zusammenhänge vollständig erfassen will, muß
%% auch semantische Aspekte, die mit der Passivierung interagieren, modellieren.
%% Eine solche semantische Theorie läßt sich prinzipiell in die hier vorgestellte Analyse
%% integrieren, würde jedoch den Rahmen dieses Buches sprengen.


\subsection{Dativpassiv}
\label{sec-anal-dativpassiv}

Das\is{Passiv!Dativ-|(}
Dativpassiv kann wie das Vorgangspassiv %und das Zustandspassiv 
über Argumentanziehung
erklärt werden. Wie im Abschnitt~\ref{sec:dat-pass} gezeigt wurde, ist das Dativ"=Passiv
mit unakkusativischen Verben nicht möglich. Der folgende Lexikoneintrag
verlangt, daß das eingebettete Verb ein designiertes Argument hat. Deshalb
werden Dativ"=Passive von unakkusativischen Verben von der Grammatik nicht zugelassen:

\eas
\label{le-bekommen-passive-da}
\mbox{\emph{bekomm-} (Dativpassivhilfsverb):} \\
\ms{
 head$|$da & \sliste{ } \\[2mm]
 subcat    & \sliste{ NP[\str]\ind{1}} $\oplus$ \ibox{2} $\oplus$ \ibox{3} $\oplus$ \liste{ \begin{tabular}{@{}l@{}l@{}}
                  \textrm{V[}&\textrm{\type{ppp}, \textsc{lex}+, \textsc{da}~\sliste{ NP$_{\type{ref}}$ },}\\[3mm]
                          &\textrm{\textsc{subcat}~\ibox{2} $\oplus$ \sliste{ NP[\type{ldat}]\ind{1}} $\oplus$ \ibox{3}]} \\
                  \end{tabular}}\\
}\iw{bekommen!passive|uu}
\zs

\noindent
Das Subjekt des Dativ"=Passiv"=Hilfsverbs ist mit dem Dativargument des eingebetteten Verbs
koindiziert. Alle Elemente der \subcatl des eingebetteten Verbs mit Ausnahme des Dativobjekts, das
zum Subjekt wird, werden in die \subcatl von \emph{bekommen} angehoben (\,\iboxt{2} $\oplus$
\iboxt{3}\,). Das Dativobjekt kann nicht direkt angehoben werden, da der Dativ ein lexikalischer Kasus
ist und somit mit dem Kasuswert, den das Subjekt im Dativpassiv haben muß, unverträglich ist.

Der Lexikoneintrag in (\mex{0}) weicht von dem von \citet[\page228]{HM94a} vorgeschlagenen ab:
In (\mex{0}) wird nicht verlangt, daß die \subcatl des eingebetteten Verbs mit einer NP mit strukturellem Kasus
anfängt. Wie im Abschnitt~\ref{sec:dat-pass} gezeigt wurde, ist das Dativpassiv nicht auf Verben
beschränkt, die neben dem Dativ auch noch einen Akkusativ regieren. Da der Wert, den \iboxt{2} 
haben kann, nicht vorgeschrieben ist, können Sätze wie (\mex{1}) ebenfalls analysiert werden:
\ea\iw{helfen}
Ich bekam (von Karl) geholfen.
\z
Bei der Analyse von (\mex{0}) ist der Wert von \ibox{2} in (\mex{-1}) die leere Liste.

In Sätzen wie \pref{ex-ich-bekommen-ein-buch-geschenkt} -- hier als (\mex{1}) wiederholt -- 
hat das eingebettete Verb ein direktes Objekt, und \iboxt{2} ist demzufolge \sliste{ NP[\str] }.
\ea\label{ex-ich-bekommen-ein-buch-geschenkt-zwei} 
Ich     bekomme ein Buch       geschenkt.
\z

% Auf die Finger ist doch ein Adjunkt, oder?
%
%% In der Analyse von (\ref{bsp-auf-die-finger-geklopft-bekommen}) -- hier als (\mex{1})
%% wiederholt -- ist die \iboxt{3} in (\ref{le-bekommen-passive-da}) nicht leer.
%% Sie enthält das PP"=Komplement, das dann als \emph{auf die Finger} realisiert wird.
%% \ea
%% daß wir noch nachsitzen mußten und auf die Finger geklopft bekamen.\footnote{
%%         Frankfurter Rundschau, 03.06.1998, S.\,2.%
%% }
%% \z

\noindent
Da der \daw des Hilfsverbs die leere Liste ist, ist eine Doppelpassivierung wie
in (\mex{1}) ausgeschlossen:
\ea[*]{
In diesem Saal sind viele Preise verliehen bekommen worden.\footnote{
        \citet{Kathol91a}\ia{Kathol} markiert diesen Satz mit einem Fragezeichen,
        eine Einschätzung, die ich nicht teile. Man könnte die Theorie so modifizieren,
        daß (\mex{1}) analysierbar wird: Man muß dazu nur den \daw von \emph{bekommen}
        anders spezifizieren.%
}
}
\z

\noindent
Zum Lexikoneintrag in (\ref{le-bekommen-passive-da}) muß noch eine Anmerkung gemacht werden:
In (\ref{le-bekommen-passive-da}) ist der referentielle Index des Dativobjekts des eingebetteten
Verbs mit dem Index des Subjekts von \emph{bekommen} identisch \iboxb{1}. Insofern
ähnelt \emph{bekommen} einem Kontrollverb\is{Verb!Kontroll-}. In den Datenkapiteln habe ich gezeigt,
daß es nicht sinnvoll ist anzunehmen, daß \emph{bekommen} einer Nominalphrase eine
semantische Rolle zuweist. Im Lexikoneintrag für \emph{bekommen} wird auch keinem Element eine
Rolle zugewiesen. Das heißt, daß \emph{bekommen} in (\ref{le-bekommen-passive-da}) kein Kontrollverb ist,
es gibt lediglich eine Koindizierung von NPen im Lexikoneintrag. Eine ähnliche Situation gibt
es bei den inhärent reflexiven Verben\is{Verb!inhärent reflexives}:
\eal
\ex Ich erhole mich.
\ex Du erholst dich.
\ex Er erholt sich.
\zl
Das Reflexivum muß mit dem Subjekt in der Person übereinstimmen, was durch Koindizierung erreicht wird.
Das Reflexivum ist ein syntaktisches Argument des Verbs, bekommt vom Verb aber keine semantische Rolle zugewiesen.
Parallel hierzu weist \emph{bekommen} seinen Argumenten keine semantische
Rolle zu, es gibt aber Koindizierungen zwischen den Argumenten. Zur Diskussion einer Analyse,
die davon ausgeht, daß die gesamten \type{synsem}"=Objekte des Dativobjekts des eingebetteten Verbs
und des \emph{bekommen}"=Subjekts identifiziert werden, siehe Kapitel~\ref{sec-alternativen-struc-dativ}.


\begin{comment}
The embedding of the dative passive under {\em sein}, which is marginally possible \citep[\page6]{Haider86}\ia{Haider},
can also be explained:
\eal
\label{ex-ist-zu-bekommen}
\ex[?]{
So   etwas     ist leicht geschenkt zu kriegen.
}
\ex[?]{
So ein Preis ist leicht zugesprochen zu kriegen.
}
\zl
Since the lexical entry of \emph{zu kriegen}, which is parallel to that of \emph{bekommen}
we saw in (\ref{le-bekommen}), is specified to raise the \textsc{acc} value of its verbal complement
\emph{geschenkt} in (\mex{0}a), the modal \sein can raise the object of \emph{geschenkt zu kriegen}
to the subject of the complete verbal complex.
Unfortunately this specification of \accf also allows sentences like (\mex{1}) which
I find unacceptable:
\ea[*]{
In diesem Saal sind viele Preise verliehen bekommen worden.\footnote{
        \citet{Kathol91a}\ia{Kathol} marks this sentences with a question mark.
}
}
\z
(\mex{-1}) and (\mex{0}) can be ruled out by assuming that the dative passive auxiliary
is unakkusativisch as the other passive auxiliaries, \ie, that the \textsc{subj} value and the
\textsc{acc} value of \emph{bekommen} are identical.
\end{comment}
\is{Passiv!Dativ-|)}




\subsection{Modale Infinitive}
\label{sec-analysis-da-modal-inf}


\is{Infinitiv!modaler|(}\is{Passiv!modaler Infinitiv|(}
Haider hat dafür argumentiert, modale Infinitive mit \sein nicht völlig parallel
zum Passiv zu behandeln. Anstatt ebenfalls das designierte Argument zu blockieren,
läßt er das Infinitiv"=\emph{zu} das syntaktische Subjekt, \dash das Element, das 
in Aktivsätzen Nominativ bekommt, blockieren. Unsere Beispielverben
haben die folgenden \da-, \subj- und \subcatwe:
{
\exewidth{(100)}
\ea
\begin{tabular}[t]{@{}l@{~}l@{~~}l@{~~~}l@{~~~}l@{}}
  &                               & \textsc{da}            & \textsc{subj}                           & \textsc{subcat}                                \\[2mm]
a.&anzukommen  (unakkusativisch): & \sliste{}           & \sliste{ NP[\type{str}] }             & \sliste{}                                   \\[2mm]
b.&zu tanzen   (unergativisch):   & \sliste{ \ibox{1} } & \sliste{ \ibox{1} NP[\type{str}] }    & \sliste{}                                   \\[2mm]
c.&aufzufallen (unakkusativisch): & \sliste{}           & \sliste{ NP[\type{str}] }             & \sliste{ NP[\type{ldat}] }                  \\[2mm]
d.&zu lieben   (transitiv):       & \sliste{ \ibox{1} } & \sliste{ \ibox{1} NP[\type{str}] }    & \sliste{ NP[\type{str}] }                    \\[2mm]
e.&zu schenken (ditransitiv):     & \sliste{ \ibox{1} } & \sliste{ \ibox{1} NP[\type{str}] }    & \sliste{ NP[\type{str}], NP[\type{ldat}] }  \\[2mm]
f.&zu helfen   (unergativisch):   & \sliste{ \ibox{1} } & \sliste{ \ibox{1} NP[\type{str}] }    & \sliste{ NP[\type{ldat}] }                   \\[2mm]
g.&zu regnen   (unergativisch):   & \sliste{ \ibox{1} } & \sliste{ \ibox{1} NP[\type{str}] }    & \sliste{}                                   \\
\end{tabular}
\z

\noindent
Das Hilfsverb \haben deblockiert das blockierte Element in \subj und das Hilfsverb \sein
beläßt es im blockierten Zustand.
%% 
%% Anhebung ist in den beiden Fällen immer Anhebung von SUBJ und SUBCAT, also
%% macht es auch keinen Unterschied, von wo etwas angehoben wird.
%%
%% Haider repräsentiert das Subjekt von Infinitiven mit \emph{zu}
%% anders als das von Infinitiven ohne \emph{zu}:
%% Das Subjekt der \emph{zu}"=Infinitive wird beim ihm blockiert, das Subjekt der Infinitive
%% ohne \emph{zu} dagegen nicht.
%% In der HPSG"=Umsetzung würde das der Repräsentation des Subjekts auf verschieden Listen entsprechen.
%% Infinitive ohne \emph{zu} hätten ihr Subjekt auf der \subcatl, bei Infinitiven
%% mit \emph{zu} wäre das Subjekt dagegen in der \subjl repräsentiert.\isfeat{subj}
%% Der Nachteil dieser Repräsentation wäre, daß die Klasse der Subjektanhebungsverben.
%% The drawback of such an encoding is that one cannot describe
%% the class of subject raising verbs in a uniform way, as was possible in Chapter~\ref{subj-rais-anal}.
%% Modal verbs which select bare infinitives would raise the subject of their verbal complements from the
%% \subcatl while verbs like \emph{scheinen} (`seem') or \emph{anfangen} (`begin') which select an infinitive with \emph{zu}
%% would raise the subject of the embedded verb from its \subjl.
%% %
%% In order to be able to capture the generalization about raising verbs, I suggest representing both
%% the subject of bare infinitives and the subject of infinitives with \emph{zu} in the \subjl.
%% The only non-finite form that has a different representation of the subject (if there is any) is the
%% participle.

% \citet[\page139]{Gunkel99a}\iaright{Gunkel} defines the external argument as the one
% argument with sufficient proto"=agent properties. This excludes expletives as external
% arguments. The referentiality of the logical subject in passive constructions is therefore
% required by the passive auxiliary rather implicitly. In Gunkel's approach being an
% external argument is a lexical"=semantic property.
% If expletives are allowed as external arguments\footnote{
%         If the choice of the auxiliary is made dependent upon there being a designated
%         argument or not, expletives must be allowed as external arguments since \word{regnen} (`to rain')
%         takes \haben. See \citew[Chapter~6.6.2]{HM94a}\ia{Heinz}\ia{Matiasek} for
%         auxiliary selection that is dependent on the external argument.
% }, one has to make the requirement that the logical subject is referential explicit.



% In case of an embedded \emph{zu}"=infinitive the \subj value corresponds
% to the syntactic subject. This subject is inserted into the \subcat list
% of \haben.


Die Beschreibung in (\mex{1})\vpageref{lr-subj-reduction} entspricht einem Supertyp
der Lexikonregeln, die Wörter in der \type{bse}- und in der \type{inf}-Form
lizenzieren. Im Gegensatz zur Partizipregel wird der \daw des Stammes
ignoriert. Statt dessen wird das erste Argument mit strukturellem
Kasus\is{Kasus!struktureller} im \subjw der Regelausgabe repräsentiert.
\textit{first-np-str} ist eine relationale Beschränkung, die die Liste
\iboxt{1} in die beiden Teile \iboxt{2} und \iboxt{3} teilt, wobei \iboxt{2}
die erste NP mit strukturellem Kasus aus \iboxt{1} enthält, wenn es eine gibt,
und \iboxt{3} die anderen Elemente von \iboxt{1} enthält. Wenn \iboxt{1} 
kein Element mit strukturellem Kasus enthält, ist \iboxt{2} die leere Liste und \iboxt{1}
und \ibox{3} sind identisch.\isfeat{subj}
%
%\begin{figure}[htbp]
\eas
\label{lr-subj-reduction}
Argumentblockierungslexikonregel für Infinitive mit und ohne \emph{zu}:\\
\begin{tabular}[t]{@{}l@{}}
\ms[stem]{
synsem$|$loc$|$cat & \ms{
head   & verb\\
subcat & \ibox{1}\\
}\\
} $\mapsto$\\
\ms[word]{
synsem$|$loc$|$cat & \ms{
head & \ms[verb]{
        vform & inf-or-bse\\
        subj & \ibox{2}\\
       }\\
subcat & \ibox{3}\\
}\\
}\\$\wedge$ \textit{first-np-str}(\,\ibox{1}, \ibox{2}, \ibox{3}\,)\\
\end{tabular}\is{Lexikonregel!Argumentblockierung}
\zs
%\vspace{-\baselineskip}\end{figure}
%
%\noindent
Werte von Merkmalen, die in einer Lexikonregel nicht erwähnt werden, werden
per Konvention von der Eingabe der Regel zur Ausgabe übertragen,
weshalb der \daw des \emph{zu}- bzw.\ \type{bse}"=Infinitivs mit dem
\daw des Stamms identisch ist. Das ist für die Analyse des \emph{lassen}"=Passivs\is{Passiv!lassen@\emph{lassen}}
wichtig (siehe Abschnitt~\ref{lassen-passive-da}).

Lexikoneinträge für das Partizip II unterscheiden sich also dadurch von den Einträgen
anderer nicht"=finiter Formen, daß sie ein wirkliches Subjekt, in \subj haben, \dash eine NP, die in
Aktivsätzen Nominativ bekommt und mit dem Verb kongruiert und keine der im
Abschnitt~\ref{sec-unakkusativitaet} besprochenen Objekteigenschaften hat,
während bei den Infinitiven mit und ohne \emph{zu} alle Subjekte, \dash auch die mit
Objekteigenschaften, in \subj repräsentiert werden.

Die Lexikoneinträge für die Hilfsverben, die in Konstruktionen mit modalen Hilfsverben
benutzt werden, sind parallel zu den Perfekthilfsverben in \pref{le-haben-perfekt} bzw.\ \pref{le-sein-perfekt}: 
\haben deblockiert das logische Subjekt des \emph{zu}"=Infinitivs und \sein läßt
das logische Subjekt blockiert: Der Lexikoneintrag für \haben verlangt vom eingebetteten
Verb nicht, daß es ein Subjekt haben muß. Es wird überhaupt keine Beschränkung in bezug auf \subj
formuliert, weshalb sowohl die Einbettung eines expletiven Prädikats 
wie in (\mex{1}a) als auch die Einbettung eines subjektlosen Prädikats wie in (\mex{1}b) möglich ist.
\eal
\ex 
Es hat zu regnen.
\ex 
Den Studenten hat vor der Prüfung zu grauen.
\zl
Im Eintrag für das \sein, das in modalen Konstruktionen benutzt wird, wird genauso wie beim Vorgangspassiv
verlangt, daß das eingebettete Verb ein referentielles Element in \textsc{da} hat, weshalb
entsprechende Beispiele mit \sein ausgeschlossen sind:
\eal
\ex[*]{ 
Heute ist zu regnen.
}
\ex[*]{ 
Den Studenten ist vor der Prüfung zu grauen.
}
\zl

\noindent
Außerdem wird der folgende von \citet[\page137]{Haider90b} erwähnte Unterschied 
korrekt vorhergesagt:
\eal
\ex[]{
daß ihm nicht zu helfen ist
}
\ex[*]{
daß ihm nicht geholfen zu werden ist
}
\zl
\emph{zu helfen} hat ein designiertes Argument, auf das das Hilfsverb \emph{ist} zugreifen kann.
Der Verbalkomplex \emph{geholfen zu werden} ist dagegen eine subjektlose Konstruktion,
die auch kein designiertes Argument hat und demzufolge nicht unter \emph{ist} eingebettet
werden kann.

Interessant ist der Satz \pref{bsp-mod-inf+werden}, der hier als (\mex{1}) wiederholt wird:
\ea[*]{
\label{bsp-mod-inf+werden-zwei}
Dieser Wagen ist von ihnen bis morgen repariert zu werden.\footnote{
        \citew[\page2]{Wilder90a}.
}
}
\z
Dieser Satz ähnelt Fernpassivkonstruktionen, in denen das Objekt eines tiefer eingebetteten
Verbs zum Subjekt des gesamten Verbalkomplexes wird. Durch die oben vorgestellten Lexikonregeln
wird das einzige Argument von \emph{zu werden} unter \subj repräsentiert, wie das für
die Analyse von Sätzen wie (\mex{1}) auch sinnvoll ist:
\ea
Dieser Wagen scheint nicht repariert zu werden.
\z
(\ref{bsp-mod-inf+werden-zwei}) ist ausgeschlossen, weil der \daw von \emph{zu werden} die
leere Liste ist und der Verbalkomplex \emph{repariert zu werden} deshalb nicht unter \emph{ist}
eingebettet werden kann.

%% Das Beispiel in (\mex{1}) kann allerdings nicht auf diese Weise erklärt werden:
%% \ea[*]{
%% daß (von ihnen) geschlafen zu haben ist.
%% }
%% \z
%% Da der \daw von \emph{haben} dem des eingebetteten Verbs gleicht, ist der \daw
%% des Verbalkomplexes \emph{geschlafen zu haben} identisch mit dem von \emph{geschlafen}.
%% Der Komplex \emph{geschlafen zu haben} erfüllt also die Anforderungen von \emph{ist}.
%% Allerdings wurde auf Seite~\pageref{subjekt-raising-kein-passiv} bereits diskutiert,
%% daß Subjektanhebungsverben nicht passivierbar sind, da sie ihrem Subjekt keine semantische
%% Rolle zuweisen, die blockiert werden könnte. Da \emph{haben} ein Anhebungsverb ist,
%% ist die Ungrammatikalität von (\mex{0}) erklärt.

%\subsection{Einbettung von Infinitiven unter Nomen}
\is{Infinitiv!modaler|)}\is{Passiv!modaler Infinitiv|)}
%
%With the feature set-up developed in this section it is possible to account for the
%various perfect and passive forms and the modal infinitives with one lexical item per
%non-finite verb. If we have only one lexical item for \emph{zu}"=infinitives 
%in normal control constructions (for instance embedded under \emph{versuchen} (`try'))
%and in modal constructions, the modal meaning necessarily has to be encoded in
%\haben and \sein. \citet[\page144]{Gunkel99a}\ia{Gunkel} argues that
%this is not correct since the \emph{zu} participle 1 in (\mex{1}) has a modal meaning
%and the phrase does not contain a \sein.
%\ea
%die zu öffnende Tür
%%\z
%\citet[\page144--145]{Gunkel99a}\ia{Gunkel} suggests two lexical entries for \emph{zu}"=infinitives:
%one for normal control constructions and one that appears in modal constructions. The two lexical
%entries have different \vform values. \emph{zu}"=infinitives with \vform{} \type{mod-zu-inf}
%are embedded under the auxiliaries \haben and \sein in modal constructions
%and \emph{zu}"=infinitives with \vform{} \type{zu-inf} are embedded under control or other
%raising verbs. In his approach the two forms have different valence specifications: The representation
%to be used in modal infinitive constructions has a blocked subject and the representation
%for control constructions has no blocked arguments.

%If one follows Gunkel in assuming two lexical items for infinitives with \emph{zu}, the account
%presented in this section has the advantage that the two infinitival forms have the same syntactic
%representation.


\subsection{Das Fernpassiv}
\label{sec-remote-passive-hpsg}
%% \NOTE{Iwanow: Das Kapitel über das Fernpassiv ist recht unüberschaulich und wenig gut erklärt. Es werden zwar Beispiele und die Bedeutung der Lexikoneinträge gegeben, jedoch nimmt man aus dem Text "`nichts mit"'.}


\is{Passiv!Fern-|(}\is{Anhebung|(}
Über den Satz (\ref{bsp-zu-reparieren-versucht-wurde}) -- hier als (\mex{1}) wiederholt --
muß man eigentlich nichts mehr sagen, denn die Analyse folgt aus der in Kapitel~\ref{sec-anhebung-anal}
vorgestellten Analyse der kohärent konstruierenden Verben und der bisher besprochenen
Passivanalyse.
\ea
\label{bsp-zu-reparieren-versucht-wurde-zwei}
weil    der Wagen oft zu reparieren versucht wurde
\z
Die Analyse kann wie folgt skizziert werden: \emph{zu reparieren versuchen} bildet einen
Verbalkomplex, der zwei Argumente verlangt (den Versuchenden bzw.\ Reparierenden und das, was
repariert werden soll). Bei der Bildung der Partizipform \emph{versucht} wird das Subjekt
unterdrückt, so daß auch der Komplex \emph{zu reparieren versucht} ein unterdrücktes Subjekt und
somit nur ein nicht blockiertes Argument hat. Das Passivhilfsverb deblockiert das blockierte
Argument nicht, weshalb bei \emph{zu reparieren versucht wurde} nur ein Argument übrigbleibt,
welches dann im Nominativ realisiert wird. Im folgenden sollen die Details erklärt werden.

Für \emph{versuchen} nehme ich den folgenden Lexikoneintrag an:\footnote{
        Der Lexikoneintrag unterscheidet sich von dem von
        \citet[\page232]{HM94a} dadurch, daß das Subjekt des Matrixverbs
        nicht mit dem Subjekt des eingebetteten Verbs identifiziert ist.
        Wie in Kapitel~\ref{sec-rais-contr-identity-coindexing} gezeigt
        wurde, beschreibt man Kontrollrelationen mit Koindizierung, nicht mit Identifikation
        der gesamten \type{synsem}"=Information.
        Außerdem wird \textsc{da} als Kopfmerkmal repräsentiert, was
        sicherstellt, daß der \daw von \stem{versuch} auch an Projektionen vorhanden ist,
        \dash auch bei \emph{zu reparieren versucht} in \pref{ex-zu-reparieren-versucht}.
        Das ist auch wichtig für die Analyse von Interaktionen zwischen Passiv
        und partieller Voranstellung, wie sie im Abschnitt~\ref{kathol-pollard}
        und~\ref{kasus-anhang} besprochen werden.%
}

\eas
\label{le-versuchen-da}
\stem{versuch}:\\
\ms{
head$|$da & \rule{0cm}{3ex}\sliste{ \ibox{1} }\\[2mm]
subcat & \sliste{ \ibox{1} NP[\str]\ind{2} } $\oplus$ \ibox{3} $\oplus$ \sliste{ \textrm{V[}%
\type{inf}, \textsc{subj} \sliste{ NP[\str]\ind{2}}, \subcat \ibox{3}]}\\[2mm]
}
\zs

\noindent
Die Argumentblockierungsregel in (\ref{lr-da-reduction}) lizenziert den Lexikoneintrag in (\mex{1}):
\eas
\label{le-versucht-hm}
\emph{versucht} (Partizip):\\
\ms{
head$|$da & \rule{0cm}{3ex}\sliste{ NP[\str]\ind{2} }\\[2mm]
subcat & \ibox{3} $\oplus$ \sliste{ V[\type{inf}, \textsc{subj} \sliste{ NP[\str]\ind{2}}, \subcat \ibox{3}] }\\[2mm]
}
\zs

\noindent
Das Ergebnis der Kombination des Partizips in (\mex{0}) mit dem \emph{zu}"=Infinitiv in
(\mex{1}) wird durch (\mex{2}) beschrieben.
\eas
\emph{zu reparieren}:\\
\ms{
head$|$subj & \rule{0cm}{3ex}\sliste{ NP[\str] }\\[2mm]
subcat      & \sliste{ NP[\str] }\\[2mm]
}
\zs

\eas
\label{ex-zu-reparieren-versucht}
\emph{zu reparieren versucht}:\\
\ms{
head$|$da   & \rule{0cm}{3ex}\sliste{ NP[\str]}\\[2mm]
subcat & \sliste{ NP[\str]}\\[2mm]
}
\zs

\noindent
Das Objekt von \emph{zu reparieren} ist in der \subcatl von \emph{zu reparieren versucht} enthalten,
und das Subjekt von \emph{versucht}, das mit dem Subjekt von \emph{zu reparieren} koindiziert ist, ist blockiert.
Da das Passivhilfsverb keine Argumente deblockiert, enthält die \subcatl von \emph{zu reparieren versucht werden}
als einziges Element das Objekt von \emph{zu reparieren}.
Abbildung~\vref{abb-remote-pass-da-hm} zeigt das im Detail.
\begin{figure}
\oneline{%
\begin{forest}
sm edges
[\ms{ head   & \ibox{1} \\
      subcat & \ibox{2}\\
    }
  [\iboxt{4}~\ms{ head & \ibox{3} \\
                  subcat & \ibox{2} \\
                }
     [\iboxt{6}~\onems{ head  \onems[verb]{ vform \type{inf} \\
                                      subj \sliste{ NP[\type{str}]\ind{5} } \\
                                    }  \\
                                subcat~\ibox{2} \sliste{ NP[\str] } \\
                              } [zu reparieren]]
     [\onems{ head~\ibox{3} \onems[verb]{ vform \type{ppp} \\
                                         da \sliste{ NP[\type{str}]\ind{5} } \\
                            } \\
            subcat ~ \ibox{2} $\oplus$ \sliste{ \ibox{6} } \\
          } [versucht]]]
  [\ms{ head & \ibox{1} \ms[verb]{ vform & fin \\
                                   subj & \sliste{} \\
                                 } \\
        subcat & \ibox{2} $\oplus$ \sliste{ \ibox{4} } \\
      } [wurde]]]
\end{forest}
}
\caption{Analyse des Verbalkomplexes \emph{zu reparieren versucht wurde} in:\ \emph{daß der Wagen oft zu reparieren versucht wurde}}\label{abb-remote-pass-da-hm}
\end{figure}
Wegen der Kontrollbeziehung ist das Subjekt von \emph{versucht} mit dem Subjekt von \emph{zu reparieren} koindiziert (\iboxt{2}
in \pref{le-versuchen-da} und \iboxt{5} in Abbildung~\ref{abb-remote-pass-da-hm}).
Da \emph{versucht} das Ergebnis der Anwendung der Lexikonregel zur Blockierung des designierten Arguments
ist, ist das Subjekt von \emph{versucht} blockiert. 
Es ist in der \dalist von \emph{versucht} repräsentiert. Die \subcatl des Komplexes \emph{zu reparieren versucht}
ist wegen der Argumentanhebung (\,\iboxt{3} im Lexikoneintrag \pref{le-versucht-hm} und \iboxt{2}
in Abbildung~\ref{abb-remote-pass-da-hm}) mit der \subcatl von \emph{zu reparieren} identisch.
%Die \subcatl von \emph{zu reparieren versucht} ist also identisch mit der von  identical
%to the \subcat list of \emph{versucht} (\,\iboxt{2} in Figure~\ref{abb-remote-pass-da-hm}). 
Das Passivhilfsverb \emph{werden} deblockiert keine Argumente, es hebt nur die
Elemente von der \subcatl des eingebetteten Verbalkomplexes an (\,\iboxt{2} in
Abbildung~\ref{abb-remote-pass-da-hm}). Außerdem verlangt \emph{wurde}, daß der eingebettete
Verbalkomplex eine referentielle NP in der \dalist hat, was bei \emph{zu reparieren versucht} der Fall ist.
Da \emph{wurde} finit ist, ist das Subjekt von \emph{wurde} nicht unter \subj, sondern
als Element der \subcatl repräsentiert. Die NP, die sich auf das Objekt von
\emph{reparieren} bezieht, ist das erste Element der \subcatl von \emph{zu reparieren versucht wurde}
und bekommt demzufolge Nominativ.


Interessanterweise funktioniert das auch für die Beispiele mit dem Objektkontrollverb \emph{erlauben}:\label{page-remote-passive-erlauben}
\eas
\emph{erlaub-}:\\
\ms{
head$|$da     & \rule{0cm}{3ex}\sliste{ \ibox{1} }\\[2mm]
subcat & \sliste{ \ibox{1} NP[\str], NP[\ldat]\ind{2} } $\oplus$ \ibox{3} $\oplus$ \liste{ V[\begin{tabular}[t]{@{}l@{}}
                                                                                            \type{inf}, \textsc{subj} \sliste{ NP[\str]\ind{2}},\\
                                                                                            \subcat \ibox{3}]\\
                                                                                            \end{tabular}}\\[2mm]
}
\zs

\noindent
Die Argumentblockierungsregel lizenziert den Eintrag in (\mex{1}):
\eas
\emph{erlaubt}:\\
\ms{
head$|$da     & \rule{0cm}{3ex}\sliste{ NP[\str] }\\[2mm]
subcat & \sliste{ NP[\ldat]\ind{2} } $\oplus$ \ibox{3} $\oplus$ \sliste{ V[\type{inf}, \textsc{subj} \sliste{ NP[\str]\ind{2}}, \subcat \ibox{3}] }\\[2mm]
}
\zs

\noindent
Kombiniert man (\mex{0}) mit dem Eintrag für \emph{auszukosten}, bekommt man (\mex{1}).
\eas
\emph{auszukosten erlaubt}:\\
\ms{
head$|$da     & \sliste{ NP[\str] }\\[2mm]
subcat & \sliste{ NP[\ldat], NP[\str] }\\[2mm]
}
\zs

\noindent
Wenn man diesen Verbalkomplex mit \emph{wurde} kombiniert, bleibt das designierte Argument
blo"ckiert, und man bekommt einen Verbalkomplex, der dieselbe \subcatl hat wie \emph{auszukosten erlaubt}.
Da das Objekt von \emph{auszukosten} das erste Element mit strukturellem Kasus in der \subcatl von
{\em auszukosten erlaubt wurde} ist, muß es im Nominativ stehen.
%
Abbildung~\vref{abb-remote-pass-da-hm-erlauben} zeigt die Details.
\begin{figure}
\oneline{%
\begin{forest}
sm edges
[\ms{ head   & \ibox{1} \\
           subcat & \ibox{2}\\
            }
  [\iboxt{4}~\ms{ head & \ibox{3} \\
                  subcat & \ibox{2} \\
                }
     [\iboxt{6}~\onems{ head  \onems[verb]{ vform \type{inf} \\
                                            subj \sliste{ NP[\type{str}]\ind{5} } \\
                                          }  \\
                                    subcat~\ibox{7} \sliste{ NP[\str] } \\
                               } [auszukosten]]
     [\onems{ head~\ibox{3} \onems[verb]{ vform \type{ppp} \\
                                         da \sliste{ NP[\type{str}]\ind{5} } \\
                            } \\
            subcat ~ \ibox{2} (\sliste{ NP[\type{ldat}] } $\oplus$ \ibox{7}\,) $\oplus$ \sliste{ \ibox{6} } \\
          } [erlaubt]]]
  [\ms{ head & \ibox{1} \ms[verb]{ vform & fin \\
                                   subj & \sliste{} \\
                                 } \\
        subcat & \ibox{2} $\oplus$ \sliste{ \ibox{4} } \\
      } [wurde]]]
\end{forest}}
\itdopt{avm für wurde weiter nach links}
\caption{Analyse des Verbalkomplexes in \emph{auszukosten erlaubt wurde} in:\ \emph{daß der Erfolg uns nicht auszukosten erlaubt wurde.}}\label{abb-remote-pass-da-hm-erlauben}
\end{figure}
Die Kontrollbeziehung zwischen dem Dativobjekt von \emph{erlaubt} und dem Subjekt von \emph{auszukosten} 
wird über Koindizierung hergestellt (\,\iboxt{5} in Abbildung~\ref{abb-remote-pass-da-hm-erlauben}). 
Die Komplemente von \emph{auszukosten} \iboxb{7} werden von \emph{erlaubt} angehoben. Die \subcatl
von \emph{auszukosten erlaubt} enthält deshalb ein Dativelement und das Objekt von \emph{auszukosten}.
Die \subcatl von \emph{auszukosten erlaubt} ist identisch mit der \subcatl von \emph{erlaubt} \iboxb{2} abzüglich des
eingebetteten Verbs.
Das Subjekt von \emph{erlaubt} ist blockiert und das Passivhilfsverb deblockiert es nicht.
Somit ist die \subcatl von \emph{auszukosten erlaubt wurde} identisch mit der von \emph{auszukosten erlaubt}. 
Die einzige NP mit strukturellem Kasus auf dieser Liste ist das Objekt von \emph{auszukosten}.
Diese NP muß also im Nominativ realisiert werden.
\is{Passiv!Fern-|)}\is{Anhebung|)}
%
% This approach to the long distance passive differs in an interesting way from the \textsc{acc}"=based approach 
% that was discussed in Section~\ref{sec-remote-passive-hpsg}: The order of elements in the \textsc{subcat} list
% in the structure (\ref{struc-auszukosten-erlaubt-wurde-erg}) is the reverse of the order in (\mex{0}).
% Since the HPSG Binding Theory refers to the order of elements in the \textsc{subcat} list in order
% to account for binding facts, this difference in order 
% should make different predictions as far as binding properties are concerned. 
% I leave this for further studies.


\subsection{\emph{lassen}-Passiv}
\label{lassen-passive-da}


% Bierwisch, Gunkel fassen die Einträge zusammen.
% Der Zusammenfall der LE kann nicht ausschließen, daß unakkusativische Verben passiviert
% werden. Gunkel unterteilt deshalb noch zwischen permissiven und direktiven Verben. Nur die
% direktiven sind dann optional anhebend. Sie verlangen ein agenitves Verb. Das Problem ist, daß es
% auch permissive Passive gibt (Reis).
Die Passiv"=Version\is{Passiv!lassen@\emph{lassen}} des Verbs \emph{lassen} ist völlig parallel
zu den Lexikoneinträgen für das Vorgangspassiv 
%, das Zustandspassiv 
und das modale Passiv. Sieht man von der Form des eingebetteten Verbs ab, ist der
einzige Unterschied, daß \emph{lassen} ein zusätzliches Argument hat \iboxb{1},
das gleichzeitig auch das designierte Argument von \emph{lassen} ist:
\eas
\label{le-lassen-passive-da}
\mbox{\emph{lass-} (Passiv-Version):} \\
\ms{
 head$|$da     & \sliste{ \ibox{1} }\\[2mm]
 subcat & \sliste{ \ibox{1} NP[\type{str}] } $\oplus$ \ibox{2} $\oplus$ \liste{ V[\type{bse}, \textsc{lex}+, \textsc{da}~\sliste{ NP$_{\type{ref}}$}, \textsc{subcat}~\ibox{2}]}\\[2mm]
% \begin{tabular}{@{}l@{}l@{}}
%                   \textrm{V[}&\textrm{\type{bse}, \textsc{lex}+, \textsc{da}~\sliste{ NP$_{\type{ref}}$},}\\
%                           &\textrm{\textsc{subcat}~\ibox{2}]} \\
%                   \end{tabular}}\\
}\iw{lassen!passive|uu}
\zs


%\is{Passiv!lassen@\emph{lassen}|)}

% Scherpenisse86a: Wie Gunkel optionales Objekt

\subsection{Adjektivische Formen}
\label{sec-adj-formation}
\is{Derivation!Adjektiv|(}\is{Passiv!adjektivisches Partizip|(}


Wie wir in Abschnitt~\ref{sec-unaccusativity} gesehen haben, gibt
es für bestimmte Partizipien eine adjektivische Form, die
pränominal verwendet wird. Das erste Beispiel in (\mex{1}) zeigt eine
adjektivische Form eines transitiven Verbs und das zweite
zeigt die partizipiale Verwendung eines unakkusativischen Verbs.
\eal
\ex\iw{reparieren}\label{ex-der-reparierte-wagen}
der reparierte Wagen
\ex\iw{ankommen}\label{ex-angekommene}
der angekommene Zug
\zl
Wenn ein transitives Verb als pränominaler Modifikator gebraucht wird,
werden das direkte Objekt\is{Objekt!direktes} des Verbs und das modifizierte Nomen koindiziert.
Bei unakkusativischen Verben wird dagegen das logische Subjekt mit dem modifizierten Nomen koindiziert.
In beiden Fällen wird das Element, das mit dem modifizierten Nomen koindiziert ist,
nicht als Argument des Partizips realisiert.

Pränominale adjektivische Partizipien sind flektiert\is{Flexion|(}, und
wenn man annimmt, daß Flexion ein lexikalischer Prozeß ist, dann muß die Eingabe
für diesen Prozeß ebenfalls lexikalisch sein (\citealp[\page412]{Dowty78a}\iaright{Dowty};
\citealp[\page 21]{Bresnan82a}\iaright{Bresnan}).
% noch andere Seitenzahlen??
%zitiert Chomsky70a. 
%%
%% Since in \citew[Chapter~7]{Mueller99a}, I assumed inflection to be analyzed with lexical rules, 
%% I suggested deriving the adjectival forms with lexical rules also.
%% The rules that I proposed in \citew[Chapter~15.5]{Mueller99a}\ia{Müller} license adjectives
%% from past participles. In \citew{Mueller99a}, I suggested deriving the inflected pronominal
%% adjectives directly from the participle. This is not appropriate since adjectival participles
%% can appear as secondary predicates in sentences like (\mex{1}):
%% \eal
%% \label{ex-adj-participle}
%% \ex
%% weil er die Äpfel gewaschen ißt.
%% \ex
%% So lange gilt   die 39-Jährige als nicht suspendiert.\footnote{
%%         taz, 31.01.2000, S.\,17.%
%% }
%% \zl
%% Instead of licensing inflected adjectival forms, I suggest lexical rules that
%% license lexical items for adjectival stems. The uninflected predicative form is derived
%% from this lexical item by another lexical rule that also applies to normal adjectives
%% and so is the inflected attributive prenominal form.
%
Die Lexikonregel für die Adjektivbildung zeigt (\mex{1})\vpageref{lr-adjective-formation-da-approach}.

\eas
\label{lr-adjective-formation-da-approach}
Adjektivableitungsregel für Partizipien:\\
%
%
\onems[word]{
  synsem$|$loc$|$cat \ms{ head   & \onems[verb]{ vform \type{ppp} \\ 
                                               } \\
                          subcat & \sliste{ \ibox{1} NP[\type{str}]$_{ref}$ } $\oplus$ \ibox{2} \\ 
                        }\\
} $\mapsto$\\
\onems[stem]{
 synsem$|$loc$|$cat \ms{ head & \ms[adj]{ subj  & \sliste{ \ibox{1} } \\
                                               }\\
                         subcat & \ibox{2}\\
                       }\\
}\is{Lexikonregel!Adjektivderivation}
\zs

\noindent
Diese Regel nimmt ein Partizip als Eingabe, das eine referentielle NP mit strukturellem
Kasus als erstes Element der \subcatl hat. Da unakkusativische Verben kein designiertes
Argument haben, gibt es kein blockiertes Element. Deshalb ist das erste
Element der \subcatl unakkusativischer Verben das Subjekt (das Element, von dem
man sagt, es habe Objekteigenschaften). Bei transitiven Verben ist das
erste Element der \subcatl mit strukturellem Kasus das direkte Objekt, da das Subjekt
blockiert ist. Dieses Element wird zum Subjekt des adjektivischen Partizips.
Subjekte von Infinitiven mit und ohne \emph{zu} und von Adjektiven werden
einheitlich als Elemente von \subj repräsentiert.\isfeat{subj} 

Lexikonregeln für die Flexion lizenzieren Lexikoneinträge für Adjektive und adjektivische
Partizipien, die pränominal benutzt werden können. Wie in Kapitel~\ref{chap-adjunkte}
dargelegt, selegieren Adjunkte den Kopf, den sie modifizieren, über das
\textsc{modified}"=Merkmal. Die Lexikonregel in (\mex{1}) bildet die Stammeinträge
normaler Adjektive und der von (\mex{0}) lizenzierten adjektivischen Partizipien
auf flektierte Formen ab, die pränominal benutzt werden können.
%
%\begin{figure}[htbp]
\eas
\label{lr-prenom-adj}
\begin{tabular}[t]{@{}l@{}}
Lexikonregel für pränominale Adjektive:\\
\onems[stem]{
 synsem$|$loc \ms{ cat & \ms{ head & \ms[adj]{ subj  & \sliste{ NP\ind{1} } \\
                                        }\\
                       }\\
                   cont & \ibox{2}\\
                 }\\
} $\mapsto$\\
\onems[word]{ synsem$|$loc \ms{ cat & \ms{ head & \ms{
%                       prd & $-$ \\
                        mod &  \textrm{$\overline{\mbox{\textrm{N}}}$:} \ms{ ind   & \ibox{1} \\
                                                                     restr & \ibox{3} \\
                                                                    } \\
                      } \\
             } \\
   cont & \ms{ ind   & \ibox{1} \\
               restr & \sliste{ \ibox{2} } $\oplus$ \ibox{3}  \\
             }\\
}\\}\is{Lexikonregel!pr"anominales Adjektiv}
\end{tabular}
\zs
%\vspace{-\baselineskip}\end{figure}
%
%\noindent
Das Subjekt des Adjektivs ist mit dem Nomen, das durch das Adjektiv modifiziert wird, koindiziert \iboxb{1}.
Der semantische Beitrag des Adjektivs \iboxb{2} wird mit dem Beitrag des Nomens \iboxb{3} verknüpft. 
Der semantische Beitrag wird in Kopf"=Adjunkt"=Strukturen von der Adjunktstruktur projiziert.
Im konkreten Fall ist das Adjunkt ein Adjektiv bzw.\ eine Adjektivphrase, und der
semantische Beitrag des Wortes oder der Phrase besteht aus einem Index, der durch den Beitrag
des Adjektivs \iboxb{2} und einer Liste von Restriktionen, die vom Nomen beigesteuert werden \iboxb{3}, restringiert wird.

Die Regel in (\mex{0}) enthält weder für das Eingabezeichen noch für das Ausgabezeichen eine Spezifikation der Phonologie.
Kongruenzinformation ist ebenfalls nicht angegeben. Die Details der Flexion werden in Kapitel~\ref{sec-morph-flex-anal}
besprochen\is{Flexion|)}.

Im folgenden wird die Regel anhand zweier Beispiele diskutiert. Zuerst betrachten wir
das transitive Verb \emph{reparieren} und dann das unakkusativische Verb \emph{ankommen}.
Die Partizipform des Verbs \emph{reparieren} zeigt (\mex{1}). Dieses Wort ist
das Ergebnis der Anwendung der Regel in (\ref{lr-da-reduction}) auf Seite~\pageref{lr-da-reduction}.

\eas
\emph{repariert} (Partizip):\\
\ms{
cat & \ms{ head    & \ms[verb]{
                     da & \rule{0cm}{3ex}\sliste{NP[\str]\ind{1}}\\
                     }\\
           subcat  & \sliste{NP[\str]\ind{2}}\\[2mm]
         }\\
cont & \ms[reparieren]{
        agens & \ibox{1}\\
        thema & \ibox{2}\\
        }\\ 
}
\zs

\noindent
Die Adjektivbildungsregel in (\ref{lr-adjective-formation-da-approach}) lizenziert
den folgenden Stamm:
\eas
\stem{repariert} (adjektivischer Stamm):\\
\ms{
cat & \ms{ head    & \ms[adj]{
                     subj & \rule{0cm}{3ex}\sliste{NP[\str]\ind{2}}\\
                     }\\
           subcat  & \sliste{}\\
         }\\
cont & \ms[reparieren]{
        agens & \etag\\
        thema & \ibox{2}\\
        }\\ 
}
\zs

\noindent
\etag steht hierbei für einen beliebigen Wert. Das Agens von \emph{repariert} ist nicht an ein
Argument des Adjektivs gelinkt.

Die pränominale adjektivische Form (\mex{1}) ist durch die Lexikonregel in (\ref{lr-prenom-adj})
lizenziert.
\eas
{\em reparierte} (attributives adjektivisches Partizip):\\
\ms{
cat & \ms{ head    & \ms[adj]{
%                     subj & \sliste{NP[\str]\ind{1}}\\
                     mod &  \textrm{$\overline{\mbox{\textrm{N}}}$:} \ms{ ind   & \ibox{1} \\
                                                                  restr & \ibox{2} \\
                                                                } \\
                     }\\
           subcat  & \sliste{}\\
         }\\
cont & \ms{ ind   & \ibox{1} \\
            restr & \liste{ \ms[reparieren]{
                                agens & \etag\\
                                thema & \ibox{1}\\
                             }} $\oplus$ \ibox{2}  \\
              } \\
}
\zs

\noindent
Da das Subjekt des adjektivischen Partizips das Objekt des Verbs ist und
da das Subjekt des adjektivischen Partizips wegen der Regel (\ref{lr-prenom-adj})
mit dem modifizierten Nomen koindiziert ist, ist erklärt, warum das Nomen \emph{Wagen} 
in (\ref{ex-der-reparierte-wagen}) die Thema"=Rolle von \emph{reparierte} füllt.

Für das Verb \emph{ankommen} lizenziert die Partizipregel in (\ref{lr-da-reduction}) einen Eintrag
mit folgendem \localw:

\eas
\emph{angekommen} (Partizip II):\\
\ms{
cat & \ms{ head    & \ms[verb]{
                     da & \sliste{}\\
                     }\\
           subcat  & \sliste{NP[\str]\ind{1}}\\[2mm]
         }\\
cont & \ms[ankommen]{
        thema & \ibox{1}\\
        }\\ 
}
\zs

\noindent
Da \emph{ankommen} kein designiertes Argument hat, wird nichts blockiert und somit ist das Subjekt des
Verbs in der \subcatl des Partizips II repräsentiert.

Da das erste Element der \subcatl von \emph{angekommen} eine NP mit strukturellem Kasus ist,
kann die Adjektivbildungsregel in (\ref{lr-adjective-formation-da-approach}) angewendet werden
und lizenziert also folgenden Stamm:
\eas
\stem{angekommen} (Adjektivstamm):\\
\ms{
cat & \ms{ head    & \ms[adj]{
                     subj & \rule{0cm}{3ex}\sliste{NP[\str]\ind{1}}\\
                     }\\
           subcat  & \sliste{}\\
         }\\
cont & \ms[ankommen]{
        thema & \ibox{1}\\
        }\\ 
}
\zs

\noindent
Dieser Stamm ist die Eingabe für die Lexikonregel (\ref{lr-prenom-adj}),
die die pränominale adjektivische Form in (\mex{1}) lizenziert:
\eas
{\em angekommene} (attributive Form):\\
\ms{
cat & \ms{ head    & \ms[adj]{
%                     subj & \sliste{NP[\str]\ind{1}}\\
                     mod &  \textrm{$\overline{\mbox{\textrm{N}}}$:} \ms{ ind   & \ibox{1} \\
                                                                  restr & \ibox{2} \\
                                                                } \\
                     }\\
           subcat  & \sliste{}\\
         }\\
cont & \ms{ ind   & \ibox{1} \\
            restr & \liste{ \ms[ankommen]{
                                thema & \ibox{1}\\
                             }} $\oplus$ \ibox{2}  \\
              } \\
}
\zs

\noindent
Dieselbe Erklärung funktioniert auch für bivalente unakkusativische Verben:
\ea[]{\iw{zustoßen}
die ihnen zugestoßenen Ereignisse\footnote{
        Die Zeit, 11.04.1986, S.\,57.%
}
}
\z
Das Subjekt von \emph{zustoßen} ist nicht blockiert.
Es ist das erste Element der \subcatl des Partizips II.
Deshalb wird es im durch (\ref{lr-adjective-formation-da-approach}) lizenzierten
Lexikoneintrag für das Adjektiv unter \subj repräsentiert.
Das Dativargument von \emph{zugestoßene} wird innerhalb der pränominalen AP realisiert.
Diese AP ist ein Beispiel für eine Instantiierung der Lexikonregel in (\ref{lr-adjective-formation-da-approach}),
in der \iboxt{2} eine Liste ist, die ein Element enthält.

Nachdem ich gezeigt habe, wie adjektivische Partizipien transitiver
und unakkusativischer Verben lizenziert sind, bleibt noch zu erklären,
weshalb Phrasen wie (\mex{1}) ausgeschlossen sind:
Da die Eingabe für die Lexikonregel in (\ref{lr-adjective-formation-da-approach})
verlangt, daß die \subcatl des Partizips eine NP mit strukturellem Kasus
enthält, kann die Regel nicht auf subjektlose Verben oder auf Verben angewendet
werden, die nicht unakkusativisch sind und kein Akkusativobjekt verlangen.
\eal
\ex[*]{
der (vor der Prüfung) gegraute Student
}
\ex[*]{\iw{tanzen}
der (eben erst) getanzte Mann
}
\ex[*]{\iw{helfen}
der (ihm) geholfene Mann
}
\zl
Da der Lexikoneintrag des Partizips II eines unergativischen intransitiven Verbs
wie \emph{tanzen} eine leere \subcatl hat, kann die Regel in (\ref{lr-adjective-formation-da-approach}) nicht angewendet
werden. Die Unakzeptabilität von (\mex{0}b), die der Wohlgeformtheit von (\ref{ex-angekommene}) mit dem unakkusativischen
intransitiven Verb \emph{ankommen} gegenübersteht, ist also erklärt.
Genauso kann kein adjektivisches Partizip \noword{geholfene} gebildet werden, da das Partizip \emph{geholfen}
eine \subcatl hat, die mit einem Dativ beginnt, \dash mit einem lexikalischen Kasus.



Im Kapitel~\ref{sec-rais-contr-identity-coindexing} wurde Höhles Test \citeyearpar[Kapitel~6]{Hoehle83a}\iaright{Höhle}
für die Bestimmung des Kasus nicht ausgedrückter Subjekte vorgestellt. Höhle hat diesen Test auf
Infintive angewendet, aber man kann natürlich parallele Beispiele mit adjektivischen Partizipien konstruieren:
\eal
\label{ex-subj-case-adjectival-part}
\ex
die [eines         nach  dem           anderen]$_i$ eingeschlafenen Kinder$_i$
\ex
die [einer         nach  dem           anderen]$_i$ durchgestarteten Halbstarken$_i$
\zl
In (\mex{0}a) hat {\em ein- nach d- ander-}\iw{ein- nach d- ander-} keinen eindeutigen Kasus. Die
Kasusform ist $nom \vee acc$. (\mex{0}b) legt jedoch nahe, daß das Subjekt des adjektivischen
Partizips im Nominativ steht. Man beachte, daß die NPen in (\mex{0}) als Subjekt oder Objekt in
einem übergeordneten Satz fungieren können, da der Kasus des modifizierten Nomens unabhängig vom
Kasus des Subjekts des adjektivischen Partizips ist. Die Lexikonregel in (\ref{lr-prenom-adj})
erfaßt das richtig. Sie stellt eine Koindizierung zwischen dem modifizierten Nomen und
dem Subjekt des Partizips her. Die \synsemwe des modifizierten Nomens und des Subjekts des Partizips
sind jedoch nicht identisch. Die Beziehung zwischen diesen beiden NPen ist eine Kontrollbeziehung,
keine Anhebungsbeziehung.
% auch in Wilder90a:8, aber ohne große Argumentation
Es ist deshalb nicht zulässig, die modifizierte NP als Subjekt des Partizips (bzw.\ in der
GB"=Terminologie als externes Argument des Partizips) zu bezeichnen, wie das \zb
\citet[\page646]{LR86a}\iawrong{Levin}\iawrong{Rappaport} und Jacobs
(\citeyear[\page 9]{Jacobs91a}, \citeyear[\page 98]{Jacobs92a-u}) tun. Jacobs, der eine Theorie im Rahmen der
Kategorialgrammatik\is{Kategorialgrammatik (CG)} entwickelt, geht davon aus, daß das modifizierte
Nomen ein Argument des adjektivischen Partizips ist. Er beschränkt den Sättigungsgrad von Elementen
in den Valenzlisten nicht, so daß auch die (ungesättigte) \nbar, die modifiziert wird, gleichzeitig
ein Argument des Partizips sein kann. Er nimmt an, daß das Subjekt einen Kasuswert bekommt, der zur 
Flexion des Adjektivs und somit zum Kasus des Bezugsnomens paßt \citep[\page 9]{Jacobs91a}. % in 92a nicht mehr enthalten
Das bedeutet, daß Subjekte von Partizipien alle vier Kasus haben können. Insbesondere wird auch angenommen, daß es
Dativsubjekte gibt, eine Möglichkeit, die von der hier entwickelten Theorie ausgeschlossen ist, da
NP"=Subjekte strukturellen Kasus haben und der Dativ lexikalisch ist. Das paßt zu der Feststellung,
daß es im Deutschen keine Dativsubjekte gibt (Siehe auch die Diskussion von
(\ref{ex-der-student-versucht-geholfen-zu-werden}) auf
Seite~\pageref{ex-der-student-versucht-geholfen-zu-werden}.).
\is{Derivation!Adjektiv|)}\is{Passiv!adjektivisches Partizip|)}%
\is{Argumentblockierung|)}


\NOTE{Haider sagt, daß wegen der folgenden Daten nur das \emph{zu} blockiert. Ich brauche noch etwas zu diesen Daten
\eal
\ex[]{
der dieses Problem lösende Linguist
}
\ex[*]{
der dieses Problem zu lösende Linguist
}
\ex[*]{
das (von ihm) lösende Problem
}
\ex[]{
das (von ihm) zu lösende Problem
}
\zl
Zwei verschiedene \suffix{d}. Eins für \emph{bse}"= und eins für \emph{zu}"=Infinitive.
}


\subsection{Agensausdrücke}
\label{sec-analyse-agensausdruecke}

Wie Höhle festgestellt hat, können Agensausdrücke von den verschiedensten PPen
ausgedrückt werden. PPen wie \emph{zwischen den Sanitätern} oder \emph{auf dem Meßgerät}
unterscheiden sich jedoch von Präpositionalphrasen mit \emph{von} bzw.\ \emph{durch} darin,
daß die Präposition in ersteren einen semantischen Beitrag leistet. Für die Präpositionen,
die normalerweise zur Realisierung des Agens benutzt werden, braucht man spezielle
Lexikoneinträge, die eben keine Relation beisteuern. Es wäre schön, wenn man eine Analyse hätte,
die die Identifikation der unterdrückten Agens"=Rolle mit dem referentiellen Index der NP in
der \vonpp nicht einer Schlußkomponente überläßt, die Weltwissen zur Bestimmung des Agens benutzt,
sondern die Verbindung direkt herstellt.
Der folgende Lexikoneintrag für die Präposition \emph{von} leistet dies:
Die \vonpp modifiziert einen verbalen Kopf (ein Verb oder ein adjektivisches Partizip) mit
einem Element in der \dalist. Der Index"=Wert dieses Elements wird mit dem referentiellen
Index des Dativarguments der Präposition identifiziert \iboxb{1}.

\eas
Präposition \emph{von} für Agensausdrücke:\\
\ms{ head & \ms[prep]{ mod$|$loc$|$cat$|$head$|$% & \ms{ % braucht man nicht, wegen DA verbal & +\\
da & \sliste{ [ \textsc{loc$|$cont$|$ind} \ibox{1} ] } \\
%                                                         }\\
                     }\\
     subcat & \sliste{ NP[\ldat]\ind{1} }\\
}
\zs

\noindent
Dieser Lexikoneintrag für die Präposition interagiert auch mit der Analyse der adjektivischen Partizipien.
In der Analyse von (\mex{1}) wird das designierte Argument von \stem{ablehn} blockiert. Das blockierte
Element ist als \subjw und als Element in der \dalist des Partizips repräsentiert. Bei der Adjektivbildungsregel
wird das erste Element der \subcatl von \emph{abgelehnt} (das Objekt von \stem{ablehn}) zum Subjekt
des Adjektivs. Der \daw wird vom Partizip unverändert übernommen und bleibt auch bei der Anwendung
der Lexikonregel, die pränominale Adjektive lizenziert, erhalten.
\ea
der von Verdi abgelehnte "`Solidarpakt"'\footnote{
  taz bremen, 11.04.2005, S.\,21.%
}
\z
Deshalb ist die Information über das designierte Argument im Eintrag für \emph{abgelehnte} vorhanden,
und die PP \emph{von Verdi} kann sich auf das designierte Argument von \stem{ablehn} beziehen.

Ein Problem, das diese Analyse hat, ist folgendes: Die Identifikation der semantischen Rolle
ist nicht an die Abbindung einer Valenzstelle gekoppelt, sondern erfolgt durch Identifikation
des Indexes mit einem Index innerhalb eines Adjunkts. Wenn keine besonderen Maßnahmen getroffen
werden, könnte das Agensargument auch syntaktisch in Kopf"=Argument"=Strukturen realisiert werden.
Die Perfektkonstruktion in (\mex{1}) ist ein Beispiel: 

\ea[*]{
Anna hat von Anna Peter geküßt.
}
\z
Das Partizip ist mit der \vonpp kombiniert worden, und außerdem wird das \subj des Partizips
deblockiert, so daß das Agens \emph{Anna} noch als Nominativ"=NP realisiert werden kann.
Man könnte argumentieren, daß der Satz in (\mex{0}) durch die Bindungstheorie\is{Bindungstheorie}\footnote{
  Die Bindungstheorie sagt etwas darüber aus, wann referentielle Nominalphrasen in einem Satz
  koreferent sein können und wann nicht. Insbesondere werden Gesetzmäßigkeiten für die
  Bindung von Reflexivpronomina formuliert. Sieht man von speziellen Äußerungskontexten ab,
  können in (i.a) die beiden Karls nicht identisch sein.
  Will man etwas Entsprechendes ausdrücken, so muß wie in (i.b) ein Reflexivpronomen verwendet werden.
\eal
\ex Karl kennt Karl.
\ex Karl kennt sich.
\zl
  Zur Bindungstheorie im Rahmen der HPSG siehe \citew[Kapitel~6]{ps2} und \citew[Kapitel~20]{Mueller99a}.%
} ausgeschlossen wird, da \emph{Anna} in \emph{von Anna} nicht mit dem Subjekt \emph{Anna} koreferent sein darf.
Aber man kann in (\mex{0}) statt \emph{von Anna} auch \emph{von sich} schreiben, 
die Sätze wären genauso ungrammatisch. In \citew{Mueller2003e} habe ich eine Analyse entwickelt,
die doppelte Realisierung des Agens nach dem Muster von (\mex{0}) ausschließt. Die Analyse
verwendet ein binäres Merkmal, das markiert, wenn ein Argument durch eine \vonpp realisiert
wird. Solcherart realisierte Argumente dürfen dann nicht als normale Subjekte in Perfektkonstruktionen
realisiert werden. Die vorgeschlagene Analyse kann zwar (\mex{0}) ausschließen, versagt aber
leider bei (\mex{1}):
\eal
\ex[*]{
Der Mann wird von Anna von Anna geliebt.
}
\ex[*]{
Der Mann wird von Anna von sich/ihr geliebt.
}
\zl
Die \vonpp ist ein Adjunkt. Es gibt keine Beschränkung für die Anzahl von Adjunkten, da
diese normalerweise keine semantischen Rollen füllen und iterierbar sind. Die \vonpp kann etwas über die Köpfe
aussagen, mit denen sie kombiniert werden kann. Sie kann aber nichts darüber aussagen, daß
es nur eine solche PP geben kann, denn die Information, die die PP unter \textsc{mod} enthält, wird mit dem
\synsemw des Kopfes identifiziert, und wenn das wiederholt wird (beim Vorhandensein mehrerer PPs wie
in (\mex{0})), ändert sich das Ergebnis der Identifikation nicht.

Es bleibt wohl nur anzunehmen, daß Sätze wie (\mex{0}) aufgrund von Ökonomiebeschränkungen
ausgeschlossen sind, die besagen, daß Argumente nicht mehrfach ausgedrückt werden dürfen.
Solche Beschränkungen würden dann auch (\mex{-1}) ausschließen. Daß man solche Beschränkungen
braucht, haben auch Höhles Sanitäter"=Sätze in (\ref{ex-sanitaeter}) auf Seite~\pageref{ex-sanitaeter} gezeigt.



%% \citet[\page18--19]{Wilder90a} diskutiert die folgenden Daten:
%% \eal
%% \ex[]{
%% Die Hoffnung, gesehen zu werden
%% }
%% \ex[*]{
%% die Hoffnung, von ihm zu schlafen
%% }
%% \zl
%% Das Beispiel (\mex{0}a) zeigt, daß die Einbettung von passivierten Infinitiven möglich ist. Die
%% Einbettung des \emph{zu}"=Infinitivs mit einer passivtypischen \emph{von}"=PP ist dagegen
%% ausgeschlossen.
%% Die Ungrammatikalität von (\mex{0}) ist dadurch zu erklären, daß \emph{Hoffnung} eine
%% Verbalphrase mit einem Argument in \subj verlangt. Diese Rolle wird durch das Nomen gebunden,
%% weshalb es durch die \emph{von}"=Phrase nicht noch einmal ausgedrückt werden kann.



\section{Alternativen}

Im Abschnitt~\ref{cxg-hpsg-linking-konstruktionen} wurden bereits vererbungsbasierte Ansätze zur
Analyse des Passivs diskutiert. In den folgenden Abschnitten sollen HPSG"=Ansätze besprochen werden,
die entweder gar keine oder andere Merkmale benutzen, um Unterschiede zwischen unakkusativischen und
unergativischen bzw.\ transitiven Verben zu modellieren. Abschnitt~\ref{sec-agensausdruecke} setzt
sich mit alternativen Behandlungen der Agensausdrücke auseinander.

\subsection{Theorien ohne zusätzliche Merkmale}

% \citet[\page139]{Gunkel99a}\iaright{Gunkel} defines the external argument as the one
% argument with sufficient proto"=agent properties. This excludes expletives as external
% arguments. The referentiality of the logical subject in passive constructions is therefore
% required by the passive auxiliary rather implicitly. In Gunkel's approach being an
% external argument is a lexical"=semantic property.
% If expletives are allowed as external arguments\footnote{
%         If the choice of the auxiliary is made dependent upon there being a designated
%         argument or not, expletives must be allowed as external arguments since \word{regnen} (`to rain')
%         takes \haben. See \citew[Chapter~6.6.2]{HM94a}\ia{Heinz}\ia{Matiasek} for
%         auxiliary selection that is dependent on the external argument.
% }, one has to make the requirement that the logical subject is referential explicit.


%\ifthenelse{\boolean{draft}}{

%% Das ist doch ein Problem, da das Reflexivum bei Doppeltpassivierung nicht realisiert werden muß.
%% Das ist kein Problem, da dann das Objekt ein Reflexivum sein müßte.
%% Das geht aber nicht, da das Reflexivum nicht Nominativ bekommen kann.
%% Höchstens wenn man das Passiv unter ein AcI einbettet:
%%
%% Das scheint aber nicht zu gehen:
%% * Er ließ ihn geschlagen werden.

Es wäre wünschenswert, die Passivierbarkeit von Verben allein aus semantischen Eigenschaften abzuleiten
und ohne ein Merkmal wie \textsc{da} auszukommen. Leider scheint das nicht möglich zu sein, 
da man -- wenn man sich nur auf die Bedeutung des eingebetteten
Verbs bezieht -- mit Daten wie (\ref{ex-double-application-of-passive}) -- hier als (\mex{1}) wiederholt --
ein Problem bekommt: Normalerweise wird davon ausgegangen, daß das Passiv bedeutungserhaltend ist, 
\dash, das Passivhilfsverb führt selbst keine Relation ein, die zur Bedeutung eines Satzes beiträgt. 
\eal
\label{ex-double-application-of-passive-zwei}
\ex[]{
weil er den Film liebt
}
\ex[]{
weil der Film geliebt wurde
}
\ex[*]{
weil geliebt worden wurde
}
\zl
Nimmt man an, daß der Bedeutungsbeitrag von \emph{geliebt wurde} mit dem von \emph{geliebt} identisch ist,
gibt es nichts, was (\mex{0}c) ausschließen könnte. \citet[\page100]{Gunkel2003b} behauptet zwar, 
daß man im Lexikoneintrag verlangen kann, daß ein bestimmtes Argument an das Agens gebunden ist, aber
das ist nicht richtig. Man kann sich das verdeutlichen, indem man folgenden hypothetischen
Eintrag für das Passivhilfsverb annimmt:
\eas
\label{le-passiv-gunkel}
Hypothetischer Eintrag für das Passivhilfsverb:\\
\ms{
cat$|$subcat \ibox{1} $\oplus$ \sliste{ \ms{ loc \ms{ cat  & \ms{ head & \ms[verb]{ vform & ppp\\
                                                                                 }\\
                                                                 subcat & \sliste{ NP\ind{2} } $\oplus$ \ibox{1}\\
                                                               }\\
                                                     cont & \ms[agens-rel]{
                                                            agens & \ibox{2}\\
                                                            }\\
                                                   }\\
                                          } }\\
}
\zs
Würde Gunkels Vorschlag funktionieren, wäre (\mex{-1}c) dadurch ausgeschlossen,
daß die NP in der Valenzliste von \emph{gelesen worden} an das Thema von \emph{lesen}
gelinkt ist. Problematisch sind aber Verben, die Argumente mit kompatiblen
Selektionsrestriktionen haben. Ein Beispiel dafür ist das Verb \emph{wählen}.
Im Satz (\mex{1}) sind das Subjekt von \emph{wählen} und das Objekt von \emph{wählen}
koindiziert, da sie auf dasselbe Objekt verweisen.
\ea
Peter wird nur von sich selbst gewählt.
\z
Somit entspricht der Komplex \emph{gewählt wird} der Struktur in (\mex{1}):
\eas
\emph{gewählt wird} bei reflexivem Agensausdruck:\\
\ms{ cat  & \ms{ head & verb\\
                 subcat & \sliste{ NP\ind{2} }\\
               }\\
     cont & \ms[wählen]{
             agens & \ibox{2}\\
             thema & \ibox{2}\\
            }\\
   }
\zs

\noindent
Das Subjekt von \emph{wählen} ist unterdrückt, und das Objekt befindet sich an der ersten
Position in der \subcatl. Wegen der Koindizierung der beiden Argumentrollen ist die
Agensrolle identisch mit dem Index der Nominalphrase in der \subcatl und die Beschränkung
in (\ref{le-passiv-gunkel}) somit erfüllt. Eine Mehrfachpassivierung kann also auf diese Weise
nicht ausgeschlossen werden.

Ein anderes Problem stellen komplexe Verben wie \emph{anlachen} dar (Die Details
der Analyse von Partikelverben werden im Kapitel~\ref{sec-lr-for-transp-pvs} besprochen.):
\eal
\ex Sie lacht ihn an.
\ex weil er angelacht wird
\zl
Das Partikelverb \emph{anlachen} ist nach einem produktiven Muster gebildet. Nach
\citew[\page956]{SW94a} bedeutet \emph{anlachen} \relation{und}(\relation{lachen}(X), \relation{gerichtet-auf}(\relation{lachen}(X),Y)).
Die oberste Relation in dieser komplexen semantischen Repräsentation ist \relation{und}.
Da \relation{und} keine Relation vom Typ \type{agens-rel} ist, können die Anforderungen
des Passivhilfsverbs in (\ref{le-passiv-gunkel}) nicht erfüllt werden, und der
Satz (\mex{0}b) wäre nicht analysierbar. Es bleibt also nur, zusätzliche Merkmale einzuführen.
Man könnte ein semantisches Merkmal einführen, das
auf den Beitrag von \emph{lachen} verweist, und die Restriktionen des Passivhilfsverbs
könnten dann auf dieses Hilfsmerkmal Bezug nehmen. Die Alternative ist, ein Argument
besonders zu kennzeichnen, und von dieser Möglichkeit wurde im vorangegangenen Abschnitt
Gebrauch gemacht.

%% Diese Daten sind problematisch, da bei der Koordination auch die Subjekte unifiziert werden.
%
%
%% Sieht man von diesen Fällen mit Reflexivpronomen ab hat Gunkels Vorschlag auch Probleme
%% mit Koordinationen wie (\mex{1}):
%% \ea
%% Peter wurde vorgeschlagen und gewählt.
%% \z
%% Nimmt man an, daß \emph{vorgeschlagen und gewählt} als Konstituente unter \emph{wurde} eingebettet
%% wird, so kann man die Restriktion in bezug auf das Agens nicht ohne weiteres formulieren,
%% denn der semantische Beitrag von \emph{vorgeschlagen und gewählt} ist und(vorschlagen(X,Y),wählen(Z,Y),
%% somit ist die oberste Relation \emph{und}, also kein Untertyp von \type{agens-rel}.



%% \subsection{Theorien ohne \subjm}


%% In the previous subsections, I have demonstrated how \hm's approach works for the agentive passive, the
%% stative passive, and the dative passive. However, the extension of their approach
%% to modal infinitives turned out to be problematic:
%% \hm assume that the subject of infinitives with \emph{zu} are represented on the \subcat list.
%% This has two disadvantages: First, incoherently constructing verbs select for a verbal projection
%% that is not fully saturated. This makes impossible a uniform characterization of maximal projections\is{maximal projection}
%% as something with an empty \subcat list. Furthermore, such a representation is incompatible with
%% Haider's approach to modal infinitives: According to Haider, the subject (external argument) of
%% infinitives is blocked. It is deblocked by \haben in the modal infinitive construction and
%% it remains blocked if the \emph{zu}"=infinitive is combined with \sein.

%% In what follows, I will extend \hm's analysis to modal infinitives and modify their analysis of
%% the passive so that Haider's ideas are formalized properly.


%% Die können ja Listen nehmen und dann das erste designierte Argument

%% \subsection{Zeiger oder binäres Merkmal zur Auszeichnung des designierten Arguments}

%% Im Kapitel~\ref{cxg-linking-konstruktionen} wurde bereits der Passivansatz von
%% Fillmore\aimention{Charles J. Fillmore}, Kay\aimention{Paul Kay}, 
%% Michaelis\aimention{Laura A. Michaelis} und Ruppenhofer\aimention{Josef Ruppenhofer} besprochen.
%% Die genannten Autoren verwenden ein binäres Merkmal \textsc{da}, das Bestandteil der
%% Beschreibung eines Elements in einer Valenzliste ist. Das heißt, der 
%% \textsc{da}"=Wert des designierten Arguments ist +, wohingegen alle anderen Elemente den \daw
%% $-$ haben. Der hier vorgestellte Ansatz verwendet dagegen ein eigenes listenwertiges Merkmal
%% zur Kennzeichnung des designierten Arguments: Wenn es ein designiertes Argument gibt,
%% dann wird eine Strukturteilung zwischen dem entsprechenden Element in der \subcatl und
%% dem Element in der \textsc{da}"=Liste hergestellt. Der Vorteil dieser Analyse ist, daß das
%% Fernpassiv mit Bezug auf die Komplexbildung und Argumentanziehung erklärt werden kann.
%% Das ist bei dem Ansatz, der ein binäres Merkmal zur Kennzeichnung verwendet, nicht möglich:
%% Das Verb \emph{reparieren} hat ein designiertes Argument, das im Passiv unterdrückt bzw.\
%% als \emph{von}"=PP realisiert werden kann:
%% \eal
%% \ex Er repariert den Wagen.
%% \ex Der Wagen wurde von ihm repariert.
%% \zl
%% Das ist jedoch für das Fernpassiv nicht relevant


In diesem Kapitel bin ich Vorschlägen von Hubert Haider zur Kennzeichnung eines Arguments 
mit Subjekteigenschaften gefolgt. In der HPSG"=Literatur gibt es Varianten dieses Ansatzes,
aber auch andere Analysen, die nicht das Element mit Subjekteigenschaften, sondern
das Element mit Akkusativ"=Eigenschaften markieren. Auch unterscheiden sich die Vorschläge
darin, welche Information zwischen Hilfsverb und eingebettetem Verb geteilt wird.
Die beiden folgenden Abschnitte beschäftigen sich mit alternativen Objekt"=zu"=Subjekt"=Anhebungsansätzen,
und Abschnitt~\ref{sec-kathol-passive-raising} und~\ref{sec-ryu-passive} diskutieren
Analysen, die auf Koindizierung basieren.



\subsection{Kathol: 1994}
\label{sec-kathol-passive}

\mbox{}\citet[Kapitel~7.3.3]{Kathol94a} schlägt für Partizipien die Repräsentationen in\NOTE{Timm
  Lichte:  - 17.3.2 und 17.3.4 haben denselben Titel, nämlich "Kathol:1994". Werden da zwei unterschiedliche Ansätze vorgestellt, die so eine Trennung rechtfertigen? In jedem Fall ist das eine Erklärung wert.
}
(\mex{1}) und für die Hilfsverben die in (\mex{2}) vor.\footnote{
        Kathol benutzt statt \subcat das Merkmal \comps. Ich habe seine Einträge an die Merkmalsgeometrie,
        die im vorliegenden Buch verwendet wird, angepaßt.%
}

\ea
\begin{tabular}[t]{@{}l@{ }l@{ }l@{ }l@{ }l@{}}
  &                               & \textsc{ext}                           & \textsc{subj}                  & \textsc{subcat}\\[2mm]
a.&angekommen  (unakkusativisch): & \sliste{ \ibox{1} NP[\type{nom}] } & \sliste{ \ibox{1} }         & \sliste{}    \\[2mm]
b.&geschlafen  (unergativisch):   & \sliste{          NP[\type{nom}] } & \sliste{}                   & \sliste{}    \\[2mm]
c.&geliebt     (transitiv):       & \sliste{          NP[\type{nom}] } & \sliste{ NP[\type{acc}] }  & \sliste{}    \\[2mm]
\end{tabular}
\z
\eal
\ex \haben (Perfekthilfsverb)\\
\ms{
subj   & \ibox{3}\\
subcat & \ibox{2} $\oplus$ \ibox{1} $\oplus$ \sliste{ V[\textsc{subj} \ibox{2}, \textsc{ext} \ibox{3}, \textsc{subcat} \ibox{1}] }\\
} $\wedge$ \ibox{2} $\neq$ \ibox{3}
\ex \sein (Perfekthilfsverb)\\
\ms{
subj   & \ibox{2}\\
subcat & \ibox{1} $\oplus$ \sliste{ V[\textsc{subj} \ibox{2}, \textsc{ext} \ibox{2}, \textsc{subcat} \ibox{1}] }\\
}
\ex\label{le-werden-kathol} \emph{werden} (Passivhilfsverb)\\
\ms{
subj   & \sliste{ NP[\type{nom}]\ind{2} }\\
subcat & \ibox{1} $\oplus$ \sliste{ V[\textsc{subj} \sliste{ NP[\type{acc}]\ind{2} }, \textsc{subcat} \ibox{1}] }\\
}
\zl
Das Merkmal \textsc{ext} wird zur Kennzeichnung des externen Arguments benutzt. Das \subjm ist bei
Kathol wie bei \citet{Pollard90a} und auch im hier vorliegenden Buch
kein Valenzmerkmal \citep[\page243]{Kathol94a}. Das heißt, daß in den Lexikoneinträgen in (\mex{-1})
sowohl die Elemente in \textsc{ext} als auch die in \subj blockiert sind und nicht direkt mit den
Partizipien kombiniert werden können. Das Perfekthilfsverb \haben 
in (\mex{0}a) deblockiert die Elemente in \textsc{ext} und in \subj. Für die unakkusativischen Verben
muß das Hilfsverb \sein in (\mex{0}b) verwendet werden, das das externe Argument deblockiert.

Der Vorteil von Kathols Ansatz ist, daß das logische Subjekt
aller Partizipien einheitlich als Element von \textsc{ext} repräsentiert ist,
der Nachteil ist aber, daß \emph{geliebt} kein Element in der \subcatl hat,
was fälschlicherweise vorhersagt, daß das Partizip nicht mit Argumenten kombiniert
werden kann. Da in Kathols Ansatz das Hilfsverb \emph{hat} das externe Argument und das
Element in \subj deblockiert, kann \emph{seine Frau} in (\mex{1}) nur als Argument des Hilfsverbs
realisiert werden, und es bleibt unklar, wodurch die Projektion des Partizips im Vorfeld
lizenziert wird.
\ea
Seine Frau geliebt hat er nie.
\z
Selbst wenn man eine Spezialregel zur Kombination des Partizips mit dem Subjekt einführen würde,
könnte man Kathols Ansatz nicht retten, denn in Sätzen wie (\ref{bsp-subjekt-im-vf-passiv}) -- hier als
(\mex{1}) wiederholt -- muß das Objekt
des Verbs im Vorfeld im Nominativ stehen (vergleiche auch die Subjekt"=Verb"=Kongruenz, die zeigt,
daß es sich um Subjekte -- also Nominative -- handelt), was den Kasusspezifikationen in (\mex{-2}) widerspricht.
\eal
\label{bsp-subjekt-im-vf-passiv-zwei}
\ex\iw{brechen}
Die Nase gebrochen wurde dem Boxer schon zum zweiten Mal.\footnote{
        \citew[\page386]{Luehr84a}.
}
\ex\iw{erschießen}
Zwei Männer erschossen wurden    während des Wochenendes.\footnote{
        \citew*[\page210]{Webelhuth85a}\ia{Webelhuth}.%
        }
\label{bsp-subjekt-im-vf-passiv-letzt-zwei}
% Worden steht aber mit vorn.
% \ex\iw{bauen}
% Ein Haus gebaut worden ist gestern nicht.\footnote{
%         \citew[\page433]{Sternefeld85a}.
% }
\zl

\noindent
Außerdem kann Kathols Ansatz modale Infinitive\is{Infinitiv!modaler}\is{Passiv!modaler Infinitiv} 
und inkohärente Infinitivkonstruktionen nicht mit einem Lexikoneintrag parallel zum Vorgangspassiv
erklären:
Da das Akkusativobjekt als Element der \subjl repräsentiert ist, kann man keine VP
bilden. Die einzige Lösung für dieses Problem scheint die Stipulation eines zusätzlichen
Lexikoneintrags für \emph{zu}"=Infinitive, die in einer VP auf"|treten, zu sein.
Wie zum Beginn des Abschnitts~\ref{sec-passive-anal} dargelegt wurde, ist eine der Hauptmotivationen
der Objekt"=zu"=Subjekt"=Anhebungsansätze, Mehrfacheinträge für Verben in den jeweiligen
Formen zu vermeiden.

Mit dem Lexikoneintrag für das Passivhilfsverb \emph{werden} in (\ref{le-werden-kathol}) 
kann Kathol das unpersönliche Passiv nicht erklären, da die eingebetteten Verben bei unpersönlichen Passiven kein
Akkusativargument haben. Er muß also noch einen weiteren Lexikoneintrag für \emph{werden}
annehmen.

Zu guter Letzt muß man anmerken, daß die Beschränkung, daß unter \haben eingebettete
Verben verschiedene \textsc{subj}"= und \textsc{ext}"=Werte haben müssen, zu stark ist,
da durch diese Beschränkung das Perfekt von subjektlosen Verben\is{Verb!subjektloses}
wie \word{grauen} (siehe (\ref{ex-hat-gegraut})) ausgeschlossen wird, denn bei solchen Verben sind
sowohl \subj als auch \textsc{ext} die leere Liste. 






\subsection{Kathol: 1991 und Pollard: 1994}
\label{kathol-pollard}


%% Toman86a ähnlicher Vorschlag, pränominales Partizip wird über Unterspezifikation und Ausgliederung
%% der Argumentvererbung in Prinzipien gemacht. Diskutiert und verwirft ein unsichtbares unakkusativisches \sein.

%% Bei Adjektiven hätte man dann
%% \ea[\#]{
%% das grün seiende Krokodil
%% }
%% \z
%% Die Frage ist dann allerdings, warum es kein leeres \haben gibt:
%% \eal
%% \ex[??]{
%% das geschlafen habende Krokodil
%% }
%% \ex[*]{
%% das geschlafene Krokodil
%% }
%% \zl

\citet{Pollard94a} benutzt nicht wie wir ein Merkmal, das auf das Element mit Subjekteigenschaften
verweist, sondern verweist statt dessen auf das Element mit Akkusativobjekteigenschaften.\footnote{
        Pollards Ansatz ist eine Ausarbeitung von Ideen aus \citew{Kathol91a}.
        Pollard vereinigt die Analysen des persönlichen und unpersönlichen Passivs und diskutiert
        außerdem auch das Fernpassiv.%
}
Pollard nimmt an, daß das Subjekt infiniter Verben nicht auf
der \subcatl repräsentiert wird, sondern wie im hier vorliegenden Buch auch als Wert von \subj.
Zur Auszeichnung eines Arguments benutzt Pollard das Merkmal \HPSGerg.\isfeat{erg}
Bei transitiven Verben ist der Wert von \HPSGerg mit dem Akkusativobjekt identisch. 
Bei unakkusativischen Verben entspricht \HPSGerg dem Subjekt.
Intransitive unergativische Verben haben als \ergw die leere Liste.
(\mex{1}) %\vpageref{kp-erg-values} 
zeigt die \subj"=, \HPSGerg"= und \subcatwe der Verben \word{ankommen},
\word{tanzen}, \word{auf"|fallen}, \word{lieben}
%, \word{schenken} (`to give as a present'), 
und \word{helfen}.
%\begin{figure}[htbp]
\ea\label{kp-erg-values}
\begin{tabular}[t]{@{}l@{ }l@{ }l@{ }l@{~~ }l@{ }}
  &                               & \textsc{subj}                          &  \textsc{erg}          & \textsc{subcat}\\[2mm]
a.&ankommen (unakkusativisch):    & \sliste{ \ibox{1} NP[\type{str}] } & \sliste{ \ibox{1} } & \sliste{}\\[2mm]
b.&tanzen   (unergativisch):      & \sliste{ NP[\type{str}] }          & \sliste{}           & \sliste{}\\[2mm]
c.&auf"|fallen (unakkusativisch): & \sliste{ \ibox{1} NP[\type{str}] } & \sliste{ \ibox{1} } & \sliste{ NP[\type{ldat}] }\\[2mm]
d.&lieben      (transitiv):       & \sliste{ NP[\type{str}] }          & \sliste{ \ibox{1} } & \sliste{ \ibox{1} NP[\type{str}] }\\[2mm]
%e.&schenken    (ditransitiv):  & \sliste{NP[\type{str}]}          & \sliste{\ibox{1}} & \sliste{\ibox{1} NP[\type{str}], NP[\type{ldat}]}\\[2mm]
e.&helfen      (unergativisch):   & \sliste{ NP[\type{str}] }          & \sliste{}           & \sliste{ NP[\type{ldat}] }\\
\end{tabular}%
\z
%\vspace{-\baselineskip}\end{figure}
Bei unakkusativischen Verben wie \emph{ankommen} und {\em auf"|fallen} ist das Element in \HPSGerg 
identisch mit dem Element in \subj. Bei transitiven Verben ist das Element in \HPSGerg mit dem direkten 
Objekt identisch (\emph{lieben}) und bei unergativischen Verben ist der \ergw die leere Liste,
da es kein Element mit Akkusativeigenschaften gibt (\emph{tanzen} und \emph{helfen}).

Das Kernstück von Pollards Passivanalyse ist der Lexikoneintrag für das Passivhilfsverb in (\mex{1}).\footnote{
        Der Eintrag wurde an die in diesem Buch genutzte Merkmalsgeometrie angepaßt.
        Pollard repräsentiert das \ergm und das \subjm nicht als Kopfmerkmal.%
}
Das Passivhilfsverb bettet ein Verb mit der \vform{} \type{ppp}\istype{ppp}, \dash ein Partizip II, ein.
Das Hilfsverb zieht den Wert von \HPSGerg \iboxb{1} von der \subcatl des eingebetteten Verbs ab.

\eas
\label{le-passiv-prelim-Pollard}
\mbox{\emph{werden} (Passivhilfsverb, nicht finite Form):} \\
\ms{
 head   & \ms[verb]{ subj & \ibox{1} \\
                     erg  & \ibox{1} \\ 
                   } \\
 subcat & \ibox{2} $\oplus$ \liste{ \begin{tabular}{@{}l@{}l@{}}
                  \textrm{V[}&\textrm{\type{ppp}, %\textsc{flip}$-$, 
                                \textsc{lex}+, \textsc{subj}~\liste{ NP[\type{str}]$_{\type{ref}}$}, \textsc{erg}~\ibox{1},}
                          \textrm{\textsc{subcat}~\ibox{1} $\oplus$ \ibox{2}]}\\
                 \end{tabular}}\\[4mm]
}\iw{werden!passive|uu}
\zs

\noindent
Die restlichen Elemente \iboxb{2} werden auf die \subcatl des Hilfsverbs angezogen.

Der Lexikoneintrag in (\mex{0}) kann sowohl das persönliche als auch das unpersönliche
Passiv erklären und schließt das Passiv unakkusativischer Verben aus:
Das Passiv unakkusativischer Verben wird dadurch verhindert, daß das \HPSGerg"=Element
unakkusativischer Verben mit deren \subj"=Element identifiziert und nicht in der
\subcatl des eingebetteten Verbs enthalten ist. Deshalb ist \HPSGerg kein Präfix
der \subcatl des eingebetteten Verbs und unakkusativische Verben sind somit als
Argument des Passivhilfsverbs ausgeschlossen.

Bei \emph{tanzen} ist der \ergw des unter das Hilfsverb eingebetteten Verbs
die leere Liste. Das Ergebnis der Subtraktion der leeren Liste von einer anderen
Liste ist die Liste selbst. Im Fall von {\em tanzen} ist \iboxt{2} die leere Liste.
Da der \ergw von \emph{tanzen} die leere Liste ist, ist der \subjw 
von \emph{getanzt werden} ebenfalls die leere Liste.

Bei \emph{helfen} ist die Situation ähnlich. \iboxt{2} wird hier durch \sliste{ NP[\type{ldat}] }
instantiiert. Der \subjw von \emph{geholfen werden} ist mit dem \ergw von \emph{geholfen} -- der leeren Liste -- identisch.
Abbildung~\vref{abb-unpers-pass} zeigt das im Detail.
\begin{figure}
\begin{forest}
[\ms{ head & \ibox{1} \\
      subcat & \ibox{2} \\
    }
  [\iboxt{4}~\onems{ loc$|$cat \onems{ head  \ms[verb]{ vform & ppp \\
                                                  subj & \sliste{ NP[\type{str}] }\\
                                                  erg  & \ibox{3} \liste{}\\
                                                }  \\
                                subcat~\ibox{3} $\oplus$ \ibox{2} \sliste{ NP[\type{ldat}] } \\
                              }} [geholfen]]
  [\onems{ head~\ibox{1} \ms[verb]{ vform & bse \\
                              subj & \ibox{3} \\
                              erg  & \ibox{3} \\
                            } \\
            subcat ~ \ibox{2} $\oplus$ \sliste{ \ibox{4} } \\[2mm]
          } [werden]]]
\end{forest}
\caption{\label{abb-unpers-pass}Pollards Analyse des Verbalkomplexes \emph{geholfen werden} in:\ \emph{daß dem Mann geholfen werden wird}}%
\end{figure}
Der \ergw von \emph{helfen} ist die leere Liste. Er wird von der Valenzliste von \emph{helfen}
abgezogen. Das Ergebnis ist \iboxt{2}, eine Liste, die das Dativobjekt enthält.
Diese Liste wird im Lexikoneintrag des Passivhilfsverbs übernommen.
Der \ergw von \emph{helfen} ist mit dem \subjw von \emph{werden} identisch.
Da dieser \subjw zu den Kopfmerkmalen von \emph{werden} gehört \iboxb{1}, wird
er durch das Kopfmerkmalsprinzip\is{Prinzip!Kopfmerkmals-} projiziert. 
Deshalb hat der Verbalkomplex \emph{geholfen werden} die leere Liste als \subjw.
Innerhalb der Verbalkomplexstruktur wird das verbale Argument des Hilfsverbs
abgebunden, die verbleibenden Element der \subcatl werden hochgereicht \iboxb{2}.
Deshalb hat der gesamte Verbalkomplex \emph{geholfen werden} eine \subcatl, die
nur das Dativobjekt enthält.

Abbildung~\vref{abb-pers-pass} zeigt eine Beispielanalyse für das persönliche Passiv
mit dem transitiven Verb \emph{lieben}.
\begin{figure}
\begin{forest}
sm edges
[\ms{ head & \ibox{1} \\
      subcat & \ibox{2} \\
    }
  [\iboxt{4}~\onems{ loc$|$cat \onems{ head  \ms[verb]{ vform & ppp \\
                                                  subj & \rule{0cm}{3ex}\liste{ NP[\type{str}] }\\[2mm]
                                                  erg  & \ibox{3} \sliste{ \ibox{5} NP[\type{str}] }\\
                                                }  \\
                                subcat~\ibox{3} \sliste{ \ibox{5} NP[\type{str}] } $\oplus$ \ibox{2} \liste{}\\
                              }} [geliebt]]
  [\onems{ head~\ibox{1} \ms[verb]{ vform & bse \\
                              subj & \ibox{3} \\
                              erg  & \ibox{3} \\
                            } \\
            subcat ~ \ibox{2} $\oplus$ \sliste{ \ibox{4} }\\[2mm]
          } [werden]]]
\end{forest}
\caption{\label{abb-pers-pass}Pollards Analyse des Verbalkomplexes \emph{geliebt werden} in:\ \emph{daß der Mann geliebt werden wird}}
\end{figure}
Der \ergw von Verben wie \emph{lieben} ist eine Liste mit einem Element \iboxb{3}. 
Diese Liste wird von der \subcatl des Partizips \emph{geliebt} abgezogen.
Das Ergebnis ist die leere Liste \iboxb{2}. Diese leere Liste wird
von  \emph{werden} angezogen, \dash, bei bivalenten transitiven Verben wie \emph{lieben}
enthält die \subcatl von \emph{werden} nur das selegierte Partizip. 
Der \ergw des Partizips \emph{geliebt} ist mit dem \subjw des Hilfsverbs \emph{werden} identisch.
Das Kopfmerkmalsprinzip sorgt dafür, daß die Kopfmerkmale von \emph{werden} projiziert werden,
und da das \subjm zu den Kopfmerkmalen gehört, hat der Verbalkomplex
\emph{geliebt werden} als \subjw eine Liste, die ein Element enthält, das identisch mit dem Akkusativobjekt
von \emph{geliebt} ist. Wenn ein finiter Verbalkomplex analysiert wird,
wird das zugrundeliegende Akkusativobjekt genauso zum Subjekt angehoben.
Da das Hilfsverb finit ist, bekommt das zugrundeliegende Akkusativobjekt
durch Pollards Kasusprinzip Nominativ zugewiesen.


Nachdem ich Pollards Analyse erklärt habe, wende ich mich nun den problematischen
Aspekten zu: Die bereits als (\ref{bsp-subjekt-im-vf-passiv}) diskutierten und als
(\ref{bsp-subjekt-im-vf-passiv-drei}) wiederholten Beispiele stellen ein Problem
für Pollards Ansatz dar, da das Subjekt der Passivkonstruktion
zusammen mit dem Partizip im Vorfeld stehen kann \citep[\page374]{Mueller99a}.
\eal
\label{bsp-subjekt-im-vf-passiv-drei}
\ex\iw{brechen}
Die Nase gebrochen wurde dem Boxer schon zum zweiten Mal.\footnote{
        \citew[\page386]{Luehr84a}.
}
\ex\iw{erschießen}
Zwei Männer erschossen wurden    während des Wochenendes.\footnote{
        \citew*[\page210]{Webelhuth85a}\ia{Webelhuth}.%
        }
\label{bsp-subjekt-im-vf-passiv-letzt-drei}
% Worden steht aber mit vorn.
% \ex\iw{bauen}
% Ein Haus gebaut worden ist gestern nicht.\footnote{
%         \citew[\page433]{Sternefeld85a}.
% }
%% \ex\iw{zurückgewinnen}\iw{sein!stative passive}
%% Ein verkanntes Meisterwerk dem Musiktheater  zurückgewonnen ist da    nicht.\label{bsp-subjekt-im-vf-zp}\footnote{
%%         ECI Multilingual Corpus CD I, Frankfurter Rundschau Korpus, File ger03a01.eci 
%%         (FR week, ending 5th of July 1992).
%%         Thanks to Wojciech Skut\ia{Skut}\aimention{Wojciech Skut}
%%         for searching this example in the \negra{} corpus and to Thorsten Brants\ia{Brants}\aimention{Thorsten Brants}
%%         for finding the exact reference.%
%%       }
%% % \ex Der Punkt erledigt ist niemals.\footnote{
%% %         Interview mit Voscherau, ARD, 20.05.1988, 22:45 Uhr, zitiert nach \citew[\page111]{Duerscheid89a}.
%% %     }
\zl
Das Objekt von \emph{erschießen} in (\mex{0}a) kann mit dem Partizip die Phrase
\emph{zwei Männer erschossen} bilden, aber dann ist es nicht mehr in der \subcatl von
\emph{zwei Männer erschossen} enthalten. Das Passivhilfsverb \emph{wurden} verlangt,
daß der \ergw des eingebetteten Partizips ein Präfix der \subcatl des Partizips ist,
was dann für die Projektion \emph{zwei Männer erschossen} nicht der Fall wäre.
Für Haiders\aimention{Hubert Haider} Ansatz besteht dieses Problem nicht: Die Argumentblockierung
findet gleichzeitig mit der Ableitung der Partizipform statt, \dash, nicht das Passivhilfsverb
blockiert das Subjekt. Da das Akkusativobjekt in der \subcatl des Partizips enthalten
ist, können Sätze wie (\mex{0}) in Haiders Analyse erklärt werden.


\subsection{Kathol: 1994}
\label{sec-kathol-passive-raising}

\citet[\page250]{Kathol94a} schlägt einen Lexikoneintrag für
das Passivhilfsverb \emph{werden} vor, der dem folgenden entspricht:

\ea
\begin{tabular}[t]{@{}l@{}}
\emph{werden} (Passivhilfsverb):\\
\ms{
subj & \ibox{3}\\
subcat & \ibox{1} $\oplus$ \liste{ V \ms{ vform & part ii\\
                                  subj  & \sliste{ np }\\[2mm]
                                  subcat & \ibox{2} $\oplus$ \ibox{1}\\
                                }}\\
}\\
constraint: (\,\ibox{2} = \liste{ NP[\type{acc}]\ind{4}} $\wedge$ \ibox{3} = \liste{ NP[\type{nom}]\ind{4}})\\[2mm]
otherwise: (\,\ibox{2} = \ibox{3} = \liste{})
\end{tabular}
\z
Wenn das eingebettete Partizip ein Akkusativobjekt hat 
(\,\ibox{2} = $\left\langle\right.$\,NP[\type{acc}]\raisebox{-0.8ex}{\ibox{4}}\,$\left.\right\rangle$), 
wird dieses als Subjekt des Hilfsverbs realisiert
(\,\ibox{3} = $\left\langle\right.$\,NP[\type{nom}]\raisebox{-0.8ex}{\ibox{4}}\,$\left.\right\rangle$).
In diesem Fall liegt ein persönliches Passiv vor. Wenn das eingebettete Partizip kein Akkusativobjekt hat,
werden alle Argumente des eingebetteten Verbs \iboxb{1} angehoben, der entstehende Verbalkomplex
hat kein Element in \subj (\iboxt{3} = \eliste{}), und es liegt somit eine unpersönliche Konstruktion vor.

Wie Pollards Ansatz versagt auch dieser Ansatz bei der Analyse von Sätzen
wie \pref{bsp-subjekt-im-vf-passiv-letzt-drei}: Die Projektion \emph{zwei Männer
erschossen} ist eine komplette VP, die nichts in der \subcatl enthält.
Der Lexikoneintrag in (\mex{0}) ist mit einem Partizip, das eine leere \subcatl hat, kompatibel,
aber das Ergebnis der Kombination ist ein subjektloser Verbalkomplex.
Subjektlose Verbalkomplexe stehen immer in der dritten Person Singular\is{Kongruenz!Subjekt"=Verb"=}, 
\dash, man würde \emph{wurde} statt \emph{wurden} als finites Verb erwarten.

%The drawback of such an approach is
%that the elements in the valence lists of the participle and in the valence lists of the auxiliary
%have differing case information. As was discussed in Chapter~\ref{sec-rais-contr-identity-coindexing},
%case agreeing adverbials can be used to determine the case of unexpressed arguments.
%\eal
%\ex[]{
%Gelesen wurden die Aufsätze einer nach dem anderen.
%}
%\ex[]{
%Einer nach dem anderen gelesen wurden die Aufsätze.
%}
%\ex[*]{
%Einen nach dem anderen gelesen wurden die Aufsätze.
%}
%\zl
%
%Die Jungen bekommen einen Ball geschenkt.
%
%Einer nach dem anderen einen Ball geschenkt bekommen die Jungen.
%Einem nach dem anderen einen Ball geschenkt bekommen die Jungen.



\subsection{Ryu: 1997}
\label{sec-ryu-passive}

\mbox{}\citet{Ryu97a} schlägt zwei neue Merkmale vor: eins zur Kennzeichnung des Indexes des
externen Arguments (\textsc{extarg}\isfeat{extarg}) und eins zur Kennzeichnung des Indexes des internen
Arguments (\textsc{intarg}\isfeat{intarg}). Diese Merkmale repräsentiert er zusammen mit einer Liste
der referentiellen Indizes aller Argumente (\textsc{args})
als Wert des Merkmals \textsc{argstr}, das für Argumentstruktur steht.
(\mex{1}) zeigt ein Beispiel für das transitive Verb \emph{schlagen}.
\ea
Argumentstruktur von \emph{schlagen} nach \citew[\page376]{Ryu97a}:\\
\ms{
extarg & \sliste{ \ibox{1} }\\
intarg & \sliste{ \ibox{2} }\\
args   & \sliste{ \ibox{1} } $\oplus$ \sliste{ \ibox{2} }\\
}
\z
Er nimmt die folgenden Lexikoneinträge für das Passivhilfsverb \emph{werden} an:\footnote{
  Ich habe seine Lexikoneinträge an die Merkmalsgeometrie, die in diesem Buch verwendet wird,
  angepaßt.%
}
\eas
\emph{werden} (Hilfsverb für das persönliche Passiv, S.\,377):\\
\resizebox{\linewidth}{!}{%
\onems{
head$|$subj \rule{0cm}{3ex}\liste{ NP[\emph{nom}]\ind{2}}\\
subcat \liste{ PP[\type{von}]\ind{1}} $\oplus$ \ibox{4}  $\oplus$ \liste{ \ms{ head   & \ms[verb]{ vform & psp\\}\\
                                                                          subcat & \liste{ NP[\emph{acc}]\ind{2}} $\oplus$ \ibox{4}\\[2mm]
                                                                          argstr  & \ms{
extarg & \sliste{ \ibox{1} }\\
intarg & \sliste{ \ibox{2} }\\
args   & \sliste{ \ibox{1} } $\oplus$ \sliste{ \ibox{2} } $\oplus$ \ibox{3}\\
}\\
}}\\
}
}
\zs

\eas
\emph{werden} (Hilfsverb für das unpersönliche Passiv, S.\,379):\\
%\resizebox{\linewidth}{!}{%
\onems{
head$|$subj \rule{0cm}{3ex}\liste{ }\\
subcat \liste{ PP[\type{von}]\ind{1}} $\oplus$ \ibox{4}  $\oplus$ \liste{ \ms{ head   & \ms[verb]{ vform & psp\\}\\
                                                                          subcat & \ibox{4}\\[1mm]
                                                                          argstr  & \ms{
extarg & \sliste{ \ibox{1} }\\
intarg & \sliste{ }\\
args   & \sliste{ \ibox{1} } $\oplus$ \ibox{3}\\
}\\
}}\\
}
\zs
% Like Kathol's approach \citeyearpar[\page250]{Kathol94a}, this is a control approach, \ie, only the
% indeces and not the \synsem values of the accusative object of the participle and the subject of
% the passive auxiliary are shared.

\noindent
Beispiele wie \pref{bsp-subjekt-im-vf-passiv-letzt-drei} und (\mex{1}) sind für
Ryus Ansatz problematisch, da er davon ausgeht, daß die Argumentstruktur
nur in Lexikoneinträgen, also nicht auf der phrasalen Ebene repräsentiert wird.
\ea
Einem Jungen    geschenkt wurde das Buch       dann doch      nicht.
\z
In \pref{bsp-subjekt-im-vf-passiv-letzt-drei} und (\mex{0}) wird das \vf durch eine
komplexe Konstituente besetzt. Diese komplexe Konstituente ist der Füller in einer
Fernabhängigkeit. \emph{wurde} wird mit einer Spur kombiniert, und die Anforderungen
des Passivhilfsverbs werden mit den Eigenschaften der Spur identifiziert.
Da die Argumentstruktur nicht projiziert wird, ist die Konstituente
\emph{einem Jungen geschenkt} entweder mit der Spur nicht kompatibel
oder die Grammatik generiert über:
Wenn der Wert von \textsc{argstr} für Phrasen \type{none} oder etwas Ähnliches ist,
schlägt die Analyse fehl, da die Restriktionen für die Spur mit dem Füller inkompatibel sind.
Wenn der Wert von \textsc{argstr} für Phrasen nicht beschränkt wird, läßt
die Grammatik nicht wohlgeformte Sätze wie (\mex{1}) zu, in denen das Partizip eines
unakkusativischen Verbs zusammen mit einem Argument vorangestellt wurde.
\ea[*]{
Dem Mann aufgefallen wurde nicht.
}
\z
(\mex{0}) kann als unpersönliches Passiv analysiert werden, da die Beschränkung,
daß das eingebettete Partizip ein Element in \textsc{extarg} haben muß, nicht 
erzwungen werden kann, da Information über die Argumentstruktur, die zu einem Widerspruch führen
könnte, bei der Projektion \emph{dem Mann aufgefallen} nicht vorhanden ist.




%% \subsection{Zeiger oder Merkmal?}

%% Die Ansätze, die bisher diskutiert wurden, verwenden alle listenwertige Merkmale, um Zeiger
%% auf bestimmte Elemente zu verwalten. In der Implementierung von Haiders Ansatz wurde das
%% \dam verwendet, um auf das Element zu verweisen, daß Subjekteigenschaften hat.
%% In weniger formalen Arbeiten werden bestimmte Argumente mitunter durch Unterstreichung
%% \citep[\page10]{Haider86} oder Kursivschreibung markiert. Eine solche Markierung kann man
%% einerseits mit den erwähnten listenwertigen Merkmalen formal umsetzen oder mit einem binären
%% Merkmal, das für jedes Argument eines Verbs einen entsprechenden Wert haben muß.
%% Die Verwendung eines solchen binären Merkmals wurde \zb von \citet{MR2001a} im Rahmen
%% der \cxg vorgeschlagen (siehe Kapitel~\ref{cxg-linking-konstruktionen}). In
%% (\mex{1}) und (\mex{2}) sind die beiden Repräsentationsformen gegenüber gestellt.

%% \ea
%% \begin{tabular}[t]{@{}l@{ }l@{\hspace{5ex}}l@{\hspace{5ex}}l@{}}
%%   &                           & \textsc{da}                            & \textsc{subcat}\\[2mm]
%% a.&ankommen (unakkusativisch):    & \sliste{}                           & \sliste{NP[\type{str}]}\\[2mm]
%% b.&lieben      (transitiv):   & \sliste{ \ibox{1} NP[\type{str}] } & \sliste{ \ibox{1}, NP[\type{str}] }\\
%% \end{tabular}
%% \z
%% \ea
%% \begin{tabular}[t]{@{}l@{ }l@{\hspace{5ex}}l@{}}
%%   &                           & \textsc{subcat}\\[2mm]
%% a.&ankommen (unakkusativisch):    & \sliste{NP[\type{str}, \da $-$]}\\[2mm]
%% b.&lieben      (unergativisch):   & \sliste{NP[\str, \da +], NP[\str, \da $-$]}\\
%% \end{tabular}
%% \z
%% Die Repräsentation in (\mex{-1}) sagt folgendes aus: Für das Verb \emph{lieben} gibt es
%% zwei Argumente. Das erste der beiden Argumente ist das designierte Argument. Wir können es
%% auf"|finden, indem wir in die \dalist gucken.

%% Die Repräsentation in (\mex{0}) sagt folgendes aus: Für das Verb \emph{lieben} gibt es
%% zwei Argumente. Das erste der beiden Argumente hat die Eigenschaft, designiert zu sein.
%% Wir können designierte Argumente finden, indem wir die \dawe der Elemente in der \subcatl
%% angucken.

%% Auf den ersten Blick scheinen beide Repräsentationen gleichermaßen geeignet zu sein, betrachtet
%% man allerdings Interaktionen mit anderen Bereichen der Grammatik, stellt man fest, daß die
%% Annahme eines binären Merkmals unerwünschte Konsequenzen hat:
%% Im Kapitel~\ref{chap-verbalkomplex} haben wir eine sehr gut funktionierende
%% Analyse des Verbalkomplexes kennengelernt. Ein wesentlicher Bestandteil dieser Analyse ist
%% die Argumentanziehung. Bei der Analyse eines Verbalkomplexes wie (\mex{1}) bilden \emph{zu reparieren}
%% und \emph{versucht} eine
%% \ea
%% weil er ihn den Wagen reparieren sah
%% \z

%% {sec-remote-passive-hpsg}



\subsection{Agensausdrücke}
\label{sec-agensausdruecke}

%% In\is{Agensausdruck|(} den beiden folgenden Abschnitten werden Vorschläge zur Analyse der Agensausdrücke untersucht. Der
%% erste Abschnitt widmet sich einem Ansatz, der ein Hilfsprädikat und Schlußregeln verwendet, und der
%% zweite beschäftigt sich mit einem recht einflußreichen Ansatz, der die Subjekte überhaupt nicht als
%% Argumente ihrer Verben analysiert.

\begin{comment}

% Bei meinem Ansatz ist die Behandlung des Agens, wenn keine PP auftritt, unklar.

\subsubsection{Ein AGENS-Prädikat und Inferenz}

\mbox{}\citet[\page740]{Wunderlich93a} schlägt vor, die Agens"=Phrase beim Passiv als Adjunkt zu behandeln.
Die Agens"=Rolle von passivierten Verben wird existenziell abgebunden, für \emph{geküßt werd-} nimmt
er die Repräsentation in (\mex{1}a) an, und (\mex{1}b) zeigt die Repräsentation des Agensausdrucks
\emph{von Anna}.
\eal
\ex \emph{geküßt werd-}: $\lambda y \lambda s \exists x$ KÜSS$(x, y)(s)$
\ex \emph{von Anna}: $\lambda s$ AGENS$(s, a)$
\zl
Hierbei steht $s$ für die vom Verb bezeichnete Situation. Die Repräsentation, die man erhält, wenn man
beide Phrasen kombiniert, zeigt (\mex{1}a):
\eal
\ex $\lambda y \lambda s (\exists x$ KÜSS$(x, y)(s)$ \& AGENS$(s, a))$
\ex $\lambda y \lambda s (\exists x$ KÜSS$(x, y)(s)$ \& AGENS$(s, a)) \to \lambda y \lambda s$ KÜSS$(a, y)(s)$ 
\zl
Die Identifikation der jeweiligen Situationsvariablen erfolgt durch Unifikation\footnote{
  Das ist parallel zur Behandlung der Kombination von Adjektiv und Nomen in Kapitel~\ref{sem-adj}.%
} bzw.\ Theta"=Identifikation\is{Theta"=Identifikation}, wie das Verfahren von \citet[\page564]{Higginbotham85a} genannt wurde.

Besitzt man Wissen darüber, an welcher Stelle der semantischen Repräsentation von Küssen"=Situationen
welche Argumentrolle steht, dann kann man wie in (\mex{0}b) aus der AGENS"=Relation ableiten, welcher
Slot von KÜSS durch $a$ gefüllt werden muß. Da die Information darüber, welches Argument ein AGENS
ist, in KÜSS(x,y) nicht explizit enthalten ist, müßte man spezielle Bedeutungspostulate wie in (\mex{1})
haben, um die entsprechenden Schlüsse ziehen zu können:
\ea
(KÜSS$(x, y)(s)$ \& AGENS$(s, a)) \to$ KÜSS$(a, y)(s)$
\z
Entsprechende Postulate müßte es für alle Verben mit Agens"=Rolle geben. Arbeitet man mit Repräsentationen,
wie sie in der HPSG verwendet werden, wäre eine Implikation ausreichend, da man auf die Klasse der Prädikate,
die ein Agens"=Argument haben, über ihren Typ zugreifen kann. Man könnte somit auf die explizite Erwähnung
der einzelnen Prädikate (KÜSS in (\mex{0})) verzichten. Allerdings hat man in die Beschreibungen der Situationen,
über die man Aussagen macht, ein Prädikat AGENS eingeführt, das nur eine Hilfsfunktion hat: Die Verbindung 
zum Agens"=Argument wurde in der Syntax nicht hergestellt und muß dann mittels Schlußverfahren ermittelt werden.
Im  Ansatz, der im Abschnitt~\ref{sec-analyse-agensausdruecke} vorgestellt wurde, ist weder das
Hilfsprädikat AGENS noch ein gesondertes Schlußverfahren nötig.
\end{comment}

%\subsubsection{Generelle Ausgliederung der Agens"=Rolle}

\citet{Kratzer96a} hat vorgeschlagen, das Agens grundsätzlich nicht als Argument von Verben zu behandeln.
Statt der Repräsentation in (\mex{1}a) bzw.\ der Repräsentation in (\mex{1}b), bei der die semantischen
Rollen in Prädikate ausgelagert sind, nimmt sie die Repräsentation in (\mex{1}c) bzw.\ (\mex{1}d)
an:\footnote{
  Zu einer Diskussion anderer Aspekte von Kratzers Analyse siehe \citew[Abschnitt~7]{MWArgSt}.
}
\eal
\label{lex-buy}
\ex $\lambda x \lambda y \lambda e~buy(x)(y)(e)$
\ex $\lambda x \lambda y \lambda e~buying(e) \& Theme(x)(e) \& Agent(y)(e)$
\ex $\lambda x \lambda e~buy(x)(e)$
\ex $\lambda x \lambda e~buying(e) \& Theme(x)(e)$
\zl
$e$ steht hier für die Event"=Variable, die der Situationsvariable anderer Autoren entspricht.
Während in (\mex{0}a,b) die Variable, die dem Agens von \emph{buy} entspricht, durch ein $\lambda$ gebunden
ist, ist das in (\mex{0}c,d) nicht der Fall. Kratzer behandelt das Agens als Element, das nicht vom Verb selegiert
wird (weder syntaktisch, \dash über Valenzinformation, noch semantisch, \dash über Zuweisung einer semantischen Rolle),
sondern über eine funktionale Projektion (VoiceP)\is{Kategorie!funktionale!Voice}
mit dem Verb verbunden wird.
Für den Satz (\mex{1}) nimmt \citet[\page121]{Kratzer96a} die Tiefenstruktur in Abbildung~\vref{voicep} an.
\ea
Mittie feeds the dog.
\z
\begin{figure}
\begin{forest}
sm edges
[VoiceP
	[DP
		[Mittie]]
	[\hphantom{$'$}Voice$'$
		[Voice
			[agent]]
		[VP
			[DP
				[the dog,roof]]
			[\hphantom{$'$}V$'$
				[V
					[feed]]]]]]
\end{forest}
\caption{\label{voicep} Das Agens"=Prädikat und die funktionale Projektion VoiceP}
\end{figure}
Kratzer nimmt an, daß
im allgemeinen Köpfe ihre Argumente in der D"=Struktur in der Spezifikatorposition realisieren
(S.\,120). Deshalb wird das Objekt von \emph{feed} nicht direkt mit V kombiniert, sondern
das V wird zu V$'$ projiziert, und erst danach werden Objekt und V$'$ zu VP verbunden.
Der Voice"=Kopf ist dabei phonologisch leer\is{leere Kategorie|(}. Das Agens"=Prädikat, das in
semantischen Repräsentationen von Lexikoneinträgen
wie (\mex{-1}c) nicht enthalten ist, wird vom Voice"=Kopf eingeführt. 
Kratzer unterscheidet
zwischen Voice"=Köpfen für Aktivstrukturen und solchen für Passivstrukturen. Der Kopf
für Aktivstrukturen realisiert das Agensargument (wie in Abbildung~\ref{voicep}) und
weist der DP in der VP Akkusativ zu. Der Kopf für Passivstrukturen realisiert
kein Agensargument und weist keinen Akkusativ zu (S.\,123). Diese Art der Kasuszuweisung
unterscheidet sich von der hier vorgestellten dadurch, daß sie nicht lokal\is{Lokalität} ist: Voice
bestimmt den Kasus eines Elements innerhalb einer eingebetteten Phrase. In früheren
Arbeiten zur \gb wurde bei Kasuszuweisungen in ähnlichen Konfigurationen immer von
\emph{Exceptional Case Marking} gesprochen. In Kratzers Analyse wird diese Art von
Kasuszuweisung zum Normalfall.\footnote{
  Siehe auch \citew[\page 223]{Abraham2005a} zur angestrebten Lokalität der Rektion im Rahmen der \gbt.
}

Kratzer geht nicht darauf ein, wie Sätze ohne (Akkusativ-)Objekt behandelt werden sollen. Für
Sätze wie (\mex{1}) muß man wohl einen weiteren Voice"=Kopf (bzw.\ eine Disjunktion
in der Beschreibung des Aktiv"=Voice"=Kopfes) annehmen, der keinen Akkusativ zuweist.
\eal
\ex Er schläft.
\ex Er hilft dem Mann.
\ex Er denkt an die Zukunft.
\zl

\noindent
\citet[\page123]{Kratzer96a} erläutert am Beispiel des Verbs \emph{own}, was mit Prädikaten passiert,
die kein Agens verlangen.
\ea
\gll Mittie owns a dog.\\
     Mittie besitzt einen Hund\\
\z
Für diese Fälle schlägt sie einen zusätzlichen leeren Kopf vor, der ein entsprechendes
Prädikat zur Beschreibung der Rolle einführt. Sie nennt das Prädikat \emph{holder}.
Kratzer führt getypte Variablen für das Situationsargument ein: Statt der Variable
$e$, die in der lexikalischen Repräsentation von \emph{buy} in (\ref{lex-buy}) eine Rolle
spielt, nimmt sie für Zustände eine Variable $s$ an. Die entsprechenden Repräsentationen
zeigt (\mex{1}):\footnote{
  Die Sternchen kennzeichnen, daß es hier um die semantische Repräsentation der entsprechenden
  Wörter geht.}$^,$\footnote{
  In Kratzers Text steht in (\mex{1}a) als Argument von \emph{holder} ein $e$ ($holder(x)(e)$),
  der Text legt aber nahe, daß ein $s$ intendiert ist.%
} 
\eal
\ex Holder$^*$ = $\lambda x_e \lambda s_s~holder(x)(s)$
\ex own the dog$^*$ = $\lambda s_s~own(the~dog)(s)$
\zl
Situationsvariablen vom Typ $s$ sind mit solchen vom Typ $e$ unverträglich, weshalb
die Kombination eines ein Agens einführenden Kopfes mit \emph{own} ausgeschlossen ist.
Genauso kann der \emph{Holder}"=Kopf nicht mit einem Aktionsprädikat kombiniert werden.

Diese Art Selektion ließe sich sehr leicht in eine HPSG"=Analyse übersetzen, wenn man
alle Relationen entsprechend typt: Relationen, die ein Agens"=Argument haben, sind vom Typ
\type{agens-rel}, die, die ein Patiens"=Argument haben, sind vom Typ \type{patients-rel}.
Relationen, die sowohl ein Agens"=Argument als auch ein Patiens"=Argument haben, sind
vom Typ \type{agens-patiens-rel}. Außerdem braucht man Untertypen für Relationen, die
ein Agens"=, aber kein Patiens"=Argument haben (\type{agens-only-rel}) usw. \emph{agens-only-rel}
würde von \type{agens-rel} erben, aber nicht von \type{patients-rel}. Gegen solch eine Analyse
sprechen aber folgende Argumente: 1) Man muß für jede semantische Rolle, die auf diese Weise
eingeführt werden soll, einen leeren Kopf annehmen. 
%2) Man muß sicherstellen, daß Argumente nur einmal realisiert werden. 
2) Analysen, die Komplexbildung annehmen, wären nicht mehr möglich. Auf diese Punkte gehe ich im folgenden einzeln ein. 

In der in diesem Buch vorgestellten Analyse des Deutschen ist für die Zuweisung
von Argumentrollen und von Kasus kein einziger leerer Kopf nötig. Aktiv und Passiv
werden über unterschiedliche Argumentrealisierungen erklärt. Die Theorie benötigt
daher weniger Annahmen in bezug auf nicht direkt sichtbare Entitäten und kann trotzdem
die Daten erklären. Die lizenzierten Strukturen sind kleiner als in Kratzers System.
Der hier vorgestellten Theorie ist also aus diesen Gründen der Vorzug zu geben.%
\is{leere Kategorie|)}

%% Will man innerhalb der Grammatik sicherstellen, daß Argumente nicht mehrfach realisiert werden, 
%% so muß man dafür sorgen, daß \zb VoiceP nur einmal in einer Struktur eines Satzes vorkommen
%% kann. Dies kann man erreichen, indem man die funktionalen Köpfe einander entsprechend
%% selegieren läßt. Wenn Voice immer eine VP als Argument nimmt, dann ist klar, daß VoiceP
%% nicht unter Voice eingebettet werden kann, und somit ist zumindest an dieser Stelle der Grammatik
%% Iteration von Voice ausgeschlossen.

Im Kapitel~\ref{sec-anhebung} wurden Anhebungsverben wie \word{scheinen} und \word{sehen} diskutiert.
Nimmt man für Sätze mit diesen Verben eine Analyse mit Prädikatskomplex an, so funktioniert Kratzers
Rollenzuweisung nicht ohne weiteres.
\eal
\ex weil die Männer [zu lächeln scheinen]
\ex weil er die Männer [lächeln sieht]
\zl
Die Analyse, die in Kapitel~\ref{sec-anhebung-anal} vorgestellt wurde, geht davon aus, daß die Verben
in (\mex{0}) jeweils einen Komplex bilden. Die Argumente der an der Komplexbildung beteiligten Verben
werden vom jeweils höchsten Verb angezogen und werden so zu Argumenten des gesamten Komplexes.
\emph{die Männer} ist deshalb das Subjekt von \emph{scheinen} und auch von \emph{zu schlafen scheinen}. Die
Bedeutung des Verbalkomplexes in (\mex{0}a) wäre in Kratzers System etwas wie (\mex{1}).
\ea
$\lambda s$ \relation{scheinen}(\relation{lächeln}(e))(s)
\z
Das heißt, die Event"=Variable, die man für die Realisierung des Agens von \emph{lächeln} braucht, wäre
nach einer Verbalkomplexbildung nicht mehr zugänglich. Das Prädikat \emph{scheinen} selbst
hat aber keine Situationsvariable, die mit einem Agens kompatibel wäre.
Für Kratzer ist das kein Problem, wenn sie annimmt, daß \emph{scheinen} einen Satz einbettet, 
eine solche Analyse ist jedoch mit allgemeinen Annahmen in der HPSG nicht verträglich: 
Zum Beispiel wird Kongruenz als eine lokale Relation behandelt, weshalb
\emph{die Männer} als syntaktisches Argument von \emph{scheinen} behandelt werden sollte.\footnote{
  Siehe Kapitel~\ref{sec-anhebung-diskontinuierliche-konstituenten}
%\citew[Abschnitt~5.1]{Kathol98b} und \citew[Kapitel~21.1]{Mueller99a} 
zu Reapes Vorschlag \citeyearpar{Reape94a},
  einen Satz unter \emph{scheinen} einzubetten.%
}
\is{Agensausdruck|)}

\section{Anhang}
\label{kasus-anhang}

In diesem Anhang soll das Kasusprinzip, das bisher nur informal angegeben wurde,
präzise formalisiert werden. Die Formalisierung kann erst jetzt erfolgen, da erst jetzt die
relevanten Phänomene wie Fernpassiv und Verbalkomplexbildung in ausreichendem
Maße diskutiert wurden.

Zur Erinnerung sei hier die Prosaform von
Meurers' (\citeyear{Meurers99b}; \citeyear[Kapitel~10.4.1.4]{Meurers2000b})
Kasusprinzip wiederholt:

\begin{itemize}
\item[a.] Das am wenigsten oblique Argument eines Verbs mit strukturellem Kasus erhält Nominativ,
          es sei denn, es ist angehoben.
\item[b.] Alle anderen nicht angehobenen Argumente eines Verbs mit strukturellem Kasus erhalten Akkusativ.
\end{itemize}

\noindent
Meurers gibt eine sehr komplexe Formalisierung des Prinzips, die vollständige Äußerungen
durchwandert und Valenzlisten daraufhin überprüft, ob Argumente angehoben wurden oder nicht (siehe
(\ref{raised-wert}) auf Seite~\pageref{raised-wert}). Ein Ansatz, der solcherart komplexe Beschränkungen 
nicht benötigt, ist dem von Meurers vorgeschlagenen vorzuziehen. 
%
\citet[\page53--57]{MdK2001a} greifen die Kritik an der Komplexität früherer Formalisierungen
auf und formalisieren das Kasusprinzip auf eine Weise, die Elemente von \prz{}s Ansatz \citeyearpar{Prze99}
verwendet. Wichtig für das Funktionieren der Analyse ist, daß es eine Repräsentation
gibt, in der alle Argumente eines Verbs -- insbesondere auch die von anderen Verben angezogenen -- unabhängig
vom Ort ihrer Realisierung repräsentiert sind, denn nur so kann man erklären,
warum sich die Kasus der Elemente innerhalb der Projektionen im Vorfeld in
den Sätzen in (\mex{1}) unterscheiden.\footnote{
  Siehe auch Kapitel~\ref{kasus-adamp} zur Diskussion der Probleme, die die Daten in (\mex{1}) für \prz{}s Ansatz aufwerfen.%
}
\eal
\label{bsp-der-aufsatz-gelesen-zwei}
\ex {}[Der Aufsatz gelesen] wurde am Wochenende.
\ex {}[Den Aufsatz gelesen] hat er am Wochenende.
\zl
Die realisierungsunabhängige Repräsentation wird dadurch erreicht, daß
Argumente, die mit ihrem Kopf verbunden werden, nicht mehr aus der \subcatl entfernt,
sondern in der \subcatl nur als realisiert markiert werden (siehe auch \citew[\page 560]{Higginbotham85a} 
%und \citew{Speas90a} 
für einen ähnlichen Vorschlag in einem anderen theoretischen Rahmen). 
Eine Phrase ist dann eine Maximalprojektion eines Kopfes, wenn alle Argumente des Kopfes als realisiert markiert sind.
Elemente in der \subcatl, die bereits als realisiert markiert sind, nennt Meurers \emph{spirits}
(Geister\is{Geist}).\isfeat{realized}

Das entsprechend angepaßte Schema sieht folgendermaßen aus:
\begin{samepage}
\begin{schema}[Kopf-Argument-Schema (binär verzweigend mit "`Geistern"')]
\label{schema-bin-mark}
\type{head"=argument"=phrase}\istype{head"=argument"=phrase} \impl\\
\onems{
      synsem$|$loc$|$cat$|$subcat            \ibox{1} $\oplus$ \liste{ \ms{ argument & \ibox{2}\\
                                                              realized & \textrm{+}\\
                                                            } } $\oplus$ \ibox{3}\\
      head-dtr$|$synsem$|$loc$|$cat$|$subcat \ibox{1} $\oplus$ \liste{ \ms{ argument & \ibox{2}\\
                                                              realized & $-$\\
                                                            } } $\oplus$ \ibox{3} \\
      non-head-dtrs \sliste{ \onems{ synsem \ibox{2} \onems{ loc$|$cat$|$subcat \type{list of spirits}\\
                                                             lex  $-$\\ } \\ 
                                   } }\\
}
\end{schema}\is{Schema!Kopf"=Argument"=}
\end{samepage}

\noindent
Die Merkmale \textsc{argument}\isfeat{argument} und \textsc{realized}\isfeat{realized} wurden von \citet{Prze99} vorgeschlagen
und bereits auf Seite~\pageref{page-realized} diskutiert. Bisher wurde im Kopf"=Argument"=Schema
immer die Relation \emph{delete} verwendet. In Schema~\ref{schema-bin-mark} tauchen nun
diverse $\oplus$"=Relationen (\emph{append}) auf. Die beiden \emph{append}"=Relationen
werden gebraucht, um die \subcatl der Kopftochter in drei Teile zu zerteilen. Im zweiten
Teil befindet sich das Element, das realisiert werden soll. Dieses wird aber nicht einfach
aus der Liste der noch zu sättigenden Argumente entfernt, sondern es wird nur der 
\textsc{realized}"=Wert verändert. Die Reihenfolge der Elemente in der \subcatl der Mutter
muß der Reihenfolge in der \subcatl der Kopftochter entsprechen. Deshalb muß
die \subcatl der Mutter mittels \emph{append}
aus den Bestandteilen zusammengesetzt werden, aus denen auch die \subcatl der Kopftochter
besteht. Die \subcatl der Mutter ist also bis auf den \textsc{realized}"=Wert eines Arguments
identisch mit der \subcatl der Kopftochter. Von der Nichtkopftochter wird verlangt, dass sie
vollständig ist, \dash, dass alle Elemente der \subcatl der Nichtkopftochter \textsc{realized}+
sind. Das wird durch die Beschränkung \type{list of spirits} ausgedrückt. 

Dadurch daß Elemente nicht von der \subcatl entfernt, sondern nur markiert werden,
erreicht man, daß auch in den für \citet{Prze99} problematischen Fällen in
(\ref{bsp-der-aufsatz-gelesen-zwei}) alle Argumente des eingebetteten Verbs angehoben
werden, obwohl sie bereits in einer anderen Umgebung realisiert wurden. (\mex{2})
zeigt die Valenzlisten für die Sätze (\ref{bsp-der-aufsatz-gelesen-zwei}).
\ea
\begin{tabular}[t]{@{}l@{~}l@{~}l}
a. & \emph{der Aufsatz gelesen}:        & \subj + \subcat \sliste{ NP[\str]$_j$, \st{NP[\str]}$_k$ }\\
b. & \emph{den Aufsatz gelesen}:        & \subj + \subcat \sliste{ NP[\str]$_j$, \st{NP[\str]}$_k$ }\\
c. & \emph{der Aufsatz gelesen wurde}:  & \subj + \subcat \sliste{ \st{NP[\str]}$_k$ }\\
d. & \emph{den Aufsatz gelesen hat er}: & \subj + \subcat \sliste{ \st{NP[\str]}$_j$, \st{NP[\str]}$_k$ }\\
\end{tabular}
\z
Innerhalb der Projektion \emph{der/den Aufsatz gelesen} wird das Objekt von \emph{gelesen} realisiert,
weshalb es in (\mex{0}a, b) durchgestrichen ist. \emph{werden} läßt das designierte Argument von
\emph{lesen} blockiert, so daß nur das Objekt von \emph{lesen} angehoben wird und als einziges Element
in (\mex{0}c) repräsentiert ist. In (\mex{-1}b) deblockiert das Perfekthilfsverb das blockierte
designierte Argument des Partizips: Sowohl das Subjekt als auch das Objekt von \emph{lesen}
sind in (\mex{0}d) repräsentiert.

Zusätzlich zum Merkmal \textsc{realized}, das etwas über den Geist"=Status eines Elements aussagt,
braucht man das Merkmal \textsc{raised}\isfeat{raised}, das festhält, ob ein Element durch ein übergeordnetes
Prädikat angehoben wird oder nicht. Werden die Argumente wie bei \emph{der/den Aufsatz gelesen}
angehoben, darf keine Kasuszuweisung durch das eingebettete Verb (\emph{gelesen}) erfolgen.
Erst wenn -- wie \zb bei (\mex{0}c, d) -- klar ist,
daß Elemente nicht weiter angehoben werden, kann Kasus zugewiesen werden.

Die modifizierten Lexikoneinträge für das Perfekt- und das Passivhilfsverb sind in (\mex{1})
und (\mex{2}) zu sehen. 
% soll eigentlich nach den AVMs stehen, dann paßt die Beschränkung in der Fußnote aber nicht
% mehr auf die Seite.
\emph{attract}\isrel{attract} ist dabei eine relationale Beschränkung, die die Elemente einer Liste als
angehoben (\textsc{raised}+) markiert.\footnote{
attract(\eliste) := \eliste.\\
attract(\liste{ \ms{ argument & \ibox{1}\\
                     realized & \ibox{2}\\
                     raised & +\\
                   } $|$ \textrm{Rest} }) := \liste{ \ms{ argument & \ibox{1} \\
                                                       realized & \ibox{2}\\
                                                     } $|$ \textrm{attract(Rest)} }.
}

\eas
\mbox{\haben (Perfekthilfsverb):}\\
\ms{
da     & \ibox{1}\\
subcat & \textrm{attract(\ibox{1} $\oplus$ \ibox{2})} $\oplus$ \sliste{ V[\type{ppp}, \textsc{lex}+, \textsc{subj} \ibox{1}, \textsc{subcat} \ibox{2}] }\\[2mm]
}
\zs

\eas
\label{le-werden-passive-da-spirits}
\mbox{\emph{werden} (Passivhilfsverb):}\\
\ms{
da     & \liste{}\\
subcat & \textrm{attract\iboxb{1}} $\oplus$ \sliste{ V[\type{ppp}, \textsc{lex}+, \textsc{da} \sliste{ NP$_{ref}$}, \textsc{subcat} \ibox{1}] }\\
}
\zs
Die Argumente der Hilfsverben in (\mex{-1}) und (\mex{0}) sind in bezug auf ihren \textsc{raised}"=Wert
unspezifiziert, aber die Argumente des eingebetteten Verbs werden durch \emph{attract}
als \textsc{raised}+ gekennzeichnet. Die folgende Implikation drückt aus, daß Argumente finiter
Verben nicht angehoben werden können:

\eas
\ms{
synsem$|$loc$|$cat$|$head & \ms[verb]{ vform & fin\\ }\\
} \impl \\\\
\mbox{}\hspace{6em}\ms{ synsem$|$loc$|$cat$|$subcat & list\_of\_non\_raised\_arguments\\}
\zs
Der Grund dafür, daß Argumente finiter Verben nicht angehoben werden können,
ist, daß es keine Verben gibt, die finite Verben einbetten und mit ihnen einen Komplex bilden.
Bei einer Komplexbildung, die ein finites Verb enthält, ist das Finitum immer das höchste
Verb. Deshalb können die Argumente finiter Verben nicht angehoben werden.\footnote{
  Bei der Analyse der scheinbar mehrfachen Vorfeldbesetzung, die ich in \citew{Mueller2005d}
  vorgeschlagen habe, werden auch Argumente finiter Verben angehoben. Dabei handelt es sich aber um
  die Argumente einer Verbspur. Da Argumente overt realisierter finiter Verben nie angehoben werden,
  kann man das Antecedens der Implikation in (\mex{0}) spezifischer machen, so daß die Implikation
  nur auf overte Verben angewendet wird.%
}

Die Implikation in (\mex{0}) ist einer globalen Beschränkung wie der von Meurers und \prz
formulierten (siehe Seite~\pageref{raised-wert}) vorzuziehen.

Für andere Kontexte, in denen nicht weiter angehoben werden kann, müssen in den entsprechenden
Lexikoneinträgen Beschränkungen formuliert werden. So schließt \zb \emph{ohne} oder \emph{um}
einen Anhebungsbereich ab:
\eal
\ex Er hat das gesehen, ohne sich zu wundern.
\ex Er hat das gelesen, um sich über diese Angelegenheit zu informieren.
\zl
Es ist nicht möglich, wie das von \citet[\page54, Fußnote~54]{MdK2001a} mit entsprechend
anderer Merkmalsgeometrie vorgeschlagen wurde, die Implikation \type{sign} \impl \textsc{synsem$|$""raised}$-$
zu verwenden. Diese Implikation würde dafür sorgen, daß alle realisierten Argumente \textsc{raised}$-$
sind, was zu Problemen mit Sätzen wie (\ref{bsp-der-aufsatz-gelesen-zwei}) führen würde, denn dann
wäre das Objekt von \emph{gelesen} als \textsc{raised}$-$ markiert und \emph{attract} würde
fehlschlagen, da es ja die Argumente des eingebetteten Verbs \emph{gelesen} als \textsc{raised}$+$
markiert.
% als müßte lokal bei der Kombination mit \emph{gelesen} Kasus bekommen.

Die Information darüber, ob ein Element angehoben wurde oder nicht, darf natürlich
bei der Sättigung von Argumenten nicht verlorengehen, weshalb die Strukturteilung
der \textsc{raised}"=Werte in Kopf"=Argument"=Strukturen\is{Schema!Kopf"=Argument"=} explizit gemacht werden muß:
\begin{samepage}
\begin{schema}[Kopf-Argument-Schema (binär verzweigend mit "`Geistern"')]
\label{schema-bin-mark-final}
\type{head"=argument"=phrase}\istype{head"=argument"=phrase} \impl\\
\onems{
      synsem$|$loc$|$cat$|$subcat            \ibox{1} $\oplus$ \liste{ \ms{ argument & \ibox{2}\\
                                                              raised   & \ibox{3}\\ 
                                                              realized & $+$\\
                                                            } } $\oplus$ \ibox{4}\\
      head-dtr$|$synsem$|$loc$|$cat$|$subcat \ibox{1} $\oplus$ \liste{ \ms{ argument & \ibox{2}\\
                                                              raised   & \ibox{3}\\ 
                                                              realized & $-$\\
                                                            } } $\oplus$ \ibox{4} \\
      non-head-dtrs \sliste{ \onems{ synsem \ibox{2} \onems{ loc$|$cat$|$subcat \eliste\\
                                                             lex  $-$\\ } \\ 
                                   } }\\
}
\end{schema}
\end{samepage}


\noindent
Nach diesen Vorarbeiten kann nun das Kasusprinzip formalisiert werden:

\medskip
\noindent
\begin{minipage}{\linewidth}
\begin{prinzip-break}[Kasusprinzip]\is{Prinzip!Kasus-}
~
\begin{itemize}
\item[\textrm{a.}]
\onems{
synsem$|$loc$|$cat \ms{ head   & \ms[verb]{ subj & \ibox{1}\\
                                          }\\
                        subcat & \ibox{2}\\
                      }\\
} $\wedge$\\
\ibox{1} $\oplus$ \ibox{2} = \type{list\_of\_not\_np\_str} $\oplus$ \sliste{ \ibox{3} NP[\str, \textsc{raised}$-$] } $\oplus$ \etag
\impl\\\\
\mbox{}\hspace{3cm} \ibox{3} = \textrm{NP[\snom]}\\\\
\end{itemize}
\end{prinzip-break}
\end{minipage}

%
%%%%%%%%%%%%%%%%%%%%%%%%%%%%%%%%

\begin{itemize}
\item[\textrm{b.}]
\begin{tabular}[t]{@{}l@{}}
\onems{
synsem$|$loc$|$cat \ms{ head   & \ms[verb]{ subj & \ibox{1}\\
                                          }\\
                        subcat & \ibox{2}\\
                      }\\
} $\wedge$\\
\ibox{1} $\oplus$ \ibox{2} =\\
\oneline{\type{list\_of\_not\_np\_str} $\oplus$ \sliste{ NP[\str, \textsc{raised}$-$] } $\oplus$ \etag{} $\oplus$ \sliste{ \ibox{3} NP[\str, \textsc{raised}$-$] } $\oplus$ \etag \impl}\\\\
\mbox{}\hspace{3cm} \ibox{3} = \textrm{NP[\sacc]}
\end{tabular}
\end{itemize}

\noindent
Der Ausdruck in a bedeutet folgendes: Wenn sich die Liste, die aus der Verknüpfung des \subjwes
mit dem \subcatw besteht, so unterteilen läßt, daß es einen Listenanfang gibt, der aus Elementen besteht,
die keine NP mit strukturellem Kasus sind, und einen Listenrest, der mit einer Nominalphrase mit strukturellem
Kasus beginnt, die nicht angehoben ist \iboxb{3}, dann muß diese Nominalphrase strukturellen
Nominativ (\type{snom}) haben. \etag steht für einen beliebigen Wert, im konkreten Fall also für eine beliebige Liste.
Die Unterteilung der Liste \ibox{1} $\oplus$ \ibox{2} in den Listenanfang vom Typ \type{list\_of\_not\_np\_str}
ist wichtig, da sonst die bereits in Kapitel~\ref{kasus-adamp} angesprochenen Fälle des Fernpassivs mit Objektkontrollverben
nicht analysierbar wären (siehe auch Seite~\pageref{page-remote-passive-erlauben}). Bei solchen
Fernpassiven kann an der ersten Stelle der \subcatl ein Dativ stehen und erst an zweiter Stelle eine
NP mit strukturellem Kasus.

Die Implikation in b regelt die Akkusativzuweisung in analoger Weise: Zusätzlich
zur Spezifikation des Listenanfangs muß es noch eine weitere nicht angehobene
Nominalphrase mit strukturellem Kasus geben (diejenige, die Nominativ bekommt). Alle weiteren
nicht angehobenen Nominalphrasen mit strukturellem Kasus \iboxb{3} müssen dann im Akkusativ stehen.
Durch die beiden \etag in (\mex{0}b) kann die Liste nach der ersten NP[\str] beliebig zerlegt werden,
so daß \zb in der Liste \sliste{ NP[\str], NP[\str], NP[\str] }, wie sie in (\mex{1}) als \subcatl
von \emph{ließ} vorkommt, die zweite und dritte NP Akkusativ bekommt.
\ea
Er ließ den Jungen den Aufsatz holen.
\z



\questions{
\begin{enumerate}
\item Welche Arten von Passiv bzw.\ passivähnlichen Konstruktionen kennen Sie?
\item Worin unterscheiden sich die Hilfsverben für das Vorgangspassiv und das Perfekt
      in der hier vorgestellten Analyse?
\end{enumerate}
}

%\section*{Übungsaufgaben}

%\begin{enumerate}
%\item %\citet{Embick2004a} behauptet, daß lexikalische Theorien Sätze nicht analysieren können,
%      die adjektivische Partizipien + Resultativkonstruktionen enthalten. Er diskutiert ein
%      englisches Beispiel, das zu (\mex{1}) parallel ist.
%      \ea
%      Das Metall ist flach gehämmert.
%      \z
%%       Überlegen Sie, wie die Lexikonregel, die Sie zur Übungsaufgabe~\ref{ue-result}
%%       von Kapitel~\ref{chap-lexikon} geschrieben haben, mit den hier vorgestellten Adjektivbildungslexikonregeln
%%       interagiert. Wie kann man das Zustandspassiv in (\mex{0}) und die NP mit dem pränominalen Partizip
%%       in (\mex{1}) analysieren?
%%       \ea
%%       der leer gefischte Teich
%%       \z
%\end{enumerate}

\exercises{
\begin{enumerate}
\item Warum gibt es in (\mex{1}b) keine Kongruenz zwischen \emph{zwei Wochen}
      und dem Passivhilfsverb?
\eal
\ex Er hat im März zwei Wochen gearbeitet.
\ex weil im März zwei Wochen gearbeitet wurde
\zl

\item Laden Sie die zu diesem Kapitel gehörende Grammatik von der Grammix"=CD
(siehe Übung~\ref{uebung-grammix-kapitel4} auf Seite~\pageref{uebung-grammix-kapitel4}).
Im Fenster, in dem die Grammatik geladen wird, erscheint zum Schluß eine Liste von Beispielen.
Geben Sie diese Beispiele nach dem Prompt ein und wiederholen Sie die in diesem Kapitel besprochenen
Aspekte.

\end{enumerate}
}

%\section*{Literaturhinweise}
\furtherreading{
Die Vorschläge zur theoretischen Behandlung des Passivs kann man in zwei Klassen teilen:
Zum einen gibt es solche, die für das Partizip Perfekt und das Partizip Passiv einen eigenen
Lexikoneintrag annehmen (\citealp{Bresnan78a,Bresnan82a}; \citealp[\page214--218]{ps}; \citealp[\page189]{Bierwisch90a};
% Gunkel2003b:62 zitiert \citealp[\page312]{Jacobs94a}; aber das scheint mir ein Argumentanziehungsansatz zu sein
\citealp[\page656]{Kunze96};
\citealp{MS98a}; \citealp[Kapitel~4]{MR2001a}; \citealp*[\page231]{VHE2003a}),
%
% Gunkel2003b:70 alle, die annehmen, daß das Passivmorphem etwas sättigt, müssen auch zwei Einträge annehmen.
%
und zum anderen gibt es Ansätze, die davon ausgehen, daß es nur einen Eintrag für das Partizip II gibt, 
der in verschiedenen Umgebungen verwendet werden kann. Die
Argumente des Partizips II werden je nach Umgebung realisiert oder unterdrückt 
(\citealp[\page37]{Bech55a}; \citealp{Hoehle78a}; \citealp{Haider86}; \citealp{Toman86a}; 
\citealp[\page165]{Fanselow87a}; % nach Gunkel, noch prüfen
\citealp[\page283]{Hoekstra87a}; \citealp[\page171]{Stechow90a}).
% Hoekstra87a: sein blockiert Projektion der Subj-Theta-Rolle

Ähnlich liegt die Sache bei modalen Infinitiven: \citet{Haider84b} und \citet[\page174]{Demske94a}
nehmen an, daß es nur eine Form des \emph{zu}"=Infinitivs gibt. \citet{Wilder90a} dagegen geht
davon aus, daß es zwei verschiedene \emph{-en}"=Morpheme gibt, die für die Aktiv/""Passiv"=Varianten
der Konstruktionen mit \emph{zu}"=Infinitiv verantwortlich sind.

% Demske nimmt aber für attributive Formen andere Infinitive an.

Beim \emph{lassen}"=Passiv nehmen \citet[\page20]{Reis76c}, \citet{Wilder90a},
\citet[\page131--149]{Fanselow87a}, % Satz im Passiv eingebettet
\citet[\page265]{Demske94a} und 
Kunze (\citeyear[\page665]{Kunze96}; \citeyear[\page161]{Kunze97a}) % zwei Einträge
an, daß es Aktiv- bzw.\ Passivstrukturen für die Infinitive
bzw.\ zwei verschiedene Infinitive oder \emph{-en}"=Morpheme gibt.
\citet[\page70]{Hoehle78a} geht dagegen von zwei verschiedenen Einträgen für \emph{lassen} aus.

In HPSG"=Grammatiken für das Englische\il{Englisch} \citep[\page214--218]{ps}
und in früheren Versionen bzw.\ Vorläufern der \lfg \citep{Bresnan78a,Bresnan82a} wurde das Passiv mittels einer Lexikonregel
analysiert, die ein Hauptverb zu einem Passivpartizip in Verbindung setzt, das eine entsprechend
veränderte Valenzanforderung hat.
%\citet{Nerbonne82}\ia{Nerbonne}, gefeiert -> gefeiert werden
%Nerbonne86:924 -> Passiv-Metaregel
\citet[\page276]{Kiss92}, 
\citet{HN98a},
\citet[\page255]{Kathol98b} und
\citet{Mueller2000h}
haben ähnliche, auf Lexikonregeln basierende HPSG"=Analysen für das Deutsche entwickelt.
}

%%% Local Variables: 
%%% mode: latex
%%% TeX-master: "hpsg-lehrbuch"
%%% End: 


\begin{comment}

liefen die Jungen von unserer Straße auf tiefer gelegenen Tümpeln in den Feldern Schlittschuh, 
wenn diese vollgeregnet waren\footnote{
  taz, 19.03.1999, LeserInnenbriefe, S.\,7.
}

Aus einem ihrer Fenster im dritten Stock haben die Mitglieder des Neuen Forums,
der Vereinigten Linken, der Umweltbibliothek und die "`mündigen Bürger"' ein Transparent herausgehängt.
Vollgeregnet fordert es die "`Zerschlagung der Stasi-Struktur -- Gegen neue Geheimdienste -- 
Keinen Polizeistaat"'.\footnote{
  taz, 12.09.1990, S.\,5.
}
Seine Landschaften sind weder klare Sehnsuchts-bilder romantischer Ferne noch kultivierte er die sentimentalen Ecken des Spaziergängers. Seine Natur ist windzerzaust, naßgeregnet, aufgewühlt.
12.4.1991 Lavie 96 Zeilen, katrin bettina müller S. 4

die armen Radfahrer, strampeln ohne jede Motorkraft ungeschützt gegen Wind und Wetter, naßgeschwitzt und naßgeregnet.
21.10.1988 taz 337 Zeilen, bernd müllender S. 11

Sie werden erst eingeschneit, dann langsam eingeregnet, 
21.2.1992 taz 248 Zeilen, elke schmitter S. 16
In taz-Berlin S.18 / Europaverlag

Wenn die besagte Küstenstraße leicht angeregnet ist,ist sie unglaublich rutschig,"wie
Schmierseife" ist fast noch eine Untertreibung.
http://www.hpn.de/forum/archiv022004/messages/40624.html



Anfang Februar hatte es in diesem Gebiet 1m bis 2,5 m Neuschnee gegeben, der nachfolgend
mehrmals erwärmt und teilweise bis 1700 m hinauf angeregnet wurde.
http://www.stadler-markus.de/files/skitour/verh1a.htm

Manche Meteorologen behaupten, an Wochenenden sei verstärkt schlechtes Wetter zu beobachten, weil da die vom Berufsverkehr emittierten Partikel abgeregnet würden.
6.9.2003 taz Berlin lokal Berlin Aktuell 32 Zeilen, S. 34


Nicht nur die Bundesregierung hatte sich blamiert, allen voran der CSU-Innenminister, der noch jede Gefahr leugnete, als er längst wusste, dass die radioaktiven Isotope auch über Westeuropa massiv abgeregnet wurden. 7.6.2000 taz Themen des Tages 84 Zeilen, REINER METZGER S. 3


Die radioaktive Wolke habe lange über der Region gehangen und sich abgeregnet.
26.4.1997 taz Wirtschaft und Umwelt 71 Zeilen, kpk S. 7

Der belastete, "saure" Regen wird unmittelbar und ohne weitere Behandlung über eine Sprinkleranlage wieder abgeregnet. 31.8.1993 taz 139 Zeilen, reimar paul S. 14

Eine Strahlenwolke von Tschernobyl wurde offenbar künstlich abgeregnet
16.4.1991 taz 88 Zeilen, manfred kriener S. 8

Doch die Minister in Moskau kennen keine Radioaktivität. Bis dorthin ist die radioaktive Wolke auch nicht gekommen. Sie sei künstlich über der Stadt Gomel abgeregnet worden, behaupten nicht wenige aus der Bürgerinitiative "Kinder von Tschernobyl". 
12.11.1990 taz 435 Zeilen, w. giebel S. 14

der bau dauerte ein jahr und ist übel vollgeregnet. 
http://www.haustechnikdialog.de/forum.asp?fid=37474&forum=5&uebersicht=1

Letzte Nacht hat es den Teich vollgeregnet,http://www.timmendorfer.de/pforum/showthread.php?id=160&eintrag=10



\end{comment}



%Problem: "der geschlagen wordene Mann"  geht

%Da das Hilfsverb unakkusativisch ist, ist das parallel zu
%"der gestorbene Mann", nur eben mit Verbalkomplex.








}% exewidth{(100)}
