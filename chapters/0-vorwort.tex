%% -*- coding:utf-8 -*-

\chapter*{Vorwort}

\begin{sloppypar}
Seit 1994 unterrichte ich die Kopfgesteuerte Phrasenstrukturgrammatik (Head-Driven Phrase
Structure Grammar). In den ersten Jahren habe ich dazu ein Vorlesungsskript verwendet, das
dann in \citew{Mueller99a} eingeflossen ist. Dieses Buch ist allerdings komplex und
für eine Einführung wohl weniger geeignet. Ich habe mich also entschlossen, ein Lehrbuch zu
schreiben, das sich auf das Wesentliche konzentriert und Problemfälle nicht erörtert.
\end{sloppypar}

Das Buch sollte für alle verständlich sein, die mit Wortarten und dem Valenzbegriff vertraut
sind. Nach der Lektüre dieses Buches soll der Leser bzw.\ die Leserin in der Lage sein, aktuelle
HPSG"=Publikationen zu verstehen. Ziel ist es deshalb, die aktuelle Merkmalsgeometrie zu motivieren
und darzustellen, wie zentrale Konzepte wie Valenz und Selektion, Modifikation und lexikalische
Prozesse modelliert werden.

In jedem der Kapitel wird ein bestimmter Aspekt der Theorie behandelt, weshalb die Kapitel teilweise
recht kurz sind. Die Kapitel bestehen aus der Besprechung des jeweiligen Aspekts, eventuell einer
Diskussion alternativer Theorien, einem kurzen Abschnitt mit Kontrollfragen, einem Abschnitt
mit Übungs\-aufgaben und bei einigen Kapiteln Hinweisen zu weiterführender Literatur. Die Diskussion
alternativer Theorien ist für den fortgeschrittenen Leser gedacht, 
dem eine Bewertung der HPSG im gegenwärtigen Forschungsumfeld wichtig ist. Die Argumentation
bezieht sich mitunter auf Konzepte, die in diesem Buch nicht ausführlich dargestellt werden können.
Es gibt dann immer Verweise auf weiterführende Literatur. Für einen ersten Überblick über
die HPSG kann man die Diskussionsabschnitte überspringen und eventuell später beim vertiefenden
Lesen zu einzelnen Abschnitten zurückkehren.

\begin{sloppy}
Zu diesem Buch gehört ein Foliensatz, der unter 
\url{https://hpsg.hu-berlin.de/~stefan/Lehre/HPSG/}
verfügbar ist. Von dieser Seite gelangt man auch zu einer Seite, die computerverarbeitbare
Grammatiken enthält, die den jeweiligen Ka\-pi\-teln in diesem Buch entsprechen. Eine CD, die
alle zum Grammatikentwickeln benötigte Software und auch die Grammatiken enthält, ist unter
\url{https://hpsg.hu-berlin.de/Software/Grammix/} verfügbar. Dem interessierten Leser
wird nahegelegt, sich mit den Grammatiken zu beschäftigen, da das anschaulicher ist,
als es jeder noch so gute Text sein könnte.
\end{sloppy}

\section*{Benutzte Korpora}

Die meisten der Beispiele in diesem Buch sind Belege aus der {\em taz}\footnote{
\url{http://www.taz.de/}}, einer überregionalen
deutschen Tageszeitung. Andere sind aus dem Magazin \emph{Der Spiegel},
aus der Computerzeitschrift c't oder aus der \emph{zitty}, einer kleinen Zeitschrift mit
Veranstaltungshinweisen aus Berlin. 
%% Auch Beispiele aus Romanen oder wissenschaftlichen Texten über
%% Linguistik wurden berücksichtigt. I have also considered examples from novels
%% and some from scientific texts on linguistics. Of course, it is clear to
%% me that the language of linguists changes according to their research topic
%% and according to the theories they have at a certain stage, but in many cases
%% I have quoted examples that show that a claim of the author is wrong and this excludes
%% the possibility that the production of the respective sentences was influenced
%% by the author's theoretical work.


Es ist sehr bequem, elektronische Korpora zu verwenden, um bestimmte Behauptungen zu rechtfertigen
oder zu widerlegen.
Für diese Zwecke habe ich die taz-CD"=Roms benutzt, die über 20 Jahrgänge der Zeitung enthalten.
Außerdem habe ich das \cosmas"=Korpus\footnote{\url{https://cosmas2.ids-mannheim.de/cosmas2-web/}} 
verwendet, das vom Institut für Deutsche Sprache (IDS) Mannheim zur Verfügung gestellt
wird. Die Version, die über das World Wide Web zugänglich ist, enthält 1,489 Milliarden Wörter.
Neben \cosmas habe ich das \negra"=Korpus\footnote{
\url{http://www.coli.uni-sb.de/sfb378/negra-corpus/}}, das
Tiger"=Korpus\footnote{\url{http://www.ims.uni-stuttgart.de/projekte/TIGER/TIGERCorpus/}}
und das Digitale Wörterbuch der deutschen Sprache\footnote{\url{http://www.dwds.de/}} der 
Berlin"=Brandenburgischen Akademie der Wissenschaften benutzt.


\section*{Danksagungen}

Ich danke allen Studierenden der Universitäten Bremen und Potsdam, 
an denen ich das Buch ausprobieren konnte.
Felix Bildhauer,
Johannes Bubenzer, % Uni Potsdam, Mann mit Bart aus Hamburg, Script
Daniel Clerc, % Kommentare zu Übungsaufgaben
Anna Iwanow, % Uni Potsdam, Script
Katarina Klein, 
Till Kolter, 
Oleg Lichtenwald, 
Haitao Liu,
Frank Richter,
Wolfgang Seeker,
Wilko Steffens,
Ralf Vogel
und
Arne Zeschel
danke ich für Kommentare zu früheren Versionen dieses Buches.
%Felix Bildhauer, 
Olivier Bonami, % Situationssemantik
Gosse Bouma, 
Ann Copestake, 
Gisbert Fanselow, 
Kerstin Fischer, 
Dan Flickinger, 
Martin Forst,
Frederik Fouvry,
Tibor Kiss, 
Valia Kordoni, 
Detmar Meurers, 
Ivan Sag, 
Manfred Sailer,       % Idioms
Jan-Philipp Soehn,    % Idioms
Anatol Stefanowitsch,
Gertjan van Noord
und
Shravan Vasishth % Performance/Competence
%und Arne Zeschel 
danke ich für Diskussionen.

Beim Tutorial \emph{Grammar Implementation}, das im Rahmen des von der DFG geförderten Netzwerks
Cogeti (Constraintbasierte Grammatik: Empirie, Theorie und Implementierung) am Seminar für
Computerlinguistik der Universität Heidelberg durchgeführt wurde,
konnte ich noch einige kleinere Fehler in der Grammix"=CD finden. Ich danke allen Teilnehmern des Tutoriums.

Bei Felix Bildhauer und Renate Schmidt möchte ich mich für das Korrekturlesen des fast fertigen
Manuskripts bedanken. Die Fehler, die jetzt noch im Buch enthalten sind, habe ich gestern bei
letzten Änderungen reingemacht.

~\medskip

\noindent
Bremen, 12.\ Februar, 2007\hfill Stefan Müller

\section*{Vorwort zur zweiten Auf"|lage}

Die zweite Auf"|lage unterscheidet sich nur geringfügig von der ersten. Ich habe einige
Literaturverweise ergänzt und einige erklärende Sätze eingefügt. Die Diskussion der
vererbungsbasierten Analyse des Passivs in Abschnitt~\ref{sec-vererbung-koenig} habe ich geändert und
die Beispiele aus dem Yukatekischen\il{Yukatekisch} durch Beispiele aus dem Türkischen\il{Türkisch}
ersetzt. Zur Motivation siehe \citew[\page 387]{Mueller2007d}. Die Analyse von Zeitausdrücken im
Akkusativ wurde entfernt, da man sie nicht -- wie in der ersten Auf"|lage vorgeschlagen -- an Lexikoneinträgen festmachen kann 
(siehe Seite~\pageref{bsp-den-groessten-Teil-der-Woche}).

Ich bedanke mich bei Jürgen Bohnemeyer\aimention{J{\"u}rgen Bohnemeyer} und bei Jacob Maché\aimention{Jacob Mach{\'e}} für Diskussion bzw.\ für Kommentare zur
ersten Auf"|lage.

~\medskip

\noindent
Berlin, 01.\ August, 2008\hfill Stefan Müller

\section*{Vorwort zur dritten Auf"|lage}

Die wichtigste Änderung von der zweiten zur dritten Auf"|lage besteht in einer Anpassung des
Kopf"=Argument"=Schemas. Ich verwende jetzt nicht mehr die \emph{del}"=Relation sondern
\emph{append} \citep[Abschnitt~8.4]{MuellerGTBuch1}. Das ermöglicht eine einfache sprachübergreifende Analyse der Konstituentenstellung
\citep{MuellerCopula}. Die entsprechenden Analysen haben sich im CoreGram"=Projekt\footnote{
  \url{http://hpsg.fu-berlin.de/Projects/CoreGram.html}. Zu einer Beschreibung des Projekts siehe \citew{MuellerCoreGram}. Das CoreGram"=Projekt beschäftigt sich mit der Implementation von Grammatiken für so
verschiedene Sprachen wie Deutsch, Dänisch\il{Dänisch}
\citep{MOe2011a,MuellerPredication,MuellerCopula,MOeDanish}, Englisch\il{Englisch}
\citep{MuellerPredication,MuellerCopula}, Jiddisch\il{Jiddisch} \citep{MOe2011a}, Spanisch\il{Spanisch}, Französisch\il{Französisch},
Maltesisch\il{Maltesisch} \citep{MuellerMalteseSketch},
Persisch\il{Persisch} \citep{MuellerPersian} und Mandarin Chinesisch\il{Mandarin Chinesisch} \citep{Lipenkova2009a,ML2009a}.
}, bewährt. 

Wir haben uns entschlossen, die CD-Rom, die bisher mit dem Buch zusammen ausgeliefert wurde, nicht
mehr gemeinsam mit dem Buch zu vertreiben. Die Grammaix-CD \citep{Mueller2007b} ist im Netz verfügbar\footnote{
  \url{https://hpsg.hu-berlin.de/Software/Grammix/}
} und die Netzverbindungen sind inzwischen so gut, dass man sich CDs herunterladen kann. Denjenigen,
die über keine ausreichend schnelle Internetanbindung verfügen, schicken wir gern eine CD zu (bitte
mein Sekretariat kontaktieren). 
Die Änderung des Kopf"=Argument"=Schemas ist auf der Grammix"=CD in der gegenwärtig auf der
Web-Seite verfügbaren Version noch nicht enthalten. Die Grammix"=CD
wird komplett umgestaltet: Es wird ein neues Betriebssystem geben, eine neue wesentlich verbesserte
und effizientere Version von TRALE und die Grammatiken aus dem CoreGram"=Projekt. Ich hoffe, die
Arbeiten an der neue Version der CD noch vor der Fertigstellung des Berliner Flughafens abschließen
zu können. Die angepassten Lehrbuchgrammatiken und die Grammatiken aus dem CoreGram"=Projekt sind
aber bereits jetzt über meine Webseite zugänglich.

Der Abschnitt~\ref{sec-vererbung-koenig} zur vererbungsbasierten Analyse des Passivs wurde nochmals
überarbeitet und enthält jetzt auch eine Diskussion von Doppeltpassivierungen im Türkischen\il{Türkisch} und
anderen Sprachen. Außerdem habe ich im Abschnitt~\ref{sec-spr} etwas erklärenden Text und eine
Abbildung hinzugefügt. Die Diskussion des \textsc{mother}"=Merkmals im Kapitel~\ref{chap-lokalitaet} wurde erweitert und auf
den Stand von \citew{MuellerGTBuch1} bzw.\ \citew{MuellerGTBuch2} gebracht. 
Ein Fehler in den Schemata~\ref{schema-bin-mark} und~\ref{schema-bin-mark-final} wurde korrigiert.

Es gibt jetzt einen Anhang mit Lösungen zu ausgewählten Übungsaufgaben.

Ich bedanke mich bei Antonio Machicao y Priemer\aimention{Antonio {Machicao y Priemer}} für Kommentare zur zweiten Auf"|lage.

~\medskip

\noindent
Berlin, 23. Januar 2013\hfill Stefan Müller


\section*{Vorwort zur vierten Auf"|lage}

% Timm Lichte: Falsches Verb in Erklärungen für Passivbeispiel.
% Tibor Kiss 6.5.2014: Merkmalstruktur statt beschreibung, Symbole in PSGen.
% Pied Piping gegen PSGen? To Do.
Ich bedanke mich bei Tibor Kiss und Timm Lichte\aimention{Timm Lichte} für Kommentare zur dritten Auf"|lage.

% ein Komma und Optionalität bei Valenz von denken
Bei Anja Herrmann bedanke ich mich für den Hinweis auf Typos und Fragen zu den Übungsaufgaben.

% 23.05.2022 Im Kausus-Kapitel Verweis auf 2 statt auf 1. Karla Jerabeck hats gefunden.
% 19) a. Er goß ihr die Blumen.
%     b. ErzündeteihrdasHausan.
% Wegener (1985a) hat jedoch gezeigt, daß diese Dative Komplementstatus haben.
%
% Erklären, wieso Komplementstatus.

% To do:
%
% Abbildung 4.5 ist nicht zentiriert

% Sophie Reule, Studentin 22.01.2018
% Auf S. 74 des HPSG-Lehrbuchs werden die Elemente der Liste Phon durch Kommata getrennt, zuvor war
% dies nicht der Fall. Welche Bedeutung hat die Notation an dieser Stelle? 

% 
% Auf Seite 75 wird vom SYNSEM-Wert gesprochen, der wurde aber gar nicht eingeführt.

% Wert von PSOA-ARG ist eine Liste, also muss Mörder in eine Liste in (21) Kapitel 6.4

% 17.05.18 Auf S. 282 war ein "ist" doppelt.

% 28.08.2018 Steve Wechsler: Kongruenzmerkmale des Nomens müssen Kopfmerkmale sein,
% denn wenn es Depiktive gibt, die mit Nomina kongruieren, müssen deren CONCORD-Merkmale nach außen
% verfügbar sein.

%On p.237 of your HPSG Lehrbuch (3rd edition) you're saying:

%% Przepiórkowski (1999a) integriert deshalb Meurers Ansatz der Kasuszuweisung (1999b;
%% 2000, Kapitel 10.4.1.4) in seine Theorie und führt zusätzlich zum REALIZED -Merkmal das
%% Merkmal RAISED ein, das den Wert + hat, wenn ein Argument angehoben wird, und den
%% Wert −, wenn das nicht der Fall ist.

%% The following part is false:

%% führt zusätzlich zum REALIZED -Merkmal das Merkmal RAISED ein

%% Rather, RAISED replaces REALIZED.

%% Please, remember when you prepare the 4th edition ;-)

%% BTW, I expect to send you the chapter on case on 31 July.  But just in case: do you think I can ask for a 3-day deadline extension (without losing the right to the free volumes)?  This might drastically increase the quality of my submission ;-)

~\medskip

\noindent
Berlin, \today\hfill Stefan Müller