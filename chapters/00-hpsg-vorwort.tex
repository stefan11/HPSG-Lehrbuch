%% -*- coding:utf-8 -*-

\chapter*{Vorwort}

\begin{sloppypar}
Seit 1994 unterrichte ich die Kopfgesteuerte Phrasenstrukturgrammatik (Head-Driven Phrase
Structure Grammar). In den ersten Jahren habe ich dazu ein Vorlesungsskript verwendet, das
dann in \citew{Mueller99a} eingeflossen ist. Dieses Buch ist allerdings komplex und
für eine Einführung wohl weniger geeignet. Ich habe mich also entschlossen, ein Lehrbuch zu
schreiben, das sich auf das Wesentliche konzentriert und Problemfälle nicht erörtert.
\end{sloppypar}

Das Buch sollte für alle verständlich sein, die mit Wortarten und dem Valenzbegriff vertraut
sind. Nach der Lektüre dieses Buches soll der Leser bzw.\ die Leserin in der Lage sein, aktuelle
HPSG"=Publikationen zu verstehen. Ziel ist es deshalb, die aktuelle Merkmalsgeometrie zu motivieren
und darzustellen, wie zentrale Konzepte wie Valenz und Selektion, Modifikation und lexikalische
Prozesse modelliert werden.

In jedem der Kapitel wird ein bestimmter Aspekt der Theorie behandelt, weshalb die Kapitel teilweise
recht kurz sind. Die Kapitel bestehen aus der Besprechung des jeweiligen Aspekts, eventuell einer
Diskussion alternativer Theorien, einem kurzen Abschnitt mit Kontrollfragen, einem Abschnitt
mit Übungsaufgaben und bei einigen Kapiteln Hinweisen zu weiterführender Literatur. Die Diskussion
alternativer Theorien ist für den fortgeschrittenen Leser gedacht, 
dem eine Bewertung der HPSG im gegenwärtigen Forschungsumfeld wichtig ist. Die Argumentation
bezieht sich mitunter auf Konzepte, die in diesem Buch nicht ausführlich dargestellt werden können.
Es gibt dann immer Verweise auf weiterführende Literatur. Für einen ersten Überblick über
die HPSG kann man die Diskussionsabschnitte überspringen und eventuell später beim vertiefenden
Lesen zu einzelnen Abschnitten zurückkehren.

\begin{sloppy}
Zu diesem Buch gehört ein Foliensatz, der unter 
\url{https://hpsg.hu-berlin.de/~stefan/Lehre/HPSG/}
verfügbar ist. Von dieser Seite gelangt man auch zu einer Seite, die computerverarbeitbare
Grammatiken enthält, die den jeweiligen Ka\-pi\-teln in diesem Buch entsprechen. Eine CD, die
alle zum Grammatikentwickeln benötigte Software und auch die Grammatiken enthält, ist unter
\url{https://hpsg.hu-berlin.de/Software/Grammix/} verfügbar. Dem interessierten Leser
wird nahegelegt, sich mit den Grammatiken zu beschäftigen, da das anschaulicher ist,
als es jeder noch so gute Text sein könnte.
\end{sloppy}

\section*{Benutzte Korpora}

Die meisten der Beispiele in diesem Buch sind Belege aus der {\em taz}\footnote{
\url{http://www.taz.de/}, 21.07.2024.}, einer überregionalen
deutschen Tageszeitung. Andere sind aus dem Magazin \emph{Der Spiegel},
aus der Computerzeitschrift c't oder aus der \emph{zitty}, einer kleinen Zeitschrift mit
Veranstaltungshinweisen aus Berlin. 
%% Auch Beispiele aus Romanen oder wissenschaftlichen Texten über
%% Linguistik wurden berücksichtigt. I have also considered examples from novels
%% and some from scientific texts on linguistics. Of course, it is clear to
%% me that the language of linguists changes according to their research topic
%% and according to the theories they have at a certain stage, but in many cases
%% I have quoted examples that show that a claim of the author is wrong and this excludes
%% the possibility that the production of the respective sentences was influenced
%% by the author's theoretical work.


Es ist sehr bequem, elektronische Korpora zu verwenden, um bestimmte Behauptungen zu rechtfertigen
oder zu widerlegen.
Für diese Zwecke habe ich die taz-CD"=Roms benutzt, die über 20 Jahrgänge der Zeitung enthalten.
Außerdem habe ich das \cosmas"=Korpus\footnote{\url{https://cosmas2.ids-mannheim.de/cosmas2-web/}, 21.07.2024.} 
verwendet, das vom Institut für Deutsche Sprache (IDS) Mannheim zur Verfügung gestellt
wird. Die Version, die über das World Wide Web zugänglich ist, enthält 1,489 Milliarden Wörter.
Neben \cosmas habe ich das \negra{}\hyp Korpus der Universität Saarbrücken,
% \footnote{
% \url{http://www.coli.uni-sb.de/sfb378/negra-corpus/}}
das Tiger"=Korpus des Instituts für maschinelle Sprachverarbeitung der Universität Stuttgart,
%\footnote{\url{http://www.ims.uni-stuttgart.de/projekte/TIGER/TIGERCorpus/}}
und das Digitale Wörterbuch der deutschen Sprache\footnote{\url{http://www.dwds.de/}, 21.07.2024.} der 
Berlin"=Brandenburgischen Akademie der Wissenschaften benutzt.

~\medskip

\noindent
Bremen, 12.\ Februar, 2007\hfill Stefan Müller

\section*{Vorwort zur zweiten Auf"|lage}

Die zweite Auf"|lage unterscheidet sich nur geringfügig von der ersten. Ich habe einige
Literaturverweise ergänzt und einige erklärende Sätze eingefügt. Die Diskussion der
vererbungsbasierten Analyse des Passivs in Abschnitt~\ref{sec-vererbung-koenig} habe ich geändert und
die Beispiele aus dem Yukatekischen\il{Yukatekisch} durch Beispiele aus dem Türkischen\il{Türkisch}
ersetzt. Zur Motivation siehe \citew[\page 387]{Mueller2007d}. Die Analyse von Zeitausdrücken im
Akkusativ wurde entfernt, da man sie nicht -- wie in der ersten Auf"|lage vorgeschlagen -- an Lexikoneinträgen festmachen kann 
(siehe Seite~\pageref{bsp-den-groessten-Teil-der-Woche}).


~\medskip

\noindent
Berlin, 01.\ August, 2008\hfill Stefan Müller

\section*{Vorwort zur dritten Auf"|lage}

Die wichtigste Änderung von der zweiten zur dritten Auf"|lage besteht in einer Anpassung des
Kopf"=Argument"=Schemas. Ich verwende jetzt nicht mehr die \emph{del}"=Relation sondern
\emph{append} \citep[Abschnitt~8.4]{MuellerGTBuch1}. Das ermöglicht eine einfache sprachübergreifende Analyse der Konstituentenstellung
\citep{MuellerCopula}. Die entsprechenden Analysen haben sich im CoreGram"=Projekt\footnote{
  \url{https://hpsg.hu-berlin.de/Projects/CoreGram.html}. Zu einer Beschreibung des Projekts siehe \citew{MuellerCoreGram}. Das CoreGram"=Projekt beschäftigt sich mit der Implementation von Grammatiken für so
verschiedene Sprachen wie Deutsch, Dänisch\il{Dänisch}
\citep{MOe2011a,MuellerPredication,MuellerCopula,MOeDanish}, Englisch\il{Englisch}
\citep{MuellerPredication,MuellerCopula}, Jiddisch\il{Jiddisch} \citep{MOe2011a}, Spanisch\il{Spanisch}, Französisch\il{Französisch},
Maltesisch\il{Maltesisch} \citep{MuellerMalteseSketch},
Persisch\il{Persisch} \citep{MuellerPersian} und Mandarin Chinesisch\il{Mandarin Chinesisch} \citep{Lipenkova2009a,ML2009a}.
}, bewährt. 

Wir haben uns entschlossen, die CD-Rom, die bisher mit dem Buch zusammen ausgeliefert wurde, nicht
mehr gemeinsam mit dem Buch zu vertreiben. Die Grammaix-CD \citep{Mueller2007b} ist im Netz verfügbar\footnote{
  \url{https://hpsg.hu-berlin.de/Software/Grammix/}
} und die Netzverbindungen sind inzwischen so gut, dass man sich CDs herunterladen kann. Denjenigen,
die über keine ausreichend schnelle Internetanbindung verfügen, schicken wir gern eine CD zu (bitte
mein Sekretariat kontaktieren). 
Die Änderung des Kopf"=Argument"=Schemas ist auf der Grammix"=CD in der gegenwärtig auf der
Web-Seite verfügbaren Version noch nicht enthalten. Die Grammix"=CD
wird komplett umgestaltet: Es wird ein neues Betriebssystem geben, eine neue wesentlich verbesserte
und effizientere Version von TRALE und die Grammatiken aus dem CoreGram"=Projekt. Ich hoffe, die
Arbeiten an der neue Version der CD noch vor der Fertigstellung des Berliner Flughafens abschließen
zu können.\footnote{Fußnote aus der Zukunft: Bis zur Eröffnung des BER am 31.10.2020 war noch
  ausreichend Zeit. 2013, also sieben Jahre vor der Eröffnung des BER, habe ich eine Virtuelle
  Maschine mit Grammix zum Download bereitgestellt.} Die angepassten Lehrbuchgrammatiken und die Grammatiken aus dem CoreGram"=Projekt sind
aber bereits jetzt über meine Webseite zugänglich.

Der Abschnitt~\ref{sec-vererbung-koenig} zur vererbungsbasierten Analyse des Passivs wurde nochmals
überarbeitet und enthält jetzt auch eine Diskussion von Doppeltpassivierungen im Türkischen\il{Türkisch} und
anderen Sprachen. Außerdem habe ich im Abschnitt~\ref{sec-spr} etwas erklärenden Text und eine
Abbildung hinzugefügt. Die Diskussion des \textsc{mother}"=Merkmals im Kapitel~\ref{Kapitel-Lokalität} wurde erweitert und auf
den Stand von \citew{MuellerGTBuch1} bzw.\ \citew{MuellerGTBuch2} gebracht. 
Ein Fehler in den Schemata~\ref{schema-bin-mark} und~\ref{schema-bin-mark-final} wurde korrigiert.

Es gibt jetzt einen Anhang mit Lösungen zu ausgewählten Übungsaufgaben.


~\medskip

\noindent
Berlin, 23. Januar 2013\hfill Stefan Müller


\section*{Vorwort zur vierten Auf"|lage}

Dieses Buch hat eine lange Geschichte. Es ist in den Jahren 2005–2006 aus Folien entstanden, die ich
ab 1994 für die Lehre benutzt habe. Meine Dissertation habe ich 1999 bei Niemeyer
veröffentlicht. Das Buch war bis zur Veröffentlichung auf meiner Webseite verfügbar. Danach musste
ich es von der Webseite entfernen. Die Veröffentlichung war also eine Entöffentlichung. Der Preis
war mit 186 DM für 486 Seiten prohibitiv hoch: Privatpersonen dürften sich das Buch höchst selten
gekauft haben. Inzwischen wurde Niemeyer von De Gruyter geschluckt und De Gruyter bietet das Buch für 150€
an. Das Buch habe ich selbst geschrieben und komplett selbst gesetzt. Mit meinem neuen Buch sollte das nicht passieren,
weshalb ich vorhatte, es bei einem unbekannten Print-On-Demand-Verlag zu veröffentlichen. Auf einer
Konferenz sprach ich mit Brigitte Narr über meine Pläne und sie sagte: "`Machen Sie das
nicht!"'. Wir kamen überein, dass ich das Buch im Stauffenburg-Verlag veröffentliche und es
gleichzeitig auf meiner Web-Seite lassen kann. Ich habe Frau Narrs Kontonummer auf meiner Web-Seite
angegeben und weiß von einigen berühmten Grammatikern, dass sie Frau Narr auch Geld überwiesen
haben. (Andere Grammatiker*innen waren so freundlich Frau Narr darauf hinzuweisen, dass ich das Buch
kostenlos zum Download auf meiner Web-Seite anbiete \ldots). 

Ich habe dann auch mein Grammatiktheorie"=Buch im Stauffenburg"=Verlag veröffentlicht. Ab 2012 habe
ich gemeinsam mit Martin Haspelmath und mit der Unterstützung von vielen, vielen Linguist*innen
weltweit (u.\,a.\ Chomsky, Pinker, Goldberg) begonnen, einen wissenschaftsgeführten Diamond Open
Access-Verlag aufzubauen \citep{MuellerOA}.\footnote{Diamond Open Access bedeutet, dass weder die Autor*innen noch die
Leser*innen für die Veröffentlichungen bezahlen müssen.} Seit 2014 gibt es Language Science Press. Ab diesem Zeitpunkt habe ich
auch meine Bücher dort veröffentlicht. 

Im April 2024 habe ich Frau Narr
gefragt, ob ich die Rechte an diesem Buch zurückbekommen könnte, und sie hat eingewilligt. Das Buch
ist jetzt also wie fast alle meine Bücher frei und ich kann es unter der CC-BY-Lizenz bei Language
Science Press veröffentlichen.

Seit 2013 hat sich einiges getan. Ich habe mich weiter mit sprachübergreifenden Fragestellungen
beschäftigt und ein Buch über germanische Sprachen veröffentlicht \citep{MuellerGermanic}. Dort habe
ich alle germanischen Sprachen so analysiert, dass die Verben in den jeweiligen Sprachen dieselbe
Argumentstruktur haben. Zum Beispiel sind die Gegenstücke zum deutschen \emph{geben} ebenfalls
dreistellig. Die Argumentstrukturen unterscheiden sich lediglich in den Kasus: Das Englische hat keinen
Dativ; die Sprachen haben unterschiedliche Verteilungen von strukturellem und lexikalischem
Kasus. Ich nehme, wie in neueren Arbeiten üblich, eine Argumentstruktur"=Liste (\argst) an, auf der 
alle Argumente eines Kopfes aufgelistet sind. Die Argumente werden von dieser Repräsentation auf
Valenzmerkmale gemappt. Diese sind \spr für Spezifikatoren und \comps für Komplemente. Die
Reihenfolge der drei Argumente in der \argst ist für alle germanischen Sprachen gleich. Sie
entspricht der Normalstellung im Deutschen (Nominativ, Dativ, Akkusativ) und der Stellung im
Englischen:
\eal
\ex dass er dem Affen den Stock gibt
\ex that he gives the monkey the stick
\zl
Das ist ein Unterschied zu früheren Auf"|lagen des Buches, in denen die Elemente in der Valenzliste
\subcat in der Reihenfolge Nominativ, Akkusativ, Dativ angeordnet waren. Die Einführung in
Phrasenstrukturgrammatiken und die Analyse der Linearisierungstheorien in Abschnitt~\ref{sec-konstituentenreihenfolge-alternativen} wurden an
diese veränderte Grundreihenfolge angepasst.

Das Semantik"=Kapitel wurde komplett überarbeitet. Statt der Situationssematik, die in den ersten
drei Auf\/lagen des Buches verwendet wurde, gibt es im Kapitel~\ref{Kapitel-Semantik} nun eine
Einführung in die Minimal Recursion Semantics \citep*{CFPS2005a}.

Ansonsten habe ich das Buch jetzt auf neue Rechtschreibung umgestellt, ich habe begonnen zu gendern
und habe die meisten Beispiele so abgeändert, dass es keine Gender-Stereotypen mehr gibt
\citep{MB97a,PCKSDMC2017a}. Für das Deutsche ist das leider nicht so einfach, weil man auf die
Verwendung von Maskulina angewiesen ist, denn nur bei diesen ist der Kasus voll ausbuchstabiert. 

Ich habe an Stellen, wo das noch nicht der Fall war, bei Literaturquellen Seitenzahlen hinzugefügt.

% ein Komma und Optionalität bei Valenz von denken
% Bei Anja Herrmann bedanke ich mich für den Hinweis auf Typos und Fragen zu den Übungsaufgaben.

% 23.05.2022 Im Kausus-Kapitel Verweis auf 2 statt auf 1. Karla Jerabeck hats gefunden.
% 19) a. Er goß ihr die Blumen.
%     b. Er zündete ihr das Haus an.
% Wegener (1985a) hat jedoch gezeigt, dass diese Dative Komplementstatus haben.
%
% Erklären, wieso Komplementstatus.

% Lexikon Typhierarchie statt acc dat. 03.06.2024

% Den Kindern wurde geholfen -> Den Kindern wird geholfen.
% Beispiele wurden fehlerhaft Haider zugeschrieben. 07.06.2024

% 11.06.2024
% Bei Lösungen wurde der Numerus-Wert bei der Adjektivlexikonregel entfernt. Dieser
% wird von der Endung bestimmt, die beim attributiven Adjektiv angehängt wird. Kasus und Genus
% wurden in der Aufgabenstellung wieder eingebaut, aber mit Anmerkung versehen, dass sie weggelassen
% werden sollen.

%21.06.2024 

% Reihenfolge im Abschnitt über Linearisierungstheorien wurde an nom, dat, acc angepasst.

% To do:
%
% Sophie Reule, Studentin 22.01.2018
% Auf S. 74 des HPSG-Lehrbuchs werden die Elemente der Liste Phon durch Kommata getrennt, zuvor war
% dies nicht der Fall. Welche Bedeutung hat die Notation an dieser Stelle? 

% 
% Auf Seite 75 wird vom SYNSEM-Wert gesprochen, der wurde aber gar nicht eingeführt.

% Wert von PSOA-ARG ist eine Liste, also muss Mörder in eine Liste in (21) Kapitel 6.4

% 17.05.18 Auf S. 282 war ein "ist" doppelt.

% 28.08.2018 Steve Wechsler: Kongruenzmerkmale des Nomens müssen Kopfmerkmale sein,
% denn wenn es Depiktive gibt, die mit Nomina kongruieren, müssen deren CONCORD-Merkmale nach außen
% verfügbar sein.

%On p.237 of your HPSG Lehrbuch (3rd edition) you're saying:

%% Przepiórkowski (1999a) integriert deshalb Meurers Ansatz der Kasuszuweisung (1999b;
%% 2000, Kapitel 10.4.1.4) in seine Theorie und führt zusätzlich zum REALIZED -Merkmal das
%% Merkmal RAISED ein, das den Wert + hat, wenn ein Argument angehoben wird, und den
%% Wert −, wenn das nicht der Fall ist.

%% The following part is false:

%% führt zusätzlich zum REALIZED -Merkmal das Merkmal RAISED ein

%% Rather, RAISED replaces REALIZED.

%% Please, remember when you prepare the 4th edition ;-)

% 03.05.2025 
% Oliva92b statt Oliva92a zitiert. COLING-Paper ist verfügbar, CLAUS-Report nicht mehr


~\medskip

\noindent
Berlin, \today\hfill Stefan Müller