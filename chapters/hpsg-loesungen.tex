%% -*- coding:utf-8 -*-

\chapter{Lösungen zu den Übungsaufgaben}

\section{Einleitung}

\begin{enumerate}
% Auf Seite~\pageref{page-unendlich-viele-grammatiken} habe ich behauptet, daß
%      es unendlich viele Grammatiken gibt, die (\ref{bsp-weil-er-das-buch-dem-mann-gibt})
%      analysieren können.
%      Überlegen Sie sich, wieso diese Behauptung richtig ist.
\item Man kann zu jeder Grammatik zusätzliche Symbole und Regeln annehmen, die unnötige Struktur
  erzeugen oder einfach nie angewendet werden, weil es keine Wörter oder Phrasen gibt, die in der
  rechten Regelseite verwendet werden können. Wenn wir unserer Grammatik zum Beispiel die folgende
  Regel hinzufügen, haben wir eine komplexere Grammatik, die allerdings denselben Sprachausschnitt
  analysieren kann.
\ea
Trallala $\to$ Trulla Trollolo
\z
% \item Stellen Sie Überlegungen dazu an, wie man ermitteln kann, welche der unendlich vielen Grammatiken
%       die beste ist bzw.\ welche der Grammatiken die besten sind.
\item Allgemein geht man davon aus, dass die Grammatik mit den wenigsten Regeln die beste ist. Somit
  kann man Grammatiken, die unnötige Regeln wie (\mex{0}) enthalten, ablehnen.

Allerdings muss man im Auge behalten, was das Ziel der Grammatiktheorie ist. Setzt man sich zum Ziel
unser menschliches Sprachvermögen zu beschreiben, so kann eine Grammatik mit mehr Regeln die bessere
sein. Das liegt daran, dass psycholinguistische Forschungen gezeigt haben, dass hochfrequente
Einheiten einfach als ganze in unseren Köpfen abgespeichert sind und nicht jedes mal neu aus
Einzelteilen aufgebaut werden, obwohl wir natürlich prinzipiell in der Lage dazu wären.\NOTE{Quelle
  und Beispiel}

Damit die letzten beiden Sätze ausgeschlossen sind, muss die Grammatik Kasusinformation
enthalten. Die folgende Grammatik leistet das Gewünschte:
\eal
\ex s $\to$ np(nom), v(nom\_dat), np(dat)
\ex s $\to$ np(nom), v(nom\_dat\_akk), np(dat), np(akk)
\ex s $\to$ np(nom), v(nom\_pp\_auf), pp(auf,akk)
%
\ex pp(Pform,Kas) $\to$ p(Pform,Kas), np(Kas)
\ex np(Kas) $\to$ d(Kas), n(Kas)
%
\ex v(nom\_dat) $\to$ hilft
\ex v(nom\_dat\_akk) $\to$ gibt
\ex v(nom\_pp\_auf) $\to$ wartet
%
\ex np(nom) $\to$ er
\ex np(dat) $\to$ ihr
%
\ex d(nom) $\to$ der
\ex d(dat) $\to$ der
\ex d(akk) $\to$ das
\ex d(akk) $\to$ ein
%
\ex n(nom) $\to$ Mann
\ex n(dat) $\to$ Frau
\ex n(akk) $\to$ Buch
\ex n(akk) $\to$ Wunder
%
\ex p(auf,akk) $\to$ auf
\zl



\end{enumerate}


\section{Der Formalismus}

\begin{enumerate}
\item Listen\is{Liste} kann man mittels rekursiver Strukturen beschreiben, die jeweils aus einem Listenanfang
  und einem Rest bestehen. Der Rest kann entweder eine nicht leere Liste (\type{ne\_list}) oder die
  leere Liste (\type{e\_list}) sein. Die Liste \relliste{ \type{a}, \type{b}, \type{c} } kann man wie folgt darstellen:
\ea
\ms[ne\_list]{
first & a\\
rest  & \ms[ne\_list]{
        first & b\\
        rest  & \ms[ne\_list]{
                first & c\\
                rest & e\_list\\
                }\\
        }\\
}
\z
\item Erweitert man die Datenstruktur in (\mex{0}) um zwei weitere Merkmale, kann man auf
  \emph{append} verzichten. Das Stichwort heißt \emph{Differenzliste}\is{Liste!Differenz-}. Eine Differenzliste besteht
  aus einer Liste und einem Zeiger auf das letzte Element der Liste. 
\ea
\ms[diff-list]{
list & \ms[ne\_list]{
       first & a\\
       rest  & \ms[ne\_list]{
               first & b\\
               rest  & \ibox{1} list\\
               }\\
       } \\
last & \ibox{1}\\
}
\z

Im Gegensatz zur
  Listenrepräsentation in (\mex{-1}) ist jedoch der \textsc{rest}"=Wert des Listenendes nicht \type{e\_list}
  sondern einfach \type{list}. Es ist damit möglich, eine Liste zu verlängern, indem man an die
  Stelle, an der sie endet, eine andere Liste anfügt. Die Verkettung von (\mex{0}) und (\mex{1}a) ist (\mex{1}b).
\eal
\ex 
\ms[diff-list]{
list & \ms[ne\_list]{
       first & c\\
       rest  & \ibox{2} list\\
       } \\
last & \ibox{2}\\
}
\ex
\ms[diff-list]{
list & \ms[ne\_list]{
       first & a\\
       rest  & \ms[ne\_list]{
               first & b\\
               rest  & \ms[ne\_list]{
                       first & c\\
                       rest  & \ibox{2} list\\
                       }\\
               }\\
       } \\
last & \ibox{2}\\
}
\zl
Für die Verkettung der Listen muss lediglich der \textsc{list}"=Wert der zweiten Liste mit dem \textsc{last}"=Wert der ersten Liste identifiziert werden. Der \textsc{last}"=Wert der Ergebnisliste
entspricht dem \textsc{last}"=Wert der zweiten Liste (im Beispiel \ibox{2}).

Information zur Differenzlistenkodierung lässt sich \zb mit einer Suchanfrage mit den Stichwörtern \emph{list} \emph{append} und
\emph{feature structure} finden. Bei dieser Suchanfrage bekommt man Grammatikentwicklungsseiten, die
Differenzlisten erklären.
\end{enumerate}



\section{Valenz und Grammatikregeln}

      \eal
      \ex er: \sliste{}
      \ex seine (in \emph{seine Ankündigung}): \sliste{}
      \ex schnarcht: \sliste{ NP[\type{nom}] } 
      \ex denkt: \sliste{ NP[\type{nom}], (PP[\type{an},\type{akk}]) } $\vee$ \sliste{
        NP[\type{nom}], (CP[\type{dass}]) }
% wenn das semantisch zwei verschiedene denken sind, dann sollten beide Objekte optional sein.
      \ex graut: \sliste{ (NP[\type{es}]), NP[\type{dat}], (PP[\type{vor}, \type{dat}]) }
      \zl


\section{Kopf-Argument-Strukturen}

\begin{enumerate}
\item Zeichnen Sie einen Syntaxbaum für (\mex{1}):
\ea
daß der Mann das spannende Buch liest
\z
Markieren Sie die Kanten im Baum mit Ad für Adjunkt, Ar für Argument und
H für Kopf.

Die Struktur von (\mex{1}) ist in Abbildung~\ref{fig-dass-der-mann} zu sehen. Dabei steht \nbar für
ein Nomen, das einen Determinator verlangt (siehe Seite~\pageref{ex-abkuerzung-nbar}), V steht für ein lexikalisches
Verb, \vbar für eine ungesättigte und VP für eine gesättigte Verbalprojektion.
\begin{figure}[htbp]
\centerline{%
\begin{tabular}{ccccccc}
\multicolumn{4}{c}{\node{cp}{CP}}\\
\\
\multicolumn{6}{l}{\hspace{0.4cm}H\hspace{2.2cm}Ar}\\[2ex]
\node{c}{C}   & \multicolumn{4}{c}{\node{vp}{VP}}\\ 
\\
\multicolumn{6}{l}{\hspace{1.7cm}Ar\hspace{3cm}H}\\[2ex]
%
              & \multicolumn{2}{c}{\node{np1}{NP}}  & \multicolumn{4}{c}{\node{v1}{\vbar}}\\
\\
\multicolumn{7}{l}{\hspace{0.8cm}Ar\hspace{1.1cm}H \hspace{1.6cm}Ar\hspace{2.5cm}H}\\[2ex]
%
              & \node{det1}{Det} & \node{n1}{\nbar} & \multicolumn{2}{c}{\node{np2}{NP}}                        & & \node{v2}{V}\\
\\
\multicolumn{7}{l}{\hspace{3cm}Ar\hspace{1.9cm}H}\\[2ex]
%
              &                  &                  & \node{det2}{Det} & \multicolumn{2}{c}{\node{ns2}{\nbar}}\\
\\
\multicolumn{7}{l}{\hspace{4.3cm}Ad\hspace{1.6cm}H}\\[2ex]
%
              &                  &                  &                  & \node{adj}{Adj}       & \node{n2}{\nbar} \\
\\[3ex]
 \node{dass}{daß} & \node{der}{der} & \node{mann}{Mann} & \node{das}{das} & \node{sp}{spannende} & \node{buch}{Buch} & \node{liest}{liest}\\
\end{tabular}
\nodeconnect{cp}{c}\nodeconnect{cp}{vp}%
\nodeconnect{c}{dass}%
\nodeconnect{vp}{np1}\nodeconnect{vp}{v1}%
\nodeconnect{np1}{det1}\nodeconnect{np1}{n1}%
\nodeconnect{v1}{np2}\nodeconnect{v1}{v2}%
\nodeconnect{np2}{det2}\nodeconnect{np2}{ns2}%
\nodeconnect{ns2}{adj}\nodeconnect{ns2}{n2}%
\nodeconnect{det1}{der}\nodeconnect{n1}{mann}%
\nodeconnect{det2}{das}\nodeconnect{adj}{sp}%
\nodeconnect{n2}{buch}\nodeconnect{v2}{liest}%
}
\caption{\label{fig-dass-der-mann}Syntaxstruktur von \emph{daß der Mann das spannende Buch liest}}
\end{figure}
\item 

\eanoraggedright
\onems[head-argument-phrase~]{
      phon  \phonliste{ das Kind schläft }\\
      head  \ibox{1}\\
      subcat \ibox{2} \sliste{ } \\
      head-dtr \onems[word]{ phon \phonliste{ schläft }\\
                                     head \ibox{1} \ms[verb]{vform & fin \\
                                                            }\\
                                     subcat \ibox{2}  $\oplus$ \sliste{ \ibox{3} } \\
                       } \\
      non-head-dtrs \liste{ \ibox{3} \onems[head-argument-phrase~]{ 
                                        phon \phonliste{ das Kind }\\
                                        head \ibox{4} \ms[noun]{ cas & \ibox{5} nom\\
                                                      } \\
                                        subcat   \eliste \\
                                        head-dtr \onems[word]{ 
                                                  phon \phonliste{ Kind }\\
                                                  head \ibox{4} \\
                                                  subcat   \sliste{ \ibox{6} } \\
                                                  }\\
                                        non-head-dtrs \liste{ \ibox{6} \onems[word]{ 
                                                                phon \phonliste{ das }\\
                                                                head \ms[det]{
                                                                      cas & \ibox{5} \\ 
                                                                      }\\
                                                                subcat \eliste{ } \\
                                                  } }\\
                                     }
                       }\\
}
\z
\end{enumerate}


\section{Semantik}

\begin{enumerate}

\item 
%% Wie kann man den semantischen Beitrag von \emph{lacht} repräsentieren? Geben Sie den
%%   Lexikoneintrag für \emph{lacht} an und berücksichtigen Sie das Linking.



\ea
\onems[head-argument-phrase~]{
      phon  \phonliste{ er lacht }\\
      cat \ms{ head  \ibox{1}\\
               subcat \ibox{2} \sliste{ } \\
             }\\
      cont \ibox{3}\\
      head-dtr \ms[word]{ phon & \phonliste{ lacht }\\
                          cat & \ms{ head \ibox{1} \ms[verb]{vform & fin \\
                                                            }\\
                                     subcat \ibox{2}  $\oplus$ \sliste{ \ibox{4} } \\
                                   }\\
                          cont & \ibox{3} \ms[lachen]{ agens & \ibox{5}\\
                                          }\\
                       } \\
      non-head-dtrs \liste{ \ibox{4} \ms[word~]{ 
                                        phon & \phonliste{ er }\\
                                        cat & \ms{ head & \ms[noun]{ cas & nom\\
                                                                   } \\
                                                   subcat &  \eliste \\
                                                 }\\
                                        cont & \ms{ ind & \ibox{5} \ms{ per & 3\\
                                                                      num & sg\\
                                                                      gen & mas\\
                                                                    }\\
                                                  restr & \eliste\\
                                                }\\
                                     }
                       }\\
}
\z

 

\end{enumerate}


\section{Spezifikation und Adjunktion}


\begin{enumerate}
\item

\ea
\ms
 { phon & \phonliste{ großem }\\
   cat & \ms{ head & \ms[adj]
                      { %prd & $-$ \\
                        mod &  \textrm{$\overline{\mbox{\textrm{N}}}$:} \ms{ ind   & \ibox{1} \\
                                                                     restr & \ibox{2} \\
                                                                    } \\
                      } \\
               subcat & \liste{} \\
             } \\
   cont & \ms{ ind & \ibox{1} \ms{ per & 3 \\
                                   num & sg \\
                                   gen & mas $\vee$ neu \\
                               } \\
                     restr & \liste{ \ms[groß]{ 
                                theme & \ibox{1} \\ 
                               }} $\oplus$ \ibox{2}  \\
              } \\
}

\z

\end{enumerate}



\section{Das Lexikon}

\begin{enumerate}
\item
\ea 
\ms{
cat & \ms{ head   & \ms[adj]{
                    mod & none\\
                    }\\
           subcat & \sliste{ NP[\nom]\ind{1} } $\oplus$ \ibox{2}\\
         }\\
cont & \ibox{3}\\
} $\mapsto$\\
\ms{
cat & \ms{ head   & \ms[adj]{
                    mod & \textrm{$\overline{\mbox{\textrm{N}}}$:} \ms{ ind   & \ibox{1} \\
                                                                restr & \ibox{4} \\
                                                              } \\
                    }\\
           subcat & \ibox{2}\\
         }\\
cont & \ms{ ind & \ibox{1} \ms{ per & 3 \\
%                                num & sg\\
%                                gen & neu\\
                              } \\
            restr & \sliste{ \ibox{3} } $\oplus$ \ibox{4}  \\
              } \\
}
\z
Der Kasus der NP im Eingabezeichen muss spezifiziert werden, da die Lexikonregel sonst auch auf
subjektlose Adjektive anwendbar wäre:
\eal
\ex[]{
Dem Mann ist schwindelig/schlecht.
}
\ex[*]{
der schwindelige/schlechte Mann
}
\zl
\emph{der schlechte Mann} ist zwar grammatisch, jedoch mit einer anderen Bedeutung als in (\mex{0}a).

Kasus, Numerus und Genus sind im Ausgabezeichen hier nicht spezifiziert. Welche Werte im
Ausgabezeichen dann stehen müssen, hängt von der gewählten Endung ab.

\end{enumerate}

\section{Ein topologisches Modell des deutschen Satzes}

\begin{enumerate}
\item \eal
\ex \field{Karl}{VF} \field{isst}{LS}.
\ex \field{Der Mann}{VF} \field{liebt}{LS} \field{eine Frau}{MF}, \field{\field{den}{VF} \field{Peter}{MF} \field{kennt}{RS}}{NF}.
\ex \field{Der Mann}{VF} \field{liebt}{LS} \field{eine Frau, \field{die}{VF} \field{Peter}{MF} \field{kennt}{RS}}{MF}.
%\ex \field{Die Studenten}{VF} \field{behaupten}{LS}, \field{\field{nur wegen der Hitze}{MF} \field{einzuschlafen}{RS}}{MF}.
\ex \field{Die Studenten}{VF} \field{haben}{LS} \field{behauptet}{RS}, \field{\field{nur wegen der Hitze}{MF} \field{einzuschlafen}{RS}}{NF}.
\ex \field{\field{Dass}{LS} \field{Peter nicht}{MF} \field{kommt}{RS}}{VF}, \field{ärgert}{LS} \field{Klaus}{MF}.
\ex \field{\field{Einen Mann}{MF} \field{küssen}{RS}, \field{\field{der}{VF} \field{ihr nicht}{MF} \field{gefällt}{RS}}{NF}}{VF}, \field{würde}{LS} \field{sie nie}{MF}.
\zl

Zu (\mex{0}c): Theoretisch könnte hier auch eine Extraposition des Relativsatzes ins Nachfeld
vorliegen. Da \emph{eine Frau, die Peter kennt} aber eine Konstituente ist, geht man davon aus, dass
keine Umstellung des Relativsatzes stattgefunden hat, sondern dass die einfachere Struktur mit
\emph{eine Frau, die Peter kennt} als kompletter NP im Mittelfeld vorliegt.

\end{enumerate}

\section{Konstituentenreihenfolge}

%% \begin{enumerate}
%% \item Skizzieren Sie die Analysebäume für folgende Sätze und erläutern Sie,
%%       wie die Unterschiede zwischen den Sätzen von der in diesem Kapitel
%%       vorgestellten Analyse erfaßt werden:
%% \eal
%% \ex daß Kurt-Martin drei Mark dem Clown gibt
%% \ex daß dem Clown Kurt-Martin drei Mark gibt 
%% \ex Gibt Kurt-Martin dem Clown drei Mark?
%% \zl

\begin{figure}[H]
\begin{tikzpicture}
\tikzset{level 1+/.style={level distance=3\baselineskip}}
\tikzset{frontier/.style={distance from root=15\baselineskip}}
\Tree[.{C[\subcat \eliste ] }
       [.{C[\subcat \sliste{ \ibox{1} }]} daß ]
       [.{\ibox{1} V[\subcat \eliste ] }
          [.{\ibox{2} NP[\type{nom}]} Kurt-Martin ]
          [.{V[\subcat \sliste{ \ibox{2} } ] } 
            [.{\ibox{3} NP[\type{acc}]} \edge[roof]; {drei Mark} ]
            [.{V[\subcat \sliste{ \ibox{2}, \ibox{3} } ] }
              [.{\ibox{4} NP[\type{dat}]} \edge[roof]; {dem Clown} ]
              [.{V[\subcat \sliste{ \ibox{2}, \ibox{3}, \ibox{4} } ] } gibt ] ] ] ]
]
\end{tikzpicture}
\end{figure}


\begin{figure}[H]
\begin{tikzpicture}
\tikzset{level 1+/.style={level distance=3\baselineskip}}
\tikzset{frontier/.style={distance from root=15\baselineskip}}
\Tree[.{C[\subcat \eliste ] }
       [.{C[\subcat \sliste{ \ibox{1} } ] } daß ]
       [.{\ibox{1} V[\subcat \eliste ] }
          [.{\ibox{4} NP[\type{dat}]} \edge[roof]; {dem Clown} ]
          [.{V[\subcat \sliste{ \ibox{4} } ] } 
            [.{\ibox{3} NP[\type{acc}]} \edge[roof]; {drei Mark} ]
            [.{V[\subcat \sliste{ \ibox{3}, \ibox{4} } ] }
              [.{\ibox{2} NP[\type{nom}]} Kurt-Martin ]
              [.{V[\subcat \sliste{ \ibox{2}, \ibox{3}, \ibox{4} } ] } gibt ] ] ] ]
]
\end{tikzpicture}
\end{figure}

\begin{figure}[H]
\oneline{%
\begin{tikzpicture}
\tikzset{level 1+/.style={level distance=3\baselineskip}}
\tikzset{level 2+/.style={level distance=4\baselineskip}}
\tikzset{frontier/.style={distance from root=19\baselineskip}}
\Tree[.{V[\subcat \eliste ] }
       [.{V[\subcat \sliste{ \ibox{1} } ] }  \edge node[auto=left]{V1-LR};
         [.{V\ibox{5}[\subcat \sliste{ \ibox{2}, \ibox{3}, \ibox{4} } ] } gibt ] ]
       [.{\ibox{1} V\feattab{\dsl \ibox{5},\\
                             \subcat \eliste} }
          [.{\ibox{2} NP[\type{nom}]} Kurt-Martin ]
          [.{V\feattab{\dsl \ibox{5},\\
                       \subcat \sliste{ \ibox{2} } } } 
            [.{\ibox{4} NP[\type{dat}]} \edge[roof]; {dem Clown} ]
            [.{V\feattab{\dsl \ibox{5},\\
                         \subcat \sliste{ \ibox{2}, \ibox{4} } } }
              [.{\ibox{3} NP[\type{acc}]} \edge[roof]; {drei Mark} ]
              [.{V\ibox{5}\feattab{\dsl \ibox{5},\\
                           \subcat \sliste{ \ibox{2}, \ibox{3}, \ibox{4} } } } \trace{} ] ] ] ]
]
\end{tikzpicture}}
\end{figure}


\section{Nichtlokale Abhängigkeiten}

\begin{figure}[H]
\oneline{%
\begin{tikzpicture}
\tikzset{level 1+/.style={level distance=4\baselineskip}}
\tikzset{level 3+/.style={level distance=5\baselineskip}}
\tikzset{frontier/.style={distance from root=23\baselineskip}}
\Tree[.{V\feattab{\subcat \eliste,\\
                  \slasch \eliste } }
       [.{NP\ibox{1}[\type{acc}]} \edge[roof]; {diese Wohnung} ]
       [.{V\feattab{ \subcat \eliste,\\
                     \slasch \sliste{ \ibox{1} } } } 
         [.{V[\subcat \sliste{ \ibox{2} } ] }  \edge node[auto=left]{V1-LR};
           [.{V\ibox{3}[\subcat \sliste{ \ibox{4}, \ibox{5} } ] } mietet ] ]
         [.{\ibox{2} V\feattab{ \dsl \ibox{3},\\
                                \subcat \eliste,\\
                                \slasch \sliste{ \ibox{1} } } } 
           [.{ \ibox{5} NP\ibox{1}\feattab{ \slasch \sliste{ \ibox{1} } } } \trace{} ]
           [.{V\feattab{\dsl \ibox{3},\\
                        \subcat \sliste{ \ibox{4}, \ibox{5} },\\
                                \slasch \eliste } }
              [.{\ibox{4} NP[\type{nom}]} er ]
              [.{V\ibox{3}\feattab{\dsl \ibox{3},\\
                                   \subcat \sliste{ \ibox{4}, \ibox{5} },\\
                                   \slasch \eliste } }  \trace{} ] ] ]
] 
]
\end{tikzpicture}
}
\end{figure}



\section{Relativsätze}

%% \item Skizzieren Sie die Analyse für die folgende Phrase:
%% \ea
%% die Blume, die allen gefällt
%% \z
%% Gehen Sie auf die \slashwe und die \relwe ein.
\begin{figure}[H]
\oneline{%
\begin{tikzpicture}
\tikzset{level 1+/.style={level distance=3\baselineskip}}
\tikzset{level 3+/.style={level distance=5\baselineskip}}
\tikzset{level 4+/.style={level distance=4\baselineskip}}
\tikzset{frontier/.style={distance from root=23\baselineskip}}
\Tree[.{NP}
       [.{\ibox{1} Det} die ]
       [.{N$'$[\spr \sliste{ \ibox{1} }] } 
          [.{\ibox{2} N$'$[\spr \sliste{ \ibox{1} }]\ind{3} } Blume ] 
          [.{RC\feattab{\mod \ibox{2},\\
                        \rel \eliste,\\
                        \slasch \eliste }} 
            [.{NP~\ibox{4}\feattab{\type{acc},\\
                                   \textsc{rel} \sliste{ \ibox{3} }}} die ]
            [.V\feattab{ \subcat \eliste,\\
                         \slasch \sliste{ \ibox{4} }} 
              [.{\ibox{5} NP\feattab{\HPSGloc \ibox{4},\\
                                     \slasch \sliste{ \ibox{4} }}} \trace{} ]
              [.V\feattab{ \subcat \sliste{ \ibox{5} },\\
                           \slasch \eliste }
                [.\ibox{6}~NP[\type{nom}] alle ]
                [.V\feattab{ \subcat \sliste{ \ibox{6}, \ibox{5} },\\
                             \slasch \eliste }  lieben ] ] ] ] ]
]
\end{tikzpicture}
}
\end{figure}


\section{Kasus}

\eal
\ex Der Junge (str) lacht.  
\ex Mich (lex) friert.    
\ex Er (str) zerstört das Auto (str).
\ex Das (str) dauert ein ganzes Jahr (lex = semantischer Kasus).
\ex Er (str) hat nur einen Tag (str) dafür gebraucht.
\ex Er (str) denkt an den morgigen Tag (lex).
\zl

\section{Der Verbalkomplex}

\resizebox{\linewidth}{!}{%
\nodemargin5pt
\begin{tabular}{@{}ccc@{}c@{}}
\multicolumn{3}{c}{%
\node{s}{\ms[cat]{ head   & \ibox{1} \\
           subcat & \eliste\\
            }}}\\
\\[3ex]
\\[5ex]
\node{np}{\ibox{2} NP[\type{nom}]} & \multicolumn{3}{c}{%
\node{1}{\ms[cat]{ head   & \ibox{1} \\
           subcat & \ibox{3}\\
            }}}\\
\\[3ex]
\\[5ex]
& \multicolumn{2}{c}{%
\node{2}{
\iboxt{5}~\onems{ loc \ms{ head & \ibox{4} \\
                                   subcat & \ibox{3} \\
                                 }
}}} &
\node{3}{
                 \ms[cat]{ head & \ibox{1} \ms[verb]{ vform & fin \\
                                                    } \\
                      subcat & \ibox{3} $\oplus$ \sliste{ \ibox{5} } \\
                    }}\\
\\[3ex]
\\[4ex]
& \node{4}{%
\iboxt{6}~\onems{ loc \onems{ head  \ms[verb]{ vform & bse  \\
                                                }  \\
                                subcat~\ibox{3} \sliste{ \ibox{2} NP[\textit{nom\/}] } \\
                              }\\
                }
} & \node{5}{
\onems[cat]{ head~\ibox{4} \ms[verb]{ vform & bse \\
                            } \\
            subcat ~ \ibox{3} $\oplus$ \sliste{ \ibox{6} } \\
          }
}\\
\\[3ex]
\node{er}{er} & \node{6}{singen} & \node{7}{können} & \node{8}{muss}\\
\end{tabular}
\nodeconnect{s}{np}\nodeconnect{s}{1}%
\nodeconnect{np}{er}%
\nodeconnect{1}{2}\nodeconnect{1}{3}%
\nodeconnect{2}{4}\nodeconnect{2}{5}%
\nodeconnect{4}{6}\nodeconnect{5}{7}\nodeconnect{3}{8}%
}


\section{Kohärenz, Inkohärenz, Anhebung und Kontrolle}


\eanoraggedright
\oneline{%
\ms{
cat & \ms{% head   & verb \\ 
           subcat & \ibox{1} $\oplus$ \sliste{ NP[\type{ldat}]\ind{2} } $\oplus$ \ibox{3} $\oplus$ \sliste{ \textrm{V[\textit{inf}, \textsc{lex}+, \textsc{subj}~\ibox{1}, \textsc{subcat}~\ibox{3} ]:\ibox{4}}}\\
         }\\
cont & \ms[scheinen]{
        experiencer & \ibox{2} \\
        soa & \ibox{4}\\
       }\\
}}\iw{scheinen|uu}
\z

\section{Passiv}

\begin{enumerate}
\item Weil \emph{zwei Wochen} kein Akkusativobjekt sondern ein Akkusativ der Zeit ist. Die NP hat
  semantischen Kasus, der sich nicht in Abhängigkeit von der syntaktischen Umgebung ändert.

\end{enumerate}


\section{Partikelverben}

\begin{enumerate}
\item \emph{vor} verhält sich ähnlich wie \emph{los}, kann aber mit zweistelligen Verben kombiniert
  werden. Das Verb, das im Partikelverb \emph{vorkochen} vorkommt, wird durch die Lexikonregel in
  (\ref{lr-pv}) auf Seite~\pageref{lr-pv} lizenziert. Es wird keine zusätzliche Lexikonregel benötigt.

\eas
\label{le-vor}
\mbox{\emph{vor}:}\\
\ms{
cat & \ms{ head & \ms[particle]{ mod  & \textrm{V} \\
                                 subj & \eliste \\
                               } \\
           subcat & \eliste \\
         }
}
\zs\iw{los|uu}

Das \emph{vor} beschränkt anders als \emph{los} die Valenz des Verbs, mit dem es zusammengeht
nicht. Somit kann es mit einstelligen (\emph{vorarbeiten}), zweistelligen (\emph{vorkochen}) und
auch sogar mit dreistelligen Verben (\emph{vorschenken}) kombiniert werden:
\ea
vater würde ihn mir zu xmas vorschenken\footnote{
  \url{http://www.elitepvpers.com/forum/crossfire-trading/2138728-b-lt-colonel-gg-lm-mg3-gold-etc-6.html}. 20.02.2013.
}
\z

\eas
\label{le-kochen}
\mbox{\stem{koch}:}\\
\ms{ head   & verb \\
     subcat & \liste{NP[\str], NP[\str]} \\[2mm]
}
\iw{kochen|uu}
\zs

\end{enumerate}


\section{Morphologie}



\begin{enumerate}
\item

\eas
Lexikonregel für \emph{ung}"=Nominalisierung:\\
\label{lr-ung-nom}%
\ms[ung-derived-noun-stem]{
phon & \ibox{1} $\oplus$ \phonliste{ ung }\\
synsem & \onems{ loc  \ms{ cat  & \ms{ head   & noun\\
                                       spr & \sliste{ Det }\\
%                                       subcat & \eliste\\
                                      } \\
                            cont & \ms{ ind & \ibox{2} \ms{ per & 3\\
                                                            %num & sg\\    nicht für den Stamm
                                                            gen & fem\\
                                                           }\\
                                         restr & \liste{%
                                                \ms[\ldots]{
                                                        inst & \ibox{2}\\
                                                        soa & \ibox{3}\\
                                                }}\\   
                                      }\\                                 
                         }\\
               }\\
lex-dtr & \onems[stem]{
           phon   \ibox{1} \\
           synsem$|$loc \onems{ cat$|$head \textit{verb}\\
                                cont    \ibox{3} \\
                        }\\
           }\\
}\is{Lexikonregel!ung Nominalisierung@{\em ung}-Nominalisierung}
\zs

Zur Semantik der \emph{ung}-Nominalisierung siehe \citew{ER2000a-u}.

\item Die \suffix{ung}-Nominalisierungsregel wird auf den Stamm des Partikelverbs angewendet. Dieser
  enthält bereits die Partikel in der \subcatl. Es muss also nicht erst aus dem Simplex"=Verb
  \emph{treten} \noword{Tretung} gebildet werden, um dann \emph{Abtretung} bilden zu können.

\end{enumerate}
