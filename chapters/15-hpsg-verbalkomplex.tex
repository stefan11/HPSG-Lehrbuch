%% -*- coding:utf-8 -*-
%%%%%%%%%%%%%%%%%%%%%%%%%%%%%%%%%%%%%%%%%%%%%%%%%%%%%%%%%
%%   $RCSfile: hpsg-verbalkomplex.tex,v $
%%  $Revision: 1.15 $
%%      $Date: 2008/09/30 09:14:41 $
%%     Author: Stefan Mueller (CL Uni-Bremen)
%%    Purpose: 
%%   Language: LaTeX
%%%%%%%%%%%%%%%%%%%%%%%%%%%%%%%%%%%%%%%%%%%%%%%%%%%%%%%%%


\chapter{Der Verbalkomplex}
\label{chap-verbalkomplex}
\is{Verbalkomplex|(}

%\section


% \citet{HN89} haben auf der Grundlage von Voranstellungs- (\mex{1}) und Oberfeldumstellungsdaten (\mex{2}) dafür
% argumentiert, dass Hilfs- und Modalverben im Deutschen mit dem Hauptverb einen Verbalkomplex bilden.
% \ea
% Geholfen haben wird er dem Mann.
% \z
% % mehr
% %Since German is assumed to be a verb second language, i.e., a language with exactly one constituent before the
% %finite verb, examples like (\mex{0}) are evidence for the existence of the constituent \emph{geholfen haben}.
% \eal
% \ex
% dass er dem Mann helfen müssen wird
% \ex
% dass er dem Mann wird helfen müssen
% \zl
% Die Beispiele in (\mex{0}) kann man leicht erklären, wenn man annimmt, dass
% \emph{helfen} und \emph{müssen} einen Komplex bilden, der dann unter \emph{wird} eingebettet wird.
% \emph{wird} kann entweder links oder rechts des eingebetteten Verbalkomplexes stehen.
%
% In Hinrichs und Nakazawas analyse bilden auch \emph{geholfen} und das Hilfsverb
% \emph{hat} in (\mex{1}) einen Verbalkomplex:
% \ea
% \label{ex-er-geholfen-hat}
% dass er dem Mann [geholfen hat]
% \z


In diesem Kapitel wird die Analyse des Verbalkomplexes vorgestellt (Abschnitt~\ref{sec-verbalkomplex}).
Die für die Analyse notwendige Technik der Argumentanziehung wird hier eingeführt. Diese spielen
auch bei der Analyse sogenannter kohärenter Konstruktionen (Kapitel~\ref{chap-anhebung}) und bei
der Analyse des Passivs (Kapitel~\ref{chap-passiv}) eine wesentliche Rolle. Abschnitt~\ref{sec-pvp}
geht auf ein interessantes Kapitel der deutschen Syntax ein: das Voranstellen von Phrasenteilen,
auch unter den Bezeichnungen \emph{Incomplete Category Fronting} oder \emph{Partial Verb Phrase Fronting}
bekannt.

\section{Die Analyse des Verbalkomplexes in der rechten Satzklammer}
\label{sec-verbalkomplex}

In verschiedenen anderen Arbeiten (\zb \citew[Kapitel~3]{Uszkoreit87a}) 
wird angenommen, dass ein Hilfsverb eine Verbphrase als Komplement verlangt. 
\ea
\label{ex-uf}
dass niemand [[[das Buch lesen] können] wird]
\z
Mit solchen Strukturen ist jedoch innerhalb der HPSG"=Theorie
die Abfolge der Verben in (\ref{ex-of}) schwer zu erklären,
da das Hilfsverb \emph{wird} zwischen Bestandteilen der Verbphrase steht.
\ea
\label{ex-of}
dass niemand das Buch wird lesen können
\z
Außerdem sind die Sätze in (\mex{1}) mit einer solchen Analyse nicht ohne weiteres auszuschließen,
da \emph{das Buch lesen} eine Phrase bildet, die im Mittelfeld nach links verschoben
werden bzw.\ in einer sogenannten Rattenfängerkonstruktion\is{Rattenfängerkonstruktion} (siehe S.\,\pageref{page-rattenfaenger})
in einem Relativsatz\is{Relativsatz} auf"|treten könnte.
\eal
\ex[*]{\label{bsp-das-Buch-lesen-niemand-wird}
dass das Buch lesen niemand wird
}
\ex[*]{\label{bsp-das-lesen-niemand-wird}
das Buch, das lesen niemand wird
}
\zl
Dass Stellungen wie die in (\mex{0}) prinzipiell möglich sind, zeigt (\mex{1}):
\eal
\ex dass [das Buch zu lesen] niemand versucht
\ex Für einige Länder Afrikas ist der Export von Kakao und Kakaobutter eine wichtige
Einnahmequelle, [die zu erhalten] sich die EU im Kakaoabkommen von 1993 ausdrücklich
verpflichtet hat.\footnote{
        taz, 23.11.1997, S.\,9.
}
\zl
Es muss also eine Unterscheidung in der Grammatik geben, die für den Kontrast
zwischen (\mex{-1}) und (\mex{0}) verantwortlich ist.
\citet*{HN94a} 
schlagen deshalb die Benutzung eines speziellen Dominanzschemas vor, 
das dafür sorgt, dass (bestimmte) verbale Argumente vor nichtverbalen gesättigt werden. Das heißt,
in der Analyse von (\ref{ex-uf}) und (\ref{ex-of}) wird zuerst \emph{lesen} mit \emph{können}
und dann der entstandene Verbalkomplex mit \emph{wird} kombiniert:
\ea
\label{ex-struktur-lesen-koennen-wird}
dass niemand das Buch [[lesen können] wird]
\z
\emph{wird} kann wie in \pref{ex-uf} rechts oder wie in \pref{ex-of} links des eingebetteten Verbalkomplexes
stehen. Nach dem Aufbau des Verbalkomplexes \emph{lesen können wird} wird dieser mit den Argumenten 
der beteiligten Verben, also mit \emph{das Buch} und \emph{niemand} kombiniert.\footnote{
        Eine solche Struktur wurde schon von 
        \citet*{Johnson86a}
        im Zusammenhang mit der Positionierung des Verbalkomplexes im
        Vorfeld vorgeschlagen.
}

\is{Koordination|(}
Für ein solches Vorgehen sprechen auch Koordinationsdaten
wie der Satz in (\mex{1}).
\ea
Ich liebte ihn, und ich fühlte, daß er mich auch geliebt hat oder doch, daß er mich
hätte lieben wollen\iw{wollen} oder lieben müssen.\footnote{
        \citew*[\page36]{Hoberg81a}.
}\iw{müssen}
\z
Auf Seite~\pageref{page-symmetrische-koordination} wurde dafür argumentiert,
syntaktische Information unter \textsc{cat} zu bündeln. In Koordinationsstrukturen werden
dann einfach die \catwe der Konjunkte identifiziert. Würde man -- wie \zb \citet*{BvN96,BvN98a} --
eine völlig flache Struktur annehmen, in der alle Verben
gleichzeitig miteinander und mit ihren Komplementen kombiniert werden,
so wäre die Koordination in (\mex{0}) nicht als symmetrische Koordination zu erklären.\footnote{
        Es existiert zur Zeit keine umfassende Analyse für Koordinationsphänomene. Es ist also
        nicht völlig auszuschließen, dass (\mex{0}) auch mit einer flachen Struktur zu erklären ist.
        Mit einem strukturierten Verbalkomplex ist (\mex{0}) jedenfalls unproblematisch.%
}
Geht man dagegen von Strukturen wie (\ref{ex-struktur-lesen-koennen-wird}) aus, bilden \emph{lieben wollen}
und \emph{lieben müssen} in (\mex{0}) jeweils eigene Konstituenten, die dann auch mit einer Konjunktion
verknüpft werden können. Die gesamte Wortgruppe \emph{lieben wollen oder lieben müssen} wird dann unter
\emph{hätte} eingebettet.
\is{Koordination|)}


Das folgende Schema, das eine Abwandlung des von
Hinrichs und Nakazawa vorgeschlagenen ist, lizenziert die Prädikatskomplexe\is{Schema!Prädikatskomplex-}:
\begin{samepage}
\begin{schema}[Prädikatskomplexschema]
\label{schema-vk}
\type{head"=cluster"=phrase}\istype{head"=cluster"=phrase} \impl\\
\ms{
 synsem & \onems{ loc$|$cat$|$subcat \ibox{1} \\ 
                } \\
 head-dtr & \onems{ synsem$|$loc$|$cat$|$subcat \ibox{1} $\oplus$ \sliste{ \ibox{2} }\\ } \\
 nonhead-dtrs & \sliste{ [ \synsem  \ibox{2} ]                     } \\
}
\end{schema}
\end{samepage}
%
Dieses Schema entspricht dem Kopf"=Argument"=Schema, das wir bis zum Kapitel~\ref{chap-Konstituentenreihenfolge}
benutzt haben, bzw.\ der angepaßten Version auf Seite~\pageref{schema-head-arg}.
Im Kapitel~\ref{chap-Konstituentenreihenfolge} wurde das \emph{append} durch \emph{delete}
ersetzt, um die relativ freie Anordnung von Konstituenten im Mittelfeld erklären zu können.
Bei der Verbalkomplexbildung ist Reihenfolge der Abbindung von verbalen Argumenten strikt: Es wird
immer mit dem letzten Argument begonnen. Deshalb wird im Prädikatskomplexschema \emph{append} ($\oplus$)
verwendet. Das Kopf"=Argument"=Schema wird auf Seite~\pageref{schema-bin} noch so revidiert,
dass es korrekt mit dem Schema~\ref{schema-vk} und den entsprechenden Lexikoneinträgen für komplexbildende
Verben interagiert.

Für das Hilfsverb\is{Verb!Hilfs-} \emph{werden} nehme ich die Repräsentation in (\mex{1}) an:\footnote{\label{subj-fn}%
        \citet{Pollard90a} und \citet*{Kiss92} haben vorgeschlagen, das Subjekt infiniter
Verben nicht in der \textsc{subcat}"=Liste des Verbs, sondern als Element einer gesonderten Liste (\subj)
zu repräsentieren. Aus Gründen der Übersichtlichkeit habe ich im folgenden die Subjekte finiter
und infiniter Verben gleichermaßen auf der \textsc{subcat}"=Liste repräsentiert. Diese Annahme
wird im Kapitel~\ref{sec-subj-merkmal} revidiert.%
%% Die separate
%% Repräsentation des Subjekts infiniter Verben sagt vorher, dass Subjekte nicht innerhalb
%% von Projektionen von infiniten Verben vorkommen können, es sei denn man formulierte Spezialregeln,
%% die entsprechende Kombinationen lizenzieren.%
%%
%% Für die Analyse der scheinbar mehrfachen Vorfeldbesetzung würde ebenfalls vorausgesagt, dass
%% Subjekte nicht zusammen mit anderen Konstituenten im Vorfeld stehen können, was empirisch korrekt
%% zu sein scheint. Zu einer Analyse von Prädikatskomplexen, die das Subjekt infiniter Verben separat repräsentiert,
%% siehe \citew{Mueller99a,Mueller2002b}.%
%Nimmt man die in \citew[Kapitel~3]{Mueller2002b}
%vorgeschlagene Passivanalyse an, so sind die Beispiele in (\ref{bsp-mehrfach-vf-subjekt}),
%in denen ein Oberflächensubjekt in Passivkonstruktionen zusammen mit einer anderen Konstruktion
%vorangestellt wurde, auch mit einem separat repräsentierten Subjekt problemlos analysierbar.%
}
\eas
\label{le-wird}
\emph{wird} (Futur-Hilfsverb):\\
\ms[cat]{
 head   & \type{verb}\\*
 subcat & \ibox{1} $\oplus$ \sliste{ \textsc{V[\type{bse}, \textsc{lex}+, subcat~\ibox{1}]}} \\
}
\zs
%% Hierbei ist V[\type{bse}, \textsc{lex}+, subcat~\ibox{1}] eine Abkürzung für (\mex{2}):
%% \ea
%% \ms{
%% loc & \ms{ cat & \ms{ head & \ms[verb]{
%%                              vform & bse\\
%%                              }\\
%%                       subcat & \ibox{1}\\
%%                     }\\
%%         }\\
%% lex & +\\
%% }
%% \z
\emph{werden} selegiert ein Verb in der \type{bse}"=Form\istype{bse}\footnote{%
  \emph{bse} ist die Abkürzung für \emph{base}.%
}, \dash einen Infinitiv ohne \emph{zu}. Bevor das \lexm erklärt wird, soll der Aufbau der \subcatl
des Hilfsverbs anhand eines Beispiels erklärt werden:
Im Satz (\mex{1}) übernimmt \emph{wird} die Teilspezifikationen der Argumente
\emph{Karl} und \emph{mir} von \emph{helfen}. 
\ea
dass Karl mir helfen wird
\z
Diese Übernahme erfolgt mit Hilfe der durch die Box \iboxt{1} ausgedrückten Strukturteilung.
Die \textsc{subcat}"=Liste von \emph{helfen wird} hat also die gleiche Form wie die 
\textsc{subcat}"=Liste für \emph{hilft}. Man sagt auch, dass das Hilfsverb die Argumente des
eingebetteten Verbs anhebt bzw.\ anzieht und spricht von Argumentanziehung\is{Argument!anziehung@-anziehung}, Argumentanhebung\is{Argument!anhebung@-anhebung}
bzw.\ Argumentkomposition\is{Argument!komposition@-komposition}. Die Kombination von \emph{helfen} und \emph{wird}
zeigt Abbildung~\vref{abb-helfen-wird}.
\begin{figure}
\centerline{
\begin{forest}
[\ms{ head  & \ibox{1}\\
      comps & \ibox{2}\\
            }
  [\iboxt{4}~\onems{ loc \onems{ head  \ms[verb]{ vform & bse \\
                                             }\\
                                 comps~\ibox{2} \liste{ NP[\type{nom}], NP[\type{dat}] }\\[2mm]
                              }\\
                } [helfen]]
  [\ms{ head & \ibox{1} \ms[verb]{ vform & fin \\
                                                    }\\
                      comps & \ibox{2} $\oplus$ \sliste{ \ibox{4} }\\
                    } [wird]]]
\end{forest}
}
\caption{\label{abb-helfen-wird}%
Analyse von \emph{helfen wird}}
\end{figure}

\noindent
Hilfsverben %verhalten sich wie Anhebungsverben: Sie 
weisen weder einem Subjekt noch Komplementen semantische Rollen\is{semantische Rolle} zu.
So ist es auch nicht verwunderlich, dass
\iboxt{1} in (\ref{le-wird}) durch die leere Liste instantiiert werden kann:
\ea
Morgen wird getanzt werden.
\z
In (\mex{0}) ist eine durch Passiv entstandene subjektlose Konstruktion (\emph{getanzt werden})
unter das Futur"=Hilfsverb eingebettet worden. Das Passiv wird im Kapitel~\ref{chap-passiv}
behandelt.

Die Spezifikation des \textsc{lex}"=Wertes\isfeat{lex} des eingebetteten Verbalkomplexes in (\ref{le-wird})
schließt unechte Mehrdeutigkeiten\is{unechte Mehrdeutigkeit} aus.
% \footnote{
%       Note that this is the only purpose \textsc{lex} has in my grammar.
%       \textsc{lex} has the value $-$ if a head has been combined with a complement and +
%       otherwise. So if an unsaturated verb is combined with an adjunct its \textsc{lex} value
%       is still +. This is not the way \textsc{lex} is seen in the standard framework, and
%       therefore it might be reasonable to choose a different feature name. However, I decided
%         to stick with the name \textsc{lex} for historical reasons.
% } 
Ohne eine solche Spezifikation wären alle drei Strukturen in (\mex{1}) möglich (\citealp[\page303]{Pollard90a};
\citealp{HN94b}):
\eal
\ex er seiner Tochter  ein Märchen [erzählen wird]
\ex er seiner Tochter [[ein Märchen erzählen] wird]]\label{pvp-ein-maerchen-erzaehlen}
\ex er [[seiner Tochter ein Märchen erzählen] wird]]
\zl
Die Spezifikation des \lexwes von \emph{erzählen} stellt sicher, dass \emph{erzählen} mit \emph{wird} kombiniert wird,
bevor \emph{erzählen} mit seinen Argumenten kombiniert wird. Da der Mutterknoten in
Kopf"=Argument"=Strukturen als \textsc{lex}$-$ spezifiziert ist (in anderen Strukturen übrigens auch), 
%(siehe Schema~\ref{schema-bin} auf Seite~\pageref{schema-bin}), 
können die Projektionen
von \emph{erzählen} in (\mex{0}b--c) nicht mit \emph{wird} kombiniert werden.
% \footnote{
% Of course, phrases like \emph{ein Märchen erzählen} and \emph{seiner Tochter ein Märchen erzählen} are needed in the analysis
% of German sentences like those in (i) and (ii):
% \ea
% weil er seiner Tochter hätte ein Märchen erzählen sollen.\\
% \z
% \eal
% \ex 
% Ein Märchen erzählen wird er seiner Tochter.\\
% \ex 
% Seiner Tochter ein Märchen erzählen wird er sicher.\\
% % \zl
% The example in (i) is an instance of the so"=called \emph{Oberfeldumstellung} (\citealp{Bech55a};
% \citealp[\page723]{Haftka81a}) and the examples in (ii) are examples of (partial) verb phrase fronting 
% \citep[\page720--721]{Haftka81a}. For an analysis of Oberfledumstellung see \citep{HN94a} and \cite[Ch.~14]{Mueller99a-unlinked}
% and for an analysis of the partial verb phrase fronting examples see \citep{Mueller97c} and \citep{Meurers99a-unlinked}.
% In the HPSG analysis of these phenomena the constraint that all arguments of embedded verbal complexes have to be raised
% to the higher predicate is relaxed for fronting or in cases like (ii) where no spurious ambiguities can
% arise.
% }

Der \textsc{lex}"=Wert der Mutter in Prädikatskomplexstrukturen ist im Unterschied zu Kopf"=Argument"=Strukturen
nicht restringiert, da Prädikatskomplexe unter andere Verben eingebettet werden können 
und mit diesen einen Prädikatskomplex bilden können, wie (\ref{ex-er-geholfen-haben-wird}) zeigt.
\ea
\label{ex-er-geholfen-haben-wird}
dass er dem Mann [[geholfen haben] wird].
\z
Will man unechte Mehrdeutigkeiten vermeiden, so muss man sicherstellen, dass Sätze wie (\mex{0}) nur auf die
in (\mex{0}) dargestellte Weise und nicht auch wie in (\mex{1}) analysiert werden können.
\ea
\label{bsp-non-complex-forming}
dass er dem Mann [geholfen [haben wird]].
\z
Der Analyse in (\mex{-1}) ist gegenüber der in (\mex{0}) der Vorzug zu geben, da \emph{geholfen
haben} als Konstituente vorangestellt werden kann, \emph{haben wird} bildet dagegen keine irgendwie
nachweisbare Einheit.
In der Analyse in (\mex{0}) wurde das verbale Argument von \emph{haben} zum Argument des Komplexes
\emph{haben wird} angehoben. Der Komplex \emph{haben wird} wird mit \emph{geholfen} über das Prädikatskomplexschema
kombiniert. Die Analyse in (\mex{0}) läßt sich ausschließen, wenn man in Lexikoneinträgen für Anhebungsprädikate
die Art der Elemente, die angehoben werden können, beschränkt. Als zusätzliche Bedingung
für (\ref{le-wird}) muss gelten, dass \iboxt{1} nur vollständig gesättigte
% nicht prädikative 
Elemente mit \textsc{lex}"=Wert $-$\isfeat{lex} enthält. Formal kann das als Beschränkung über \iboxt{1}
ausgedrückt werden:\footnote{
        \citet{BvN98a} formulieren in Prosa eine äquivalente Beschränkung. Sie unterscheiden
        im Satz eine \emph{Inner Zone} und eine \emph{Outer Zone}. Die \emph{Inner Zone}
        entspricht dem Prädikatskomplex. Elemente, die von dem sie regierenden Kopf
        als zur \emph{Inner Zone} gehörig markiert werden, dürfen nicht angehoben werden.

Mit der hier angegebenen Beschränkung für angehobene Elemente wird
        meine Kritik \citep[\page351--352]{Mueller99a} an Kiss' Behandlung der obligatorischen Kohärenz
        als Unterfall der optionalen Kohärenz \citep[\page183]{Kiss95a} hinfällig: Für optional kohärent konstruierende
        Verben reicht in der hier vorgestellten Analyse ein Lexikoneintrag aus. Zur Behandlung
        kohärenter Konstruktionen siehe Kapitel~\ref{sec-anhebung-anal}.%
%
}
\eas
\label{constr-non-complex-forming}
list\_of\_non\_c\_forming\_synsems(\eliste).\\\\
list\_of\_non\_c\_forming\_synsems(\liste{ \onems{ loc$|$cat \onems{ %head$|$prd $-$\\
                                                          subcat \eliste\\
                                                        }\\
                                                lex $-$\\
                                                } $|$ \textrm{Rest} }) :=\\\\
\hfill        list\_of\_non\_c\_forming\_synsems(Rest).\\\\
\zs
Eine Liste besteht aus Elementen, die keinen Prädikatskomplex bilden, wenn die Liste
die leere Liste ist (erste Klausel) oder wenn das erste Element gesättigt ist
und einen \textsc{lex}"=Wert $-$ 
%und einen \textsc{prd}"=Wert $-$ 
hat und wenn der Rest der Liste dieselbe Bedingung erfüllt.\footnote{\label{fn-lex-intransitiv}%
        Auf die Erwähnung des \textsc{lex}"=Wertes kann man in (\ref{constr-non-complex-forming}) nicht verzichten,
        da intransitive Verben -- wenn das Subjekt separat repräsentiert wird -- eine
        leere Valenzliste haben. Der \textsc{lex}"=Wert intransitiver Verben wird
        im Lexikon nicht spezifiziert. Sie können deshalb sowohl an Stellen auf"|treten,
        an denen nur Phrasen erlaubt sind (in sogenannten inkohärenten Konstruktionen \citep{Bech55a})
        als auch an Stellen, an denen nur lexikalische Elemente erlaubt sind (in kohärenten Konstruktionen).
        Das ist auch der Grund dafür, dass der \textsc{lex}"=Wert der Mutter in Prädikatskomplexstrukturen nicht als \textsc{lex}+ spezifiziert
        ist, wie \zb bei \citet{HN94a,dKM2001a}, da Kombinationen aus Verben, die ein intransitives Verb einbetten,
        vollständig gesättigt sein können. Solche vollständig gesättigten Verbalkomplexe können dann eine
        inkohärente Konstruktion mit einem Matrixverb eingehen. Der \textsc{lex}"=Wert eines Verbalkomplexes
        wird also nur durch das übergeordnete Verb restringiert.%
}
%% Das \textsc{prd}"=Merkmal wurde von \citet[\page64--67]{ps} zur Unterscheidung von prädikativen
%% und nicht"=prädikativen Elementen eingeführt.

Weiter unten wird erklärt, warum diese Beschränkung nicht nur für den Ausschluß
unechter Mehrdeutigkeiten, sondern auch für den Ausschluß unmöglicher Vorfeldbesetzungen
eine Rolle spielt.

Wie die Analyse von (\ref{ex-er-geholfen-haben-wird}) im Detail funktioniert, zeigt die Abbildung~\vref{abb-kombin1}.
\begin{figure}
\oneline{
\begin{forest}
sm edges
[{\ms[cat]{ head   & \ibox{1} \\
           comps & \ibox{2}\\
            }}
   [\iboxt{4}~\onems{ loc \ms{ head & \ibox{3} \\
                               comps & \ibox{2} \\
                             }} 
     [\iboxt{5}~\onems{ loc \onems{ head  \ms[verb]{ vform & ppp  \\
                                                   }  \\
                                    comps~\ibox{2} \sliste{ NP[\type{nom}], NP[\type{dat}] } \\
                                  }\\
                      } [geholfen]]
     [{\onems[cat]{ head~\ibox{3} \ms[verb]{ vform & bse \\
                            } \\
            comps ~ \ibox{2} $\oplus$ \sliste{ \ibox{5} } \\
          }} [haben]]]
   [{\ms[cat]{ head & \ibox{1} \ms[verb]{ vform & fin \\
                                          } \\
                 comps & \ibox{2} $\oplus$ \sliste{ \ibox{4} } \\
               }} [wird]]]
\end{forest}
}
\caption{Analyse des Verbalkomplexes in:\ \emph{dass Aicke dem Kind geholfen haben wird}}\label{abb-kombin1}%
\end{figure}
Das Perfekthilfsverb \emph{haben} bettet das Partizip \emph{geholfen} (ein Verb mit \textsc{vform} \type{ppp}\istype{ppp})
ein. Es übernimmt die Argumente dieses Verbs \iboxb{2}. Der resultierende Verbalkomplex hat dieselbe
Valenz wie \emph{geholfen}. Dieser Komplex wird unter \emph{wird} eingebettet. \emph{wird} zieht
ebenfalls die Argumente des eingebetteten Komplexes an, so dass der gesamte Komplex \emph{geholfen haben wird}
dieselben Argumente wie \emph{geholfen} verlangt.

Mit den bisher vorgestellten Komponenten der Analyse kann man erklären, warum
(\ref{bsp-das-Buch-lesen-niemand-wird}) -- hier als (\mex{1}) wiederholt --
ausgeschlossen ist:
%\NOTE{WS: Vielleicht hab ich schon zu lange draufgesehen, aber \emph{dass das Buch lesen niemand wird}
%find ich gar nicht so inakzeptabel.}
\ea[*]{
\label{bsp-das-Buch-lesen-niemand-wird-zwei}
dass [das Buch lesen] niemand wird
}
\z
Eine Kombination von \emph{wird} und \emph{das Buch lesen} wird weder durch das Verbalkomplexschema
noch durch das Kopf"=Argument"=Schema lizenziert, da die Kombination von \emph{das Buch} und \emph{lesen} 
eine Verbalphrase ergibt, die den \lexw $-$\isfeat{lex} hat, 
und demzufolge nicht unter \emph{wird} eingebettet werden kann, da \emph{wird} ein Verb mit
dem \lexw + selegiert.

Diese Beschränkungen schließen aber die folgenden Sätze noch nicht aus:
\eal
\ex[*]{
dass lesen er den Aufsatz wird
}
\ex[*]{
dass er lesen den Aufsatz wird
}
\zl
Auch der Satz (\ref{bsp-das-lesen-niemand-wird}) -- hier als (\mex{1}) wiederholt --
wird nicht ausgeschlossen.
\ea[*]{
\label{bsp-das-lesen-niemand-wird-zwei}
das Buch, das lesen niemand wird
}
\z
Die Sätze in (\mex{-1}) können mit dem Kopf"=Argument"=Schema analysiert werden,
da dieses eine Kombination der Elemente in der \subcatl von \emph{wird} in beliebiger
Reihenfolge zuläßt. Genauso gibt es eine Analyse für (\mex{0}): \emph{lesen} und \emph{niemand}
werden mit \emph{wird} über das Kopf"=Argument"=Schema kombiniert. Eine Spur nimmt den Platz
für das Relativpronomen in einer Kopf"=Argument"=Struktur ein, und das Relativsatzschema lizenziert
dann den gesamten Relativsatz:
\ea[*]{
das Buch, [\sub{rc} das$_i$ [\sub{h-arg} \_$_i$ [\sub{h-arg} lesen [\sub{h-arg} niemand wird]]]]
}
\z
Es ist klar, dass einzelne Verben bzw.\ Verbalkomplexe nicht als Argumente in \kasen
vorkommen sollen. Ein solches Vorkommen kann man dadurch ausschließen, dass man verlangt,
dass die Argumente immer den \lexw $-$ haben. Schema~\ref{schema-bin} ist
das entsprechend angepaßte Schema.\is{Schema!Kopf"=Argument"=}

\begin{samepage}
\begin{schema}[Kopf-Argument-Schema (binär verzweigend, vorläufige Version)]
\label{schema-bin}
\type{head"=argument"=phrase}\istype{head"=argument"=phrase} \impl\\
\onems{
      synsem$|$loc$|$cat$|$subcat \ibox{1} $\oplus$ \ibox{3}\\
% das gilt für mehrere Schemata
%                     lex $-$\\
%                }\\
      head-dtr$|$cat$|$subcat \ibox{1} $\oplus$ \sliste{ \ibox{2} } $\oplus$ \ibox{3} \\
      non-head-dtrs \sliste{ [ \synsem  \ibox{2} \textrm{[\textsc{lex}  $-$ ]} ] }\\
}
\end{schema}\isfeat{lex}
\end{samepage}

\noindent
Der aufmerksame Leser wird sich fragen, was die Theorie in bezug auf Sätze wie (\mex{1}) vorhersagt.%
\NOTE{WS: Die Erklärung, warum (18) nicht geht, verstehe ich nicht. \emph{wird} kann doch keine anderen Argumente als \emph{lachen} haben, die müßte es ja von lachen übernehmen und da kommt es ja nicht ran.}
\ea[*]{
dass lachen er wird
}
\z
In Fußnote~\ref{fn-lex-intransitiv} wurde schließlich gesagt, dass der \lexw intransitiver Verben
im Lexikon nicht spezifiziert wird, so dass diese an Positionen vorkommen können, an denen Phrasen
verlangt werden, aber auch innerhalb von Verbalkomplexen. Man könnte nun annehmen, dass das intransitive Verb
\emph{lachen} in (\mex{0}) als phrasales Argument realisiert werden kann, da der \lexw von \emph{lachen}
mit den Anforderungen des Kopf"=Argument"=Schemas kompatibel ist. Der Satz in (\mex{0}) wird aber dennoch
nicht von der Theorie zugelassen, da das regierende Verb, also \emph{wird}, von seinem verbalen Argument verlangt,
dass dessen \lexw + ist. Somit kommt es in \kasen zu konfligierenden \lexwen, 
weshalb der Satz (\mex{0}) auch ausgeschlossen ist.


\section{Voranstellung von Verbalphrasenteilen}
\label{sec-pvp}

Dass\is{Vorfeldbesetzung|(}\is{Extraktion|(}\is{Partial Verb Phrase Fronting@\emph{Partial Verb Phrase Fronting} (PVP)|(} man Phrasen wie \emph{ein Märchen erzählen} für Sätze braucht, in denen sich diese Wortgruppe im Vorfeld
befindet, scheint auf den ersten Blick problematisch zu sein: Während man diese Phrase als Komplement
in \pref{pvp-ein-maerchen-erzaehlen} -- hier als (\mex{1}a) wiederholt -- ausschließen will,
soll sie in (\mex{1}b) als Binder der Fernabhängigkeit
für die Vorfeldbesetzung auf"|treten:
\eal
\ex er ihr [[ein Märchen erzählen] muss]
\ex Ein Märchen erzählen wird er ihr müssen.
\zl
Sätze wie in (\mex{0}b) sind aber unproblematisch, wenn man \textsc{lex} nicht wie \cite[\page 22]{ps2}
unter \textsc{cat}, also innerhalb von \textsc{local}, sondern unter \textsc{synsem}, also außerhalb von
\textsc{local}, repräsentiert \citep{Mueller96a,Mueller99a,Mueller2002b,Meurers99a}.
Da ein Füller einer Fernabhängigkeit nur die Merkmale mit der Spur teilt, die unter \textsc{local} stehen,
kann ein Verb von einer eingebetteten Spur verlangen, dass diese den \textsc{lex}-Wert\isfeat{lex} + hat. Der \textsc{lex}-Wert
der Spur muss nicht mit dem \textsc{lex}-Wert der Konstituente im Vorfeld identisch sein,
\dash, Wortgruppen mit einem \textsc{lex}-Wert $-$ sind als Füller durchaus
zulässig.\footnote{
        Das heißt, dass es für HPSG"=Grammatiken nicht sinnvoll ist, ein Strukturerhaltungsprinzip\is{Prinzip!Strukturerhaltungs-}
        zu formulieren,
        das besagt, dass eine bewegte Konstituente mit ihrer Spur identisch sein muss. (Siehe \zb
        \citew{Emonds76a-u} zur Formulierung eines solchen Prinzips für Transformationen\is{Transformation}). Ein solches Strukturerhaltungsprinzip
        ist für HPSG"=Grammatiken ohnehin nicht sinnvoll, da overte Realisierungen sich meist von
        ihren Spuren dadurch unterscheiden, dass die overten Realisierungen Töchter haben, was bei Spuren
        nicht der Fall ist. In HPSG"=Grammatiken wird für gewöhnlich nur die Information unter \textsc{local}
        geteilt. Alles andere (\textsc{phon}, \textsc{head-dtr}, \textsc{non-head-dtr}, \textsc{synsem$|$nonlocal},
        \textsc{synsem$|$lex}, \ldots) kann bei Spur und Füller verschiedene Werte haben. Dass die Theorie nicht übergeneriert,
        wird über allgemeine Beschränkungen zur Extraktion geregelt, die mit Bezug auf lokale Kontexte
        spezifiziert werden.%
}
Abbildung~\vref{abb-seiner-tochter-erzaehlen} zeigt die Analyse von (\mex{1}).
\ea
Seiner Tochter erzählen wird er das Märchen.
\z
\begin{figure}
\resizebox{0.935\textwidth}{!}{\begin{sideways}%
\begin{forest}
sm edges
[V\feattab{\type{fin},\\
           \comps \sliste{ },\\
           \textsc{slash} \sliste{ } }
  [{V[\begin{tabular}[t]{@{}l@{}}
             \lex $-$,\\
             \textsc{loc} \ibox{1} [\begin{tabular}[t]{@{}l@{}}
                                    \type{bse},\\
                                    \comps \ibox{2} \sliste{ \ibox{3}, \ibox{4} } ]]\\
                                    \end{tabular}\end{tabular}}
     [{\ibox{5} NP[\type{dat}]} [seiner Tochter, roof]]
     [V\feattab{
       \type{bse},\\ 
       \comps \sliste{ \ibox{3}, \ibox{4}, \ibox{5} } } [erzählen]]]
  [V\feattab{\type{fin},\\
              \comps \sliste{ },\\
              \textsc{slash} \sliste{ \ibox{1} } }
     [V\feattab{\type{fin},\\
                \comps \sliste{ \ibox{7} } }
       [{V[\comps \ibox{2} $\oplus$ \sliste{ \ibox{6} } ]} [wird]]]
     [\ibox{7} V\feattab{\type{fin},\\
                         \comps \sliste{ },\\
                         \textsc{slash} \sliste{ \ibox{1} } }
       [{\ibox{3} NP[\type{nom}]} [er]]
       [V\feattab{\type{fin},\\
                  \comps \sliste{ \ibox{3} },\\
                  \textsc{slash} \sliste{ \ibox{1} } }, s sep+=3ex
         [{\ibox{4} NP[\type{acc}]} [das Märchen, roof]]
         [V\feattab{\type{fin},\\
                    \comps \ibox{2} \sliste{ \ibox{3}, \ibox{4} }\\
                    \textsc{slash} \sliste{ \ibox{1} } }
           [\ibox{6} V\feattab{\lex +,\\
                               \textsc{loc}~\ibox{1}, \\
                               \textsc{slash} \sliste{ \ibox{1} } } [\trace]]
           [V\feattab{\type{fin},\\
                      \comps \ibox{2} \sliste{ \ibox{3}, \ibox{4} } $\oplus$ \sliste{
                        \ibox{6} } } [\trace]]]]]]]
\end{forest}
\end{sideways}}
\caption{\label{abb-seiner-tochter-erzaehlen}%
Analyse von \emph{Seiner Tochter erzählen wird er das Märchen.}}
\end{figure}
Ungrammatische Sätze wie der in (\mex{1}) werden durch die Bedingung
in (\ref{constr-non-complex-forming}) ausgeschlossen.
\ea[*]{
Müssen wird er ihr ein Märchen erzählen.
}
\z
\emph{wird} verlangt einen Infinitiv in der \emph{bse}"=Form, dessen Argumente
es anzieht. Die angezogenen Elemente müssen aber \textsc{lex}$-$ sein. \emph{erzählen}
kann deshalb nicht angezogen werden, weshalb eine Struktur wie (\mex{1}) ausgeschlossen ist:
\ea[*]{
Müssen$_i$ wird$_j$ er ihr ein Märchen [erzählen [\_$_i$ \_$_j$]].
}
\z
Siehe hierzu auch die Diskussion von (\ref{bsp-non-complex-forming}) auf Seite~\pageref{bsp-non-complex-forming}.

Die Analyse in (\mex{1}) ist durch eine allgemeine Bedingung ausgeschlossen,
die Extraktionsspuren in Kopfpositionen verbietet.
\ea[*]{
Müssen$_i$ wird$_j$ er ihr ein Märchen [[erzählen \_$_i$] \_$_j$].
}
\z
%% Der Kontrast in (\mex{1}) ist dadurch erklärt, dass in (\mex{1}a) eine prädikative PP
%% angezogen werden müßte, was bei (\mex{1}b) nicht der Fall ist.
%% \eal
%% \ex[\#]{
%% Halten wird er ihn für den Präsidenten.
%% }
%% \ex[]{
%% Interessieren wird er sich für den Präsidenten.
%% }
%% \zl
%% Die hier vorgeschlagene Analyse ist also durchaus mit der in \citew[Kapitel~2]{Mueller2002b} 
%% vorgeschlagenen Analyse von Konstruktionen wie \emph{halten für} als komplexes Prädikat
%% kompatibel.
\is{Vorfeldbesetzung|)}\is{Extraktion|)}\is{Partial Verb Phrase Fronting@\emph{Partial Verb Phrase Fronting} (PVP)|)}

\section{Alternativen}
\label{sec-alternativen}%

In diesem Abschnitt werden drei alternative Vorschläge zur Analyse der Verbalkomplexe diskutiert:
In Abschnitt~\ref{sec-vcomp} wird die Verwendung eines speziellen Valenzmerkmals für komplexbildende
Prädikate verworfen. Abschnitt~\ref{sec-pvp-vdrei} zeigt, dass man erhebliche Probleme bekommt,
wenn man Voranstellungen ungesättigter Phrasen mit ganz flachen Strukturen erklären will,
und Abschnitt~\ref{sec-restbewegung} widmet sich Ansätzen, die annehmen, dass im Vorfeld nur maximale
Projektionen stehen können und dass die im Abschnitt~\ref{sec-pvp} diskutierten Voranstellungen
Voranstellungen von Maximalprojektionen sind, aus denen vor der Voranstellung ins Vorfeld
Phrasenteile herausbewegt wurden.

\subsection{Spezielle Valenzmerkmale für komplexbildende Argumente}
\label{sec-vcomp}

\citet*{Chung93a} hat für das Koreanische\il{Koreanisch}
und \citet*{Rentier94} für das Niederländische\il{Niederländisch} vorgeschlagen,
ein spezielles Valenzmerkmal (\textsc{gov}\isfeat{gov}) für die Selektion von Elementen zu verwenden,
die mit ihrem Kopf einen Verbalkomplex bilden. Dieser Vorschlag wurde
von \citet{Kathol98b,Kathol2000a} und mir \citep{Mueller97c,Mueller99a} für das Deutsche übernommen.
Die Merkmale heißen bei uns \textsc{vcomp}\isfeat{vcomp} bzw.\ \xcomp.\isfeat{xcomp}
In \citew{Mueller2002b} habe ich die Analysen um Resultativkonstruktionen\is{Resultativkonstruktion} und Subjekts-\is{Subjektsprädikativ} und
Objektsprädikative\is{Objektsprädikativ} nach dem Muster \emph{jemanden für etwas/jemanden halten} erweitert.
Die eingebetteten Prädikate werden in der Analyse ebenfalls über ein spezielles Valenzmerkmal selegiert.

Die hier vorgestellte Theorie kommt ohne ein solches zusätzliches Merkmal aus.
Das hat den Vorteil, dass man optionale Kohärenz als Spezialfall der Kohärenz analysieren
kann, wie das von \citet{Kiss95a} vorgeschlagen wurde. Für Verben wie \emph{versprechen}
benötigt man dann nur noch einen Lexikoneintrag statt der zwei verschiedenen, die für
die kohärente bzw.\ inkohärente Konstruktion benötigt wurden. Zu den Details der Analyse siehe
Kapitel~\ref{sec-anhebung-anal}.

Durch die Reduktion der Anzahl der Valenzmerkmale ist es außerdem möglich geworden,
die Analyse der scheinbar mehrfachen Vorfeldbesetzung wesentlich zu vereinfachen.
Für die Analyse von Sätzen wie (\ref{bsp-smvfb}) auf Seite~\pageref{bsp-smvfb} schlage ich in \citew{Mueller2005d}
eine Lexikonregel vor, die zur Verbbewegungsregel in (\ref{lr-verb-movement2})
völlig parallel ist. Bisherige Vorschläge von mir zur Behandlung der 
scheinbar mehrfachen Vorfeldbesetzung \citep{Mueller2002f,Mueller2002c} haben
noch ein spezielles Valenzmerkmal verwendet, wodurch die Parallelität der beiden
Verbbewegungsregeln verdeckt blieb. Mit der hier verwendeten Merkmalsgeometrie kann
die scheinbar mehrfache Vorfeldbesetzung als optional komplexbildende Variante
der einfachen Verbbewegung verstanden werden \citep{Mueller2005d}.



%% Die ganzen Argumente ziehen alle nicht mehr, wenn man Spirits hat.
%%
%% \subsection{Domänenvereinigung}
%% \label{sec-domain-union-fuer-verbalkomplex}
%% \is{clause union|(}%
%% \is{Konstituente!diskontinuierliche|(}\is{Linearisierung!-sdomäne}
%%
%% \citet{Reape94a} schlägt vor, kohärente Konstruktionen\footnote{
%%   Zum Begriff der Kohärenz siehe Kapitel~\ref{chap-anhebung}. 
%%   Die kohärenten Konstruktionen sind die, die hier mit einem Verbalkomplex analysiert wurden.%
%% }
%% als Satzvereinigung (\emph{Clause Union})
%% zu analysieren. Für (\ref{ex-weil-es-ihm-jemand-zu-lesen-versprochen-hat-zwei}) -- hier
%% als (\mex{1}) wiederholt -- nimmt er an, dass \emph{es zu lesen} eine Phrase
%% bildet, die unter \emph{ihm versprochen} eingebettet wird. Die entstehende Phrase
%% bildet mit \emph{jemand} und \emph{hat} eine Konstituente.
%% \ea
%% \label{ex-weil-es-ihm-jemand-zu-lesen-versprochen-hat-drei}
%% weil    es       ihr       jemand   zu lesen versprochen hat.\footnote{
%% Haider, \citeyear[\page110]{Haider86c}; \citeyear[\page128]{Haider90b}.%
%% }
%% \z
%% Die Phrase \emph{es zu lesen} bildet hierbei eine diskontinuierliche
%% Maximalprojektion\is{Maximalprojektion}. Die Elemente, die in der Konstituentenstellungsdomäne
%% dieser Phrase enthalten sind (\emph{es} und \emph{zu lesen}) werden in die übergeordnete
%% Domäne, \dash in die Domäne des Kopfes \emph{versprochen}, aufgenommen (siehe Kapitel~\ref{sec-Reape-Linearisierung}
%% zu Linearisierungsdomänen).
%%
%% %\subsubsection{Agreement}
%%
%% Für Verben wie \word{scheinen} nimmt Reape an, dass das Verb einen 
%% assumes that the raising verb
%% embeds a non"=finite clause that contains the subject.
%% \ea
%% weil    der Fritz die Maria zu lieben scheint.
%% \z
%% This means that \emph{der Fritz die Maria zu lieben} is a clause that is embedded
%% under \emph{scheint}. \emph{der Fritz} agrees with \emph{scheint} since it
%% is the subject in (\mex{0}). This fact cannot be accounted for in Reape's approach unless
%% one assumes that the non"=finite verb \emph{zu lieben} has agreement\is{agreement!subject-verb} features that
%% can be checked with the subject of \emph{zu lieben} and that are simultaneously present
%% at \emph{scheint} (\citew[Kapitel~5.1]{Kathol98b}; \citew[Kapitel~21.1]{Mueller99a}). 
%% As there is no morphological reflex of the agreement features
%% on non"=finite forms, such a solution would be pretty ad hoc.

%% Furthermore, the so-called remote passive, which will be discussed in Chapter~\ref{sec-remote-passive-phen},
%% cannot be explained in Reape's framework \citep[Kapitel~5.2]{Kathol98b}.
%% \is{clause union|)}%
%% \is{discontinuous constituent|)}



\subsection{Ganz flache Strukturen für Satz inklusive Vorfeld}
\label{sec-pvp-vdrei}
\is{Vorfeldbesetzung|(}

\citet[\page170--171]{Gunkel2003b} schlägt vor, Sätze wie (\mex{1}) als Verbdrittsätze
mit einer ganz flachen Struktur zu analysieren.
\eal
\ex Reparieren müssen wird er die Brille.
\ex Die Brille reparieren wird er müssen.
\zl 
\emph{reparieren} und \emph{müssen} bzw.\ \emph{die Brille} und \emph{reparieren}
stehen dabei nicht als Konstituente, sondern einzeln im Vorfeld.
Wie die Linearisierungsbeschränkungen
für solche Sätze aussehen, läßt er offen. Analysiert man Sätze wie (\mex{1}) mit total
flachen Strukturen und mit drei Konstituenten im Vorfeld, kann man nicht erklären,
wieso sich die Konstituenten vor dem finiten Verb genauso verhalten als gäbe es dort
ein Mittelfeld\is{Feld!Mittel-}, eine rechte Satzklammer\is{Satzklammer} und ein Nachfeld\is{Nachfeld}.
\ea
Den Kunden sagen, dass die Ware nicht lieferbar ist, wird er wohl müssen.
\z
Nimmt man dagegen an, dass die Konstituenten vor dem finiten Verb eine Verbalprojektion
bilden, können die Bestandteile der Verbalprojektion wieder topologischen Feldern
zugeordnet werden, und die Anordnung der Wortgruppen vor dem Finitum muss nicht gesondert
erklärt werden. Siehe auch \citew[\page82]{Reis80a} und Kapitel~\ref{sec-topo-rekursion}.

Unabhängig davon, ob man den Wörtern vor dem finiten Verb Konstituentenstatus
zuspricht \citep{Kathol95a} oder nicht \citep{Gunkel2003b}, können Ansätze,
die Sätze wie (\mex{0}) über lokale Umstellung erklären, Sätze wie (\mex{1})
nicht erfassen.
\eal
\ex Das Buch gelesen glaube ich nicht, dass er hat.\footnote{
  \citew[\page 82]{Sabel2000a}.
}
\ex Angerufen denke ich, daß er den Fritz nicht hat.\footnote{
  \citew[\page 110]{Fanselow2002a}.
}
\zl
In (\mex{0}) kommen die Wortgruppen vor dem Finitum aus dem eingebetteten Satz,
können also nicht durch lokale Umstellung nach vorn gelangt sein.

Ein anderes Problem ergibt sich im Zusammenhang mit der Kasusvergabe:
In \citew[\page93--94]{Mueller2002b} habe ich gezeigt, dass die Tatsache,
dass \emph{den Wagen} in (\mex{1}) Akkusativ hat, nicht zu erklären wäre, wenn man annehmen würde,
dass sich in (\mex{1}) zwei unabhängige Konstituenten im Vorfeld befinden.\footnote{
  Siehe auch \citew{Mueller2005d}.%
}
\eal 
\ex[]{
Den Wagen zu reparieren wurde versucht.
}
\ex[*]{
Der Wagen     zu reparieren wurde versucht.
}
\zl
In Konstruktionen mit dem sogenannten Fernpassiv (siehe Kapitel~\ref{sec-remote-passive-phen})
kann das Objekt von \emph{reparieren} durchaus im Nominativ stehen, wie (\mex{1}a) zeigt.
Betrachtet man (\mex{1}b), stellt man fest, dass die Nominativ-NP allein vorangestellt werden kann.
\eal
\ex[]{
weil der Wagen zu reparieren versucht wurde
}
\ex[]{
Der Wagen wurde zu reparieren versucht.
}
\zl
Auch der \emph{zu}"=Infinitiv kann einzeln vorangestellt werden, wie (\mex{1}) zeigt:\NOTE{JB: ungrammatisch}
\ea
Zu reparieren wurde der Wagen versucht.
\z
Bei einer solchen Voranstellung ist der Nominativ von \emph{der Wagen} zwingend.
Läge nun bei (\mex{-2}) eine Voranstellung des Infinitivs und der Nominalphrase 
als einzelne Konstituente vor, so müßte auch hier ein Nominativ möglich sein, 
was nicht den beobachtbaren Fakten entspricht.
\is{Vorfeldbesetzung|)}


\subsection{Restbewegung}
\label{sec-restbewegung}

Im Prinzipien-und-Parameter-Framework\indexgb werden die im Abschnitt~\ref{sec-pvp} diskutierten
Voranstellungen unvollständiger Projektionen
oft als Restbewegung\is{Restbewegung} (\emph{Remnant Movement}) analysiert (siehe \zb G. \citealp{GMueller96a,GMueller98a}).
\citet[\page281]{Haider93a}, \citet[Kapitel~4.2.5]{deKuthy2002a}, \citet[Abschnitt~2]{dKM2001a}\NOTE{Mehr zu den Einwänden} und \citet{Fanselow2002a} haben jedoch gezeigt, 
dass Restbewegungsansätze mit empirischen Problemen zu kämpfen haben, 
die Argumentkompositionsansätze wie der hier vorgestellte nicht haben.%
\is{Verbalkomplex|)}


\questions{
\begin{enumerate}
\item Welche Teile des Verbalkomplexes können vorangestellt werden?
\end{enumerate}
}

\exercises{
\begin{enumerate}
\item Zeichnen Sie einen Analysebaum für den Satz (\mex{1}):
      \ea
      {}[weil] Aicke singen können muss
      \z
      Geben Sie die Lexikoneinträge für die beteiligten Verben an.

\item Laden Sie die zu diesem Kapitel gehörende Grammatik von der Grammix"=CD
(siehe Übung~\ref{uebung-grammix-kapitel4} auf Seite~\pageref{uebung-grammix-kapitel4}).
Im Fenster, in dem die Grammatik geladen wird, erscheint zum Schluß eine Liste von Beispielen.
Geben Sie diese Beispiele nach dem Prompt ein und wiederholen Sie die in diesem Kapitel besprochenen
Aspekte.

\end{enumerate}
}


\furtherreading{
Die Analyse der Verbalkomplexe in der HPSG"=Theorie ist wesentlich von \citet{HN89b,HN89a,HN94a} geprägt
worden. Hinrichs und Nakazawas Arbeiten gehen auf Arbeiten im Rahmen der
Kategorialgrammatik\is{Kategorialgrammatik (CG)} zurück \citep{Geach70a}. Zum Deutschen gibt es
Arbeiten von \citet{Kiss95a} und \citet{Meurers2000b}.

Komplexe Prädikate im Deutschen (Verbalkomplexe und Resultativkonstruktionen werden in
\citew{Mueller2002b} besprochen. Eine Übersicht über Arbeiten zu komplexen Prädikaten im
Persischen, Französischen, Italienischen, Spanischen, Portugiesischen, Koreanischen und Deutschen bieten \citet{GS2021a}.
}