%% -*- coding:utf-8 -*-
%%%%%%%%%%%%%%%%%%%%%%%%%%%%%%%%%%%%%%%%%%%%%%%%%%%%%%%%%
%%   $RCSfile: 6-hpsg-adjunkte.tex,v $
%%  $Revision: 1.22 $
%%      $Date: 2008/07/28 13:31:07 $
%%     Author: Stefan Mueller (CL Uni-Bremen)
%%    Purpose: 
%%   Language: LaTeX
%%%%%%%%%%%%%%%%%%%%%%%%%%%%%%%%%%%%%%%%%%%%%%%%%%%%%%%%%

\chapter{Spezifikation und Adjunktion}
\label{chap-adjunkte}


Im Kapitel~\ref{sec-intro-arg-adj} wurde der Unterschied zwischen Argumenten und\NOTE{JB: sind Kapitel ganze Zahlen oder haben sie einen Punkt}
Adjunkten erklärt. Im vorvorigen Kapitel wurde gezeigt, wie die Selektion von
Argumenten durch ihren Kopf modelliert werden kann, und im vorigen Kapitel wurde
erklärt, wie die Bedeutungskomposition erfolgt. In diesem Kapitel sollen nun
syntaktische und semantische Eigenschaften von Kopf"=Adjunkt"= und
Spezifikator"=Kopf"=Strukturen beleuchtet werden.



\section{Die Syntax von Kopf"=Adjunkt"=Strukturen}
\label{sec-Syntax-Kopf-Adjunkt}

Im Kapitel~\ref{sec-intro-arg-adj}\is{Adjunkt|(}\is{Modifikator|(} wurde festgestellt, dass die Form von Adjunkten,
die mit bestimmten Köpfen vorkommen können, relativ wenig beschränkt ist. Andererseits
können Adjunkte oft nur mit Köpfen einer bestimmten Wortart vorkommen. Flektierte
Adjektive sind \zb nur als Modifikatoren von Nomina möglich. 
\eal
\ex[]{
ein interessantes Buch
}
\ex[*]{
Peter schläft interessantes.
}
\zl
Analog zur Selektion von Argumenten durch Köpfe über \comps kann
man Adjunkte ihren Kopf über ein Merkmal (\textsc{modified})\isfeat{mod} selegieren lassen.
Adjektive, Nomina modifizierende Präpositionalphrasen und Relativsätze
selegieren eine fast vollständige Nominalprojektion, \dash ein Nomen, das
nur noch mit einem Determinierer kombiniert werden muss, um eine vollständige
NP zu bilden. Ausschnitte des Lexikoneintrags für \emph{interessantes} zeigt (\mex{1}):
\ea\is{Adjektiv}
\label{le-interessantes}
\emph{interessantes}:\\
\ms{ 
   phon & \phonliste{ interessantes }\\
   cat & \ms{ head & \ms[adj]{ %prd & $-$ \\
                        mod &  \nbar\\
                      } \\
              comps & \liste{} \\
            } \\
}
\z
\emph{interessantes} ist ein Adjektiv, das selbst keine Argumente mehr zu sich nimmt,
und das deshalb eine leere \compsl hat. Adjektive wie \emph{treu} in (\mex{1}) würden entsprechend
eine Dativ"=NP in ihrer \compsl haben.
\ea
ein dem König treues Mädchen
\z
Den Lexikoneintrag zeigt (\mex{1}):
\ea
\label{le-treue}
\emph{treues}:\\
\ms{ 
   phon & \phonliste{ treues }\\
   cat & \ms{ head & \ms[adj]{ %prd & $-$ \\
                        mod &  \nbar\\
                      } \\
              comps & \liste{ NP[\type{dat}] } \\
            } \\
}
\z
\emph{dem König treues} bildet dann eine Adjektivphrase, die \emph{Mädchen} modifiziert.

Im Gegensatz zum Selektionsmerkmal \comps, das zu den Merkmalen unter \textsc{cat}  gehört,
ist \textsc{mod} ein Kopfmerkmal.
Der Grund dafür ist, dass das Merkmal, das den zu modifizierenden Kopf selegiert, an der
Maximalprojektion des Adjunkts vorhanden sein muss. Die Eigenschaft der Adjektivphrase \emph{dem
König treues}, dass sie \nbar{}s modifiziert, muss in der Repräsentation der gesamten AP enthalten sein,
genauso wie sie im Lexikoneintrag für Adjektive in (\ref{le-interessantes}) auf lexikalischer Ebene
vorhanden ist. Die Adjektivphrase \emph{dem König treues} hat dieselben syntaktischen Eigenschaften wie das einfache
Adjektiv \emph{interessantes}:
\ea
\label{avm-dem-koenig-treues}
\emph{dem König treues}:\\
\ms{ 
   phon & \phonliste{ dem König treues }\\
   cat & \ms{ head & \ms[adj]{ %prd & $-$ \\
                        mod &  \nbar\\
                      } \\
              comps & \eliste{ } \\
            } \\
}
\z
Wenn \textsc{mod} ein Kopfmerkmal ist, sorgt das Kopfmerkmalsprinzip (siehe Seite~\pageref{prinzip-hfp})
dafür, dass der \modw der gesamten Projektion mit dem \modw des Lexikoneintrags für \emph{treues} identisch ist.


Alternativ zur Selektion des Kopfes durch den Modifikator könnte man eine
Beschreibung aller möglichen Adjunkte beim Kopf vornehmen. Dies wurde von
\citet[\page 161]{ps} vorgeschlagen. \citet[Abschnitt~1.9]{ps2} rücken von diesem Ansatz aber wieder
ab, da die Semantik der Modifikation mit diesem Ansatz nicht ohne weiteres
beschreibbar ist.\footnote{
        Siehe jedoch \citew*{BMS2001a}. \citet*{BMS2001a} verfolgen einen hybriden Ansatz, in dem es sowohl Adjunkte gibt,
        die den Kopf selegieren, als auch solche, die vom Kopf selegiert werden.
        Als Semantiktheorie liegt diesem Ansatz die \textit{Minimal Recursion Semantics}
        (MRS)\is{Minimal Recursion Semantics@\emph{Minimal Recursion Semantics} (MRS)}
        zugrunde. Mit dieser Semantik treten die Probleme bei der Beschreibung der Semantik
        von Modifikatoren, die \citet*{ps} hatten, nicht auf.
}


Abbildung~\vref{fig-ha-selektion} zeigt ein Beispiel für die Selektion in Kopf"=Adjunkt"=Strukturen.
\begin{figure}
\begin{forest}
sm edges
[\nbar
  [{AP[\textsc{head$|$mod} \ibox{1}]}
    [interessantes]]
  [\ibox{1} \nbar
    [Buch]]]
\end{forest}
\caption{\label{fig-ha-selektion}Kopf"=Adjunkt"=Struktur (Selektion)}
\end{figure}
Kopf"=Adjunkt"=Strukturen sind durch das folgende Schema lizenziert:\istype{head"=adjunct"=phrase}
%\begin{figure}[htbp]
\begin{samepage}
\begin{schema}[Kopf-Adjunkt-Schema (vorläufige Version)]
\label{ha-schema-prel}
~\\
\textit{head"=adjunct"=phrase} \impl\\
\is{Schema!Kopf"=Adjunkt"=}
\ms{ 
head"=dtr      & \ibox{1} \\[2mm]
non-head"=dtrs & \liste{ \ms{ cat & \ms{ head$|$mod & \ibox{1} \\
                                         comps   & \liste{} \\
                                       } \\
                           }}\\
}
\end{schema}
\end{samepage}
%\vspace{-\baselineskip}\end{figure}
Der \modw des Adjunkts \iboxb{1} wird mit dem \synsemw der Kopf"|tochter identifiziert, wodurch
sichergestellt wird, dass die Kopf"|tochter die vom Adjunkt spezifizierten Eigenschaften hat. Die \compsl
der Nicht"=Kopf"|tochter ist die leere Liste, weshalb nur vollständig gesättigte Adjunkte 
in Kopf"=Adjunkt"=Strukturen zugelassen sind. Phrasen wie (\mex{1}b) werden somit korrekt ausgeschlossen:
\eal
\ex[]{
die Wurst in der Speisekammer
}
\ex[*]{
die Wurst in
}
\zl
Das Beispiel in (\mex{0}a) soll noch genauer diskutiert werden. Für die Präposition \emph{in} (in der
Verwendung in (\mex{0}a)) nimmt man den folgenden Lexikoneintrag an:

\ea
Lexikoneintrag für \emph{in}:\\
\ms{
phon & \phonliste{ in } \\
cat & \ms{ head & \ms[prep]{
                   mod &   \nbar\\
                   } \\
           comps & \sliste{ NP[\type{dat}] } \\
         } \\
}
\z
Nach der Kombination mit der Nominalphrase \emph{der Speisekammer} bekommt man:
\ea
Repräsentation für \emph{in der Speisekammer}:\\
\ms{
phon & \phonliste{ in der Speisekammer } \\
cat & \ms{ head & \ms[prep]{
                   mod &   \nbar\\
                   } \\
           comps & \liste{ } \\
         } \\
}
\z
Diese Repräsentation entspricht der des Adjektivs \emph{interessantes} und kann 
-- abgesehen von der Stellung der PP -- auch genauso verwendet
werden: Die PP modifiziert eine \nbar.

Köpfe, die nur als Argumente verwendet werden können und nicht selbst modifizieren,
haben als \modw{} \type{none}. Dadurch können sie in Kopf"=Adjunkt"=Strukturen nicht an die Stelle
der Nicht"=Kopf"|tochter treten, da der \modw der Nicht"=Kopf"|tochter mit der Kopf"|tochter kompatibel
sein muss.





\section{Die Semantik in Kopf"=Adjunkt"=Strukturen}
\label{sem-adj}

Es ist noch nicht erklärt worden, wie der semantische Beitrag des Mutterknotens in
Abbildung~\ref{fig-ha-selektion} bestimmt wird. Der Bedeutungsbeitrag des Nomens
\emph{Buch} steht fest: \relation{buch}(X). Eine naheliegende Möglichkeit, die Gesamtbedeutung
des Ausdrucks \emph{interessantes Buch} zu bestimmen, ist, dass man die Bedeutungen 
der beiden Töchter nach oben gibt und diese konjunktiv verknüpft.
Aus der Kombination von \emph{interessantes} (\relation{interessant}(X)) mit \emph{Buch} (\relation{buch}(Y))
würde \relation{interessant}(X) \& \relation{buch}(X). Ein solcher Ansatz ist aber für Fälle wie (\mex{1})
problematisch:
\ea
der angebliche Mörder
\z
Die Kombination von \emph{angebliche} (\relation{angeblich}(X)) und {\em Mörder\/} (\relation{mörder}(Y)) ist eben
gerade nicht \relation{angeblich}(X) \& \relation{mörder}(x), da die Information, dass der Referent der Nominalphrase
ein Mörder ist, durch \emph{angeblich} als ungesicherte Information gekennzeichnet wird.
Statt \relation{angeblich}(x) \& \relation{mörder}(x) ist also \relation{angeblich}(\relation{mörder}(x)) als semantische Repräsentation
anzunehmen.
Die Alternative für die semantische Komposition von Adjunkt- und Kopfbeitrag
besteht darin, die Gesamtbedeutung des Ausdrucks am Adjunkt festzumachen:
Im Lexikoneintrag für \emph{interessantes} bzw.\ \emph{angebliche}
steht, wie der Bedeutungsbeitrag der Mutter aussehen wird.
Die Bedeutung des modifizierten Kopfes wird im Lexikoneintrag des Modifikators 
in die Bedeutung des Modifikators integriert. Für das Beispiel \emph{interessantes Buch} ist das
in Abbildung~\vref{fig-ha-synsem} dargestellt.
\begin{figure}
\oneline{%
\begin{forest}
sm edges
[{\nbar[\cont \rnode{cont1}{\ibox{1}}]}
  [AP\feattab{\textsc{head$|$mod} \ibox{2},\\
              {\rnode{cont2}{\cont} \ibox{1} [\textsc{restr} \nliste{ interessant(\ibox{3}) } $\oplus$ \ibox{4}}]} [interessantes]]
  [{\ibox{2}} \nbar{[\textsc{cont$|$restr} \ibox{4} \nliste{ buch(\ibox{3}) }]}
    [Buch]]]
\end{forest}
\nccurve[angleA=90,angleB=175,ncurvB=1]{<->}{cont1}{cont2}
}
\itdopt{remove the pstricks code}
\caption{\label{fig-ha-synsem}Kopf"=Adjunkt"=Struktur (Selektion und Bedeutungsbeitrag)}
\end{figure}
Das Kopf"=Adjunkt"=Schema identifiziert den Kopf mit dem \textsc{mod}"=Wert der Adjunkttochter \iboxb{2}.
Dadurch wird es möglich, innerhalb eines Lexikoneintrags lokal die semantische Komposition zu
regeln, denn man kann auf alle Information innerhalb von \textsc{mod} mittels Strukturteilung
verweisen. Das Adjunkt muss dazu noch nicht mit dem Kopf kombiniert sein.
Das Adjektiv \emph{interessantes} vereinigt in seiner semantischen Repräsentation den
semantischen Beitrag des modifizierten Nomens \iboxb{4}
mit der Liste der Restriktionen, die es selbst beiträgt (\semliste{ \relation{interessant}\iboxb{3} }).
Wird der \modw des Adjektivs durch Verwendung des Adjektivs in einer Kopf"=Adjunkt"=Struktur
instantiiert, so wird auch der Wert von \iboxt{4} instantiiert. Im vorliegenden Beispiel
ist \iboxt{4} dann \semliste{ \relation{buch}\iboxb{3} }. Auf diese Weise ist es möglich, den
gesamten semantischen Beitrag der Phrase beim Adjunkt zu repräsentieren. Er wird 
dann auch von dort projiziert (\ibox{1} in Abbildung~\ref{fig-ha-synsem}).

Die Struktur in (\mex{1}) zeigt den vervollständigten Eintrag für \emph{interessantes}:\footnote{
  Die Repräsentation der Person-, Numerus- und Genus"=Werte wird im Kapitel~\ref{sec-np-kongruenz}
  über die Kongruenz innerhalb von Nominalphrasen noch revidiert.%
}
\ea\is{Adjektiv}
Adjektiveintrag mit Bedeutungsrepräsentation:\\
\label{le-interessantes-sem}%
\ms
 { phon & \phonliste{ interessantes }\\
   cat & \ms{ head & \ms[adj]
                      { %prd & $-$ \\
                        mod &  \textrm{$\overline{\mbox{\textrm{N}}}$:} \ms{ ind   & \ibox{1} \\
                                                                     restr & \ibox{2} \\
                                                                    } \\
                      } \\
               comps & \liste{} \\
             } \\
   cont & \ms{ ind & \ibox{1} \ms{ per & 3 \\
                                   num & sg \\
                                   gen & neu \\
                               } \\
                     restr & \liste{ \ms[interessant]{ 
                                theme & \ibox{1} \\ 
                               }} $\oplus$ \ibox{2}  \\
              } \\
}
\z
Das Adjektiv selegiert das zu modifizierende Nomen über \textsc{mod}. Deshalb kann
das Adjektiv auf den \contw und damit auf die Restriktionen des Nomens \iboxb{2} zugreifen
und diese bei sich in den semantischen Beitrag einbauen.
Die Teilung des Indexes \iboxb{1} sorgt dafür, dass Adjektiv und Nomen sich auf dasselbe Objekt beziehen.

Die Gesamtstruktur, die dem Mutterknoten in Abbildung~\ref{fig-ha-synsem} entspricht, beschreibt (\mex{1}):

\ea
\textit{interessantes Buch}:\\
\samepage
\ms
{ cat & \ms{ head & \type{noun} \\
             comps & \sliste{ Det } \\
           } \\
  cont &  \ms
           { ind & \ibox{1} \ms{ per & 3 \\
                                 num & sg \\
                                 gen & neu \\
                               } \\
             restr & \liste{ \ms[interessant]{ theme & \ibox{1} \\ },
                             \ms[buch]{ inst & \ibox{1} \\ }} \\
           } \\
}
\z

\noindent
Die Projektion des semantischen Beitrags erfolgt also nicht wie \zb in Kopf"=Argument"=Strukturen
entlang des Kopfpfades, sondern von der Adjunkttochter zur Mutter:
\ea
\type{head"=adjunct"=phrase} \impl\\
\onems{ 
cont \ibox{1}\\
non-head"=dtrs \sliste{ [ \cont  \ibox{1} ] }\\
}
\z
Integriert man diese Strukturteilung in die vorläufige Version des Kopf"=Adjunkt"=Schemas
auf Seite~\pageref{ha-schema-prel}, erhält man das Schema~\ref{ha-schema}.
%
%\begin{figure}[htbp]
\begin{schema}[Kopf-Adjunkt-Schema (vorläufige Version)]
\label{ha-schema}
\begin{tabular}[t]{@{}l@{}}
\type{head"=adjunct"=phrase}\istype{head"=adjunct"=phrase} \impl\\
\ms{ 
cont & \ibox{1} \\
head"=dtr      & \ibox{2} \\[2mm]
non-head"=dtrs & \liste{ \ms{ cat & \ms{ head$|$mod & \ibox{2} \\
                                        comps   & \liste{} \\
                                      } \\
                             cont & \ibox{1} \\
                           }}\\
}
\end{tabular}
\end{schema}
%\vspace{-\baselineskip}\end{figure}%
\is{Adjunkt|)}\is{Modifikator|)}

\section{Prinzipien}
\label{sec-prinzipien}

Im Kapitel~\ref{sec-semp-i} wurde eine vorläufige Version des Semantikprinzips eingeführt. Diese
soll jetzt präzisiert und vervollständigt werden. Außerdem wird noch das Valenzprinzip vorgestellt, das etwas über
die Argumentabbindung in verschiedenen Strukturtypen aussagt.

\subsection{Das Semantikprinzip}
\is{Prinzip!Semantik-|(}

Die auf Seite~\pageref{semp-i} formulierte vorläufige Version des Semantikprinzips besagt,
dass in Strukturen mit Kopf der \contw der Mutter mit dem \contw der Kopf"|tochter
identisch ist. Wie wir im vorigen Abschnitt gezeigt haben, ist ein solches Vorgehen für
Kopf"=Adjunkt"=Strukturen nicht sinnvoll. Das Semantikprinzip wird deshalb in zwei Teile geteilt:
einen für Kopf"=Adjunkt"=Strukturen und einen für alle anderen Strukturen:
\begin{prinzip-break}[Semantikprinzip]
\label{semp}
In Strukturen mit Kopf, die keine Kopf"=Adjunkt"=Strukturen sind, ist der
semantische Beitrag der Mutter identisch mit dem der Kopf"|tochter.

In Kopf"=Adjunkt"=Strukturen ist der semantische Beitrag der Mutter
identisch mit dem der Adjunkttochter.
\end{prinzip-break}
Formal sieht das wie in (\mex{1}) aus:
\ea
\type{head"=non"=adjunct"=phrase}\istype{head"=non"=adjunct"=phrase} \impl
\ms{
cont & \ibox{1} \\
head"=dtr$|$cont & \ibox{1} \\
}\\[2mm]

\type{head"=adjunct"=phrase}\istype{head"=adjunct"=phrase} \impl
\ms{ 
cont & \ibox{1} \\
non-head"=dtrs & \sliste{ [ \cont  \ibox{1} ] }\\
}
\z
Strukturen mit Kopf sind entweder vom Typ \type{head"=adjunct"=phrase} oder Untertypen von
\type{head"=non"=adjunct"=phrase}.
\is{Prinzip!Semantik-|)}


\subsection{Das Valenzprinzip}
\label{sec-valp}
\is{Prinzip!Valenz-|(}


In Kopf"=Adjunkt"=Strukturen ändert sich die Valenz der gesamten Phrase im Vergleich
zur Valenz des Kopfes nicht: \emph{Buch} hat die gleiche Valenz wie \emph{interessantes Buch}:
Sowohl \emph{Buch} als auch \emph{interessantes Buch} braucht noch einen Determinator, um als
vollständige Projektion verwendet werden zu können.
Deshalb muss in Kopf"=Adjunkt"=Strukturen die Valenzinformation am Mutterknoten der Valenzinformation 
der Kopf"|tochter entsprechen. Man kann das noch verallgemeinern auf alle Strukturen, in denen
kein Argument mit einem Kopf kombiniert wird. Diesen Strukturen wird der Typ \type{head"=non"=argument"=phrase}
zugeordnet. Bisher wurden nur Kopf"=Adjunkt"=Strukturen diskutiert, aber es gibt \zb auch
noch Kopf"=Füller"=Strukturen (siehe Kapitel~\ref{chap-nla}), in denen wie in Kopf"=Adjunkt"=Strukturen
keine Argumente gesättigt werden. Solche Strukturen sind allesamt Untertypen des Typs \type{head"=non"=argument"=phrase}.

Formal sieht die Beschränkung der \compswe in Strukturen vom Typ \type{head"=non"=argument"=phrase}\istype{head"=non"=argument"=phrase}
wie folgt aus:
\ea
\label{def-head-non-arg-phrase}
\type{head"=non"=argument"=phrase}\istype{head"=non"=argument"=phrase} \impl
\onems{
      cat$|$comps \ibox{1} \\
      head"=dtr$|$cat$|$comps \ibox{1}\\
      }
\z
Da in Strukturen vom Typ \type{head"=non"=argument"=phrase}
keine Argumente gesättigt werden, ist der \compsw der Mutter mit dem der Kopf"|tochter identisch.

In der HPSG"=Literatur wird das Valenzprinzip immer in Prosa formuliert:

\begin{prinzip-break}[Valenzprinzip]
In Strukturen mit Kopf entspricht die \compsl des Mutterknotens der \compsl der Kopf"|tochter
minus den als Nicht"=Kopf"|tochter realisierten Argumenten.
\end{prinzip-break}

\noindent
Zur vollständigen Formalisierung dieses in Prosa angegebenen Prinzips gehören neben (\mex{0})
natürlich auch die Beschränkungen für Strukturen vom Typ \type{head"=argument"=phrase}, die im
Schema~\ref{schema-bin-prel} auf Seite~\pageref{schema-bin-prel} formuliert wurden, und hier der
Übersichtlichkeit halber noch einmal wiederholt seien:\istype{head"=argument"=phrase}
\ea
\type{head"=argument"=phrase}\istype{head"=argument"=phrase} \impl\\
\onems{
cat$|$comps \ibox{1} \\
head"=dtr$|$cat$|$comps  \ibox{1} $\oplus$ \sliste{ \ibox{2} } \\
non-head"=dtrs  \sliste{ \ibox{2} } \\
} 
\z

\noindent
Strukturen mit Kopf (\type{headed"=phrase}) sind entweder Untertypen von
\emph{head"=argument"=phrase} oder von \emph{head"=non"=argument"=phrase}.
Einen Überblick über die Untertypen des Typs \type{sign} zeigt Abbildung~\vref{abb-sign-non-ha-non-har-struc}.
\begin{figure}
\centerline{
\begin{forest}
type hierarchy,
 for tree={
   calign=fixed angles,
   calign angle=60
 } 
[sign %,s sep+=4em
  [word]
  [phrase
    [non-headed-phrase]
    [headed-phrase, for tree={l sep+=\baselineskip}
      [head-non-adjunct-phrase, calign=first
        [head-argument-phrase]
        [\ldots]]
      [head-non-argument-phrase, calign=last
       [,identify=!r2212]
       [head-adjunct-phrase]]]]]
\end{forest}}
\itdopt{Linien treffen sich nicht}
\caption{\label{abb-sign-non-ha-non-har-struc}Typhierarchie für \type{phrase}}
\end{figure}
\type{head"=non-adjunct"=phrase} und \type{head"=non"=argument"=phrase} haben
mehrere gemeinsame Untertypen. Hier ist nur einer beispielhaft durch die drei Punkte
angedeutet.

Bevor wir uns der Analyse von Beispielen wie \emph{der angebliche Mörder} zuwenden,
soll das Zusammenwirken aller bisher formulierten Prinzipien gezeigt werden.
Abbildung~\vref{fig-ha-val-sem-hfp} zeigt, wie das Kopfmerkmalsprinzip (siehe \iboxt{1}),
das Valenzprinzip (siehe \iboxt{2}) und
das Semantikprinzip (siehe \iboxt{3}) die
Eigenschaften der Gesamtstruktur determinieren.
\begin{figure}[htbp]
%\centerline{
\hspace{2em}\resizebox{0.96\textwidth}{!}{%
\begin{forest}
sm edges
[\nbar\feattab{\head   \rnode{h1}{\ibox{1}},\\
               \comps \rnode{sc1}{\ibox{2}},\\
               \rnode{cont1}{\cont} \ibox{3}}
  [AP\feattab{\textsc{head$|$mod} \ibox{4},\\
              \rnode{cont2}{\cont} \ibox{3} [\textsc{restr} \nliste{ interessant(\ibox{5}) } $\oplus$ \ibox{6}]}
    [interessantes]]
  [\ibox{4} \nbar\feattab{\head   \rnode{h2}{\ibox{1}},\\
                          \comps \rnode{sc2}{\ibox{2}} \sliste{ det },\\
                          \textsc{cont$|$restr} \ibox{6} \nliste{ buch(\ibox{5}) }}
    [Buch]]]
\end{forest}
\nccurve[angleA=0,angleB=45]{<->}{h1}{h2}
\nccurve[angleA=0,angleB=45]{<->}{sc1}{sc2}
\nccurve[angleA=170,angleB=175,ncurvB=1.2]{<->}{cont1}{cont2}}
\caption{\label{fig-ha-val-sem-hfp}Kopf"=Adjunkt"=Struktur (HFP, Valenz, Semantik, \ldots)}
\end{figure}
\is{Prinzip!Valenz-|)}





\section{Kapselnde Modifikation}
\label{sec-kapselnde-Modifikation}

Die\is{Adjektiv|(} Behandlung der Bedeutungskomposition in Kopf"=Adjunkt"=Strukturen ist relativ komplex.
Sie wurde damit motiviert, dass man Beispiele wie \emph{potentielle Mörder} in (\mex{1})
nicht einfach durch die Projektion der koordinativen Verknüpfung
von \relation{potentiell}(x) und \relation{mörder}(x) analysieren kann.
\ea
\label{soldat}
%% Darin heißt es, dass "Soldaten nicht nur potentielle Mörder sind, sondern im wahrsten Sinne des Wortes bezahlte Killer".21.10.1994 taz Inland 97 Zeilen, hans-hermann kotte S. 5
%% Augst hatte 1989 bei einer Diskussion Soldaten als "potentielle Mörder" bezeichnet. 10.10.1994 taz Aktuelles 14 Zeilen, S. 2

%% Er hatte sich in einem Leserbrief mit dem Ausspruch "Alle Soldaten sind potentielle Mörder" solidarisiert.22.9.1994 taz Tagesthema 91 Zeilen, kotte S. 3
Gewalt provoziere immer Gegengewalt und: "`Soldaten sind potentielle Mörder."'\footnote{
23.12.1993, taz berlin, S.\,18.
}
\z
Die Formel in (\mex{1}) wäre für die Repräsentation des Prädikates im Satz (\mex{2}) angemessen, für
die Prädikation in (\mex{0}) ist sie es nicht.
\ea
$\ll m"order, instance:X\gg$
\z
\ea
Soldaten sind Mörder.\footnote{
  Tucholsky, Kurt, (1931), "`Der bewachte Kriegsschauplatz"', \emph{Die Weltbühne}, S.\,31.
}
\z
Sätze wie (\mex{-2}) machen keine Aussage darüber, dass Soldaten Mörder sind, sie sagen vielmehr,
dass es möglich ist, dass Soldaten zu Mördern werden. Die \relation{mörder}"=Relation ist unter \relation{potentiell}
eingebettet und nicht direkt für logische Schlußfolgerungen zugänglich:
\ea
\label{pot}
$\ll potentiell, \textit{psoa-arg} : ~\{ \ll m"order, instance:X\gg \} \gg$
\z
Dabei wird \textsc{psoa-arg}\isfeat{psoa-arg} als die Bezeichnung für die Argumentrolle der \relation{potentiell}"=Relation
verwendet, wobei \textsc{psoa} für \emph{parametrized state of affairs} und \textsc{arg} für Argument
steht.

Adjektive wie \emph{mutmaßlich-} oder \emph{angeblich-} verhalten sich genauso wie \emph{potentiell}:
Zeitungen dürfen noch nicht verurteilte Personen nicht als Mörder bezeichnen, auch wenn es starke Evidenz
für eine Schuld zu geben scheint. Es muss immer von \emph{mutmaßlichen Mördern} gesprochen werden.

Die Bedeutungskomposition für solche Fälle ist mit der eingeführten Maschinerie einfach: Statt
wie bei \emph{interessantes} eine Verkettung der Restriktionsliste des modifizierten Nomens mit
der Restriktionsliste des Adjektivs vorzunehmen, wird die Restriktionsliste des modifizierten
Nomens (\iboxt{2} in (\mex{1})) als Argument der Adjektiv"=Relation eingebettet:

\ea
\emph{mutmaßlich-} nach \citep*[\page330]{ps2}:\\
\samepage
\ms{
  cat & \ms{ head & \ms[adj]
                     { %prd & $-$ \\
                       mod & \baro{N}\textrm{:} \ms{ ind & \ibox{1} \\
                                                                          restr & \ibox{2} \\
                                                                        } \\
                     } \\
             comps & \liste{ } \\
           } \\
  cont & \ms{ ind & \ibox{1} \\
                    restr & \liste{ \ms[mutmaßlich]{ 
                                      psoa-arg & \ibox{2} \\ 
                                      }} \\
                   } \\
}
\z

\noindent
Der referentielle Index des Adjektivs wird mit dem referentiellen Index des Nomens identifiziert.
Das ist wichtig, da der semantische Beitrag von Kopf"=Adjunkt"=Strukturen ja durch das Adjunkt
bestimmt wird. Die \iboxt{1} in (\mex{0}) entspricht dem X in (\mex{-1}), \dash, der referentielle
Index dient zur Bezugnahme auf das Objekt, über das gerade gesprochen wird.  Die Information, dass
wir über X bzw.\ \iboxt{1} reden, ist also für den Beitrag der gesamten Phrase \emph{potentielle
Mörder} bzw.\ \emph{mutmaßlicher Mörder} relevant. Wir sagen aber nicht aus,
dass X ein Mörder ist, sondern dass X ein potentieller bzw.\ mutmaßlicher Mörder ist. Die Kombination von
\emph{mutmaßlicher} und \emph{Mörder} zeigt (\mex{1}):
\ea
\emph{mutmaßlicher Mörder}:\\
\samepage
\ms{
  cat & \ms{ head   & noun \\
             comps & \sliste{ Det } \\
           } \\
  cont & \ms{ ind & \ibox{1} \ms{ per & 3 \\
                                 num & sg \\
                                 gen & mas \\
                               } \\
                    restr & \liste{ \ms[mutmaßlich]{ 
                                      psoa-arg & \ms[mörder]{
                                                  inst & \ibox{1}\\ 
                                                 }\\
                                      }} \\
                   } \\
}
\z


Komplexere Fälle wie (\mex{1}) können auf ähnliche Weise behandelt werden.
\ea
ein scheinbar einfaches Beispiel
\z
Aus Platzgründen kann ich darauf hier aber nicht eingehen. Zu den Einzelheiten siehe \citew{Kasper95a} und
\citew{Mueller99a} bzw.\ \citew*[Abschnitt~6.3]{CFPS2005a}.%
\is{Adjektiv|)}

Mit der Diskussion dieser Beispiele ist die Behandlung von Kopf"=Adjunkt"=Strukturen abgeschlossen.
Im folgenden Abschnitt soll noch das Spezifikatorprinzip diskutiert werden. Es erlaubt die korrekte
Behandlung der Bedeutungskomposition bei Possessivpronomina. Das dazu verwendete Merkmal ähnelt
dem \textsc{mod}"=Merkmal, weshalb das Spezifikatorprinzip noch in diesem Kapitel diskutiert werden soll.




\section{Spezifikator-Kopf-Strukturen}
\label{sec-spec-kopf}

Wir\is{Determinator|(} haben im vorigen Kapitel für den Satz in (\mex{1}) die semantische
Repräsentation in (\ref{ex-sachverhalt-drei}) angegeben.
\ea
\label{bsp-mann-schlaegt-hund-zwei}
Der Mann schlägt den Hund.
\z
\ea
\label{ex-sachverhalt-drei}
$\ll schlagen, agens:X, patiens:Y\gg$\\
$X|\ll mann, instance:X\gg,$\\
$Y|\ll hund, instance:Y\gg$
\z
Dabei handelte es sich um eine Vereinfachung, denn der Bedeutungsbeitrag
des Artikels ist in (\mex{0}) nicht erfaßt. Betrachtet man (\mex{1}) ist klar,
dass Determinatoren eine wichtige Rolle spielen, da die Aussage von (\mex{1})
von der Aussage von (\ref{bsp-mann-schlaegt-hund-zwei}) wesentlich verschieden ist.
\ea
Alle Männer schlagen einen Hund.
\z
\citet[\page 48]{ps2} stellen den Quantorenteil des semantischen Beitrags einer Nominalphrase wie \emph{alle Männer}
analog zu (\mex{1}) dar:
\ea
\ms[all]{
restind & \ms{
                       ind & \ibox{1} \ms{ per & 3\\
                                           num & pl\\
                                           gen & mas\\
                                         }\\
                       restr & \liste{ \ms[mann]{
                                       inst & \ibox{1}\\
                                       }}\\
                       }\\
}
\z
Der semantische Beitrag des Nomens wird also als Wert des Merkmals \textsc{restricted index} (\textsc{restind}\isfeat{restind})
repräsentiert. Der Typ der Merkmalstruktur ist für definite Artikel \type{def}, für Determinatoren wie \emph{alle} 
oder \emph{ein} entspricht der Typ dem Quantor.
%
Ich gehe wie \citet[\page 49]{ps2} davon aus, dass in Nominalstrukturen das Nomen der Kopf ist. Eine NP besteht
aus einem Determinator und aus einer \nbar. Der Determinator wird von der \nbar selegiert.
Im semantischen Beitrag des Quantors\is{Quantor|(} wird auf das Nomen Bezug genommen, wie (\mex{0}) zeigt. Da das
Nomen aber der Kopf ist, stellt sich die Frage, wie die Bedeutungskomposition im Determinator vorgenommen werden kann.
Damit der Determinator die Bedeutung des Nomens in seine eigene Bedeutung integrieren kann,
muss er Zugriff auf die Bedeutung des Nomens haben. Dies wird durch die Einführung des
\textsc{specified}"=Merkmals\isfeat{spec} erreicht, das zum im vorigen Abschnitt besprochenen \modm
analog ist. Auf den semantischen Beitrag des Kopfnomens kann über \textsc{spec} zugegriffen werden,
der entsprechende \contw wird mit dem Wert von \textsc{restind} identifiziert. (\mex{1})
zeigt den Lexikoneintrag für \emph{alle}:
\eas
\emph{alle}:\\
\ms{
cat & \ms{ head   & \ms[det]{
                    spec$|$cont & \ibox{1}\\ 
                    }\\
           comps & \eliste\\
         }\\
cont & \ibox{2} \ms[all]{
               restind & \ibox{1}\\ 
               }\\
qstore & \sliste{ \ibox{2} }\\
}
\zs
In (\mex{0}) ist auch das Merkmal \textsc{qstore}\isfeat{qstore} enthalten. \textsc{qstore} steht für \emph{Quantifier Store}.
In einem solchen Speicher\is{Speicher} werden alle quantifizierten Ausdrücke, die innerhalb einer Phrase auf"|treten,
gespeichert. \citet[Kapitel~8]{ps2} und \citet{PY98a-u} zeigen,\is{Skopus} wie man mit
Hilfe solcher Quantoren"=Speicher die richtigen Lesarten für Quantoren enthaltende Sätze
bekommt. Eine Diskussion dieser Mechanismen würde jedoch den Rahmen des vorliegenden Buches
sprengen.%
\is{Quantor|)}


Die Identifikation des \textsc{spec}"=Wertes eines Determinators mit der Kopf"|tochter
wird durch das folgende Prinzip geregelt:
\begin{prinzip-break}[Spezifikatorprinzip (\textsc{spec}-Principle)] 
\label{prinzip-spec}\is{Prinzip!Spezifikator-}
Wenn eine Tochter, die keine Kopf"|tochter ist, in einer Kopf"|struktur
einen von \type{none} verschiedenen \textsc{spec}-Wert besitzt, so ist dieser token-identisch mit
%dem \textsc{synsem}-Wert 
der Kopf"|tochter.
\end{prinzip-break}
In Nominalstrukturen gibt es somit eine gegenseitige Selektion: Zum einen selegiert das
Nomen den Determinator über \comps und zum anderen selegiert der Determinator das Nomen
über \textsc{spec}.

Ähnlich liegen die Verhältnisse in Nominalstrukturen mit Possessivpronomen.
In Beispielen wie (\mex{1}) füllt das Kopfnomen eine semantische Rolle in der Relation des Possessivums:
\ea
sein Geschenk
\z
Possessiva drücken eine -- wie auch immer geartete -- Zugehörigkeitsrelation
zwischen dem Referenten der Possessivphrase und dem Kopf"|nomen
aus. Die Relation \relation{besitzen} ist hierbei ein Supertyp
der Relation, die durch den Äußerungskontext und durch Weltwissen\is{Weltwissen} 
bestimmt wird.\footnote{
        Siehe auch \citew[\page 205--206]{Chomsky70a} und \citew*[\page13]{Jackendoff77}.           % zitiert Chomsky (Remarks S. 45 und 205-206
%Jaegli86a:609
%Haider88a:54
        Zur kontextabhängigen Disambiguierung von Possessiva siehe
        \citew*{Nerbonne92}.
}
 
Ein schönes Beispiel von Jürgen Kunze\aimention{J{\"u}rgen Kunze} dafür, 
dass Possessivpronomina nicht unbedingt eine Besitzrelation ausdrücken müssen,
ist die folgende Situation: Karl schenkt Max ein Buch.
Wird jetzt über \emph{sein Geschenk} gesprochen, so kann sich \emph{sein} sowohl auf
Max als auch auf Karl beziehen. \emph{Karl lag diesmal genau richtig. Sein Geschenk
gefällt Max am besten.} oder: \emph{Max bewundert sein Geschenk}.
Im ersten Fall bezieht sich \emph{sein} in der naheliegenden Lesart
auf \emph{Karl}, obwohl \emph{Max} der Besitzer ist.

Wenn also das Kopfnomen eine Rolle in der \relation{besitzen}"=Relation füllt, muss der
referentielle Index des Kopfnomens für das Possessivum zugänglich sein. Da der Determinator
aber nicht der Kopf ist, hat er das Nomen nicht in seiner \compsl. Auch hier hilft
wieder das Selektionsmerkmal \textsc{spec}. In Determinator"=Nomen"=Strukturen
ist der Wert von \textsc{spec} identisch mit dem Nomen. (\mex{1}) zeigt den Lexikoneintrag
für die feminine Form im Nominativ/""Akkusativ von \emph{sein}:\NOTE{FB: location erklären}

\eas
\label{le-seine}
\textit{seine}:\\
\ms{
cat & \ms{ head & \ms[det]{
                       cas & nom $\vee$ acc \\
                       gen & fem \\
%                       dtype & 3 \\
                       spec & \textrm{N:[\textsc{ind} \ibox{1}, \textsc{restr} \ibox{2}]} \\
                      } \\
            comps & \liste{} \\
           } \\
cont & \ms{ ind & \ibox{3} \ms{ 
                            per & 3 \\
                            num & sg \\
                            gen & mas $\vee$ neu \\
                           }} \\
qstore & \liste{ \ms[def]{
                 restind & \ms{ ind & \ibox{1}\\
                                restr & \liste{ \ms[besitzen]{ 
                                           location & \ibox{3} \\
                                           thema    & \ibox{1} \\
                                           } } $\oplus$ \ibox{2} \\
                        }\\
                 }}  \\
}
\zs
Kasus\isfeat{case} und Genus\isfeat{gen} gehören zu den Kopfmerkmalen von Determinatoren. Diese Merkmale sind
für die Herstellung von Kongruenz\is{Kongruenz} zwischen Nomen und Determinator wichtig (zur ausführlichen
Behandlung der Kongruenz siehe Kapitel~\ref{chap-kongruenz}). Die Genusmerkmale
des referentiellen Indexes können durchaus von syntaktischen Genusmerkmalen abweichen.
Mit \emph{sein} bezieht man sich auf einen maskulinen Referenten, die Nominalphrase
kann aber syntaktisch feminin sein: \emph{seine Idee}.\is{Determinator|)} 

%% Vielleicht geht das ja semantisch.
%%
%% \citet{Soehn03a} und \citet{SS2003a} haben vorgeschlagen, das \spec- und das \textsc{mod}"=Merkmal
%% zu einem Merkmal für externe Selektion (\textsc{xsel}) zusammenzufassen. Dieses Merkmal benutzen
%% sie innerhalb der Analyse idiomatischer Wendungen. Diese Analyse ist sehr elegant und deckt
%% viele idiomatische Wendungen ab, eine Zusammenlegung von \textsc{spec} und \textsc{mod} scheint aber
%% problematisch zu sein, da unklar wäre, was Nominalphrasen wie die in (\mex{1}) ausschließt:
%% \eal
%% \ex die seine Freundin
%% \ex die die Freundin
%% \zl



%% Wir haben gesehen, dass Lexikoneinträge bereits jetzt sehr komplexe Gebilde sind. In späteren
%% Kapiteln werden wir noch weiter Merkmale einführen, so dass die Komplexität noch weiter zunehmen
%% wird. Im folgenden Kapitel wird gezeigt, wie man diese Fülle von Information kompakt und
%% redundanzfrei repräsentieren und wie man Generalisierungen in Bezug auf Wortklassen und
%% Valenzalternationen bzw.\ morphologische Alternationen erfassen kann.


\section{Alternativen}

Im folgenden sollen DP"=Analysen, \dash Analysen, die davon ausgehen, dass in Nominalstrukturen
Determinatoren der Kopf sind, diskutiert werden. Daran anschließend wird ein Analyse"=Vorschlag
für Nominalstrukturen ohne Determinator besprochen.

\subsection{Die DP-Analyse}
\label{sec-dp-analyse}

\mbox{}\is{Determinator!-phrase|(}%
Die Analyse der Nominalstrukturen ist relativ komplex: Das Selektionsmerkmal \textsc{spec} wird benötigt,
das Spezifikatorprinzip muss formalisiert werden, und außerdem brauchen wir ein neues
Valenzmerkmal und eine neue Grammatikregel für Determinator"=Nomen"=Strukturen.
Diese wurde hier noch nicht eingeführt, das wird aber im Kapitel~\ref{sec-spr}
erfolgen. Eine Analyse, die davon ausgeht, dass der Determinator der Kopf\is{Kopf} ist, scheint wesentlich
einfacher zu sein. Für den Quantor \emph{alle} könnte man die Struktur in (\mex{1}) annehmen:

\eas
\emph{alle} in der DP-Analyse:\\
\ms{
cat    & \ms{ head   & det\\
           comps & \sliste{ NP:\ibox{1} }\\
         }\\
cont   & \ibox{1}\\
qstore & \sliste{  \ms[all]{
                   restind & \ibox{1}\\ 
                   } }\\
}
\zs

\noindent
Eine mit diesem Determinator gebildete DP hätte (vom \headw abgesehen) dasselbe Aussehen
wie die NP in der bisher diskutierten NP"=Analyse.

Problematisch an dieser DP"=Analyse sind aber die Possessivpronomina. Man könnte sich
hier folgenden Lexikoneintrag vorstellen:
\eas
\textit{seine} (falsch):\\
\ms{
cat & \ms{ head    & det\\
           comps & \sliste{ NP:[\textsc{ind} \ibox{1}, \textsc{restr} \ibox{2}] } \\
         } \\
cont & \ms{ ind & \ibox{3} \ms{ 
                            per & 3 \\
                            num & sg \\
                            gen & mas $\vee$ neu \\
                           }} \\
qstore & \liste{ \ms[def]{
                 restind & \ms{ ind & \ibox{1}\\
                                restr & \liste{ \ms[besitzen]{ 
                                           location & \ibox{3} \\
                                           thema    & \ibox{1} \\
                                           } } $\oplus$ \ibox{2} \\
                        }\\
                 }}  \\
}
\zs
Das Problem besteht nun darin, dass die Repräsentation für \emph{seine Mutter},
die mit (\mex{0}) gebildet werden kann, die falsche Index"=Information hat, denn das Semantikprinzip
sorgt dafür, dass der semantische Beitrag in einer Kopf"=Argument"=Struktur vom Kopf kommt. In der
DP"=Analyse ist der Determinator der Kopf und steuert somit den semantischen Hauptbeitrag bei. Man
würde also eine Struktur bekommen, die durch (\mex{1}) beschrieben wird:
\eas
\textit{seine Mutter} (falsch):\\
\ms{
cat & \ms{ head    & det\\
           comps & \sliste{ } \\
         } \\
cont & \ms{ ind & \ibox{3} \ms{ 
                            per & 3 \\
                            num & sg \\
                            gen & mas $\vee$ neu \\
                           }} \\
qstore & \liste{ \ms[def]{
                 restind & \ms{ ind & \ibox{1}\\
                                restr & \liste{ \ms[besitzen]{ 
                                           location & \ibox{3} \\
                                           thema    & \ibox{1} \\
                                           }, \ms[mutter]{
                                              inst &\ibox{1}\\
                                              }} \\
                        }\\
                 } }  \\
}
\zs
Der Index dieser DP entspricht dem Possessivpronomen, nicht aber dem Nomen.
Das Nomen ist aber das Element, das eine semantische Rolle eines Verbs füllen kann.
Würde man mit dem Eintrag in (\mex{-1}) den Satz in (\mex{1}) analysieren,
bekäme man eine falsche Repräsentation, in der \emph{lachen} über \emph{seine}
prädiziert. Die semantische Repräsentation wäre dann (\mex{1}c) statt (\mex{1}b):
\eal
\ex Seine Mutter lacht.
\ex $\ll lachen, agens:X\gg$\\
    $X|\ll mutter, instance:X\gg$, $\ll besitzen, location: Y, thema: X \gg$
   
\ex $\ll lachen, agens:Y\gg$\\
    $X|\ll mutter, instance:X\gg$, $\ll besitzen, location: Y, thema: X \gg$
\zl

\noindent
Man könnte versuchen, das Problem zu beheben, indem man den Index
des Nomens zum Index von \emph{seine} macht. Somit würde dieser Index
dann auch zum Index der DP werden. Die Index"=Information des Possessivums
wäre nur noch in der Argumentstelle der \relation{besitzen}"=Relation
repräsentiert:
\eas
\textit{seine} (falsch):\\
\oneline{%
\ms{
cat & \ms{ head    & det\\
           comps & \sliste{ NP:[\textsc{ind} \ibox{1}, \textsc{restr} \ibox{2}] } \\
         } \\
cont & \ms{ ind & \ibox{1} \\
          } \\
qstore & \liste{ \ms[def]{
                 restind & \ms{ ind & \ibox{1}\\
                                restr & \liste{ \ms[besitzen]{ 
                                           location & \ms{ 
                                                      per & 3 \\
                                                      num & sg \\
                                                      gen & mas $\vee$ neu \\
                                                      } \\
                                           thema    & \ibox{1} \\
                                           } } $\oplus$ \ibox{2} \\
                        }\\
                 }}  \\
}
}
\zs

\noindent
Dieser Vorschlag funktioniert aber auch nicht, da die Information über das Possessivpronomen
für die Bindungstheorie\is{Bindungstheorie|(} gebraucht wird. Zur Bindungstheorie siehe
\citew[Kapitel~6]{ps2}. Man könnte vorschlagen, die Bindungstheorie auf semantischen Repräsentationen
operieren zu lassen, aber das funktioniert nicht, da man auch die Bindung von Reflexivpronomen
im Zusammenhang mit inhärent reflexiven Verben\is{Verb!inhärent reflexives} erklären will:
Das Reflexivum\is{Pronomen!Reflexiv-} muss in Sätzen wie (\mex{1}) immer mit dem Subjekt in Person\is{Person} und Numerus\is{Numerus} übereinstimmen.
\eal
\ex Ich erhole mich.
\ex Du erholst dich.
\ex Er erholt sich.
\ex Ihr erholt euch.
\zl
Die Reflexivpronomina in (\mex{0}) sind keine semantischen, aber syntaktische Argumente.
Eine Bindungstheorie muss also auf syntaktischen Strukturen bzw.\ mit Bezug auf Valenzinformation
operieren. Somit muss die Information über den Index des Possessivums außerhalb der Relationen
in \textsc{restr} verfügbar sein.\is{Bindungstheorie|)} 
Eine Analyse mit Possessivum als Kopf scheidet also aus. Damit bleiben
nur zwei Möglichkeiten: die NP"=Analyse und eine Analyse, die einen leeren Determinator\is{leere Kategorie} als Kopf
nimmt, der dann mit Possessivum und NP verbunden wird \citep[\page 50, \page 53]{Abney87a}.
Diese Analyse ist dann aber wieder komplizierter
als die NP"=Analyse, so dass man der NP"=Analyse wohl den Vorzug geben muss.

%%  Dan Flickinger: 2008 in Kyoto
%%  Er arbeitet einen großen Teil der Woche. -> Man braucht eine Regel, die eine NP mit Zeitnomen projiziert.
%%
%% \Citet{vanLangendonck94a} und \citet{Hudson2004a} führen ein weiteres Argument 
%% für eine Analyse mit dem Nomen als Kopf an: Bestimmte Nomina können als Adjunkte auf"|treten:
%% \eal
%% \ex Er hat \emph{den ganzen Tag} gearbeitet.
%% \ex \emph{Eines Tages} werd' ich mich rächen, [\ldots]\footnote{
%%   Die Ärzte, \emph{Debil}, CBS Schallplatten GmbH, 1984.
%% }
%% \zl
%% Geht man von einer Analyse aus, in der der Modifikator bestimmt, welche Köpfe er modifizieren
%% kann, dann muss man in Nomina wie \emph{Tag} den \modw so spezifizieren, dass diese Nomina Verben
%% modifizieren können. Außerdem muss der Kasus des Nomens im Lexikoneintrag spezifiziert sein,
%% denn die Bedeutung des Modifikators hängt mit dem Kasus des Nomens zusammen (zu semantischen
%% Kasus\is{Kasus!semantischer} siehe Kapitel~\ref{sec-sem-kasus}).
%% Es ist dann eine Eigenschaft des Nomens, 
%% dass es als Modifikator verwendet werden kann, der Determinator ist dafür nicht verantwortlich.
%% Das Einfachste ist es anzunehmen, dass die Kopfmerkmale alle vom Nomen kommen. Alternativ könnte man
%% Determinatoren natürlich so spezifizieren, dass sie den \modw vom Nomen übernehmen oder die Kopf"=Informationen
%% des Determinators generell mit denen des Nomens vereinigen \citep{Netter96a}. Beide Möglichkeiten sind
%% aber komplizierter als die hier vorgestellte Analyse.
%% % Netter96a:100 SemP -> Semantik kommt vom Determinator

Ein zweites Argument gegen DP"=Strukturen liefern Beispiele mit relationalen Nomina, in denen der
Determinator eine semantische Rolle füllt:
\eal
\ex Peters Bruder
\ex Peters Zerstörung der Stadt
\zl
Für beide Beispiele gibt es eine Lesart, in der Peter ein semantisches Argument des Nomens ist. Wäre
\emph{Peters} kein syntaktisches Argument der jeweiligen Nomina, wäre unklar, wie das Linking
innerhalb der Lexikoneinträge für \emph{Bruder} bzw.\ \emph{Zerstörung} vorgenommen werden
könnte. In einer NP"=Analyse kann man dagegen das Linking im Lexikon vornehmen, da der Determinator
vom Nomen selegiert wird.%
\is{Determinator!-phrase|)}

\subsection{Lexikalische Abbindung von Determinatoren}

\mbox{}\is{Konstruktionsgrammatik (CxG)|(}\citet[\page 80]{Michaelis2006a} schlägt im Rahmen der
Konstruktionsgrammatik eine Analyse von Nominalphrasen wie (\mex{1}) vor, die davon ausgeht, dass der
Determinator über eine Lexikonregel abgebunden wird.
\eal
\ex Er trinkt Milch.
\ex Er mag Kinder.
\zl
Für Stoffnomina und Nomina im  Plural nimmt sie an, dass diese eine leere \sprl haben, also keinen
Determinator selegieren (zum \sprm siehe Kapitel~\ref{sec-spr}). Der semantische Beitrag des
Determinators wird bereits im Lexikon in die semantische Repräsentation des Nomens (in die \textsc{frames}"=Liste) aufgenommen. Folgende Repräsentation zeigt, wie Stoffnomina behandelt werden:\is{Skopus|(}
\ea
Schematischer Eintrag für Stoffnomina nach \citet[\page 80]{Michaelis2006a}:\\
\ms{
syn & \ms{ head & \ms[noun]{
                  count & $-$\\
                  agr & \ms{ per & 3\\
                             num & sg\\
                           }\\
                  }\\
           val & \ms{ spr   & \eliste\\
                      comps & \eliste\\
                    }\\
         }\\
sem & \ms{
      index & c\\
      frames & \liste{ \ibox{1} \ms[nominal]{
                                arg & c\\
                                }, \ms[exist]{
                                   arg   & c\\
                                   restr & \ibox{1}\\
                                   } }\\
      }\\
}
\z
Dabei steht \emph{nominal} für eine nominale Relation, die je nach Nomen noch spezifischer wird, und
\emph{exist} für den semantischen Beitrag des Quantors. Der Quantor hat unmittelbaren Skopus über
den Beitrag des Nomens, was man an der Strukturteilung \iboxt{1} sehen kann.

%% Diese Analyse hat zwei Probleme: Die Möglichkeit der Modifikation von Stoffnomina bzw.\ von Nomina im Plural und den
%% Skopus von skopustragenden Adjektiven. Das erste Problem soll anhand der Beispiele in (\mex{1})
%% erläutert werden, auf die bereits \citet[\page 319]{Netter94} hingewiesen hat:
Diese Analyse hat ein Problem mit der Modifikation von Stoffnomina bzw.\ von Nomina im Plural durch
Adjektive, das anhand der Beispiele in (\mex{1}) erläutert werden soll, auf die bereits \citet[\page 319]{Netter94} hingewiesen hat:
\eal
\ex[]{
der gute Wein
}
\ex[]{
guter Wein
}
\ex[*]{
guter der Wein
}
\zl 
Wenn man davon ausgeht, dass \emph{Wein} einen Lexikoneintrag hat, der keinen Determinator selegiert,
dann muss das Adjektiv in (\mex{0}b) eine vollständige NP modifizieren. Dann stellt sich aber die
Frage, wieso das Adjektiv eine vollständige NP, die einen Determinator enthält, nicht modifizieren
darf (\mex{0}c). Nimmt man dagegen an, dass in (\mex{0}b) ein leerer Determinator an der Stelle
steht, an der Determinatoren normalerweise stehen würden, dann folgt die Ungrammatikalität von
(\mex{0}c) automatisch aus der Analyse, die im Abschnitt~\ref{sec-Syntax-Kopf-Adjunkt} vorgestellt
wurde: Adjektive modifizieren \nbar{}s. Da \emph{Wein} nur einen Determinator verlangt, ist \emph{Wein} eine
\nbar. Die Kombination mit Adjunkten verändert die Valenz eines Kopfes nicht, so dass \emph{guter
  Wein} genauso wie \emph{Wein} einen Determinator verlangt. Bei Stoffnomina kann der Determinator
leer\is{leere Kategorie} sein, was in (\mex{1}) der Fall ist. 


\ea
{}[\sub{NP} \_ [\sub{\nbar} guter Wein]]
\z

\noindent
Als Alternative zu einem leeren Element bietet sich eine
Spezialregel an, die einen Kopf, der einen Determinator selegiert, zu einem vollständigen Kopf
projiziert und den semantischen Beitrag leistet, den der leere Determinator leisten
würde (siehe \citealt{Wunderlich87d} zur Vermeidung leerer Elemente). Abbildung~\vref{Abb-NP-Analysen} zeigt die Analysemöglichkeiten.
\begin{figure}
\hfill
\begin{forest}
sm edges
[NP
  [Det [\trace]]
  [\nbar
    [Adj [guter]]
    [\nbar [Wein]]]]
\end{forest}
%
\hfill
\begin{forest}
sm edges
[NP
  [\nbar
    [Adj [guter]]
    [\nbar [Wein]]]]
\end{forest}
\hfill
\begin{forest}
sm edges
[NP
  [Adj [guter]]
  [NP
    [\nbar [Wein]]]]
\end{forest}
\hfill\mbox{}

\caption{\label{Abb-NP-Analysen}Analysemöglichkeiten für determinatorlose Nominalphrasen: leerer
  Determinator, unäre Projektion, Lexikonregel}
\end{figure}
Die beiden ersten Analysen sind formale Varianten voneinander und machen in Bezug auf die
Modifikation durch Adjektive dieselben Vorhersagen. Allerdings ist ein Vorteil der ersten Analyse,
dass man ein Objekt annehmen muss, dass bis auf den phonologischen Gehalt Objekten entspricht, die es
bereits in der Grammatik gibt (siehe Kapitel~\ref{chap-lexikon} zur Organisation des Lexikons in der
HPSG). Folgt man der Analyse zwei, muss man eine sehr spezielle Grammatikregel stipulieren, die ein
Element aus der \sprl abbindet und eine Determinator"=Semantik mit der Kopfsemantik verrechnet. Die
Analyse mit dem leeren Determinator ist also vorzuziehen.\footnote{
  Man kann sich jetzt fragen, warum hier einem leeren Kopf der Vorzug gegeben wird, im letzten
  Abschnitt jedoch im Zusammenahng mit Possessivpronomina gegen einen leeren Determinator
  argumentiert wurde. Die Fälle unterscheiden sich dahingehend, dass für den hier angenommenen leeren
  Determinator ein direktes sichtbares Gegenstückt existiert (i), beim leeren Kopf, der eine NP und
  ein Possessivum selegiert, ist das jedoch nicht der Fall. Man muss erklären, wieso Possessiva
  dieselbe Distribution haben wie Determinatoren, und das ist erklärt, wenn man sie als
  Determinatoren analysiert. In der DP"=Analyse haben sowohl die Possessiva als auch der leere
  Determinator, der das Possesivum selegiert, einen besonderen Status.
\eal
\ex ein guter Wein
\ex \_ guter Wein
\zllast
}

%% Das zweite Problem, das sich für die lexikalische Abbindung des Determinators ergibt, ist ein
%% semantisches: Im Abschnitt~\ref{sec-kapselnde-Modifikation} wurden Adjektive vorgestellt, die den semantischen Gehalt der \nbar, die
%% sie modifizieren unter ihre Relation einbetten. Würde man nun den semantischen Beitrag des Quantors
%% im Lexikon einführen, bekäme man für den entsprechenden Ausschnitt aus (\mex{1}a) statt (\mex{1}b)
%% die Repräsentation in (\mex{1}c):
%% \eal
%% \ex Wenn die Post Hundefutter und "`Die Zeit"' "`angeblichen Wein"' vertreibt, darf die Weltwoche
%% nicht passen.\footnote{
%%   \url{http://www.weltwoche.ch/forum/threads.asp?AssetID=14074&ThreadID=0&ThreadOpen=1}. 24.01.2007
%% }
%% \ex $\exists$ x angeblich(wein(x)) $\wedge$ vertreiben(Zeit,x)
%% \ex angeblich($\exists$ x wein(x)) $\wedge$ vertreiben(Zeit,x)
%% %% \ex Mutmaßliche Mörder erholen sich (auf einer Südseeinsel).
%% %% \ex $\exists$ x mutmaßlich(mörder(x)) $\wedge$ erholen(x)
%% %% \ex mutmaßlich($\exists$ x mörder(x) $\wedge$ erholen(x))
%% \zl
%% In (\mex{0}b) wird gesagt, dass es ein Objekt x gibt, das \emph{Die Zeit} vertreibt und das angeblich
%% Wein ist. In (\mex{0}c) wird dagegen gesagt, dass \emph{Die Zeit} etwas vertreibt, was angeblich
%% existiert und Wein ist. Das ist semantisch nicht richtig, denn, dass es etwas gibt, was die Zeit vertreibt steht
%% außer Frage, fraglich ist nur, ob das, was sie vertreibt, Wein ist. Davon abgesehen, ist die Formel
%% in (\mex{0}c) nicht wohlgeformt, denn die Variable x in vertreiben(Zeit,x) ist nicht durch den
%% Quantor $\exists$ gebunden.

%% In den beiden linken Strukturen in Abbildung~\ref{Abb-NP-Analysen} bildet der Determinator bzw.\ das
%% Schema, das die Determinatorensemantik beisteuert, das höchste Element im Baum. Der Quantor schließt
%% die NP ab und hat Skopus über alles innerhalb der NP. In der dritten Analyse mit der Lexikonregel
%% hat der Quantor immer unmittelbaren Skopus über das Nomen, zu dem er gehört, was nicht den
%% beobachtbaren Fakten entspricht.%
%% \is{Skopus|)}
\is{Konstruktionsgrammatik (CxG)|)}

%\section*{Kontrollfragen}

\questions{
\begin{enumerate}
\item Wozu braucht man das \textsc{mod}- und das \textsc{spec}"=Merkmal?
\end{enumerate}
}

%\section*{Übungsaufgaben}

\exercises{
\begin{enumerate}
\item Wie sieht der Lexikoneintrag für das Adjektiv \emph{großem}, wie es in (\mex{1})
vorkommt, aus?
\eal
\ex mit großem Tamtam
\ex mit großem Eifer
\zl

\item Laden Sie die zu diesem Kapitel gehörende Grammatik von der Grammix"=CD
(siehe Übung~\ref{uebung-grammix-kapitel4} auf Seite~\pageref{uebung-grammix-kapitel4}).
Im Fenster, in dem die Grammatik geladen wird, erscheint zum Schluß eine Liste von Beispielen.
Geben Sie diese Beispiele nach dem Prompt ein und wiederholen Sie die in diesem Kapitel besprochenen
Aspekte.

\end{enumerate}
}