%% -*- coding:utf-8 -*-
%%%%%%%%%%%%%%%%%%%%%%%%%%%%%%%%%%%%%%%%%%%%%%%%%%%%%%%%%
%%   $RCSfile: hpsg-anhebung.tex,v $
%%  $Revision: 1.17 $
%%      $Date: 2008/03/31 11:54:36 $
%%     Author: Stefan Mueller (CL Uni-Bremen)
%%    Purpose: 
%%   Language: LaTeX
%%%%%%%%%%%%%%%%%%%%%%%%%%%%%%%%%%%%%%%%%%%%%%%%%%%%%%%%%


\chapter{Kohärenz, Inkohärenz, Anhebung und Kontrolle}
\label{Kapitel-anhebung}
\is{Verbalkomplex|(}

Im vorigen Kapitel wurde gezeigt, wie man Verbalkomplexe mittels
Argumentanhebung analysieren kann. Die Analyse wurde unter Bezug auf
Modalverben und Futur- und Perfekthilfsverben erklärt. Diese bilden
mit den Verben, die sie einbetten, immer einen Komplex.
Das ist jedoch nicht bei allen Verben, die verbale Argumente nehmen,
so. Gunnar \citet{Bech55a} hat das in einer herausragenden
Arbeit untersucht und Kriterien dafür zusammengetragen,
wann Komplexbildung vorliegt und wann nicht. Die Tests sollen
in diesem Kapitel besprochen werden. 

Außerdem werde ich die Unterschiede zwischen Anhebung und Kontrolle diskutieren.

\section{Die Phänomene}


Gunnar Bech hat zur Unterscheidung der sogenannten kohärenten von der inkohärenten
Konstruktion topologische Felder definiert. Die genaue Definition der
Begriffe ist leider sehr komplex, aber man kommt wohl nicht umhin,
sich damit auseinanderzusetzen, wenn man der Literatur zum deutschen
Verbalkomplex folgen können will. Im folgenden Abschnitt werden die Bechschen
Begriffe eingeführt, und im Abschnitt~\ref{sec-coh-incoh} werden dann
Tests zur Unterscheidung der kohärenten und inkohärenten Konstruktion vorgestellt.

\subsection{Die topologische Einteilung Bechs}
\label{topo-bech}

Um Phänomene wie Extraposition, 
%Oberfeldumstellung, 
Umstellung von Argumenten, verschiedene Stellungen von Verben im Verbalkomplex
und Skopus von Adverbien erklären zu können, 
definiert Bech die Begriffe Verbalfeld, Restfeld und Schlussfeld, % Oberfeld und Unterfeld, 
die ich im folgenden erklären will.

%\subsubsection{Die subordinative Kette}

Verbale Köpfe können eine Verbalprojektion als Argument verlangen. Der Kopf bestimmt
Eigenschaften seines Arguments, und bei verbalen Argumenten gehört die Verbform zu 
den vom Kopf bestimmten Eigenschaften. In den Sätzen (\mex{1}) bestimmt \emph{darf}
die Verbform von \emph{behaupten} und \emph{behaupten} die von \emph{zu kennen}.
\eal
\ex dass Conny den Mann zu kennen behaupten darf
\ex dass Conny behaupten darf, den Mann zu kennen
\zl
Eine Kette von Verben, die in Kopf"=Argument"=Beziehung stehen, nennt Bech eine subordinative
bzw.\ hypotaktische Kette.%
\is{Kette!subordinative}%
\is{Kette!hypotaktische}%
\is{hypotaktische Kette|see{Kette}}%
\is{subordinative Kette|see{Kette}}%
%% Braucht man nur für Oberfeld.
%% Er numeriert die Verben in Ketten durch und kennzeichnet sie mit verschiedenen Indizes.
%% Indizes rechts oben entsprechen dem Grad der Einbettung. $V^1$ ist das maximal übergeordnete
%% Verb. In (\mex{0}) ist $V^1$ = \emph{darf}, $V^2$ = \emph{behaupten} und $V^3$ = \emph{zu kennen}.

%\subsubsection{Verbalfeld, Kohärenzfeld, Restfeld und Schlußfeld}

\is{Verbalfeld|see{Feld}}%
\is{Feld!Verbal-}%
Zu jedem Verb gehört ein Verbalfeld (F), das das Verb selbst und alle nichtverbalen
Argumente des Verbs und alle Adjunkte des Verbs enthält.\footnote{
        Diese Festlegung ist in einer Grammatik mit Argumentanziehung
        (siehe Kapitel~\ref{Kapitel-Verbalkomplex}) etwas
        problematisch, da ja zum Beispiel in (i) \emph{die Frau} sowohl ein
        Argument von \emph{erkannt} als auch ein Argument von \emph{hat} ist.
        Es wird nur nicht als Argument von \emph{erkannt} gesättigt.
\ea
Conny hat die Frau erkannt.
\zlast%
}
Im Satz (\mex{1}) gibt es zwei Verbalfelder: $F^1$ = \emph{ich bitte ihn} und $F^2$ =
\emph{morgen zu kommen}.
\ea
Ich bitte ihn, morgen zu kommen.
\z
Die Zugehörigkeit zu Verbalfeldern ist nicht immer eindeutig:
\ea
\label{zu_kommen_versprach}
da Aicke nicht zu kommen versprach
\z
Folgende Auf"|teilungen in Verbalfelder sind möglich: $F^1$ = \emph{Aicke} + \emph{versprach},
$F^2$ = \emph{nicht zu kommen} oder $F^1$ = \emph{Aicke} + \emph{nicht} + \emph{versprach}, 
$F^2$ = \emph{zu kommen}.

Des weiteren führt Bech den Begriff des Kohärenzfeldes (K) ein.%
\is{Feld!Kohärenz-}
Ein Kohärenzfeld besteht aus einem Schlussfeld (S) und einem Restfeld (R).%
\is{Schlussfeld|see{Feld}}%
\is{Feld!Schluss-}%
\is{Restfeld|see{Feld}}%
\is{Feld!Rest-}
Das Schlussfeld steht immer nach dem Restfeld. Ein Schlussfeld enthält im allgemeinen
alle Verben des Kohärenzfeldes (\mex{1}a). Eine Ausnahme bildet -- wenn es existiert --
das Verb in der linken Satzklammer (\mex{1}b).
\eal
\ex dass ${\underbrace{\rule[-0.5ex]{0cm}{2.5ex}\mbox{Aicke nicht}
                              }_{R}~
                   \underbrace{\mbox{zu kommen versprach}
                              }_{\mbox{\footnotesize S}}}$
\ex ${ \underbrace{\mbox{Aicke versprach nicht}
                        }_{R}~
             \underbrace{\rule[-0.5ex]{0cm}{2.5ex}\mbox{zu kommen}
                        }_{\mbox{\footnotesize S}}.}$
\zl
Eine hypotaktische Kette von Verbalfeldern besteht aus einem (\mex{1}a) oder mehreren
(\mex{1}b) Kohärenzfeldern. Jedes Kohärenzfeld umfasst mindestens ein Verbalfeld. Bech trennt
Kohärenzfelder durch `$|$'\is{$\vert$} voneinander ab \citep[\S 77]{Bech55a}. 
%
%\is{\textvertline} % + = quote wegen german.sty und makeindex -g
%
Dieses Symbol entspricht einer Grenzpause.\is{Grenzpause}
%% \footnote{
%%         Zur Grenzpause siehe auch \citew[\page 33]{Drach37}.
%% }
`$|$' markiert die Stelle in einem Satz, an der beim Sprechen des Satzes 
eine Pause gemacht wird.
%% \NOTE{JB: Mir ist nicht klar, wie genau sich die VFer ergeben (ist sehr
%%         intuitiv/wenig konkret). Basiert (5b) auf der ersten Einteilung in VF? Oder ist das völlig unabhängig?}
\eal
\ex dass ${\overbrace{\underbrace{\rule[-0.5ex]{0cm}{2.5ex}{\textrm{Aicke~nicht}}
                                         }_{R}~
                   \underbrace{\textrm{zu~kommen~versprach}}_{\mbox{\footnotesize S}}}^{K}}$
%
\ex dass ${\overbrace{\underbrace{\rule[-0.5ex]{0cm}{2.5ex}{\textrm{Aicke}}
                                         }_{R_1}~
                              \underbrace{\textrm{versprach}}_{S_1}
                             }^{K_1}, |~
                   \overbrace{\underbrace{\rule[-0.5ex]{0cm}{2.5ex}{\textrm{nicht}}
                                         }_{R_2}~
                              \underbrace{\rule[-0.5ex]{0cm}{2.5ex}{\textrm{zu~kommen}}
                                         }_{S_2}
                             }^{K_2}}$
\zl
(\mex{0}b) unterscheidet sich von (\mex{-1}b) dadurch, dass es sich um
einen Verbletztsatz handelt, \dash, \emph{versprach} steht in der rechten Satzklammer und somit im
Schlussfeld. In (\mex{-1}b) steht \emph{versprach} in der linken Satzklammer und zählt deshalb zum
Restfeld. 

Ein Kohärenzfeld ist eine Gruppe von Verbalfeldern. Das schließt den Fall ein,
dass ein Kohärenzfeld aus genau einem Verbalfeld besteht. Das Kohärenzfeld umfasst alle Bestandteile der
zum Kohärenzfeld gehörenden Verbalfelder. Es bildet in topologischer
Hinsicht eine geschlossene Einheit. Ein Element eines Kohärenzfeldes kann nie zwischen
zwei Elementen eines anderen Kohärenzfeldes stehen. Elemente eines Verbalfeldes dagegen
können sehr wohl zwischen zwei Elementen eines anderen Verbalfeldes stehen (siehe (\ref{zu_kommen_versprach})).

Zwei Verbalfelder, die zur selben hypotaktischen Kette gehören, werden \emph{kohärent} genannt,%
\is{Kohärenz}
wenn sie zum selben Kohärenzfeld gehören, und \emph{inkohärent},% 
\is{Inkohärenz}
wenn sie zu zwei verschiedenen Kohärenzfeldern gehören.\footnote{
        \citeauthor{Bech55a} untersucht Kohärenzphänomene nur für Verbalfelder.
        In Analogie zum Begriff des Verbalfelds kann man
        auch den Begriff des Adjektivfeldes\is{Feld!Adjektiv-}
        einführen.
        In (i) gibt es die Felder $F^1$ = \emph{er wollte}, $F^2$ = \emph{sein} und
        $F^3$ = \emph{ihm immer treu} bzw.\ $F^1$ = \emph{er immer wollte}, $F^2$ = \emph{sein} und
        $F^3$ = \emph{ihm treu}.
        \ea
        dass er ihm immer treu\iwf{treu} sein wollte
        \z
        $F^1$, $F^2$ und $F^3$ bilden ein Kohärenzfeld.
%       In (i) gibt es die Felder $F^1$ = \emph{zu regieren},
%       $A^1$ = \emph{nicht fähig} und $F^2$ = \emph{jemand ist}. 
%       \ea
%       Wenn jemand nicht zu regieren fähig ist, soll er auch die Ämter nicht
%        besetzen. (Tagesschau, 12.10.95, Friedhelm Brebeck)
%       \z
%        $F^1$, $A^1$ und $F^2$ bilden ein Kohärenzfeld.
}
Der Satz in (\mex{1}) besteht zum Beispiel aus zwei Kohärenzfeldern \citep[\S 58]{Bech55a}.
\ea
${\overbrace{\rule{0cm}{2ex}{\textrm{Er~soll~den~Vater~gebeten~haben}}
                 }^{K_1},|~ 
      \overbrace{\rule{0cm}{2ex}{\textrm{den~Jungen~laufen~zu~lassen}}
                }^{K_2}.}$
\z
$F^1$ = \emph{er soll}, $F^2$ = \emph{haben}, $F^3$ = \emph{den Vater gebeten},
$F^4$ = \emph{den Jungen zu lassen}, $F^5$ = \emph{laufen}. $F^1 + F^2 + F^3$ und $F^4 + F^5$
bilden jeweils ein Kohärenzfeld. Keines der Felder $F^1$, $F^2$, $F^3$ ist mit einem Feld
außerhalb dieser Gruppe kohärent. Dasselbe gilt für $F^4$ und $F^5$.

Bech unterscheidet zwischen finiten und infiniten Kohärenzfeldern. Ein Kohärenzfeld ist genau
dann finit, wenn es ein finites Verb enthält. Es ist möglich, dass finite Kohärenzfelder 
kein Schlussfeld haben (\mex{1}).
\ea
Otto läuft nach Hause.
\z
Bei infiniten Kohärenzfeldern muss es nicht unbedingt ein Restfeld geben. (\mex{1}) ist ein Beispiel
für einen solchen Fall.
%% \NOTE{JB: Warum
%%   sind das zwei Kohärenzfelder? St.Mü.: Noch mal bei Bech nachlesen. Im Kohärenzfeld gibt es immer
%%   hypotaktische Kette, oder?}
\ea
${weil~\overbrace{\underbrace{\rule[-0.5ex]{0cm}{2.5ex}{\textrm{er~mir}}
                                   }_{R_1}~
                        \underbrace{\textrm{versprochen~hat}
                                   }_{\mbox{\footnotesize $S_1$}}}^{K_1}|~
        \overbrace{\underbrace{\rule[-0.5ex]{0cm}{2.5ex}{\textrm{zu~kommen}}
                              }_{\mbox{\footnotesize S$_2$}}}^{K_2}}$
\z
Für Schlussfelder gilt (mit Ausnahme der sogenannten Oberfeldumstellung\is{Oberfeldumstellung} und
der sogenannten dritten Konstruktion\is{dritte Konstruktion}, die hier nicht besprochen werden
können\NOTE{vielleicht Müller99a zitieren}), dass Verben links von den sie einbettenden Verben stehen. Da \emph{zu kommen} in (\mex{0})
nicht vor \emph{versprochen} steht, sondern rechts von allen übergeordneten Verben, wird es einem
anderen Kohärenzfeld zugeordnet. Es bildet ein eigenes Schlussfeld \citep[\S 78]{Bech55a}. Außerdem
ist das Kohärenzfeld K$_2$ intonatorisch durch die Grenzpause von K$_1$ abgegrenzt.

\subsection{Tests zur Unterscheidung kohärenter und inkohärenter Konstruktionen}
\label{sec-coh-incoh}

Ob\is{Kohärenz|(}\is{Inkohärenz|(} Verben in eine inkohärente Konstruktion eingehen können oder ob sie immer kohärent konstruieren,
ist eine Eigenschaft, die für die Klassifizierung von Verben wichtig ist. Im folgenden sollen einige
Tests vorgestellt werden, die Auskunft darüber geben, ob Verbalfelder zum gleichen Kohärenzfeld
gehören oder nicht, \dash ob ein Verb mit dem ihm untergeordneten Verb eine kohärente Konstruktion
bildet oder nicht.

\subsubsection{Skopus von Adjunkten}
\label{sec-skopus-kohaerent}

Adjunkte\is{Skopus|(}\is{Adverb!Skopus|(} können nur Skopus über Elemente haben, die sich im selben Kohärenzfeld
wie das Adjunkt befinden.
\ea
\label{ex-darf-zu-lesen-versuchen}
$\overbrace{ \mbox{Conny darf\iw{dürfen} das Buch nicht zu lesen versuchen} }^{K}$.
\z
Der Satz in (\mex{0}) ist dreideutig, wenn alle drei Verben zum selben Kohärenzfeld
gehören. Er kann bedeuten, dass es Conny gestattet ist zu versuchen,
das Buch nicht zu lesen (\mex{1}a), oder dass es Conny gestattet ist, nicht zu versuchen, das Buch zu lesen (\mex{1}b),
oder dass es Conny nicht gestattet ist zu versuchen, das Buch zu lesen (\mex{1}c).
\eal
\label{la}
\ex \relation{dürfen}(\relation{versuchen}($\neg$ \relation{lesen}(conny, buch)))
\ex \relation{dürfen}($\neg$ \relation{versuchen}(\relation{lesen}(conny, buch)))
\ex $\neg$ \relation{dürfen}(\relation{versuchen}(\relation{lesen}(conny, buch)))
\zl
In (\mex{1}) und (\mex{2}) liegen jeweils zwei Kohärenzfelder vor, so dass
sich die Anzahl der Lesarten pro Satz entsprechend reduziert.
\ea
$\overbrace{ \mbox{Conny darf nicht versuchen}}^{K_1},~\overbrace{\mbox{das Buch zu lesen}}^{K_2}$.
\z
\ea
$\overbrace{ \mbox{Conny darf versuchen}}^{K_1},~\overbrace{\mbox{das Buch nicht zu lesen}}^{K_2}$.
\z
In (\mex{-1}) gehört das \emph{nicht} zum selben Kohärenzfeld wie \emph{darf}
und \emph{versuchen}. Der Satz hat die beiden Lesarten in (\ref{la}b) und (\ref{la}c).
In (\mex{0}) dagegen gehört das \emph{nicht} in das Kohärenzfeld
von \emph{zu lesen}. (\mex{0}) hat nur die Lesart in (\ref{la}a). In (\mex{-1}) und (\mex{0})
konstruieren \emph{darf} und \emph{versuchen} jeweils kohärent und \emph{versuchen} und \emph{zu
lesen} inkohärent. In (\ref{ex-darf-zu-lesen-versuchen}) konstruieren \emph{darf} und
\emph{versuchen} und \emph{versuchen} und \emph{zu lesen} kohärent.
\is{Skopus|)}\is{Adverb!Skopus|)}

\subsubsection{Permutation im \mf}

Im folgenden Satz bilden \emph{zu lesen}, \emph{versprochen} und \emph{hat}
ein gemeinsames Schlussfeld.
\ea
\label{zu-lesen-versp}
weil $\overbrace{ \underbrace{\rule[-0.5ex]{0cm}{2.5ex}\mbox{es ihm jemand} }_{R}~
                    \underbrace{\rule[-0.5ex]{0cm}{2.5ex}\mbox{zu lesen versprochen\iw{versprechen} hat}}_{\mbox{\footnotesize S}}
                  }^{K}$\footnote{
        Siehe \textcites[\page 110]{Haider86c}[\page 128]{Haider90b}.
}
\z
Das Objekt von \emph{zu lesen} (\emph{es})
und das Subjekt und Objekt von \emph{versprochen} (\emph{jemand} und \emph{ihm})
bilden gemeinsam ein Restfeld. Die Elemente des Restfelds können in Bezug
auf einander in relativ freier Reihenfolge stehen, so dass das Komplement von \emph{zu lesen}
wie in (\mex{0}) nicht unbedingt adjazent zu seinem logischen Kopf sein muss.

Eine alternative Anordnung mit anderer lexikalischer Besetzung zeigt (\mex{1}):
\ea 
weil    ihm den Aufsatz jemand   zu lesen versprochen hat
\z
Mitunter ist die Umstellung von Argumenten in kohärenten Konstruktionen nur eingeschränkt möglich,
was auf Performanzfaktoren\is{Performanz}\footnote{
  Man unterscheidet zwischen Performanz und Kompetenz\is{Kompetenz}. Kompetenz beschreibt unser linguistisches
  Wissen, \dash Wissen über die mögliche Struktur sprachlicher Äußerungen. Performanz bezeichnet
  dagegen unsere Fähigkeit, mit diesem Wissen umzugehen. Das klassische Beispiel für ein
  Performanzproblem ist die Selbsteinbettung bei Relativsätzen\is{Relativsatz}. Je tiefer man Relativsätze nach dem
  Muster von (i) einbettet, desto schwerer sind die Sätze für Menschen zu verarbeiten.
\ea
Der Hund, [\sub{RS} der die Katze, [\sub{RS} die die Maus gefangen hat,] jagt, ] bellt.
\z
Die Regeln, die man für solche Einbettungen braucht, sind Bestandteil unseres Wissens, wir können
die Sätze aber mitunter dennoch nicht verarbeiten, da die Kapazitäten des Gehirns bei besonders
komplexen Strukturen nicht ausreichen \citep{MC63a-u}. Siehe auch
\citew[Kapitel~16]{MuellerGT-Eng5} für einen Überblick.%
} zurückzuführen ist \citep[\page 37]{BK89a}.
%\citealp[\page 530]{Abraham95a-u}.
%Ich werde darauf an entsprechender Stelle noch zurückkommen.
\citeauthor{BK89a} geben die Beispiele in (\mex{1}) und (\mex{2}):
\eal
\ex daß ihm ihr der Hans erlaubte zu helfen
\ex daß ihm ihr der Hans verbot zu helfen 
\zl
\eal
\ex daß ihn sie der Hans zwang zu suchen
\ex daß ihn sie der Hans überredete zu suchen
\zl
Sowohl \emph{erlauben} als auch \emph{helfen} verlangen einen Dativ. Genauso verlangen
\emph{zwingen} und \emph{suchen} einen Akkusativ. Hörer*innen können die Argumente nicht zuordnen,
so dass es zu einem Zusammenbruch der Verarbeitung kommt.




\subsubsection{Intraposition}

Ein\is{Intraposition|(} weiteres Kriterium zur Unterscheidung zwischen der kohärenten und der
inkohärenten Konstruktion ist die Möglichkeit, in Relativsätzen das eingebettete Verb mit
seinen abhängigen Elementen voranzustellen:
\eal
\label{inkohaerenz-rs}
\ex[]{
den Keks, den zu essen Conny versucht
}
\ex[*]{
den Keks, den essen Conny darf /  wird
}
\ex[*]{
den Keks, den gegessen Conny hat
}
\zl
Diese Konstruktion wird nach \citet*[\page 108]{Ross67} 
auch Rattenfängerkonstruktion\is{Rattenfängerkonstruktion} genannt,
weil das Relativpronomen das Material in der Verbphrase \emph{den zu essen} mitzieht (siehe auch S.\,\pageref{page-rattenfaenger}).

Verben, die obligatorisch kohärent konstruieren (Modalverben wie \emph{dürfen} und Hilfsverben wie
\emph{haben} und \emph{werden}), erlauben die Rattenfängerkonstruktion nicht (\mex{0}b,c). 

Für Sätze wie (\mex{0}a) gibt es mehrere Analysemöglichkeiten: Das Relativpronomen kann aus der
Infinitiv"=Phrase mit \emph{zu essen} extrahiert worden sein (\mex{1}a)
oder \emph{den zu essen} bildet die Relativphrase und ist als ganze Phrase extrahiert 
worden (\mex{1}b).\footnote{
  Solche Rattenfängerkonstruktionen wurden in den 80er Jahren kontrovers diskutiert
  (\citealp*{Riemsdijk85,Haider85c,Grewendorf86,Trissler88,Riemsdijk94}). Die Autoren
  versuchten zu zeigen, welche der beiden Strukturen in (\mex{1}) die besser geeignete ist.
  Meiner Meinung nach muss man sowohl die Voranstellung der gesamten Infinitivphrase
  als auch die Voranstellung des einzelnen Relativpronomens zulassen \citep[Kapitel~10.7]{Mueller99a}.
}
\eal
\ex den Keks, den$_i$ [ \_$_i$ zu essen] Conny versucht
\ex den Keks, [den zu essen]$_i$ \_$_i$ Conny versucht
\zl
In jedem Fall muss es bei der Analyse von (\mex{-1}) eine Verbalphrase
geben, die nicht adjazent zu den Verben im Verbalkomplex ist: entweder \emph{den zu essen} oder
\emph{zu essen} mit einer Lücke für die Relativphrase.

In Relativsätzen wie (\ref{inkohaerenz-rs}a) bilden die Verben mit den von ihnen abhängenden Elementen ein
eigenständiges Kohärenzfeld, \dash, \emph{zu essen} und das Relativpronomen stehen zusammen
und bilden ein Restfeld und ein Schlussfeld. Genauso bilden \emph{Conny} und \emph{versucht} ein
Restfeld und ein Schlussfeld. Die Analysen, die in (\mex{0}) skizziert sind, gehen davon aus, dass
\emph{versuchen} in der inkohärenten Konstruktion eine Verbalphrase einbettet. Diese kann dann als
Einheit umgestellt werden. Wenn man annimmt, dass sich inkohärent konstruierende Verben von
kohärent konstruierenden Verben dadurch unterscheiden, dass sie eine VP statt eines Verbs einbetten, dann kann man
auch die Umstellungen in (\mex{1}) als Inkohärenz"=Test heranziehen:
(\mex{1}a--b) sind ungrammatisch, da Modalverben\is{Verb!Modal-} wie \emph{dürfen} und Hilfsverben\is{Verb!Hilfs-} wie
\emph{haben} und \emph{werden} obligatorisch kohärent konstruieren. 
(\mex{1}c) dagegen ist möglich, da \emph{versuchen} inkohärent konstruieren kann.\NOTE{JB: Findet
  (\mex{1}a+b) bei entsprechend langer Pause okay, FB auch. FB: Nichts ist so wie es scheint zu sein}
\eal
\ex[*]{
dass Conny den Keks essen nicht darf
}
\ex[*]{
dass Conny den Keks essen nicht wird
}
\ex[]{
dass Conny, den Keks zu essen, nicht versucht
}
\zl
Wie auch bei der Extraposition ist die Infinitiv"=VP in Sätzen wie (\mex{0}c) oft
durch Pausen vom restlichen Material abgehoben.%
\is{Intraposition|)}

\subsubsection{Extraposition}

Wenn\is{Extraposition|(} das Matrixverb in einer inkohärenten Konstruktion vorkommen kann,
dann ist die Extraposition der Projektion des eingebetteten verbalen Kopfes möglich:
\ea
Conny hat versucht, den Keks zu essen.
\z
Das Verb \emph{versuchen} kann inkohärent konstruieren, und in (\mex{0}) 
bildet die Phrase \emph{den Keks zu essen} ein eigenes Kohärenzfeld.

Nicht alle \emph{zu}"=Infinitive können extraponiert werden. Anhebungsverben
wie \emph{scheinen} konstruieren obligatorisch kohärent.
Das Verb, das unter \emph{scheinen} eingebettet wird, befindet sich immer
im selben Kohärenzfeld wie \emph{scheinen}.
\eal
\ex[]{
dass Conny den Keks zu essen scheint
}
\ex[*]{
dass Conny scheint den Keks zu essen\NOTE{JB: findet das okay, bei vergessen statt essen wunderbar}
}
\zl
Die Extraposition von Infinitiven ohne \emph{zu} und von Partizipien
ist nicht möglich:
\eal
\ex[]{
dass Conny den Keks zu essen versucht
}
\ex[]{
dass Conny versucht, den Keks zu essen
}
\ex[]{\iw{werden!future}
dass Conny den Keks essen wird
}
\ex[*]{\iw{werden!future}
dass Conny wird den Keks essen
}
\ex[]{\iw{haben!perfect}
dass Conny den Keks gegessen hat
}
\ex[*]{\iw{haben!perfect}
dass Conny hat den Keks gegessen
}
\zl
\is{Extraposition|)}

\subsubsection{Voranstellung ins Vorfeld}
\label{sec-verb-fronting}
\is{Vorfeldbesetzung|(}

Die Voranstellung von VPen mit \emph{zu}"=Infinitiven ist immer möglich.
\ea
Den Keks zu essen, hat Conny versucht.
\z
Wie bei der Intraposition und der Extraposition bildet die vorangestellte
VP ein separates Kohärenzfeld.

Zusätzlich zu solchen Voranstellungen sind auch Voranstellungen von Verben
bzw.\ Verbalprojektionen möglich, die nicht intraponiert oder extraponiert werden können.\footnote{
        Die Beispiele in (\mex{1}) sind von \citet*[\page 720--721]{Haftka81a}.
}
\eal
\label{bsp-erzaehlen-wird}
\ex
Erzählen wird er seiner Tochter ein Märchen.
\ex
Ein Märchen erzählen wird er seiner Tochter.
\ex  
Seiner Tochter ein Märchen erzählen wird er.
\zl
Das Hilfsverb \emph{wird} konstruiert obligatorisch kohärent.
In (\mex{0}) gibt es verschiedene Arten von Voranstellungen:
In (\mex{0}a) ist das eingebettete Verb vorangestellt, und die Elemente, die
von ihm abhängen -- das direkte und indirekte Objekt -- bleiben im \mf zurück.
In (\mex{0}b) ist das Akkusativobjekt zusammen mit dem Verb vorangestellt,
und das Dativobjekt bleibt zurück, und in (\mex{0}c) befinden sich beide Objekte
zusammen mit dem Verb im \vf.

Man beachte, dass das vorangestellte Material eine separate\is{Skopus}\is{Adverb!Skopus} Skopusdomäne bildet.
(\mex{1}a) ist mehrdeutig, (\mex{1}b) dagegen hat nur die eine Lesart, in der \emph{nicht}
Skopus über \emph{gewinnen} hat.
\eal
\ex
dass er das Rennen nicht gewinnen darf
\ex
Das Rennen nicht gewinnen darf er.
\zl

\noindent
Eine interessante Eigenschaft solcher Voranstellungen ist, dass sie beliebig komplex
sein können, dass es jedoch nicht möglich ist, Teile aus der Mitte eines Verbalkomplexes
voranzustellen. Elemente, die ein anderes Element in einer kohärenten Konstruktion regieren,
können nicht ohne dieses Element vorangestellt werden:\footnote{
        \citet*[\page 720--721]{Haftka81a} gibt Beispiele ähnlicher Struktur,
        die andere ausgeschlossene Voranstellungen zeigen, die in den nächsten Kapiteln
        diskutiert werden.
}
\eal
\ex[]{
dass er ihr ein Märchen erzählen müssen wird
}
\ex[]{\label{ex-erzaehlen-muessen-wird-er-ihr}
Erzählen müssen wird er ihr ein Märchen.
}
\ex[*]{
Müssen wird er ihr ein Märchen erzählen.
}\label{ex-muessen-wird-er-ihr}
\zl
Es ist nicht der Fall, dass das Zurücklassen von Hilfsverben in der
rechten Satzklammer bei Voranstellung eingebetteter Verbalkomplexe im Deutschen
nicht möglich ist, wie das von \citet[\page 942]{SW94a}
behauptet wird. Ihr Satz (\mex{1}a) ist wegen allgemeiner Prinzipien zur Organisation
der Informationsstruktur\is{Informationsstruktur} eines Satzes merkwürdig und nicht wegen allgemeiner Beschränkungen
für Voranstellungen.
\eal
\label{bsp-gegessen-wird-er-wohl-den-braten-haben}
\ex[\S]{
Gegessen wird er wohl     den Braten haben.
%He probably will have eaten the roast
}
\ex[]{
Gegessen wird er den Braten wohl     haben.
}
\zl
% SW92:31
% Geschlafen kann er bis 10 Uhr haben.
% Geschlagen kann er von jedem Anwesenden sein.
Mit anderer Anordnung und daraus resultierendem anderen Skopus\is{Skopus}\is{Adverb!Skopus} von \emph{wohl} ist der Satz
wohlgeformt und selbst für (\mex{0}a) kann man sich einen speziellen Kontext überlegen, in dem der
Satz geäußert werden kann:\footnote{
  Ich danke Felix Bildhauer\aimention{Felix Bildhauer} für die Konstruktion dieses Kontextes.
}
\ea
Wir haben ihm verschiedene Getränke angeboten und auch Braten und Tofu. Getrunken hat er Wein und
gegessen wird er wohl den Braten haben.
\z
%(\mex{0}b)
%is well-formed and naturally occurring examples are easy to find. (\mex{1})
%is just one example.
%\ea
%\z

\noindent
\citet[\page 283]{Haider93a} behauptet etwas Ähnliches, nämlich, dass
die Komplemente von nicht"=finitem \emph{haben} nicht voranstellbar sind.
\eal
\ex[]{
Im Radio gehört hat er die Nachricht.
}
\ex[*]{
Im Radio gehört glaubt er die Nachricht zu haben.
}
\zl
Der Kontrast zwischen (\mex{0}a) und (\mex{0}b) ist klar, aber das liegt nicht
an \emph{haben}. \citet[\page 93]{Meurers2000b} gibt das Beispiel
in (\mex{1}).
\ea
Im Radio gehört wird er die Nachricht sicher nicht haben.
\z
Das Trennbarkeitsprinzip (\emph{Principle of Separability}\is{Prinzip!Trennbarkeits-}),
das \citet[\page 942]{SW94a}
formulieren, um das Voranstellen eines Basisverbs aus einer Partikelverbkombination
zu erklären, würde grammatische Sätze wie (\ref{ex-erzaehlen-muessen-wird-er-ihr}) 
und (\ref{bsp-gegessen-wird-er-wohl-den-braten-haben}) ausschließen und muss deshalb verworfen werden.
In Kapitel~\ref{sec-pvp} wurde bereits gezeigt, dass die Unmöglichkeit der
Voranstellung wie in Fällen wie (\ref{ex-muessen-wird-er-ihr}) aus einer allgemeinen
Bedingung für die Voranstellbarkeit von Teilen des Prädikatskomplexes folgt.
\is{Vorfeldbesetzung|)}\is{Kohärenz|)}\is{Inkohärenz|)}



\subsection{Anhebung und Kontrolle}
\label{sec-anhebung}
\is{Anhebung|(}
\is{Verb!Kontroll-|(}\is{Kontrolle|(}

Ob Verben in eine inkohärente Konstruktion eingehen können
oder ob sie immer kohärent konstruieren, ist eine Eigenschaft, die für die Klassifizierung
von Verben wichtig ist. Eine andere wichtige Klassifizierung betrifft die Zuordnung zur Klasse
der Anhebungs- bzw.\ Kontrollverben. Im folgenden sollen Eigenschaften der beiden Verbklassen 
vorgestellt werden.

\subsubsection{Einbettung expletiver Prädikate und subjektloser Konstruktionen}
\label{expl-pred-subj-constr}

Der wichtigste Unterschied zwischen Kontroll- und Anhebungsverben ist, dass das Subjekt
des eingebetteten Verbs bei Kontrollverben eine semantische Rolle\is{semantische Rolle} füllt.
\eal
\ex Conny versucht zu schlafen.
\ex \relation{versuchen}(Conny, \relation{schlafen}(Conny))
\zl
Man sagt, dass das Subjekt des eingebetteten Verbs kontrolliert wird. Die Kontrolle
kann wie in (\mex{0}a) durch das Subjekt oder durch ein Objekt erfolgen. Ein Beispiel
mit Kontrolle durch ein Akkusativobjekt zeigt (\mex{1}a):
\eal
\ex Conny zwingt den Mann, das Buch zu lesen.
\ex \relation{zwingen}(Conny, Mann, \relation{lesen}(Mann, Buch))
\zl
In (\mex{0}a) ist der Mann sowohl derjenige, der zu etwas gezwungen wird, als auch
derjenige, der das Buch lesen muss.

Anhebungsverben weisen dagegen keinem der Argumente des eingebetteten Verbs eine
semantische Rolle zu.\footnote{
        Siehe auch \citew[\page 347]{Puetz82a}.
        Zur Unterscheidung von Anhebungs"= und Kontrollverben an Hand
        der Einbettbarkeit expletiver Prädikate siehe auch
        \citew[\page 353]{Puetz82a}.
}
In (\mex{1}) füllt \emph{Conny} nur eine Rolle der Relation \emph{schlafen}.
\eal
\ex Conny scheint zu schlafen.
\ex \relation{scheinen}(\relation{schlafen}(Conny))
\zl
Dass es dennoch eine Beziehung zwischen dem Subjekt von \emph{schlafen} und dem Anhebungsverb
\emph{scheint} gibt, zeigt Subjekt"=Verb"=Kongruenz\is{Kongruenz!Subjekt"=Verb"=}:
\eal
\ex[]{
Die Männer scheinen zu schlafen.
}
\ex[*]{
Die Männer scheint zu schlafen.
}
\zl
Man sagt, dass das Subjekt von \emph{schlafen} zum Subjekt von \emph{scheinen} angehoben wurde.

Ein weiterer Unterschied zwischen Anhebungs"= und Kontrollverben, der sich daraus ableiten lässt, dass
Kontrollverben eine Rolle zuweisen, ist, dass Kontrollverben keine subjektlosen\is{Verb!subjektloses} Konstruktionen einbetten
können, wohingegen die Einbettung subjektloser Konstruktionen unter die meisten Anhebungsverben
möglich ist:
\eal
\ex[]{\iw{grauen}
dass    (es) dem Studenten       vor der Prüfung graut
}
\ex[*]{
Der Professor versucht, dem Studenten vor der Prüfung zu grauen.
}
\zl
Das Verb \emph{grauen} in (\mex{0}a) verlangt ein Dativobjekt und eine Präpositionalphrase.
Optional kann es auch mit Subjekt verwendet werden, doch das Subjekt muss dann ein Expletivum\is{Pronomen!Expletiv-} -- ein semantisch
leeres Füllwort -- sein. Wie (\mex{0}b) zeigt, kann \emph{grauen} nicht unter Kontrollverben eingebettet werden.
Das zeigt, dass weder die Variante mit dem Expletivum noch die subjektlose unter Kontrollverben
zulässig ist. Die Einbettung unter ein Anhebungsverb ist jedoch möglich. 
\ea
dass (es) dem Studenten vor der Prüfung zu grauen schien
\z

\noindent
Das Beispiel (\mex{1}b) ist eine andere Art subjektlose Konstruktion, das sogenannte unpersönliche Passiv.\is{Passiv!unpersönliches}
\eal
\ex[]{
Der Student arbeitet.
}
\ex[]{
dass gearbeitet wurde
}
\ex[*]{
Der Student versucht, gearbeitet zu werden.\label{ex-versucht-gearbeitet-zu-werden-c}
}
\zl
Die Einbettung der subjektlosen Konstruktion unter ein Kontrollverb ist wieder ungrammatisch (\mex{0}c), die
unter Anhebungsverben dagegen grammatisch, wie (\mex{1}) zeigt:
\ea
Dort schien noch gearbeitet zu werden.
\z
Kontrollverben haben Selektionsrestriktionen für das kontrollierende Argument, \dash für das Argument,
das mit dem Subjekt des eingebetteten Verbs koreferiert. Die Einbettung von expletiven Prädikaten,
also Prädikaten mit expletivem Subjekt, unter Kontrollverben ist ungrammatisch. Ein prominentes
Beispiel für expletive Prädikate sind die Witterungsverben:\footnote{
        Verschiedene \emph{es} werden mitunter als Quasi"=Argumente behandelt \citep[\page 324]{Chomsky93a},
        die Quasi"=Theta"=Rollen\is{semantische Rolle!Quasi"=Theta"=Rolle} zugewiesen bekommen.
        Als Argument hierfür wird immer angeführt, dass diese \emph{es} kontrolliert werden können:
        \ea
        Es blitzt, ohne gleichzeitig zu donnern.
        \z
        Solche Sätze sind mit der Behandlung des \emph{es} als echtes Expletivum durchaus
        kompatibel. Es wird nur verlangt, dass der Index des Subjekts des unter \emph{ohne} eingebetteten
        Infinitivs mit dem Index des Matrixsubjekts kompatibel ist, was in (i) der Fall ist.
        Der folgende Satz, der einem Beispiel von \citet[\page 84]{Fanselow91a} ähnelt,
        wird dadurch ausgeschlossen, dass \emph{ohne} das Vorhandensein eines Subjekts
        verlangt.
        \ea[*]{
        Heute wird getanzt, ohne gesungen zu werden.
        }
        \z
        Siehe hierzu auch Kapitel~\ref{sec-expletivum-in-obj-position}.%
}
\eal
\ex[*]{
Es versucht zu regnen.
}
\ex[*]{
Conny zwingt es zu regnen.
}
\zl
Egal ob man die \emph{es} in (\mex{0}) als Wetter"=\emph{es} oder als ein referentielles \emph{es}, das
Argument von \emph{zwingen} bzw.\ \emph{versuchen} sein kann, interpretiert, die Sätze sind ungrammatisch.
Anhebungsverben können dagegen Witterungsverben einbetten:
\eal
\ex Es scheint zu regnen.
\ex Er sah es regnen.
\zl


\subsubsection{Identität vs.\ Koindizierung}
\label{sec-rais-contr-identity-coindexing}

Das\is{Koindizierung|(} Verb \word{sehen} ist ein Anhebungsverb, was in Fällen wie
(\mex{1}), in denen ein Wetterverb\is{Verb!Wetter-} bzw.\ eine subjektlose Konstruktion eingebettet
ist, offensichtlich ist \parencites[\page 66]{Reis76c}[\page 70]{Hoehle78a}.
% Hoehle nur für sehen, sagt dass lassen ungrammatisch ist
\eal
\ex[]{ 
Conny sah es regnen.
}
\ex[?]{
Ich sah ihm       schlecht werden.
}
\zl
% Suche nach schlecht werden / warm/kalt werden fühlen in COSMAS erfolglos
%
Für Sätze wie (\mex{0}) kann man annehmen, dass das Subjekt des eingebetteten Verbs
mit dem Objekt des Matrixverbs identisch ist. Wenn das eingebettete Verb
kein Subjekt hat, wie in (\mex{0}b), dann hat das Matrixverb kein zusätzliches Objekt.
Das kann man erfassen, indem man festlegt, dass bei solchen Objektanhebungsverben das Subjekt
des eingebetteten Verbs und das Objekt des Matrixverbs identisch sind.
Ob das eingebettete Prädikat ein Subjekt hat oder nicht, ist dabei irrelevant.
Wenn das eingebettete Prädikat kein Subjekt hat, dann bekommt das Anhebungsverb kein
(zusätzliches) Objekt.

Die Frage ist, ob das bei Kontrollkonstruktionen auch so ist oder ob sich der Anhebungssatz in
(\mex{1}) von dem Kontrollsatz unterscheidet.
\eal
\ex\iw{sehen}
Der Wächter sah den Einbrecher   und seinen Helfer stehenbleiben.
\ex\iw{erlauben}
Der Wächter zwang den Einbrecher und seinen Helfer stehenzubleiben.
\zl
\word{zwingen} ist ein Objektkontrollverb, \dash, das Akkusativobjekt und das
nicht ausgedrückte Subjekt des eingebetteten Infinitivs sind koreferent.
Wegen der Daten in (\mex{-1}) scheint es angebracht anzunehmen, dass das Subjekt
von \emph{stehen bleiben} in Anhebungskonstruktionen wie (\mex{0}a)
mit \emph{den Einbrecher und seinen Helfer} identisch ist.
Im folgenden werde ich untersuchen, ob die Annahme einer solchen Identität für
Kontrollkonstruktionen wie in (\mex{0}b) sinnvoll ist.

\citet*[Kapitel~6]{Hoehle83a}
hat gezeigt, wie man den Kasus nicht sichtbar realisierter Elemente bestimmen kann. Mit der Phrase
{\em ein- nach d- ander-}\iw{ein- nach d- ander-|(uu} kann 
man sich auf mehrzahlige Konstituenten beziehen. Dabei muss {\em ein- nach d- ander-}
in Kasus und Genus mit der Bezugsphrase übereinstimmen.\is{Kongruenz!Adverbiale"=Subjekt} Höhles Beispiele wurden
bereits im Kapitel~\ref{sec-kasus-nicht-realisierter-subj} diskutiert (siehe Seite~\pageref{bsp-tueren-hoehle}).
Hier seien nur die Beispiele aus (\ref{bsp-nominativ-inkoh}) noch einmal wiederholt:
\ealnoraggedright
\label{bsp-nominativ-inkoh-zwei}
\ex Ich habe den Burschen geraten, im Abstand von wenigen Tagen einer nach dem anderen
      zu kündigen.\label{bsp-nominativ-inkoh-geraten-zwei}
\ex Die Türen sind viel zu wertvoll, um eine nach der anderen verheizt zu werden.
\ex Wir sind es leid, eine nach der anderen den Stuhl vor die Tür gesetzt zu kriegen.
\ex Es wäre fatal für die Sklavenjäger, unter Kannibalen zu fallen und einer nach dem
      anderen verspeist zu werden.
\zl
In (\mex{0}) ist {\em ein- nach d- ander-} nicht das Subjekt der Infinitivverbphrase,
da dieses in dieser Form Infinitivkonstruktion nie realisiert wird. {\em Ein- nach d- ander-} bezieht sich
jedoch auf das Subjekt. Daraus dass {\em ein- nach d- ander-} in (\mex{0}) im Nominativ
steht, kann man schließen, dass das nicht realisierte Subjekt ebenfalls im Nominativ
stehen muss.

In (\mex{0}a) ist der Kasus der kontrollierenden NP \emph{den Burschen} Dativ, wohingegen
der Kasus des kontrollierten Subjekts des \emph{zu}"=Infinitivs Nominativ ist,
wie man aus dem Kasus von \emph{einer nach dem anderen} schließen kann.\footnote{
        Adam Przepiórkowski\ia{Przepiórkowski, Adam} (P.\,m.\,1999) hat mich darauf hingewiesen,
        dass es im Polnischen\il{Polnisch} zwei Klassen von im Kasus übereinstimmenden Elementen
        gibt: zum einen die, die im Instrumental\is{Kasus!Instrumental} stehen, wenn sie sich auf ein
        unrealisiertes Subjekt beziehen, und zum anderen die, die dann im Dativ stehen.
        Würde man diese Elemente benutzen, um Rückschlüsse auf nicht realisierte Subjekte zu
        machen, käme man zu der Schlussfolgerung, dass diese im Polnischen sowohl Dativ als auch Instrumental
        sein können. In Sätzen ohne kasuskongruierendes Adverbial ist das problematisch, da man ja
        davon ausgeht, dass die Modelle maximal spezifische Werte haben (siehe
        Abschnitt~\ref{sec-modelle-theorien}). Man hätte dann für solche Sätze zwei Strukturen: eine
        mit Dativsubjekt und eine mit Instrumentalsubjekt.
        Aufgrund der polnischen Daten könnte man annehmen, dass nicht ausgedrückte Subjekte kasuslos sind
        und dass sich auf kasuslose NP beziehende Adverbialphrasen im Deutschen im Nominativ stehen müssen
        und im Polnischen im Dativ bzw.\ Instrumental.

        \citet{Hennis89} diskutiert Daten der Sprache Malayalam\is{Malayalam}. In
        Malayalam gibt es sowohl Nominativ- als auch Dativsubjekte. Sätze, in denen eine VP mit
        einem Nominativsubjekt mit einer VP mit Dativsubjekt koordiniert wurde, sind ungrammatisch.
        Sie schließt daraus, dass nicht ausgedrückte Subjekte Kasus haben müssen.
        Adam Przepiórkowski\aimention{Adam Przepiórkowski} hat mir mitgeteilt, dass das für das Polnische nicht gilt,
        \dash, man kann eine VP mit einer Adverbialphrase im Instrumental und eine VP
        mit einer Adverbialphrase im Dativ koordinieren.

% Bresnan noch mal prüfen
% 
        \citet{Andrews82b}, \citet{Neidle82a} und 
        \citet[\page 396]{Bresnan82c} diskutieren isländische\il{Isländisch} und
        russische\il{Russisch} Daten und schlagen vor, prädikative Adjunkte mit Kasuskongruenz
        parallel zu Anhebungskonstruktionen zu analysieren (funktionale Kontrolle\is{funktionale Kontrolle}
        in ihrer Terminologie) und prädikative Adjunkte ohne Kasuskongruenz parallel zu
        Kontrollkonstruktionen zu behandeln (anaphorische Kontrolle\is{anaphorische Kontrolle}\is{Kontrolle!anaphorische}
        in ihrer Terminologie).
        \citet[\page 404]{Neidle82a} diskutiert russische Daten und nimmt mit Comrie
        an, dass Subjekte nicht"=finiter Sätze im Russischen Dativ tragen. Sekundäre Prädikate,
        die als Adjunkte analysiert werden, stimmen mit ihrer Bezugsphrase im Kasus überein und
        stehen demzufolge bei Bezug auf ein nicht ausgedrücktes Subjekt im Dativ. Die prädikativen
        Phrasen, die im Instrumental stehen, behandelt sie als Komplemente, die nicht mit
        ihrem Subjekt im Kasus kongruieren. Wenn diese Analyse auch für das Polnische verwendbar
        ist, dann ist die Tatsache, dass sich sowohl Elemente im Dativ als auch solche im Instrumental
        auf dasselbe Element beziehen können, unproblematisch.

        Daten aus dem Isländischen, Russischen und Polnischen zeigen, dass sich die Sprachen darin unterscheiden, wie sie
        ihren (nicht ausgedrückten) Subjekten Kasus zuweisen. Da mir keine weiteren
        Tests bekannt sind, die man benutzen könnte, um den Kasus nicht ausgedrückter Subjekte
        im Deutschen zu bestimmen, werde ich bei der Annahme bleiben, dass Subjekte im Nominativ
        stehen.
        Selbst wenn man ein kasusloses Subjekt annehmen würde, könnte dieses nicht identisch
        mit der kasustragenden NP des Matrixverbs sein.
}
Das zeigt, dass das Subjekt des eingebetteten Verbs nicht identisch mit dem Objekt des Kontrollverbs
sein kann.

Ändert man das Genus des Pronomens in {\em ein- nach d- ander-}, ändert sich die Bedeutung des Satzes.
\ea
Ich habe [den Burschen]$_i$ geraten, im Abstand von wenigen Tagen [eine nach der an\-de\-ren]$_{*i}$ zu kündigen.
\z
(\mex{0}) ist nur grammatisch, wenn {\em ein- nach d- ander-} nicht ein Adjunkt ist, das 
sich auf das nicht"=overte Subjekt bezieht, sondern ein direktes Objekt von \emph{kündigen}.
Das ist erklärt, wenn man Kontrolle als Koindizierung der kontrollierenden Phrase mit dem nicht"=overten
Subjekt des eingebetteten Infinitivs beschreibt. Der Index von \emph{den Burschen} ist identisch
mit dem Index des nicht an der Oberfläche realisierten Subjekts. Deshalb kann kein Adjunkt,
das\NOTE{JB: Satz zu kompliziert.}
in Genus bzw.\ Numerus mit seinem Bezugsausdruck übereinstimmen muss, in der Domäne
des eingebetteten Infinitivs auf"|treten, wenn es sich auf das nicht"=overte Subjekt bezieht,
aber keine zum kontrollierten Element passenden Genus"= und Numerus"=Werte hat.%
\iw{ein- nach d- ander-|)uu}

Zu guter Letzt zeigen auch Beispiele wie (\mex{1}), dass eine Identität von kontrollierendem Element und kontrolliertem
Subjekt nicht adäquat ist, da in (\mex{1}) eine PP eine NP kontrolliert.\footnote{
        \citet*[\page 139]{ps2} geben das folgende englische\is{Englisch}
        Beispiel:
        \ea
        Kim appealed to Sandy to cooperate.
        \zlast
}
\ea\iw{erwarten}
Die Lehrer, von denen erwartet wird, diesen auf"|geputschten Kohlehydratkolossen etwas beizubringen,
verdienen jedermanns Anteilnahme.\footnote{
        Max Goldt, \emph{Die Kugeln in unseren Köpfen}. München: Wilhelm Heine Verlag. 1997, S.\,145.
}
\z
Die Kontrollbeziehungen sind im folgenden einfacheren Satz leichter zu durchschauen:
\eal
\ex Man erwartet von Aicke, den Kindern etwas beizubringen.
\ex \relation{erwarten}(man, Aicke, \relation{beibringen}(Aicke, Kindern, etwas))
\zl
Im skizzierten Bedeutungsbeitrag von (\mex{0}a) in (\mex{0}b) sieht man, dass Aicke sowohl eine Rolle
von \emph{erwarten} als auch von \emph{beibringen} füllt. Die syntaktische Kategorie der jeweiligen
Argumente von \emph{erwarten} und \emph{beibringen} ist aber PP bzw.\ NP. Die syntaktische Kategorie
kann also nicht identisch sein, der semantische Index jedoch schon.

Die Kongruenzeigenschaften von {\em ein- nach d- ander-} helfen übrigens auch
bei der Bestimmung des Skopus\is{Skopus} in kohärenten Konstruktionen:
\eal
\ex Der Wächter erlaubte den Einbrechern    einem     nach  dem anderen wegzulaufen.
\ex Der Wächter erlaubte den Einbrechern,   einer     nach  dem anderen wegzulaufen.
\zl
In (\mex{0}a) ist nur Skopus über \emph{erlauben} möglich, da die Adjunktphrase mit einem Objekt dieses Verbs
übereinstimmt. In (\mex{0}b) dagegen ist nur der Skopus über \emph{weglaufen} möglich, da die Adjunktphrase
mit dem nicht"=overten Subjekt von \emph{weglaufen} übereinstimmt.

Interessanterweise ist das bei Anhebungsprädikaten anders:\footnote{
       Wie Kordula De Kuthy (persönliche Mitteilung, 1998) festgestellt hat, scheint der Satz (\mex{1}b) besser zu werden,
       wenn man statt \emph{die Männer} das Pronomen \emph{sie} verwendet.
        \ea[?*]{
        Der Wächter sah sie$_{i}$    [einer     nach  dem anderen]$_{i}$ weglaufen.
        }
        \z
        Das Pronomen ist hinsichtlich seines Kasus morphologisch unterspezifiziert. Für manche Sprecher
        ist der Nominativ auch mit vollen, morphologisch eindeutig markierten NPen möglich.
}
\eal
\ex[]{\iw{sehen}
Der Wächter sah den Einbrecher und seinen Helfer einen nach  dem  anderen weglaufen.
}
\ex[*]{
Der Wächter sah den Einbrecher und seinen Helfer einer nach  dem anderen weglaufen.%\NOTE{JB findet       den grammatisch.}
}
\zl
Bei Anhebungsverben sind die Nominativ"=Adjunktphrasen ungrammatisch, was darauf hindeutet,
dass das Subjekt des eingebetteten Prädikats identisch mit dem Objekt des Matrixverbs ist, \dash
sowohl syntaktische als auch semantische Information wird geteilt, und deshalb stehen sowohl das Objekt
des Matrixsatzes als auch das Subjekt des eingebetteten Prädikats im Akkusativ.
\is{Anhebung|)}
\is{Verb!Kontroll-|)}\is{Kontrolle|)}\is{Koindizierung|)}

Nach der generellen Diskussion von Kohärenz und Inkohärenz und von Anhebung und Kontrolle
sollen nun die Eigenschaften bestimmter Verbklassen untersucht werden. Dabei ist besonders
interessant, ob die Eigenschaften unabhängig voneinander sind.

\subsection{Subjektanhebungsverben}
\label{sec-subj-raising-verbs-phen}
\label{sec-phase-verbs-phen}

Die\is{Kohärenz|(}\is{Inkohärenz|(}\is{Verb!Subjektsanhebungs-|(} meisten Subjektanhebungsverben kommen nur in kohärenten Konstruktionen vor.
Die Phasenverben (Verben wie \emph{beginnen}, \emph{anfangen} und \emph{aufhören})
können allerdings auch in inkohärenten Konstruktionen vorkommen.

\subsubsection{Skopus von Adjunkten}
\label{sec-subj-rais-scope}

Das Beispiel in (\mex{1}) zeigt, dass bei \word{scheinen} sowohl enger als auch weiter
Skopus\is{Skopus}\is{Adverb!Skopus} des Adjunkts möglich ist.
\eal
dass Conny Kirby nicht zu lieben scheint
\zl
In der einen Lesart scheint es der Fall zu sein, dass Conny Kirby nicht liebt (enger Skopus) und in der
anderen Lesart scheint es nicht der Fall zu sein, dass Conny Kirby liebt.


%\subsubsection{Expletive Predicates and Subjectless Constructions}
%\label{sec-subj-rais-expl}
%\eal
%\ex\iw{es!expletiv}\iw{regnen}
%\gll Es scheint zu regnen.\\
%     it seems   to rain\\
%\ex
%\gll Dem Mann scheint geholfen zu werden.\\
%     the man  seems   helped   to get\\
%\zl


\subsubsection{Permutation im \mf}
\label{sec-subj-rais-perm-mf}

Die Beispiele in (\mex{1}) zeigen, dass NPen, die vom eingebetteten Verb abhängen,
mit NPen, die vom Matrixverb abhängen, vertauscht werden können:
\eal
\ex 
dass niemandem    der Mann zu schlafen scheint
\ex 
dass der Mann niemandem zu schlafen scheint
\zl
%Es hatte schon zu dunkeln begonnen.\footnote{
%        \citep[\page 112]{Bech55a}.
%}
%Da es stärker zu regnen anfing, lenkten sie ihre Schritte gegen das Haus.\footnote{
%        \citep[\page 112]{Bech55a}.
%}
Das Subjekt eines Phasenverbs\is{Verb!Phasen-} kann ebenfalls mit einem Objekt des
eingebetteten Verbs vertauscht werden:
\eal
\ex 
Leise   begann der Tote     sich zu bewegen.
\ex 
Leise   begann sich der Tote     zu bewegen.\footnote{
        \citew[\page 121]{Bech55a}.
    }
\zl


\subsubsection{Intraposition und Extraposition}
\label{sec-subj-rais-ie}

Die\is{Extraposition|(}\is{Intraposition|(} meisten Anhebungsverben erlauben weder die Intraposition (\mex{1}b) noch
die Extraposition des Infinitivs (\mex{1}c), so dass manchmal behauptet wird, dass Anhebungsverben
immer obligatorisch kohärent konstruieren (siehe \zb \citew[\page 128]{Haider90b}).
\eal
\label{bsp-intra-extra-scheinen}
\ex[]{
dass Conny Kirby zu lieben scheint
}
\ex[*]{
dass Conny Kirby zu lieben zumindest scheint
}
\ex[*]{
dass Conny scheint, Kirby zu lieben\NOTE{FB: findet den okay}
}
\zl

% Haider93a:244 Idioms
\noindent
Sogenannte Phasenverben\is{Verb!Phasen-|(} wie \word{anfangen}, \word{aufhören} und 
\word{beginnen} sind jedoch Ausnahmen \citep[\page 18]{Kiss95a}. 
In (\mex{1}b) ist ein Verb mit expletivem Subjekt unter \emph{anfangen} eingebettet
und der Infinitiv ist extraponiert.
\eal
% \ex 
% Er hatte das Buch zu lesen begonnen.
% \ex
% Er hatte begonnen, das Buch zu lesen.
\ex
dass    es   wie aus Kübeln zu regnen begonnen hatte
\ex 
dass    es  begonnen hatte, wie aus Kübeln zu regnen
% Kontrolle
%\ex Da hatte das Schwein angefangen zu schreien.\footnote{
%        \citep[\page 120]{Bech55a}.
%}
%\ex sobald die Buchen anfangen, farbig zu werden.\footnote{
%        \citep[\page 120]{Bech55a}.
%}
\zl

\noindent
Dass es sich bei den Phasenverben wirklich um Anhebungsverben handelt, sieht
man daran, dass sie Prädikate mit expletivem Subjekt (\mex{0}) und subjektlose Konstruktionen (\mex{1})
einbetten können.

\ea
dass dem Studenten vor der Prüfung zu grauen anfing
\z
%% \eal
%% \ex Es hatte zu regnen angefangen.
%% \ex Es hatte angefangen zu regnen.
%% \zl

\noindent
Bisher wurden in der Literatur als Beleg dafür, dass Phasenverben auch inkohärent konstruieren können,
nur Extrapositionsbeispiele angeführt. \citet{Reis2005a} hat darauf hingewiesen, dass
die Extrapositionsdaten auch als eine andere Konstruktion, nämlich als die sogenannte
"`dritte Konstruktion"'\is{dritte Konstruktion|(}\footnote{
	Die Bezeichnung dritte Konstruktion stammt von \citet*{dBR89}.%
	Sie haben sie für eine ähnliche Konstruktion im 
	Niederländischen\is{Niederländisch} verwendet. 
	\Citeauthor{dBR89} zeigen, dass es sich bei der Konstruktion weder 
	um die reine Bildung eines Verbalkomplexes noch um normale Extraposition,
	sondern eben um eine dritte Konstruktion handelt. \citet*{BBHR95}
        nennen die dritte Konstruktion \emph{Remnant Extraposition}.
%
        \citet*[\page 145]{Wunderlich80} nennt diese Konstruktion "`Extraposition schwerer Infinitivketten"', und
	\citet*[\page 151]{Uszkoreit87a} verwendet den Begriff \emph{Focus Raising}.
        Auf die Analyse der dritten Konstruktion\is{dritte Konstruktion} kann hier nicht eingegangen werden.
        Siehe jedoch \citew[Kapitel~17.5]{Mueller99a}.%
} analysiert werden könnten. Die folgenden Beispiele zeigen jedoch,
dass auch die Intraposition von unter Phasenverben eingebetteten Infinitiven möglich ist:
\eal
\ex von seiner Gewandtheit, alte Bilder wiederherzustellen,  darf ich [zu erzählen] nicht anfangen,%
%weil man zugleich die schwere  Aufgabe und die glückliche Lösung , womit sich diese eigene Handwerkskunst  beschäftigt , entwickeln müßte .
\footnote{
 Goethe, \emph{Italienische Reise}, Hamburger Ausgabe, Band 11, S.\,207.
}
\ex so ging es auch mir, der ich, in Ermangelung einer vorzüglichen Bühne, [über das deutsche Theater zu denken] nicht auf"|hörte,\footnote{
  Goethe, \emph{Dichtung und Wahrheit}, Hamburger Ausgabe, Band 9, S.\,566.
}
\ex\label{bsp-seefahrer} 
In der Tat ist es auch heute noch beeindruckend, 
    mit welcher nie erlahmenden Energie der Seefahrer 
    [seine Idee einer Westfahrt nach Asien, allen Widerständen und Rückschlägen zum Trotz, 
    zu propagieren] nicht auf"|hörte.\footnote{
      Salzburger Nachrichten, 31.12.1991; URS BITTERLI.%
    }
\zl
\eal
\ex Für manche einfache Schützen wurde aus dem Krieg ein großes Abenteuer, [von dem zu erzählen] 
    sie kaum mehr aufhören konnten.\footnote{
  Oberösterreichische Nachrichten, 22.11.1996; Vom Grauen des Vernichtungskrieges im Osten.%
}
\ex Von dieser Wirkstätte aus hatte es ihn auch häufig in die belgischen Industrielandschaften gezogen, [die zu zeichnen] er bis heute nicht aufgehört hat.\footnote{
  Die Presse, 30.06.1998; Jubiläum.%
}
\zl
In (\mex{-1}) liegen Umordnungen des eingebetteten Infinitivs im Mittelfeld vor. Bei (\mex{0}) handelt
es sich um die Voranstellung einer Relativphrase.\is{Intraposition|)}

Das Verb \word{versprechen} gibt es sowohl als Anhebungs"= als auch als Kontrollverb. Der Satz
(\mex{1}a) hat zwei Lesarten. 
\eal
\label{bsp-kontrast-raising-extraposition}
\ex
dass Aicke ein erfolgreicher Sportler zu werden versprach
\ex\label{ex-versprach-ein-x-zu-werden} 
dass Aicke versprach, ein erfolgreicher Sportler zu werden
\zl
In einer Lesart versprach Aicke etwas, in der anderen
wird ausgesagt, dass es wahrscheinlich war, dass Aicke ein erfolgreicher
Sportler werden würde. In der ersten Lesart handelt es sich um das Kontrollverb, in der
zweiten um das Anhebungsverb.
Das Anhebungsverb konstruiert obligatorisch kohärent und das Kontrollverb
\emph{versprechen} optional kohärent. In (\mex{0}b) ist das Infinitivkomplement extraponiert.
Es liegen zwei Kohärenzfelder vor, \dash, es handelt sich um inkohärente Konstruktionen, 
und wie zu erwarten hat (\mex{0}b) auch nur eine Lesart, nämlich die des optional kohärent
konstruierenden Kontrollverbs \emph{versprechen}, also die Lesart, in der
Aicke etwas versprach. Siehe hierzu auch \citew[\page 5]{Netter91}.

\citet[\page 43]{Meurers2000b} benutzt die von mir gefundenen Beispiele 
in (\mex{1}) zusammen mit Beispielen, die Phasenverben enthalten, um zu zeigen,
dass Kohärenz und Anhebung unabhängige Phänomene sind.
\eal\iw{drohen}
\ex 
Im Herbst schließlich stoppte Apple die Auslieferung einiger Power Books, weil sie drohten
     sich zu überhitzen und in Flammen auf"|zugehen.\footnote{
         taz 20./21.01.1996, S.\,7.
    }
\ex
Das elektronische Stabilitätsprogramm ESP überwacht die Fahrzeugbewegungen und greift 
     in kritischen Situationen ein, wenn der Wagen droht, außer Kontrolle zu geraten.\footnote{
        Spiegel, 41/1999, S.\,103.
    }
\zl
\citet[\page 189]{Fanselow87a} diskutiert das Beispiel in 
(\mex{1}a) und \citet[\page 279]{Rosengren92a} und
\citet[\page 16]{Cook2001a} diskutieren die Beispiele in (\mex{1}b) bzw.\ (\mex{1}c):\footnote{
  Es gibt eine Verwendung von \emph{weil} mit V2-Satz. Ein solcher könnte in
  (\mex{1}b) vorliegen. Für eine korrekte Beurteilung der Daten sollte man also
  (i) betrachten.
  \ea
  dass das Wetter verspricht, heiter zu werden
  \zlast
}
\eal
\ex Ludwig der Deutsche glaubt nicht, dass der Rhein droht, über die Ufer zu treten.
\ex weil das Wetter verspricht, heiter zu werden
\ex obwohl heute verspricht, ein wunderschöner Tag zu werden
\zl

%% Die Sätze in (\mex{-1}) und (\mex{0}) scheinen mir nicht vollständig akzeptabel zu sein und der Grund für ihre Markiertheit
%% dürfte sein, dass sie in das Linearisierungsmuster der inkohärenten Konstruktion gezwungen wurden,
%% das nur mit der Kontrollesart von \emph{drohen} möglich ist. 

\noindent
Ich würde die Sätze in (\mex{-1}) und (\mex{0}) als Ausnahmen einordnen, aber \citet{Reis2005a} 
zitiert eine Korpusuntersuchung von \emph{drohen}/""\emph{versprechen} in der Anhebungsversion,
derzufolge im COSMAS"=Korpus bei 5,4\% der Belege für \emph{drohen} (45 von 828 Belegen)
und bei 6,6\% der Belege für \emph{versprechen} (30 von 458 Belegen) Extraposition vorliegt.
Es scheint also adäquater zu sein, für Verben wie \emph{drohen} und \emph{versprechen} 
beide Muster zuzulassen. Den Kontrast in (\ref{bsp-kontrast-raising-extraposition}) müsste man dann über Präferenzen erklären.
Reis argumentiert -- wie bereits erwähnt -- dafür, die Extrapositionsvarianten 
nicht als inkohärente Konstruktion, sondern als "`dritte Konstruktion"' zu analysieren. Da die
dritte Konstruktion nicht von allen Sprechern gleichermaßen akzeptiert wird, wären bei
einer solchen Analyse auch die Unterschiede in der Bewertung von Daten wie (\mex{-1}) und (\mex{0})
erklärt. Außerdem ist erklärt, warum \emph{drohen} und \emph{versprechen} mit der entsprechenden Lesart
keine Intraposition zulassen.%
\is{dritte Konstruktion|)}\is{Extraposition|)}

Die Phasenverben wären so die einzige Unterklasse der Anhebungsverben,
die eine inkohärente Konstruktion erlauben.
\is{Verb!Phasen-|)}\is{Verb!Subjektsanhebungs-|)}

\subsection{Subjektkontrollverben}

Die\is{Verb!Kontroll-|(}\is{Verb!Subjektkontroll-|(} meisten der Beispiele, die in diesem Abschnitt diskutiert werden, habe
ich schon im Abschnitt~\ref{sec-coh-incoh} diskutiert, wo es um Kohärenztests ging.


\subsubsection{Skopus von Adjunkten}
\label{sec-subj-contr-scope}

Wie\is{Skopus|(}\is{Adverb!Skopus|(} bereits im Abschnitt~\ref{sec-coh-incoh} gezeigt wurde, können Subjektkontrollverben
kohärent konstruieren.
In kohärenten Konstruktionen ist weiter Adjunktskopus möglich.
\ea\iw{versprechen}
dass Conny ihm nicht einzuschlafen verspricht
\z
%
Der Satz (\mex{0}) kann bedeuten, dass Conny jemanden verspricht, nicht einzuschlafen. Er kann aber
auch bedeuten, dass Conny jemandem nicht verspricht einzuschlafen.


Das Beispiel in (\mex{1}) ist ein Korpus"=Beleg für eine kohärente Konstruktion, in der
\word{nicht} und \word{etwas} semantisch zu \emph{nichts} verschmolzen sind.
\citet[{\S}\,80]{Bech55a} nennt solche Verschmelzungen \emph{Kohäsion}\is{Kohäsion}.
\ea\iw{trauen}
\label{bsp-trauen-kohaerent}
daß ich den Betrug mitbekommen hatte und mich nur nichts zu sagen traute\footnote{
        Jochen Schmidt, \emph{Müller haut uns raus.} München: Verlag C.\,H.\,Beck. 2002, S.\,277.%
}
\z
Das \emph{nicht} hat, obwohl es Bestandteil eines Arguments von \emph{sagen} ist,
Skopus über \emph{trauen}.\is{Skopus|)}\is{Adverb!Skopus|)}


\subsubsection{Permutation im \mf}
\label{sec-subj-contr-perm-mf}

Wie die Beispiele in \fromto{\mex{1}}{\mex{2}} zeigen, gibt es Subjektkontrollverben, die die Permutation
der Argumente des Matrixverbs und des eingebetteten Verbs erlauben.
\eal
\ex dass niemand das Buch zu lesen versucht
\ex dass das Buch niemand zu lesen versucht
\zl
In (\mex{1}) hat das Subjektkontrollverb zusätzlich zum kontrollierenden Subjekt noch ein Dativobjekt.
\eal\iw{versprechen}
\ex dass Conny dem Mann das Buch zu lesen verspricht
\ex dass Conny das Buch dem Mann zu lesen verspricht
\zl

\noindent
In Beispielen mit Pronomina ist die Anordnung des kurzen Pronomens \emph{es}
links der Argumente des Matrixverbs die bevorzugte Anordnung.
\ea
\label{ex-weil-es-ihm-jemand-zu-lesen-versprochen-hat-zwei}
weil    es       ihr       jemand   zu lesen versprochen hat\footnote{
\textcites[\page 110]{Haider86c}[\page 128]{Haider90b}.
}
\z
Es wird oft behauptet, dass Kontrollverben, die ein Objekt verlangen, nicht in kohärenten
Konstruktionen vorkommen.
\emph{versprechen} ist ein Subjektkontrollverb mit Dativobjekt, das
in kohärenten Konstruktionen vorkommt.
Im Abschnitt~\ref{sec-obj-control-phen} werde ich zeigen, dass kohärente Konstruktionen
auch mit Objektkontrollverben vorkommen, obwohl das oft ausgeschlossen wird.
%\eal
%\gll weil   es ihn niemand lesen zu dürfen    gebeten hat

\subsubsection{Intraposition und Extraposition}
\label{sec-subj-contr-ie}

Subjektkontrollverben, die \emph{zu}"=Infinitive regieren, 
erlauben sowohl die Intraposition\is{Intraposition} (\mex{1}) als auch die Extraposition\is{Extraposition}
(\mex{2}) ihrer Infinitivkomplemente.
\ea
dass Conny, das Rennen zu gewinnen, nicht versuchen will
\z
\ea
dass    Conny versuchen will,    das Rennen zu gewinnen
\z
\is{Verb!Kontroll-|)}\is{Kohärenz|)}\is{Inkohärenz|)}%
\is{Verb!Subjektkontroll-|)}

%---------------------------------------------------------------------------------------------
\subsection{Objektanhebungsverben: AcI-Verben}
\label{sec-aci}

AcI\is{Verb!Objektanhebungs-|(}\is{Verb!AcI-|(}\is{Kohärenz|(}
steht für \emph{Akkusativ mit Infinitiv}. 
%AcI"=Verben werden auch \emph{Exceptional Case Marking Verbs} (ECM"=Verben) genannt. 
Beispiele für AcI"=Verben sind Wahrnehmungsverben\is{Verb!Wahrnehmungs-} wie \word{hören} und 
\word{sehen}, sowie das permissive und kausative\is{Verb!kausatives} \word{lassen}. 

\subsubsection{Skopus von Adjunkten}
\label{sec-aci-scope}

Im\is{Skopus|(}\is{Adverb!Skopus|(} Beispiel (\mex{1}) kann die Negation\is{Negation} wie bei anderen kohärenten Konstruktionen
Skopus über beide Verben haben.
\ea
dass ich den Jungen das Buch nicht holen ließ
\z
%oder: That I didn't let the boy get the book
%
Bei Wahrnehmungsverben kann man die verschiedenen Skopen der Negation
aus semantischen Gründen nicht beobachten, da man zum Beispiel
nicht hören kann, wie jemand nicht singt. Allerdings kann man
die Skopusphänomene bei anderen Adjunkten genauso feststellen, wie
das folgende Beispiel von \citet[\page 340]{Puetz82a} zeigt:
\ea
Peter hat es im Laboratorium blitzen sehen.
\z
In der einen Lesart blitzt es im Laboratorium und Peter sieht das, und
in der anderen Lesart befindet sich Peter im Laboratorium und sieht,
wie es blitzt. Wo es blitzt, ist nicht gesagt. Das kann außerhalb
des Laboratoriums sein.\is{Skopus|)}\is{Adverb!Skopus|)}


\subsubsection{Permutation im \mf}
\label{sec-aci-perm-mf}


Das Subjekt von \emph{lassen} kann nach dem Subjekt des eingebetteten Verbs stehen, wie (\mex{1})
zeigt.
\ea
daß ihn der Max nicht schlafen ließ\footnote{
        \citew[\page 260]{Haider93a}.
}
\z

\noindent
Es wird manchmal behauptet, dass der Akkusativ des Matrixverbs
vor dem Akkusativ des eingebetteten Verbs stehen muss \citep[\page 356]{Eisenberg99a}.
Die Beispiele (\mex{1}b), (\mex{2}) und (\mex{3}) zeigen, dass das nicht richtig ist.
%
(\mex{1}a) zeigt die Reihenfolge, in der das Komplement von \emph{holen} adjazent zum Verb ist,
und in (\mex{1}b) ist das Objekt des eingebetteten Verbs von diesem durch den Akkusativ
getrennt, der das logische Subjekt von \emph{holen} ist.
\eal
\label{ex-liess-den-Jungen-das-Buch-holen}
\ex\iw{lassen} 
Ich ließ den Jungen    das Buch       holen.
\ex 
Ich ließ es       (das Buch)      den Jungen    holen.\footnote{
        \citew*[\page 136]{Bech55a}.
}
\zl
In (\mex{1}) sind die beiden Akkusative Pronomina. Aus dem Kontext wird klar, dass \word{sie}
in (\mex{1}a) das Objekt von \word{verbrennen} und in (\mex{1}b) das Objekt von \emph{bringen}
ist, in (\mex{1}c) ist das \emph{es} Objekt von \emph{machen}:
\eal\iw{lassen}
\ex \label{ex-lass-sie-uns-verbrennen}
Schau auf zum Himmel \\
      Diese Erde, sie ist gelb wie Stroh \\
      Komm, laß\iw{lassen} \emph{sie} \emph{uns} verbrennen \\
      Ich will es so \\
      Jetzt weißt du, wer ich bin\footnote{
        Herwig Mitteregger, Spliff, \emph{Herzlichen Glückwunsch}, CBS Schallplatten GmbH, Germany, 1982.
      }
\ex Was ist denn mit den alten Zeitungen? Laß sie mich gleich zum Altpapier bringen!\footnote{
        \citew[5]{GS97a}.
    }
\ex Mr.\, King fasste derweil schon eine Neuauf"|lage dieser harmonischen WM"=Nacht ins Auge: "`Lasst
    es uns doch noch mal machen."'\footnote{
        taz, 17.03.2003, S.\,19.
    }
%\ex
% Jedenfalls ließ sich der Student, der ich einst war, dies alles gern gesagt sein.\footnote{
%       taz, 11.12.2003, S.\,17.
%}
\zl


\noindent
Das folgende Beispiel von \citet[\page 142]{Lenerz93a}, % check Seite
zeigt, dass eine solche Umstellung
auch möglich ist, wenn beide Akkusative als nichtpronominale Nominalphrasen realisiert sind:
\ea
Wenn du das Buch eine Kundin lesen siehst, die dir verdächtig vorkommt \ldots
\z

\noindent
Es ist auch möglich, Dativobjekte links des AcI"=Akkusativs anzuordnen:
%\eal
\ea Man ließ der Feuerwehr           am     nächsten Tag die Polizei      helfen.\footnote{
        \citew[\page 125]{Bierwisch63a}.
      }
% \ex wenn du ihren Kindern die Mutter abends etwas vorsingen hörst.\footnote{
%         \citep[\page 284]{GS2003a}.
\z
Für Sätze wie (\mex{1}) ist die Reihenfolge, in der der Dativ dem Akkusativ vorangeht,
die bevorzugte, da es im Deutschen eine Tendenz gibt, NPen, die auf belebte\is{Belebtheit} Elemente referieren,
NPen, die auf unbelebte Elemente referieren, voranzustellen \citep*[\page 46]{Hoberg81a}.
% Beispiel ist von mir
\ea
Klaus sieht seinem Gläubiger einen Ziegel auf den Kopf fallen.\label{fallen-sehen}
\z
%
Sogar das Subjekt des Matrixverbs kann dem Akkusativ"= oder Dativobjekt folgen,
was auch oft bestritten wird \parencites[\page 138]{Grewendorf87a}[\page 284]{Grewendorf88a}[\page 207]{Wurmbrand98a}[\page 117, 306]{Cook2001a}.
\ea
dass  ihn      (den Erfolg) uns      niemand      auskosten ließ\footnote{
        \citet*[\page 78]{Haider94b-u} gibt ein ähnliches Beispiel mit \emph{erlauben} und schreibt
        Tilman Höhle ein entsprechendes Beispiel mit Fernpassiv zu. Siehe auch \citew[\page 136]{Haider90b}.
% Das sind andere Beispiel. Teils mit Fernpassiv, teils mit erlauben
}
%\ex dass niemand uns den Erfolg auskosten ließ
%\ex dass uns niemand den Erfolg auskosten ließ
%\ex dass uns ihn niemand auskosten ließ
\z
M.\ \citet[\page 249]{MRichter2002b} behauptet, dass die Voranstellung vor das Subjekt des Matrixverbs
nur möglich ist, wenn die beteiligten Elemente Pronomina sind:
\eal
\ex[]{
weil ihn niemand singen hörte
}
\ex[*]{
weil den Mann der Chef singen hörte
}
\zl
Ich teile seine Beurteilung der Daten in (\mex{0}) nicht. Bei entsprechender Intonation
ist (\mex{0}b) durchaus akzeptabel. Man kann die Intonation durch weiteres Material nahelegen:
\ea
weil diesen Mann der Chef höchstpersönlich singen hörte
\z

\noindent
Permutationen wie die in \fromto{\ref{ex-liess-den-Jungen-das-Buch-holen}}{\mex{0}} sind nur dann
möglich, wenn die Sätze verstehbar\is{Performanz} bleiben, \dash, wenn die Vertauschung der NPen sich für den Hörer
rekonstruieren lässt \citep[\page 172]{Mueller99a}. 


\subsubsection{Intraposition und Extraposition}
\label{sec-aci-ie}

Der vom AcI"=Verb abhängige Infinitiv kann nicht intraponiert\is{Intraposition} werden:
\eal
\ex[]{
dass ich den Jungen das Buch holen ließ / sah
}
\ex[*]{
dass ich das Buch holen den Jungen ließ / sah
}
\ex[*]{
dass den Jungen das Buch holen niemand ließ / sah
}
\zl

\noindent
Extraposition\is{Extraposition} des Infinitivs ist ebenfalls ausgeschlossen:
\eal
\ex[*]{
dass ich ließ / sah, den Jungen das Buch holen
}
\ex[*]{
dass ich den Jungen ließ / sah, das Buch holen
}
\zl
\is{Kohärenz|)}

\subsubsection{Einbettung expletiver Prädikate und subjektloser Konstruktionen}
\label{sec-aci-expl}

Wie bereits im Abschnitt~\ref{sec-rais-contr-identity-coindexing} gezeigt wurde,
sind die Wahrnehmungsverben\is{Verb!Wahrnehmungs-} zu den Anhebungsverben zu zählen.
Sie erlauben die Einbettung expletiver und subjektloser Prädikate \parencites[\page 66]{Reis76c}[\page 70]{Hoehle78a}.\footnote{
  In (\ref{ex-sah-ihm-schlecht-werden}) liegt eine Konstruktion ohne Akkusativ vor.
  Man kann also hier eigentlich nicht von der AcI"=Konstruktion sprechen.%
}
%%
%% klugscheißerisch
%% \footnote{
%%        Siehe auch \citew[\page 12]{Kiss95a}. Auf S.\,217 gibt Kiss allerdings 
%%         einen Lexikoneintrag für \emph{sehen} an, der einen Verbalkomplex mit Subjekt verlangt.%
%%
%% zu detailiert
%% }$^,$\footnote{
%%         Das Beispiel (\mex{1}b) ist ein Beispiel mit einem Wahrnehmungsverb und einer eingebetteten
%%         Kopula mit \emph{sein}\iwf{sein}. Entgegen der Behauptung von \citet[\page 66]{Reis76c}
%%         kommt auch \emph{lassen}\iwf{lassen} mit eingebettetem \emph{sein} vor, wie die folgenden
%%         Beispiele zeigen:
%%         \eal
%%         \ex Es ist möglich, die Subjekts"=Anhebung, so wie sie in (97) syntaktisch
%%               dargestellt wurde, auch für Sätze wie (144) und (145) relevant sein zu lassen.
%%               (Im Haupttext von \citep[\page 350]{Puetz82a})
%%         \ex das "`Dativisierungs"'"=Phänomen, das den Satz [\ldots] ungrammatisch sein läßt,
%%               (Im Haupttext von \citep[\page 141]{Grewendorf83a}).
%%         \ex Besonders die neuesten Erkenntnisse über die funktionale Arbeitsweise des menschlichen
%%             Gehirns aus der Neurowissenschaft lassen die Computermethapher mehr und mehr zu einem
%%             wissenschaftshistorischen Relikt werden. (Im Haupttext von Schwarz, Monika. 1996.
%%             \emph{Einführung in die Kognitive Linguistik}. Tübingen, Basel: A. Francke Verlag)
%%         \zl
%%         Die allgemeinere Behauptung von Suchland (\citeyear[\page 72]{Suchsland95a};
%%         \citeyear[\page 149]{Suchsland97a}), dass alle AcI"=Verben keine Kopulakonstruktionen einbetten
%%         können, wird sowohl von (i) als auch von (\mex{1}b) widerlegt.
%%         Die Unakzeptabilität der Beispiele, die die beiden Autoren geben, ist allerdings
%%         offensichtlich. Sie ist jedoch nicht auf ein generelles Einbettungsverbot zurückzuführen.%
%% }
% klugscheißerisch
%% $^,$\footnote{
%%         Die Beispiele in (\mex{1}) zeigen, dass eine Kontrollanalyse für \emph{sehen},
%%         wie sie von \citet[\page 231]{HM94a}
%%         vorgeschlagen wird, nicht adäquat ist.
%% }
\eal
\ex[]{ 
Conny sah es regnen.
}
\ex[?]{\label{ex-sah-ihm-schlecht-werden}
Ich sah ihm       schlecht werden.
}
\zl
Dasselbe gilt für \word{lassen}.
\ea
\label{bsp-er-laesst-es-regnen}
Er lässt es regnen.
\z
(\mex{0}) hat die Lesart, wo er den Regen Regen sein lässt (permissiv\is{Verb!permissives}), aber auch die
Lesart, in der er das Regnen verursacht (kausativ\is{Verb!kausatives}\is{Kausativ!-konstruktion}).\footnote{
        \citet[\page 238]{Gunkel2003b} unterscheidet zwischen direktiver und
        faktitiver Lesart und setzt für die direktive Kausation eine Patient"=Rolle
        an. Bei faktitiver Lesart gibt es dagegen keine Rolle. Einbettung
        von Witterungsverben behandelt er als faktitive Kausation.%
}
In der Sowjetunion wurden zu jedem ersten Mai vor den Paraden die Wolken abgeregnet.
Solche Techniken werden auch angewendet, um Hagelschäden zu verhindern.
Sowohl das kausative als auch das permissive \emph{lassen} lässt also die
Einbettung expletiver Prädikate zu.
(\mex{1}) hat einen anderen Kontext, aber es gibt genauso zwei Lesarten:
\ea
Er lässt es Konfetti regnen.
\z
Manchmal wird behauptet, dass es sich beim \emph{es} der Wetterverben\is{Verb!Wetter-}
nicht um expletive Elemente handelt (\citealp[\page 35]{Paul1919a}),
% The following example by Marga Reis (DGfS 2000) leaves no doubt about 
% the possibility to embed expletive predicates under \emph{lassen}.
% \ea
% Er ließ es        auf den Konflikt ankommen.
% \z
aber die folgenden Beispiele lassen keinen Zweifel daran, dass expletive Prädikate unter
\emph{lassen} eingebettet werden können:
\eal
\ex Er lässt es        sich gut  gehen.
\ex Schließlich hätten es Kriminalpolizei, Staatsanwaltschaft und Gericht zu diesem Prozeß kommen lassen.\footnote{
  taz, 09.09.2005, S.\,7.
}
\zl
%
Bei subjektlosen Konstruktionen ist die Sache weniger klar:

\eal
\judgewidth{??}
\ex[?]{
Er ließ ihm schlecht werden und kümmerte sich nicht drum.
}
\ex[??]{
Der Versuchsleiter  gab  ihm die Probe und ließ ihm schlecht werden.
}
\ex[?]{
Er ließ den Studenten vor    der Prüfung grauen und kümmerte sich nicht drum.
}
\ex[*]{
Er gab den Studenten eine schwere Probeklausur und ließ ihnen vor    der Prüfung grauen.
}
\zl
Die Einbettung subjektloser Prädikate unter das permissive \emph{lassen} (\mex{0}a,c)
scheint besser zu sein als die Einbettung unter die kausative Version (\mex{0}b,d).

Mit einem unbelebten Subjekt von \emph{lassen} kann man auch Belege für
das kausative \emph{lassen} und Einbettung subjektloser Konstruktionen finden:
\eal
\ex allein der Gedanke daran was passieren würde, würde Reno hinter sein Geheimnis kommen, ließ ihm schlecht werden.\footnote{%
\url{http://www.chibi-fich.de/Geschichten/o_foto07_09.htm}. \urlchecked{09}{07}{2005}.%
}
\ex und ein automatischer Blick nach unten ließ ihm schlecht werden.\footnote{
\url{http://www.foren.de/system/printview.php?threadid=215003}. \urlchecked{09}{07}{2005}.
}
\ex Allein der Geruch der Lebensmittel ließ ihr schlecht werden.\footnote{
\url{http://www.web-site-verlag.de/images/dds-tp.pdf}. \urlchecked{09}{07}{2005}.
}
\zl


\noindent
\label{page-start-aci-verbs-role}%
AcI"=Verben weisen dem Subjekt des eingebetteten Verbs keine thematische Rolle zu.
In Fällen, wo das eingebettete Verb ein referentielles Subjekt hat, wird manchmal
behauptet, dass das Matrixverb eine thematische Rolle zuweist.\is{Verb!Wahrnehmungs-}
\citet[\page 387]{Eisenberg94a} \zb behauptet, dass
(\mex{1}b) aus (\mex{1}a) folgt.
\eal
\ex\iw{sehen}
Ich sehe Hans rauchen.
\ex 
Ich sehe Hans.
\zl
Das ist jedoch nicht notwendigerweise der Fall, wie (\mex{1}) zeigt:
\ea 
Ich sehe jemanden rauchen.
\z
(\mex{0}) kann in einer Situation geäußert werden, in der sich ein Rauchender hinter einer Abtrennwand befindet
und man nur den Rauch der Zigarette sieht.
Wie \citet{KT76a} schlüssig darlegen, ist die Information, 
dass man, wenn man Hans rauchen sieht, gewöhnlich auch Hans sieht, 
nicht in der Bedeutung von \emph{sehen} enthalten, sondern wird
aus Weltwissen\is{Kontext}\is{Weltwissen}\is{Pragmatik} erschlossen.
Auf Seite~209 führen sie Beispiele mit unterschiedlichen Wahrnehmungsverben\is{Verb!Wahrnehmungs-} an, die ich
ins Deutsche übertragen habe:
\eal
\ex Wir haben das unsichtbare Nervengas alle Schafe töten sehen, aber natürlich haben
      wir das unsichtbare Nervengas selbst nicht gesehen.
%
\ex Ich fühlte\iw{fühlen} Georg sich auf das andere Ende des Wasserbetts setzen, aber
      natürlich habe ich ihn selbst nicht gefühlt.
%
\ex Ich roch\iw{riechen} Sylvia das Wohnzimmer aussprühen, aber ich konnte Sylvia selbst nicht riechen.
%
\ex Von meinem Beobachtungspunkt, der fünfzehn Kilometer weit entfernt war,
      sah\iw{sehen|)} ich sie die Brücke sprengen, aber es erübrigt sich zu sagen, dass
      ich die einzelnen Arbeiter aus der Entfernung nicht sehen konnte.
\ex Wir hörten den Bauer das Schwein schlachten.\footnote{
        \Citet[\page 45]{deGeest70a} gibt ein analoges niederländisches\il{Niederländisch}
        Beispiel. Was man wahrscheinlich hört, ist nicht der Bauer, sondern das Schwein.
      }
\zl
Die Beispiele zeigen, dass Situationen als Ganzes wahrgenommen werden können, ohne dass dabei
der Referent des Subjekts des eingebetteten Verbs wahrgenommen wird.%
\label{page-end-aci-verbs-role}%
% Siehe auch Heokstra88: 117
\is{Verb!Objektanhebungs-|)}
\is{Verb!AcI-|)}

\subsection{Objektkontrollverben}
\label{sec-obj-control-phen}
\is{Verb!Kontroll-|(}\is{Kohärenz|(}\is{Inkohärenz|(}

Einige Autoren haben behauptet, dass kohärente Konstruktionen mit Objektkontrollverben
nicht möglich sind (\citealp[\page 276]{Sternefeld85b}).
% Gunkel2003b sagt Haider86:28, Kiss95a:33
% Haider93 nur bei Akkusativ
% Abraham2005:503 nur bei Akkusativ
Im folgenden werde ich zeigen, dass kohärente Konstruktionen sowohl mit Objektkontrollverben,
die den Dativ regieren, als auch mit solchen, die ein Akkusativobjekt verlangen, möglich
sind.

\subsubsection{Skopus von Adjunkten}
\label{sec-obj-contr-scope}

\citet[\page 20]{Jacobs91a}\is{Skopus|(}\is{Adverb!Skopus|(} diskutiert die folgenden Sätze:
\eal
\ex\iw{verbieten}
weil    er       dem Mann      den Kindern        sicher zu helfen verbietet
\ex
weil    er       das Buch      den Kindern        sicher zu lesen verbietet
\zl
Beide Sätze haben die Lesart mit weitem Skopus, die der Anordnung in (\mex{1}a) entspricht:
\eal
\ex weil er dem Mann sicher verbietet, den Kindern zu helfen
\ex weil er dem Mann verbietet, den Kindern sicher zu helfen
\zl
Die Lesart mit weitem Skopus wäre in (\mex{-1}a) nicht möglich, wenn \emph{den Kindern sicher zu
  helfen} wie in (\mex{0}b) ein separates Kohärenzfeld wäre. Jacobs versieht das Beispiel mit den beiden Dativen
mit einem Fragezeichen, (\mex{-1}b) ordnet er als völlig akzeptabel ein.
Er nimmt an, dass eine Valenzliste, die aus dem Transfer von Argumenten vom
eingebetteten Verb zum Matrixverb entsteht, eine Form haben muss, die es auch
bei einfachen Lexikoneinträgen gibt.\footnote{
  Diese Annahme ist wohl von Bakers \emph{Case Frame Preservation Principle}
  inspiriert.
  \begin{quote}
    A complex \xnull of category A in a given language can have
    at most the maximal Case assigning properties allowed to a morphologically
    simple item of category A in that language. \citep[\page 122]{Baker88a}.
  \end{quote}%
  Baker bezieht sich auf \citew{GM85a}. \citet[\page 15]{GM85a} merken aber
  an, dass es in Inuit Eskimo\il{Inuit Eskimo}, der von ihnen untersuchten Sprache, schon
  Ausnahmen zu einer solchen Beschränkung gibt.%
}$^,$\footnote{
        \textcites[\page 94]{Haider86c}[\page 131]{Haider90b},
        \citet[\page 215]{Kiss95a} und
        \citet[\page 32]{Kathol2000a} machen dieselbe Annahme.
        Kiss räumt ein, dass diese Annahme mit einem Argumentanziehungsansatz für
        AcI"=Konstruktionen inkompatibel ist.

        Auch bei kohärenten Konstruktionen mit Subjektanhebungsverben\is{Verb!Subjektanhebungs-} gibt es Beispiele für Valenzrahmen,
        die bei Simplexverben\is{Verb!Simplex-} (\dash bei einfachen Verben im Gegensatz zu komplexen Verben
        bzw.\ Verbalkomplexen) nicht belegt sind:
        \eal
        \ex dass der Helden mir niemand zu gedenken schien
        \ex weil sie mir dem Hans untreu zu werden scheint
        \ex weil mir dem Hans die Sache über den Kopf zu wachsen scheint
        \zl
        \citet[\page 34]{Grewendorf94a} diskutiert diese Beispiele von \citet[\page 136]{Olsen81a}
        und \citet{Fanselow89b} und verwirft aber nicht das \emph{Case Frame Preservation Principle},
        sondern benutzt die Beispiele -- wie auch Fanselow selbst -- als Argument gegen eine monosententiale Struktur
        deutscher Anhebungsinfinitive. \citet[\page 7]{Fanselow89b} zeigt, dass
        das \emph{Case Frame Preservation Principle} in Bezug auf die Verteilung von Kasus
        bei Verben wie \emph{anheften} eine Rolle spielt: Der Kasus
        des durch \emph{an} lizenzierten Elements ist nicht Akkusativ, wie man es aufgrund
        der Kasusvergabe der Präposition \emph{an} erwarten könnte, sondern Dativ.
        \eal
        \ex dass ich an das Buch einen Zettel hefte
        \ex dass ich dem Buch einen Zettel anhefte
        \zl
        Dies ist erklärt, wenn man sagt, dass das \emph{Case Frame Preservation Principle}\is{Prinzip!Case Frame Preservation Principle}
        den Akkusativ an dieser Stelle ausschließt und das Argument deshalb als Dativ
        realisiert werden muss. Gegen dieses Argument lässt sich vorbringen, dass die Verbalkomplexe
        sich klar von den Präposition"=Verb"=Komplexen unterscheiden und dass man -- so wie
        das ja auch die Autoren tun, die eine bisententiale Struktur, \dash eine Struktur, in der zwei
        vollständige Teilsätze miteinander kombiniert werden, annehmen -- diese
        Komplexe anders behandeln muss. Es stellt sich nun die Frage, ob die Andersartigkeit 
        der Behandlung sich in einer anderen syntaktischen Grundstruktur oder in anderen
        Beschränkungen, die auf Grundstrukturen operieren, äußern sollte. Ich plädiere
        hier für den zweiten Weg, nämlich dafür anzunehmen, dass das \emph{Case Frame Preservation Principle}
        für Verbalkomplexe im Deutschen nicht gilt.

        Die Bindungsdaten, die Fanselow gegen eine monosententiale Struktur anführt,
        sind für valenz- bzw.\ argumentstrukturbezogene Bindungstheorien wie die von \citew{ps2}
        kein Problem.
}
Wie er selbst feststellt, sollten Beispiele wie (\mex{-1}a) nicht möglich sein,
da das Deutsche keine einfachen Köpfe hat, die zwei Dative regieren\is{Kasus!Dativ}.

Den Satz (\mex{1}a) gibt Jacobs ohne Fragezeichen.\footnote{%
\label{fn-scrambling-pronoun}%
        Man beachte, dass beide Sätze in (\mex{1}) mehrdeutig sind.
        Das Pronomen \word{es} kann sich \zb auf ein Buch, ein Kind oder ein Mädchen beziehen.
        Genauso kann \word{sie} sich auf eine Zeitung und \word{ihn} sich auf einen Aufsatz
        beziehen. Je nach Referenz der Pronomina haben die Sätze in (\mex{1})
        umgestellte oder nicht"=umgestellte Elemente im \mf.%
}
% Bayer96a:246 dass mich Hans niemanden zu grüßen gezwungen hat
% hat auch andere Skopuslesart
\eal
\label{ex-coherent-bitten}
\ex
weil    er       es       sie       tatsächlich zu reparieren bat\label{bsp-er-es-sie-zu-reparieren-bat}\footnote{
        \citew[\page 20]{Jacobs91a}.
}
\ex\iw{bitten}
weil    der Fritz       es       ihn       nicht zu lesen bat\label{bsp-es-ihn-nicht-zu-lesen-bat}\footnote{
        \citew[\page 174]{Reape94a}.
}
\zl
In diesen Sätzen sind beide Skopen möglich, da beide Prädikate mit dem Adverb kompatibel sind.
Wenn man einen Argumentkompositionsansatz annimmt, enthält die Argumentstruktur, die man bei einer
Kombination der Argumente des eingebetteten Verbs mit den Argumenten des Matrixverbs bekommt,
zwei Akkusative\is{Kasus!Akkusativ}, was es im Deutschen bei einfachen Verben nicht gibt.\footnote{
        Es gibt natürlich Verben wie \emph{lehren}, aber bei \emph{lehren} ist ein Akkusativ
        strukturell und einer lexikalisch. In den Komplexen, die man für (\mex{0}) annehmen
        muss, gibt es zwei strukturelle Akkusative, und so etwas kommt im Deutschen bei einfachen
        Verben nicht vor. Zur Unterscheidung zwischen lexikalischem und strukturellem Kasus
        siehe Kapitel~\ref{sec-struc-lex-kas}.

        In der kohärenten Konstruktion in (\ref{bsp-trauen-kohaerent}) auf Seite~\pageref{bsp-trauen-kohaerent}
        liegen ebenfalls zwei Akkusative vor. Da es sich bei \word{trauen} aber um ein inhärent
        reflexives Verb\is{Verb!inhärent reflexives} handelt, könnte man behaupten, dass der Kasus
        des Reflexivums lexikalisch ist.%
}

\citet[\page 37]{BK89a}, \citet[\page 136]{Haider90b},
\citet[\page 79]{VS98a} und \citet[\page 503]{Abraham2005a}
behaupten explizit, dass kohärente Konstruktionen mit Kontrollverben, die ein Akkusativobjekt verlangen,
nicht möglich sind.
Wie Jacobs nimmt auch \textcites[\page 94]{Haider86c}[\page 131]{Haider90b} an,
dass Verbalkomplexe in kohärenten Konstruktionen eine Argumentstruktur haben, die auch bei
einfachen Verben gefunden werden kann. Da es im Deutschen aber keine einfachen Verben gibt,
die eine Argumentstruktur der Art haben, wie man sie für (\mex{0}) braucht, ist Haiders Annahme
widerlegt.

%The translation of (\ref{bsp-es-ihn-nicht-zu-lesen-bat}) already showed that two readings
%are possible with object control verbs.
Wie \citet[\page 13]{Askedal88} festgestellt hat, muss es sich bei (\mex{1})
auch um eine kohärente Konstruktion handeln:
%
% stimmt das? Eventuell kann ja der Skopus durch die Extraktion bedingt sein
%
\ea
Keine Zeitung         wird ihr       zu lesen erlaubt.\footnote{
        Stefan Zweig. \emph{Marie Antoinette}. Leipzig: Insel-Verlag. 1932, S.\,515, 
        zitiert nach \citew[\page 309]{Bech55a}.
}
\z
Die Negation\is{Negation} in \emph{keine} kann Skopus über \emph{erlauben} haben,
was für ein Argument von \emph{lesen} in einer inkohärenten Konstruktion nicht möglich wäre.
Siehe \citew[{\S}\,80]{Bech55a} und S.\,\pageref{bsp-trauen-kohaerent} des vorliegenden
Buches zu Beispielen mit sogenannter Kohäsion\is{Kohäsion}. \emph{erlauben} ist wie \emph{verbieten}
ein Objektkontrollverb, das ein Dativobjekt regiert. Es wurde also gezeigt, dass es sowohl
Objektkontrollverben mit Dativobjekt als auch Objektkontrollverben mit Akkusativobjekt gibt, die
kohärent konstruieren können.\is{Skopus|)}\is{Adverb!Skopus|)}


\subsubsection{Permutation im \mf}
\label{sec-obj-contr-perm-mf}

Die Beispiele in (\mex{1}) zeigen, dass die Permutation der Elemente im \mf möglich ist.
\eal
\ex\iw{erlauben}
weil     dieses Machwerk        kein Vater      seinen Kindern     zu lesen erlauben würde\footnote{
        \citew[\page 174]{Reape94a}.
}
\ex 
daß ihn       (den Erfolg) uns      niemand      auszukosten erlaubte\footnote{
        \citet*[\page 78]{Haider94b-u} gibt dieses Beispiel und schreibt  Tilman
        Höhle\aimention{Tilman N. Höhle} ein ähnliches zu.
}
\zl
Die Beispiele in (\ref{ex-coherent-bitten}) sind ebenfalls Beispiele für Permutationen,
wenn das \emph{es} sich auf einen unbelebten Diskursreferenten bezieht. Siehe auch
Fußnote~\ref{fn-scrambling-pronoun}.


\subsubsection{Intraposition und Extraposition}
\label{sec-obj-contr-ie}

Sowohl Intraposition\is{Intraposition} (\mex{1}a) als auch Extraposition\is{Extraposition} (\mex{1}b) ist möglich.
\eal
\ex 
dass Conny [den Aufsatz zu lesen] niemandem versprochen hat
\ex
dass Conny niemandem versprochen hat, [den Aufsatz zu lesen]
\zl
\is{Verb!Kontroll-|)}\is{Kohärenz|)}\is{Inkohärenz|)}

\noindent
Zusammenfassend kann man festhalten, dass Kontrollverben sowohl kohärent als auch inkohärent
konstruieren können. Bei Subjektanhebungsverben erlauben nur die Phasenverben die inkohärente
Konstruktion. Alle anderen Subjektanhebungsverben und die Objektanhebungsverben konstruieren obligatorisch
kohärent.


\section{Die Analyse}
\label{sec-anhebung-anal}

Im Abschnitt~\ref{sec-subj-merkmal} führe ich kurz das \subjm ein. Danach werden die Lexikoneinträge
für Subjektanhebungsverben, Subjektkontrollverben, Objektanhebungsverben und Objektkontrollverben
vorgestellt.

\subsection{Das \subjm}
\label{sec-subj-merkmal}

\is{Merkmal!\textsc{subj}|(}
Bisher wurden alle Argumente eines Verbs in der \compsl repräsentiert.
In Grammatiken für das Englische wird zwischen Subjekten und anderen Argumenten
unterschieden, da das Subjekt eines Verbs sich unter anderem durch sein
Stellungsverhalten von den anderen Argumenten unterscheidet. Im Englischen
muss das Subjekt immer links vom Verb stehen, die anderen Argumente stehen
jedoch rechts. Im Deutschen steht das Verb (bzw.\ die Verbspur) rechts von seinen
Argumenten. In Bezug auf die Stellung des Kopfes relativ zu seinen Argumenten
gibt es keinen Unterschied zwischen Subjekten und anderen Argumenten.
\ea
\gll He gives the man a book.\\
     er gibt dem Mann ein Buch\\
\z
Für das Englische wird ein Dominanzschema für eine VP"=Projektion
angenommen, die das Verb mit allen vom Subjekt verschiedenen Argumenten enthält.
Ein weiteres Dominanzschema kombiniert dann die VP mit dem Subjekt. Für das Deutsche
ist eine solche Sonderbehandlung nicht angebracht, da sich das Subjekt finiter Verben
genauso wie die restlichen Argumente verhält. Allerdings ist es so, dass das Subjekt
in Kontrollkonstruktionen nie sichtbar realisiert wird.
\eal
\ex Kim hat versucht, ein Buch zu schreiben.
\ex Kim hat Sandy gezwungen, das Buch zu lesen.
\zl
\citet[295]{Pollard90a} und \textcites[268]{Kiss92}[Kapitel~3.1.1]{Kiss95a} haben deshalb vorgeschlagen, das Subjekt infiniter
Verben als Element einer \subjl zu repräsentieren, wohingegen das Subjekt
finiter Verben weiterhin als Element der \compsl repräsentiert wird (Bei Pollard und Kiss war das
noch die \subcatl.). Da dann
bei einer infiniten Form eines Verbs wie \emph{lesen} nur das Objekt Element
der \compsl ist, bildet \emph{das Buch zu lesen} eine vollständige Projektion.

Auch bei adjektivischen Verwendungen kommt das Subjekt nicht sichtbar
vor, sondern ist nur mit dem modifizierten Nomen koindiziert. Siehe Seite~\pageref{bsp-nominativ-adj}.
\ea
das den Roman lesende Kind
\z
Bei der Repräsentation des Subjekts außerhalb der \compsl ist \emph{den Roman lesende} ebenfalls gesättigt.
Man kann somit die Generalisierung aufrecht erhalten, dass in Adjunktionspositionen\is{Adjunkt}
und im Nachfeld\is{Extraposition} nur Maximalprojektionen stehen können. 

In Kontrollkonstruktionen wie in (\mex{-1}) ist das Subjekt des eingebetteten
Infinitivs mit einem Argument des Matrixverbs koindiziert. In Anhebungskonstruktionen
wird das Subjekt des eingebetteten Verbs als Argument des Matrixverbs realisiert.
Das Subjekt des eingebetteten Verbs muss also verfügbar sein, wenn die Kombination
(der Projektion) des eingebetteten Verbs mit dem Matrixverb stattfindet. Kiss
hat deshalb vorgeschlagen, das \subjm zum Kopfmerkmal zu machen. Es wird dadurch
automatisch durch das Kopfmerkmalsprinzip entlang des Kopfpfades einer Projektion
nach oben gereicht und ist auch innerhalb der Maximalprojektion zugänglich.

Bevor ich mich der Analyse der Anhebungsverben zuwende, möchte ich noch beispielhaft die Einträge
für den Verbstamm (\mex{1}), die finite (\mex{2}) und die infinite Form (\mex{3}) des Verbs
\emph{helfen} zeigen:

{\exewidth{(100)}
\ea
Verbstamm \stem{helf}:\\
\ms{
head   & verb\\
arg-st & \liste{ NP[\str], NP[\ldat] }
}
\z 
\ea
finite Form \emph{hilft}:\\
\ms{
head   & \ms[verb]{
         subj  & \ibox{1} \eliste\\
         vform & fin\\
         }\\
comps  & \ibox{2} \liste{ NP[\str], NP[\ldat] }\\
arg-st & \ibox{1} $\oplus$ \ibox{2}\\
}
\z 
\ea
infinite Form \emph{helfen}:\\
\ms{
head   & \ms[verb]{
         subj  & \ibox{1} \liste{ NP[\str] }\\
         vform & bse\\
         }\\
comps  & \ibox{2} \liste{ NP[\ldat] }\\
arg-st & \ibox{1} $\oplus$ \ibox{2}\\
}
\z
\todostefan{check wie das mit modalen Infinitiven geht}
Sowohl die finite als auch die infinite Form sind mittels Lexikonregeln vom Verbstamm abgeleitet
(zu Lexikonregeln siehe Kapitel~\ref{sec-lr}). Beide Formen haben dieselbe \argstl wie der
Verbstamm. Die \compsl des finiten Verbs in (\mex{-1}) enthält alle Elemente der \argstl; die \subjl
ist leer. Beim infiniten Verb ist das Subjekt nicht in der \compsl enthalten, sondern in der
\subjl (zu den Details siehe Kapitel~\ref{sec-analysis-da-modal-inf}).
\is{Merkmal!\textsc{subj}|)}


\subsection{Subjektanhebung}
\label{sec-subjektanhebung-anal}

\is{Anhebung|(}
Im Abschnitt~\ref{sec-rais-contr-identity-coindexing} wurde gezeigt,
dass Anhebungskonstruktionen als Identität sowohl der syntaktischen
als auch der semantischen Information des Subjekts bzw.\ Objekts analysiert werden
sollten, wohingegen bei Kontrollkonstruktionen nur eine Koindizierung
zwischen kontrollierendem und kontrolliertem Element hergestellt werden sollte.
Diese Analyse wird auch in der \lfg
\parencites{Andrews82b}{Neidle82a}[\page 396]{Bresnan82c}
und in den Arbeiten von \citet[Kapitel~7]{ps2}, \citet{Kiss95a}
und anderen Autoren im Rahmen der HPSG vertreten.

Die Analyse von kohärenten und inkohärenten Konstruktionen und Anhebungs- und
Kontrollverben, die in den nächsten Abschnitten präsentiert wird,
geht auf \citet{Kiss95a} zurück. 

(\mex{1}) zeigt den \locw von \emph{scheinen}:
%% \footnote{
%%         \citet[\page 37]{dSW88a} propose a function composition\is{function composition}
%%         approach for English\il{English} \emph{seem}. They assume that \emph{seems} is a functor
%%         and that the argument structure of \emph{seems sick} is identical to the argument structure
%%         of \emph{sick}.
%% }

\eas
\label{le-scheinen}%
\stem{schein} (Subjektanhebungsverb, obligatorisch kohärent\is{Kohärenz|(}):\\
\ms{
cat & \ms{% head   & verb \\ 
           arg-st & \ibox{1} $\oplus$ \ibox{2} $\oplus$ \sliste{ \textrm{V[\type{inf},
               \textsc{lex}+, \textsc{subj}~\ibox{1}, \textsc{comps}~\ibox{2}, \ltop \ibox{3} ]}}\\
         }\\
cont & \ms{
          ind  & \ibox{4} event\\
          ltop & \ibox{5} }\\
rels & \liste{ \ms[scheinen]{
               lbl  & \ibox{5}\\
               arg0 & \ibox{4}\\ 
               arg1 & \ibox{6}\\
               } }\\[8mm]
hcons & \liste{ \ms[qeq]{
                harg & \ibox{6}\\
                larg & \ibox{3}} }
}\iw{scheinen|uu}
\zs

\noindent
Der \subjw des eingebetteten Verbs \iboxb{1} wird mit dem \compsw des
eingebetteten Verbs \iboxb{2} verknüpft und zum \argstw des Verbs \emph{scheinen} angehoben.
Der Lexikoneintrag in (\mex{0}) ist parallel zu dem für das Futur"=Hilfsverb in (\ref{le-wird})
auf Seite~\pageref{le-wird}. Er unterscheidet sich in der \vform des selegierten Verbs und in der Erwähnung des \subjms.
Wie bei der Kombination von \emph{werden} mit einem Verb werden auch bei der Kombination
eines Verbs mit \emph{scheinen} alle Argumente des eingebetteten Verbs angezogen.
Für die angezogenen Argumente (\ibox{1} $\oplus$ \ibox{2} in (\ref{le-scheinen})) muss wie bei \emph{werden}
die Beschränkung (\ref{constr-non-complex-forming}) auf Seite~\pageref{constr-non-complex-forming} gelten,
die sicherstellt, dass keine lexikalischen Verben angezogen werden.

Wird ein subjektloses Verb wie \emph{grauen} unter \emph{scheinen} eingebettet, so
ist \iboxt{1} die leere Liste, und der Komplex \emph{zu grauen scheint} ist ebenfalls
subjektlos. Nur die Formen der dritten Person Singular von \emph{scheinen}
sind mit Prädikaten bzw.\ Prädikatskomplexen kompatibel, 
die keine Nominalphrase mit strukturellem Kasus verlangen. Siehe auch Seite~\pageref{page-kongruenz-scheinen}.

Wird ein Verb wie \emph{schlafen} unter \emph{scheinen} eingebettet, so ist
\iboxt{1} eine Liste, die eine NP mit strukturellem Kasus enthält. Diese wird
zum ersten Element der \argstl von \emph{scheinen} und bekommt
deshalb Nominativ und muss auch mit \emph{scheinen} kongruieren\is{Kongruenz},
wenn \emph{scheinen} finit ist. Die NP und das verbale Komplement von \emph{scheint} werden, da
\emph{scheint} finit ist, beide auf die \compsl gemappt. Der Komplex \emph{zu schlafen scheint}
verlangt dann genau eine NP mit strukturellem Kasus.

Wird das intransitive Verb \emph{schlafen} unter \emph{scheinen} eingebettet,
ist \iboxt{2} die leere Liste, und der Komplex aus eingebettetem Verb und \emph{scheinen}
verlangt außer dem Subjekt kein weiteres Argument. Hat das eingebettete Verb 
außer dem Subjekt noch weitere Argumente, wie \zb bei transitiven Verben wie \emph{kennen},
so werden diese entsprechend angehoben. Der Komplex \emph{zu kennen scheint} verlangt
demzufolge zwei Nominalphrasen mit strukturellem Kasus, von denen die erste Nominativ
und die zweite Akkusativ erhält. Die erste muss wieder mit dem Verb kongruieren.

Dass die beiden Argumente umgestellt werden können, wird von der bisher entwickelten
Analyse vorausgesagt, denn \emph{zu kennen scheint} verhält sich als komplexer Kopf
genauso wie der einfache Kopf \emph{kennen}. Die Argumentanordnungen können genauso
analysiert werden wie die Umordnung der Argumente von \emph{kennen}. Siehe dazu
Kapitel~\ref{sec-mf}.

Die Spezifikation des \lexwes des eingebetteten Verbs schließt die Kombination
mit Verbprojektionen aus. Somit ist erklärt, warum Sätze wie (\ref{bsp-intra-extra-scheinen}b,c) -- hier
als (\mex{1}) wiederholt -- ausgeschlossen sind.
\eal
\ex[*]{
dass Conny Kirby zu lieben zumindest scheint
}
\ex[*]{
dass Conny scheint, Kirby zu lieben
}
\zl


\noindent
Diese Analyse obligatorisch kohärent konstruierender Verben ist eigentlich völlig parallel
zur Analyse der Hilfsverben, die im Kapitel~\ref{Kapitel-Verbalkomplex} vorgestellt wurde. Die Komplexität
ist leicht höher, da inzwischen noch das \subjm eingeführt wurde. Zum besseren Verständnis
wird in Abbildung~\vref{abb-zu-helfen-scheint} die zu Abbildung~\vref{abb-helfen-wird} parallele Analyse von \emph{zu helfen scheint}
wiedergegeben.

\begin{figure}
\centerline{
\begin{forest}
sm edges
[\ms{ head   & \ibox{1}\\
           comps & \ibox{2} $\oplus$ \ibox{3}\\
            }
   [\iboxt{4}~\onems{ loc \onems{ head  \ms[verb]{ subj  & \ibox{2} \sliste{ NP[\type{nom}] }\\
                                               vform & inf \\
                                             }\\
                                comps~\ibox{3} \sliste{ NP[\ldat] }\\[2mm]
                              }\\
                } [zu helfen]]
   [\ms{ head & \ibox{1} \ms[verb]{ subj  & \sliste{ }\\
                                    vform & fin \\
                                  }\\
         comps & \ibox{2} $\oplus$ \ibox{3} $\oplus$ \sliste{ \ibox{4} }\\
                    } [scheint]]]
\end{forest}
}
\caption{\label{abb-zu-helfen-scheint}%
Analyse von \emph{zu helfen scheint}}
\end{figure}

\noindent
Da \emph{zu helfen} eine infinite Form ist, ist das Subjekt als Wert von \subj repräsentiert
\iboxb{2}. Der \subjw wird von \emph{scheint} angehoben, genauso wie die Elemente der
\compsl \iboxb{3}. Da \emph{scheint} finit ist, ist das Subjekt des Verbs in der \compsl
von \emph{scheint} repräsentiert. Die komplette \argstl von \emph{scheint} ist identisch mit der
\compsl. Deshalb ist das Subjekt dann auch ein Element der \compsl des gesamten Komplexes. 

Den Fall mit infinitem \emph{scheinen} zeigt Abbildung~\vref{abb-zu-helfen-scheinen}.
\begin{figure}
\centerline{
\begin{forest}
sm edges
[\ms{ head   & \ibox{1}\\
           comps & \ibox{2}\\
            }
  [\iboxt{4}~\onems{ loc \onems{ head  \ms[verb]{ subj  & \ibox{3} \sliste{ NP[\type{nom}] }\\
                                               vform & inf \\
                                             }\\
                                comps~\ibox{2} \sliste{ NP[\ldat] }\\[2mm]
                              }\\
                } [zu helfen]]
  [\onems{ head \ibox{1} \ms[verb]{ subj  & \ibox{3} \sliste{ NP[\type{nom}] }\\
                                                 vform & bse \\
                                                    }\\
                      comps \ibox{2} $\oplus$ \sliste{ \ibox{4} }\\
                    } [scheinen]]]
\end{forest}
}
\caption{\label{abb-zu-helfen-scheinen}%
Analyse von \emph{zu helfen scheinen}} % Google: Sollte nichts zu helfen scheinen eine mögliche Option
\end{figure}
Diese Analyse gleicht der von \emph{zu helfen scheint}, nur dass das Subjekt und die Komplemente von
\emph{scheinen} auf verschiedene Valenzmerkmale verteilt sind. Das Subjekt ist Element der \subjl
und Teil des Wertes von \head und die Komplemente stehen unter \comps.

Der semantische Beitrag des unter \emph{scheinen} eingebetteten Verbs wird unter die \relation{scheinen}"=Relation
eingebettet. Das wird wie bei \emph{glauben} auf S.\,\pageref{le-glauben} mithilfe von
\qeq\hyp Beschränkungen gemacht. Der Wert des \textsc{arg1}"=Merkmals in (\ref{le-scheinen}) ist \qeq mit dem \ltopw des
eingebetteten Verbs. Damit können die berühmten \emph{Einhorn}"=Sätze analysiert werden
\citep[316--317]{mrs}.
\ea
\label{ex Ein Einhorn scheint sich zu nähern}
Ein Einhorn scheint sich zu nähern.
\z
Sätze wie (\mex{0}) haben zwei Lesarten: In der ersten gibt es ein Einhorn, das sich zu nähern
scheint, und in der zweiten scheint es so zu sein, dass es etwas gibt, das ein Einhorn ist und sich
nähert:
\eal
\ex $\exists$x einhorn(x) $\wedge$ scheinen(nähern(x)))
\ex scheinen($\exists$x einhorn(x) $\wedge$ nähern(x)))
\zl
Für (\mex{0}b) muss es also nicht unbedingt ein Einhorn geben. Das, was sich nähert, könnte auch ein
Patronus sein. Für den Satz in (\ref{ex Ein Einhorn scheint sich zu nähern}) ergibt sich die
folgende MRS:
\ea
\textmrs{ h0, \{ h1:exists\_q(x, h2, h3), h4:einhorn(x), h5:scheinen(e1, h6),\\
\hphantom{\textlangle~h0, \{~}h7:nähern(e2, x) \}, \{ h2 \qeq h4, h6 \qeq h7 \}  }
\z
Für die MRS in (\mex{0}) gibt es zwei mögliche Lösungen, da der Quantor die Variable x von
nähern(e2, x) binden muss: h3 kann identisch mit h5 sein, was der
Lesart mit weitem Skopus in (\mex{-1}a) entspricht. Oder h3 wird mit h7 identifiziert, was der
Lesart in (\mex{-1}b) entspricht. In beiden Fällen ist h2 identisch mit h4, weil es nur einen
Quantor gibt und kein anderer Quantor zwischen den Existenzquantor und h4:einhorn(x) treten kann.

Für\is{Inkohärenz|(} Phasenverben gibt es den Eintrag in (\ref{le-beginnen}), der dem Eintrag für \stem{schein}
in vielen Punkten gleicht.\footnote{\label{fn-phase-verbs-control}%
        Für Phasenverben benötigt man immer zwei Lexikoneinträge, da Phasenverben
        mit agentivem\is{Agentivität} Subjekt sich wie Kontrollverben verhalten. Siehe auch \citew{Perlmutter70}.%
}

\exewidth{(100)}
\eas
\label{le-beginnen}%
\stem{beginn} (Phasenverb, optional kohärent):\\
\ms{
cat & \ms{% head   & verb \\
           arg-st & \ibox{1} $\oplus$ \ibox{2} $\oplus$ \sliste{ \textrm{V[\type{inf},
               \textsc{subj}~\ibox{1}, \textsc{comps}~\ibox{2}, \ltop \ibox{3} ]} }\\
         }\\
cont & \ms{
          ind  & \ibox{4} event\\
          ltop & \ibox{5} }\\
rels & \liste{ \ms[beginnen]{
        lbl  & \ibox{5}\\
        arg0 & \ibox{4}\\
        arg1 & \ibox{3}\\
       } }\\[8mm]
hcons & \eliste
}
\iw{beginnen|uu}
\zs
Der Eintrag unterscheidet sich von dem für \stem{schein} -- abgesehen vom Bedeutungsbeitrag -- in
zwei Punkten: Erstens wird der \ltopw des eingebetteten Verbs direkt mit dem \argone-Wert der
\relation{beginnen}\hyp Relation identifiziert. Daraus ergibt sich, dass die Quantoren in der
Bedeutung von (\mex{1}) nicht zwischen \relation{beginnen} und \relation{lesen} skopen können:
\ea
Alle Kinder beginnen ein Buch zu lesen.
\z
Es gibt nur Lesarten mit weitem Quantorenskopus.\todostefan{check}

Und zweitens ist der \lexw des eingebetteten Verbs nicht spezifiziert. Deshalb kann sowohl eine VP als auch ein einzelnes
Verb bzw.\ ein Verbalkomplex unter \emph{beginnen} eingebettet werden. Man kann mit diesem Eintrag also auch Sätze
wie die in (\mex{1}) analysieren:
\eal
\ex dass [den Aufsatz zu lesen] niemand beginnt
\ex dass niemand [den Aufsatz zu lesen] morgen beginnt
\zl
Bei der Kombination von \emph{niemand beginnt} mit \emph{den Aufsatz zu lesen} mittels \kasch
ist das Argument eine vollständig gesättigte Phrase. Die bisherige Version des
Kopf"=Argument"=Schemas (siehe Seite~\pageref{schema-bin}) sagt nichts über den phrasalen Status des Arguments aus: Sowohl gesättigte
als auch ungesättigte Argumente können mit Köpfen kombiniert werden. Normalerweise verlangen Köpfe,
dass ihre Argumente vollständig gesättigt sind (\zb indem sie eine NP selegieren), bei den optional
kohärent konstruierenden Verben ist das jedoch nicht der Fall. Die entsprechende Restriktion für
Kopf"=Argument"=Strukturen wird deshalb ins Kopf"=Argument"=Schema integriert:

\begin{schema}[Kopf-Argument-Schema (binär verzweigend, vorläufige Version)]
\label{schema-bin-sat}
\begin{tabular}[t]{@{}l@{}}\is{Schema!Kopf"=Argument"=}
\type{head"=argument"=phrase}\istype{head"=argument"=phrase} \impl\\
\onems{
      synsem$|$loc$|$cat$|$comps \ibox{1} $\oplus$ \ibox{3}\\
% das gilt für mehrere Schemata
%                     lex $-$\\
%                }\\
      head-dtr$|$cat$|$comps \ibox{1} $\oplus$ \sliste{ \ibox{2} } $\oplus$ \ibox{3} \\
      non-head-dtrs \sliste{ \ms{ \synsem  \ibox{2} \onems{ loc$|$cat$|$comps \eliste\\
                                                            lex  $-$\\ } } }\\
}
\end{tabular}
\end{schema}
Dadurch, dass verlangt wird, dass die Nicht"=Kopftochter eine leere \compsl hat, ist ausgeschlossen,
dass das Verb \emph{zu lesen} direkt mit \emph{beginnt} oder einer Projektion von \emph{beginnt}
kombiniert wird. Eine Kombination von \emph{zu lesen} und \emph{beginnt} ist nur mittels
Prädikatskomplexschema möglich.

In der hier vorgeschlagenen Analyse ist die obligatorische Kohärenz, wie sie bei Verben wie
\emph{scheinen} vorkommt, ein Unterfall der optionalen Kohärenz, da bei der obligatorischen
Kohärenz der \lexw des eingebetteten Verbs spezifiziert ist, bei der optionalen Kohärenz dagegen
nicht.
\is{Anhebung|)}\is{Kohärenz|)}\is{Inkohärenz|)}




\subsection{Subjektkontrollverben}
\label{sec-subj-control-analysis}
\label{sec-subject-control-anal}


(\mex{1})\is{Kohärenz}\is{Inkohärenz} zeigt den Lexikoneintrag für das optional kohärent konstruierende Subjektkontrollverb
\word{versuchen}.

\eas
\label{le-versuchen-incoh}%
\stem{versuch} (Kontrollverb, optional kohärent):\\
%\resizebox{\linewidth}{!}{%
\onems{
cat$|$arg-st  \sliste{ NP[\type{str}]\ind{1}  } $\oplus$ \ibox{2} $\oplus$ \nliste{
  \textrm{V[}\type{inf}, \textsc{subj} \sliste{NP[\type{str}]\ind{1}}, \textsc{comps}~\ibox{2}, \ltop \ibox{3}]  }\\
cont \ms{ ind  & \ibox{4} event\\
          ltop & \ibox{5}\\
          }\\ 
rels \liste{ \ms[versuchen]{
        lbl  & \ibox{5}\\
        arg0 & \ibox{4}\\ 
        arg1 & \ibox{1}\\
        arg2 & \ibox{3}\\
       } }\\[10mm]
hcons \eliste 
}
\iw{versuchen|uu}%
%}
\zs
Im Gegensatz zu den bisher diskutierten Anhebungsverben ist das Subjekt des eingebetteten Verbs
nicht mit dem Subjekt des Matrixverbs identisch. Wie gezeigt wurde, können das kontrollierende
Element und das kontrollierte Subjekt verschiedene Kasuswerte haben, ja sie können sich sogar
in der Wortart unterscheiden (siehe Abschnitt~\ref{sec-rais-contr-identity-coindexing}).
Selbst für Subjektkontrollverben, bei denen der Kasusunterschied zwischen kontrolliertem
Subjekt und kontrollierendem Subjekt nicht ohne weiteres sichtbar gemacht werden kann, da
es sich ja bei beiden Nominalphrasen um Subjekte handelt, die normalerweise im Nominativ
stehen, kann man sich Beispiele ausdenken, die zeigen, dass die jeweiligen Kasuswerte
unabhängig voneinander sein müssen. So zeigt (\mex{1}) einen Satz, in dem das Kontrollverb
unter ein \aciv eingebettet wurde und in dem somit das Subjekt des Kontrollverbs Akkusativ bekommt.
%
% Huch, was hab ich mir denn da ausgedacht?
%
% das ist Gender-Resolution, a la, Dalrymple Kaplan
%\ea
%\gll Er ließ den Mann      und die Frau versuchen, einer neben dem anderen einzuschlafen.\\
%     he let   the man\acc{} and the woman try       one   next  the other   \partic(in).to.sleep\\
%\glt `He let the man and the woman try to sleep next to each other.'
%\z
\ea
Er ließ den Jungen und den Mann versuchen, einer nach dem anderen über den Zaun zu klettern.
\z
Das Subjekt des eingebetteten Verbs muss aber im Nominativ stehen, wie die Kongruenz mit dem Adverbial zeigt.
Eine Adjunktphrase im Akkusativ wäre ungrammatisch. Da die Kasuswerte der beiden Subjekte unabhängig voneinander sind
und nur der semantische Index identifiziert ist, erfasst die Analyse Sätze wie (\mex{0}) korrekt.

Die Bezugnahme auf das kontrollierte Subjekt schließt die Einbettung unpersönlicher Konstruktionen aus.
Die Abkürzung NP\ind{1} steht für eine referentielle Nominalphrase, weshalb auch die Einbettung
expletiver Prädikate ausgeschlossen ist.





\subsection{Objektanhebungsverben: AcI-Verben}
\is{Verb!Objektanhebungs-|(}
\is{Verb!AcI-|(}%
\is{Anhebung|(}



Der Lexikoneintrag in (\mex{1}) zeigt den \localw für das \aciv \emph{sehen}:\footnote{
        \citet[\page 231]{HM94a} und \citet[\page 164]{Suchsland97a} nehmen an,
        dass \emph{sehen} eine VP einbettet. Mit einer solchen Analyse wird
        es schwierig zu erklären, warum das Subjekt von \emph{sehen} zwischen
        Argumenten des eingebetteten Verbs stehen kann. Man braucht
        dazu eine andere Analyse für lokale Umstellungen
        oder muss wie \citet{Reape94a} diskontinuierliche Konstituenten
        annehmen. Zu Reapes Ansatz siehe auch Kapitel~\ref{sec-Reape-Linearisierung}.%
%Einige Probleme, die sich für Reapes Ansatz ergeben,
%        wurden bereits im Kapitel~\ref{sec-domain-union-fuer-verbalkomplex} diskutiert.
}
%% $^,$\footnote{
%%         \citet[\page 217]{Kiss95a} gives a similar lexical entry for \emph{sehen},
%%         but he requires that the embedded verb have a subject by instantiating \iboxt{2}
%%         with \sliste{ NP }. This rules out sentences like (\ref{ex-sah-ihm-schlecht-werden}).
%% }
\eas
\label{le-sehen}%
\stem{seh} (AcI-Verb, obligatorisch kohärent\is{Kohärenz|(}):\\
\onems{ cat$|$arg-st \sliste{ NP[\type{str}]\ind{1} } $\oplus$ \ibox{2} $\oplus$ \ibox{3} $\oplus$ \sliste{ \textrm{V[}%
\type{bse}, \textsc{lex}+, \textsc{subj}~\ibox{2}, \textsc{comps}~\ibox{3}, \ltop \ibox{4}] }\\[2mm]
cont \ms{ ind  & \ibox{5} event\\
          ltop & \ibox{6}\\
        }\\
rels \liste{  \ms[sehen]{
              lbl  & \ibox{6}\\
              arg0 & \ibox{5}\\
              arg1 & \ibox{1}\\
              arg2 & \ibox{7}\\
             } }\\[8mm]
hcons \liste{ \ms[qeq]{
              harg & \ibox{7}\\
              larg & \ibox{4}\\
              } }
        }
\zs\iw{sehen|uu}
Das Subjekt des eingebetteten Verbs wird angehoben \iboxb{2}, wenn es eins gibt.
Genauso werden die anderen Argumente des eingebetteten Verbs \iboxb{3} angehoben.
Verlangt das eingebettete Verb ein Subjekt, so steht dieses an zweiter Stelle in
der \argstl von \emph{sehen} und bekommt deshalb in Aktivsätzen
Akkusativ und in Passivsätzen -- in denen das erste Argument von \emph{sehen} unterdrückt wird -- 
Nominativ (vergleiche das Kasusprinzip\is{Prinzip!Kasus-} auf S.\,\pageref{case-p}). 
Regiert das eingebettete Verb wie in (\mex{1}) noch weitere Nominalphrasen mit strukturellem Kasus, bekommen
diese ebenfalls Akkusativ.
\ea
Aicke sieht ihn den Aufsatz lesen.
\z

\noindent
Alle angehobenen Argumente und das Argument, das \emph{sehen} selbst mitbringt,
können in Bezug zueinander umgestellt werden. Wie in Abschnitt~\ref{sec-aci-perm-mf}
diskutiert, unterliegen solche Umstellungen von Argumenten eines Verbalkomplexes
bestimmten Performanzfaktoren\is{Performanz}, so dass die Umstellungen nicht im gleichen Maße
wie bei einfachen Verben möglich sind. Diese Performanzfaktoren müssen aber in
unserer Grammatik nicht modelliert werden.

Wie das Subjektanhebungsverb \emph{scheinen} konstruieren AcI"=Verben obligatorisch
kohärent. Dies wird durch die Spezifikation des \lexwes des eingebetteten Verbs
sichergestellt.

Der \ltopw des eingebetteten Verbs \iboxb{4} ist \qeq mit dem \argtwo von \emph{sehen}
und das Subjekt von \emph{sehen} mit der \textsc{experiencer}"=Rolle, hier \argone \iboxb{1}. Das angehobene
Subjekt des eingebetteten Verbs -- so es denn eins gibt -- füllt keine Argumentrolle von \emph{sehen}. 
Das wurde \vpagerefrange{page-start-aci-verbs-role}{page-end-aci-verbs-role} motiviert.
%on page~\pageref{page-start-aci-verbs-role}.
Die \qeq\hyp Beschränkung lässt die Einbettung von Quantoren unter \relation{sehen} zu. So, wie
Einhörner nicht existieren müssen, wenn sie sich zu nähern scheinen, kann es sein, dass man etwas
sieht, das nicht wirklich da ist:
\eal
\ex Conny sieht eine Hexe tanzen.
\ex Aicke sieht ein Einhorn kommen.
\zl

\is{Verb!Objektanhebungs-|)}%
\is{Verb!AcI-|)}%
\is{Anhebung|)}\is{Kohärenz|)}


\subsection{Objektkontrollverben}
\label{sec-object-control-anal}

\is{Verb!Objektkontroll-|(}%
(\mex{1}) zeigt den \localw des Lexikoneintrags des Objektkontrollverbs \emph{erlauben}:


\eas
\stem{erlaub} (Objektkontrollverb, optional kohärent\is{Kohärenz|)}\is{Inkohärenz}):\\
\ms{
cat & \ms{% head   & verb \\
           arg-st & \begin{tabular}[t]{@{}l@{}}
                    \sliste{ NP[\str]\ind{1}, NP[\type{ldat}]\ind{2} } $\oplus$ \ibox{3} $\oplus$\\[2mm]
                    \sliste{ \textrm{V[\type{inf}, \textsc{subj}~\sliste{NP[\type{str}]\ind{2}},
                      \textsc{comps}~\ibox{3}, \ltop \ibox{4}]}}\\
                    \end{tabular}
% \begin{tabular}{@{}l@{}l}
%                             \textrm{V[}&\textrm{\type{inf}, \textsc{lex}+, \textsc{subj}~\sliste{NP[\type{str}]\ind{2}},}\\
%                                     & \textrm{\textsc{comps}~\ibox{3} ]:\ibox{4}}\\
%                             \end{tabular}}\\
         }\\
cont & \ms{ ind  & \ibox{5} event\\
            ltop & \ibox{6}\\
          }\\ 
rels & \liste{ \ms[erlauben]{
        lbl  & \ibox{6}\\
        arg0 & \ibox{5}\\
        arg1 & \ibox{1}\\
        arg2 & \ibox{2}\\
        arg3 & \ibox{4}\\
       }}\\[8mm]
hcons & \eliste\\
}
\zs\iw{erlauben|)uu}

\noindent
Das Dativobjekt ist das kontrollierende Element und wie beim Eintrag des Subjektkontrollverbs
mit dem kontrollierten Subjekt koindiziert. Die Koindizierung des Subjekts des eingebetteten
Verbs mit einer referentiellen Nominalphrase schließt sowohl die Einbettung expletiver Prädikate
als auch die Einbettung subjektloser Konstruktionen aus.

Wird dieser Lexikoneintrag mit finiter Form von \emph{erlauben} in der kohärenten Konstruktion genutzt,
so sind die Komplemente des eingebetteten Verbs auch Argumente des gesamten Verbalkomplexes, und
es ist erklärt, warum Argumente des Matrixverbs und Argumente des eingebetteten Verbs
relativ zueinander umgeordnet werden können.

Bettet \emph{erlauben} ein Verb ein, das ein Objekt mit strukturellem Kasus regiert, so steht
dieses in der kohärenten Konstruktion an der dritten Stelle in der \argst von \emph{erlauben}. Das Kasusprinzip\is{Prinzip!Kasus-} weist dem ersten Element in der \argstl, das strukturellen
Kasus hat (dem Subjekt von \emph{erlauben}), Nominativ zu. Das angehobene Objekt des eingebetteten Verbs
mit strukturellem Kasus erhält dagegen Akkusativ. Das Dativobjekt von \emph{erlauben} steht
an zweiter Stelle der \argstl, hat aber lexikalischen Kasus und wird deshalb bei der Kasuszuweisung ignoriert.
In Passivsätzen wird das Subjekt von \emph{erlauben} unterdrückt. Bei Verbalkomplexbildung
und Einbettung eines transitiven Verbs ergibt sich für den Komplex die \argstl in (\mex{1}):
\ea
\argst \sliste{ NP[\type{ldat}], NP[\type{str}], V }
\z
Das erste Element mit strukturellem Kasus in dieser Liste ist das Objekt des eingebetteten
Verbs. Das Kasusprinzip weist diesem Element Nominativ zu. Haiders Satz in (\mex{1}),
der bereits auf Seite~\pageref{erfolg-auszukosten-erlaubt-kasus} diskutiert wurde,
wird also von der hier vorgeschlagenen Analyse korrekt erfasst.
\ea
\iw{auskosten}
Der Erfolg        wurde uns      nicht auszukosten erlaubt.\footnote{
        \citew[\page 110]{Haider86c}.%
}
\z
Zu den Details der Passivanalyse siehe Kapitel~\ref{sec-remote-passive-hpsg}.
\is{Verb!Objektkontroll-|)}%
\is{Kohärenz|)}%

\section{Alternativen}

\begin{comment}
\subsection{Einbettung von Verbalphrasen oder Sätzen in AcI-Konstruktionen}

Für die Behandlung von AcI"=Konstruktionen gibt es viele verschiedene Vorschläge.  So werden sie \zb
als Kontrollkonstruktionen behandelt. In der Transformationsgrammatik werden solche Sätze mittels
Equi"=NP"=Deletion analysiert, \dash, (\mex{1}a) wird durch eine Löschung von einer doppelt vorhandenen
NP aus den beiden Sätzen in (\mex{1}b) erzeugt \citep{Bierwisch63a}:\addpages

\eal
\ex Ich sehe ihn kommen.
\ex Ich sehe ihn. Er kommt.
\zl
Wie die Diskussion auf den Seiten \vpagerefrange{page-start-aci-verbs-role}{page-end-aci-verbs-role}
jedoch gezeigt hat, ist das semantisch nicht adäquat, denn es ist nicht immer so, dass das Agens des
eingebetteten Verbs wahrgenommen wird.

Alternativ wurde vorgeschlagen, dass das AcI"=Verb einen Satz mit infinitem Verb einbettet, \dash eine
Projektion des Verbs, die das Subjekt enthält. Bei solchen Ansätzen muss entweder das AcI"=Verb in
eine abgeschlossene Projektion hinein Kasus zuweisen, was auch als \emph{Exceptional Case Marking}
bezeichnet wird oder man geht davon aus, dass es spezielle Infinitivformen des Verbs gibt, die ihrem
Subjekt Akkusativ zuweisen. In der hier vorgestellten Theorie ist eine lokale Kasuszuweisung
möglich, und man kommt mit einer Infinitivform für Futurkonstruktionen (\mex{1}a),
AcI"=Konstruktionen (\mex{1}b) und sogar auch das \emph{lassen}"=Passiv (\mex{1}c) aus.
\eal
\ex Er wird den Wagen reparieren.
\ex Sie lässt ihn den Wagen reparieren.
\ex Sie lässt den Wagen reparieren.
\zl
Zu den Details der Kasuszuweisung und zum \emph{lassen}"=Passiv siehe Kapitel~\ref{Kapitel-passiv}.

\citet[\page 1423--1425]{Zifonun97c} nimmt an, dass AcI"=Verben eine Verbalprojektion einbetten,
die bis auf die Nominativstelle gesättigt ist. Statt des Nominativarguments des eingebetteten Verbs
wird dann vom Matrixverb ein Akkusativobjekt verlangt. Überträgt man diese Analyse in die
HPSG"=Notation, ergibt sich folgendes:

\eas
AcI"=Verb nach \citet[\page 1425]{Zifonun97c} übertragen in HPSG-Format:\\
\ms{ head   & verb \\
           comps & \begin{tabular}{@{}l@{}}
                    \sliste{ NP[\type{nom}], NP[\type{acc}]\ind{1} } $\oplus$ \sliste{ \textrm{ VP[\type{bse}, \textsc{subj}~\sliste{ NP[\type{nom}]\ind{1}}]}}\\
                    \end{tabular}
% \begin{tabular}{@{}l@{}l}
%                             \textrm{V[}&\textrm{\type{inf}, \textsc{lex}+, \textsc{subj}~\sliste{NP[\type{str}]\ind{2}},}\\
%                                     & \textrm{\textsc{comps}~\ibox{3} ]:\ibox{4}}\\
%                             \end{tabular}}\\
}
\zs
Diese Analyse ähnelt den Analysen, die für Kontrollverben vorgeschlagen wurden, da das
Akkusativobjekt des Matrixverbs mit dem Subjekt des eingebetteten Verbs koindiziert ist. Man kann
aber nicht von einer Kontrollanalyse sprechen, wenn nicht gleichzeitig noch behauptet wird, dass das
Akkusativobjekt eine semantische Rolle des AcI"=Verbs füllt. Man vergleiche auch den Lexikoneintrag
für das Dativpassivhilfsverb im Kapitel~\ref{sec-anal-dativpassiv}.

Außerdem hat die Datendiskussion gezeigt, dass Argumente des Infinitivs vor Argumenten des AcI"=Verbs
angeordnet werden können. Geht man davon aus, dass 

%Hier muss man mit den unpersönlichen Einbettungen argumentieren. Dafür gibt es aber bei Google keinen Beleg.

\subsection{Diskontinuierliche Konstituenten und \emph{Clause Union}}
\end{comment}
\label{sec-anhebung-diskontinuierliche-konstituenten}


\mbox{}\citet{Reape94a} arbeitet in einer HPSG"=Variante, die diskontinuierliche
Konstituenten zulässt (siehe auch Kapitel~\ref{sec-Reape-Linearisierung}). 
Er schlägt vor, kohärente Konstruktionen
als Satzvereinigung (\emph{Clause Union})\is{Domain Union@\emph{Domain Union}|(} zu analysieren.
Für (\ref{ex-weil-es-ihm-jemand-zu-lesen-versprochen-hat-zwei}) -- hier als (\mex{1})
wiederholt -- nimmt er an, dass \emph{es zu lesen} eine Phrase ist, die unter
\emph{ihm versprochen} eingebettet ist, und die so entstandene Phrase ist unter \emph{jemand hat}
eingebettet.
\ea
\label{ex-weil-es-ihm-jemand-zu-lesen-versprochen-hat-drei}
weil    es       ihm       jemand   zu lesen versprochen hat\footnote{
\textcites[\page 110]{Haider86c}[\page 128]{Haider90b}.%
}
\z
Die Phrase \emph{es zu lesen} ist eine diskontinuierliche Maximalprojektion, \dash
bei der Analyse von (\mex{0}) befindet sich zwischen \emph{es} und \emph{zu lesen} 
ein Zwischenraum. Die Bestandteile der Phrase \emph{es zu lesen} werden einzeln
zu den Bestandteilen der einbettenden Phrasen hinzugefügt. Projektionen verfügen
über eine Liste mit Elementen, aus denen sie bestehen (die \doml, siehe
Kapitel~\ref{sec-Reape-Linearisierung}). Die Elemente in solchen Listen 
können frei angeordnet werden, vorausgesetzt, es werden keine Linearisierungsbeschränkungen
verletzt. Da sich \emph{es}, \emph{ihm} und \emph{jemand} nach der Einsetzung von \emph{es}
und \emph{ihm} in die übergeordneten Listen innerhalb derselben Liste befinden, können
diese Elemente in jeder Reihenfolge angeordnet werden. Das wurde im vorangegangenen
Abschnitt durch Argumentanziehung und freie Abbindung der Argumente im Kopf"=Argument"=Schema
erreicht.

Für Anhebungsverben wie \word{scheinen} nimmt Reape an, dass das Anhebungsverb einen nicht"=finiten
Satz einbettet, der auch das Subjekt enthält. Für (\mex{0}) wäre
\emph{niemand diese Bücher zu lesen} also ein Satz, der unter  \emph{scheint} eingebettet ist.
\ea
weil    niemand diese Bücher zu lesen scheint
\z
Das Problem ist nun, dass die Nominativ"=NP in (\mex{0}) mit \emph{scheint} kongruiert, was man noch besser
sieht, wenn man den Satz mit der Singular-NP mit einem mit einer Plural"=NP vergleicht:
\ea
weil    alle diese Bücher zu lesen scheinen
\z
Diese Tatsache kann man in Reapes Ansatz nicht erklären, es sei denn man würde
annehmen, dass das infinite Verb \emph{zu lesen} Kongruenzmerkmale hat, die mit dem
Subjekt von \emph{zu lesen} und außerdem noch mit denen von \emph{scheint} übereinstimmen
müssen \parencites[Abschnitt~5.1]{Kathol98b}[Kapitel~21.1]{Mueller99a}. 
Da es bei der infiniten Verbform keinen morphologischen Reflex der Kongruenzmerkmale
gibt, wäre eine solche Lösung ad hoc. Die in diesem Buch vorgestellte Analyse, in der
ein eventuell vorhandenes Subjekt des eingebetteten Verbs zum Subjekt des Matrixverbs \emph{scheinen}
wird und dann mit diesem kongruieren kann, wenn das Matrixverb finit ist, ist also vorzuziehen.

Außer den Kongruenzdaten stellt auch das sogenannte Fernpassiv, das im Kapitel~\ref{sec-remote-passive-phen}
diskutiert wird, ein Problem für Reapes Analyse dar \citep[Abschnitt~5.2]{Kathol98b}: In
Sätzen wie (\mex{1}) wird ein Objekt eines eingebetteten Infinitivs zum Subjekt.
\ea
weil der Wagen oft zu reparieren versucht wurde\footnote{
  Siehe \citew[\page 176]{Hoehle78a} für ein ähnliches Beispiel.
}
\z
Das lässt sich mit Theorien, die Argumentanziehung für die Analyse des Verbalkomplexes verwenden,
leicht erklären: Das Objekt von \emph{zu reparieren} ist bei Verbalkomplexbildung gleichzeitig
auch ein Objekt von \emph{zu reparieren versuchen}. Da die Argumentanziehung bereits im Lexikoneintrag
des Verbs \emph{versuchen} angelegt ist, kann das Objekt bei Passivierung auch zum Subjekt werden.
Geht man dagegen davon aus, dass \emph{versuchen} eine Infinitiv"=VP einbettet, ist \emph{Wagen}
kein Argument von \emph{versuchen}. Eine solche Theorie sagt fälschlicherweise voraus, dass nur das unpersönliche
Passiv in (\mex{1}) möglich ist:
\ea
weil oft den Wagen zu reparieren versucht wurde
\z
Die im vorigen Abschnitt vorgestellte Theorie kann dagegen sowohl das unpersönliche Passiv
in (\mex{0}) als auch das Fernpassiv in (\mex{-1}) erklären: In (\mex{-1}) konstruiert
\emph{versuchen} kohärent und in (\mex{0}) inkohärent. Der Kasus des Objekts von \emph{reparieren}
ergibt sich ganz normal aus der Interaktion zwischen Passivanalyse und Zuweisung von strukturellem
Kasus. Zu den Details siehe Kapitel~\ref{sec-remote-passive-hpsg}.%
\is{Domain Union@\emph{Domain Union}|)}

\section{Anhang}

In diesem Anhang möchte ich einige offene Punkte kurz ansprechen. Der Anhang ist eher als eine
ausführlichere Sammlung von Literaturhinweisen zu verstehen. 

\citet{Bech55a}
beschreibt neben den hier diskutierten Anordnungen von Verben in kohärenten Konstruktionen noch die sogenannte
Oberfeldumstellung\is{Oberfeldumstellung}\is{Feld!Ober-}. (\mex{1}) zeigt ein Beispiel für dies Konstruktion:
\ea
dass er das Lied wird haben singen können
\z
In (\mex{0}) stehen die Verben \emph{wird} und \emph{haben} vor dem Verbalkomplex, den
sie einbetten, obwohl in kohärenten Konstruktionen Verben normalerweise rechts des Verbalkomplexes
stehen, den sie einbetten. Die Analyse dieses Phänomens wurde ebenfalls von
\citet{HN89b,HN94a} ausgearbeitet und ist recht einfach in das hier vorgestellte Grammatikfragment zu
integrieren. Aus Platzgründen habe ich jedoch darauf verzichtet. %Siehe aber \citew{MuellerGermanic}.

Eine weitere Stellungsvariante, die von Bech jedoch nicht beschrieben wird, ist die sogenannte
dritte Konstruktion\is{dritte Konstruktion} \citep*{dBR89}. Bei dieser Konstruktion handelt es sich
auch um eine Komplexbildung: ein Infinitiv ist evtl.\ mit Argument nachgestellt, aber mindestens ein Argument
ist im Mittelfeld realisiert. \citet[344--345]{Mueller99a} gibt folgende Beispiele:\footnote{%
(\ref{bsp-es-faellt-auf})--(\ref{bsp-ganz-gleich}) sind von \citet[86]{vdVelde77}
und (\ref{bsp-ernuechtert}) ist von \citet[155]{Kvam80}.}
\eal
\ex Langsam fing mir die Sache an,\iw{anfangen} Spaß zu machen.\iw{Spaß machen}\footnote{
	taz, 29.09.95, S.\,20
	}
\ex Aber daß du nicht gleich wieder umgekehrt bist, als es dir anfing schlecht zu gehen!\footnote{
        Hesse, zitiert nach \citep[112]{Bech55a}
}
\ex es fällt auf, wie häufig dieser Mensch sich glaubte\iw{glauben} entschuldigen zu müssen.\footnote{
        Max Frisch. {\em Stiller\/}
      }\label{bsp-es-faellt-auf}
\ex gerade in den Bereichen, für die er in seiner Dissertation die überzeugendsten Signifikanzen
        des schichtspezifischen Sprachverhaltens glaubt nachweisen zu können
\ex ganz gleich, wieviel Abschriften man zwischen beide noch glaubt legen
        zu müssen\label{bsp-ganz-gleich}
\ex Ernüchtert lebte der Heimkehrer in einem Land, das "`übelste preußische
      Provinz"' war, in der man alles meinte\iw{meinen} kommandieren zu können.\footnote{
        Stern, 50/76, S.\,144
      }\label{bsp-ernuechtert}
\ex Die Abneigung Fillmores gegen die Logik geht so weit, daß er auch einfache
      logische Anforderungen an Theorien in seiner Darstellung nicht glaubt berücksichtigen
      zu müssen, \ldots\footnote{
        Im Haupttext von \citep[48]{Heringer73a}.
      }
\zl
Eine Analyse dieser Konstruktion im
Rahmen der HPSG findet man in \citew[Kapitel~17.5]{Mueller99a}.

Im\is{Adjunkt|(}\is{Skopus|(} 
Analyseteil wurde auf Adjunkte nicht eingegangen. Wie in \citew{Mueller2006b} dargelegt, gibt es
mehrere Möglichkeiten für die Analyse von Adjunkten im Zusammenhang mit kohärenten
Konstruktionen. Wenn man eine Kombination des Adjunkts mit einem Verbalkomplex vor der Einbettung
unter andere Verben erlaubt, kann man folgende Strukturen annehmen:
\eal
\label{anal-adjunct-VP-Einbettung}
\ex weil er den Wagen [[gestern reparieren] wollte]
\ex weil er [[gestern den Wagen reparieren] wollte]
\zl
Eine solche Analyse setzt voraus, dass der \lexw der Mutter in Kopf"=Adjunkt"=Strukturen
unspezifiziert ist. In diesem Zusammenhang sind die folgenden Beispiele von \citet[\page
  407]{Fanselow2001a} interessant, die zu zeigen scheinen, dass Umstellungen auf Argumente beschränkt
sind.
\eal
\judgewidth{§}
\label{bsp-adjunkte-in-kohaerenten-konstr}
\ex[]{
dass niemand morgen ein Buch zu lesen versprach
}
\ex[§]{
dass morgen niemand ein Buch zu lesen versprach
}
\zl
Der Satz in (\mex{0}b) ist abweichend, was Fanselow darauf zurückführt, dass \emph{morgen}
durch seine Stellung bedingt nur Skopus über \emph{versprach} haben kann und dass das
Präteritum von \emph{versprach} nicht mit der im Adverb enthaltenen Aussage, dass etwas in
der Zukunft stattfinden wird, verträglich ist.

Ich denke aber, dass auch hier Performanzphänomene\is{Performanz} eine Rolle spielen, denn wenn Skopus
über das übergeordnete Verb (das Matrixverb\is{Matrixverb}) unplausibel ist, ist die Voranstellung eines Adjunkts möglich, wie
(\mex{1}) zeigt:
\eal
\ex dass so schnell niemand zu laufen versuchte
\ex Bei dem Wetter wird ohne Regenmantel ein besorgter Vater seine Kinder niemals aus dem Haus gehen lassen.\footnote{
\citew[\page 383]{Crysmann2004a}.
}
\ex Der diensthabende Beamte gab zu Protokoll, dass in der Dachwohnung zum fraglichen Zeitpunkt ein Rentner von der
    anderen Straßenseite aus die Angeklagte mehrmals auf das Opfer einstechen sah.\footnote{
\citew[\page 384]{Crysmann2004a}.
}
\zl
Wenn strukturelle Gründe für die Abweichung in (\mex{-1}) verantwortlich wären, wäre die Analyse in
(\ref{anal-adjunct-VP-Einbettung}) ausreichend. (\mex{0}a) kann man jedoch nicht so analysieren, da
\emph{so schnell} und \emph{zu laufen} nicht adjazent sind und somit keine Phrase bilden
können. Entsprechendes gilt für die anderen Beispiele in (\mex{0}). Es
gibt mehrere Vorschläge in der HPSG"=Literatur, die dieses Problem lösen: \citet{NB94} schlagen eine
Lexikonregel vor, die Modifikatoren in \compsln einfügt. Damit sind die Modifikatoren Argumente des
Verbs, über das sie Skopus haben. Bei der Verbalkomplexbildung werden auch die Modifikatoren genau wie die
anderen Argumente vom einbettenden Verb angehoben. Somit ist erklärt, wieso Modifikatoren, die
eingebettete Verben modifizieren, vor Argumenten übergeordneter Verben stehen können. \citet*{MSI99a}
schließen sich \citet{NB94} an und machen einen entsprechenden Vorschlag für
Japanische\is{Japanisch} Kausativkonstruktionen.\is{Kausativ!-konstruktion} Solche lexikalischen
Analysen wurden von Bob Levin stark kritisiert, da sie mit bestimmten Koordinationsdaten\is{Koordination} sowohl
syntaktisch als auch semantisch unverträglich sind
\citep{Levine2003a,LH2006a}. \citet[Abschnitt~6]{Cipollone2001a} stellt eine Analyse vor, die
Speichermechanismen\is{Speicher} für semantische Repräsentationen verwendet. Die semantischen Beiträge der
Verben, die an einer Komplexbildung beteiligt sind, werden nicht direkt verrechnet, sondern in einer
Liste mit $\lambda$"=Termen repräsentiert, die für vollständige Sätze über $\beta$"=Konversion zu
einer semantischen Repräsentation des Gesamtsatzes umgewandelt werden können. Da die semantischen
Beiträge einzelner Verben auch in der semantischen Repräsentation des Verbalkomplexes zugänglich
sind, können Adjunkte, die syntaktisch mit dem gesamten Komplex kombiniert werden, Skopus über tief
eingebettete Verben haben. Siehe auch \citet{Crysmann2004a} zu einem ähnlichen Vorschlag mit einer \mrs"=Semantik.




\begin{comment}
\subsection{Lexicalized Tree Adjoining Grammar und die Kompetenz/""Performanz"=Unterscheidung}

In der formalen Grammatiktheorie wird erforscht, welcher Komplexitätsklasse Grammatiken
angehören müssen, um natürliche Sprache adäquat beschreiben zu können. \citet{Chomsky56a-u} hat vier
Klassen definiert: die regulären Grammatiken, die kontextfreien Grammatiken, die kontextsensitiven
und die unrestringierten Grammatiken. Man ist sich inzwischen einig, dass kontextfreie Grammatiken
zwar große Ausschnitte von natürlichen Sprachen beschrieben können, aber nicht allen Phänomene in
allen Sprachen eine Analyse zuordnen können. Man ist sich relativ sicher, dass die natürlichen
Sprachen einer Teilklasse der kontextsensitiven Sprachen entsprechen. Eine solche Teilklasse sind
zum Beispiel die \emph{Mildly Context Sensitive Grammars}.

\citet*{JBR2000a} diskutieren Beispiele für die Selbsteinbettung von Relativsätzen wie die in (\mex{1})
und folgen \citet{MC63a-u} in der Annahme, dass die Tatsache, dass solche Einbettungen nur bis zur
Ebene drei möglich sind, nicht in der Grammatik beschrieben werden soll, sondern auf Verarbeitungsprobleme
des Hörers zurückzuführen ist, die unabhängig von seinen prinzipiellen Fähigkeiten in Bezug
auf Grammatik sind.
\ea
Der Hund, [\sub{RS} der die Katze, [\sub{RS} die die Maus gefangen hat,] jagt ] bellt.
\z
Interessant in diesem Zusammenhang ist, dass man die Beispiele für Selbsteinbettung so konstruieren
kann, dass sie für Hörer leichter verarbeitbar sind. Auf diese Weise kann man die Zahl der prinzipiell
verarbeitbaren Selbsteinbettungen um eins erhöhen und zeigen, dass alle Grammatiken, die eine Beschränkung
auf zwei eingebettete Relativsätze formuliert haben, falsch sind.
Das folgende Beispiel von Uszkoreit ist leichter verarbeitbar, weil die eingebetteten Relativsätze als
solche isoliert sind und die Verben durch Material aus dem übergeordneten Satz voneinander getrennt sind.
\ea
\z

\citet{JBR2000a} diskutieren außerdem Verbalkomplexe mit umgestellten Argumenten. Das allgemeine Muster,
das sie betrachten hat die Form in (\mex{1}):
\ea
$\sigma$(NP$_1$ NP$_2$ \ldots NP$_n$) V$_{n}$V$_{n}$ \ldots V$_{1}$
\z
Dabei is $\sigma$ eine beliebige Permutation der Nominalphrasen und V$_{1}$ ist das finite Verb.
Die Autoren untersuchen die Eigenschaften von Lexicalized Tree Adjoining Grammar (LTAG) in Bezug auf
diese Muster und stellen fest, dass LTAG die Abfolge in (\mex{1}) nicht analysieren kann, wenn man
Wert auf die richtige Semantik legt.
\ea
NP$_2$ NP$_3$ NP$_1$ V$_{3}$V$_{2}$V$_{1}$
\z
Sie schlagen deshalb eine Erweiterung von LTAG vor, die sogenannte \emph{tree"=local multi"=component LTAG}
(Tree"=local MC"=LTAG). Sie zeigen, dass MC-LTAG zwar (\mex{0}) nicht aber (\mex{1}) mit korrekter Semantik
analysieren kann. Sie behaupten, dass solche Muster im Deutschen nicht möglich sind und argumentieren, dass 
man hier im Gegensatz zu den Relativsatzbeispielen die Möglichkeit hat, diese Tatsache sowohl als
Performanz- als auch als Kompetenzphänomen zu behandeln.
\ea
\label{ex-mc-ltag-fails}
NP$_2$ NP$_4$ NP$_3$ NP$_1$ V$_{4}$V$_{3}$V$_{2}$V$_{1}$
\z
Wenn man es als Performanzphänomen behandelt, bezieht man sich auf die Komplexität der Konstruktion
und die sich daraus ergebenden Verarbeitungsprobleme für den Hörer. Das Nichtvorkommen solcher Abfolgen
in Korpora kann man unter Bezugname auf das Prinzip der Kooperativität erklären. Sprecher wollen normalerweise
verstanden werden und formulieren ihre Sätze deshalb so, dass der Hörer sie auch verstehen kann.
Da man hochkomplexe kohärente Konstruktionen oft dadurch vereinfachen kann, dass man Teile extraponiert
und da man auf diese Weise auch Ambiguität vermeidet (siehe Abschnitt~\ref) sind Verbalkomplexe
mit mehr als vier Verben kaum zu finden.

Die Alternative besteht eben darin, einen Grammatikformalismus zu benutzen, der genau so mächtig
ist, dass er die Einbettung von zwei Verben mit Umstellung erlaubt, aber die Einbettung von drei
Verben mit entsprechend umgestellten Argumenten ausschließt. \citet{JBR2000a} entscheiden sich für
diese Lösungen und ordnen somit die Unmöglichkeit der Umstellung von Argumenten in (\mex{0})
der Kompetenz zu.

Im vorliegenden Buch habe ich vorgeschlagen, Verbalkomplexe über Argumentkomposition zu analysieren.
Die Verbalkomplexe verhalten sich dann wie Simplexverben, weshalb auch alle Argumente der beteiligten
Verben in beliebiger Reihenfolge angeordnet werden können. Die Grammatik enthält keine Beschränkung
in Bezug auf die Anzahl der kombinierbaren Verben bzw.\ Beschränkungen, die ab einer gewissen Einbettungstiefe
Umstellungen verbieten. Ich gehe statt dessen davon aus, dass viele Umstellungen durch Regeln ausgeschlossen
sind, die schon bei einfachen zweistelligen Verben angewendet werden. Die Unmöglichkeit der Einbettung
von vier und mehr Verben schreibe ich tatsächlich auch der Performanz zu. 

Bevor ich meine Argumente gegen einen kompetenzbasierten Ausschluss von (\mex{1}) vorbringe,
möchte ich noch eine generelle Anmerkung machen: Es ist sehr schwierig im vorliegenden Phänomenbereich
Entscheidungen über Analysen zu treffen: Korpora helfen hier nicht weiter, da man in Korpora keine
Belege mit vier oder mehr eingebetteten Verben findet. \citet{Bech55a} stellt eine extensive Materialsammlung
zur Verfügung, aber die Beispiele mit vier eingebetteten Verben sind konstruiert.
Meurers zitiert deskriptive Grammatiken, die Beispiele mit sechs Verben diskutieren, die mehrere Modalverben
enthalten. Diese Beispiele sind aber wirklich nicht mehr verarbeitbar und auch für die Diskussion hier
nicht relevant, da die Verben eigene Argumente selegieren müssen. Für die Konstruktion von Beispielen
bleiben also nicht viele Verben übrig. Man kann für die Konstruktion von Beispielen nur Subjektkontrollverben
mit zusätzlichem Objekt (\zb \emph{versprechen}), Objektkontrollverben (\zb \emph{zwingen})
oder AcI"=Verben (\zb \emph{sehen} oder \emph{lassen}) verwenden. Bei der Konstruktion von Beispielen
muss man außerdem beachten, dass die beteiligten Nomina möglichst in Bezug auf ihren Kasus und auf 
Selektionsrestriktionen verschieden sind, denn das sind die Merkmale, die ein Hörer/""Leser verwenden
kann, um eventuell umgestellte Argumente ihren Köpfen zuzuordnen. An dieser Beschreibung sieht man schon,
dass die Wahrscheinlichkeit, dass in einem Zeitungstext ein entsprechender Satz gefunden werden kann,
extrem gering ist, was schon daran liegt, dass es wenige Situationen gibt, in denen entsprechende Äußerungen
denkbar sind. Hinzu kommt, dass mit Ausnahme von \emph{helfen} alle Kontrollverben einen Infinitiv mit
\emph{zu} verlangen und auch inkohärent realisiert werden können. Wie oben bereits erwähnt, wird
der kooperative Sprecher/""autor eine weniger komplexe Konstruktion verwenden, was die 
Auf"|tretenswahrscheinlichkeit entsprechender Sätze noch weiter verringert.

Man beachte, dass MC-LTAG die Anzahl der Verben im Satz nicht beschränkt. Der Formalismus lässt
beliebig viele Verben zu. Man muss also genauso, wie ich es hier für die HPSG"=Analyse tue, davon ausgehen,
dass Performanzbeschränkungen dafür zuständig sind, dass man keine Belege für Verbalkomplexe mit fünf
Verben finden kann. MC-LTAG macht Vorhersagen über die Umstellbarkeit von Argumenten. Ich halte
es für falsch, Beschränkungen in Bezug auf Umstellbarkeit an der Mächtigkeit des Grammatikformalismus
fest zu machen, denn die Beschränkungen, die man finden kann, sind unabhängig von Verbalkomplexen
bereits bei einfachen Verben mit zwei Argumenten wirksam.
Das Problem bei Umstellungen ist, dass es irgendwie möglich sein muss die Nominalphrasen ihren Köpfen
zuzuordnen. Wenn eine solche Zuordnung zu Ambiguitäten führt, die nicht durch Bezug auf
Kasus, Selektionsrestriktionen oder Intonation auf"|lösbar sind, wird die unmarkierte Konstituentenstellung
angenommen.
\citep*[\page 68]{Hoberg81a} demonstriert das sehr schön mit Beispielen, die den folgenden
ähneln:
\eal
\ex[]{
Hanna hat immer schon gewußt, dass das Kind sie verlassen will.\footnote{        
        \citep*[\page 68]{Hoberg81a} gibt ähnliche Beispiele. Allerdings verwendet sie statt \emph{das} das
        Possessivpronomen \emph{ihr}. Das macht die Sätze semantisch plausibler, allerdings
        bekommt man Interferenzen mit Linearisierungsbedingungen, die für gebundene
        Pronomina gelten (siehe oben). Ich habe deshalb das Pronomen durch einen Artikel ersetzt.
}
}
\ex[\#]{
Hanna hat immer schon gewußt, dass sie das Kind verlassen will.
}
\ex[]{
Hanna hat immer schon gewußt, dass sie der Mann verlassen will.
}
\zl
Man kann den Satz (\mex{0}a) nicht zu (\mex{0}b) umstellen, ohne eine andere
Bedeutung zu bekommen. Das liegt daran, dass sowohl \emph{sie} als auch \emph{das Kind}
morphologisch nicht eindeutig als Nominativ oder Akkusativ markiert sind. (\mex{0}b)
wird deshalb so interpretiert, dass Hanna das Kind verlassen will. Ist dagegen mindestens
eins der beiden Argumente eindeutig markiert, wird die Umstellung wieder möglich, wie (\mex{0}c)
zeigt. Bei Zählnomina fallen die Formen je nach Genus im Nominativ und Akkusativ bzw.\
im Genitiv und Dativ zusammen, bei Stoffnomina ist es jedoch noch schlimmer: Werden diese
ohne Artikel verwendet, fallen alle Kasus zusammen. Bei folgendem Beispiel
von \citep*[\page 45]{Wegener85b} kann man Dativ und Akkusativ nicht ohne Weiteres
vertauschen, obwohl das möglich ist, wenn die Nomina mit Artikel verwendet werden:
\eal
\ex Sie mischt Wein Wasser bei.\footnote{
        \citep*[\page 45]{Wegener85b}.
}\iw{beimischen}
\ex Sie mischt Wasser Wein bei.
\zl
Die beiden Nomina können nur vertauscht werden, wenn durch den Kontext klar ist,
was gemeint ist (\zb durch explizite Negation einer gegenteiligen Aussage) und wenn
der Satz entsprechend betont wird.

Das Problem mit den Verbalkomplexen ist jetzt, dass man, wenn man vier Nominalphrasen hat,
fast zwangsläufig zwei gleiche Kasus hat, wenn man nicht auf die wenigen Verben zurückgreifen
will, die einen Genitiv verlangen. Ein nicht besonders schönes
Beispiel mit vier verschiedenen, morphologisch eindeutig markierten Kasus ist (\mex{1}):
\ea
weil er den Mann dem Jungen des Freundes gedenken helfen lassen sieht
\z
Eine andere Strategie ist, Verben auszuwählen, die belebte und unbelebte Argumente
selegieren, so dass man durch die Belebtheit der Argumente Interpretationshilfen bekommt.
Ich habe ein solches Beispiel konstruiert, wobei das am tiefsten eingebettete Prädikat
kein Verb sondern ein Adjektiv ist. Bei \emph{leer fischen} handelt es sich um eine
sogenannte Resultativkonstruktion, die parallel zum Verbalkomplex analysiert werden
sollte \citep{Mueller2002b}.
\ea
weil niemand$_1$ [den Mann]$_2$ [der Frau]$_3$ [diesen Teich]$_4$  leer$_4$ fischen$_3$ helfen$_2$ sah$_1$
\z
Liest man den Satz mit entsprechenden Pausen, ist er noch verständlich. Der Kasus
der belebten Nominalphrasen ist jeweils eindeutig markiert, und unser Weltwissen
hilft uns, \emph{diesen Teich} als Argument von \emph{leer} zu interpretieren.

Der Satz in (\mex{0}) wird von einer entsprechend geschriebenen MC-LTAG und von der hier
vorgestellten HPSG mit entsprechenden Erweiterungen für Resultativkonstruktionen korrekt analysiert.
Der Satz in (\mex{1}) ist eine Variante von (\mex{0}), die genau dem Muster in (\ref{ex-mc-ltag-fails})
entspricht:
\ea
weil [der Frau]$_2$ [diesen Teich]$_4$ [den Mann]$_3$ niemand$_1$ leer$_4$ fischen$_3$ helfen$_2$ sah$_1$
\z
(\mex{0}) ist markierter als (\mex{-1}), aber das ist bei lokalen Umstellungen immer der Fall (Gisbert Fanselow, p.\,M.\ 2006).
Der Satz sollte jedenfalls nicht von der Grammatik ausgeschlossen werden. Seine Markiertheit ist vielmehr
auf dieselben Faktoren zurückzuführen, die auch für die Markiertheit von einfachen Umstellungen verantwortlich
sind. MC-LTAG kann Sätze wie (\mex{0}) nicht korrekt verarbeiten, womit gezeigt ist, dass diese Art Grammatik für die Analyse von natürlichen
Sprachen ungeeignet ist. 
%Man könnte sich jetzt auf die Suche nach einem geringfügig komplexeren Grammatikformalismus
%begeben, der (\mex{0}) gerade noch analysieren kann. Ich denke jedoch, dass man


% weil niemand den Mann dem Jungen das Buch lesen helfen lassen sieht

Gisbert:
weil so einer Frau solche Teiche einen Mann niemand leer zu fischen versprechen lassen würde 
weil so einer alten Schachtel solche blöden Teiche einen Mann wie Günter Jauch bestimmt doch niemand leer zu fischen versprechen lassen würde, der bei Verstand ist.
\end{comment}


\questions{
\begin{enumerate}
\item Wodurch unterscheiden sich Anhebungsverben von Kontrollverben?
\item Wodurch unterscheiden sich kohärente von inkohärenten Konstruktionen?
\end{enumerate}
}

\exercises{
\begin{enumerate}
\item Handelt es sich bei folgenden Verben um Kontroll- oder um Anhebungsverben?
      \eal
      \ex scheinen
      \ex anfangen
      \ex versprechen
      \ex drohen
      \ex verbinden
      \ex zwingen
      \ex hören
      \zl
      Stützen Sie Ihre Aussagen auf Belege aus Korpora, aus der Zeitung oder aus dem World Wide Web.

\item Schreiben Sie einen Lexikoneintrag für das Verb \emph{scheinen}, wie es in (\mex{1}) vorkommt:
      \eal
      \ex Mir scheint es zu regnen.
      \ex Das Problem scheint mir nicht lösbar zu sein.
      \zl


\item Laden Sie die zu diesem Kapitel gehörende Grammatik (siehe Übung~\ref{uebung-grammix-kapitel4} auf Seite~\pageref{uebung-grammix-kapitel4}).
Im Fenster, in dem die Grammatik geladen wird, erscheint zum Schluss eine Liste von Beispielen.
Geben Sie diese Beispiele nach dem Prompt ein und wiederholen Sie die in diesem Kapitel besprochenen
Aspekte.
      
\end{enumerate}
}

%\section*{Literaturhinweise}

\furtherreading{
Das Standardwerk zu Infinitivkonstruktionen im Deutschen stellt die Arbeit von Gunnar Bech aus den
Jahren \citeyear{Bech55a} und 1957 dar, die 1983 bei Niemeyer einer breiteren Öffentlichkeit zugänglich gemacht
wurde. \citet{Kiss95a} hat auf den bereits im vorigen Kapitel besprochenen Arbeiten von
\citet{HN89a,HN94a} aufbauend HPSG"=Analysen für die infiniten Verben im Deutschen entwickelt. 
In den Lexikoneinträgen für Kontrollverben, die hier vorgestellt wurden, wurde einfach die
Koindizierung eines Arguments des Matrixverbs mit dem Subjekt des eingebetteten Infinitivs
stipuliert. \citet[Kapitel~7]{ps2} zeigen, wie man die jeweilige Koindizierung aus der Bedeutung
des Verbs ableiten kann.
\is{Adjunkt|)}%
\is{Skopus|)}
}% end exewidth{(100)}
}
