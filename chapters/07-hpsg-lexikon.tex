%% -*- coding:utf-8 -*-
%%%%%%%%%%%%%%%%%%%%%%%%%%%%%%%%%%%%%%%%%%%%%%%%%%%%%%%%%
%%   $RCSfile: 7-hpsg-lexikon.tex,v $
%%  $Revision: 1.16 $
%%      $Date: 2008/09/30 09:14:41 $
%%     Author: Stefan Mueller (CL Uni-Bremen)
%%    Purpose: 
%%   Language: LaTeX
%%%%%%%%%%%%%%%%%%%%%%%%%%%%%%%%%%%%%%%%%%%%%%%%%%%%%%%%%


\chapter{Das Lexikon}
\label{chap-lexikon}

In\is{Lexikon|(} diesem Kapitel wird gezeigt, wie das Lexikon organisiert werden kann
und wie Generalisierungen erfaßt werden können.


\section{Vertikale Generalisierungen: Typhierarchien}

Durch\is{Typhierarchie|(} die Lexikalisierung der Grammatik haben wir eine enorme Reduktion der Anzahl der Dominanzschemata
erreicht. Statt viele verschiedene Grammatikregeln für verschiedene Valenzmuster zu haben, verwenden
wir nur ein sehr abstraktes Dominanzschema für Kopf"=Argument"=Strukturen. Dafür sind jetzt aber
die Lexikoneinträge sehr komplex geworden. Man kann jedoch verschiedene Generalisierungen in bezug auf
Lexikoneinträge formulieren. Zum Beispiel kann man Verben nach ihrer Valenz klassifizieren
und bestimmten Typen zuordnen, die in einer Hierarchie von Untertypen des Typs \type{word}
organisiert sind. Solcherart Generalisierungen nennt man vertikale Generalisierungen, da
es immer um die Zuordnung von Beschreibungen zu Typen in einer Hierarchie geht, wobei die
allgemeinste Beschreibung oben in der Hierarchie steht und spezifischere Beschreibungen
zu in der Hierarchie untergeordneten Typen gehören. Den Gegensatz zu solchen vertikalen
Generalisierungen bilden horizontale Generalisierungen, denen wir uns im nächsten Abschnitt zuwenden werden.


Im folgenden soll gezeigt werden, wie man einen Lexikoneintrag für ein Nomen wie \emph{Frau} repräsentieren
kann. Der Lexikoneintrag für das Nomen \emph{Frau} ist in (\mex{1}) angegeben:
\ea
\ms{
 phon & \phonliste{ \highlight{Frau}} \\
 cat  & \ms{ head & noun \\
               comps & \sliste{ Det } \\
               \ldots & \ldots\\
           } \\
 cont & \ms{
           ind   & \ibox{1} \ms{ per & 3 \\
% kommt aus der Morphologie            num & sg \\
                                       gen & \highlight{fem} \\
                                     } \\
           restr & \liste{ \ms[\highlight{frau}]{ inst & \ibox{1} \\ }} \\
           } \\
\ldots & \ldots\\
}
\z
Nur ein kleiner Teil dieser Information -- nämlich der grau unterlegte -- ist idiosynkratisch.
Die Punkte in (\mex{0}) sollen darauf hinweisen, dass in den folgenden Kapiteln noch weitere Merkmale
eingeführt werden, die auch für Nomina relevant ist, so dass sich das Verhältnis von wortspezifischer
Information zu allgemeiner Information noch zu Ungunsten der wortspezifischen Information
verändert.

Man kann die Information in (\mex{0}) wie folgt zerlegen: (\mex{1}) enthält
die Merkmal"=Wert"=Paare, die für alle Nomina gleich belegt sind:
\ea
\onems{ cat$|$head \type{noun} \\
            cont \type{nom-obj} \\
        }
\z
\type{noun} steht dabei für die Wortart und \type{nom"=obj}\istype{nom-obj} für den Typ,
den Strukturen mit einem \textsc{index}"=Merkmal und einem \textsc{restrictions}"=Merkmal
haben.

(\mex{1}) enthält die Information, die für alle referentiellen, nichtpronominalen Nomina,
die einen Determinator verlangen, zusätzlich zu (\mex{0}) relevant ist:
\ea
\onems{ cat$|$comps \sliste{ Det } \\
                       cont  \ms{ ind   & \ibox{1} \textrm{[ \textsc{per} \type{3} ]}\\
                                  restr & \liste{ \ms[psoa]{ inst & \ibox{1} \\ }, \ldots } \\
                     } \\ 
      }
\z
Solche Nomina haben mindestens eine Relation zur Beschränkung des Indexes in der Liste
ihrer Restriktionen. Das genaue Prädikat (\emph{buch}' oder \emph{haus}') hängt vom einzelnen
Wort ab und ist hier nicht festgelegt. An Stelle des Prädikats steht in (\mex{0})
der maximal unspezifische Typ für Prädikate: \type{psoa} (\type{psoa} steht für
\emph{parametrized state of affairs}. Siehe auch Seite~\pageref{page-psoa}).

%% % nur zusätzlich zu 0, denn nicht alle Feminina nehmen einen Artikel
%% Alle femininen Nomina haben zusätzlich zu (\mex{0})
%% noch folgende Eigenschaft:
%% \ea
%% \ms{ cont$|$ind$|$gen & fem \\
%%    }
%% \z
Wir haben damit Information isoliert, die für alle Nomina oder bestimmte Teilklassen
relevant ist. Wir betrachten noch die Zerlegung der Information, die zur Beschreibung
des Verbstamms\footnote{
  Bezeichnungen wie Lexem\is{Lexem}, Wurzel\is{Wurzel} und Stamm\is{Stamm} 
  werden in der Linguistik leider recht unterschiedlich verwendet. 
  Ich nehme hier an, dass \emph{Stamm} sowohl Wurzeln als auch komplexe Gebilde wie
  \emph{be+sing} bezeichnet. Im Unterschied zu Wörtern\is{Wort} sind Stämme nicht flektiert.%
% auch Bußmann
}
\stem{helf} gebraucht wird, und schauen uns dann eine Typhierarchie an,
die die Information aus beiden Zerlegungen hierarchisch gliedert.

(\mex{1}) zeigt den Lexikoneintrag für den Verbstamm \stem{helf}. Wieder nur die
grau unterlegten Werte sind idiosynkratisch.
\ea
\ms{
phon & \phonliste{ \highlight{helf}}\\
cat  & \ms{ head & verb \\
            comps & \liste{ NP[\type{nom}]\ind{1}, NP[\type{dat}]\ind{2}  } \\
          } \\
cont &  \ms[\highlight{helfen}]{ agens       & \ibox{1} \\
                                 experiencer & \ibox{2} \\
                               } \\
}
\z

\noindent
Die Information kann wie folgt aufgeteilt werden: Alle Verben teilen die Information
in (\mex{1}).
\ea
\onems{  cat$|$head \type{verb} \\
         cont       \type{psoa} \\
}
\z
Alle Verben mit Subjekt und Dativobjekt  haben zusätzlich zu (\mex{0}) eine \compsl
mit einer Nominativ- und einer Dativ"=NP:
\ea
\onems{ 
cat$|$comps \liste{ NP[\type{nom}], NP[\type{dat}] } \\
}
\z
Alle zweistelligen Verben mit \textsc{agens} und \textsc{experiencer} haben zusätzlich
zu (\mex{-1}) die folgenden Eigenschaften:
\ea
\label{bsp-linking-agens-experiencer}
\ms{ cat & \ms{ comps & \liste{ \onems{ cont$|$ind \ibox{1}\\}, \onems{ cont$|$ind \ibox{2}\\}}\\ 
              } \\
     cont & \ms[agens-exp-rel]{
                agens & \ibox{1} \\
                experiencer & \ibox{2} \\
                }\\
}
\z
Siehe auch Kapitel~\ref{sec-linking} zum Linking\is{Linking}.

Abbildung~\vref{abb-hierarchie-lexikon} zeigt, wie ein Ausschnitt einer Typhierarchie
aussehen könnte, in die man die den oben vorgenommenen Informationsaufteilungen
entsprechenden Typen integrieren müßte. \type{root} steht dabei für Wurzel.
%
\begin{figure}
\oneline{%
\begin{forest}
type hierarchy
[sign, l sep*=3
  [root,name=root]
  [{\begin{tabular}[t]{@{}c@{}}
   verb-sign\\
   \upshape [\head \type{verb}]\end{tabular}}, l sep+=8\baselineskip
    [nom-dat-verb-root, edge to= root, edge to=nom-dat-arg, edge to=agens-exp, l sep+=2\baselineskip
      [helf, instance]]]
  [{\begin{tabular}[t]{@{}c@{}}
   noun-sign\\
   \upshape [\head \type{noun}]\end{tabular}}, l sep+=8\baselineskip
    [count-noun-root, edge to=root, edge to=det-sc, edge to=3rd, l sep+=2\baselineskip
      [Frau, instance]]]
  [{\begin{tabular}[t]{@{}c@{}}
    saturated\\
   \upshape [\comps \eliste]\end{tabular}}]
  [{\begin{tabular}[t]{@{}c@{}}
    unsaturated\\
   \upshape [\comps \sliste{ [], \ldots }]\end{tabular}}
    [{\begin{tabular}[t]{@{}c@{}}
     det-sc\\
     \upshape [\comps \sliste{ Det }]\end{tabular}},name=det-sc]
    [{\begin{tabular}[t]{@{}c@{}}
     nom-arg\\
     \upshape [\comps \sliste{ NP[\type{nom}] }]\end{tabular}}]
    [{\begin{tabular}[t]{@{}c@{}}
     nom-dat-arg\\
     \upshape [\comps \sliste{ NP[\type{nom}], NP[\type{dat}] }]\end{tabular}},name=nom-dat-arg]]
  [nominal-sem-sign
    [1]
    [2]
    [{\begin{tabular}[t]{@{}c@{}}
     3\\
     \upshape [\textsc{ind|per} \type{3}]\end{tabular}},name=3rd]]
  [{\begin{tabular}[t]{@{}c@{}}
    verbal-sem-sign\\
    \upshape [\cont \type{psoa}]\end{tabular}}
    [{\begin{tabular}[t]{@{}c@{}}
     agens\\
    \upshape [\cont \type{agens-rel}]\end{tabular}}
    [agens-exp, edge to=!un, name=agens-exp %before drawing tree={x/.option=!r.x}
    ]]
    [{\begin{tabular}[t]{@{}c@{}}
     experiencer\\
    \upshape [\cont \type{exp-rel}]\end{tabular}}
      ]]]
\end{forest}}
\caption{\label{abb-hierarchie-lexikon}Auszug aus einer möglichen Typhierarchie}
\end{figure}
Bei den Merkmalsspezifikationen in Abbildung~\ref{abb-hierarchie-lexikon}
sind entsprechende Pfade dazuzudenken: So steht \zb [\comps \sliste{}] für [\textsc{cat$|$comps} \sliste{}].
Wie in Typhierarchien üblich, gelten Beschränkungen für Typen auch für deren Untertypen (Vererbung\is{Vererbung|uu}).
Instanzen\is{Instanz} -- also Beschreibungen für wirkliche Objekte -- sind mit Strichlinie mit einem bestimmten Typ verbunden.
(\mex{1}) und (\mex{2}) zeigen die Information, die dann bei der Spezifikation zweier Instanzen angegeben
werden muss. Das ist genau die Zuordnung zu einem Typ, der entsprechende Beschränkungen von seinen
Obertypen erbt, und die Spezifikation einiger weniger idiosynkratischer Werte für Phonologie
und semantischen Beitrag.
\ea
\ms[count-noun-root]{
 phon & \phonliste{ Frau } \\
 cont & \ms{ ind$|$gen & fem \\
             restr  & \sliste{ \type{frau} } \\
           }\\
}
\z
\ea
\onems[nom-dat-verb-root]{
 phon \phonliste{ helf } \\
 cont \type{helfen} \\
}
\z
\is{Typhierarchie|)}



\section{Horizontale Generalisierungen: Lexikonregeln}
\label{sec-lr}


In Typhierarchien werden linguistische Objekte (Lexikoneinträge, Schemata)
kreuzklassifiziert. Man kann so Generalisierungen über Klassen von linguistischen Objekten ausdrücken.
Man kann \zb sagen, was die Wörter in (\mex{1}) gemeinsam haben.
\eal
\ex \emph{Frau} und \emph{Mann}
\ex \emph{Frau} und \emph{Salz}
\ex \emph{Frau} und \emph{Plan}
\zl
So sind die Nomina in (\mex{0}a) beides Zählnomina und Konkreta. Die Nomina in (\mex{0}b) sind verschieden,
weil eines ein Zählnomen ist und das andere ein Stoffnomen, aber beide sind Konkreta, und in (\mex{0}c)
liegen wieder zwei Zählnomina vor, von denen das eine ein Konkretum und das andere ein Abstraktum ist.

Aber es gibt auch andere Regularitäten, die sich nicht gut in Hierarchien erfassen lassen:
\eal
\ex \emph{treten} und \emph{getreten} wie in \emph{wurde getreten}
\ex \emph{lieben} und \emph{geliebt}  wie in \emph{wurde geliebt}
\zl
Die Wörter könnten ebenfalls in der Hierarchie repräsentiert werden (als Untertypen von intransitiv und transitiv),
aber dann wäre nicht erfaßt, dass die Valenzänderung in (\mex{0}a) und (\mex{0}b) durch denselben Prozeß
ausgelöst wird (siehe Abschnitt~\ref{sec-vererbung-koenig} zur Diskussion von vererbungsbasierten Ansätzen).

Man verwendet statt dessen Lexikonregeln, die zwei Wörter, zwei Stämme oder einen Stamm und ein
Wort zueinander in Beziehung setzen. 
Zur Verdeutlichung des Konzepts soll im folgenden eine Lexikonregel erklärt werden,
die die Beschreibung eines Stamms zur Beschreibung einer Passivform in Beziehung setzt.

Es gibt verschiedene Auf"|fassungen darüber, welchen formalen Status Lexikonregeln haben.
Man unterscheidet \emph{Meta Level Lexical Rules} (MLR) und \emph{Description Level Lexical Rules} (DLR)).
Eine detaillierte Diskussion der unterschiedlichen Ansätze findet man bei \citew{Meurers2000b}.
Bei MLR"=Ansätzen geht man davon aus, dass Lexikonregeln nicht Bestandteil des Formalismus sind,
sondern auf einer Metaebene Aussagen über beschriebene lexikalische Objekte machen. Beim
DLR"=Ansatz dagegen sind Lexikonregeln in den Formalismus integriert und haben denselben
Status wie Lexikoneinträge und Dominanzschemata.




Die Lexikonregel in (\mex{2}) lizenziert einen Eintrag für das Passivpartizip,
das man zur Analyse von (\mex{1}b) braucht.\footnote{
  Diese Lexikonregel wird hier nur benutzt, um die Funktionsweise von Lexikonregeln zu erklären.
  Dem Passiv und verwandten Konstruktionen ist das Kapitel~\ref{chap-passiv} gewidmet.%
}
\eal
\ex Der Mann schlägt den Hund.
\ex Der Hund wird geschlagen.
\zl
\ea
\label{pass-lr-mlr}
Lexikonregel für persönliches Passiv nach \citet{Kiss92}:\\
\onems[stem]{
  cat~ \ms{ head & verb  \\ 
            comps & \liste{ NP[\type{nom}], NP[\type{acc}]$_{\ibox{1}}$ } $\oplus$ \ibox{2} \\
          } \\
} $\mapsto$ \\
\onems[word]{
  cat \ms{ head & \ms{ vform & passiv-part } \\
           comps & \liste{ NP[\type{nom}]$_{\ibox{1}}$ } $\oplus$ \ibox{2} \\
         } \\
}
\z
Die Lexikonregel nimmt als Eingabe einen Verbstamm\footnote{
  Der Begriff \emph{Stamm} schließt Wurzeln (\stem{helf}),
  Derivationsprodukte (\stem{besing}) und Zusammensetzungen ein. 
  Die Lexikonregel kann also sowohl auf \stem{helf} als auch auf \stem{besing}
  angewendet werden.%
}, der ein Nominativargument, ein Akkusativargument und eventuell noch weitere Argumente (falls \iboxt{2}
nicht die leere Liste ist) verlangt, und lizenziert einen Lexikoneintrag, der ein Nominativargument 
und eventuell in \ibox{2} enthaltene Argumente verlangt. Die Ausgabe der Lexikonregel spezifiziert
den \vformw des Ausgabewortes. Das ist wichtig, da das Hilfsverb zur Verbform passen muss. Es darf \zb
nicht das Partizip Perfekt statt des Partizips Passiv verwendet werden:
\eal
\ex[]{
Der Mann hat den Hund geschlagen.
}
\ex[*]{
Der Mann wird den Hund geschlagen.
}
\zl

\noindent
Für die Bedeutung von Lexikonregeln gibt es einige Konventionen: Alle Information, 
die im Ausgabezeichen nicht erwähnt wird, wird vom Eingabezeichen per Konvention übernommen.
So wird \zb die Verbbedeutung in der Passivregel nicht erwähnt, was auch sinnvoll ist,
da Passivierung eine bedeutungserhaltende Umformung ist. Die \textsc{cont}"=Werte von Ein- und Ausgabe sind identisch.
Wichtig ist dabei, dass Linking"=Information erhalten bleibt. Das kann man sich anhand der Anwendung
der Passivregel auf den Verbstamm \stem{schlag} klarmachen:
\eal
\label{lr-passiv-beispiel}
\ex Eingabe:\\
\ms{
cat & \onems{ comps      \liste{ NP[\type{nom}]\ind{1}, NP[\type{acc}]\ind{2} } \\
       }\\
cont & \ms[schlagen]{
      agens       & \ibox{1}\\
      patiens & \ibox{2}\\
      }\\ 
}
%\flushright sieht doof aus
%\mbox{}\hspace{1em}
\ex Ausgabe:\\
\ms{
cat  & \onems{ comps      \liste{ NP[\type{nom}]\ind{2} } \\
       }\\
cont & \ms[schlagen]{
      agens       & \ibox{1}\\
      patiens & \ibox{2}\\
      }\\ 
}
\zl
Die Agens"=Rolle ist mit dem Subjekt von \stem{schlag} verbunden. Bei der Passivierung wird
das Subjekt unterdrückt, und das Argument, das mit dem Patiens von \stem{schlag} verbunden
ist, wird zum Subjekt des Partizips. Das Argument"=Linking ist davon aber nicht betroffen,
weshalb das Nominativ"=Argument im Passiv korrekterweise die Patiens"=Rolle füllt.

Folgt man dem DLR"=Ansatz, kann man Lexikonregeln wie (\ref{pass-lr-mlr}) mit Merkmalbeschreibungen
beschreiben:\label{pageref-lr-mit-dtr}
\ea
\label{passiv-lr-mit-dtr}
\onems[acc-passive-lexical-rule]{
     cat \ms{ head & \ms{ vform & passiv-part \\
                        } \\
              comps & \liste{ NP[\type{nom}]$_{\ibox{1}}$ } $\oplus$ \ibox{2} \\
            } \\
lex-dtr \onems[stem]{
        cat~ \ms{ head & verb \\ 
                  comps & \liste{ NP[\type{nom}], NP[\type{acc}]$_{\ibox{1}}$ } $\oplus$ \ibox{2} \\
                } \\
     }\\
}
\z
Was in (\ref{pass-lr-mlr}) auf der linken Regelseite steht, ist in (\mex{0}) als
Wert von \textsc{lex-dtr}\isfeat{lex-dtr} enthalten. Diese Lexikonregel ähnelt einer unären Grammatikregel,
ist jedoch auf das Lexikon beschränkt. Da die DLRs vollständig in den Formalismus integriert
sind, haben die den Lexikonregeln entsprechenden Merkmalstrukturen auch einen Typ. Ist das Ergebnis der
Regelanwendung ein flektiertes Wort, so ist der Typ der Lexikonregel (\type{acc-passive-lexical-rule} in unserem Beispiel)
ein Untertyp von \type{word}. Da Lexikonregeln einen Typ haben, ist es auch möglich, Generalisierungen über Lexikonregeln
zu erfassen.



Bisher wurde noch nicht gesagt, wie die morphologische Veränderung modelliert wird.
Die entsprechende Veränderung kann man einfach in die Regel in (\ref{passiv-lr-mit-dtr}) integrieren:
\ea
\label{passive-lr-mit-phon}
\onems[acc-passive-lexical-rule]{
phon $f\iboxb{1}$\\
cat \ms{ head & \ms{ vform & passiv-part \\
                   } \\
         comps & \liste{ NP[\type{nom}]$_{\ibox{2}}$ } $\oplus$ \ibox{3} \\
        } \\
lex-dtr \onems[stem]{
           phon  \ibox{1}\\
           cat  \ms{ head & verb \\ 
                       comps & \liste{ NP[\type{nom}], NP[\type{acc}]$_{\ibox{2}}$ } $\oplus$ \ibox{3} \\
                     } \\
     }\\
}
\z
In (\mex{0}) ist $f$ eine Funktion, die aus dem \textsc{phon}"=Wert der \textsc{lex-dtr} die Partizipform berechnet,
also zu \stem{red} \emph{geredet} und zu \stem{schlag} \emph{geschlagen} bildet.
Diese Funktion mag dem Leser obskur erscheinen, wir werden aber im Kapitel~\ref{sec-morph-flex-anal}
sehen, was sich dahinter verbirgt.

\section{Flexion und Derivation}
\label{sec-lexikon-felx-deriv}

\is{Morphologie!Derivation|(}%
\is{Morphologie!Flexion|(}%
Im vorangegangenen Abschnitt haben wir gesehen, dass Lexikonregeln benutzt werden können, um einen
Lexikoneintrag zu einem anders flektierten Eintrag mit eventuell anderer Valenz in Beziehung zu setzen.
Natürlich kann man Lexikonregeln auch zur Modellierung der derivationellen Morphologie benutzen.
Für die Analyse des Nomens \emph{Lesbarkeit} braucht man eine Lexikonregel, die den Verbstamm
\stem{les} auf den Adjektivstamm \stem{lesbar} abbildet. Dieser wiederum wird dann auf den nominalen
Stamm \stem{Lesbarkeit} abgebildet.

Für das Verständnis interessanter morphologischer Regeln müssen Konzepte wie struktureller Kasus
und die Repräsentation von Subjekten bei infiniten Verben und Adjektiven noch eingeführt werden.
Die Analyse von Kongruenzphänomenen steht ebenfalls noch aus. Nach der Behandlung dieser Themen
können wir uns dann im Kapitel~\ref{chap-morphologie} ausführlicher mit der Morphologie beschäftigen.
Dort wird auch eine alternative Analyse morphologischer Strukturen besprochen, die den Analysen
entspricht, die wir bisher für syntaktische Strukturen kennengelernt haben.%
\is{Morphologie!Derivation|)}%
\is{Morphologie!Flexion|)}%

\section{Zwischenzusammenfassung der bisher eingeführten Beschreibungsmittel}

An dieser Stelle bietet es sich an, eine kurze Zwischenzusammenfassung des bisher eingeführten
Inventars zu geben, da jetzt nichts qualitativ Neues mehr hinzukommen wird. In den kommenden
Kapiteln wird die Merkmalsgeometrie zwar noch verändert, und wir werden komplexere Analysen
linguistisch interessanter Phänomene kennenlernen, die verwendeten Mechanismen sind aber
bereits eingeführt. 

Eine linguistische Theorie im Rahmen der HPSG besteht aus folgenden Bestandteilen:
\begin{enumerate}
\item einer Anzahl von Lexikoneinträgen, deren Eigenschaften durch getypte Merkmalbeschreibungen beschränkt werden,
\item einer Anzahl von Lexikonregeln, die ebenfalls mittels Merkmal"=Wert"=Paaren beschrieben werden und
      die weitere Lexikoneinheiten lizenzieren,
\item einer Anzahl von Dominanzschemata, die die Kombination von Lexikoneinheiten zu größeren Einheiten
      lizenzieren,
\item Prinzipien, \dash Beschränkungen, die für linguistische Objekte erfüllt sein müssen,
\item Linearisierungsbeschränkungen, die die Abfolge der Töchter in den Dominanzschemata regeln und
\item einer Anzahl relationaler Beschränkungen, die in den Dominanzschemata, den Lexikonregeln bzw.\
      den Merkmalbeschreibungen verwendet werden.
\end{enumerate}
Die durch die Dominanzschemata lizenzierten Einheiten können je nach Schema
selbst wieder mit anderen linguistischen Objekten kombiniert
werden, so dass rekursiv\is{Rekursion} beliebig komplexe Strukturen aufgebaut werden können.

Da die Lexikoneinträge, die Lexikonregeln und die Dominanzschemata mit Hilfe getypter Merkmalstrukturen
modelliert werden, können Generalisierungen über alle drei Arten von Objekten in der Typhierarchie
erfaßt werden. Wird ein Prinzip formalisiert, so geschieht das meistens mit Bezug auf einen Typ. Das Prinzip
gilt dann für alle Untertypen des betreffenden Typs.


\section{Alternativen}

\mbox{}\is{Vererbung|(}%
In diesem Abschnitt wird untersucht, wofür sich die in diesem Kapitel vorgestellten Techniken und Beschreibungsmittel
einsetzen lassen. In bestimmten Varianten der Konstruktionsgrammatik werden Typhierarchien verwendet, um Generalisierung in bezug
auf phrasale Muster oder morphologische Prozesse auszudrücken. Das ist mitunter nicht adäquat, und Lösungen,
die Lexikonregeln bzw.\ äquivalente Beschreibungsmittel verwenden, sind vorzuziehen.

\subsection{Konstruktionsgrammatik}

Die Konstruktionsgrammatik\is{Konstruktionsgrammatik (CxG)|(}\is{Typhierarchie|(} teilt viele Ansichten mit der HPSG. Wichtig ist
den Vertretern beider Theorien, dass immer Form"=Bedeutungspaare betrachtet werden. Die beschriebenen
linguistischen Objekte werden in beiden Theorien in Hierarchien klassifiziert. Die formalen Grundlagen
der Konstruktionsgrammatik sind noch nicht ausgearbeitet,\footnote{%
  Die Vorschläge in \citew{KF99a}, \citew{Kay2005a} und \citew{Kay2002a} sind teilweise
  in sich inkonsistent bzw.\ miteinander inkompatibel.
\citet{Kay2002a} nimmt ungetypte Merkmalstrukturen an,
  verwendet aber in einem 2000 fertiggestellten Aufsatz \citep{Kay2005a} getypte Merkmalstrukturen. Relationale Beschränkungen
  wie \emph{append} kommen ebenfalls in \citew{Kay2002a} nicht vor, werden aber in \citew{Kay2005a}
  verwendet. Eine Formalisierung wird wohl darauf hinauslaufen, dass
  dieselben formalen Mittel verwendet werden wie in HPSG: getypte Merkmalstrukturen und
  relationale Beschränkungen. 
%Zu einer Bemerkung in diese Richtung siehe \citew{Michaelis2006a}.
% Die Relationen werden da aber nicht erwähnt.
  Zur Diskussion der formalen Grundlagen von Konstruktionsgrammatik siehe auch \citew[\page
    856--859 und Abschnitt~3]{Mueller2006d}.%
} aber oft ist klar, dass 
Konstruktionsgrammatikanalysen direkt mit Merkmalstrukturen und Vererbungshierarchien
modelliert werden können.

\subsubsection{Phrasale vs.\ Argumentstruktur-Konstruktionen}
\label{sec-phrasal-lexikalisch-rc}

Obwohl sich beide Theorien ähnlich sind, wird die Frage, welche Phänomene auf phrasaler Ebene
beschrieben werden sollten und welche im Lexikon, in HPSG und in CxG sehr unterschiedlich
beantwortet. In der Konstruktionsgrammatik tendiert man zu phrasenbasierten Analysen, wohingegen in
der HPSG oft lexikonbasierte Analysen vorgezogen werden. Problematisch an phrasenbasierten Ansätzen ist, 
dass man, sobald man den Bereich des Lexikons verlassen hat, nicht mehr
dahin zurückkehren kann. Wenn man also -- wie es \citet{Goldberg95a} tut -- die Bedeutung
der Resultativkonstruktion in (\mex{1}a) an einer bestimmten phrasalen Konfiguration festmacht,
dann muss man auch die Bedeutung von (\mex{1}b) an einer phrasalen Konfiguration festmachen.
\eal
\label{bsp-res-passiv}
\ex Er hat den Teich leer gefischt.
\ex Der Teich wurde leer gefischt.
\zl
Man kann dann nicht, so wie das im Abschnitt~\ref{sec-lr} gemacht wurde, einen Verbstamm
mittels Lexikonregel zu einem Partizip in Beziehung setzen, um (\mex{0}b) zu analysieren,
da der Verbstamm ja die resultative Bedeutung nicht enthält. Diese wird in Goldbergs Analyse
von der Konstruktion auf phrasaler Ebene beigesteuert. Es muss also auch eine phrasale
Passiv"=Resultativ"=Konstruktion geben. Um diese beiden phrasalen Konstruktionen in Beziehung
zueinander zu setzen, braucht man dann Transformationen\is{Transformation}, die über komplexen phrasalen
Objekten operieren. Solcherart komplexe Transformationen wurden in den 70er und 80er Jahren
in den auf Chomskys Arbeiten basierenden Grammatikmodellen\is{Government and Binding (GB)@\emph{Government and Binding} (GB)}
eliminiert. Die genauen Gründe hierfür sind gut in \citew[Kapitel~3.1]{Klenk2003a} zusammengefaßt.

In HPSG"=Analysen nimmt man dagegen oft Lexikonregeln für Resultativkonstruktionen an
(\citealp{Wechsler97a,WN2001a,Verspoor97a}; \citealp[Kapitel~6]{Mueller2002b}).\footnote{
  Siehe auch \citew[\page 45]{Wunderlich92a-u-kopiert} und
  \citew[\page 120--126]{Wunderlich97c}
zu einem äquivalenten Vorschlag in einem
  anderen theoretischen Rahmen.%
}
Das Verb \stem{fisch} lizenziert einen Lexikoneintrag, der zusätzlich zum Subjekt von \stem{fisch}
ein Adjektiv und dessen Subjekt selegiert. Dieser Lexikoneintrag kann dann wieder Eingabe
für die Passivierungsregel sein. Die Ausgabe der Passivierungsregel kann zur Analyse von (\mex{0}b)
benutzt werden. 

\citet{Kay2005a} schlägt Argumentstruktur"=Konstruktionen vor,
die zu den HPSG"=Lexikonregeln äquivalent sind. Somit ließen sich die HPSG"=Ansätze auch direkt
in Konstruktionsgrammatik übersetzen.
Zur Diskussion phrasaler Ansätze siehe \citew{Mueller2006d}.
%
%% \citet[\page26]{Croft2001a}:
%% \begin{figure}[htbp]
%% \nodemargin0pt
%% \begin{tabular}{c@{\hspace{4ex}}c}
%% \rnode{subjintrverb}{\fbox{Subj IntrVerb}}               & \rnode{SubjAux}{\fbox{Sbj Aux-n't Verb}}\\[5ex]
%% \multicolumn{2}{@{}c@{}}{\hspace{0ex}\rnode{Ididntsleep}{\fbox{I didn't sleep}}}\\
%% \end{tabular}
%% \ncline{subjintrverb}{Ididntsleep}\ncline{SubjAux}{Ididntsleep}%
%% \end{figure}
%% \citet{Croft2001a} in seiner \emph{Radical Construction Grammar FAQ}:
%% \begin{quote}
%% Although categorial grammar is another way of modeling Radical Construction Grammar,
%% it has the drawback that all combinations must be represented as binary,
%% whereas many constructions have more than two elements. \citep[\page49]{Croft2001a}
%% \end{quote}

\subsection{Linking"=Konstruktionen}
\label{cxg-hpsg-linking-konstruktionen}
\is{Linking!-Konstruktionen|(}

Zu Beginn des Abschnitts~\ref{sec-lr} wurde behauptet, dass man Alternationen wie die
Aktiv/""Passiv"=Alternationen nicht adäquat in Typhierarchien erfassen kann. 
Genau das wurde aber von Kay und Fillmore im Rahmen der Konstruktionsgrammatik und 
von \citet{Koenig99a}, \citet{DK2000b-u} und Davis und Koenig folgend
auch von \citet{Kordoni2001b-u} im Rahmen der HPSG vorgeschlagen. Im
folgenden sollen die jeweiligen Vorschläge diskutiert werden.\footnote{
  Entsprechende Vorschläge wurden auch in TAG\indextag (\citealp{Candito96a}; \citealp[\page 188]{CK2003a-u}) und Simpler Syntax\is{Simpler Syntax} \citep[Kapitel~6.3]{CJ2005a} gemacht. Diese können hier nicht diskutiert werden. Der interessierte Leser sei jedoch auf \citew{MuellerUnifying} und \citew{MWArgSt} verwiesen.
}

\subsubsection{Konstruktionsgrammatik}
\label{cxg-linking-konstruktionen}

\mbox{}\citet[\page12]{KF99a}  erwähnen Linking"=Konstruktionen nur am Rande, ihr Ansatz wird aber
von \citet[Kapitel~4]{MR2001a} genauer erklärt. In \citew{KF99a} wird Valenzinformation
in Mengen\is{Menge|(} repräsentiert.\footnote{
  Siehe jedoch \citew[Abschnitt~4.1]{Kay2005a} und \citew{Michaelis2006a} zu einem listenbasierten Ansatz und
  Linking"=Konstruktionen.
}
Die Auffassungen in bezug auf Mengenunifikation unterscheiden sich sehr
stark von denen, die in der HPSG gemacht werden. Kay und Fillmore nehmen an, dass
die Unifikation der Menge \{ a \} mit der Menge \{ b \}, wobei a und b nicht unifizierbar sind,
die Vereinigung der beiden Mengen, also \{ a, b \} ist.
Durch ihr spezielles Verständnis von Mengen ist es möglich, die Anzahl von Elementen in Mengen
zu erweitern. Die Unifikation zweier Mengen, die kompatible Elemente enthalten, ist eine disjunktive
Verknüpfung von Mengen, die die jeweiligen Unifikationen der Elemente enthalten. Das hört sich kompliziert
an, hier ist jedoch nur ein bestimmter Fall von Interesse: die Unifikation einer beliebigen
Valenzmenge mit einer einelementigen Menge:
\ea
\{ NP[\nom], NP[\acc] \} $\wedge$ \{ NP[\nom] \} = \{ NP[\nom], NP[\acc] \}
\z
Nach dieser Auffassung führt die Unifikation einer Menge mit einer Menge, die ein kompatibles
Element enthält, nicht zur Erhöhung der Anzahl der Elemente. Auch der folgende Fall ist denkbar:
\ea
\{ NP, NP[\acc] \} $\wedge$ \{ NP[\nom] \} = \{ NP[\nom], NP[\acc] \}
\z
In (\mex{0}) ist NP in der ersten Menge in bezug auf Kasus unterspezifiziert. In der zweiten Menge
ist der Kasus als Nominativ spezifiziert. NP[\nom] ist nicht mit  NP[\acc] unifizierbar, aber mit NP.

Bevor ich jetzt die Linking"=Konstruktionen und die sich aus dieser Analyse ergebenden Probleme
diskutiere, muss ich noch auf eine Konsequenz der gerade erklärten Mengenunifikation hinweisen: Unifikation
ist normalerweise wie folgt definiert:
\ea
Die Unifikation\is{Unifikation} zweier Strukturen FS$_1$ und FS$_2$ ist diejenige Struktur FS$_3$, die sowohl von
FS$_1$ als auch von FS$_2$ subsumiert\is{Subsumtion} wird und selbst nicht von einer anderen Struktur subsumiert wird,
für die das auch gilt.
\z
Dabei subsumiert eine Struktur FS$_1$ die Struktur FS$_3$, wenn FS$_3$ alle Merkmal"=Wert"=Paare und
alle Strukturteilungen aus FS$_1$ enthält. FS$_3$ kann darüber hinaus noch weitere Merkmal"=Wert"=Paare
oder Strukturteilungen enthalten. Die Konsequenz ist, dass wenn die Unifikation von Valenzmengen
wie in (\mex{1}a) erfolgen kann, die Subsumtionsverhältnisse in (\mex{1}b,c) gelten müssen:
\ea
Eigenschaften der Mengenunifikation nach \citew{KF99a}:\\
\begin{tabular}{@{}l@{}}
a. \{ NP[\nom] \} $\wedge$ \{ NP[\acc] \} = \{ NP[\nom], NP[\acc] \}\\
b. \{ NP[\nom] \} $\succeq$ \{ NP[\nom], NP[\acc] \}\\
c. \{ NP[\acc] \} $\succeq$ \{ NP[\nom], NP[\acc] \}\\
\end{tabular}
\z
Das heißt, eine Merkmalstruktur mit einer Valenzliste, die nur eine NP[\nom] enthält, ist allgemeiner
als eine Merkmalstruktur mit einer Valenzliste, die eine NP[\nom] und eine NP[\acc] enthält (\mex{0}b).
Somit ist die Menge der transitiven Verben eine Unterklasse der intransitiven Verben. Das ist unintuitiv,
aber mit dem System zur Lizenzierung von Argumenten, das Kay und Fillmore verwenden, kompatibel. Allerdings
ergeben sich andere Probleme, wie sich gleich zeigen wird.

\is{Passiv|(}%
\citet[\page55--57]{MR2001a} geben die folgenden Linking"=Konstruktionen\is{Linking} an:\footnote{
  In der \emph{Transitive Construction} ist im Original als $\theta$"=Wert \textsc{da}$-$
  angegeben, \textsc{da} ist aber ein Merkmal. Ich habe das in (\mex{1}a) entsprechend korrigiert.

  In den folgenden Strukturen steht \textsc{gf} für \emph{grammatical function} und \textsc{da} für \emph{distinguished argument}.%
}
\eal
\label{linking-konstruktionen}
\ex\label{transitiv-Konstruktion} die \emph{Transitive Construction}:\\
\ms{
syn & \ms{ cat & v\\
           voice & active\\
       }\\
val & \menge{ \onems{ role \ms{ gf & obj \\
                                %$\theta$ & \textsc{da}$-$\\
                                \textsc{da} & $-$\\
                              }\\
                    }}\\
}
\ex die \emph{Subject Construction}:\\
\ms{
syn & \ms{ cat & v\\
       }\\
val & \menge{ \onems{ role \onems{ gf \type{subj} } }}\\
}
\ex die \emph{Passive Construction}:\\
\ms{
syn & \ms{ cat  & v\\
           form & PastPart\\
       }\\
val & \menge{ \ms{ role & \ms{ gf & obl\\
                               da & +\\
                             }\\
                   syn  & \textrm{P[von]/}zero\\
                 }}\\
}
\zl
%% Diese Konstruktionen sind sehr nachlässig aufgeschrieben worden: Mal gibt es ein $\theta$"=Merkmal,
%% mal nicht, mal steht \textsc{gf} unter \textsc{role}, mal nicht. Man kann sich aber überlegen, wofür die
%% Strukturen stehen sollen: 
Die Struktur in (\mex{0}a) sagt, dass es in einer Valenzmenge eines von
der Transitiv"=Konstruktion beschriebenen linguistischen Objekts ein Element geben muss, das die grammatische Funktion
\emph{Objekt} hat und dessen \textsc{da}"=Wert $-$ ist. Die Subjekt"=Konstruktion besagt, dass ein
Element der Valenzmenge die grammatische Funktion \emph{Subjekt} haben muss. Die Passiv"=Konstruktion
besagt, dass es ein Element geben muss, das die grammatische Funktion \emph{Oblique} und einen
\textsc{da}"=Wert + haben muss. Dieses Element wird entweder als \emph{von}-PP oder gar nicht (\emph{zero})
realisiert.

Das Zusammenspiel der Konstruktionen in (\mex{0}) soll am Verb \emph{schlagen} erklärt werden:
\ea
Lexikoneintrag für \stem{schlag}:\\
\ms{
syn & \ms{ cat & v\\
       }\\
val & \menge{ \onems{ role \ms{ $\theta$ & agent\\
                                da       & +\\
                              }\\
                    }, 
              \onems{ role \ms{ $\theta$ & patient\\
                              }\\
                    }}\\
}
\z
Kombiniert man diesen Lexikoneintrag mit der Transitiv- und der Subjekt"=Konstruktion,
erhält man (\mex{1}a), wohingegen eine Kombination mit Subjekt- und Passiv"=Konstruktion
(\mex{1}b) ergibt:
\eal
\label{ex-schlagen-linking}
\ex 
\begin{tabular}[t]{@{}l@{}}
\stem{schlag} + Subjekt- und Transitiv"=Konstruktion:\\
\ms{
syn & \ms{ cat & v\\
           voice & active\\
       }\\
val & \menge{ \onems{ role \ms{ $\theta$ & agent\\
                                gf       & subj\\
                                da       & +\\
                              }\\
                    }, 
              \onems{ role \ms{ $\theta$ & patient\\
                                gf       & obj\\
                                da       & $-$\\
                              }\\
                    }}\\
}
\end{tabular}
\ex \stem{schlag} + Subjekt- und Passiv"=Konstruktion:\\
\ms{
syn & \ms{ cat & v\\
           form & PastPart\\
       }\\
val & \menge{ \ms{ role & \ms{ $\theta$ & agent\\
                                gf       & obl\\
                                da       & +\\
                              }\\
                   syn  & \textrm{P[von]/}zero\\
                  }, 
              \onems{ role \ms{ $\theta$ & patient\\
                                gf       & subj\\
                              }\\
                    }}\\
}
\zl
Mit den Einträgen in (\mex{0}) kann man dann die Sätze in (\mex{1}) analysieren:
\eal
\ex Er schlägt den Weltmeister.
\ex Der Weltmeister wird (von ihm) geschlagen.
\zl
Die Frage, wie man von den Lexikoneinträgen für Verben zu den Repräsentationen in (\mex{-1})
kommt, ist noch nicht beantwortet. \citet{Kay2002a} schlägt für (phrasale) Konstruktionen
die automatische Berechnung aller kompatiblen Kombinationen von maximal spezifischen
Konstruktionen vor. Ein solches Verfahren würde die Beschreibungen in (\mex{-1}) liefern,
und man könnte die wohlgeformten Sätze in (\mex{0}) analysieren.
Probleme ergeben sich aber bei ungrammatischen Sätzen wie (\mex{1}). \emph{grauen} ist ein
subjektloses Verb\is{Verb!subjektloses}. Würde man einfach alle kompatiblen Linking"=Konstruktionen mit \emph{grauen}
verbinden, so würde die Auffassung von Mengenunifikation, die Kay und Fillmore vertreten, dazu führen,
dass ein Subjekt in die Valenzmenge von \emph{grauen} eingeführt wird. Damit wäre dann (\mex{1})
analysierbar.
\ea[*]{
Ich graue dem Studenten vor der Prüfung.
}
\z
Man könnte dieses Problem dadurch beheben, dass man bereits im Lexikoneintrag für \emph{grauen}
ein Element mit der grammatischen Funktion \emph{Subjekt} spezifiziert und zusätzlich festlegt,
dass es nie realisiert werden kann (\textsc{syn} \type{zero}) und nichts bedeutet. Man hätte somit
ein phonologisch nicht realisiertes Expletivpronomen, das nur in Valenzlisten sein leeres Dasein
fristet. Eine Analyse, die ohne solche Hilfskonstruktionen auskommt, ist vorzuziehen.
%Solcherart leere Elemente werden in der
%Konstruktionsgrammatik jedoch strikt abgelehnt \citep[\page49--50]{MR2001a}.

\citet{KF99a} repräsentieren den Bedeutungsbeitrag sprachlicher Zeichen genauso wie deren Valenz in Mengen.
Damit ist es ausgeschlossen, unerwünschte Unifikationen von Linking"=Konstruktionen über Bezugnahme
auf semantische Eigenschaften zu verhindern, denn es tritt genau derselbe Effekt wie bei den
Valenzmengen ein: Wenn die semantischen Beschreibungen inkompatibel sind, wird die Menge erweitert. Das bedeutet, dass bei
automatischer Unifikation alle Verben mit der Transitiv"=Konstruktion in (\ref{transitiv-Konstruktion}) kompatibel sind,
was dann neben (\mex{0}) auch noch Sätze wie (\mex{1}) lizenziert:
\eal
\ex[*]{
Der Mann schläft das Buch.
}
\ex[*]{
Der Mann denkt an die Frau das Buch.
}
\zl
In (\mex{0}a) wurde ein intransitives Verb mit der Transitiv"=Konstruktion verbunden und in
(\mex{0}b) ein Verb, das ein Präpositionalobjekt verlangt, \dash, man kann die Repräsentationen in
(\ref{ex-schlagen-linking}) nicht automatisch berechnen lassen. In Kays und Fillmores System müßte
man also für jedes Verb einzeln Unterkonstruktionen für das Verb im Aktiv, im Passiv, in der
Medialkonstruktion usw.\ festlegen, wodurch man nicht erfaßt, dass neu entstehende transitive Verben
von Sprechern nach dem Erwerb des neuen Verbs auch ohne weiteres passiviert werden können.\is{Menge|)}%
\is{Passiv|)}%

\citet{MR2001a} benutzen keine Mengen für die Repräsentation von semantischer Information.
Daher könnten sie in der Transitiv"=Konstruktion Beschränkungen für die Bedeutung der Verben
formulieren, die mit dieser Konstruktion kompatibel sein sollen.\footnote{
  Hierzu ist eine Repräsentation der Relationen als getypte Merkmalstruktur notwendig, so
  wie sie in Kapitel~\ref{sec-rep-semantik-merkmalstruktur} eingeführt wurde. Mit einer solchen
  Repräsentation ist es möglich, abstrakt über zweistellige Relationen zu sprechen. 
  Siehe \zb auch die Diskussion von (\ref{bsp-linking-agens-experiencer}) auf
  Seite~\pageref{bsp-linking-agens-experiencer}.%
}
Die Unifikation mit der Subjekt"=Konstruktion kann aber nicht aus semantischen Gründen ausgeschlossen
werden, da es durchaus Verben mit Subjekt gibt, die semantisch nichts mit dem Subjekt zu tun haben:
Das ist bei den sogenannten Anhebungsverben\is{Verb!Anhebungs-} der Fall (zu Anhebungsverben
siehe Kapitel~\ref{sec-anhebung}). Wie man an der Subjekt"=Verb"=Kongruenz in (\mex{1})
sehen kann, ist \emph{du} das Subjekt von \emph{scheinst}. Der Referent von \emph{du} ist aber nicht
"`der Scheinende"'.
\ea
Du scheinst gleich einzuschlafen.
\z
Man ist also gezwungen, entweder für subjektlose Verben wie \emph{grauen} ein leeres expletives
Subjekt anzunehmen oder explizit zu spezifizieren, welche Verben von der Subjekt"=Konstruktion
erben dürfen und welche nicht.


Parallel zu (\mex{0}) gibt es Objektanhebungskonstruktionen, in denen ein Akkusativobjekt
existiert, das auch durch Passivierung zum Subjekt des gesamten Satzes werden kann, das
aber keine semantische Rolle vom Verb bekommt:
\eal
\ex Richard lacht ihn an.
\ex Richard fischt den Teich leer.
\zl
In (\mex{0}) ist das Objekt ein semantisches Argument von \emph{an} bzw.\ \emph{leer}, nicht aber
semantisches Argument der Verben \emph{lacht} bzw.\ \emph{fischt}.
Partikelverben\is{Verb!Partikel-} wie \emph{anlachen} werden im Kapitel~\ref{chap-partikel} ausführlich diskutiert.
Zu Resultativkonstruktionen\is{Resultativkonstruktion} wie (\mex{0}b)  siehe \citew{Mueller2002b}.
Will man diese Aktivformen und die entsprechenden Passivierungen über die Linkingkonstruktionen
in (\ref{linking-konstruktionen}) erklären, kann man sich dabei nicht auf semantische Eigenschaften
des Matrixprädikats beziehen. Hier bleibt also nur eine explizite Aufzählung der mit
der Transitiv"=Konstruktion kompatiblen Lexikoneinträge. Man ist also gezwungen, für jedes Verb
Aussagen bzgl.\ seines Vorkommens in Aktiv"= und Passivsätzen zu machen.
Zu Resultativkonstruktionen und
Interaktionen mit anderen Konstruktionen siehe auch \citew{Mueller2006c,Mueller2006d}.
\is{Konstruktionsgrammatik (CxG)|)}


\subsubsection{HPSG}
\label{sec-vererbung-koenig}

Die Verwendung von Mengen nach Kay und Fillmore sollte man möglichst schnell wieder vergessen,
da sie mit nichts anderem in diesem Buch kompatibel ist. In HPSG ist die Unifikation von Mengen\is{Menge},
die inkompatible Elemente enthalten, nicht möglich. Es gibt eine schwierige Definition von Mengenunifikation
bei \citet[\page47--49]{ps} und \citet{PM90a}. Mengen werden dazu verwendet, sogenannte Schmarotzerlücken\is{Lücke!Schmarotzer-} (\emph{parasitic gaps})
zu analysieren. Auf Schmarotzerlücken werde ich hier nicht eingehen, da es dieses Phänomen im Deutschen
nicht gibt.\footnote{
        \citet*{Felix85} gibt Beispiele aus dem  Bairischen.\il{Bairisch} 
        Diese gehören aber garantiert nicht zum Standard"=Deutschen und
        werden auch nicht von allen Sprechern des Süddeutschen akzeptiert. 
        Zu einer Kritik an Felix' Ansatz siehe auch 
        \citet*[\page230]{Oppenrieder91a}.
}
Es reicht also aus, Listen zu verwenden.

Listen kann man bei Verwendung normaler Unifikation nicht verkürzen oder verlängern. Den Effekt
einer Verkürzung könnte man in der HPSG erreichen, indem man ein binäres Merkmal \textsc{realized}
einführt und dann bei einem entsprechenden Element den Wert auf $+$ setzt. 
Eine Liste mit einem auf diese Weise bereits als realisiert markierten Element entspricht
einer \compsl, aus der ein Element entfernt wurde. Für die Abbindung von Elementen
in der Valenzliste, die einen \textsc{realized}"=Wert $-$ haben, brauchte man ein entsprechend
angepaßtes Kopf"=Argument"=Schema, das Bezug auf den \textsc{realized}"=Wert von Argumenten nimmt
(siehe Kapitel~\ref{kasus-anhang} für eine Ausarbeitung dieses Vorschlags).
Ein größeres Problem stellen Fälle dar, für die man eine Erweiterung der Valenzliste
benötigt. Zum Beispiel benötigt man für die Analyse der Medialkonstruktion\is{Medialkonstruktion} in (\mex{1}b)
ein zusätzliches Reflexivpronomen in der Valenzliste:
\eal
\ex Er          liest den Aufsatz.
\ex Der Aufsatz liest sich leicht.
\zl
Das logische Subjekt\is{Subjekt!logisches}\footnote{
  Das logische Subjekt ist das Argument, das im Aktiv als Subjekt realisiert wird.%
}
von \emph{lesen} wird unterdrückt, und es erscheint ein Reflexivum als
formale Markierung im Satz.

\is{Passiv|(}%
Dieses Problem läßt sich ebenfalls lösen, wenn man mehrere Repräsentationsmöglichkeiten für
Argumente hat. Einen entsprechenden Vorschlag macht \citet{Koenig99a}: In neueren Arbeiten
zur HPSG wird ein listenwertiges Merkmal \argst\isfeat{arg-st} für die Repräsentation der Argumentstruktur
lexikalischer Elemente verwendet. Die \argstl enthält dieselbe Information, die wir
bisher in \comps repräsentiert haben. Wenn man also im Lexikoneintrag für ein Verb
wie \stem{schlag} die Valenzinformation in der \argstl repräsentiert, dann können entsprechende
Linking"=Konstruktionen\is{Linking} diese Information je nach Aktiv- oder Passivrealisierung auf
die \compsl übernehmen. Wie das im Detail aussieht, soll im folgenden erklärt werden.

Koenig diskutiert ein Beispiel, das eine Hierarchie verwendet, die der in Abbildung~\vref{bsp-partitionen}
ähnelt.
\begin{figure}
% \centerline{\begin{pspicture}(0.4,0.8)(9.2,5.4)
% %\psgrid
% \rput[B](5,5){\rnode{lexeme}{\type{lexeme}}}
% \rput[B](2,3){\rnode{valence}{\fbox{\type{valence}}}}
% %\rput[B](5,3){\rnode{inflection}{\fbox\type{inflection}}}
% \rput[B](8,3){\rnode{root}{\fbox{\type{root}}}}
% %
% \rput[B](1,1){\rnode{passive}{\type{passive}}}
% \rput[B](3,1){\rnode{active}{\type{active}}}
% %
% %\rput[B](4,1){\rnode{3sg}{\type{3sg}}}
% %\rput[B](6,1){\rnode{past}{\type{past}}}
% \rput[B](7,1){\rnode{lesen}{\type{lesen}}}
% \rput[B](9,1){\rnode{essen}{\type{essen}}}
% \psset{angleA=-90,angleB=90,arm=0pt}
% %
% \ncdiag{lexeme}{valence}\ncdiag{lexeme}{inflection}\ncdiag{lexeme}{root}
% \ncdiag{valence}{passive}\ncdiag{valence}{active}
% \ncdiag{inflection}{3sg}\ncdiag{inflection}{past}
% \ncdiag{root}{lesen}\ncdiag{root}{essen}
% %
% \end{pspicture}}
\begin{forest}
  type hierarchy,
  for level=3{
    instance, % Apply "instance" style to nodes on (absolute) level 3.
    l*=1.5, % increases the distance to the second level by a factor of 1.5
  },
  [lexeme
    [valence,partition,
      [passive]
      [active]]
    [root,partition,
      [lesen]
      [essen]
    ]
  ]
\end{forest}
\caption{\label{bsp-partitionen}Typhierarchie mit Partitionen}
\end{figure}
Die eingekästelten Elemente entsprechen sogenannten Partitionen\is{Partition}. Für beschriebene Objekte
muss jeweils ein von einer solchen Partition dominierter Typ ausgewählt werden, \dash
ein Lexem muss entweder \type{passive} %\type{middle} 
oder \type{active} und entweder \type{lesen} oder \type{essen} sein. 
Die jeweils möglichen Kombinationen
läßt Koenig automatisch berechnen. Ist die für die entsprechenden Typen definierte Information
kompatibel, wird ein entsprechender Subtyp in die Hierarchie eingefügt.
Für die Hierarchie in Abbildung~\ref{bsp-partitionen} ergibt sich die Hierarchie
mit den zusätzlichen Typen in Abbildung~\vref{bsp-partitionen-ausgerechnet}.
\begin{figure}
\begin{forest}
  type hierarchy,
  for level=2{
%    instance, % Apply "instance" style to nodes on (absolute) level 3.
    l*=1.5, % increases the distance to the second level by a factor of 1.5
  },
  [lexeme 
    [passive, name=passive
      [passive $\wedge$ lesen, edge to=lesen]]
    [active, name=active
      [active $\wedge$ lesen, edge to=lesen]]
    [lesen, name=lesen
      [passive $\wedge$ essen, edge to=essen, edge to=passive, no edge]]
    [essen, name=essen
      [active $\wedge$ essen, edge to=active]]]
\end{forest}
\caption{\label{bsp-partitionen-ausgerechnet}Typhierarchie mit automatisch hinzugefügten Typen}
\end{figure}
Der Vorteil einer solchen automatischen Typberechnung ist, dass man nicht
für jedes Verb einzeln sagen muss, dass es eine Aktiv- und eine Passivvariante
gibt. Fügt man \zb eine Wurzel für \emph{schlafen} in die Hierarchie ein, wird
\type{active} $\wedge$ \type{schlafen} und \type{passive} $\wedge$ \type{schlafen}
automatisch berechnet.

\noindent
Für \stem{les} könnte man den unterspezifizierten Lexikoneintrag in (\mex{1})
annehmen, der dem von \citet[\page59]{Koenig99a} für \emph{play} vorgeschlagenen ähnelt.
\ea
\ms[lesen]{
stem-phon & \phonliste{ les }\\
cat       & \ms{
            arg-st & \sliste{ NP\ind{1}, NP\ind{2} } \\
            }\\
cont      & \ms[lesen-rel]{
            agens & \ibox{1}\\
            thema & \ibox{2}\\
            }\\
}
\z
Die für unsere Merkmalsgeometrie angepaßten Typen \type{active} und \type{passive} zeigt (\mex{1}):
\eal
\ex 
\begin{tabular}[t]{@{}l@{}}
Typ für Aktiv-Linking:\\
\ms[active]{
cat       & \ms{
            comps & \ibox{1}\\
            arg-st & \ibox{1}\\
            }\\
}
\end{tabular}
\ex\label{constr-passive-linking}
\begin{tabular}[t]{@{}l@{}} 
Typ für Passiv-Linking:\\
\ms[passive]{
cat       & \ms{
            comps & \ibox{1}\\
            arg-st & \sliste{ NP } $\oplus$ \ibox{1}\\
            }\\
}
\end{tabular}
\zl
Der Aktiv"=Linking"=Typ übernimmt einfach alle Elemente der Argumentstruktur auf die \compsl.
Der Passiv"=Linking"=Typ übernimmt alle Elemente aus der \argstl außer dem Subjekt.
Das Ergebnis der Unifikationen von \type{lesen} mit den entsprechenden Typen zeigt (\mex{1}):
\eal
\ex 
\begin{tabular}[t]{@{}l@{}}
\stem{les} mit Aktiv-Linking:\\
\ms[lesen $\wedge$ active]{
stem-phon & \phonliste{ les }\\
cat       & \ms{
            comps & \ibox{1}\\
            arg-st & \ibox{1} \sliste{ NP\ind{1}, NP\ind{2} } \\
            }\\
cont      & \ms[lesen-rel]{
            agens & \ibox{1}\\
            thema & \ibox{2}\\
            }\\
}
\end{tabular}
\ex
\begin{tabular}[t]{@{}l@{}}
\stem{les} mit Passiv-Linking:\\
\ms[lesen $\wedge$ passive]{
stem-phon & \phonliste{ les }\\
cat       & \ms{
            comps & \ibox{1}\\
            arg-st & \sliste{ NP\ind{2} } $\oplus$ \ibox{1} \sliste{ NP\ind{3} }\\
            }\\
cont      & \ms[lesen-rel]{
            agens & \ibox{2}\\
            thema & \ibox{3}\\
            }\\
}
\end{tabular}
\zl
Der aufmerksame Leser wird festgestellt haben, dass in den Spezifikationen keine
Kasusinformation enthalten ist. In der Lexikonregel in (\ref{lr-passiv-beispiel})
weichen die Kasus des logischen Objekts im Eintrag für die Aktivform und die Passivform
voneinander ab: Im Aktiv erhält das Objekt Akkusativ, im Passiv dagegen Nominativ.
Entsprechende Beschränkungen müßten für die Aktiv- und die Passiv"=Linking"=Typen
formuliert werden. Das ist unproblematisch, man würde auf die Kasusprinzipien,
die im Kapitel~\ref{chap-kasus} erläutert werden, zurückgreifen.

Der Linking"=Typ für die Medialkonstruktion\is{Medialkonstruktion} läßt sich analog zum Passiv definieren:

\eas\label{constr-middle-linking}
Typ für Medial-Linking:\\
\ms[middle]{
cat       & \ms{
            comps & \ibox{1} $\oplus$ \sliste{ NP[\type{refl}] }\\
            arg-st & \sliste{ NP } $\oplus$ \ibox{1}\\
            }\\
}
\zs
Das logische Subjekt eines Verbs wird wie beim Passiv
nicht auf die \compsl übernommen. Zusätzlich wird noch ein Reflexivpronomen
(und eventuell ein Modifikator) an die Liste der verbleibenden Argumente \iboxb{1}
angehängt.

\eas
\stem{les} mit Medial-Linking:\\
\ms[lesen $\wedge$ middle]{
stem-phon & \phonliste{ les }\\
cat       & \ms{
            comps & \ibox{1} $\oplus$ \sliste{ NP[\type{refl}] }\\
            arg-st & \sliste{ NP\ind{2} } $\oplus$ \ibox{1} \sliste{ NP\ind{3} }\\
            }\\
cont      & \ms[lesen-rel]{
            agens & \ibox{2}\\
            thema & \ibox{3}\\
            }\\
}
\zs

\noindent
Dieses Verfahren funktioniert gut für die diskutierten Fälle. Probleme ergeben sich aber
an zwei Stellen: Zum einen gibt es Sprachen, in denen valenzverändernde Prozesse mehrfach
angewendet werden können, und zum anderen gibt es Probleme bei Interaktionen mit weiteren valenzverändernden Phänomenen.

Ich werde kurz das erste Problem diskutieren und mich danach ausführlich mit
dem zweiten beschäftigen: Vererbungsbasierte Ansätze scheitern an Phänomenen, bei denen ein argumentstrukturverändernder
Prozess mehrfach angewendet werden kann. Ein Beispiel hierfür ist die Kausativierung im Türkischen\il{Türkisch}
\citep{Lewis67a-u}: 
\ea
Öl-dür-t-tür-t- \\
`bewirken, dass jemand bewirkt, dass jemand jemanden tötet'\\
(töten = verursachen zu sterben)
\z
Das Kausativmorphem \suffix{t} wird dreimal mit dem Verb verknüpft. Diesen argumentstrukturverändeernden
Prozess kann man nicht in einer Vererbungshierarchie modellieren, denn wenn man sagen würde, dass
ein Wort drei mal von \type{causative} erbt, bekommt man nichts anderes, als wenn das Wort
einmal erbt. Für solche Phänomene braucht man Regeln, die ein
linguistisches Objekt zu einem anderen, komplexeren in Beziehung setzen, also Lexikonregeln, unäre
Regeln, die die Phonologie eines sprachlichen Zeichens verändern, oder binäre Regeln, die ein
bestimmtes Zeichen mit einem Derivationsmorphem kombinieren. Diese Regeln können dann das
Ursprungszeichen semantisch einbetten (also \zb \emph{cause} zu \emph{kill} hinzufügen). 

Auch gibt es Sprachen wie das Litauische\il{Litauisch} \citep[Abschnitt~5]{Timberlake82a}, das
Irische\il{Irisch} \citep{Noonan94a} und das Türkische\il{Türkisch} (\citealp{Ozkaragoez86a};
\citealp[Abschnitt~2.3.3]{Knecht85a-u}), in denen Doppelpassivierungen möglich sind. (\mex{1}) zeigt
die türkischen Beispiele von \citet[\page 77]{Ozkaragoez86a}:
%% \eal
%% \ex
%% \gll Vėjas nupūte tą lapelį.\\
%%      wind.\nom{} blow that leaf.\acc\\ %\jambox{(Lithuanian)}
%% \glt `The wind blew down that leaf.'
%% \ex
%% \label{ex-lit-personal-passive}
%% \gll Tas lapelis vėjo nupūstas.\\
%%      that leaf.\nom.\mas.\sg{} wind.\gen{} blown.\nom.\mas.\sg{}\\
%% \glt `That leaf was blown down by the wind.'
%% \ex
%% \label{ex-lit-double}
%% \gll To     lapelio               būta  vėjo        nupūsto.\\
%%      that leaf.\gen.\mas.\sg{} been.\nom.\neu.\sg{} wind.\gen{} blown.\gen.\mas.\sg{}\\
%% \glt `That leaf was (presumably) blown down by the wind.'
%% \zl
%% As Timberlake argues the personal passive in (\mex{0}b) should be seen as the source on which the
%% impersonal passives operates which ultimately results into the impersonal passive in
%% (\mex{0}c). 
\eal
\ex\label{ex-double-passivization-strangle}
\gll Bu şato-da boğ-ul-un-ur.\\
     dieses Schloss-\textsc{loc} erwürgt-\textsc{pass}-\textsc{pass}-\textsc{aor}\\
\glt `In diesem Schloss wird man (von jemandem) erwürgt.'
\ex\label{ex-double-passivization-hit}
\gll Bu oda-da döv-ül-ün-ür.\\
     dieser Raum-\textsc{loc} geschlagen-\textsc{pass}-\textsc{pass}-\textsc{aor}\\
\glt `In diesem Raum wird man (von jemandem) geschlagen.'
\ex
\gll Harp-te vur-ul-un-ur.\\
     Krieg-\textsc{loc} erschießen-\textsc{pass}-\textsc{pass}-\textsc{aor}\\
\glt `Im Krieg wird man (von jemandem) erschossen.'
\zl
\suffix{In}, \suffix{n} und \suffix{Il} sind Allomorphe des Passivmorphems. Laut Özkaragöz werden
die Daten durch eine Analyse am besten erfasst, die davon ausgeht, dass ein passiviertes transitives
Verb erneut passiviert werden kann, so dass sich dann ein unpersönliches Passiv ergibt. Solche
Doppelpassivierungen lassen sich ebenfalls nicht über Vererbung analysieren, denn wenn man die
entsprechenden Beschränkungen merhfach erbt, kommt keine neue Information hinzu bzw.\ ändern sich
Strukturen nicht.

Wenn man davon ausgeht, dass Valenzänderungen wie Passiv, Kausativierung und Medialkonstruktionen
sprachübergreifend mit denselben Mitteln beschrieben werden sollten, dann ist die aufgeführte
Evidenz ein Argument gegen vererbungsbasierte Analysen des Passivs
\citep{Mueller2006d,Mueller2007d}. Die Aktiv/Passiv"=Partition wurde eingeführt, um Lexikonregeln zu
vermeiden. Die diskutierten Beispiele zeigen jedoch, dass man Lexikonregeln bzw.\ analoge unäre
Regeln oder Morphemkombinationen braucht, wenn man valenzverändernde Prozesse sprachübergreifend
erklären will.\is{Passiv|)}\is{Kausativ!-konstruktion|)}

Soweit zum ersten Problem. Das zweite Problem sind andere valenzverändernde Prozesse,
die mit der Passivierung und untereinander interagieren. Hier können auch Beispiele aus dem Deutschen
herangezogen werden: So können \zb die bereits im Abschnitt~\ref{sec-phrasal-lexikalisch-rc} diskutierten
Resultativkonstruktionen sowohl im Passiv (\mex{1}) als auch in Medialkonstruktionen\is{Medialkonstruktion} (\mex{2})
vorkommen, wie die folgenden Beispiele von \citet[\page118]{Wunderlich97c} zeigen:\footnote{
        \citet[\page17]{KR95a} diskutieren ähnliche Beispiele.%
}
\eal
\label{bsp-res-passiv-zwei}
\ex Er hat den Teich leer gefischt.
\ex Der Teich wurde leer gefischt.
\zl
\eal
\label{ex-res-nonselected-obj-passive}
\ex\iw{trinken}\iw{leer}
Der Weinkeller  trinkt sich schnell leer.
\ex\iw{laufen}\iw{platt}
Der Rasen läuft sich leicht platt.
\zl
Das Problem besteht darin, dass das Objekt in (\mex{-1}a) kein Objekt von \emph{fischen} ist.
Damit es beim Passiv zum Subjekt werden kann, muss es vorher eingeführt worden sein. Man könnte
vorschlagen, das analog zur Einführung des Reflexivums zu tun, das funktioniert jedoch nicht,
wie deutlich wird, wenn man den entsprechenden Typ aufschreibt:
\ea
Typ für Resultativkonstruktion (funktioniert nicht):\\
\ms[resultative-construction]{
cat       & \ms{
            comps & \ibox{1} $\oplus$ \sliste{ NP, Predicate }\\
            arg-st & \ibox{1} \sliste{ NP }\\
            }\\
}
\z
Das Problem ist, dass wir in (\mex{0}) die \compsl spezifiziert haben. Dieser
Typ ist deshalb mit dem Passiv"=Linking"=Typ (\ref{constr-passive-linking}) und mit dem
Medial"=Linking"=Typ (\ref{constr-middle-linking}) unvereinbar. Damit das oben erörterte System zur Passivierung
funktioniert, müßte die Information, die in (\mex{0}) auf der \compsl steht, auf der \argstl stehen.

Im folgenden diskutiere ich einen Vorschlag von Koenig zur Lösung eines ähnlichen Problems
und zeige, dass seine Analyse das Problem nur verschiebt:
Koenig diskutiert Extrapositionsdaten wie das Beispiel in (\mex{1}b), die bisher mit der
Lexikonregel in (\mex{2}) analysiert wurden.
\eal
\ex That they will miss the deadline is likely.
\ex It is likely that they will miss the deadline.
\zl
\ea
\ms{
cat & \ms{ comps & \sliste{ \ibox{1} S[\type{fin}] }\\
           arg-st & \sliste{ \ibox{1} }\\
         }\\
} $\mapsto$ %
\ms{
cat & \ms{ comps & \sliste{ \ibox{2}, \ibox{1} }\\
           arg-st & \sliste{ \ibox{2} NP$_{it}$, \ibox{1} }\\
         }\\
}
\z
Er schlägt vor, die Argumentstruktur feiner zu gliedern und zwischen von einem bestimmten
Verb eingeführten Argumenten und zusätzlichen Argumenten zu unterscheiden.
Er gibt einen Eintrag für \emph{likely} an, der dem folgenden entspricht:\footnote{
  Koenig gibt auch die leere Liste als Wert von \comps an (bei ihm \subj und \comps). Hierbei
  scheint es sich um ein Versehen zu handeln, denn mit spezifiziertem \compsw
  würde die Analyse nicht funktionieren.%
}
\ea
\label{koenig-likely}
\ms[likely]{
cat & \ms{ arg-st & \ms{ sem-arg  & \ibox{1} \sliste{ S[\type{fin}]:\ibox{2} }\\
                         add-arg  & \ibox{3}\\
                         arg-list & \ibox{1} $\bigcirc$ \ibox{3}\\
                       }\\
         }\\
cont & \ms[likely-rel]{
       soa-arg & \ibox{2}\\
       }\\
}
\z
Unter \textsc{sem-arg} steht eine Liste der Argumente, die eigentlich zu einem
Lexikoneintrag gehören. Zusätzlich gibt es eine Liste \textsc{add-arg}. Der Wert
ist in (\mex{0}) nicht restringiert. An diese Stelle kann durch Unifikation mit
anderen Typen eine Liste mit einem oder mehreren Elementen gelangen. \textsc{arg-list}
ist dann eine Liste, die die Elemente von \textsc{sem-arg} und \textsc{add-arg} enthält.
`$\bigcirc$'\is{$\bigcirc$}\is{Relation!$\bigcirc$}\isrel{shuffle}
steht dabei für die Shuffle"=Relation, die zwei Listen miteinander so
verknüpft, dass sich eine Liste ergibt, die die Elemente der beiden Teillisten enthält,
wobei Elemente aus der ersten Liste zwischen Elementen aus der zweiten Liste stehen können.
Die Reihenfolge der Elemente der jeweiligen Listen untereinander darf jedoch nicht geändert
werden. Siehe auch Seite~\pageref{rel-shuffle}. Für unser konkretes Beispiel in
(\mex{0}) heißt das, dass S[\type{fin}] vor, zwischen oder nach Elementen stehen kann,
die durch Unifikation mit anderen Typen zu Elementen von \textsc{add-arg} und somit
von \textsc{arg-list} werden. Koenig spezifiziert einen Typ für die Extraposition,
der dem folgenden ähnelt:\footnote{
  Bei ihm heißt der Typ \type{extr-verb}, da \emph{likely} aber kein Verb ist,
  wären die Typen nicht unifizierbar, weshalb es sich hier um ein weiteres
  Versehen handeln muss. Bei Koenig steht außerdem S[\type{comp}] statt S[\type{fin}]
  wie im Eintrag von \emph{likely} und dem Unifikat von \emph{likely} und \type{extr-verb}.%
}
\ea
\label{linking-extraposition}
\ms[extraposition]{
cat & \ms{ comps & \sliste{ \ibox{1} \ldots \ibox{2} \ldots }\\
           arg-st & \ms{
                    add-arg  & \sliste{ \ldots \ibox{1} NP$_{it}$  \ldots }\\
                    arg-list & \sliste{ \ibox{1} \ldots \ibox{2} S[\type{fin}] \ldots }\\
                    }\\
         }\\
}
\z
Unifiziert man (\mex{-1}) mit (\mex{0}), wird die expletive NP aus der \textsc{add-arg}"=Liste
in die \textsc{arg-list} von (\mex{-1}) integriert. Der \textsc{arg-list}"=Wert in (\mex{0})
ist dafür eigentlich nicht nötig, wurde aber wohl spezifiziert, damit das Expletivum
vor S[\type{fin}] in \textsc{arg-list} eingefügt wird, denn durch die Listenverknüpfung
mittels `$\bigcirc$' ist ja keine Reihenfolge vorgegeben.
Die Unifikation von (\mex{-1}) und (\mex{0}) ist aber nicht (\mex{1}), wie das von Koenig
behauptet wird:
\ea
\ms[likely $\wedge$ extraposition]{
cat & \ms{ comps & \sliste{ \ibox{1}, \ibox{2} }\\
           arg-st & \ms{ sem-arg  & \sliste{ \ibox{2} S[\type{fin}]:\ibox{3} }\\
                         add-arg  & \sliste{ \ibox{1} NP$_{it}$ }\\
                         arg-list & \sliste{ \ibox{1}, \ibox{2} }\\
                       }\\
         }\\
cont & \ms[likely-rel]{
       soa-arg & \ibox{3}\\
       }\\
}
\z
Das Ergebnis der Unifikation ist vielmehr (\mex{1}):
\ea
\ms[likely $\wedge$ extraposition]{
cat & \ms{ comps & \sliste{ \ibox{1} \ldots \ibox{2} \ldots }\\
           arg-st & \ms{ sem-arg  & \sliste{ \ibox{2} S[\type{fin}]:\ibox{3} }\\
                         add-arg  & \sliste{ \ldots \ibox{1} NP$_{it}$  \ldots }\\
                         arg-list & \sliste{ \ibox{1} \ldots \ibox{2} S[\type{fin}] \ldots }\\
                       }\\
         }\\
cont & \ms[likely-rel]{
       soa-arg & \ibox{3}\\
       }\\
}
\z
Die Punkte in (\mex{-2}) heißen soviel wie: "`Vor der NP können noch beliebig viele Elemente
stehen und danach auch"'. Durch die Unifikation mit (\ref{koenig-likely}) werden die Punkte
nicht eliminiert, denn (\ref{koenig-likely}) sagt nichts über \textsc{add-arg}, und über
\textsc{arg-list} wird nur gesagt, dass es S[\type{fin}] enthalten muss.
Die Struktur in (\mex{0}) ist aber für eine HPSG nicht brauchbar, da sie für
die Analyse beliebig vieler ungrammatischer Sätze mit \emph{likely} benutzt werden kann,
denn die Valenzliste von (\mex{0}) ist wegen der Punkte am Ende nicht geschlossen.

Man könnte einwenden, dass (\mex{0}) ja noch mit einem der Typen \type{intransitive} oder
\type{transitive} kombiniert werden muss. Das ist richtig, aber da diese Typunifikation
automatisch erfolgen soll, würden ebenfalls Typkonjunktionen mit drei- und vierstelligen
Valenzlisten berechnet. Somit hätte die \compsl dann zwar ein Ende, ungrammatische
Sätze könnten aber dennoch analysiert werden.

Auf Seite~75 gibt Koenig folgenden Typ für die Extraposition von Indefinita an:\footnote{
  Bei ihm steht die \ibox{1} nach \textsc{add-arg} in der Liste.%
}
\ea
\ms[extr-ind]{
cat$|$arg-st & \ms{
                    add-arg  & \ibox{1} \sliste{ NP$_{il}$ }\\
                    sem-arg  & \ibox{2} \\
                    arg-list & \ibox{1} $\oplus$ \ibox{2}\\
                    }\\
}
\z
Bei einer solchen Spezifikation ergibt sich das Problem mit den offenen Listen,
das man mit (\ref{linking-extraposition}) bekommt, nicht. Allerdings sieht man jetzt
sehr klar das eigentliche Problem des Ansatzes: Man kann höchstens einmal über \textsc{add-arg}
etwas hinzufügen. Ein Typ wie der in (\mex{0}) ist mit Typen, die andere Elemente
in \textsc{add-arg} haben, nicht unifizierbar.

Mit dem Koenigschen System könnte man die Resultativkonstruktion wie folgt definieren:
\eas
Typ für Resultativkonstruktion (funktioniert auch nicht):\\
\ms[resultative-construction]{
cat$|$arg-st$|$add-arg & \sliste{ NP, Predicate }\\
}
\zs
Ein Verb wie \stem{fisch} bringt unter \textsc{sem-arg} sein Subjekt ein, und der
Typ in (\mex{0}) steuert das Objekt und das Resultativprädikat bei. Die Argumente
würden von \textsc{arg-list} je nach Aktiv-, Passiv- oder Medial"=Linking
auf die \compsl gesetzt. Dieser Ansatz versagt aber ebenfalls, da sogenannte freie
Dative\is{Kasus!Dativ} auch mit Resultativkonstruktionen vorkommen. Freie
Dative\is{Dativ!freier}\is{Dativ!incommodi@\emph{incommodi}}\is{Dativ!commodi@\emph{commodi}} wie in
(\mex{1}b) sagen etwas über den Nutznießer einer Handlung aus oder über den, dem sie schadet.
Wie (\mex{1}c) zeigt, können diese Dative vom \emph{bekommen}"=Passiv
erfaßt werden, was dafür spricht, sie als Argumente zu behandeln.
\eal
\ex Er bemalt den Tisch.
\ex Er bemalt ihr den Tisch.
\ex Sie bekommt den Tisch bemalt.
\zl
Wenn man diese Dative als Argumente behandeln will und Lexikonregeln zur Einführung
dieser Argumente nicht zur Verfügung stehen, dann muss man sie über einen Typ einführen.
Damit hat man aber dann Probleme mit Sätzen wie (\mex{1}):
\eal
\ex Er fischt ihm den Teich leer.
\ex Der Teich wurde ihm leer gefischt.
\ex Er bekam den Teich leer gefischt.
\zl
Da \textsc{add-arg} bereits im Resultativkonstruktionstyp verwendet wird, kann es in einem
Dativtyp nicht anders belegt werden.

Dieses Problem läßt sich nur durch die Stipulation eines \textsc{add-arg-2}"=Merkmals lösen. Finden
sich noch weitere Kombinationen von Argumenterweiterungen, muss man weitere \textsc{add-arg-n}"=Merkmale
in die Grammatik aufnehmen.

Ein weiteres Problem im Zusammenhang mit Resultativkonstruktionen ergibt sich daraus, dass
diese einen semantischen Beitrag leisten. (\mex{1}a) könnte man als (\mex{1}b) paraphrasieren
und als (\mex{1}c) repräsentieren:
\eal
\ex Er fischt den Teich leer.
\ex Dass er fischt, verursacht, dass der Teich leer wird.
\ex cause(fischen(er),become(leer(Teich)))
\zl
Man sieht, dass die Resultativbedeutung die Bedeutung des Verbs einbettet. Um das mittels Vererbung
machen zu können, muss man sich ebenfalls der Koenigschen Tricks bedienen: Man braucht ein Merkmal
zur Repräsentation der eigentlichen Semantik des Verbs (\textsc{lex-cont}), ein Merkmal für
Bedeutung, die von anderen Typen kommt (\textsc{add-cont}) und ein Merkmal, das dann den letztendlichen
Wert enthält (\textsc{cont-out}):\footnote{
  Siehe auch \citew[\page262]{Kathol94a} zur Umkodierung von Lexikonregeln in Vererbungshierarchien.
%% Man könnte ein Proto-Verb auf ein Proto-Verb abbilden.
%%
%% Das Ergebnis der LR-Anwendung muss dann spezifischer gemacht werden. Dabei kommen
%% dann wieder die Adjektive oder Verben raus.
%%
%%   Kathol erwähnt auch die Ergebnisse von Krieger und Nerbonne, die gezeigt haben,
%%   dass man Derivation nicht über Vererbung modellieren kann, schlägt dann aber vor,
%%   adjektivische Partizipien wie \emph{geliebte} in (i) über Vererbung abzuleiten.
%%   \ea
%%   der geliebte Junge
%%   \z
%%   Dieses Verfahren funktioniert nur mit einfachen Verben. Will man Verben wie \emph{beladen}
%%   oder \emph{umfahren} zu ihren Morphembestandteilen in Beziehung setzen, also als von
%%   \emph{laden} bzw.\ \emph{fahren} abgeleitet analysieren, dann kann man die Partizipien
%%   in (\mex{1}) nicht über Vererbung erklären:
%%   \eal
%%   \ex der beladene Wagen
%%   \ex der umgefahrene Pfosten
%%   \zl
%%   \emph{beladene} bzw.\ \emph{umgefahrene} kann man nur bilden, wenn man \emph{beladen}
%%   bzw.\ \emph{umgefahren} hat. Um diese Verben zu bekommen braucht man aber Derivation.
%%   Derivationsprodukte sind entweder ausgaben lexikalischer Regeln oder Ergebnis einer
%%   Kombination von Morphemen, die der syntaktischen Kombination ähnelt. In jedem Fall
%%   hat das Ergebnis keinen speziellen Typ
}
\ea
Resultativkonstruktion mit Semantik (funktioniert immer noch nicht):\\
\onems{
cat$|$arg-st$|$add-arg  \sliste{ NP, Predicate:\ibox{1} }\\
cont \ms{
       lex-cont & \ibox{2}\\
       add-cont   & \ibox{3} \textrm{cause(\ibox{2}, become\iboxb{1})}\\
       cont-out & \ibox{3}\\
       }
}
\z
Typen, die nichts an der Semantik ändern, unifizieren einfach \textsc{lex-cont} mit \textsc{cont-out},
solche, die etwas ändern, fügen Information in \textsc{add-cont} ein und können dabei auf \textsc{lex-cont}
Bezug nehmen. \textsc{cont-out} übernimmt das Ergebnis von \textsc{add-cont}. Das Problem ist hierbei ebenfalls,
dass das nur einmal funktioniert. Soll der Resultativkonstruktion noch die Information über einen
freien Dativ hinzugefügt werden, scheitert das Verfahren.


Wie\is{Morphologie!Derivation|(}\is{Affix|(} 
\citet{KN93a} festgestellt haben, kann man derivationelle Morphologie nicht allein mit Vererbung
beschreiben (zu den Details siehe Abschnitt~\ref{sec-derivation-vererbung}). Koenig nimmt deshalb
für Derivation einbettende Strukturen an. Aus Gründen der Einheitlichkeit geht er auch für die
Flexion\is{Morphologie!Flexion} von Einbettung aus.
Auf Seite~97 gibt er Beschreibungen für \emph{cat} (`Katze') und \emph{cats} (`Katzen') an, die den folgenden ähneln:\footnote{
  Ich habe seinen Pfad \textsc{$\mu$-struc$|$dghtr} durch \textsc{lex-dtr} ersetzt und den semantischen
  Beitrag von \emph{cat} weg\-ge\-lassen.%
}
\eal
\ex 
\ms[cat]{
phon & \ms{
       form \phonliste{ k\ae{}t }\\
       }\\
cat & \ms{ head & noun\\
         }\\
}
\ex
\ms[plural]{
phon & \ms{ aff$|$suff \phonliste{ s }\\
          }\\
lex-dtr & lexeme\\
}
\zl
Für die flektierte Form gibt er folgende Beschreibung an:
\ea
\ms[plural]{
phon & \ms{ form \phonliste{ k\ae{}ts }\\
          }\\
lex-dtr & \ms[lexeme $\wedge$ cat]{
                    phon & \ms{
                           form \phonliste{ k\ae{}t }\\
                           }\\
                    cat & \ms{ head & noun\\
                             }\\
                    }\\
} 
\z
                
\noindent
Die in (\mex{0}) dargestellte Einbettung entspricht dem, was in Abschnitt~\ref{sec-lexikon-felx-deriv}
skizziert wurde. Eine solche Einbettung kann man aber nicht in einer Vererbungshierarchie
darstellen. In einer Vererbungshierarchie könnte man nur die Information aus (\mex{-1}a) und
(\mex{-1}b) unifizieren, nicht aber (\mex{-1}a) in (\mex{-1}b) einbetten, wie das in (\mex{0})
der Fall ist. Eine zu Abbildung~\vref{bsp-partitionen} analoge Abbildung wie die in
Abbildung~\vref{bsp-partitionen-numerus} würde also nicht zu dem Ergebnis in (\mex{0})
führen.\footnote{
  \citet[\page93]{Koenig99a} unterscheidet zwischen Wurzeln und komplexen Lexemen (Stämmen und Wörtern).
  Komplexe Lexeme haben interne Struktur. Der Eintrag für die Wurzel \emph{vendre} (`verkaufen'), den er
  auf Seite~72 angibt, hat ebenfalls eine komplexe Struktur (siehe auch (\ref{le-vendre})\vpageref{le-vendre}
  in diesem Buch). Ich verzichte deshalb in den folgenden Abbildungen auf diese Unterscheidung.%
}
\begin{figure}
\begin{forest}
type hierarchy
[lexeme
  [number,partition,
    [sg]
    [pl]]
  [root,partition,
    [cat]
    [dog]]]
\end{forest}
\caption{\label{bsp-partitionen-numerus}Typhierarchie für Flexion}
\end{figure}
Auf Seite~72 gibt Koenig Einträge für das französische \emph{vendre} und Passiv- bzw.\
Aktiv"=Linking an, die hier vereinfacht wiedergegeben sind.
\eal
\ex
\label{le-vendre} 
\ms[vendre]{
%phon & \phonliste{ v{\~a} }\\
lex-dtr$|$cat$|$arg-st$|$sem-arg & \sliste{ NP, NP }\\
}
\ex 
\ms[active]{
cat$|$arg-st$|$sem-arg & \ibox{1}\\
lex-dtr$|$cat$|$arg-st$|$sem-arg & \ibox{1}\\
}
\ex 
\ms[medio-passive]{
cat$|$arg-st$|$sem-arg & \ibox{1}\\
lex-dtr$|$cat$|$arg-st$|$sem-arg & \sliste{ NP } $\oplus$ \ibox{1}\\
}
\zl
Im Gegensatz zu \type{cat} und \type{plural} können für \type{vendre}, \type{active}
und \type{medio-passive} automatisch die erwünschten Subtypen errechnet werden. Der Trick
besteht darin, im Lexikoneintrag für \emph{vendre} die relevante Information eine Etage
tiefer, nämlich bei den Töchtern zu spezifizieren. Rechnet man dann automatisch die Strukturen
für \type{vendre} $\wedge$ \type{active} bzw.\ \type{vendre} $\wedge$ \type{medio-passive}
aus, bekommt man auf der obersten Ebene die erwünschte Valenzrepräsentation.

Das ist aber nicht die Lösung des Problems. Das System, das Koenig entwickelt, läßt sich
nicht umsetzen, wenn die Typhierarchie Derivationstypen und Wurzeln in Partitionen enthält.
Mit einer Hierarchie wie der in Abbildung~\vref{bsp-derivation-wurzeln-partitionen} können
wir \emph{lesbar} und \emph{traurigkeit} ableiten, nicht aber \emph{Lesbarkeit}.
\begin{figure}
\begin{forest}
type hierarchy
[lexeme
  [root,partition,
    [les]
    [traurig]]
  [affix,partition,
    [bar]
    [keit]]]
\end{forest}
\caption{\label{bsp-derivation-wurzeln-partitionen}Derivation mit Wurzeln in Partitionen}
\end{figure}
Das liegt daran, dass das Ergebnis der Typberechnung \type{les} $\wedge$ \type{bar}
und \type{traurig} $\wedge$ \type{keit} enthält. In der Struktur für \type{traurig} $\wedge$ \type{keit}
ist der Tochterwert instantiiert (als Adjektiv mit \phonw \emph{traurig}).
Da wir keine andere Struktur in der Grammatik haben,
die das Affix \suffix{keit} mit von \emph{traurig} verschiedenem Material kombinieren
kann, wäre das Wort \emph{Lesbarkeit} nicht analysierbar. Daraus ergibt sich, dass
Affixe, die mit komplexen Morphemkombinationen kombinierbar sein sollen, nie mit
Wurzeln konjunktiv verknüpft werden dürfen. Ohne eine solche Verknüpfung ist aber die Attraktivität
der Online"=Typberechnung dahin und man ist bei einem Ansatz angelangt, der genau den im
Abschnitt~\ref{sec-lr} besprochenen Lexikonregeln entspricht. Auf Hilfsmerkmale wie
\textsc{add-arg} oder \textsc{add-cont} kann man dann verzichten.


%% Das stimmt nicht, da es nicht um Derivation sondern um Flexion geht, bei Flexion
%% könnte man auf einen Proto-Typ abbilden, der ganz allgemein ist und Untertypen
%% mit entsprechender Wortart hat. Da die Wortart selbst vom Kopf vorgegeben ist, geht das
%% Ganze dann auf.
%% \begin{comment}
%% \citet[\page263]{Kathol94a} schlägt vor, Fälle, in denen nur Argumentstrukturänderungen,
%% Wortartänderungen oder Flexionsänderungen vorliegen, über Vererbung zu erklären. Er schlägt vor,
%% adjektivische Partizipien auf diese Weise zu analysieren. Er nimmt an, dass es einen Typ
%% \type{proto-verb} gibt, der Information zur Valenz und zum semantischen Beitrag in Hilfsmerkmalen
%% enthält. Dieser Typ hat dann Untertypen für \type{verb-word} und \type{pass-part}, die jeweils
%% für die flektierten Verben und die Passivpartizipien stehen. Adjektive müssen entweder prädikativ
%% oder attributiv sein.
%% \begin{figure}[htb]
%% \centerline{\begin{pspicture}(-0.8,0.8)(10,5.4)
%% %\psgrid
%% \rput[B](1,5){\rnode{protoverb}{\type{proto-verb}}}
%% \rput[B](5,5){\rnode{adjword}{\type{adj-word}}}
%% %
%% \rput[B](7,3){\rnode{paaword}{\fbox\type{pred/attr-adj-word}}}
%% %
%% \rput[B](0,1){\rnode{verbword}{\type{verb-word}}}
%% %
%% \rput[B](3,1){\rnode{pass-part}{\type{pass-part}}}
%% \rput[B](6,1){\rnode{prd-adj-word}{\type{prd-adj-word}}}
%% \rput[B](9,1){\rnode{attr-adj-word}{\type{attr-adj-word}}}
%% \psset{angleA=-90,angleB=90,arm=0pt}
%% %
%% \ncdiag{protoverb}{verbword}\ncdiag{protoverb}{pass-part}
%% \ncdiag{adjword}{pass-part}\ncdiag{adjword}{paaword}
%% \ncdiag{paaword}{prd-adj-word}\ncdiag{paaword}{attr-adj-word}
%% %
%% \end{pspicture}}
%% \caption{\label{bsp-derivation-kathol}Derivation adjektivischer Partizipien}
%% \end{figure}
%% In der HPSG wird angenommen, dass die Merkmalstrukturen, die Objekte modellieren,
%% maximal spezifisch sind. Das heißt, dass es kein Objekt vom Typ \type{proto-verb}
%% geben kann, sondern nur Objekte vom Typ \type{verb-word} bzw. den automatisch
%% berechneten Typen \type{pass-part} $\wedge$ \type{prd-adj-word} und \type{pass-part} $\wedge$ \type{attr-adj-word}.
%% Man muss also in die Hierarchie in Abbildung~\ref{bsp-derivation-kathol} noch einen Typ
%% \type{verb-stem} einfügen, von dem die Objekte beschrieben werden, die man für
%% derivationelle Morphologie benötigt. Das Ergebnis der Kombination des Präfixes \prefix{be}
%% mit dem Stamm \stem{lad} muss dann vom Typ \type{proto-verb} sein, damit Kathols System
%% funktioniert und Partizipien wie das in (\mex{1}) von der Grammatik lizenziert werden.
%% \ea
%% der beladene Wagen
%% \z
%% Mit einem Ansatz, der Morpheme mittels Regeln verknüpft, die den Syntaxregeln ähneln, bekommt man
%% dieses Ergebnis nicht ohne weiteres. Das Ergebnis von Affix/Wurzel- bzw. Affix/Stamm"=Kombinationen
%% ist ein komplexer Stamm. Würde man explizit angeben wollen, dass das Ergebnis der Kombination von 
%% \prefix{be} und \stem{lad} vom Typ \type{proto-verb} ist, müßte man sehr spezielle Derivationsschemata 
%% haben, nämlich jeweils eins für Derivation mit Proto"=Verb als Ergebnis, eins mit Proto"=Nomen als Ergebnis, usw.\footnote{
%%   Wenn man für Nomina, Adjektive usw.\ nicht ähnliche Effekte wie für Verben erreichen will, dann braucht man
%%   keine Proto"=Typen. Allerdings braucht man dann mindestens einen Typ Nicht"=Proto"=Verb.%
%% }
%% Mit entsprechend formulierten Lexikonregeln müßte man Subtypen für Lexikonregeln vom Typ
%% \type{proto-verb-lexical-rule}, \type{proto-noun-lexical-rule} usw.\ definieren. Zusätzlich zu
%% den fünf Hilfsmerkmalen, die Kathol verwendet, und zu dem zusätzlichen Typ \type{proto-verb} brauchte
%% man also noch weitere Typen. Alles in Allem wird der Ansatz dadurch sehr unattraktiv.
%% %
%% %%   \emph{beladene} bzw.\ \emph{umgefahrene} kann man nur bilden, wenn man \emph{beladen}
%% %%   bzw.\ \emph{umgefahren} hat. Um diese Verben zu bekommen braucht man aber Derivation.
%% %%   Derivationsprodukte sind entweder ausgaben lexikalischer Regeln oder Ergebnis einer
%% %%   Kombination von Morphemen, die der syntaktischen Kombination ähnelt. In jedem Fall
%% %%   hat das Ergebnis keinen speziellen Typ
%%\end{comment}
\is{Morphologie!Derivation|)}%
\is{Linking!-Konstruktionen|)}%
\is{Affix|)}


\subsection{Defaults}
\label{sec-defaults}

In\is{Default|(} der Linguistik ist oft von Default"=Werten die Rede. 
Zum Beispiel wird der Nominativ als Default"=Kasus\is{Kasus!Default-} bezeichnet (\citealp[\page
  78]{Sternefeld95a}; \citealp[\page3]{Jacobs91a}; \citealp[\page 265, 
\page 417]{Abraham2005a}), %für das Deutsche Svenonius2005:19 für das Isländische
andere Autoren bezeichnen den Akkusativ als Default"=Kasus (\citealp[\page969]{Zwicky86};
\citealp[\page688]{Hoeksema91c}) und \citet[\page 733]{Wunderlich93a} nennt alle strukturellen Kasus
(\dash die meisten Nominative und Akkusative und je nach Sichtweise auch einige Dative) Default"=Kasus.
%accusative by default (GB & GPSG, GKPS85).
%
Dass etwas einen Default"=Wert hat, heißt soviel wie: "`Wenn nirgends etwas anderes gesagt
wird, dann hat ein Merkmal den als Default"=Wert angegebenen Wert"'. Solche Defaults sind
in constraint"=basierten Grammatiken aber nicht ungefährlich: Zum Beispiel kann Information
in Valenzlisten unterspezifiziert sein. Dadurch dass ein Kopf mit einer selegierten Phrase
kombiniert wird, wird auch die Information in der Valenzliste spezifischer; die
Information wird ja unifiziert.\NOTE{!Unifiziert \ldots} Würde man so etwas sagen wie: "`Wenn der Kasus nirgends
sonst in der syntaktischen Struktur spezifiziert wird, ist er Nominativ."', bekäme man falsche
Analysen, da die Kombination des Kopfes mit einer Dativ"=NP dazu führt, dass innerhalb der
syntaktischen Struktur der Kasus spezifiziert ist und zwar unerwünschterweise als Dativ.
Man muss also vor der Überprüfung von Valenzanforderungen\is{Valenz} dafür sorgen, dass entsprechende
Defaultwerte zu echten, nicht"=veränderbaren Werten werden.

Es kann durchaus sinnvoll sein, von Default"=Werten zu sprechen, doch mitunter lassen Autoren im Dunkeln,
was genau sie damit meinen. Wenn man keine genaue Erklärung dafür geben kann, 
was es bedeuten soll, wenn man sagt, dass ein Merkmal einen Default"=Wert hat,
sollte man Defaults vermeiden.

\subsubsection{Defaults in Vererbungshierarchien}

Das klassische Beispiel zur Default"=Vererbung aus der Künstlichen Intelligenz
ist die Subhierarchie für das Konzept \emph{Vogel}: Vögel haben normalerweise
Flügel und können fliegen. Diese Eigenschaften erbt der Subtyp \type{Spatz},
beim Subtyp \type{Pinguin} wird aber die Eigenschaft des Fliegen"=Könnens überschrieben.
Die entsprechende Hierarchie mit Eigenschaften zeigt Abbildung~\vref{pinguin}.
\begin{figure}
\begin{forest}
type hierarchy
[Lebewesen
   [Vogel: \textnormal{hat Flügel, fliegt}
     [Pinguin: \textnormal{fliegt nicht}]
     [Spatz]]]
\end{forest}
\caption{\label{pinguin}Typhierarchie für \type{Vogel} mit Default-Vererbung}
\end{figure}
Wenn ein solches Überschreiben von Information möglich ist, spricht man von 
nicht"=monotoner\is{Monotonie} Vererbung. Die Information in der Typhierarchie in Abbildung~\ref{pinguin}
läßt sich auch ohne Defaults darstellen, wie Abbildung~\vref{mon-pinguin} zeigt.
\begin{figure}
\begin{forest}
type hierarchy
[Lebewesen
   [Vogel: \textnormal{hat Flügel}, name=Vogel
     [Pinguin: \textnormal{fliegt nicht}]]
   [fliegendes Lebewesen: \textnormal{fliegt}
     [Spatz, edge to=Vogel]]]
\end{forest}
\caption{\label{mon-pinguin}Typhierarchie für \type{Vogel} ohne Default-Vererbung}
\end{figure}
Man sagt eben nicht, dass alle Vögel fliegen können, was ja faktisch auch falsch ist.
Statt dessen repräsentiert man am Typ \type{Vogel} nur die tatsächlichen Eigenschaften,
die alle Vögel besitzen. Die Vögel, die fliegen können, erben diese Eigenschaft von
einem Typ \type{fliegendes Lebewesen}, von dem \zb auch gleichzeitig noch bestimmte Klassen
von Insekten erben können. Die Frage, ob man Default"=Vererbung in der Linguistik
braucht, wird kontrovers diskutiert. Sind Generalisierungen, die für ganze Subklassen nicht gelten,
Generalisierungen? 
%Wird erfaßt, wie stark ein bestimmter Typ von anderen abweicht?

Ein Problem mit nicht-monotoner Vererbung
ergibt sich, wenn man Typhierarchien mit Mehrfachvererbung\is{Vererbung!Mehrfach-} hat und
die Information, die von Obertypen ererbt wird, inkompatibel ist. Das klassische
Beispiel für diese Situation ist der sogenannte Nixon"=Diamant\is{Nixon"=Diamant} in Abbildung~\vref{abb-nixon}.
\NOTE{geht auf \citep{Touretzky86a-u} zurück}
\begin{figure}
\begin{forest}
type hierarchy
[Mensch
   [Quaker: \textnormal{Pazifist}, name=Vogel
     [Nixon: \textnormal{?}, edge to=!un, before drawing tree={x/.option=!r.x}]]
   [Republikaner: \textnormal{kein Pazifist}]]
\end{forest}
\caption{\label{abb-nixon}Der Nixon-Diamant: Das Problem bei nicht-monotoner Mehr\-fach\-ver\-er\-bung}
\end{figure}
Man kann die Eigenschaften des Typs \type{Nixon} nicht automatisch bestimmen. Deshalb muss
man irgendeinen Weg finden, festzulegen, welcher Wert `gewinnt'. Man kann das auf verschiedene
Weisen tun. Eine Möglichkeit ist, überschreibbare Information explizit zu kennzeichnen. Man würde dann
sagen, dass Quaker nur per Default Pazifisten sind. Wenn die Information über die Republikaner keine
Default"=Information ist, dann würde das dazu führen, dass \emph{Nixon} die Information "`kein Pazifist"' erbt.
Für den Fall, dass beide Eigenschaften als Default"=Information gekennzeichnet sind, muss man sich
ebenfalls ein Vorgehen überlegen: Ist der Wert am erbenden Knoten dann eine disjunktive Verknüpfung
der Werte des Obertyps, der allgemeinste Typ, der auf beide Werte paßt, oder einfach nicht definiert?
In unserem Beispiel hier ist die Entscheidung für das eine oder das andere relativ harmlos, aber
in einer umfangreichen Theorie mit vielen im Typsystem kodierten Unterscheidungen können die Konsequenzen
enorm sein.


\subsubsection{Mißbrauch von Defaults}

Eine interessante Möglichkeit, die die Formalisierung der Default"=Unifikation in \citew{LC99a}
bietet, ist es, Strukturteilung als Default zu spezifizieren. Dabei kann die Information über die
Strukturteilung überschrieben werden, Information über Unterstrukturen wird aber dennoch identifiziert.
Diese Eigenschaft der Default"=Unifikation nutzen \citet[\page33]{GSag2000a-u} in ihrer Formulierung
des Generalisierten Kopfmerkmalsprinzips\is{Prinzip!Generalisiertes Kopfmerkmals-} aus:


\eas
Generalisiertes Kopfmerkmalsprinzip \citep[\page33]{GSag2000a-u}:\\
\type{headed"=phrase}\istype{headed"=phrase} \impl
\onems{
synsem  / \ibox{1}\\
head-dtr$|$synsem / \ibox{1}\\
}
\zs


\noindent
Das Prinzip besagt, dass alle syntaktische und semantische Information der Kopf"|tochter
mit der syntaktischen und semantischen Information der Mutter identisch ist (zu \synsem
siehe Kapitel~\ref{chap-lokalitaet}). Diese Aussage kann bei Untertypen von \type{headed"=phrase} überschrieben werden,
was dann dazu führt, dass \zb nur noch die Kopf"|information (alles unter \textsc{head}) oder nur
noch die \compswe identifiziert werden, also nur noch Teile der eigentlich identifizierten
Strukturen.

Diese Notation ermöglicht eine sehr kompakte Darstellung von Identitätsanforderungen, stellt
meiner Meinung nach aber einen Mißbrauch von Defaults dar, denn es gibt keine einzige Struktur,
in der das Generalisierte Kopfmerkmalsprinzip ohne Überschreibung bestimmter Werte gilt: Bei allen Untertypen von \type{headed"=phrase}
wird die Default"=Pfadgleichung überschrieben. Eine Generalisierung, die nie gilt, ist keine Generalisierung.


Einen anderen interessanten Vorschlag zur Verwendung von Defaults macht \citet[\page86--87]{Villavicencio2000a-u}.
Sie schlägt vor, das Ende einer Liste als Default"=Information zu spezifizieren. In der folgenden Beschreibung
für intransitive Verben steht \type{e-list}\istype{e-list} für die leere Liste und
\type{ne-list}\istype{ne-list} für eine nicht leere Liste. 
Der Wert von \textsc{hd}\isfeat{hd} ist das erste Element der Liste, der Wert von \textsc{tl}\isfeat{tl} ist der Rest der Liste.
\ea
Beschränkungen für intransitive Verben (nach \citet{Villavicencio2000a-u}):\\
\type{intransitive-verb}\istype{intransitive-verb}  \impl
\ms{
 comps & \ms[ne-list]{ hd & \textrm{NP} \\
                        tl & / e-list\\
                      }\\
}
\z
Da der Listenrest als Default"=Wert markiert ist, kann er von Untertypen überschrieben werden.
\citet[\page86--87]{Villavicencio2000a-u} schlägt vor, \type{transitive-verb} als Untertyp
von \type{intransitive-verb} zu klassifizieren:
\ea
Beschränkungen für transitive Verben (nach \citet{Villavicencio2000a-u}):\\
\type{transitive-verb}\istype{transitive-verb} \impl 
\ms{
 comps & \ms[ne-list]{ tl & \ms[ne-list]{ hd & \textrm{NP} \\
                                           tl & / e-list\\
                                         }\\
                    }\\
}
\z
Der Wert von \textsc{tl} in (\mex{-1}) wird in (\mex{0}) überschrieben: Die Liste wird erweitert,
so dass sie zwei NPen enthält. Die Beschränkungen von (\mex{0}) zusammen mit den nicht
überschriebenen Beschränkungen aus (\mex{-1}) zeigt (\mex{1}):
\ea
\ms[transitive-verb]{
 comps & \ms[ne-list]{ hd & \textrm{NP} \\
                        tl & \ms[ne-list]{ hd & \textrm{NP} \\
                                           tl & / e-list\\
                                         }\\
                    }\\
}
\z



\noindent
Die Klassifikation von transitiven Verben als Untertyp der intransitiven Verben dürften die meisten
Linguisten unintuitiv finden. Läßt man solcherart Klassifikationen zu, stellt sich sofort die Frage,
wo die Grenzen sind: Warum kann ein intransitives Verb nicht von einem Nomen erben und dann die
Wortart überschreiben? In der KI"=Literatur wurde die Frage wie folgt gestellt: \emph{Wenn
ein Pinguin ein Vogel ist, der nicht fliegen kann, warum kann man dann nicht sagen, dass ein Holzklotz
ein Vogel ist, der nicht fliegt, keine Federn hat und keine Eier legt} \citep[\page330]{LS99a}
oder dass ein Pferd ein Tisch ist, außer das es lebt, denn immerhin hat ein Tisch ja vier Beine?
Die Antwort, die darauf gegeben wurde, ist das \emph{Best Fit Principle} \citep{Winograd76a-u,Hudson2003a}.
Nach dem \emph{Best Fit Principle} wird nur dann etwas als Vogel klassifiziert, wenn es mehr
Eigenschaften mit Vögeln teilt als mit allen anderen Objekten. Das Problem am \emph{Best Fit Principle}
ist, dass es im vorliegenden Fall nicht hilft, denn bei einer Klassifikation von Valenzmustern
ist zwischen den beiden Möglichkeiten in Abbildung~\vref{abb-valenz-defaults} zu wählen.
\begin{figure}
\hfill
\begin{forest}
type hierarchy
[verb: \type{list}
  [intrans-verb\textnormal{:} \sliste{ NP } 
     [trans-verb\textnormal{:} \sliste{ NP, NP, \ldots{} } ] ] ]
\end{forest}
\hfill
\begin{forest}
type hierarchy
[verb: \type{list}
  [\begin{tabular}{c}
   intrans-verb\textnormal{:}\\ \sliste{ NP }
   \end{tabular} ] 
  [\begin{tabular}{c}
   trans-verb\textnormal{:}\\ \sliste{ NP, NP, \ldots{} }
   \end{tabular} ] ]
\end{forest}
\hfill\mbox{}
\caption{\label{abb-valenz-defaults}\emph{Best Fit Principle} und Valenzklassen}
\end{figure}
Geht man bei der Integration der transitiven Verben in eine Typhierarchie danach,
ob transitive Verben mehr Eigenschaften mit intransitiven Verben oder mit Verben
allgemein teilen, dann muss man die linke Hierarchie wählen, was ich -- wie bereits erwähnt --
sehr unintuitiv finde. Die Alternative ist die rechte Hierarchie. Wenn man die Gemeinsamkeiten der
intransitiven und der transitiven Verben erfassen will, dann kann man noch einen weiteren Typ
einführen, von dem beide Typen erben \citep[\page205]{ps}. 

%\subsubsection{Unmögliche Anwendung von Defaults}


\subsection{Derivationelle Morphologie und Vererbung}
\label{sec-derivation-vererbung}

\is{Morphologie!Derivation|(}%
In diesem Kapitel wurden zwei grundlegende Mittel vorgestellt, die dazu dienen,
Generalisierungen zu erfassen und die Information im Lexikon kompakt zu repräsentieren.
Argumentstrukturverändernde Prozesse wurden dabei über Lexikonregeln erklärt. In der Literatur
wird immer wieder vorgeschlagen, auch derivationelle Morphologie über Vererbung zu beschreiben
(\citealp*[\page 218]{RCWA91a}, \citealp{MR2001a}).
Bei derivationeller Morphologie kann sich die syntaktische Kategorie oder die Bedeutung ändern:
\eal
\ex Les+bar+keit (V $\to$ Adj $\to$ Nomen)
\ex be+regnen
\zl
Ableitungen wie (\mex{0}a) lassen sich mit einfacher Vererbung nicht richtig modellieren, da
die Kategoriewerte der beteiligten Morpheme inkompatibel sind. Verwendet man Defaults, kann man
zwar einzelne Wörter über Vererbung beschreiben, aber man kann kein vernünftiges morphologisches
System zusammenstellen, da explizit angegeben werden muss, welche syntaktische Kategorie \emph{lesbar}
bzw.\ \emph{Lesbarkeit} hat.

\citet[\page38]{MR2001a} geben das folgende Prinzip an, das regeln soll, welche Information
überschrieben wird:
\begin{quote}
OVERRIDE PRINCIPLE.\is{Prinzip!"Uberschreibungs-} If lexical and structural meanings conflict, the semantic
constraints of the lexical element conform to those of the grammatical structure
with which it is combined. \citep[\page38]{MR2001a}
\end{quote}
Man könnte nun annehmen, dass die Kategorie von \emph{lesbar} bzw.\ \emph{Lesbarkeit}
durch ein solches Prinzip bestimmt wird. Das Override Principle läßt sich aber
in bezug auf Vererbungshierarchien nicht formalisieren, und wenn man Einbettung hat,
braucht man es nicht. Das soll im folgenden kurz erklärt werden: Man könnte
das Wort \emph{Lesbarkeit} in einer Vererbungshierarchie wie in Abbildung~\vref{abb-lesbarkeit} repräsentieren.\footnote{
  Eine komplexere Analyse mit explizit als überschreibbar gekennzeichneten
  Defaults nach \citew{LC99a} findet man in \citew{MuellerDefaults}. Die Behandlung
  der Derivation über Default"=Vererbung ist auch in einem solchen System
  nicht ohne erhebliche Erweiterungen des Formalismus (zum Beispiel um reguläre
  Ausdrücke in Merkmalstrukturen) möglich. Valenzänderungen, wie sie
  für den Übergang vom Adjektiv zum Nomen nötig sind, können bei der Annahme
  von listenwertigen Valenzmerkmalen nicht so modelliert werden, dass das
  morphologische System automatisch Wörter erzeugen kann.%
}
\begin{figure}
\begin{forest}
    for tree={grow=north,parent anchor=north,child anchor=south,where n children=0{tier=word}{},l sep+=\baselineskip}
    [lesbarkeit
      [keit]
      [lesbar
        [bar]
        [les]]]
    \end{forest}
\caption{\label{abb-lesbarkeit}\emph{Lesbarkeit} über Defaultvererbung}
\end{figure}
Das Override Principle würde man dann so interpretieren, dass bei Konflikten in bezug
auf die syntaktische Kategorie immer das Affix gewinnt. Damit hätte man dann irgendwie
zentral geregelt, wie Überschreibung vorgenommen wird (Wie das zu formalisieren wäre,
ist eine offene Frage.).
% MR2001a:54 -> Unifikation
Allerdings hat man dann das Problem, dass eine automatische
Berechnung aller von einer Grammatik lizenzierten Wörter, wie sie bei Koenig
möglich ist (siehe \figuresref{bsp-partitionen}{bsp-partitionen-ausgerechnet}), nicht mehr durchgeführt werden kann.
Der Grund hierfür ist, dass Restriktionen, die ein Affix in bezug auf das Element spezifiziert, mit dem es
kombiniert werden kann, nicht mehr zur Wirkung kommen, da sie einfach überschrieben werden.
Wenn \zb \bars verlangt, dass der Verbstamm, mit dem es kombiniert wird, transitiv ist, würde es bei
der Kombination mit einem intransitiven Stamm die Valenzinformation des intransitiven Stammes
einfach überschreiben.
In einem System, das Derivation über Einbettung beschreibt, braucht man das Override Principle
überhaupt nicht, da man die Information des Mutterknotens unabhängig von der des eingebetteten
Objekts spezifizieren kann: (\mex{1})\vpageref{lr-bard-skizze} skizziert eine Lexikonregel für die \bard (für die
vollständige Version der Lexikonregel für die \bard siehe (\ref{lr-bar-adj}) auf Seite~\pageref{lr-bar-adj}).
\bard ist nur für Verben mit Subjekt und Akkusativobjekt produktiv. Die \textsc{lex-dtr} ist in
(\mex{1}) entsprechend spezifiziert. \etag\is{\etag} steht für einen beliebigen Wert.

\begin{figure}[htbp]
\eas
\label{lr-bard-skizze}
Lexikonregel für \bard (Skizze):\\
\ms{
phon & \ibox{1} $\oplus$ \phonliste{ bar }\\
cat  & \ms{
                      head & adj\\
                      comps & \sliste{ NP[\nom]\ind{2} }\\
                      }\\
lex-dtr & \ms{ phon & \ibox{1}\\
               cat  & \ms{
                      head   & verb\\
                      comps & \sliste{ NP[\nom], NP[\acc]\ind{2} } $\oplus$ \etag\\ 
                      }\\
             }\\
}
\zs
\vspace{-\baselineskip}\end{figure}

%\noindent
Für die Behandlung von Ausnahmen wie \zb \emph{brennbar}, das nicht von einem transitiven Verb
abgeleitet ist, benötigt man keine Defaults. Es reicht, eine entsprechend allgemeine (nicht
produktive) Lexikonregel zu haben, die keine (oder andere) Restriktionen in bezug auf die Valenzliste des Verbs enthält. 
Diese allgemeinere Lexikonregel ist Bestandteil der im Lexikon gespeicherten Information
zum idiosynkratischen \emph{brennbar}, für die produktiven Formen wird die Lexikonregel in (\mex{0})
verwendet \citep{Riehemann98a}.



Es gibt einen weiteren Grund dafür, warum man derivationelle Morphologie nicht mit Vererbung beschreiben kann:
Mit Vererbung läßt sich keine rekursive\is{Rekursion} Einbettung erzeugen. Man kann also ein Nomen wie
\emph{Vorvorvorvorversion}\footnote{%
   \url{http://www.sgaf.de/viewtopic.php?t=21551&postdays=0&postorder=asc&start=0}. 11.11.2006.%
} bzw.\ \emph{Vorvorvorvorvorversion}\footnote{%
  \url{http://forum.geizhals.at/t393036,3147329.html} 11.11.2006.%
}
nicht auf sinnvolle Weise analysieren, denn wenn man die Information, die in \emph{Vorversion} enthalten
ist, noch einmal mit der Information von \prefix{vor} kombiniert, kommt nichts Neues hinzu \citep{KN93a}.
Man könnte nun behaupten, dass die zehnfache Präfigierung von
\prefix{vor} an ein Nomen ohnehin nicht vorkommt und dass man alle vorkommenden Wörter über Vererbung
kompakt repräsentieren kann, aber eine solche Analyse bedeutet, dass
man alle möglichen Stamm"=Affix"=Kombinationen aufschreiben muss, und entspricht dem in der Einleitung 
%auf Seite~\pageref{sec-wozu-syntax} 
diskutierten Aufschreiben aller möglichen Sätze, ist somit wenig erhellend und spiegelt die
menschliche Sprachfähigkeit nicht adäquat wider. Auch ist die Festlegung einer solchen Grenze ad
hoc. Bei der Diskussion von Stammbäumen findet man durchaus auch noch eine größere Anzahl von
Präfixen:
\eal
\ex Somit startet mit diesem Ur$_9$"-ur$_8$"-ur$_7$"-ur$_6$"-ur$_5$"-ur$_34$"-ur$_3$"-ur$_2$"-ur$_1$"-großvater die Ahnentafel.\footnote{
  \url{http://www.rothschopf.eu/}. 06.06.2008.
}
\ex Deshalb weiß ich jetzt auch, dass mein Ur$_{10}$"-ur$_9$"-ur$_8$"-ur$_7$"-ur$_6$"-ur$_5$"-ur$_4$"-ur$_3$"-ur$_2$"-ur$_1$"-großvater (kein Witz!) Jakob Ehl um
1550 in Voelfling geboren wurde, eine Frau namens Margreth Laux aus Ittersdorf hatte und im Alter
von sechzig Jahren (also 1610) noch "`lang und stark"' war. Unfassbar.\footnote{
  \url{http://jan.powerdan.de/blog/?p=369}. 06.06.2008.
}
\ex Mein Ur$_{11}$"-ur$_{10}$"-ur$_9$"-ur$_8$"-ur$_7$"-ur$_6$"-ur$_5$"-ur$_4$"-ur$_3$"-ur$_2$"-ur$_1$"-großvater und mein Ur$_{13}$"-ur$_{12}$"-ur$_{11}$"-ur$_{10}$"-ur$_9$"-ur$_8$"-ur$_7$"-ur$_6$"-ur$_5$"-ur$_4$"-ur$_3$"-ur$_2$"-ur$_1$"-großvater hatten diesen
Vornamen um 1600.\footnote{
  \url{http://www.baby-vornamen.de/Jungen/S/St/Stoffel/}. 06.06.2008
}
\ex Nein, jeder Deutsche hat mindestens einen
ur$_{16}$"-ur$_{15}$"-ur$_{14}$"-ur$_{13}$"-ur$_{12}$"-ur$_{11}$"-ur$_{10}$"-ur$_9$"-ur$_8$"-ur$_7$"-ur$_6$"-ur$_5$"-ur$_4$"-ur$_3$"-ur$_2$"-ur$_1$"-großvater,
der aus einem anderen Land kommt\footnote{
  \url{http://iq.lycos.de/exp/show/Myrt/?e4882eb7d690b0c92ed1ab7dcab0e104}. 06.06.2008
}
\ex Das war eine sehr seltene Gabe für Einhörner und das letze Einhorn mit dieser Fähigkeit war
Moonlight Shadow, Cystals Ur$_{18}$"-ur$_{17}$"-ur$_{16}$"-ur$_{15}$"-ur$_{14}$"-ur$_{13}$"-ur$_{12}$"-ur$_{11}$"-ur$_{10}$"-ur$_{9}$"-ur$_{8}$"-ur$_{7}$"-ur$_{6}$"-ur$_{5}$"-ur$_{4}$"-ur$_{3}$"-ur$_{2}$"-ur$_{1}$"-großvater.\footnote{
\url{http://www.forenfuchs.de/archiv/1/56000/54760/mein-leben-bevor-ich-hierher-kam-F_5299-40.html}.
06.06.2008
}
\zl

\is{Morphologie!Derivation|)}%
\is{Default|)}%
\is{Vererbung|)}%
\is{Lexikon|)}%
\is{Typhierarchie|)}

\begin{comment}
\subsection{Typen vs.\ Makros}
\is{Makro|(}

Im Kapitel~\ref{sec-index} auf Seite~\pageref{abkuerzungen-sem} haben wir Abkürzungen eingeführt,
die dann in den nachfolgenden Kapiteln verwendet wurden. Auch für das explizite
Aufschreiben von Theorien ist die Verwendung solcher Abkürzungen wichtig, ja sie macht
ein effektives Arbeiten erst möglich. In vielen Computersystem zur Sprachverarbeitung
gibt es deshalb entsprechende Konstrukte \citep{KS93a,DISCO94,BKNS99a-ed}. 
Solche Abkürzungen werden auch Makros genannt. Für die Definition von Makros 
kann man auch Makros benutzen, so dass man Makrohierarchien aufbauen kann, die den
bereits vorgestellten Typhierarchien entsprechen. Während manche Systeme Makros als
zusätzliches Beschreibungsmittel neben Typen zulassen (Page, Trale), erlaubt das LKB"=System
\citep{Copestake2002a} die Verwendung von Makros nicht. In frühren Version
des Systems konnten Makros wie in TDL \citep{KS93a} definiert verwendet werden, diese
Funktionalität wurde aber bewußt entfernt. Die Begründung dafür ist,
dass Makros keinen Status in der linguistischen Theorie haben. Das ist richtig, und
sie werden deshalb in linguistischen Aufsätzen auch nicht zum Gegenstand von Abhandlungen
gemacht. Allerdings ist es ohne Abkürzungen nicht möglich, sich auf Merkmalsbündel zu beziehen,
es sei denn, man führt entsprechende Typen ein.

\ea
\begin{tabular}[t]{@{}ll@{}}
NP$_{[3,sg,fem]}$     & \onems[sign]{ synsem \onems[synsem]{ cat \ms{ head & \type{noun} \\
                                                                      comps & \liste{} \\
                                                                    } \\
                                                             cont$|$ind \ms{ per & 3 \\
                                                                             num & sg \\
                                                                             gen & fem \\
                                                                           } \\
                                                            }\\
                                    }\\
\end{tabular}
\z


\is{Makro|)}
\end{comment}

\questions{
\begin{enumerate}
\item Welche Möglichkeiten gibt es in HPSG"=Theorien, Generalisierungen in bezug
      auf das Lexikon zu erfassen?
\end{enumerate}
}

\exercises{
\begin{enumerate}
\item\label{ue-result}
Schreiben Sie eine Lexikonregel, die für Adjektivstämme wie den
in (\mex{1}a) einen Lexikoneintrag für die attributive%\NOTE{FB: Attributiv erklären?} 
Verwendung wie in (\mex{1}b)
lizenziert.
\eal
\ex 
\begin{tabular}[t]{@{}l@{}}
\stem{reif}:\\
\ms{
cat & \ms{ head   & \ms[adj]{
                    mod & none\\
                    }\\
           comps & \liste{ NP\ind{1} }\\
         }\\
cont & \ms[reif]{
       thema & \ibox{1}\\
       }\\
}\end{tabular}
\ex \begin{tabular}[t]{@{}l@{}}
\emph{reifes}:\\
\ms{
cat & \ms{ head   & \ms[adj]{
                    cas & nom $\vee$ acc\\
                    mod & \textrm{$\overline{\mbox{\textrm{N}}}$:} \ms{ ind   & \ibox{1} \\
                                                                restr & \ibox{2} \\
                                                              } \\
                    }\\
           comps & \liste{ }\\
         }\\
cont & \ms{ ind & \ibox{1} \ms{ per & 3 \\
                                num & sg\\
                                gen & neu\\
                              } \\
            restr & \liste{ \ms[reif]{ 
                             theme & \ibox{1} \\ 
                          }} $\oplus$ \ibox{2}  \\
              } \\
}\end{tabular}
\zl
Die \phon-, Kasus-, Numerus- und Genus"=Werte können dabei unberücksichtigt bleiben. Wichtig ist, dass die
Regel für alle Adjektivstämme funktioniert, also \zb auch für \stem{groß}/""\emph{großem}.

\item Überlegen Sie, ob Ihre Lexikonregel auch für Adjektive wie \emph{stolz} in (\mex{1})
  funktioniert.
\ea
der auf seinen Sohn stolze Mann
\z
Sollte das nicht der Fall sein, formulieren Sie Ihre Regel um.

\item Laden Sie die zu diesem Kapitel gehörende Grammatik von der Grammix"=CD
(siehe Übung~\ref{uebung-grammix-kapitel4} auf Seite~\pageref{uebung-grammix-kapitel4}).

Oben in der Menüleiste des Fensters, in dem die Grammatik geladen wird, gibt es einen Menüpunkt \texttt{Trale}.
Gehen Sie zum  Unterpunkt \texttt{Draw|""Hierarchy}. 
In der Statusleiste unten im Fenster erscheint der Text: \emph{Enter type to display [bot]:}.
Geben Sie dort \emph{word} bzw.\ \emph{phrase} ein, um sich die Typhierarchie unter
\type{word} bzw.\ \type{phrase} anzusehen. Die Beschränkungen, die zu den Typen gehören, finden
Sie in der Datei \texttt{le\_macros.pl}, die Sie über das Menü \texttt{File|Open File} öffnen können.
Die genaue Syntax der Typbeschränkungen ist im Trale"= bzw.\ ALE"=Manual beschrieben.\footnote{
\url{http://www.ale.cs.toronto.edu/docs/}. 24.05.2024.%
}
\end{enumerate}
}


%\section*{Literaturhinweise}

\furtherreading{
Zu Lexikonregeln in den verschiedenen Grammatikframeworks siehe \citew{Jackendoff75a},
\citew{Williams81a}, \citew{Bresnan82a}, \citew*{SURT83a}, \citew*{FPW85a}, \citew{Flickinger87},
\citew{CB92a} und \citew{Meurers2000b}. Einen guten Überblick über das Lexikon in der HPSG bietet
\citew{Meurers2001a}. Meurers geht auch auf den formalen Hintergrund der Lexikonregeln ein.
}

