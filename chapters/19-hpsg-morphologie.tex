%% -*- coding:utf-8 -*-
%%%%%%%%%%%%%%%%%%%%%%%%%%%%%%%%%%%%%%%%%%%%%%%%%%%%%%%%%
%%   $RCSfile: hpsg-morphologie.tex,v $
%%  $Revision: 1.13 $
%%      $Date: 2008/09/30 09:14:41 $
%%     Author: Stefan Mueller (CL Uni-Bremen)
%%    Purpose: 
%%   Language: LaTeX
%%%%%%%%%%%%%%%%%%%%%%%%%%%%%%%%%%%%%%%%%%%%%%%%%%%%%%%%%



\chapter{Morphologie}
\label{chap-morphologie}

Die\is{Morphologie|(} Morphologie ist eine eigenständige Teildisziplin der Linguistik mit vielen interessanten
Fragen, die hier nicht einmal gestreift werden können. Im folgenden Kapitel beschränke ich
mich also nur auf einige Grundprobleme. Ich werde die bereits im Kapitel~\ref{sec-lr}
erwähnten Lexikonregeln vorstellen und den alternativen Ansatz, der Affixe als eigenständige
Morpheme annimmt, diskutieren.


\section{Die Phänomene}

In der Morphologie beschäftigt man sich mit Morphen\is{Morph}. Ein \emph{Morph} ist
die kleinste bedeutungstragende Einheit. Haben zwei oder mehr Morphe dieselbe Bedeutung
bei verschiedener Verteilung, spricht man von \emph{Allomorphen}.\is{Allomorph} Eine entsprechende Gruppe
von Allomorphen mit derselben Bedeutung nennt man \emph{Morphem}. 
%% Es gibt freie Morpheme
%% wie \emph{Buch} und \emph{Stuhl}, die ohne

In diesem Kapitel sollen Flexion (Abschnitt~\ref{sec-phen-flexion})
und Derivation (Abschnitt~\ref{sec-phen-derivation}) behandelt werden.

\subsection{Flexion}
\label{sec-phen-flexion}


In\is{Flexion|(} (\mex{1}) liegt eine bestimmte Form der Flexion vor: die Deklination\is{Deklination}.
\eal
\ex Hund\label{bsp-hund-nom}
\ex Hundes\label{bsp-hund-gen}
\ex Hunde
\ex Hunden
\zl
Der Stamm\is{Stamm} \emph{Hund} wird mit bestimmten Endungen (\emph{Suffixen}\is{Suffix})
kombiniert. Welche Endung für den Plural verwendet werden muss, hängt von der Flexionsklasse
des Nomens ab. So kann neben der Endung \suffix{e}, wie sie bei (\mex{0}c) zu beobachten ist, auch
\suffix{en}, \suffix{n}, \suffix{s}, $\varnothing$ den Plural markieren (\emph{Betten},
\emph{Katzen}, \emph{Omas}, \emph{Himmel}). Diese verschiedenen Realisierungsmuster der Pluralendung
faßt man unter dem Begriff Pluralmorphem zusammen. Die verschiedenen Realisierungsformen eines
Morphems\is{Morphem} werden Allomorphe genannt. \suffix{e}, \suffix{en}, \suffix{n}, \suffix{s} und
$\varnothing$ sind also die Allomorphe des Pluralmorphems. $\varnothing$ steht für das Null(allo)morph.
%% Neben Morphemen wie dem Pluralmorphem, das nur gebunden -- also in Kombination mit einem Wortstamm
%% wie \emph{Hund} vorkommen kann -- gibt es auch noch so genannte freie Morpheme. Das sind Morpheme,
%% die nicht mehr mit anderen Morphemen kombiniert werden müssen, sondern gleich so verwendet werden können.

Wie (\mex{1}) zeigt, muss bei einigen Nomina bei der Pluralbildung zusätzlich noch eine Stammumlautung erfolgen:
\eal
\ex Mann\label{bsp-mann-nom}
\ex Mannes\label{bsp-mann-gen}
\ex Männer
\ex Männern
\zl


\noindent
Außer der Flexion der Nomina ist für den in diesem Buch beschriebenen Sprachausschnitt die 
Flexion der Verben (Konjugation\is{Konjugation}) relevant. (\mex{1}) zeigt die finiten Formen für
den Stamm \stem{lach}:
\begin{multicols}{2}
\eal
\label{verb-konjugation-praesens}
\ex Ich lache.
\ex Du  lachst.
\ex Er  lacht.
\ex Wir lachen.
\ex Ihr lacht.
\ex Sie lachen.
\zl
\eal
\label{verb-konjugation-praeteritum}
\ex Ich lachte.
\ex Du  lachtest.
\ex Er  lachte.
\ex Wir lachten.
\ex Ihr lachtet.
\ex Sie lachten.
\zl
\end{multicols}

\noindent
Betrachtet man ein anderes Verb wie \zb \stem{lieb}, stellt man fest, dass die
Endungen gleich sind:
\begin{multicols}{2}
\eal
\ex Ich liebe.
\ex Du  liebst.
\ex Er  liebt.
\ex Wir lieben.
\ex Ihr liebt.
\ex Sie lieben.
\zl
\eal
\ex Ich liebte.
\ex Du  liebtest.
\ex Er  liebte.
\ex Wir liebten.
\ex Ihr liebtet.
\ex Sie liebten.
\zl
\end{multicols}
\noindent
Es liegt also nahe, \zb das Suffix \suffix{st} als Markierung der zweiten Person Singular
anzusehen. Es gibt jedoch auch andere Formen, wie die in (\mex{1}) für das Verb \stem{red}:
\begin{multicols}{2}
\eal
\ex Ich rede.
\ex Du  redest.
\ex Er  redet.
\ex Wir reden.
\ex Ihr redet.
\ex Sie reden.
\zl
\eal
\ex Ich redete.
\ex Du  redetest.
\ex Er  redete.
\ex Wir redeten.
\ex Ihr redetet.
\ex Sie redeten.
\zl
\end{multicols}
\noindent
Im Vergleich zu \stem{lieb} und \stem{lach} steht bei einigen Formen in (\mex{-1})
und (\mex{0}) noch ein zusätzliches \emph{e}. Man könnte jetzt annehmen, dass \stem{red}
einfach ein Verb ist, dass in der zweiten Person Singular die Endung \suffix{est} haben
muss. Es gibt allerdings eine allgemeinere Gesetzmäßigkeit, auf die sich das Auf"|treten
des \emph{e} zurückführen läßt: Das \emph{e} wird eingeführt, wenn an der Morphemgrenze
(der Stelle an der zwei Morpheme zusammenstoßen) ein \emph{d} oder ein \emph{t} auf ein
\emph{s} oder ein \emph{t} treffen. Die entsprechende Regel heißt \emph{e}-Epenthese\is{Epenthese}
und wird wie folgt aufgeschrieben:\footnote{
  Zum Beispiel \citet[\page183]{Eisenberg98a} weist darauf hin, dass diese Regel nicht
  obligatorisch ist, da Formen wie \emph{du botst} und \emph{du rietst} möglich sind.%
}
% du rietst
% du botst
\ea
\label{Regel-e-einfuegung}
+:e $\leftrightarrow$ \{d, t\} \_ \{s, t\}
\z
In (\mex{0}) steht das `+' für die Morphemgrenze. Die Regel besagt, dass die Morphemgrenze
durch ein \emph{e} ersetzt wird, wenn eins der Elemente der Menge \{d, t\} links der Morphemgrenze
steht und eins der Elemente der Menge \{s, t\} rechts der Morphemgrenze.
Eine alternative Schreibweise ist:
\ea
+ $\to$ e / \{d, t\} \_ \{s, t\}
\z
Dies ähnelt den Phrasenstrukturregeln aus Kapitel~\ref{sec-psg}, wobei die Information hinter dem `/'
den Anwendungskontext für die Regel angibt. Verwendet man diese morphophonologischen Regeln,
reicht es anzunehmen, dass \emph{st} die zweite Person Singular markiert. Die Regel erklärt auch,
warum in der dritten Person Singular ein \emph{e} zwischen \stem{red} und \suffix{t} steht,
und es wird plötzlich möglich, die Präteritummarkierung allein dem \infix{t} zuzuordnen 
(\emph{lieb + e} vs.\ \emph{lieb + t + e}): Die Person-\is{Person} und
Numerus"=Markierung\is{Numerus} ist nämlich im Präsens und im Präteritum gleich. Die Formen der zweiten Person unterscheiden sich
lediglich in bezug auf ein vorhandenes \emph{e}: \emph{lieb+st} vs.\ \emph{lieb+t+e+st}. Dieses
eingefügte \emph{e} wird durch die Regel in (\mex{-1}) erklärt, da das Präteritumsmorphem
auf \emph{t} endet und die Person"=Numerus"=Markierung mit \emph{s} beginnt.\NOTE{liebt+t = liebte erklären}

Neben den bisher diskutierten Fällen gibt es aber noch Verben wie \stem{geb}, die unregelmäßige
Flexionsformen haben:
\begin{multicols}{2}
\eal
\ex Ich gebe.   
\ex Du  gibst.  
\ex Er  gibt.   
\ex Wir geben.  
\ex Ihr gebt.   
\ex Sie geben. 
\zl
\eal
\ex Ich gab.
\ex Du  gabst.
\ex Er  gab.
\ex Wir gaben.
\ex Ihr gabt.
\ex Wir gaben. 
\zl
\end{multicols}
\noindent
Solche Verben werden \emph{starke Verben}\is{Verb!starkes} genannt,
Verben wie \stem{red} werden \emph{schwache Verben}\is{Verb!schwaches} genannt.
Bei den starken Verben wird das Präteritum nicht durch das Einfügen eines \infix{t} gebildet,
sondern durch die Verwendung einer abgelauteten\is{Ablaut} Stammform.
\is{Flexion|)}


\subsection{Derivation}
\label{sec-phen-derivation}

Bei\is{Derivation|(} der Flexion ändert sich die Wortart des flektierten Elements nicht, 
bei der Derivation dagegen ändert sich die Wortart -- und damit
gewöhnlich auch die Bedeutung -- (\mex{1}a,b) bzw.\ (\mex{1}c,d) 
oder die Wortart bleibt gleich (\mex{1}b,c) und die Bedeutung wird geändert.
\eal
\ex \stem{schlag} (Verb)
\ex schlagbar (Adjektiv)
\ex unschlagbar (Adjektiv)
\ex Unschlagbarkeit (Nomen)
\zl
Wie (\mex{1}) zeigt, kann sich bei Derivation auch die Art der Argumente ändern:
\eal
\ex X\sub{akk} auf Y streuen 
\ex Y\sub{akk} mit X bestreuen
\zl
Betrachtet man (\ref{bsp-hund-nom}) und (\ref{bsp-hund-gen}), so ergibt sich bei den beiden
Flexionsformen kein Unterschied in der Bedeutung, es wird lediglich der Kasus des Nomens
markiert. Bei den Pluralformen sieht es schon anders aus: Hier könnte man vermuten,
dass die Pluralflexion gemeinsam mit einer Pluralsemantik auf"|tritt. In Abhängigkeit von
Annahmen, die man in bezug auf die Syntax der Nominalphrase macht, ist es jedoch
nicht zwingend, die Pluralsemantik den Nomen zuzuordnen. Es könnte auch hier -- wie bei der Kasusmarkierung --
eine rein formale Markierung vorliegen, und die Pluralsemantik würde dann durch den Plural"=Determinator beigesteuert.
Betrachtet man die Konjugationsmuster in (\ref{verb-konjugation-praesens}) und (\ref{verb-konjugation-praeteritum}),
sieht man, dass es hier einen Unterschied in der zeitlichen Lokalisierung des Ereignisses gibt.
Da man diese auch negieren kann, wie (\mex{1}) zeigt, muss sich dieser Zeitbezug auch in der
logischen Repräsentation des Verbs widerspiegeln.\NOTE{FB: Das versteht man nicht ohne weiteres. Was
  hat die Negation damit zu tun?}
\ea
Sie lacht nicht, aber sie wird lachen.
\z

\noindent
Die Linguisten sind sich nicht einig, wie sie Derivation und Flexion genau definieren.
So zählen \citet*[\page263--264]{SWB2003a} die Bildung des Partizips Präsens und Präteritum im Englischen
zur Derivation, weil diese Formen im Französischen noch flektiert werden müssen.
\citet*[\page313]{SWB2003a} behandeln das Passiv\is{Passiv} mittels einer valenzverändernden Lexikonregel.
Alle valenzverändernden Regeln ordnen sie der Derivation zu, weshalb bei ihnen Passivierung
zur Derivation gezählt wird.
\is{Derivation|)}

\section{Die Analyse}

In den beiden folgenden Abschnitten zeige ich, wie Flexion und Derivation mittels lexikalischer Regeln
analysiert werden können. Der Abschnitt~\ref{morph-pv} zeigt, wie die Analyse der Partikelverben, die
im Kapitel~\ref{chap-partikel} vorgestellt wurde, mit der Analyse von Flexion und Derivation zusammenwirkt.

\subsection{Flexion}
\label{sec-morph-flex-anal}
\label{sec-inflection-hpsg}

Im\is{Flexion|(} Kapitel~\ref{chap-lexikon} haben wir bereits eine Lexikonregel gesehen, die einen Verbstamm zu einer
flektierten Form in Verbindung setzt (S.\,\pageref{passive-lr-mit-phon}). Wie der \phonw aber genau berechnet
wird, wurde bisher nicht erklärt. Betrachtet man die Präsensformen in (\ref{verb-konjugation-praesens}),
so liegt \zb für die zweite Person Singular die folgende Spezifikation
nahe:\is{Kongruenz|(}\NOTE{FB: list-of... muss man erklären}

\eas
\label{lr-verbal-inflection}
Lexikonregel für die zweite Person Singular Präsens (vorläufig):\\
\onems[fin-verb-infl-lr]{
phon   \textit{f\textrm{(\,\ibox{1}, \phonliste{ st })}}\\
synsem$|$loc \onems{ cat   \ibox{2} \onems{ head$|$vform  \type{fin} \\
%                                                        subj  & \eliste \\
                                                  comps  \type{list\_of\_not\_np\_str} $\oplus$ \liste{ NP[\str]$_{2,sg}$ } $\oplus$ \etag\\
                                                } \\
                             cont  \ms[present]{
                                      soa & \ibox{4}\\
%                                       arg2 & \ms[t-overlap]{
%                                              arg1 & \ibox{4}\\
%                                              arg2 & now\\
%                                              }\\
                                      }\\
                       }\\
lex-dtr  \onems[stem]{
           phon   \ibox{1} \\
           synsem$|$loc \ms{ cat  & \ibox{2} \ms{ head  & verb\\
                                                }\\
                             cont & \ibox{4} \\
                         }\\
           }\\
}\is{Lexikonregel!Verbflexion}
\zs
Diese Lexikonregel bildet einen Verbstamm (unter \textsc{lex-dtr}) auf ein Wort ab. Der Typ
\type{fin-verb-infl-lr} ist ein Untertyp von \type{word}. Der \phonw des Ausgabezeichens wird durch
die Funktion \textit{f} berechnet. Der \phonw ist entweder die direkte Verknüpfung des Eingabestamms
mit der Endung \suffix{st} (\emph{lachst}) oder, wenn der Eingabestamm auf \emph{d} oder \emph{t}
endet (siehe Regel (\ref{Regel-e-einfuegung}) auf Seite~\pageref{Regel-e-einfuegung}), die
Verknüpfung des Eingabestamms mit einem -\emph{e}- und dem \suffix{st} (\emph{redest}).  

Eine finite Form kongruiert mit dem Subjekt, wenn es eins gibt. Im Kapitel~\ref{chap-kongruenz}
wurde eine Analyse der Kongruenz vorgestellt, die auf die Index"=Merkmale des Subjekts Bezug nimmt.
In der Lexikonregel in (\mex{0}) wird verlangt, dass das erste Element in der \compsl des Verbs mit
strukturellem Kasus einen Index mit der Person \emph{2} und dem Numerus \emph{sg} haben muss.
Somit ist die Kongruenz zwischen Subjekt und einem Verb wie \emph{lachst} gewährleistet. Die Regel
ist auch so formuliert, dass sie nicht auf das subjektlose Verb \emph{grauen} angewendet werden kann,
denn \emph{grauen} hat kein Argument mit strukturellem Kasus. Die Unterteilung der \compsl in einen
Anfang, der Elemente enthalten kann, die keine Nominalphrasen mit strukturellem Kasus sind, ist
notwendig, um die im Kapitel~\ref{sec-remote-passive-hpsg} diskutierten Beispiele des Fernpassivs\is{Passiv!Fern-} mit
Objektkontrollverben\is{Verb!Objektkontroll-} erfassen zu können.%
\is{Kongruenz|)}

Der \catw des Stammes und der \catw des Wortes sind identisch \iboxb{2}, aber der semantische
Beitrag des Stammes wird unter die Tempus"=Information,\is{Tempus} die vom Suffix beigesteuert wird,
eingebettet.\footnote{ 
  Die Tempus"=Repräsentation ist eine Vereinfachung.%
}

%%\begin{comment}
%% In der Lexikonregel in (\ref{lr-verbal-inflection}) ist nur der Fall für die zweite Person Singular
%% erfaßt. Es stellt sich die Frage, wie die anderen Flexionsformen abgeleitet werden. Eine Möglichkeit besteht
%% darin, parallele Regeln für alle Suffixe zu formulieren. Alternativ kann man die Information aber auch
%% so organisieren, dass alle Flexionsformen (das gesamte Paradigma\is{Paradigma}) am selben Ort spezifiziert
%% sind. Das kann man \zb mit der folgenden Typbeschränkung erreichen:\footnote{
%%   In diesen Repräsentationen ist der Modus\is{Modus} (Indikativ\is{Indikativ}, Konjunktiv\is{Konjunktiv})
%%   der Übersichtlichkeit halber nicht berücksichtigt. In einem vollständigen Flexionsparadigma müssen die entsprechenden Formen
%%   natürlich repräsentiert sein.%
%% }
%% \ea
%% \type{fin-verb-infl-suffix} \impl\\
%% \ms{ phon   & \phonliste{ e }\\
%%                                        tense  & present\\
%%                                        per    & 1\\
%%                                        num    & sg\\
%%                                      } $\vee$
%% \ms{ phon   & \phonliste{ st }\\
%%                                        tense  & present\\
%%                                        per    & 2\\
%%                                        num    & sg\\
%%                                      } $\vee$
%% \ms{ phon   & \phonliste{ t }\\
%%                                        tense  & present\\
%%                                        per    & 3\\
%%                                        num    & sg\\
%%                                      } $\vee$    \ldots
%% \z
%%
%% Die Flexionslexikonregel kann dann wie folgt allgemein formuliert werden:
%% \eas
%% \label{lr-verbal-inflection}
%% Lexikonregel für Verbflexion (vorläufig):\\
%% \ms[fin-verb-infl-lr]{
%% phon   & f\textrm{(\,\ibox{1}, \ibox{2}\,)}\\
%% synsem & \onems{ loc$|$cat   \ms{ head & \ms[verb]{ vform & fin \\
%% %                                                     subj  & \eliste \\
%%                                                        } \\ 
%%                                        comps & \textrm{\ibox{3} (\liste{ NP[\str]$_{\ibox{4},\ibox{5}}$ } $\oplus$ \etag)}\\
%%                                      } \\
%% %%                              cont & \ms[present]{
%% %%                                       soa & \ibox{4}\\
%% %% %                                       arg2 & \ms[t-overlap]{
%% %% %                                              arg1 & \ibox{4}\\
%% %% %                                              arg2 & now\\
%% %% %                                              }\\
%% %%                                       }\\
%%                }\\
%% lex-dtr & \onems[stem]{
%%            phon   \ibox{1} \\
%%            infl \ms[fin-verb-infl-suffix]{
%%                 phon & \ibox{2}\\
%%                 per  & \ibox{4}\\
%%                 num  & \ibox{5}\\
%%                 }\\
%%            synsem$|$loc$|$cat  \ms{ head  & \type{verb}\\
%%                                         comps & \ibox{3}\\
%%                                       }\\
%% %                             cont & \ibox{4} \\
%%            }\\
%% }\is{Lexikonregel!Verbflexion}
%% \zs
%% Durch die Typbeschränkung in (\mex{-1}) ist sichergestellt, dass die Werte der Merkmale unter \textsc{infl}
%% Flexionsaffixen\is{Affix} entsprechen. Der \phonw des Stammes (\ibox{1} in (\mex{0}))
%% wird mit dem \phonw eines Affixes aus der Disjunktion in (\mex{-1}) (\ibox{2} in (\mex{0})) verknüpft.
%% Die entsprechenden Person- und Numerus"=Werte (\iboxt{3} und \iboxt{4}) werden mit den Index"=Werten des Subjekts identifiziert.

%% Die Behandlung der Kongruenz in (\mex{0}) ist jedoch noch nicht für alle bisher diskutierten
%% Fällen verwendbar. Die flektierte Form \emph{graut} kann man mit (\mex{0}) nicht ableiten, da \emph{graut}
%% keine NP mit strukturellem Kasus selegiert. Genauso sind unpersönliche Passive (\mex{1}a) und Anhebungsverben, die Verben
%% ohne Subjekt einbetten (\mex{1}b), nicht analysierbar. Auch der Satz (\ref{bsp-auskosten-fernpassiv-haider}) aus dem Kapitel~\ref{chap-passiv}
%% -- hier als (\mex{1}c) wiederholt -- kann mit der Regel in (\mex{0}) nicht analysiert werden. Das liegt daran, dass
%% die Valenzliste von \emph{wurde} an erster Stelle eine NP mit lexikalischem Dativ enthält (vergleiche Seite~\pageref{abb-remote-pass-da-hm-erlauben}).
%% \eal
%% \ex weil getanzt wurde
%% \ex weil ihm vor der Prüfung zu grauen schien
%% \ex\iw{auskosten}
%% Der Erfolg        wurde uns      nicht auszukosten erlaubt.\footnote{
%%         \citew[\page110]{Haider86c}.%
%% }
%% \label{bsp-auskosten-fernpassiv-haider}
%% \zl
%%
%% Damit die Regel allgemein verwendbar ist, muss man sie wie folgt spezifizieren:
%% \eas
%% \label{lr-verbal-inflection}
%% Lexikonregel für Verbflexion:\\
%% \ms[fin-verb-infl-lr]{
%% phon   & f\textrm{(\,\ibox{1}, \ibox{2}\,)}\\
%% synsem & \onems{ loc \ms{ cat  & \ms{ head & \ms[verb]{ vform & fin \\
%% %                                                     subj  & \eliste \\
%%                                                        } \\ 
%%                                       comps & \ibox{3} \\
%%                                      } \\
%%                           cont & \ms[present]{
%%                                       soa & \ibox{4}\\
%% %                                       arg2 & \ms[t-overlap]{
%% %                                              arg1 & \ibox{4}\\
%% %                                              arg2 & now\\
%% %                                              }\\
%%                                       }\\
%%                       }\\
%%                }\\
%% lex-dtr & \onems[stem]{
%%            phon   \ibox{1} \\
%%            infl \ms[fin-verb-infl-suffix]{
%%                 phon & \ibox{2}\\
%%                 per  & \ibox{4}\\
%%                 num  & \ibox{5}\\
%%                 }\\
%%            synsem$|$loc \ms{ cat  & \ms{ head  & \type{verb}\\
%%                                         comps & \ibox{3}\\
%%                                       }\\
%%                              cont & \ibox{4} \\
%%                            }\\
%%            }\\
%% } $\wedge$ \\
%% subject\_verb\_agreement(\ibox{3},\ibox{4},\ibox{5})\is{Lexikonregel!Verbflexion}
%% \zs
%% Dabei ist subject\_verb\_agreement eine relationale Beschränkung, die erfüllt ist,
%% wenn \ibox{4} = 3 ist \ibox{5} = sg und \ibox{3} keine NP mit strukturellem Kasus enthält
%% oder wenn die erste NP in \ibox{3}, die strukturellen Kasus hat, den \perw \ibox{4} und
%% den \numw \ibox{5} hat.
%% \end{comment}
%% \begin{comment}
%% \subsubsection{Irreguläre Formen}
%%
%% Starke Verben
%%
%% Kopula
%%
%% Lakoff70a
%%
%% corrode, corrosion, corrosive
%% *agress, agression, agressive
%% incite, *incition, *incitive
%%
%% \subsubsection{Fehlende Formen}
%%
%% Im vorigen Abschnitt wurden die unregelmäßigen Formen behandelt. Mitunter kommt es auch vor,
%% dass Formen, die normalerweise zu einem Paradigma gehören, nicht auf"|treten. Beispiele sind
%% Pluraliatanta\is{Pluraliatantum} wie \emph{Leute} und das Futur"=Hilfsverb \emph{werden}, 
%% das nur finit auf"|tritt und auch nicht über Präteritumsformen verfügt.
%%
%% Solche Fälle kann man leicht erfassen, indem man die Information, die Normalerweise im Stamm
%% nicht spezifiziert ist, sondern von den jeweiligen Affixen beigesteuert wird, bereits im Stammeintrag
%% spezifiziert. Dadurch dass man \emph{Leute} mit einem \numw \type{pl} ins Lexikon schreibt, kann
%% kein Singular"=Affix mit dem Nomen kombiniert werden. Nur ein Plural"=Affix wie \zb \suffix{en}
%% ist mit dem Stamm \stem{Leut} kombinierbar, so dass dann die Formen \emph{Leute} und \emph{Leuten}
%% gebildet werden können.
%%
%%\end{comment}
\is{Flexion|)}



%% \ea
%% \ms[derived-stem]{
%% head-dtr & derivational\_affix\\
%% non-head-dtrs & \liste{ stem-or-word }\\
%% }
%% \z

%% \ea
%% \ms[complex-word]{
%% head-dtr & inflectional\_affix\\
%% non-head-dtrs & \liste{ stem }\\
%% }
%% \z

%% \ea
%% \ms[derivation-lexical-rule]{
%% lex-dtr & stem-or-word\\
%% }
%% \z


\subsection{Derivation}
\label{sec-bard}

Die\is{Derivation|(} Behandlung der Derivation soll am Beispiel der \bard erklärt werden. Neben einigen idiosynkratischen
\emph{bar}"=Adjektiven wie \zb \emph{brennbar} gibt es produktiv gebildete \emph{bar}"=Adjektive.
Bei diesen Adjektiven wurde ein Verbstamm eines transitiven Verbs mit dem Suffix \suffix{bar} kombiniert.
\eal
\ex Er löst das Problem.
\ex Das Problem ist lösbar.
\ex Das Problem kann gelöst werden.
\zl
Wie (\mex{0}c) zeigt, ist die \bard äquivalent zu einer Einbettung eines Passivs unter ein Modalverb. Neben
der Modalbedeutung \emph{können} sind auch andere Modalbedeutungen wie \zb \emph{sollen} oder \emph{müssen}
möglich, wie durch Bildungen wie \emph{zahlbar} belegt ist. Für die Bedeutungsrepräsentation der \bard wird
deshalb oft ein allgemeiner Modaloperator\is{Modaloperator} angenommen.

Man sagt, dass ein Muster produktiv\is{Produktivität} ist, wenn es auch auf neu in eine Sprache
aufgenommene Wörter anwendbar ist. So kann man \zb die Wörter \emph{mailbar} und \emph{faxbar}
bilden. Die dazugehörigen Verben \emph{mailen} und \emph{faxen} sind erst im vorigen Jahrhundert in
die deutsche Sprache aufgenommen worden und sind also relativ neue Wörter.

Die produktive Lexikonregel zur Bildung von \emph{bar}"=Adjektiven zeigt (\mex{1}) (siehe auch \citew{Riehemann98a}
zu einer ähnlichen Regel).
\begin{figure}
\eas
\label{lr-bar-adj}
Lexikonregel für die Derivation von Adjektiven mit \bars:\\
\resizebox{\linewidth}{!}{%
\ms[reg-bar-adj-stem]{
phon   & \ibox{1} $\oplus$ \phonliste{ bar }\\
synsem & \onems{ loc  \ms{ cat  & \ms{ head   & \ms[adj]{ 
                                                 subj & \sliste{ \ibox{2} }\\
                                                 }\\
                                        comps & \ibox{3}\\
                                      } \\
                            cont & \ms[modal-op]{
                                    soa & \ibox{4}\\
                                   }\\
                         }\\
               }\\
lex-dtr & \onems[stem]{
           phon   \ibox{1} \\
           synsem$|$loc \ms{ cat  & \ms{ head   & \ms[verb]{
                                                  da & \sliste{ \ibox{5} }\\
                                                  }\\
                                         comps & \sliste{ \ibox{5} NP[\str], \ibox{2} NP[\str] } $\oplus$ \ibox{3}\\
                                       } \\
                             cont & \ibox{4} \\
                          }\\
           }\\
}\is{Lexikonregel!bar-Derivation@\emph{-bar}-Derivation}%
}
\zs
\vspace{-\baselineskip}\end{figure}
Eingabe der Regel ist ein transitives Verb, \dash ein Verb mit zwei Nominalphrasen mit strukturellem Kasus.
Die erste NP -- die auch das designierte Argument sein muss \iboxb{5} -- wird wie bei der Passivierung unterdrückt, 
und die zweite NP \iboxb{2} wird zum Subjekt des durch die Regel lizenzierten Adjektivs.
Der Bedeutungsbeitrag des Verbs \iboxb{4} wird unter einen Modaloperator eingebettet.
\newpage
\noindent
Als Beispiel sei die Lexikonregel auf den Stamm \stem{lös} in (\mex{1})
angewendet. Das Ergebnis der Regelanwendung zeigt (\mex{2}):
\eas
\mbox{\stem{lös}:}\\
\ms{
cat & \ms{ head   & \ms[verb]{
                    da & \sliste{ \ibox{1} }\\
                    }\\
           comps & \liste{ \ibox{1} NP[\str]\ind{2}, NP[\str]\ind{3}} \\
         }\\
cont & \ms[lösen]{ agens & \ibox{2} \\
                   thema & \ibox{3} \\
                 } \\
}
\zs\iw{lösen|uu}

\noindent
\eas
\label{le-fahrbar}
\stem{lösbar}:\\
\ms{
cat & \ms{ head & \ms[adj]{ subj & \sliste{ NP[\type{str}]\ind{3}} \\
                           } \\
           comps & \eliste \\
         }\\
cont & \ms[modal-op]{
       soa & \ms[lösen]{ agens & \etag \\
                    thema & \ibox{3} \\
                  } \\
       }\\
}\iw{fahrbar|uu}
\zs
Die Agens"=Rolle von \emph{lösen} ist nicht an ein Argument des
Adjektivs gelinkt. Das wird durch das leere Kästchen in (\mex{0})
repräsentiert.

Das Ergebnis der Regelanwendung ist ein Adjektivstamm. Dieser muss noch flektiert werden, 
bevor er in der Syntax verwendet werden kann. Je nach Flexion kann das Adjektiv dann
prädikativ oder attributiv gebraucht werden.

Die \bard ist ein Beispiel für die Ableitung eines Stammes aus einem anderen Stamm, aber auch Wörter
können Eingaben für eine Ableitung sein. Ein Beispiel sind die pränominalen Partizipien: In (\mex{1}a) handelt es sich
beim Partizip um ein flektiertes Verb. Die Adjektivbildungsregel, die bereits im Kapitel~\ref{sec-adj-formation}
auf Seite~\pageref{lr-adjective-formation-da-approach} diskutiert wurde, lizenziert einen Adjektivstamm, der dann wieder
flektiert werden muss. In (\mex{1}b) liegt die maskuline Nominativ"=Singular"=Form vor.
\eal
\ex Der Weltmeister wurde geschlagen.
\ex der geschlagene Weltmeister
\zl
\is{Derivation|)}


\subsection{Partikelverben}
\label{morph-pv}

Im folgenden diskutiere ich angebliche Klammerparadoxa, die im Zusammenhang mit der Analyse der
morphologischen Eigenschaften von Partikelverben von vielen Autoren diskutiert wurden. Es handelt sich
dabei um Paradoxa, in denen sowohl morphologische als auch syntaktische und semantische Paradoxa vorzuliegen
scheinen. Würde es sich nur um ein semantisches Paradoxon handeln, so könnte man die von \citet{Egg2004a}
vorgeschlagenen Techniken verwenden, da aber andere Bereiche betroffen sind, ist eine grundlegendere Lösung
des Problems nötig.

Im folgenden sollen drei Fälle betrachtet werden: Der erste Fall stammt aus dem Bereich der
Flexion, die Fälle zwei und drei stammen aus dem Bereich der Derivation.

\citet[\page163]{Bierwisch87a} diskutiert die beiden Analysen für \emph{aufhören}
in Abbildung~\vref{structures-pv}.\footnote{
  Bierwisch nimmt eine morphembasierte Analyse an, ich habe dagegen im vorigen
  Abschnitt eine lexikonregelbasierte Analyse angenommen. Die Analysen sind bis zu einem
  gewissen Grad ineinander übertragbar (siehe Abschnitt~\ref{morphem-vs-lr}).
  Das Bild~\ref{structures-pv}a entspricht der folgenden Struktur:
\vspace{-\baselineskip}
\begin{figure}[H]
\begin{forest}
for tree={font=\footnotesize}
[V
  [P [auf]]
  [V en
    [hör]]]
\end{forest}
%\caption{Struktur mit unärer Regel für Affigierung}
\end{figure}
\vspace{-\baselineskip}
\noindent
Statt einer binären Regel, die ein Affix\is{Affix} mit dem Verbstamm verbindet, wird
im lexikonregelbasierten Ansatz die phonologische Information des Affix innerhalb
der Regel mit der phonologischen Information des Stammes kombiniert.%
}
%
\begin{figure}
a. \begin{forest}
   baseline
   [V
     [P [auf]]
     [V
       [V [hör]]
       [en]]]
\end{forest}
\hspace{2.5cm}b. \begin{forest}
   baseline
   [V
     [V [P [auf]]
        [V [hör]]]
     [en]]
\end{forest}
\caption{Alternative Strukturen für \emph{aufhören}}
\label{structures-pv}
\end{figure}
%
Er argumentiert, dass die Flexionsendung direkt mit dem Stamm verbunden
wird, da sie sensitiv für die Eigenschaften des Stammes ist. Partizipien
wie \emph{aufgehört} machen eine solche Analyse plausibel. Allerdings ist
es so, dass der semantische Beitrag der Flexion sich auf den gesamten Beitrag
des Partikelverbs bezieht und nicht nur auf den Beitrag des Basisverbs.
In \citew{Mueller2002b} habe ich dargelegt, dass diese Beispiele nicht problematisch
sind, da \emph{aufhören} nicht kompositional gebildet ist. Wenn die Partikel
keine Bedeutung beisteuert und die Gesamtbedeutung bereits im Basisverb
enthalten ist, dann entsteht das Problem gar nicht erst. Allerdings gibt
es parallele Fälle mit kompositional gebildeten Partikelverben und für diese
gilt es, das scheinbare Paradoxon aufzulösen. Ich werde im folgenden die
Struktur~\ref{structures-pv}a annehmen und zeigen, dass die im Kapitel~\ref{chap-partikel}
entwickelte Analyse der Partikelverben aufs beste mit der Analyse der
Flexion und Derivation harmoniert.

Für die Struktur~\ref{structures-pv}a sprechen auch die Partizipformen:
\eal
\ex anekeln
\ex angeekelt
\zl
Die Partizipmarkierung \prefix{ge} und \suffix{t} stehen direkt links und rechts
des Verbstammes, die Partikel muss links von \prefix{ge} stehen.
Mit einer Struktur wie der in \ref{structures-pv}b müßte man erklären, wieso
das \emph{ge} zwischen Partikel und Verb steht, da ja bei Annahme dieser
Struktur \emph{aufhör} mit \prefix{ge} und \suffix{t} kombiniert würde.
Auch gibt es eine Regel, die bei der Partizipbildung auf phonologische
Eigenschaften des Verbs Bezug nimmt: Wenn die erste Silbe betont ist,
wird das Partizip mit  \prefix{ge} gebildet (\mex{1}a), ist sie nicht
betont, wird das Präfix \prefix{ge} weggelassen (\mex{1}b).
\eal
\ex ger\'edet, ge\'arbeitet
\ex diskut{\'\i}ert, krak\'eelt
\zl
Für die Anwendung dieser Regel ist es unerheblich, ob das Verb ein
Partikelverb ist oder nicht, einzig die phonologischen Eigenschaften des Verbstamms
sind entscheidend.
\eal
\ex rumgeredet, losgearbeitet
\ex rumdiskutiert, loskrakeelt
\zl
Das wird mit einer Struktur wie der in \ref{structures-pv}a direkt erfaßt.

Das zweite scheinbare Klammerparadoxon ist die \geen.\is{Nominalisierung}
Die \geen ist die einzige diskontinuierliche Nominalderivation im Deutschen.
Sie besteht aus dem Präfix \emph{Ge}- und dem Suffix \suffix{e}. 
Deverbale \emph{Ge}- "~\emph{e}"=Nomina haben die Bedeutung `andauernd/wiederholtes V-en'
und sind mit einer negativen Konnotation verbunden:
\ea
Sein Gepfeife ging ihm auf die Nerven.
\z
Die \geen ist auch bei Partikelverben möglich, und in der Literatur
wurden die beiden Strukturen in Abbildung~\vref{structures-herumgerenne}
vorgeschlagen. 
\begin{figure}
a. \begin{forest}
   baseline
   [N
     [P [herum]]
     [N
       [V [renn]]
       [\gee]]]
\end{forest}
\hspace{2.5cm}b. \begin{forest}
   baseline
   [N
     [V [P [herum]]
        [V [renn]]]
     [\gee]]
\end{forest}
\caption{Alternative Strukturen für \textit{Herumgerenne}}
\label{structures-herumgerenne}
\end{figure}
Die erste Struktur kann ohne weiteres erklären, warum Präfix
und Suffix direkt an den Verbstamm gehen, wohingegen die zweite Struktur
erklären kann, warum das \gee Skopus\is{Skopus} über das gesamte Partikelverb hat.
Die zweite Struktur scheint nötig zu sein, denn \emph{Herumgerenne}
bedeutet  \relation{wiederholt}(\relation{ziellos}(\relation{rennen})) 
und nicht \relation{ziellos}(\relation{wiederholt}(\relation{rennen})).
Siehe zu diesem Punkt \citew[\page106]{Luedeling2001a}. Die letzte Bedeutung
würde man bekommen, wenn man einfach \emph{Gerenne} (\relation{wiederholt}(\relation{rennen}))
mit \emph{herum} (\relation{ziellos}) kombinieren würde.

\noindent
Als drittes Beispiel soll die \bard diskutiert werden: Auch hier scheint
es so zu sein, dass man eigentlich beide Strukturen in Abbildung~\ref{structures-anfahrbar}
braucht: Man möchte, dass das Affix wie bei der \geen direkt mit dem Verbstamm
kombiniert wird, auf der anderen Seite ist aber \bard nur mit transitiven Verben
produktiv, und das Akkusativargument von \emph{anfahren} ist -- wie (\mex{1}) zeigt -- nur durch die
Partikel lizenziert.
\ea
"`Die Kneipen,        Theater  und Geschäfte müssen \emph{anfahrbar} bleiben."'\footnote{
taz, 05.06.1997, S.\,22.% %TAZ-HAMBURG Nr. 5244 Seite 22 vom 05.06.1997
}
\z
\begin{figure}
a. \begin{forest}
   baseline
   [A
     [P [an]]
     [A
       [V [fahr]]
       [bar]]]
\end{forest}
\hspace{2.5cm}b. \begin{forest}
   baseline
   [A
     [V [P [an]]
        [V [fahr]]]
     [bar]]
\end{forest}
\caption{Alternative Strukturen für \emph{anfahrbar}}
\label{structures-anfahrbar}
\end{figure}
Das heißt, dass man sowohl aus Skopusgründen\is{Skopus} als
auch aus Gründen, die mit der Valenz\is{Valenz} von Verben zu tun haben, die
Struktur in Abbildung~\ref{structures-anfahrbar}b zu brauchen scheint.

In der Literatur sind sehr komplexe Verfahren wie \zb Umklammerung\is{Umklammerung} (`rebracketing')
vorgeschlagen worden (\citew[\page165]{Bierwisch87a}; \citew[\page934]{SW94a}; \citew[\page46]{Stiebels96a}), um mit diesem scheinbaren Paradoxon fertig zu werden. Im folgenden
soll gezeigt werden, dass dies nicht nötig ist und dass man aufbauend auf der
im vorigen Kapitel vorgestellten Analyse der Partikelverben die Flexionsdaten
und Derivationsdaten mit einer Struktur erklären kann, in der die Affixe
jeweils direkt mit dem Verbstamm kombiniert werden. Die Lexikonregel für
die Lizenzierung von Partikelverben hat ja nicht die Partikel eingeführt,
sondern nur eine entsprechende Valenzstelle für die Partikel und eventuell
weitere Argumente bereitgestellt. Die Ausgabe dieser Regel kann dann Eingabe
für die Flexions- bzw.\ Derivationsregeln sein.

Wendet man die Lexikonregel in (\ref{lr-verbal-inflection}) auf das Simplexverb
\stem{lach} in (\ref{le-lachen}) auf S.~\pageref{le-lachen} an, so erhält man 
ein Objekt mit dem \synsemw in (\mex{1}).

\eas
\label{le-lachst}
\mbox{\emph{lachst}:}\\
\ms{
cat & \ms{ head & \ms[verb]{ vform & fin\\
%                             subj  & \eliste \\
                           } \\
           comps & \liste{NP[\type{str}]$_{\ibox{1}~2, sg}$} \\
         }\\
cont & \ms[present]{
        soa & \ms[lachen]{ agens & \ibox{1} \\
                          } \\
        }\\
}\iw{lachen}
\zs

\noindent
Abbildung~\vref{fig-losfaehrst-sem} zeigt, was passiert, wenn man die
Flexionslexikonregel auf die Ausgabe der Partikelverblexikonregel (S.\,\pageref{lr-pv})
anwendet. Diese Abbildung ähnelt der Abbildung~\vref{fig-loslachen-sem}, 
die zur Erklärung der semantischen Kombination von Partikel und Verb benutzt wurde.
Sie enthält zusätzlich jedoch noch Flexionsinformation.
\begin{figure}
\begin{forest}
sm edges
[{V[\cont \ibox{1}]}
   [\ibox{2} Part\feattab{
              \textsc{mod} \ibox{3} [\cont \ibox{4} ]\\
              \cont \ibox{5} \relation{beginnen}\iboxb{4} } [los]]
   [{V[\comps \sliste{ \ldots, \ibox{2} }, \cont \ibox{1} \emph{present}\iboxb{5}]}, l sep+=\baselineskip
%Flexions-LR
      [V\feattab{
          \comps \nliste{ \ldots, \ibox{2} Part[\textsc{mod} \ibox{3}, \cont \ibox{5}] },\\
          \cont  \ibox{5} }, l sep+=\baselineskip,edge label={node[midway,right]{Flexions-LR}}
% PV LR
         [{\ibox{3} V[\cont \ibox{4} \emph{lachen}\iboxb{6}]},edge label={node[midway,right]{PV-LR}}
            [lach]]]]] 
\end{forest}
\caption{Flexion von \stem{lach} und Kombination mit \emph{los}}\label{fig-losfaehrst-sem}
\end{figure}
In der Ausgabe der Partikelverblexikonregel wird der \contw mit dem \contw der Partikel
\iboxb{5} geteilt. Dieser \contw wird in der Ausgabe der Lexikonregel für die Verbflexion
unter die \emph{present}"=Relation eingebettet. Wenn die Partikel mit der
flektierten Form des Stammes \stem{lach} kombiniert wird, wird \ibox{5} innerhalb
der Repräsentation im Verb entsprechend dem von der Partikel beigesteuerten Wert
instantiiert. Im Fall von \emph{los} ist der semantische Beitrag der Partikel
\relation{beginnen}\iboxb{4}, wobei \iboxt{4} der semantische Beitrag des Basisverbs ist.

\begin{comment}
Turning to morphological aspects of inflectional rules,
the participle inflection is dependent on the stress\is{stress} pattern of the verb: If the first
syllable is stressed the participle is formed with \prefix{ge} (\mex{1}a), 
if it is not stressed the \prefix{ge} is omitted (\mex{1}b).
\eal
\ex ger\'edet, ge\'arbeitet
\ex diskut{\'\i}ert, krak\'eelt
\zl
The distribution of \prefix{ge} is the same for simplex and particle verbs. Therefore it is sufficient
to assume that the lexical rule that licenses the participle form is sensitive to the phonological
form of the base verb. The phonological contribution of the particle that will be combined with
the verb is totally irrelevant for the distribution of \prefix{ge}. Since the form of the particle
does not matter as far as the phonology of the participle inflection is concerned, it is unproblematic
that the particle and the base verb are discontinuous in verb-initial sentences.

\citet{Geilfuss-Wolfgang98a} develops an OT\is{Optimality Theory (OT)} 
analysis for the distribution of \prefix{ge}, including the distribution in particle verbs. 
He tries to capture the data on a purely phonological\is{phonology} basis. In order to achieve this, he has to stipulate 
four constraints, one specific to \prefix{ge} and one specific to particle verbs. 
Such stipulations are not necessary in the approach suggested in this book:
\emph{ge-} and \emph{-t} are attached to the stem by an inflectional lexical rule
and the particle is added in a later step as part of the predicate complex.
\end{comment}

Bevor wir uns der derivationellen Morphologie zuwenden, möchte ich die vollständige
Analyse von \emph{loslachst} vorstellen:
Das Ergebnis der Anwendung der Flexionsregel auf den Lexikoneintrag
für das Partikelverb mit dem Stamm \stem{lach}
in (\ref{le-lach-+particle}) auf Seite~\pageref{le-lach-+particle} ist in (\mex{1}) zu sehen:
%\vpageref{le-lachst-+particle}.
%
%\begin{figure}
\newsavebox{\boxxcompvier}
\savebox{\boxxcompvier}{
\onems{ l \onems{ c \onems{ h \onems[particle]{ mod$|$l \onems{
                                                                                            c \ms{ head & verb\\
                                                                                                       comps & \ibox{1}\\
                                                                                                     }\\
                                                                                            cont \ms[lachen]{ agens & \ibox{2} \\
                                                                                                            } \\
                                                                                            }\\
                                                                                   subj \ibox{3} \\
                                                                                 } \\
                                                         comps \ibox{4} \\
                                                       } \\
                                                cont \ibox{5} \\
                                              } \\
                                 }
}
\eas
\label{le-lachst-+particle}
\mbox{\emph{lachst} (Präsens + Selektion einer Partikel):}\\
%\resizebox{\linewidth}{!}{%
\ms{
cat & \onems
\zs
%\vspace{-\baselineskip}\end{figure}

\noindent
Obwohl der Bedeutungsbeitrag der Partikelverbkombination \iboxb{5} noch unterspezifiziert
ist, weil das Verb noch nicht mit der Partikel kombiniert wurde, kann man sich
auf den Beitrag beziehen. Der Beitrag, der von der Partikel kommen wird, wird unter
die \emph{present}"=Relation eingebettet. Wenn die Partikel \emph{los}
mit dem lexikalischen Zeichen in (\mex{0}) kombiniert wird, ergibt sich die Struktur in (\mex{1}).

\eas
\mbox{\emph{los lachst}:}\\
\ms{
cat & \onems{ 
%           head \ms[verb]{ vform & fin\\ } \\           
           comps \liste{NP[\type{str}]$_{\ibox{1}~2, sg}$} \\
         }\\
cont & \ms[present]{
         soa & \ms[beginnen]{ soa & \ms[lachen]{ agens & \ibox{1} \\
                                               }  \\
                           } \\
      }\\
}
\zs
Die Kombination von Partikel und Verb erfolgt so, wie es im Abschnitt~\ref{sec-lr-for-transp-pvs}
beschrieben wurde. Die einzigen Dinge, die hier noch hinzugefügt wurden, sind die
Kongruenzspezifikationen und die semantische Information über Tempus.\is{Tempus}



\subsubsection{Derivationelle Morphologie und Partikelverben}
\label{sec-derivation-hpsg}

In den folgenden Abschnitten zeige ich, dass die \geen und die \bard
mit der bisher eingeführten Analyse der Partikelverben erklärt werden können,
ohne dass irgendwelche Klammerparadoxa entstehen. Die Lexikonregel für die
\bard wurde bereits im Abschnitt~\ref{sec-bard} vorgestellt, die Interaktion mit der
Partikelverbregel wird dann im Abschnitt~\ref{sec-bard-pv} genauer untersucht.
Zuvor möchte ich aber noch die \geen erklären.

\subsubsubsection{Nominalisierungen}

\begin{comment}
As is clear from looking at the examples discussed
in the data section, there are various ways in which
the arguments of a verb can be realized after nominalization has been applied.
The subject or object of the verb can be realized as a \emph{von}"=PP (\mex{1}a), as a postnominal
genitive NP (\mex{1}b), or it may be left implicit (\mex{1}c).
\eal
\ex\iw{Angebrülle}
das Angebrülle von Norbert\footnote{
        taz, 15.10.1993, S.\,16.% %TAZ Nr. 4138 Seite 16 vom 15.10.1993
}
\ex\iw{Rumgeheule}
das Rumgeheule der FDP\footnote{
        taz, 07.01.1998, S.\,3.% %TAZ Nr. 5425 Seite 3 vom 07.01.1998
}
\ex\iw{Herumgerenne}
das Herumgerenne\footnote{
        taz, 01.02.1999, S.\,16.% %TAZ Nr. 5750 Seite 16 vom 01.02.1999
}
\label{ex-herumgerenne-zwei}
\zl
% Accusative objects can also be realized as a \emph{von}"=PP, as a postnominal
% genitive NP, or they may be left implicit.
% \eal
% \ex\iw{Gutfinden}
% \gll das Gutfinden    von Harald Juhnke\\
%      the good.finding of  Harald Juhnke\\
% \glt `Appreciation of Harald Juhnke'
% \ex\iw{Kaputtsanierung}
% \gll die Kaputtsanierung   der Stadt\\
%      the broken.renovation of.the town\\
% \glt `the destructive over-renovation of the town'
% \ex\iw{Kaputtindustrialisierung}
% \gll die Kaputtindustrialisierung\\
%      the broken.industrialization\\
% \glt `the destructive over-industrialization'
% \zl
Rather than giving a detailed account of the various ways in which arguments
can be realized, I will consider the case where all arguments are suppressed.
The main purpose of this subsection is not to provide all the details of argument realizations
in nominal environments, but rather to show how \geens can be accounted
for without any bracketing paradox.
\end{comment}

Die\is{Derivation!Nomen|(}
Lexikonregel in (\mex{2}) lizenziert Nominalisierungen
wie die in (\mex{1}):\footnote{
  In dieser Regel werden die Argumente des Eingabeverbs ignoriert. Natürlich
  gibt es vielfältige Möglichkeiten, diese zu realisieren, was in einem vollständigen
  Ansatz auch berücksichtigt werden muss. Damit die Analyse der Nominalisierung von Partikelverben
  funktioniert, müssen mindestens die komplexbildenden Argumente (die Verbpartikel) der \textsc{lex-dtr} übernommen werden. Dafür werden weiter unten noch Beispiele gegeben.%
}
\ea
das Herumgerenne\footnote{
        taz, 01.02.1999, S.\,16.% %TAZ Nr. 5750 Seite 16 vom 01.02.1999
}
\label{ex-herumgerenne-zwei}
\z

\eas
Lexikonregel für \geenen:\\
\label{lr-gee-nom}%
\ms[ge-e-derived-noun-stem]{
phon & f\textrm{(\,\phonliste{ ge }, \ibox{1}, \phonliste{ e }\,)}\\
synsem & \onems{ loc  \ms{ cat  & \ms{ head   & noun\\
                                       spr & \sliste{ Det }\\
%                                       comps & \eliste\\
                                      } \\
                            cont & \ms{ ind & \ibox{2} \ms{ per & 3\\
                                                            num & sg\\
                                                            gen & neu\\
                                                           }\\
                                         restr & \liste{%
                                                \ms[wiederholtes-ereignis]{
                                                        inst & \ibox{2}\\
                                                        soa & \ibox{3}\\
                                                }}\\   
                                      }\\                                 
                         }\\
               }\\
lex-dtr & \onems[stem]{
           phon   \ibox{1} \\
           synsem$|$loc \onems{ cat$|$head \type{verb}\\
                                cont    \ibox{3} \\
                        }\\
           }\\
}\is{Lexikonregel!ge- -e Nominalisierung@\emph{Ge- -e}-Nominalisierung}
\zs

\noindent
$f$ ist dabei wieder eine Funktion, die den \phonw des Eingabezeichens
mit \gee kombiniert. 
Das \suffix{e} ist optional, wenn das Eingabenomen -- wie \zb in \emph{Rumgeballer} --
mit einer unbetonten\is{Betonung} Silbe \suff{er}, \suff{el}, \suff{en} endet.
% Gallmann phonologisch gesteuert =/= phonologisch bedingt
Das Ergebnis der Regelanwendung ist ein nominaler Stamm. Dieser Stamm
muss flektiert werden, damit er in der Syntax benutzt werden kann.\footnote{
        Siehe auch \citet[\page118]{Koenig99a} für einen
        ähnlichen Vorschlag zur Interaktion von Flexion und Derivation.%
}
Nullflexion\is{Flexion!Null-} führt zu einer Form\is{Kasus!morphologischer}
im Nominativ, Dativ bzw.\ Akkusativ. Wird bei der Flexion ein \suffix{s} angehängt,
bekommt man die Genitiv"=Form.

\begin{comment}
% Olsen stellt Gemeinsamkeiten zwischen lexikalisierten Ge-Nomina (Geklimper) und motivierten
% Versionen (Geklimper) her. Diese kann man durch Typen modellieren.

The rule applies to all verbs. The valence properties of the nominalized verb
are ignored since this lexical rule licenses only the bare noun with a determiner
without any complements that could be inherited from the verb. Following 
\citet[Chapter~1]{ps2} and \citet{Demske2001a}, I assume that the noun selects a determiner,
\ie, I assume an NP analysis rather than a DP\is{determiner phrase}
analysis, but the rule in (\mex{0}) could be easily changed. For a DP
analysis in HPSG see \citew{Abb94}. A special variant of a DP analysis
can be found in \citew{Netter94} and \citew{Netter98-Eng}.
\end{comment}

Da die Nomina, die mittels \geen abgeleitet sind, Neutra sind, lizenziert
die Lexikonregel ein Nomen mit einem referentiellen Index mit dem Genus"=Wert \type{neu}.
\geenen haben keine Pluralformen\is{Plural} \citep[\page34]{Bierwisch89a}, was dadurch
erfaßt wird, dass der Numerus"=Wert\is{Numerus} in der Ausgabe der Lexikonregel ebenfalls
spezifiziert ist. Pluralaffixe können deshalb nicht mit durch die Lexikonregel
in  \pref{lr-gee-nom} lizenzierten Stämmen kombiniert werden. 
Der referentielle Index \iboxb{2} ist identisch mit dem Wert des \textsc{inst}"=Merkmals
der \relation{wiederholtes-ereignis}"=Relation.


% das Grauen vor dem Henker -> auch subjektlose Verben koennen nominalisiert werden

Als Beispiel für die Anwendung der Regel soll zuerst die Ableitung von \emph{Gerenne} diskutiert
werden. Als Eingabe zur Lexikonregel dient der Stamm des Simplexverbs \stem{renn}.
Der \locw für \stem{renn}, der in (\mex{1}) angegeben ist, ist parallel
zu dem bereits diskutierten für \stem{lach}.

\eas
\stem{renn}:\\
\ms{
cat & \ms{ head & verb \\
           comps & \liste{NP[\str]\ind{1}} \\
         }\\
cont & \ms[rennen]{ agens & \ibox{1} \\
                  } \\
}\iw{rennen|uu}
\zs

\noindent
Das Ergebnis der Anwendung der Lexikonregel in (\ref{lr-gee-nom}) auf (\mex{0})
ist (\mex{1}).

\eas
\label{le-gerenne}
\stem{Gerenne}:\\
%\resizebox{\linewidth}{!}{%
\ms{ cat  & \ms{ head   & noun\\
                              spr & \sliste{ Det }\\
                              comps & \eliste\\
                            } \\
                  cont & \ms{ ind & \ibox{2} \ms{ per & 3\\
                                                  num & sg\\
                                                  gen & neu\\
                                                }\\
                              restr & \liste{ 
                                      \ms[wiederholtes-ereignis]{
                                                        inst & \ibox{2}\\
                                                        soa & \ms[rennen]{ agens & \etag \\
                                                                        } \\
                                                        }}\\   
                            }\\ 
}
%}
\zs

\noindent
Die Agens"=Rolle von \emph{rennen} ist nicht an ein Element in der Valenzrepräsentation
des Nomens gelinkt, weshalb der Wert des \textsc{agens}"=Merkmals in (\mex{0}) 
als leeres Kästchen visualisiert ist.
% Jacobs94a -> definitheitsneutrale Variablen
% The nominalization
% rule has to take care of the existential quantification of this argument.

Als nächstes soll die Analyse des Wortes \emph{Herumgerenne} vorgeführt werden.
Wie \emph{los} kann die Partikel \word{herum} in der relevanten Lesart nur mit intransitiven Verben kombiniert werden.
Das zeigt (\mex{1}):
\eal
\ex[]{\iw{herumrennen}\iw{herumhüpfen}
Karl rennt / hüpft herum.
}
\ex[]{\iw{herumlesen}
Karl liest (in dem Buch) herum.
}
\ex[*]{
Karl liest das Buch herum.
}
\zl
Es gibt viele Bedeutungen von \emph{herum}. Die Bedeutung,
die hier von Interesse ist, fügt der Bedeutung des Basisverbs
eine Komponente hinzu, in der enthalten ist, dass die Aktion
des Basisverbs ziellos ist.
(\mex{1}) zeigt den \locw des Lexikoneintrags von \emph{herum}.
Der \locw ist parallel zu dem von \emph{los}, der auf Seite~\pageref{le-los-asp}
angegeben wurde.
%
\eas
\label{le-herum-part}
\mbox{\emph{herum}:}\\
\ms{
cat & \ms{ head & \ms[particle]{ mod  & \textrm{V[\textsc{comps} \sliste{ NP[\str] }, \textsc{cont} \ibox{1}]} \\
                                subj & \eliste \\
                              } \\
           comps & \eliste \\
         }\\
cont & \ms[ziellos]{ soa & \ibox{1} \\
                   } \\
}
\zs

%
\noindent
Die Analyse von \emph{Herumgerenne} zeigt Abbildung~\vref{fig-herumgerenne}.
\begin{figure}
\begin{forest}
sm edges
[{N[\cont \ibox{1}]}
  [\ibox{2} Part\feattab{
              \textsc{mod} \ibox{3} [\cont \ibox{4} ]\\
              \cont \ibox{5} \relation{ziellos}\iboxb{4} } [herum]]
  [{N[\comps \sliste{ \ldots, \ibox{2} }, \cont \ibox{1} \ldots{}
    \relation{wiederholtes-ereignis}\iboxb{5}]}, l sep+=\baselineskip
%& \hspace{27ex}\gee-Nominalisierungs-LR\\[4ex]
    [V\feattab{
       \comps \nliste{ \ldots, \ibox{2} Part[\textsc{mod} \ibox{3}, \cont \ibox{5}] },\\
       \cont  \ibox{5} }, l sep+=\baselineskip, edge label={node[midway,right]{\gee-Nominalisierungs-LR}}
%&\hspace{8ex}PV LR\\[4ex]
     [{\ibox{3} V[\cont \ibox{4} \relation{rennen}\iboxb{6}]}, edge label={node[midway,right]{PV-LR}}
       [renn]]]]]
\end{forest}
\caption{Analyse von \emph{Herumgerenne}}\label{fig-herumgerenne}
\end{figure}
%
Um \emph{Herumgerenne} abzuleiten, muss zuerst die Lexikonregel für produktive Partikelverbbildungen
((\ref{lr-pv}) auf Seite~\pageref{lr-pv}) auf den im Lexikon gelisteten Eintrag \stem{renn} angewendet
werden. Das Ergebnis der Regelanwendung ist ein lexikalisches Zeichen,
das eine Partikel selegiert \iboxb{2}.
Der Bedeutungsbeitrag dieser Partikel \iboxb{5} wird mit der Bedeutung
des lexikalischen Zeichens identifiziert, das durch die Partikelverblexikonregel
lizenziert wird. Die Nominalisierungsregel wird auf dieses Zeichen angewendet
und bettet dessen semantischen Beitrag unter die Relation \relation{wiederholtes-ereignis} ein.
Im nächsten Schritt wird das Nomen flektiert (in Abbildung~\ref{fig-herumgerenne} nicht dargestellt)
und danach mit der Partikel kombiniert. Da das Nomen 
der Kopf in einer head"=cluster"=Struktur ist, ist sein Bedeutungsbeitrag \iboxb{1} 
identisch mit dem Beitrag der Mutter in der Struktur. Die Bedeutung der Partikel
ist bei der Kombination dann bekannt (sowohl die Bildung von \emph{Losgerenne}
als auch die von \emph{Herumgerenne} wäre mit dem \emph{gerenne}, das eine Partikel
selegiert, möglich). Über ihren \modw kann die Partikel auf den semantischen
Beitrag des Verbs zugreifen \iboxb{4} und diesen unter die \relation{ziellos}"=Relation
einbetten. Das Ergebnis ist \relation{ziellos}(\relation{rennen}\iboxb{6}). 
Da der semantische Beitrag durch die Nominalisierungsregel unter \relation{wiederholtes-ereignis}
eingebettet wird, ist die semantische Repräsentation für die gesamte Nominalisierung
\relation{wiederholtes-ereignis}(\relation{ziellos}(\relation{rennen}\iboxb{6})), was genau dem
erwünschten Resultat entspricht.

Nach dieser Skizze der Analyse sollen jetzt die genauen Merkmalbeschreibungen angegeben werden:
(\mex{1}) zeigt die Merkmalbeschreibung des Lexikoneintrags, der durch die Partikelverblexikonregel
in \pref{lr-pv} auf Seite~\pageref{lr-pv} lizenziert wird. Dieser Eintrag ähnelt dem von \stem{lach}, 
der in (\ref{le-lach-+particle}) gezeigt wurde. Der einzige Unterschied besteht
in dem Teil der semantischen Repräsentation, der sich unterscheidet,
da er vom Basisverb (\stem{lach} vs.\ \stem{renn}) kommt.

\newsavebox{\boxxcompfuenf}
\savebox{\boxxcompfuenf}{
\onems{ l \onems{ c \onems{ h \onems[particle]{ mod$|$l \onems{
                                                                                            cat \ms{ head & verb\\
                                                                                                   comps & \ibox{1} \\
                                                                                                 }\\
                                                                                            cont \ms[rennen]{ agens & \ibox{2} \\
                                                                                                            } \\
                                                                                            }\\
                                                                                   subj \ibox{3} \\
                                                                                 } \\
                                                         comps \ibox{4} \\
                                                       } \\
                                                cont \ibox{5} \\
                                              } \\
                                 }
}
\eas
\label{le-renn-+particle}
\mbox{\emph{renn-} (mit Selektion einer Partikel):}\\
%\resizebox{\linewidth}{!}{%
%\scalebox{0.9}{%
\onems{
cat \ms{ %head & verb \\
       comps & \begin{tabular}[t]{@{}l@{}}
                \textrm{(}\,\ibox{1} \liste{NP[\str]\ind{2}} \textrm{)} $\oplus$ \ibox{3} $\oplus$ \ibox{4} $\oplus$ \\[3mm]
                \liste{\usebox{\boxxcompfuenf} } \\
                \end{tabular}
     }\\
cont \ibox{5} \\
}
%}
\zs
%

\noindent
Die Lexikonregel für \geen wird auf diesen Eintrag angewendet.
Das Ergebnis zeigt (\mex{1}).
\newsavebox{\boxxcompsechs}
\savebox{\boxxcompsechs}{
\onems{ l \onems{ c \onems{ h \onems[particle]{ mod$|$l \onems{
                                                                                            c \ms{ head & verb \\
                                                                                                   comps & \liste{NP[\str]\ind{1}} \\[1mm]
                                                                                                 }\\
                                                                                            cont \ms[rennen]{ agens & \ibox{1} \\
                                                                                                            } \\
                                                                                            }\\
%                                                                                   subj \ibox{2} \\
                                                                                 } \\
%                                                         comps \ibox{3} \\
                                                       } \\
                                                cont \ibox{2} \\
                                              } \\
                                 }
}
\eas
\label{le-gerenne+particle}
\mbox{\emph{gerenne-} (mit Selektion einer Partikel):}\\
\onems{
cat \onems{ head \type{noun} \\
            spr \sliste{ Det }\\
           comps \sliste{ \usebox{\boxxcompsechs}} \\
         }\\
cont \ms{ ind & \ibox{3} \ms{ per & 3\\
                                                            num & sg\\
                                                            gen & neu\\
                                                           }\\
                                         restr & \liste{ 
                                                \ms[wiederholtes-ereignis]{
                                                        inst & \ibox{3}\\
                                                        soa & \ibox{2}\\
                                                }}\\   
                                      }\\       
}
\zs

\noindent
Dieser Stamm wird durch eine Flexionsregel flektiert. Das Ergebnis 
für Nominativ, Dativ und Akkusativ unterscheidet
sich von (\mex{0}) nur durch die Instantiierung der Kasuswerte, weshalb
die entsprechende Merkmalbeschreibung hier nicht gesondert aufgeführt wird.
In der folgenden Struktur werden die Kasusinformation und auch die Strukturteilungen,
die für die Kongruenz mit dem Determinator benötigt werden, der Übersichtlichkeit
halber weggelassen.

Die Bedeutung von \emph{rennen} + Partikel \iboxb{2} ist ein Argument der
Relation \relation{wiederholtes-ereignis}. In
(\mex{0}) ist der Wert von \iboxt{2} noch unterspezifiziert,
aber wenn (\mex{0}) mit einer Partikel kombiniert wird, wird \iboxt{2}
instantiiert. Das Ergebnis der Kombination der Partikel \emph{herum} in \pref{le-herum-part}
mit (\ref{le-gerenne+particle}) zeigt (\mex{1}).

\eas
\label{fs-Herumgerenne}
\emph{Herumgerenne}:\\
%\resizebox{\linewidth}{!}{%
\ms{
cat & \ms{ head & noun \\
           spr & \sliste{ Det } \\
           comps & \eliste\\
         }\\
cont & \ms{ ind & \ibox{2} \ms{ per & 3\\
                                                            num & sg\\
                                                            gen & neu\\
                                                           }\\
                                         restr & \liste{ 
                                                \ms[wiederholtes-ereignis]{
                                                        inst & \ibox{2}\\
                                                        soa & \ms[ziellos]{ soa & \ms[rennen]{ agens & \etag \\
                                                                                                    } \\ 
                                                              }\\
                                                }}\\   
                                      }\\        
}
%}
\zs

\noindent
Wie beim einfachen \emph{Gerenne} in (\ref{le-gerenne}) ist das
Agens von \emph{rennen} in (\mex{0}) nicht spezifiziert. 
% The nominalization rule takes care of the 
% existential quantification of this argument. 
Die Skopusbeziehungen zwischen Partikel und dem semantischen Material,
das von der Derivation beigesteuert wird, ist korrekt, Mechanismen
wie Umklammerung werden nicht gebraucht, da in dieser Analyse kein
Klammerparadoxon vorliegt.

\begin{comment}
The derivation with object predicatives and resultatives is completely analogous:
The rule in (\ref{lr-gee-nom}) is applied to the lexical entry for the object predicative
\emph{find}- (`find') producing \stem{gefinde}, which is then combined with 
\emph{schön} (`beautiful') to yield \emph{Schöngefinde}\iw{Schöngefinde} (`beautiful.finding').
In the case of resultative constructions, the listed entry for \emph{schlag}- (`to hit')
is fed into the lexical rule (\ref{lr-result}) for resultative constructions.
The output of this rule is the input to (\ref{lr-gee-nom}), yielding
\stem{geschlage}, which is then combined with \emph{tot} (`dead'),
resulting in \word{Totgeschlage} (`dead.beating').
\end{comment}
\is{Derivation!Nomen|)}

Nach der Behandlung von Flexion und \geen wende ich mich nun
dem schwierigsten Fall zu: der \bard.

\subsubsubsection{Adjektivderivation}
\label{sec-bard-pv}

Die\iw{\bars|(}%
\is{Derivation!Adjektiv|(}
Interaktion der \bard mit Partikelverben ist die
komplexeste, da sowohl Beschränkungen in bezug auf Valenz\is{Valenz}
als auch Skopusbeziehungen\is{Skopus} eine Rolle spielen.
Auf Seite~\pageref{lr-bar-adj} haben wir bereits die
Lexikonregel \pref{lr-bar-adj} für \bard vorgestellt.

Im folgenden sollen komplexe Derivationen wie \emph{anfahrbar}
in (\mex{1}) diskutiert werden:
\ea
 "`Die Kneipen,        Theater  und Geschäfte müssen \emph{anfahrbar} bleiben."'\footnote{
taz, hamburg, 05.06.1997, S.\,22.% %TAZ-HAMBURG Nr. 5244 Seite 22 vom 05.06.1997
}
\z
\emph{anfahren} ist nach einem produktiven Muster gebildet, und die entsprechende
Partikelverbregel wurde bereits im Kapitel~\ref{sec-lr-for-transp-pvs} besprochen.
Die Diskussion der Interaktion der Partikelverbregel mit der \bard wird in zwei Teile geteilt: Zuerst
diskutiere ich die Beschränkungen in bezug auf die Verbvalenz, und dann zeige ich,
dass die bisher formulierten Regeln die Bedeutung von \barden mit Partikelverben adäquat
erfassen.

Abbildung~\vref{fig-p-fahrbar-pv-lr} zeigt die Anwendung der Partikelverbregel.
\begin{figure}
\begin{forest}
sm edges
[{V[\comps \ibox{1} $\oplus$ \ibox{2} $\oplus$ \ibox{3} $\oplus$ \nliste{ Part[\subj \ibox{2}, \comps \ibox{3}] }]}, l sep+=\baselineskip
  [{V[\comps \ibox{1} \sliste{ NP[\str] }]}, edge label={node[midway,right]{PV-LR}} 
    [fahr]]]
\end{forest}
\caption{Anwendung der Partikelverbregel auf \stem{fahr}}\label{fig-p-fahrbar-pv-lr}
\end{figure}
Das Ergebnis der Anwendung ist ein Lexikoneintrag mit einem unterspezifizierten
\compsw. Der letztendliche Wert wird durch die Partikel bestimmt, sobald die Partikel
mit dem Kopf"=Verb kombiniert wird.

Die Lexikonregel für die \bard verlangt als Eingabe ein Verb mit einem Objekt mit strukturellem
Kasus. Da die Ausgabe der Partikelverbregel kompatibel mit dieser Anforderung ist, kann
die Lexikonregel für die \bard auf das Partikelverb angewendet werden. Das wird in Abbildung~\vref{fig-p-fahrbar-val}
gezeigt.
\begin{figure}
\begin{forest}
sm edges
[{Adj[\subj \sliste{ \ibox{4} }, \comps \ibox{5} $\oplus$ \nliste{ Part[\subj \ibox{2}, \comps \ibox{3}] }]}, l sep+=\baselineskip
%\hspace{19ex}\bard LR\\[4ex]
  [{V[\begin{tabular}[t]{@{}l@{}}
      \comps \ibox{1} $\oplus$ \ibox{2} $\oplus$ \ibox{3} $\oplus$ \nliste{ Part[\subj \ibox{2},
        \comps \ibox{3}] }]\\
                $\wedge$ \ibox{1} $\oplus$ \ibox{2} $\oplus$ \ibox{3} = \sliste{ NP[\str], \ibox{4}
                  NP[\str] } $\oplus$ \ibox{5} 
      \end{tabular}}, l sep+=\baselineskip, edge label={node[midway,right]{\bard-LR}}
%
%\hspace{8ex}PV LR\\[4ex]
     [{V[\comps \ibox{1} \sliste{ NP[\str] }]}, edge label={node[midway,right]{PV-LR}}
        [fahr]]]]
\end{forest}
\caption{Anwendung der Lexikonregel für die \bard auf \stem{fahr} mit selegierter Partikel}\label{fig-p-fahrbar-val}
\end{figure}
Der \compsw des Eingabezeichens muss eine Liste sein, die mit zwei Nominalphrasen mit strukturellem Kasus
beginnt (\sliste{ NP[\str], \ibox{4} NP[\str] } $\oplus$ \ibox{5}\,).
Da der \compsw des Eingabezeichens für die \bard in Abbildung~\ref{fig-p-fahrbar-val} die
Verknüpfung der \compsl des Basisverbs mit dem \subjw und dem \compsw der selegierten Partikel ist, können nur Partikeln, die eine
NP[\str] in ihrer \subjl oder in ihrer \compsl haben, mit dem Ergebnis der \bard kombiniert werden.

Abbildung~\vref{fig-anfahrbar-val} zeigt die Kombination von \emph{an}$_5$ und \emph{fahrbar}.\footnote{
  Zu den verschiedenen Formen der Partikel \emph{an} siehe Seite~\pageref{verschiedene-ans}.%
}
\begin{figure}
\begin{forest}
sm edges
[{Adj[\subj \sliste{ \ibox{4} }, \comps \ibox{5} \eliste]}
   [\ibox{6} Part\feattab{
                 \subj \ibox{2} \sliste{ NP[\str] },\\
                 \comps \ibox{3} \eliste } [an]]
   [{Adj[\subj \sliste{ \ibox{4} }, \comps \ibox{5} $\oplus$ \sliste{ \ibox{6} }]}, l sep+=\baselineskip
%& \hspace{19ex}\bard LR\\[4ex]
      [{V[\begin{tabular}[t]{@{}l@{}}
                \comps \ibox{1} $\oplus$ \ibox{2} $\oplus$ \ibox{3} $\oplus$
                \nliste{ Part[\subj \ibox{2}, \comps \ibox{3}] }]\\
                $\wedge$ \ibox{1} $\oplus$ \ibox{2} $\oplus$ \ibox{3} = \sliste{ NP[\str], \ibox{4} NP[\str] } $\oplus$ \ibox{5}\\
               \end{tabular}}, l sep+=\baselineskip, edge label={node[midway,right]{\bard-LR}}
%&\hspace{8ex}PV LR\\[4ex]
         [{V[\comps \ibox{1} \sliste{ NP[\str] }]}, edge label={node[midway,right]{PV-LR}}
           [fahr]]]]] 
\end{forest}
\caption{Kombination von \emph{an}$_5$ und \emph{fahrbar} (Valenz)}\label{fig-anfahrbar-val}
\end{figure}
Die Partikel hat eine NP[\str] in ihrer \subjl \iboxb{2}. Der \compsw der Partikel \iboxb{3} ist die leere Liste.
Die Verknüpfung von \ibox{2} and \ibox{3} ist deshalb eine Liste, die genau eine NP[\str] enthält. 
Dieses Element wird mit dem Element \iboxt{4} identifiziert, das von der Lexikonregel für die \bard zum Subjekt angehoben
wird. Deshalb hat das Adjektiv \emph{anfahrbar} als Subjekt das Element, das durch die Partikel eingeführt
wurde, und die \compsl von \emph{anfahrbar} ist leer. Weil es außer den beiden NPen kein anderes
Element in der Verknüpfung von \ibox{2} and \ibox{3} gibt, ist \iboxt{5} die leere Liste.

Interessanterweise ermöglichen die Regeln nicht nur die Analyse von \emph{anfahrbar}, sondern
schließen auch ungrammatische Beispiele wie das in (\mex{1}b) aus.\footnote{
  Das Wort \emph{losfahrbaren} ist vorstellbar in Wortgruppen wie \emph{die losfahrbaren Autos}.
  In dieser Verwendung liegt aber nicht die Partikel \emph{los} vor, die bisher diskutiert
  wurde, sondern ein \emph{los} mit der Bedeutung \emph{ab} oder \emph{lose}. Die \bard
  nimmt dann als Eingabe einen Verbstamm, wie er in
  Resultativkonstruktionen\is{Resultativkonstruktion} verwendet werden 
  kann. Siehe hierzu \citew[\page380--381]{Mueller2002b}.%
}
\eal
\ex[]{\iw{anfahrbar|)}
die anfahrbaren Geschäfte
}
\ex[*]{
die losfahrbaren Geschäfte
}
\zl
Der Grund hierfür ist, dass \emph{los} keine Argumente einführt. Da \emph{los} nur mit intransitiven
Verben kombiniert werden kann, ist das Ergebnis einer solchen Kombination wieder ein intransitives Verb.
Obwohl es eine Form \emph{fahrbare} gibt, die eine Partikel selegiert, kann diese nicht mit \emph{los}
kombiniert werden, da die Beschränkung, die durch die Lexikonregel für die \bard eingeführt wurde
(\,\ibox{1} $\oplus$ \ibox{2} $\oplus$ \ibox{3} = \sliste{ NP[\str], \ibox{4} NP[\str] } $\oplus$ \ibox{5}\,) 
verletzt wäre: \ibox{1} $\oplus$ \iboxt{2} $\oplus$ \iboxt{3} würde nur ein Element enthalten,
nämlich das Subjekt von \stem{fahr}.

Wenden wir uns nun der Bedeutungsrepräsentation in der Analyse von \emph{anfahrbar} zu,
die in Abbildung~\vref{fig-anfahrbar-sem} gezeigt wird.
\begin{figure}
\begin{forest}
sm edges
[{Adj[\cont \ibox{1}]}
   [\ibox{2} Part\feattab{
                 \textsc{mod} \ibox{3} [\cont \ibox{4} ]\\
                 \cont \ibox{5} \relation{gerichtet-auf}(\,\ibox{4}, \ibox{6}\,) } [an]]
   [{Adj[\comps \sliste{ \ibox{2} }, \cont \ibox{1} \emph{modal-op}(\,\ibox{5}\,)]}, l sep+=\baselineskip
%& \hspace{19ex}\bard LR\\[4ex]
      [V\feattab{
        \comps \nliste{ \ldots, \ibox{2} Part[\textsc{mod} \ibox{3}, \cont \ibox{5}] },\\
        \cont  \ibox{5} }, l sep+=\baselineskip, edge label={node[midway,right]{\bard-LR}}
%&\hspace{8ex}PV LR\\[4ex]
          [{\ibox{3} V[\cont \ibox{4} \emph{fahren}\iboxb{7}]}, edge label={node[midway,right]{PV-LR}}
            [fahr]]]]]
\end{forest}
\caption{Kombination von \emph{an}$_5$ und \emph{fahrbar} (Semantik)}\label{fig-anfahrbar-sem}
\end{figure}
Die Partikelverblexikonregel führt eine Partikel in die \compsl ein,
die die Eingaberepräsentation über \textsc{mod} selegiert \iboxb{3}. In der Ausgabe der
Lexikonregel wird der \contw \iboxb{5} mit dem \contw der Partikel in der \compsl identifiziert.
Die Lexikonregel für die  \bard bettet diesen \contw unter einen Modaloperator \emph{modal-op} ein.
Zu diesem Zeitpunkt ist keine Partikel vorhanden, und der Wert von \iboxt{5}
ist nicht beschränkt. Im nächsten Schritt wird die Partikel mit \emph{fahrbar} kombiniert.
Die Partikel hat die Form eines Adjunkts, ihr \modw wird mit dem \synsemw
des Stammes \stem{fahr} identifiziert, da das in der \compsl des abgeleiteten
Eintrags für  \stem{fahr} so spezifiziert ist (innerhalb von \iboxt{2}\,).
Dadurch kann die Partikel \emph{an} auf den semantischen Beitrag des Basisverbs \stem{fahr} zugreifen
und ihn unter den semantischen Beitrag der Partikel einbetten.
Das Ergebnis ist \relation{gerichtet-auf}(\,\ibox{4}, \ibox{6}\,),
wobei \iboxt{4} für \emph{fahren}\iboxb{7} steht, \dash, wir bekommen \relation{gerichtet-auf}(\relation{fahren}(\,\iboxt{7}, \ibox{6}\,).
\iboxt{6} und \iboxt{7} sind an das Objekt bzw.\ Subjekt von \emph{anfahren} gelinkt.
%
Erst nach der Kombination von \emph{an} und \emph{fahrbar} ist klar, wie der Wert von \iboxt{5}
aussieht. Dieser Wert ist das Argument des Modaloperators \emph{modal-op}, der von der \bard beigesteuert wird.
Da \emph{fahrbar} in Abbildung~\ref{fig-anfahrbar-sem} der Kopf von \emph{anfahrbar} ist, ist die Bedeutung von \emph{anfahrbar} identisch
mit der von \emph{fahrbar} \iboxb{1}.

Nach dieser Skizze der Analyse werde ich nun wieder die Details erörtern.
Leser, die an den Details nicht interessiert sind, können
zur Diskussion von \pref{pvs-in-morphology} auf Seite~\pageref{pvs-in-morphology} weiterblättern.

(\mex{1}) zeigt den Lexikoneintrag, der durch die Partikelverblexikonregel lizenziert
wird. Er ist parallel zu \stem{lach} mit selegierter Partikel in (\ref{le-lach-+particle}) auf Seite~\pageref{le-lach-+particle}.
\newsavebox{\boxxcompfuenfa}
\savebox{\boxxcompfuenfa}{
\onems{ l \onems{ c \onems{ h \onems[particle]{ mod$|$l \onems{
                                                                                            cat \ms{ head & verb \\
                                                                                                       comps & \ibox{1} \\
                                                                                                     }\\
                                                                                            cont \ms[fahren]{ agens & \ibox{2} \\
                                                                                                            } \\
                                                                                            }\\
                                                                                   subj \ibox{3} \\
                                                                                 } \\
                                                         comps \ibox{4} \\
                                                       } \\
                                                cont \ibox{5} \\
                                              } \\
                                 }
}
\ea
\label{le-fahr-+particle}
\begin{minipage}[t]{\linewidth}
\mbox{\stem{fahr}:}\\
%\resizebox{\linewidth}{!}{%
\onems{
cat \ms{ %head & verb \\
           comps & \textrm{(\,\ibox{1} \liste{NP[\str]\ind{2}})} $\oplus$ \ibox{3} $\oplus$ \ibox{4} $\oplus$\\[3mm]
                  & \liste{\usebox{\boxxcompfuenfa} } \\
         }\\
cont \ibox{5} \\
}
%}
\end{minipage}
\z
(\mex{1}) zeigt das Ergebnis der Identifikation von (\ref{le-fahr-+particle}) 
mit der \textsc{lex-dtr} der \bard"=Lexikonregel in (\ref{lr-bar-adj}).

\newsavebox{\boxxcompvierb}
\savebox{\boxxcompvierb}{
\liste{ \onems{ l \onems{ c \onems{ h \onems[particle]{ mod$|$l \onems{
                                                                                            c \ms{ head & verb \\
                                                                                                       comps & \ibox{1'} \liste{ NP[\str]\ind{2'}} \\
                                                                                                     }\\
                                                                                            cont \ms[fahren]{ agens & \ibox{2'} \\
                                                                                                            } \\
                                                                                            }\\
                                                                                   subj \ibox{3'} \\
                                                                                 } \\
                                                         comps \ibox{4'} \\
                                                       } \\
                                                cont \ibox{5'} \\
                                              } \\
                                 }}%
}
\eas
\label{le-fahr-+bar+particle}
\mbox{\emph{fahrbar}:}\\
\resizebox{\linewidth}{!}{%
\begin{tabular}[t]{@{}l@{}}
\onems{ 
phon   \ibox{1} $\oplus$ \phonliste{ bar }\\
synsem \onems{ loc  \ms{ cat  & \ms{ head   & \ms[adj]{ 
                                                 subj & \liste{ \ibox{2} NP[\str] }\\
                                                 }\\
                                     comps & \ibox{3} $\oplus$ \ibox{5} \\
                                   } \\
                            cont & \ms[modal-op]{
                                    soa & \ibox{4}\\
                                   }\\
                         }\\
               }\\
l-dtr
\onems{ phon \ibox{1} \phonliste{ fahr } \\
        synsem$|$loc\\ 
\onems{
c \ms{  head  & verb \\
           comps & $\ibox{1'} \oplus \ibox{3'} \oplus \ibox{4'}~\oplus$  \\
                  & \ibox{5} \usebox{\boxxcompvierb} \\
         }\\
cont \ibox{4} = \ibox{5'} \\
}\\
}\\
}\\ $\wedge \liste{ NP[\str], \ibox{2} NP[\str] } \oplus \ibox{3} = \ibox{1'} \oplus \ibox{3'} \oplus \ibox{4'}$ \\
\end{tabular}
}
\zs

\noindent
Ich habe die Nummern in den Kästchen innerhalb der Lexikonregel auch in (\mex{0})
beibehalten.\NOTE{FB: Das ist verwirrend. Würde tags von fahr- ohne Apostrophen verwenden und grau
  unterlegen und in (41) neue Tags für neue Sachen einführen.}
Die Nummern, die im Lexikoneintrag für \stem{fahr} verwendet wurden, sind mit einem Apostroph markiert.
Zusätzlich zu den Nummern, die in der Lexikonregel vorkommen, habe ich die \iboxt{5} benutzt,
um die Identität der Partikelbeschreibung bei der \textsc{lex-daughter} und der Mutter zu verdeutlichen.
Betrachtet man nur den Mutterknoten von (\mex{0}), bekommt man (\mex{1}).

\newsavebox{\boxxcompsieben}
\savebox{\boxxcompsieben}{
\onems{ l \onems{ c \onems{ h \onems[particle]{ mod$|$l \onems{
                                                                                            c \ms{ head & verb\\
                                                                                                       comps & \ibox{1'} \liste{NP[\str]\ind{2'}} \\
                                                                                                     }\\
                                                                                            cont \ms[fahren]{ agens & \ibox{2'} \\
                                                                                                            } \\
                                                                                            }\\
                                                                                   subj \ibox{3'} \\
                                                                                 } \\
                                                         comps \ibox{4'} \\
                                                       } \\
                                                cont \ibox{5'} \\
                                              } \\
                                 }
}
\eas
\label{le-fahrbar+particle}
\mbox{\emph{fahrbar} + Partikel:}\\
\resizebox{\linewidth}{!}{%
\begin{tabular}{@{}l@{}}
\onems{ c \ms{ head   & \ms[adj]{ 
                                                 subj & \liste{ \ibox{2} NP[\str] }\\
                                                 }\\
                                     comps & \ibox{3} $\oplus$
                                             \liste{\usebox{\boxxcompsieben} }
                                   } \\
                            cont  \ms[modal-op]{
                                    soa & \ibox{5'}\\
                                   }\\
                         }\\ $\wedge \liste{ NP[\str], \ibox{2} NP[\str] } \oplus \ibox{3} = \ibox{1'} \oplus \ibox{3'} \oplus \ibox{4'}$ \\
\end{tabular}
}
\zs

\noindent
Die Beschränkung, die zusätzlich zu den Beschränkungen innerhalb der Struktur erfüllt werden muss,
besagt, dass die Valenzliste des Partikelverbs (\,\ibox{1'} $\oplus$ \ibox{3'} $\oplus$ \ibox{4'}\,) 
in zwei Listen auf"|teilbar sein muss: eine Liste mit zwei NPen mit strukturellem Kasus und eine
Restliste \iboxb{3}. Die Restliste ergibt zusammen mit der Partikel die \compsl des Mutterknotens.
%With the assumption that the \subj list of the particle has zero
%or one element, this relational constraint can be reformulated into a disjunction\is{disjunction}.

Der Bedeutungsbeitrag, der unter \emph{modal-op} eingebettet wird, ist nicht
der von \emph{fahren}, sondern der von \stem{fahr} + selegierter Partikel. Das, was
die Partikel beisteuert, wird dann unter \emph{modal-op} eingebettet.

Wenn die Struktur in (\mex{0}) mit der Partikel \emph{an$_5$} in (\ref{le-an5}) auf Seite~\pageref{le-an5}
kombiniert wird, bekommt man (\mex{1})\vpageref{le-anfahrbar}.
%\begin{figure}
\eas
\label{le-anfahrbar}
\mbox{\emph{an+fahrbar} (mit \emph{an$_5$}):}\\
\ms{ cat  & \ms{ head   & \ms[adj]{ 
                           subj & \liste{ NP[\str]\ind{2''} }\\
                          }\\
                 comps & \eliste \\
               } \\
      cont & \ms[modal-op]{
                 soa & \ms[gerichtet-auf~]{ 
                         arg1 & \ms[fahren]{ agens & \etag \\
                                           }  \\
                         arg2 & \ibox{2''} \\
                        } \\
              }\\
}
\zs
%\vspace{-\baselineskip}\end{figure}
%

\noindent
In dieser Struktur sind die Nummern, die durch die Partikel instantiiert werden, mit zwei Apostrophen markiert.
Die Partikel steuert ein NP"=Argument bei und instantiiert \iboxt{3'} mit 
$\left\langle\right.$\,NP[\str]\raisebox{-0.8ex}{\ibox{2''}}\,$\left.\right\rangle$.
Da die \compsl der Partikel leer ist, wird \iboxt{4'} durch die leere Liste instantiiert. 
Die Subtraktion von $\left\langle\right.$\,NP[\str], \iboxt{2}~NP[\str]\,$\left.\right\rangle$ von
\iboxt{1'} $\oplus$ \iboxt{3'} $\oplus$ \iboxt{4'} ergibt die leere Liste, und deshalb ist
\iboxt{3} ebenfalls die leere Liste. Das Subjekt des \baradjs in (\mex{0}) ist identisch
mit dem Subjekt, das von der Partikel eingeführt wurde. Es ist das zweite Argument der
\relation{gerichtet-auf}"=Relation. Das Agens von \emph{fahren} wird unterdrückt.
\iw{\bars|)}

Bevor wir uns im nächsten Abschnitt alternativen Analysen zuwenden, möchte
ich kurz noch die Beispiele in (\mex{1}) diskutieren, die zeigen, dass Elemente, die von Partikelverben
abgeleitet sind, weitere morphologische Veränderungen zulassen:
\eal
\label{pvs-in-morphology}
\ex 
unannehmbar
\ex 
das Pseudo-Herumgerede\footnote{
        \citew[\page 40]{Stiebels96a}.%
    }
\zl
%\citet[\page 940]{SW94a}
%Vor-[ab-druck]
In (\mex{0}a) wurde \word{annehmbar} mit \prefix{un} präfigiert und in (\mex{0}b) wurde
\word{Herumgerede} mit \prefix{Pseudo} kombiniert. Man muss deshalb zulassen,
dass das Schema, das die Partikel mit dem derivierten Nomen oder Adjektiv kombiniert,
innerhalb der Morphologiekomponente angewendet wird bzw.\ Eingabe zu dieser Komponente sein
kann. 
\NOTE{FB: Aber wie würde das bei Dir gehen? Partikel+Nomen/adj. werden doch schon syntaktisch
  kombiniert. Oder darf man den Mutterknoten in head-cluster-strukturen wieder in die Morphologie
  stecken, weil er LEX unterspezifiziert ist und es sich gar nicht um eine richtige syntaktische
  Phrase handelt? Dazu müßtest Du vielleicht noch was sagen.}
%Das kann man erreichen, indem man Strukturen vom Typ \type{head"=cluster"=phrase} nicht nur als
%Untertypen von \type{phrase} sondern auch als Untertypen von \type{word} zuläßt. 
% Hm. Irgendwie sollte un- aber an einen Stamm gehen, oder?
Das Ergebnis der
Kombination von Partikel und Adjektiv bzw.\ Nomen 
%ist dann ein Wort und 
bildet dann die Basis für die
Kombination mit Elementen wie \prefix{un} oder \prefix{Pseudo}.
\begin{comment}
The combination of particle and verbal stem in the morphological component is also needed
for compounds like those in (\mex{1}):\footnote{
        See \citew[\page 55]{Groos89a} for similar Dutch\il{Dutch} examples.
}
\eal
\ex 
\gll Einschreibformular\\
     in.write.form\\
\glt `registration form'
\ex  
\gll Einwickelpapier\\
     in.wrap.paper\\
\glt `wrapping paper'
\zl
\end{comment}
\is{Morphologie|)}%
\is{Klammerparadoxon|)}%
\is{Derivation!Adjektiv|)}




%\subsection{Morphophonologie}
%\label{sec-morphophonologie}


\section{Alternativen}

Im folgenden sollen Kopf"=Affix"=Strukturen als Alternative zu Lexikonregeln diskutiert werden.
Abschnitt~\ref{sec-flex-mark} beschäftigt sich mit einer Analyse der Flexion als Kopf"=Marker"=Struktur.

\subsection{Kopf-Affix-Strukturen vs.\ Lexikonregeln}
\label{morphem-vs-lr}

Alternativ\is{Affix|(} zu Description"=Level Lexikonregeln für die Beschreibung von morphologischen Prozessen
(\citealp{Orgun96a}; \citealp{Riehemann98a}; \citealp{AW98a}; \citealp{Koenig99a};
\citealp[Kapitel~6.2.5]{Mueller2002b}; \citealp{Crysmann2002a}) wurden innerhalb des HPSG"=Rahmens
auch Kopf"=Affix"=Strukturen, die binär verzweigenden syntaktischen Strukturen ähneln, vorgeschlagen 
(\citealp{KN93a}; \citealp{Krieger94a}; \citealp{Eynde94}; \citealp{Lebeth94}).
Zum Beispiel selegiert ein Derivationsmorphem dann einen Stamm und bestimmt die Valenz und Wortart
des entstehenden komplexen Stamms. So verlangt das Derivationsmorphem \suffix{bar} ein transitives
Verb. Die Kombination aus transitivem Verb und \suffix{bar} ergibt ein Adjektiv, das über
das Objekt des Verbs prädiziert:
\eal
\ex Er löst das Problem.
\ex das lösbare Problem
\zl


\noindent
Manchmal wird es als Vorteil des Lexikonregelansatzes betrachtet, dass man ohne hunderte von leeren
Affixen\is{leere Kategorie} für Nullflexion und Konversion\is{Konversion} auskommt. Abstrakte Morpheme, die Stämme verkürzen, werden bei Verwendung
von LR nicht gebraucht, da Morphemveränderungen über entsprechende Funktionen in den Lexikonregeln
geregelt werden. 

\citet{Zwicky85a,Zwicky92a} und \citet[\page166]{Koenig99a} verwenden die folgenden Argumente
gegen eine Behandlung von Affixen als Kopf:
\begin{enumerate}
\item Affixe kongruieren nie mit ihren Argumenten, obwohl das in der Syntax oft vorkommt und deshalb zu erwarten
      wäre, wenn Affixe Köpfe wären.
\item Syntaktische Köpfe können in elliptischen Konstruktionen weggelassen werden, Affixe dagegen nie.
\end{enumerate}
Diese Argumentation ist jedoch aus den folgenden Gründen nicht korrekt: Es ist durchaus sinnvoll,
von einer Menge von Objekten zu behaupten, dass sie bestimmte Eigenschaften haben, auch wenn diese Menge
sich in bezug auf weitere Eigenschaften in Teilmengen auf"|teilen läßt. So kann man durchaus von Köpfen
sprechen, auch wenn syntaktische Köpfe sich von morphologischen Köpfen unterscheiden. Dass das so ist,
kann man sich auch anhand der Grafik~\vref{abb-syn-morph-kopf} verdeutlichen, die dem entspricht, was wir auch über
Vererbungshierarchien gelernt haben.
\begin{figure}
\begin{forest}
[Kopf
  [syntaktischer Kopf]
  [morphologischer Kopf]]
\end{forest}
\caption{\label{abb-syn-morph-kopf}Syntaktische und morphologische Köpfe}
\end{figure}
Es gibt gewisse Eigenschaften, die für alle Köpfe zutreffen, und außerdem auch noch Eigenschaften,
die jeweils nur für syntaktische bzw.\ morphologische Köpfe zutreffen.

%% Davon ganz abgesehen gibt es natürlich Argumente, die nicht mit ihrem Kopf kongruieren. Im Deutschen
%% gibt es \zb keine Objektkongruenz. Somit könnte die Affix"=Argument"=Beziehung parallel 
%% zur Verb"=Objekt"=Beziehung sein. Analog kann man für die zweite Eigenschaft Beispiele finden,
%% in denen der Kopf nicht weggelassen werden kann.
%% the man
%%
%% Man kann die Fehlerhaftigkeit der angeführten Argumentation auch mit einem Gleichnis verdeutlichen:
%% Menschen kann man aufgrund ihrer biologischen Eigenschaften in männliche und weibliche Wesen
%% einteilen. Man kann dann sagen, dass Männer 


In vielen Fällen sind die Ansätze ineinander überführbar (\citealp[\page168--169]{Koenig99a}; \citealp[Kapitel~6.2.5.2]{Mueller2002b}),
es gibt jedoch ein paar Kleinigkeiten, in denen sie sich unterscheiden. Das soll im folgenden
anhand der bereits diskutierten \bard diskutiert werden.
Für die \bard könnte man zum Beispiel folgendes Suffix für \suffix{bar} annehmen:
\eas
\oneline{\onems{
phon   \phonliste{ bar }\\
synsem$|$loc  \ms{ cat  & \ms{ head   & \ms[adj]{ 
                                                 subj & \sliste{ \ibox{2} NP[\str] }\\
                                                 }\\
                                        comps & \ibox{3} $\oplus$ \sliste{ V[\comps \liste{ NP[\str], \ibox{2} NP[\str] } $\oplus$ \ibox{3}]:\ibox{4}} \\
                                      } \\
                            cont & \ms[modal-op]{
                                    soa & \ibox{4}\\
                                   }\\
                         }\\
}}\is{Morphem!-\emph{bar}-Derivation}%
\zs
Das \bars ist in der Struktur für den Stamm \emph{lesbar} der Kopf. Es verlangt als Argument ein transitives
Verb, dessen Objekt \iboxb{2} es zu seinem Subjekt macht. Das logische Subjekt des transitiven Verbs wird unterdrückt.
Weitere eventuell vorliegende Argumente des Verbs \iboxb{3} werden angezogen.
Die Kombination von Verbstamm und \suffix{bar} wird durch eine Version des Prädikatskomplexschemas\is{Schema!Prädikatskomplex-} lizenziert,
was die Intuition von \citet{Bierwisch90a} erfassen würde, der die Verbalkomplexbildung als quasi"=morphologische
Bildung versteht. Hierbei gibt es jedoch ein kleines Problem: Es muss sichergestellt werden, dass die Kombination
von \stem{lös} und \suffix{bar} einen Stamm ergibt, da \emph{lösbar} noch flektiert werden muss, bevor es in
der Syntax verwendet wird. Das normale Prädikatskomplexschema aus Kapitel~\ref{chap-verbalkomplex} lizenziert
aber Phrasen und keine Stämme. Um dieses Problem zu lösen, müßte man einen allgemeinen Typ definieren,
von dem sowohl das Prädikatskomplexschema für die Syntax als auch ein weiteres Schema für die Morphologie erbt.
Die beiden Schemata würden sich dann nur darin unterscheiden, welchen Typ das Ergebnis der Kombination hat und
welchen Typ die kombinierten Elemente haben. Das syntaktische Schema lizenziert nur die Kombination von Wörtern
bzw.\ Phrasen miteinander, wohingegen das morphologische Schema die Kombination von Stämmen bzw.\ Wörtern mit
Affixen lizenziert.

Hier gibt es einen weiteren Unterschied zwischen den lexikonregelbasierten und den morphembasierten Ansätzen:
Es muss irgendwie sichergestellt werden, dass Affixe mit einem Objekt des jeweils richtigen Typs verbunden werden,
\dash, manche Affixe müssen mit Wörtern kombiniert werden, andere mit Stämmen. 
Im lexikonregelbasierten Ansatz kann
man die \textsc{lex-dtr} entsprechend spezifizieren, im morphembasierten Ansatz gibt es jedoch keine Möglichkeit
zu sagen, von welchem Typ das Element sein soll, das mit einem bestimmten Affix kombiniert werden soll. Das liegt daran,
dass nur \type{synsem}"=Objekte selegiert werden, und die Typ"=Information \type{stem} bzw.\ \type{word} außerhalb
von \textsc{synsem} repräsentiert ist. Dieses Problem kann man nun auf verschiedene Weisen lösen:
Man kann die Beschränkung in bezug auf die Lokalität\is{Lokalität} der Selektion, die im Kapitel~\ref{chap-lokalitaet}
eingeführt wurde, wieder lockern und statt \type{synsem}"=Objekten Objekte vom Typ \type{sign} selegieren, oder
man kann Hilfsmerkmale innerhalb von \textsc{synsem} einführen, die eine Unterscheidung zwischen Stämmen und Wörtern
ermöglichen.

Außerdem gibt es das Problem, dass die Kombination von Affix und einem weiteren Element entweder ein
Stamm oder ein Wort sein kann. Ein Dominanzschema kann aber nur entweder ein
Untertyp des Typs \type{word} oder des Typs \type{stem} sein, nicht jedoch beides
gleichzeitig. Dieses Problem könnte man dadurch lösen, dass man zwei Untertypen des morphologischen
Schemas annimmt, wobei ein Untertyp ein Untertyp von \type{word} und der andere ein Untertyp von
\type{stem} ist. Würde man alle Affixe als Flexions- bzw.\ Derivationsaffixe kennzeichnen, so könnte
man auch die beiden Dominanzschemata für die Art des Affixes sensitiv machen: Das Schema, das ein
Flexionsaffix als Tochter nimmt, lizenziert Wörter, und das Schema mit dem Derivationsaffix als
Tochter lizenziert Stämme, die noch flektiert werden müssen, bevor sie in der eigentlichen Syntax
verwendet werden können.

% ge t 
% von Montag an


%% \subsection{Bezugname auf lexikalische Klassen in Lexikonregeln}

%% Die Lexikonregel für die \bard in (\ref{lr-bar-adj}) ähnelt sehr derjenigen, die von\citet[\page68]{Riehemann98a}
%% vorgeschlagen wurde. Es gibt jedoch einen wichtigen Unterschied: In (\ref{lr-bar-adj}) wird nicht direkt
%% auf die Verbklasse des Verbstamms Bezug genommen, sondern die Eigenschaften der einbettbaren Verbstämme
%% werden explizit beschrieben. Bei Riehemann wird für die Tochter verlangt, dass sie vom Typ \type{trans-verb}
%% ist.

%% \eas
%% Lexikonregel für Derivation von Adjektiven mit \bars:\\
%% \resizebox{\linewidth}{!}{%
%% \ms[reg-bar-adj-stem]{
%% phon   & \ibox{1} $\oplus$ \phonliste{ bar }\\
%% synsem & \onems{ loc  \ms{ cat  & \ms{ head   & \ms[adj]{ 
%%                                                  subj & \sliste{ \ibox{2} NP[\str] }\\
%%                                                  }\\
%%                                         comps & \ibox{3}\\
%%                                       } \\
%%                             cont & \ms[modal-op]{
%%                                     soa & \ibox{4}\\
%%                                    }\\
%%                          }\\
%%                }\\
%% lex-dtr & \onems[trans-verb]{
%%            phon   \ibox{1} \\
%%            synsem$|$loc \ms{ cat  & \ms{ head   & verb\\
%%                                          comps & \sliste{ NP[\str], \ibox{2} NP[\str] } $\oplus$ \ibox{3}\\
%%                                        } \\
%%                              cont & \ibox{4} \\
%%                           }\\
%%            }\\
%% }\is{Lexikonregel!bar-Derivation@\emph{bar-}-Derivation}%
%% }
%% \zs


Zum Abschluß dieses Abschnitts soll noch darauf hingewiesen werden,
dass es Sprachen wie Portugiesisch\il{Portugiesisch} gibt, in denen mehrere Argumente eines Verbs
durch ein einziges Klitikon\is{Klitisierung} realisiert werden, welches sich als Affix mit dem Verb
verbindet \citep[Kapitel~2.1.1.4 und S.\,169--171]{Crysmann2002a}. 
Dies kann man in einem Ansatz mit Lexikonregeln gut modellieren: Die Lexikonregel bindet die jeweiligen
Argumente ab und sorgt für die entsprechenden phonologischen Veränderungen. In einem morphembasierten
Ansatz sind solche Fälle schwer zu erfassen, da man ein Morphem annehmen müßte, das für zwei Argumente
steht. Die Auf"|fassung, dass Morpheme Form"=Bedeutungspaare sind, läßt sich für solche Fälle wohl nicht
aufrechterhalten.
Man könnte annehmen, dass diese Fälle von Klitisierung anders behandelt werden und für die
restliche Morphologie davon ausgehen, dass Affixe Köpfe sind, aus Gründen der Sparsamkeit ist aber
eine Analyse, die nur einen Mechanismus für die morphologische Analyse verwendet, vorzuziehen.%
\is{Affix|)}

\subsection{Flexion als Markierung}
\label{sec-flex-mark}

\Citet{Eynde94} schlägt vor, Flexion mit dem Kopf"=Markierer"=Schema\is{Schema!Kopf"=Markierer"=}
\citep[\page46]{ps2} zu beschreiben. In Kopf"=Markierer"=Strukturen trägt der Markierer Information
zur Gesamtstruktur bei, ist aber selbst nicht der Kopf. Die vom Markierer beigesteuerte Information
wird als Wert des Merkmals \textsc{marking}\isfeat{marking} repräsentiert. In Strukturen ohne
Markierung ist der Wert dieses Merkmals \type{unmarked}. Das Merkmal \textsc{marking} befindet sich
unter dem Pfad \textsc{synsem$|$loc$|$cat}, also innerhalb von \textsc{synsem}, und sein Wert kann somit
von anderen Elementen im Rahmen einer Selektion beschränkt werden. Behandelt man Flexionsaffixe als
Markierer, ist also automatisch erklärt, warum sich bei Flexion die Wortart nicht ändert, denn die
Kopf"|information wird ja vom Stamm übernommen. Der Markierer kann das markierte Element über das
Merkmal \textsc{spec} selegieren und kann somit verlangen, dass er mit einem bisher nicht markierten
Element kombiniert wird. So kann man sicherstellen, dass ein Element nicht mehrfach flektiert wird,
und man kann auch sicherstellen, dass \zb das \bars nur mit einem Stamm und nicht mit einem bereits
flektierten Word kombiniert wird.  Bildungen wie (\mex{1}b) sind somit ausgeschlossen:

\eal
\ex[]{
lesbar
}
\ex[*]{
liestbar
}
\zl

\noindent
Der Marker"=Ansatz hat allerdings ein Problem, wenn man davon ausgeht, dass Tempus"=Morpheme
wie die für das Präsens und Präteritum in (\mex{1}) und das Futur"=Morphem im Französischen\il{Französisch}
(\mex{2}) einen semantischen Beitrag leisten.
\eal
\ex Er schläft.
\ex Er schlief.
\zl
\ea
\gll Je le verrai.\\
     ich ihn sehen.werde\\
\glt `Ich werde ihn sehen.'
\z
In Kopf"=Markierer"=Strukturen kommt der semantische Beitrag nämlich vom Kopf \citep[\page 56]{ps2}, in der Analyse
von (\mex{-1}) und (\mex{0}) hat das Tempus"=Morphem Skopus über den semantischen Beitrag des Stammes,
der semantische Beitrag muss also vom Flexionsaffix kommen.\is{Morphologie|)}

%% Man kann doch einfach sagen, dass Stämme oder Wörter vorn kein Plus haben dürfen. Das ist doch
%% wohl so, oder? 23.11.2006
%%
%% \subsection{Morphemabfolge über Markierung der Morphemgrenze in \phon}
%%
%% \Citet{Eynde94} entwickelt eine morphembasierte Analyse der Derivation. Er argumentiert
%% dagegen Derivationsaffixe explizit als Präfixe und Suffixe zu markieren. Statt dessen schlägt er
%% vor, ein diakritisches Zeichen im \phonw zu verwenden. Das Affix \suffix{ed} bekommt den
%% \phonw \phonliste{ +ed } und ist damit als Suffix gekennzeichnet. Morphemkombinationen, in denen
%% Suffixe vor dem Element stehen, an das sie eigentlich angehängt werden müssen, sind nach van Eynde
%% dadurch ausgeschlossen, dass Wörter nie mit einem \phonliste{ + } anfangen dürfen (S.\,79).
%% Leider reicht eine solche Kennzeichnung aber nicht aus, da mehrere Morpheme miteinander kombiniert
%% werden können. Zum Beispiel kann das abgeleitete Adjektiv \emph{schlagbar} mit \prefix{un} präfigiert
%% werden. Die von van Eynde formulierte Beschränkung schließt dann aber (\mex{1}b) nicht aus:
%% \eal
%% \ex[]{
%% un+schlag++bar+
%% }
%% \ex[*]{
%% un++bar+schlag+
%% }
%% \zl
%% Man könnte jetzt einwenden, dass man (\mex{0}b) durch eine weiter Beschränkung ausschließt,
%% die besagt, dass Präfixe nicht mit etwas kombiniert werden können, das mit einem `+' beginnt.
%% Diese Beschränkung läßt sich aber nicht ausdrücken, wenn man die Tatsache, dass es sich bei einem
%% Element um ein Präfix handelt, nicht explizit kenntlich gemacht hat. Im Lexikoneintrag für das Präfix selbst
%% könnte man natürlich verlangen, dass der Stamm mit dem das Präfix kombiniert wird, keinen \phonw
%% hat, der mit `+' beginnt. Eine solche Beschränkung ließe sich aber nur formulieren, wenn man
%% \phon zu den selegierbaren Merkmalen zählt und also die in Kapitel~\ref{chap-lokalitaet} diskutierten
%% Lokalitätsbeschränkungen aufgibt.%
\label{last-page-hpsg-teil}






\questions{
\begin{enumerate}
\item Zerlegen Sie die folgenden Wörter in Morpheme:
\eal
\ex Sprengung
\ex Verarbeitung
\ex klüger
\ex gelesen
\ex salzig
\zl
Welche Morpheme sind Flexions- und welche Derivationsmorpheme?

\end{enumerate}
}

\exercises{
\begin{enumerate}
\item Schreiben Sie eine Lexikonregel für die \ungn.
\item Erklären Sie, wie das Wort \emph{Abtretung}
%\footnote{
% Ausraubung
%        Neues Deutschland, 29.04.1954, S.\,6 oder auch Frankfurter Rundschau, 27.09.1997, S. 14}
 abgeleitet wird.
%\NOTE{WS: Ich finde das Wort "Ausraubung" ziemlich merkwürdig. Gibts das wirklich?}
      Wieso ist es für die in diesem Kapitel vorgestellte Analyse
      kein Problem, dass es das Wort \noword{Tretung} nicht gibt?
% Raubung gibt es bei Google aus alten Texten und bei Versch raubung.
% In COSMAS kommt es nicht vor. 28.12.2006

\item Laden Sie die zu diesem Kapitel gehörende Grammatik von der Grammix"=CD
(siehe Übung~\ref{uebung-grammix-kapitel4} auf Seite~\pageref{uebung-grammix-kapitel4}).
Im Fenster, in dem die Grammatik geladen wird, erscheint zum Schluß eine Liste von Beispielen.
Geben Sie diese Beispiele nach dem Prompt ein und wiederholen Sie die in diesem Kapitel besprochenen
Aspekte.
\end{enumerate}
}


% To Do: ER2000a-u

\begin{comment}
% AW98a: 11 LFG -> co-heads
% französisches Futur




van Eynde sagt, dass Affix keine Kategorie beisteuert, da es Adjektivaffixe
geben muss, die an Partizipien gehen.

Diese müssen dann aber disjunktiv ein Verb oder ein Adjektiv selegieren und genauso
könnte man sagen, dass sie disjunktiv ein Adjektiv oder ein Verb sind.























\end{comment}
