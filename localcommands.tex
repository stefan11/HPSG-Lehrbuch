%% -*- coding:utf-8 -*-


\hypersetup{bookmarksopenlevel=0}

\setcounter{secnumdepth}{4}

\let\citew\citet

\newcommand{\danish}{\jambox{(\ili{Danish})}}
\newcommand{\dutch}{\jambox{(\ili{Dutch})}}
\newcommand{\english}{\jambox{(\ili{English})}}
\newcommand{\german}{\jambox{(\ili{German})}}
\newcommand{\yiddish}{\jambox{(\ili{Yiddish})}}
\newcommand{\icelandic}{\jambox{(\ili{Icelandic})}}




%\newcommand{\rot}[1]{{\color{red}#1}}
% like Lab Phon
%\newcommand{\rot}[1]{{\color{lsLightWine}#1}}
% like Niger Congo
\newcommand{\rot}[1]{{\color{lsRed}#1}}
\newcommand{\rotbf}[1]{\rot{#1}}
\newcommand{\rotit}[1]{\rot{#1}}
%\newcommand{\gruen}[1]{{\color{green}#1}}
% like OGS
\newcommand{\gruen}[1]{{\color{lsDarkGreenOne}#1}}
\newcommand{\gruenbf}[1]{\gruen{#1}}
\newcommand{\gruensc}[1]{\gruen{#1}}
\newcommand{\gruenit}[1]{\gruen{#1}}

%\newcommand{\blau}[1]{{\color{blue}#1}}

% Topics at the grammar icours interface
% lsMidDarkBlue

% EOTMS
% lsMidBlue
\newcommand{\blau}[1]{{\color{lsMidBlue}#1}}

\newcommand{\blaubf}[1]{\blau{#1}}
\newcommand{\blausc}[1]{\blau{#1}}
\newcommand{\blauit}[1]{\blau{#1}}



\let\mc=\multicolumn


\newcommand{\sigle}[1]{\tiny{#1}}


%% now loaded by the langsci class
% \iftoggle{draft}{
% \usepackage{todonotes}
% }{
% \usepackage[disable]{todonotes}
% }

\iftoggle{draft}{}{
\presetkeys{todonotes}{disable}{}
}

\newcommand{\todostefan}[1]{\todo[color=orange!80]{\footnotesize #1}\xspace}
\newcommand{\todosatz}[1]{\todo[color=red!40]{\footnotesize #1}\xspace}

\newcommand{\inlinetodostefan}[1]{\todo[color=green!40,inline]{\footnotesize #1}\xspace}

\newcommand{\inlinetodoopt}[1]{\todo[color=green!40,inline]{\footnotesize #1}\xspace}
\newcommand{\inlinetodoobl}[1]{\todo[color=red!40,inline]{\footnotesize #1}\xspace}

\newcommand{\itd}[1]{\iftoggle{draft}{\inlinetodoobl{#1}}{}}
\newcommand{\itdobl}[1]{\iftoggle{draft}{\inlinetodoobl{#1}}{}}
\newcommand{\itdopt}[1]{\iftoggle{draft}{\inlinetodoopt{#1}}{}}

% for editing, remove later
%\usepackage{xcolor}
\newcommand{\iaddpages}{\iftoggle{draft}{\yel[add pages]{pages}\xspace}}

\newcommand{\addpages}{\iftoggle{draft}{\todostefan{add pages}}\xspace}
\newcommand{\addsource}{\iftoggle{draft}{\todostefan{add source}}\xspace}
\newcommand{\addglosses}{\iftoggle{draft}{\todostefan{add glosses}}\xspace}



%% % This sets the default for the positioning
%% % will be in the main class
%% \makeatletter
%% \renewcommand{\fps@figure}{htbp}
%% \renewcommand{\fps@table}{htbp}
%% \makeatother
%% \renewcommand{\floatpagefraction}{0.7}	% require fuller float pages
%% 	% N.B.: floatpagefraction MUST be less than topfraction !!




\robustify\textsc
\robustify\textit


% https://tex.stackexchange.com/questions/95014/aligning-overline-to-italics-font/95079#95079
\newbox\usefulbox

\makeatletter
    \def\getslant #1{\strip@pt\fontdimen1 #1}

    \def\skoverline #1{\mathchoice
     {{\setbox\usefulbox=\hbox{$\m@th\displaystyle #1$}%
        \dimen@ \getslant\the\textfont\symletters \ht\usefulbox
        \divide\dimen@ \tw@ 
        \kern\dimen@ 
        \overline{\kern-\dimen@ \box\usefulbox\kern\dimen@ }\kern-\dimen@ }}
     {{\setbox\usefulbox=\hbox{$\m@th\textstyle #1$}%
        \dimen@ \getslant\the\textfont\symletters \ht\usefulbox
        \divide\dimen@ \tw@ 
        \kern\dimen@ 
        \overline{\kern-\dimen@ \box\usefulbox\kern\dimen@ }\kern-\dimen@ }}
     {{\setbox\usefulbox=\hbox{$\m@th\scriptstyle #1$}%
        \dimen@ \getslant\the\scriptfont\symletters \ht\usefulbox
        \divide\dimen@ \tw@ 
        \kern\dimen@ 
        \overline{\kern-\dimen@ \box\usefulbox\kern\dimen@ }\kern-\dimen@ }}
     {{\setbox\usefulbox=\hbox{$\m@th\scriptscriptstyle #1$}%
        \dimen@ \getslant\the\scriptscriptfont\symletters \ht\usefulbox
        \divide\dimen@ \tw@ 
        \kern\dimen@ 
        \overline{\kern-\dimen@ \box\usefulbox\kern\dimen@ }\kern-\dimen@ }}%
     {}}
    \makeatother



\newcommand{\questions}[1]{~\newline\vspace*{-10mm}
{\memoizeset{disable}%
\tblssy{people}{Comprehension questions}{\setlist{leftmargin=*}#1}}}
%\tblssy{people}{Comprehension questions}{#1}}

\newcommand{\exercises}[1]{{\memoizeset{disable}%
\tblssy{pencil}{Exercises}{\setlist{leftmargin=*}#1}}}
%\tblssy{pencil}{Exercises}{#1}}

\newcommand{\furtherreading}[1]{%~\newline\vspace*{-10mm}
{\memoizeset{disable}%
\tblssy{book}{Further reading}{#1}}}

\newcommand{\greyboxrest}[1]{
{\memoizeset{disable}%
\begin{mdframed}[style=greyexercise]
#1
\end{mdframed}}}

\mdfdefinestyle{greyexercisenologo}{%
	everyline=true,ignorelastdescenders=true,
	linewidth=0pt,backgroundcolor=\tblsboxcolor,
	innerleftmargin=5mm, innerrightmargin=5mm, innerbottommargin=5mm, innertopmargin=5mm,
	frametitleaboveskip=15mm, frametitlebelowskip=5mm,frametitlerule=false, repeatframetitle=false
}


% get rid of these morewrite messages:
% https://tex.stackexchange.com/questions/419489/suppressing-messages-to-standard-output-from-package-morewrites/419494#419494
\ExplSyntaxOn
\cs_set_protected:Npn \__morewrites_shipout_ii:
  {
    \__morewrites_before_shipout:
    \__morewrites_tex_shipout:w \tex_box:D \g__morewrites_shipout_box
    \edef\tmp{\interactionmode\the\interactionmode\space}\batchmode\__morewrites_after_shipout:\tmp
  }
\ExplSyntaxOff

\newcommand{\term}[1]{\emph{\isi{#1}}}



% Felix 09.06.2020: copy code from the third line into localcommands.tex: https://github.com/langsci/langscibook#defined-environments-commands-etc
\patchcmd{\mkbibindexname}{\ifdefvoid{#3}{}{\MakeCapital{#3} }}{\ifdefvoid{#3}{}{#3 }}{}{\AtEndDocument{\typeout{mkbibindexname could not be patched.}}}



% This does a linebreak for \gll for long sentences leaving space for the language at the right
% margin.
% St.Mü. 17.06.2021
% \newcommand{\longexampleandlanguage}[2]{
% \begin{tabularx}{.99\linewidth}[t]{@{}X@{}p{\widthof{(#2)}}@{}}
% \begin{minipage}[t]{\linewidth-1em}
% #1
% \end{minipage} & (\ili{#2})
% \end{tabularx}}

% From stackexchange:
% https://tex.stackexchange.com/questions/600695/limiting-space-used-by-gb4e-and-adding-language-information-to-the-right/649414#649414
\newcommand{\longexampleandlanguage}[2]{
%\begin{tabularx}{\linewidth}[t]{@{}X@{}p{\widthof{(#2)}}@{}}
\begin{minipage}[t]{\linewidth-1em-\widthof{(#2)}}
#1
\end{minipage} 
\hfill
\begin{minipage}[t]{\widthof{(#2)}}
 (\ili{#2})
\end{minipage}
%\end{tabularx}
}


% This is needed for the index since the ! will be interpreted as subitem.
\newcommand{\dslmath}{$/\!/$}



% http://tex.stackexchange.com/questions/230300/doing-something-like-psframebox-in-tikz#230306
\tikzset{
frbox/.style={
  rounded corners,
  draw,
  thick,
  inner sep=5pt
  }
}
\newcommand\TZbox[1]{\tikz{\node[frbox,baseline] {#1};}}

\newgray{hell}{.85}
\newcommand{\highlight}[1]{\psframebox[linecolor=hell,fillcolor=hell,fillstyle=solid]{#1}}

\newcommand{\nom}{\textit{nom}}
\newcommand{\gen}{\textit{gen}}
\newcommand{\dat}{\textit{dat}}
\newcommand{\acc}{\textit{acc}}

%\newcommand{\mod}{\textsc{mod}\xspace}  % wegen beamer.cls nicht in abbrev.sty
\newcommand{\rel}{\textsc{rel}\xspace}  % wegen avm.sty nicht in abbrev.sty

