\is{AcI-Verb|see{Verb}}
\is{Adjektiv!-feld|see{Feld}}
\is{Agreement@\emph{Agreement}|see{Kongruenz}}
\is{Akkusativ|see{Kasus}}
\is{Aktiv-Passiv-Relation|see{Passiv}}
%\is{Anhebung|see{Objekt-zu-Subjekt-An\-he\-bung, Subjektanhebung}}
\is{Anhebung!-sverb|see{Verb}}
\is{Ambiguität|see{Mehrdeutigkeit}}
\is{Argument Attraction@\emph{Argument Attraction}|see{Argumentanziehung}}
\is{Artikel|see{Determinator}}
\is{attribute-value matrix (AVM)@\emph{attribute-value matrix} (AVM)|see{Merkmalbeschreibung}}
\is{Ausklammerung|see{Extraposition}}
\is{Auxiliary@\emph{Auxiliary}|see{Hilfsverb}}
%
\is{Dativ|see{Kasus}}
\is{Dativus@\emph{Dativus}!ethicus@\emph{ethicus}|see{Dativ, ethischer}}
%\is{Domäne|see{Wortstellungsdomäne}}
\is{Dominanz|see{Prinzip}}
\is{DP|see{Determinatorphrase}}
%
\is{ECM-Verb|see{Verb (AcI-)}}
\is{Eintakt"=Passiv|see{Passiv}}
\is{exceptional case marking@\emph{exceptional case marking}|see{Verb (AcI-)}} 
%\is{Empfänger-Diathese|see{Passiv-Pa\-ra\-phra\-se}}
\is{Ergänzung|see{Argument}}
\is{ergatives Verb|see{Verb (unakkusativisches)}}
\is{equi-Konstruktion@\emph{equi}-Konstruktion|see{Kontrolle}}
\is{Expletivum|see{Pronomen}}
\is{Expletivum|see{positionales \emph{es}}}
%
\is{Fernpassiv|see{Passiv}}
\is{freier Relativsatz|see{Relativsatz}}
\is{Funktion|see{Relation}}
%
\is{gap@\emph{gap}|see{Spur}}
\is{GB|see{\emph{Government and Binding}}}
\is{GPSG|see{\emph{Generalized Phrase Structure Grammar}}}
\is{Grammatikregel|see{Schema}}
\is{grammatische Kategorie|see{Kategorie}}
%
\is{Head@\emph{Head}|see{Kopf}}
\is{head-movement@\emph{head-movement}|see{Kopfbewegeung}}
\is{Hilfsverb|see{Verb}}
%
\is{ID-Regel|see{\emph{Immediate Dominance Rule}}}
\is{Infinitiv|see{Inkohärenz, Kohärenz}}
\is{Insel|see{Extraktion}}
\is{Incomplete Category Fronting@\emph{Incomplete Category Fronting}|see{Partial Verb Phrase Fronting}}
%
\is{Kontrollverb|see{Verb}}
%
\is{lexikalischer Kasus|see{Kasus}}
\is{Linear Precedence Rule@\emph{Linear Precedence Rule}|see{Linearisierungsregel}}
\is{LP"=Regel|see{Linearisierungsregel}}
\is{Lücke|see{Spur}}
%
\is{Markierer|see{Schema}}
\is{Maximalprojektion|see{Projektion}}
\is{Mehrfachvererbung|see{Vererbung}}
\is{Modalverb|see{Verb}}
\is{modaler Infinitv|see{Infinitiv}}
\is{Modifikator|see{Adjunkt, Adjektiv, Adverb, Konditionalsatz, Prä\-po\-si\-tion\-al\-phra\-se, Relativsatz}}
%\is{Modifikator|see{Adverb}}
%\is{Modifikator|see{Relativsatz}}
\is{multiple inheritance@\emph{multiple inheritance}|see{Vererbung}}
\is{MRS|see{\emph{Minimal Recursion Semantics}}}
%
\is{Nominativ|see{Kasus}}
\is{Null-Topik|see{Vorfeldellipse}}
\is{Nullkasus|see{Kasus}}
%
%
\is{parasitic gap@\emph{parasitic gap}|see{Lü"cke}}
\is{parametrized state of affairs@\emph{parametrized state of affairs}|see{Sachverhalt, parametrisierter}}
\is{partitive Topikalisierung|see{NP-Auf\-spaltung}}
\is{Personalpronomen|see{Pronomen}}
\is{persönliches Passiv|see{Passiv}}
\is{Pertinenzdativ|see{Dativ, possessiver}}
\is{Perfekthilfsverb|see{Hilfsverb}}
\is{Phasenverb|see{Verb}}
\is{Phrasenstrukturregel|see{Schema}}
\is{Pied Piping@\emph{Pied Piping}|see{Rattenfänger"=Konstruktion}}
\is{Platzhalter|see{positionales \emph{es}}}
%\is{PP|see{Präpositionalphrase}}
%\is{Präverb|see{Verbzusatz}}
\is{Prä\-fix\-verb|see{Verb}}
\is{Partikelverb|see{Verb}}
\is{pronoun zap@\emph{Pronoun Zap}|see{Vorfeldellipse}}
%
\is{quantifier storage@\emph{quantifier storage}|see{Speicher}}
%
\is{raising@\emph{raising}|see{Anhebung}}
\is{Reflexivpronomen|see{Pronomen}}
\is{Relativpronomen|see{Pronomen}}
%\is{reziprokes Pronomen|see{Pronomen}}
\is{Regel|see{Schema}}
\is{Relativsatz|see{Kongruenz, Rattenfänger"=Konstruktion}}
\is{Remnant Extraposition@\emph{Remnant Extraposition}|see{dritte Konstruktion}}
\is{Remnant Movement@\emph{Remnant Movement}|see{Restbewegung, \emph{Partial Verb Phrase Fronting}}}
\is{Remote Passive@\emph{Remote Passive}|see{Passiv, Fern-}}
\is{Rolle|see{semantische Rolle}}
%
\is{Satzwertigkeit|see{Inkohärenz}}
\is{Schmarotzerlücke|see{Lücke, parasitäre}}
\is{Selektion|see{Subkategorisierung}}
\is{separables Präfix|see{Präfix}}
%\is{Serialisierung|see{Linearisierung}}
\is{sequence union@\emph{sequence union}|see{Relation, \emph{shuffle}}}
%\is{SILR|see{lexikalische Regel}}
%\is{Singleton rel Constraint@\emph{Singleton \textsc{rel} Constraint}|see{RP-Be\-schrän\-kung}}
\is{storage@\emph{storage}|see{Speicher}}
\is{Substantiv|see{Nomen}}
%\is{Subject Insertion Lexical Rule@\emph{Subject Insertion Lexical Rule}|see{lexikalische Regel}}
\is{subjektloses Verb|see{Verb}}
%\is{sort-resolved@\emph{sort-resolved}|see{sortenauf"|gelöst}}
\is{Split-NP@\emph{Split-NP}|see{NP-Auf"|spal\-tung}}
\is{Spur|see{leere Kategorie}}
\is{struktureller Kasus|see{Kasus}}
\is{Sättigung|see{Prinzip, Valenz}}
\is{state of affairs@\emph{state of affairs} (soa)|see{Sachverhalt}}
\is{Subjekt|see{Kongruenz}}
\is{Subkategorisierung|see{Prinzip, Valenz}}
%\is{substantive head@\emph{substantive head}|see{Kopf}}
\is{Tiefenkasus|see{semantische Rolle}}
%
\is{Third Construction@\emph{Third Construction}|see{dritte Konstruktion}}
\is{Topic Drop@\emph{Topic Drop}|see{Vorfeldellipse}}
%\is{Topik|see{Thema}}
\is{Topikalisierung|see{Vorfeldbsetzung}}
\is{Topologisches Feld|see{Feld}}
%\is{Tough-Konstruktion@\emph{Tough}-Konstruktion|see{\emph{easy}-Ad\-jek\-tiv}}
\is{Trace@\emph{Trace}|see{Spur}}
%
\is{unbounded dependency construction (UDC)@\emph{unbounded dependency construction} (UDC)|see{nichtlokale Ab\-hängig\-keit}}
\is{unpersönliches Passiv|see{Passiv}}
\is{unpersönliches es@unpersönliches \emph{es}|see{Expletivum}}
%\is{Unterfeld|see{Feld}}
%
%\is{Valenz|see{Subkategorisierung}}
%\is{Verb|see{Morphologie}}
\is{Verb!Hilfs-|see{Hilfsverb}}
%\is{Verb!Partikel-|see{Verbzusatz}}
\is{Verb!Prä\-fix-|see{Verb, Partikel-}}
\is{Verb!trennbares|see{Verb, Partikel-}}
\is{Verb!ergatives|see{Verb, unakkusativisches}}
%\is{Verbpartikel|see{Verbzusatz}}
\is{verb cluster@\emph{verb cluster}|see{Verbalkomplex}}
\is{Verbalkomplex|see{Schema}}
%\is{Verb-Projection-Raising@\emph{Verb-Projection-Raising}|see{Verbalkomplexunterbrechung}}
\is{Verbum sentiendi|see{Wahrnehmungsverb}}
\is{Vorfeld!-\emph{es}|see{positionales \emph{es}}}
%\is{VP|see{Verbphrase}}
%
%\is{w"=Element@\emph{w}"=Element|see{Pronomen}}
\is{Wortstellung|see{Linearisierung}}
%
%\is{X-Schema@\xbar-Schema|see{Schema}}
%
\is{Zustandspassiv|see{Passiv}}
